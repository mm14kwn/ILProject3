\documentclass[10pt, a4paper]{article}
% The next line loads some packages you will need
\usepackage{graphicx, amsmath, amssymb, fancyhdr, setspace}
\usepackage[round]{natbib}
\usepackage{siunitx}
\usepackage{pgfplots}
\pgfplotsset{compat=1.11}
\usepackage{amsthm}
\usepackage{bm}
\usepackage{nicefrac}
\usepackage{cleveref}
\usepackage{hyperref}
% Page formatting
\addtolength{\textwidth}{5mm}
\addtolength{\textheight}{12mm}
\addtolength{\topmargin}{-10mm}
\pretolerance = 10000000
\setlength{\parindent}{0pt}
\setlength{\parskip}{\baselineskip}
\onehalfspacing   
\newtheorem{exmp}{Example}[section]
\numberwithin{equation}{section}
\definecolor{color1}{RGB}{0,0,90} % Color of the article title and sections
\definecolor{color2}{RGB}{0,20,20} % Color of the boxes behind the abstract and headings
\hypersetup{hidelinks,colorlinks,breaklinks=true,urlcolor=color2,citecolor=color1,linkcolor=color1,bookmarksopen=false,pdftitle={Title},pdfauthor={Author}}
\newcommand{\rhob}{\bar{\rho}}
\newcommand{\srho}{\rhob + (\rho(\bm{x},t) -\rhob)}
\newcommand{\srhos}{\rhob + (\rho -\rhob)}
\newcommand{\DD}[2]{\frac{D#1}{D#2}}
\newcommand{\Dt}[1]{\DD{#1}{t}}
\newcommand{\vel}{\bm{u}}
\newcommand{\grav}{\bm{g}}
\newcommand{\del}{\nabla}
\newcommand{\deldot}{\nabla \cdot}
\newcommand{\delcross}{\nabla \times}
\newcommand{\inv}[1]{\frac{1}{#1}}
\newcommand{\bO}{\bm{\Omega}}
\newcommand{\bo}{\bm{\omega}}
\newcommand{\qv}{\bm{q}}
\newcommand{\delh}{\nabla_h}
\newcommand{\DDh}[2]{\frac{D_h#1}{D#2}}
\newcommand{\Dth}[1]{\DDh{#1}{t}}
\newcommand{\Dthexp}[1]{\partial_t {#1} + u \partial_x {#1} + v \partial_y {#1}}
\newcommand{\io}{\iota}
% Header / footer
\pagestyle{fancy}
\lhead{Kieran Newman, 200901399}
\chead{}
\rhead{\em MATH 5825M, assignment 3}
\lfoot{}
\cfoot{\thepage}
\rfoot{}
\setlength{\headheight}{20pt}
\renewcommand{\headrulewidth}{0.4pt}
\renewcommand{\footrulewidth}{0pt}

\begin{document}
\begin{center}
\textbf{\Large Vortex Leapfrogging} \\
\end{center}
\section{Introduction}
Vortex interactions are an area of fluid dynamics still undergoing significant research.
In this piece of work, we will first explore the dynamics of vortices and their interactions, then analyse the stability of a system of four vortices, following the work by \citet{acheson00}.
A program was written in MATLAB to plot the paths of vortices in this system, and this is given in \hyperref[sec:ap1]{Appendix 1}.

\section*{Appendix I: MATLAB Code}\label{sec:ap1}
MATLAB code for the vortex leapfrogging program is given below.
\begin{verbatim}
clear all
N=input('number of vortices = ');
%set time step and scales
t=0;
T=input('number of time steps = ');
%set length step and scales
h=input('step-size = ');
%define vortex strengths
for i=1:N
    k(i)=input(['strength of vortex ',num2str(i),' = ']);
end
%define variable for position
x=NaN(1000,N);
y=NaN(1000,N);
for i=1:N
    x(1,i)=input(['x-coordinate of vortex ',num2str(i),' = ']);
    y(1,i)=input(['y-coordinate of vortex ',num2str(i),' = ']);
end
for a=2:T
    clear x1 x2 x3 x4 y1 y2 y3 y4 k1 k2 k3 k4 l1 l2 l3 l4
    x1=x(a-1,:);
    y1=y(a-1,:);
    k1=NaN(1,N);
    k2=NaN(1,N);
    k3=NaN(1,N);
    k4=NaN(1,N);
    l1=NaN(1,N);
    l2=NaN(1,N);
    l3=NaN(1,N);
    l4=NaN(1,N);
    for i=1:N
        k1(i)=h*vortexf(i,k,x1,y1,N);
        l1(i)=h*vortexg(i,k,x1,y1,N);
    end
    x2=x1+(k1/2);
    y2=y1+(y1/2);
    for i=1:N
        k2(i)=h*vortexf(i,k,x2,y2,N);
        l2(i)=h*vortexg(i,k,x2,y2,N);
    end
    x3=x1+(k2/2);
    y3=y1+(y2/2);
    for i=1:N
        k3(i)=h*vortexf(i,k,x3,y3,N);
        l3(i)=h*vortexg(i,k,x3,y3,N);
    end
    x4=x1+(k3);
    y4=y1+(y3);
    for i=1:N
        k4(i)=h*vortexf(i,k,x4,y4,N);
        l4(i)=h*vortexg(i,k,x4,y4,N);
    end
    x(a,:)=x(a-1,:)+((1/6)*(k1+2*k2+2*k3+k4));
    y(a,:)=y(a-1,:)+((1/6)*(l1+2*l2+2*l3+l4));
    percent=100*a/T;
    display([num2str(percent),'\% done'])
end
clf
multicomet(x,y)
title([num2str(T),' time steps, step size = ',num2str(h)])
\end{verbatim}
The scheme used is a fourth order Runge-Kutta method with non-adaptive step size. The equations plotted are from \citet{acheson00}.
\bibliographystyle{dcu}
\bibliography{kwnrefs}
\end{document}
