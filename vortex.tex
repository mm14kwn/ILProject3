\documentclass[10pt, a4paper]{article}
% The next line loads some packages you will need
\usepackage{graphicx, amsmath, amssymb, fancyhdr, setspace}
\usepackage[round]{natbib}
\usepackage{siunitx}
\usepackage{pgfplots}
\pgfplotsset{compat=1.11}
\usepackage{amsthm}
\usepackage{bm}
\usepackage{datetime}
\usepackage{nicefrac}
\usepackage{hyperref}
\usepackage[noabbrev]{cleveref}
% Page formatting
\addtolength{\textwidth}{5mm}
\addtolength{\textheight}{12mm}
\addtolength{\topmargin}{-10mm}
\pretolerance = 10000000
\setlength{\parindent}{0pt}
\setlength{\parskip}{\baselineskip}
\onehalfspacing   
\newtheorem{exmp}{Example}[section]
\numberwithin{equation}{section}
\definecolor{color1}{RGB}{0,0,90} % Color of the article title and sections
\definecolor{color2}{RGB}{0,20,20} % Color of the boxes behind the abstract and headings
\hypersetup{hidelinks,colorlinks,breaklinks=true,urlcolor=color2,citecolor=color1,linkcolor=color1,bookmarksopen=false,pdftitle={Title},pdfauthor={Author}}
\newcommand{\rhob}{\bar{\rho}}
\newcommand{\srho}{\rhob + (\rho(\bm{x},t) -\rhob)}
\newcommand{\srhos}{\rhob + (\rho -\rhob)}
\newcommand{\DD}[2]{\frac{D#1}{D#2}}
\newcommand{\Dt}[1]{\DD{#1}{t}}
\newcommand{\vel}{\bm{u}}
\newcommand{\grav}{\bm{g}}
\newcommand{\del}{\nabla}
\newcommand{\deldot}{\nabla \cdot}
\newcommand{\delcross}{\nabla \times}
\newcommand{\inv}[1]{\frac{1}{#1}}
\newcommand{\bO}{\bm{\Omega}}
\newcommand{\bo}{\bm{\omega}}
\newcommand{\qv}{\bm{q}}
\newcommand{\delh}{\nabla_h}
\newcommand{\DDh}[2]{\frac{D_h#1}{D#2}}
\newcommand{\Dth}[1]{\DDh{#1}{t}}
\newcommand{\Dthexp}[1]{\partial_t {#1} + u \partial_x {#1} + v \partial_y {#1}}
\newcommand{\io}{\iota}
\newcommand{\ba}{\bm{A}}
\newcommand{\bb}{\bm{B}}
\newcommand{\bc}{\bm{C}}
\newcommand{\half}{\frac{1}{2}}
\newcommand{\nicehalf}{\nicefrac{1}{2}}
\newcommand{\dxa}{\partial_{x_0}}
\newcommand{\dxb}{\partial_{x_1}}
\newcommand{\dya}{\partial_{y_0}}
\newcommand{\dyb}{\partial_{y_1}}
% Header / footer
\pagestyle{fancy}
\lhead{Kieran Newman, 200901399}
\chead{\today}
\rhead{\em MATH 5825M, assignment 3}
\lfoot{}
\cfoot{\thepage}
\rfoot{}
\setlength{\headheight}{20pt}
\renewcommand{\headrulewidth}{0.4pt}
\renewcommand{\footrulewidth}{0pt}

\begin{document}
\begin{center}
\textbf{\Large Vortex Leapfrogging} \\
\end{center}
\section{Introduction}
Vortex interactions are an area of fluid dynamics still undergoing significant research.
In this piece of work, we will first explore the dynamics of vortices and their interactions, then analyse the stability of a system of four vortices, following the work by \citet{acheson00}.
A program was written in MATLAB to plot the paths of vortices in this system, and this is given in \hyperref[sec:ap1]{Appendix 1}.
\subsection{Notation}
\label{sec:notation}
Many differing notations are used in the literature for differential operators, vorticity, vectors and other terms. In this piece of work, the first partial derivative is represented by
\begin{equation}
\label{eq:1partial}
\frac{\partial}{\partial t} = \partial _t,
\end{equation}
and the second by
\begin{align}
\label{eq:doublepartial}
\frac{\partial^2}{\partial t^2} &:= \partial_{tt},\\
\label{eq:mixedpartial}
\frac{\partial^2}{\partial x\partial y} &:= \partial_{xy}
\end{align}
for double and mixed derivatives. Vectors are written
\begin{equation}
\label{eq:vector}
\bm{a}=\left(\begin{array}{c} a_1\\\vdots\\a_n \end{array}\right),
\end{equation}
and velocity is written (in three dimensional cartesian coordinates) as
\begin{equation}
\label{eq:veloc}
\vel=\left(\begin{array}{c} u_x\\u_y\\u_z \end{array}\right).
\end{equation}
\subsection{Vector Identities}
The following vector identities are used in this piece of work.
There is much use of the so-called ``Del'' operator, represented by (for Cartesian coordinates)
\begin{equation}
\label{eq:deldef}
\del=\left(\begin{array}{c} \partial_{x}\\\partial_{y}\\\partial_{z}\end{array}\right).
\end{equation}
In this notation, the divergence of a fluid is given by the inner product of ``Del'' with the velocity
\begin{equation}
\label{eq:divdef}
\mbox{div} (\vel) = \deldot \vel = \partial_x u_x + \partial_y u_y + \partial_z u_z,
\end{equation}
and the curl is given by the outer product
\begin{equation}
\label{eq:curldef}
\mbox{curl} (\vel) = \delcross \vel = \left(\begin{array}{c} \partial_y u_z -\partial_z u_y\\\partial_z u_x - \partial_x u_z\\\partial_x u_y - \partial_y u_x\end{array}\right).
\end{equation}
The triple outer product of $\vel$ and $\del$ can be written using the following identity (from \citet{harlen14c6})
\begin{equation}
\label{eq:vecid1}
\vel\times (\delcross \vel)=\inv{2}\del(\vel^2) -\vel\cdot\del\vel.
\end{equation}
\clearpage
\section{Dynamics of Vortices}
The general motion of a fluid is given by the Navier-Stokes equations, which are usually written in vector form:
\begin{equation}
\label{eq:navierstokes}
\partial_t\vel + \vel\cdot\del\vel = \nu \del^2 \vel -\inv{\rho} \del P,
\end{equation}
where $\rho$ is the fluid density, $P$ is the pressure and $\nu$ is the \emph{kinematic viscosity}, $\nicefrac{\mu}{\rho}$.
When working with incompressible fluids, we also use the continuity equation
\begin{equation}
\label{eq:cont}
\deldot \vel =0.
\end{equation}
This can be reworked to find an equation for the local vorticity of the fluid by remembering that the vorticity $\bo$ is equivalent to the vector curl of the velocity $\vel$.
Following \citet{harlen14c6}, we can rewrite \cref{eq:navierstokes} using the identity in \cref{eq:vecid1}
\begin{equation}
\label{eq:navs2}
\partial_t \vel + \inv{2}\del(\vel^2) - \vel\times (\delcross \vel) = \nu \del^2 \vel -\inv{\rho} \del P.
\end{equation}
Taking the curl of the whole equation, we obtain
\begin{equation}
\label{eq:veq1}
\partial_t \delcross\vel + \inv{2}\delcross(\del(\vel^2))-\delcross(\vel\times (\delcross \vel))=\delcross(\nu \del^2 \vel) -\delcross(\inv{\rho} \del P).
\end{equation}
Now rewriting in terms of $\bo$ and noting that the outer product of $\del$ with a gradient is zero always, many parts of \cref{eq:veq1} disappear and we are left with
\begin{equation}
\label{eq:veq2}
\partial_t \bo -\delcross(\vel\times\bo) = \nu \del^2 \bo.
\end{equation}
Using another expanded version of the triple vector product (again following \citet{harlen14c6})
\begin{equation}
\delcross(\vel \times \bo) = \vel\deldot\bo+\bo\cdot\del\vel -\bo\deldot\vel - \vel\cdot\del\bo,
\label{eq:vecid2}
\end{equation}
we can then write 
\begin{equation}
\partial_t \bo -  \vel\deldot\bo+\bo\cdot\del\vel -\bo\deldot\vel - \vel\cdot\del\bo = \nu \del^2 \bo.
\label{eq:veq3}
\end{equation}
However, from the continuity equation (\cref{eq:cont}) we have that $\deldot\vel=0$, and we also know that the divergence of a curl is always zero, so as $\bo=\delcross\vel$, $\deldot\bo=0$.
This means we can further rewrite \cref{eq:veq3} to obtain
\begin{equation}
\partial_t \bo + \vel\cdot \del\bo = \bo \cdot \del\vel + \nu \del^2 \bo,
\label{eq:veq4}
\end{equation}
the \emph{vorticity equation}.
When working in two dimensions only, which will be the situation for this piece of work, this equation reduces to a scalar equation in $\omega(x,y,t)$, the scalar vorticity (noting that the velocity $\vel$ and differential operator $\del$ are now two dimensional also)
\begin{equation}
\partial_t \omega + \vel \cdot \del \omega = \nu \del^2 \omega
\label{eq:2dveq}
\end{equation}
\citep{wayne11}.
Such vortices are known as \emph{Line Vortices}, comprising a central singularity with a circulation speed at distance $r$ of $\nicefrac{k}{r}$, for constant $k$, the vortex strength.
\subsection{Streamfunctions}
When considering line vortices in two dimensional, incompressible flow, it is generally simpler to use a \emph{streamfunction} to describe the fluid motion. 
We define a scalar function $\psi(x,y)$, known as the streamfunction, which is related to the fluid velocity via partial differentials
\begin{align}
\label{eq:sf1}
u_x=\partial_y \psi,\\
\label{eq:sf2}
u_y=-\partial_x \psi
\end{align}
\citep{harlen14c1}.
\clearpage
\section{Vortex Interactions}\label{sec:vortint}
In \citeyear{helmholtz67}, \citeauthor{helmholtz67} found that two vortices in the same plane (as in the two-dimensional flow we consider here) would each affect the other, with the form of the effect depending on the polarity and strength of each vortex.
\citeauthor{helmholtz67} described the interaction of two vortex ``rings'' ( toroidal vortex tubes in three dimensions \citep{saffman92}) as a ``game'', noting that, providing there was not too great a difference in the velocities of the vortices, they would travel together in the same direction with an oscillating motion.
This interaction can be thought of as the interaction of two smoke rings.
One ring shrinks, and passes through the other, before widening and slowing.
The second ring then shrinks, speeds up and passes through the first, before the cycle repeats again.
If a horizontal cross-section is taken through the toroid, and if the radius of the toroidal section (i.e. the radius of the vortex tube itself, not the ring radius) is taken infinitesimally small, the system becomes one of line vortices in a single plane. 
A single ring of diameter $\alpha$, centred at $y=0$, would represent as two line vortices at $\pm\alpha$, with the same strength but opposite polarity.
These two vortices would move in the $x$ direction together, as the ring would move in space, and this system can be shown by using the MATLAB code in \hyperref[sec:ap1]{Appendix 1}, with the following inputs:
\begin{verbatim}
EDU>> vortexleap
number of vortices = 2
number of time steps = 10000
step-size = 0.01
strength of vortex 1 = 1
strength of vortex 2 = -1
x-coordinate of vortex 1 = 0
y-coordinate of vortex 1 = 0.5
x-coordinate of vortex 2 = 0
y-coordinate of vortex 2 = -0.5
\end{verbatim}
giving \cref{fig:2vort}.
\begin{figure}[ht]
\centering
\newlength\figureheight 
\newlength\figurewidth 
\setlength\figureheight{10cm} 
\setlength\figurewidth{\textwidth}
\input{2vortex.tikz}
\caption{Interaction of two line vortices of opposite polarity, representing a single vortex ring.}
\label{fig:2vort}
\end{figure}

The mentions of the two-ring interaction made by \citeauthor{helmholtz67} were not a detailed description of the motion of this system, and this was the basis for a paper by \citet{love94}.
\citeauthor{love94} modelled the system also as that of line vortices, with two pairs symmetrically arranged about the $x$ axis, with opposite polarity either side, as in \cref{fig:lovevort}.
\begin{figure}[ht]
\includegraphics[width=\textwidth]{lovevort}
\caption{Point vortex pairs (A-B and C-D), from \citet[p.186]{love94}}
\label{fig:lovevort}
\end{figure}
If the vortex coordinates are
\begin{align}
\mbox{A:}&\hspace{0.5in}(x_0,y_0)\\
\mbox{B:}&\hspace{0.5in}(x_0,-y_0)\\
\mbox{C:}&\hspace{0.5in}(x_1,y_1)\\
\mbox{D:}&\hspace{0.5in}(x_1,-y_1)
\end{align}
and they have strengths $k$ (A and C) and $-k$ (B and D), then the motion of the fluid can be described by a streamfunction
\begin{equation}
\psi(x,y)=\frac{k}{2\pi}\ln\left(\frac{(x-x_0)^2 + (y+y_0)^2}{(x-x_0)^2 + (y-y_0)^2}\right) + \frac{k}{2\pi}\ln\left(\frac{(x-x_1)^2 + (y+y_1)^2}{(x-x_1)^2 + (y-y_1)^2}\right)
\label{eq:lovesf}
\end{equation}
\citep{love94}.

Now, with the aim of using the definitions of the streamfunction (\cref{eq:sf1,eq:sf2}), we differentiate \cref{eq:lovesf} with respect to $y$
\begin{align}
\partial_y \psi = \frac{k}{\pi}&\left(\frac{y+y_0}{(x-x_0)^2 + (y+y_0)^2}-\frac{y-y_0}{(x-x_0)^2 + (y-y_0)^2}\right.\nonumber\\&\left.+\frac{y+y_1}{(x-x_1)^2 + (y+y_1)^2}-\frac{y-y_1}{(x-x_1)^2 + (y-y_1)^2}\right),
\label{eq:lovesfdy}
\end{align}
and with respect to $x$
\begin{align}
\partial_x \psi = \frac{k}{\pi}&\left(\frac{x-x_0}{(x-x_0)^2 + (y+y_0)^2}-\frac{x-x_0}{(x-x_0)^2 + (y-y_0)^2}\right.\nonumber\\&\left.+\frac{x-x_1}{(x-x_1)^2 + (y+y_1)^2}-\frac{x-x_1}{(x-x_1)^2 + (y-y_1)^2}\right).
\label{eq:lovesfdx}
\end{align}
Now, we wish for the velocity of the vortices, i.e. we are looking at the above equations when $(x,y)\rightarrow(x_0,y_0)$ (or $(x_1,y_1)$).
Again following \citeauthor{love94}, we remove the terms that become infinite in this limit, thus obtaining,
\begin{alignat}{3}
\frac{dx_0}{dt}&=u_x (x_0,y_0)&&= \frac{k}{\pi} \left( \frac{y_1-y_0}{(x_0-x_1)^2 + (y_0-y_1)^2} + \frac{y_0+y_1}{(x_0-x_1)^2 +(y_0+y_1)^2} + \frac{1}{2y_0}\right),\label{eq:vortm1}\\
\frac{dy_0}{dt}&=u_y (x_0,y_0)&&= \frac{k}{\pi} \left( \frac{x_0-x_1}{(x_0-x_1)^2 + (y_0-y_1)^2} + \frac{x_1-x_0}{(x_0-x_1)^2 +(y_0+y_1)^2}\right),\label{eq:vortm2}\\
\frac{dx_1}{dt}&=u_x (x_1,y_1)&&= \frac{k}{\pi} \left( \frac{y_0-y_1}{(x_1-x_0)^2 + (y_1-y_0)^2} + \frac{y_1+y_0}{(x_1-x_0)^2 +(y_1+y_0)^2} + \frac{1}{2y_1}\right),\label{eq:vortm3}\\
\frac{dy_1}{dt}&=u_y (x_1,y_1)&&= \frac{k}{\pi} \left( \frac{x_1-x_0}{(x_1-x_0)^2 + (y_1-y_0)^2} + \frac{x_0-x_1}{(x_1-x_0)^2 +(y_1+y_0)^2} \right).\label{eq:vortm4}
\end{alignat}

\citet{acheson00} generalised these equations for a system of $N$ vortices (with strengths $k_1,k_2,\ldots ,k_N$), obtaining
\begin{align}
\frac{dx_i}{dt}&=\sum^N_{\substack{j=1\\j\neq i}} k_j \frac{y_j - y_i}{(x_i - x_j)^2 + (y_i - y_j)^2},\label{eq:vortposx}\\
\frac{dy_i}{dt}&=\sum^N_{\substack{j=1\\j\neq i}} k_j \frac{x_i - x_j}{(x_i - x_j)^2 + (y_i - y_j)^2},\label{eq:vortposy}
\end{align}
for $i=1,2,\ldots, N$.
The summations in \cref{eq:vortposx,eq:vortposy} are implemented in \hyperref[vortexf]{\texttt{vortexf.m}} and \hyperref[vortexg]{\texttt{vortexg.m}} respectively.
\clearpage
\section{Conserved Quantities}
Returning to \crefrange{eq:vortm1}{eq:vortm4}, and again following \citeauthor{love94}, we can express the right hand side of each equation by the partial derivative of some function $\chi$ (related to the streamfunction $\psi$), with respect to one of the vortex coordinates $x_0,x_1,y_0$ or $y_1$.
\citeauthor{love94} says the differential equations can ``clearly'' be put in the form
\begin{equation}
\frac{dx_0}{\partial_{y_0}\chi}=\frac{dy_0}{-\partial_{x_0}\chi}=\frac{dx_1}{\partial_{y_1}\chi}=\frac{dy_1}{-\partial_{x_1}\chi}=dt.
\label{eq:lovechi}
\end{equation}
To find this function, we must integrate the right hand side as follows:
\begin{itemize}
\item For \cref{eq:vortm1}, we integrate with respect to $y_0$:
\begin{align}
\chi &=\frac{k}{\pi}\int \frac{y_1-y_0}{(x_0-x_1)^2 + (y_0-y_1)^2} + \frac{y_0+y_1}{(x_0-x_1)^2 +(y_0+y_1)^2} + \frac{1}{2y_0}dy_0\nonumber\\
&= \frac{k}{\pi} \left( \int  \frac{y_1-y_0}{(x_0-x_1)^2 + (y_0-y_1)^2} dy_0\right. \nonumber\\ &\left.+ \int \frac{y_0+y_1}{(x_0-x_1)^2 +(y_0+y_1)^2} dy_0 + \int \frac{1}{2y_0}dy_0\right)\nonumber\\
&= \frac{k}{\pi}\left(\inv{2} \ln y_0 -\half \ln((y_0-y_1)^2 +(x_0-x_1)^2)\right. \nonumber\\ &\left. + \half \ln((y_0+y_1)^2 +(x_0-x_1)^2) +C_1 (x_0,x_1,y_1)\right)\nonumber\\
&=\frac{k}{2\pi}\ln y_0 \frac{(y_0+y_1)^2 +(x_0-x_1)^2}{(y_0-y_1)^2 +(x_0-x_1)^2}\cdot C_1 \label{eq:chi1}
\end{align}
\item For \cref{eq:vortm2}, we integrate the negative with respect to $x_0$:
\begin{align}
\chi &=-\frac{k}{\pi}\int \frac{x_0-x_1}{(x_0-x_1)^2 + (y_0-y_1)^2} - \frac{x_0-x_1}{(x_0-x_1)^2 +(y_0+y_1)^2} dx_0\nonumber\\
&= -\frac{k}{\pi} \left( \int  \frac{x_0-x_1}{(x_0-x_1)^2 + (y_0-y_1)^2} dx_0\right. \nonumber\\ &\left.- \int \frac{x_0-x_1}{(x_0-x_1)^2 +(y_0+y_1)^2} dx_0 \right)\nonumber\\
&= \frac{k}{\pi}\left(-\half \ln((y_0-y_1)^2 +(x_0-x_1)^2)\right. \nonumber\\ &\left. + \half \ln((y_0+y_1)^2 +(x_0-x_1)^2) +C_2 (x_1,y_0,y_1)\right)\nonumber\\
&=\frac{k}{2\pi}\ln \frac{(y_0+y_1)^2 +(x_0-x_1)^2}{(y_0-y_1)^2 +(x_0-x_1)^2}\cdot C_2 \label{eq:chi2}
\end{align}
\item For \cref{eq:vortm3}, we integrate with respect to $y_1$:
\begin{align}
\chi &=\frac{k}{\pi}\int \frac{y_0-y_1}{(x_1-x_0)^2 + (y_1-y_0)^2} + \frac{y_0+y_1}{(x_1-x_0)^2 +(y_0+y_1)^2} + \frac{1}{2y_1}dy_1\nonumber\\
&= \frac{k}{\pi} \left( \int  \frac{y_0-y_1}{(x_1-x_0)^2 + (y_1-y_0)^2} dy_1\right. \nonumber\\ &\left.+ \int \frac{y_0+y_1}{(x_1-x_0)^2 +(y_0+y_1)^2} dy_1 + \int \frac{1}{2y_1}dy_0\right)\nonumber\\
&= \frac{k}{\pi}\left(\inv{2} \ln y_1 -\half \ln((y_1-y_0)^2 +(x_1-x_0)^2)\right. \nonumber\\ &\left. + \half \ln((y_0+y_1)^2 +(x_1-x_0)^2) +C_3 (x_0,x_1,y_0)\right)\nonumber\\
&=\frac{k}{2\pi}\ln y_1 \frac{(y_0+y_1)^2 +(x_1-x_0)^2}{(y_1-y_0)^2 +(x_1-x_0)^2} \cdot C_3\label{eq:chi3}
\end{align} 
\item For \cref{eq:vortm4}, we integrate the negative with respect to $x_1$:
\begin{align}
\chi &=-\frac{k}{\pi}\int \frac{x_1-x_0}{(x_1-x_0)^2 + (y_1-y_0)^2} - \frac{x_1-x_0}{(x_1-x_0)^2 +(y_0+y_1)^2} dx_1\nonumber\\
&= -\frac{k}{\pi} \left( \int  \frac{x_1-x_0}{(x_1-x_0)^2 + (y_1-y_0)^2} dx_1\right. \nonumber\\ &\left.- \int \frac{x_1-x_0}{(x_1-x_0)^2 +(y_0+y_1)^2} dx_1 \right)\nonumber\\
&= \frac{k}{\pi}\left(-\half \ln((y_1-y_0)^2 +(x_1-x_0)^2)\right. \nonumber\\ &\left. + \half \ln((y_0+y_1)^2 +(x_1-x_0)^2) +C_4 (x_0,y_0,y_1)\right)\nonumber\\
&=\frac{k}{2\pi}\ln \frac{(y_0+y_1)^2 +(x_1-x_0)^2}{(y_1-y_0)^2 +(x_1-x_0)^2} \cdot C_4\label{eq:chi4}
\end{align}
\end{itemize}
Combining \crefrange{eq:chi1}{eq:chi4}, we can see that, noting that $(a-b)^2 = (b-a)^2$, we must have, for all equations to be equal
\begin{align}
C_1&=y_1,\\
C_3&=y_0,\\
C_2&=C_4=y_1 y_2.
\end{align}
So, we have
\begin{equation}
\chi=\frac{k}{2\pi}\ln y_0 y_1 \frac{(y_0+y_1)^2 +(x_1-x_0)^2}{(y_1-y_0)^2 +(x_1-x_0)^2},
\label{eq:chifinal}
\end{equation}
which agrees with the result in \citet{love94}.
\subsection{Momentum}
Now we have found $\chi$, we can use it (with \crefrange{eq:vortm1}{eq:vortm4}) to derive the conserved analogues of energy and momentum, as stated in \citet{love94}.
Equating the $dy_0$ and the $dy_1$ parts of \cref{eq:lovechi}, we have
\begin{align}
\frac{dy_0}{-\partial_{x_0}\chi} &= \frac{-dy_1}{\partial_{x_1} \chi},\\
\Rightarrow \frac{\dxb \chi\cdot dy_0 + \dxa \chi \cdot dy_1}{-\partial_{x_0 x_1} \chi}&=0,\\
\Rightarrow \dxb \chi\cdot dy_0 + \dxa \chi \cdot dy_1 &=0,
\end{align}
and, as
\begin{equation}
\dxb \chi = \dxa \chi,
\end{equation}
we have
\begin{align}
dy_0 + dy_1 &=0\\
\Rightarrow y_0 + y_1 &= \mbox{CONSTANT}\label{eq:consty},
\end{align}
as in the paper.
This equation shows that the centre of the rotating system composed of the vortices A and C, or the vortices B and D (expressed by \citeauthor{love94} as the midpoint ot the line between the vortices) moves parallel to the $x$ axis at constant distance (provided the conditions set earlier by \citeauthor{helmholtz67} for the ring vortices are met).

\citet{musgrave09} give an analogue to momentum for systems of N point vortices as
\begin{equation}
\bm{p}=\half \rho \left(\begin{array}{c} \sum_i^N k_i y_i\\-\sum_{i}^N k_i x_i\\0\end{array}\right).
\end{equation}
We are interested in momentum in the $x$ direction, so we use, for each vortex pair
\begin{align}
p_x&=\half \rho \sum_{i=0}^1 k_i y_i,\\
&= \half \rho (k_0 y_0 + k_1 y_1).
\end{align}
From the definition of our system, we have $k_0=k_1=\mbox{constant}$ and $\rho=\mbox{constant}$, and from \cref{eq:consty} we have that $y_0 + y_1 = \mbox{constant}$, therefore we have $p_x=\mbox{constant}$.
This means momentum is conserved.
\subsection{Speed of Translation}
The centrepoint can be used to find the speed of translation of the vortex pairs (provided we have stable interactions).
Thinking again of the midpoint $m$ of the line between the two vortices, the $x$ coordinate of $m$ is given by
\begin{equation}
m_x=\frac{x_0 +x_1}{2}.
\end{equation}
The speed (in the $x$ direction) of this midpoint is given by
\begin{align}
\frac{dm_x}{dt}&=\frac{d}{dt}\left(\frac{x_0 +x_1}{2}\right),\\
&=\half \left(\frac{dx_0}{dt} + \frac{dx_1}{dt}\right),\\
&=\frac{k}{2\pi}\left(\frac{2y_1+2y_0}{(x_0-x_1)^2 + (y_0+y_1)^2} + \inv{2y_0} + \inv{2y_1}\right)
\end{align}
i.e. the average velocity of the two vortices.
\subsection{Energy}
Now equating the $y_0$ and the $x_0$ terms, we have 
\begin{align}
\frac{dy_0}{-\dxa \chi} &= \frac{dx_0}{\dya \chi},\\
\Rightarrow \dya \chi dy_0 &= - \dxa \chi dx_0,\\
\Rightarrow \int d\chi &= - \int d\chi,\\
\Rightarrow \chi +c_1 &= -\chi + c_2,\\
\Rightarrow \chi &= \mbox{CONSTANT}.
\end{align}
$\chi$ represents the energy of the fluid \citep{love94}, so we have conservation of energy in the system, provided we ignore viscous effects.
\clearpage
\section{Conditions for Leapfrogging}
\citet{love94} quantified the conditions necessary for the ``leapfrogging'' described in \cref{sec:vortint} to take place.
At the point when all four vortices are aligned vertically, we define $y_{0,a}$ and $y_{1,a}$ to be the $y$ coordinates of vortices A and C respectively.
From \cref{eq:consty} we have that
\begin{equation}
y_{0,a} + y_{0,b} = const = 2c,\label{eq:lc1}
\end{equation}
where the factor of two in the second equality is to simplify later expressions.
As we found $\chi$ constant, we also have (as $x_0=x_1$) 
\begin{equation}
y_{0,a}y_{0,b}\left(\frac{y_{0,a} + y_{0,b}}{y_{0,a}-y_{0,b}}\right)^2 = const = a^2,\label{eq:lc2}
\end{equation}
where again the second equality is to simplify later expressions.

For periodicity of the motion we must have $a^2 > c^2$ \citep{love94}.
Feeding \cref{eq:lc1,eq:lc2} into this condition, we obtain a condition in terms of our $y$ coordinates:
\begin{equation}
6y_{0,a}y_{1,a}-y_{0,a}^2 -y_{1,a} >0 \label{eq:lc3}.
\end{equation}
We are looking for a ratio of $y_{1,a}/y_{0,a}$, so we set $y_{1,a}=1$
in \cref{eq:lc3} and solve for $y_{0,a}$, obtaining
$y_{1,a}/y_{0,a}<3+2\sqrt{2}$.  This is the condition for periodic
leapfrogging.  In the following section, and the MATLAB code, we use
notation closer to \citeauthor{acheson00}, and thus, keeping the $y$ coordinates of one pair as $1,-1$, we define
$\alpha=y_{0,a}/y_{1,a}> 3-2\sqrt{2}$, the reciprocal of the previous ratio.
\clearpage
\section{Stability of the 4-Vortex System}
Taking a system of four point vortices, as in \citet{acheson00}, a program (\hyperref[sec:ap1]{Appendix I}) was written in MATLAB to numerically integrate the vortex velocity equations (\cref{eq:vortposx,eq:vortposy}).
The scheme used is a fourth order Runge-Kutta method with static step size.
The question to be answered is, what values of $\alpha$ permit stable, periodic leapfrogging when the initial position of one vortex is slightly perturbed.
\citeauthor{acheson00} presents four regions of (in)stability, and the below list contains figures from his \citeyear{acheson00} paper, as well as those generated in the MATLAB program.
There appears to be numerical instabilities generated in the numerical integration used, which contribute most of the variation from the figures by \citeauthor{acheson00}.
\begin{description}
\item[No leapfrogging ($\alpha<3-2\sqrt{2}\approx
0.172$)]
In this region, as per the condition in \cref{eq:lc3}, there is no leapfrogging and the vortices merely translate, as can be seen in \cref{fig:noleap}.
\begin{figure}[ht]
\centering
\setlength\figureheight{7.5cm} 
\setlength\figurewidth{\textwidth}
% This file was created by matlab2tikz.
% Minimal pgfplots version: 1.3
%
%The latest updates can be retrieved from
%  http://www.mathworks.com/matlabcentral/fileexchange/22022-matlab2tikz
%where you can also make suggestions and rate matlab2tikz.
%
\definecolor{mycolor1}{rgb}{0.00000,1.00000,1.00000}%
%
\begin{tikzpicture}

\begin{axis}[%
width=0.95092\figurewidth,
height=\figureheight,
at={(0\figurewidth,0\figureheight)},
scale only axis,
every outer x axis line/.append style={black},
every x tick label/.append style={font=\color{black}},
xmin=0,
xlabel={x},
every outer y axis line/.append style={black},
every y tick label/.append style={font=\color{black}},
ymin=-1.5,
ymax=1.5,
ylabel={y},
title={10000 time steps, step size = 0.005},
axis x line*=bottom,
axis y line*=left
]
\addplot [color=blue,solid,forget plot]
  table[row sep=crcr]{%
0	1\\
0.000953612251194968	0.999991569056121\\
0.00190798729678	0.999963601161843\\
0.00286396019577596	0.999916173293784\\
0.0038223562787305	0.999849420394275\\
0.00478398574829361	0.999763533861643\\
0.0057496385504451	0.99965875939445\\
0.00672007960576098	0.999535394245381\\
0.00769604447755355	0.999393783954496\\
0.00867823553891636	0.999234318642431\\
0.00966731868445496	0.999057428951598\\
0.0106639206156271	0.9988635817273\\
0.0116686267119624	0.998653275531109\\
0.0126819794847081	0.998427036076018\\
0.0137044775952622	0.998185411667346\\
0.014736575408559	0.997928968725522\\
0.0157786830416588	0.997658287457446\\
0.0168311668603032	0.997373957732574\\
0.0178943503711121	0.997076575208879\\
0.0189685154542912	0.996766737742879\\
0.0200539038809719	0.996445042107388\\
0.0211507190603203	0.996112081030925\\
0.0222591279640144	0.995768440564039\\
0.0233792631792637	0.995414697770303\\
0.0245112250459113	0.995051418733452\\
0.0256550838380166	0.994679156867197\\
0.0268108819554154	0.994298451510375\\
0.0279786360958673	0.99390982678745\\
0.0291583393833613	0.993513790712613\\
0.0303499634328282	0.993110834514872\\
0.0315534603358096	0.992701432161321\\
0.0327687645555044	0.992286040056183\\
0.0339957947230267	0.99186509689403\\
0.0352344553296577	0.991439023646738\\
0.0364846383123741	0.991008223665093\\
0.0377462245320113	0.990573082877498\\
0.0390190851451084	0.990133970069768\\
0.0403030828718157	0.989691237231613\\
0.0415980731632802	0.989245219956958\\
0.0429039052726805	0.98879623788673\\
0.044220423234625	0.988344595184161\\
0.0455474667579683	0.987890581033938\\
0.0468848720372929	0.987434470157766\\
0.0482324724883683	0.986976523339946\\
0.0495900994128629	0.98651698795759\\
0.0509575825974728	0.986056098510929\\
0.0523347508524641	0.985594077149959\\
0.0537214324944103	0.985131134194338\\
0.0551174557776738	0.98466746864402\\
0.0565226492789171	0.984203268678633\\
0.0579368422386703	0.983738712144056\\
0.0593598648637108	0.98327396702498\\
0.060791548593747	0.982809191902609\\
0.0622317263356417	0.982344536396883\\
0.0636802326681619	0.981880141592852\\
0.0651369040200065	0.981416140451011\\
0.066601578823637	0.980952658201549\\
0.0680740976472316	0.980489812722622\\
0.0695543033068806	0.980027714902813\\
0.0710420409609634	0.979566468988088\\
0.0725371581884741	0.979106172913558\\
0.0740395050529106	0.978646918620451\\
0.0755489341531933	0.978188792358722\\
0.0770653006629513	0.97773187497574\\
0.0785884623593899	0.977276242191551\\
0.0801182796428433	0.976821964861173\\
0.0816546155480142	0.976369109224434\\
0.0831973357478098	0.975917737143837\\
0.0847463085505986	0.975467906330936\\
0.0863014048916339	0.97501967056171\\
0.0878624983193236	0.974573079881413\\
0.0894294649769562	0.974128180799347\\
0.0910021835804397	0.973685016474027\\
0.092580535392554	0.97324362688916\\
0.0941644041941709	0.972804049020873\\
0.0957536762528503	0.97236631699659\\
0.0973482402891837	0.971930462245957\\
0.0989479874412172	0.971496513644193\\
0.100552811227255	0.971064497648223\\
0.102162607507317	0.970634438425961\\
0.103777274443486	0.970206357979052\\
0.105396712459376	0.969780276259413\\
0.107020824198905	0.969356211279867\\
0.108649514484559	0.968934179219167\\
0.110282690275297	0.968514194521683\\
0.111920260624243	0.96809626999203\\
0.113562136636292	0.967680416884871\\
0.11520823142573	0.967266644990151\\
0.116858460073987	0.966854962713989\\
0.118512739587591	0.966445377155438\\
0.120170988856418	0.966037894179336\\
0.121833128612288	0.965632518485425\\
0.123499081387987	0.965229253673952\\
0.12516877147675	0.964828102307902\\
0.126842124892257	0.964429065972059\\
0.128519069329185	0.964032145329033\\
0.130199534124343	0.963637340172433\\
0.131883450218418	0.963244649477297\\
0.133570750118359	0.962854071447955\\
0.135261367860419	0.962465603563423\\
0.136955238973864	0.962079242620478\\
0.138652300445369	0.961694984774514\\
0.140352490684108	0.961312825578304\\
0.142055749487534	0.960932760018763\\
0.143762018007875	0.960554782551828\\
0.145471238719322	0.960178887135538\\
0.147183355385936	0.959805067261404\\
0.148898313030239	0.959433315984174\\
0.150616057902523	0.959063625950049\\
0.152336537450844	0.95869598942345\\
0.154059700291704	0.958330398312393\\
0.155785496181417	0.95796684419256\\
0.157513875988152	0.957605318330112\\
0.159244791664635	0.957245811703321\\
0.160978196221513	0.956888315023083\\
0.162714043701357	0.956532818752353\\
0.16445228915331	0.956179313124574\\
0.166192888608347	0.955827788161139\\
0.167935799055163	0.955478233687942\\
0.169680978416652	0.955130639351056\\
0.17142838552698	0.954784994631594\\
0.173177980109241	0.954441288859778\\
0.174929722753671	0.954099511228272\\
0.176683574896424	0.953759650804807\\
0.178439498798886	0.953421696544129\\
0.180197457527526	0.953085637299323\\
0.181957414934258	0.952751461832518\\
0.183719335637314	0.952419158825028\\
0.185483185002611	0.952088716886938\\
0.187248929125601	0.951760124566185\\
0.189016534813593	0.951433370357127\\
0.190785969568535	0.951108442708666\\
0.192557201570241	0.950785330031909\\
0.194330199660063	0.950464020707415\\
0.196104933324982	0.950144503092041\\
0.197881372682119	0.949826765525408\\
0.199659488463647	0.949510796335996\\
0.201439252002103	0.949196583846911\\
0.203220635216078	0.948884116381309\\
0.205003610596287	0.948573382267521\\
0.206788151191999	0.948264369843881\\
0.208574230597822	0.94795706746327\\
0.210361822940838	0.947651463497397\\
0.212150902868068	0.947347546340828\\
0.213941445534267	0.947045304414774\\
0.215733426590037	0.94674472617065\\
0.217526822170251	0.946445800093421\\
0.219321608882775	0.946148514704736\\
0.221117763797485	0.945852858565872\\
0.222915264435571	0.945558820280494\\
0.224714088759116	0.945266388497231\\
0.226514215160944	0.94497555191209\\
0.228315622454732	0.944686299270713\\
0.230118289865379	0.944398619370478\\
0.231922197019621	0.944112501062461\\
0.233727323936885	0.943827933253256\\
0.235533651020384	0.943544904906676\\
0.237341159048436	0.943263405045319\\
0.239149829166002	0.942983422752028\\
0.240959642876453	0.942704947171234\\
0.242770582033528	0.942427967510198\\
0.244582628833514	0.942152473040151\\
0.24639576580761	0.941878453097344\\
0.2482099758145	0.941605897083998\\
0.250025242033095	0.941334794469186\\
0.251841547955475	0.941065134789615\\
0.253658877379999	0.940796907650349\\
0.255477214404587	0.94053010272545\\
0.257296543420181	0.940264709758552\\
0.259116849104352	0.940000718563368\\
0.260938116415083	0.939738119024145\\
0.262760330584697	0.939476901096048\\
0.264583477113936	0.939217054805497\\
0.266407541766191	0.938958570250447\\
0.268232510561869	0.938701437600627\\
0.270058369772903	0.938445647097719\\
0.271885105917392	0.938191189055505\\
0.273712705754377	0.937938053859964\\
0.275541156278742	0.937686231969335\\
0.277370444716237	0.93743571391414\\
0.279200558518625	0.937186490297171\\
0.281031485358948	0.936938551793447\\
0.282863213126898	0.936691889150135\\
0.284695729924316	0.936446493186446\\
0.286529024060777	0.9362023547935\\
0.288363084049305	0.935959464934167\\
0.290197898602166	0.935717814642881\\
0.292033456626783	0.935477395025433\\
0.293869747221731	0.935238197258739\\
0.295706759672837	0.935000212590594\\
0.29754448344937	0.934763432339398\\
0.299382908200314	0.93452784789387\\
0.301222023750738	0.934293450712746\\
0.303061820098246	0.934060232324454\\
0.30490228740951	0.933828184326782\\
0.306743416016883	0.93359729838653\\
0.308585196415091	0.933367566239147\\
0.310427619258006	0.933138979688355\\
0.312270675355485	0.932911530605771\\
0.314114355670287	0.932685210930506\\
0.315958651315059	0.932460012668761\\
0.317803553549393	0.932235927893419\\
0.31964905377695	0.932012948743617\\
0.321495143542641	0.931791067424323\\
0.323341814529887	0.931570276205896\\
0.325189058557924	0.931350567423644\\
0.327036867579184	0.931131933477383\\
0.328885233676726	0.930914366830976\\
0.330734149061721	0.93069786001188\\
0.332583606071007	0.930482405610684\\
0.334433597164682	0.930267996280645\\
0.336284114923763	0.930054624737216\\
0.338135152047889	0.929842283757579\\
0.33998670135308	0.929630966180168\\
0.341838755769539	0.929420664904196\\
0.343691308339509	0.929211372889173\\
0.345544352215173	0.929003083154434\\
0.347397880656598	0.928795788778651\\
0.34925188702973	0.928589482899358\\
0.351106364804425	0.928384158712466\\
0.352961307552526	0.928179809471785\\
0.354816708945984	0.927976428488538\\
0.356672562755013	0.927774009130881\\
0.358528862846291	0.927572544823426\\
0.360385603181194	0.927372029046753\\
0.362242777814071	0.927172455336937\\
0.364100380890556	0.926973817285066\\
0.365958406645911	0.926776108536766\\
0.367816849403407	0.926579322791723\\
0.369675703572744	0.926383453803209\\
0.371534963648494	0.926188495377613\\
0.373394624208582	0.925994441373965\\
0.375254679912804	0.925801285703468\\
0.377115125501361	0.925609022329037\\
0.37897595579344	0.925417645264826\\
0.380837165685813	0.925227148575772\\
0.382698750151471	0.925037526377134\\
0.38456070423828	0.924848772834031\\
0.386423023067672	0.924660882160995\\
0.388285701833356	0.924473848621513\\
0.390148735800059	0.92428766652758\\
0.392012120302291	0.92410233023925\\
0.393875850743138	0.923917834164198\\
0.395739922593073	0.923734172757273\\
0.397604331388797	0.923551340520063\\
0.3994690727321	0.923369332000461\\
0.401334142288749	0.923188141792231\\
0.403199535787389	0.923007764534581\\
0.405065249018478	0.922828194911739\\
0.40693127783323	0.922649427652528\\
0.408797618142595	0.922471457529949\\
0.410664265916239	0.922294279360765\\
0.412531217181565	0.922117888005087\\
0.414398468022736	0.921942278365967\\
0.416266014579727	0.921767445388991\\
0.418133853047393	0.921593384061878\\
0.420001979674554	0.921420089414078\\
0.421870390763098	0.921247556516379\\
0.423739082667103	0.921075780480514\\
0.425608051791974	0.920904756458771\\
0.427477294593596	0.920734479643607\\
0.429346807577507	0.920564945267268\\
0.431216587298085	0.920396148601409\\
0.433086630357744	0.920228084956716\\
0.434956933406159	0.920060749682539\\
0.436827493139491	0.919894138166517\\
0.438698306299638	0.919728245834218\\
0.440569369673493	0.919563068148775\\
0.442440680092219	0.919398600610526\\
0.444312234430538	0.919234838756659\\
0.446184029606031	0.919071778160863\\
0.448056062578452	0.918909414432974\\
0.449928330349059	0.918747743218635\\
0.451800829959947	0.918586760198949\\
0.453673558493406	0.918426461090145\\
0.45554651307128	0.918266841643238\\
0.457419690854347	0.918107897643698\\
0.459293089041702	0.917949624911124\\
0.461166704870157	0.917792019298912\\
0.463040535613647	0.917635076693937\\
0.464914578582657	0.917478793016233\\
0.466788831123644	0.917323164218675\\
0.468663290618482	0.917168186286665\\
0.470537954483914	0.917013855237824\\
0.472412820171009	0.916860167121685\\
0.474287885164634	0.916707118019385\\
0.476163146982935	0.91655470404337\\
0.478038603176823	0.916402921337092\\
0.479914251329476	0.916251766074716\\
0.481790089055844	0.916101234460829\\
0.483666114002162	0.915951322730149\\
0.485542323845481	0.915802027147238\\
0.487418716293196	0.915653344006223\\
0.489295289082586	0.915505269630513\\
0.491172039980369	0.915357800372518\\
0.493048966782251	0.915210932613382\\
0.494926067312497	0.915064662762705\\
0.496803339423501	0.914918987258276\\
0.498680780995364	0.914773902565807\\
0.500558389935482	0.914629405178669\\
0.502436164178144	0.91448549161763\\
0.504314101684125	0.914342158430599\\
0.506192200440301	0.91419940219237\\
0.50807045845926	0.914057219504367\\
0.509948873778925	0.913915606994396\\
0.51182744446218	0.913774561316397\\
0.513706168596506	0.913634079150198\\
0.515585044293618	0.913494157201272\\
0.517464069689116	0.913354792200499\\
0.519343242942133	0.913215980903927\\
0.521222562234995	0.913077720092536\\
0.523102025772883	0.912940006572008\\
0.524981631783508	0.912802837172493\\
0.526861378516779	0.912666208748385\\
0.528741264244488	0.912530118178095\\
0.530621287259993	0.912394562363826\\
0.532501445877912	0.912259538231354\\
0.534381738433817	0.91212504272981\\
0.536262163283934	0.911991072831463\\
0.538142718804853	0.911857625531505\\
0.540023403393236	0.911724697847842\\
0.541904215465531	0.91159228682088\\
0.543785153457697	0.911460389513322\\
0.545666215824924	0.911329003009961\\
0.547547401041368	0.911198124417476\\
0.549428707599877	0.911067750864232\\
0.551310134011736	0.910937879500081\\
0.553191678806407	0.910808507496166\\
0.555073340531272	0.910679632044725\\
0.55695511775139	0.910551250358898\\
0.558837009049245	0.910423359672538\\
0.560719013024511	0.910295957240019\\
0.562601128293807	0.910169040336055\\
0.564483353490469	0.910042606255508\\
0.566365687264318	0.90991665231321\\
0.568248128281432	0.909791175843781\\
0.570130675223925	0.90966617420145\\
0.572013326789726	0.909541644759874\\
0.573896081692366	0.909417584911969\\
0.575778938660759	0.909293992069731\\
0.577661896439002	0.909170863664067\\
0.579544953786161	0.909048197144624\\
0.581428109476073	0.908925989979621\\
0.583311362297144	0.908804239655681\\
0.585194711052153	0.90868294367767\\
0.587078154558059	0.908562099568528\\
0.588961691645812	0.908441704869113\\
0.590845321160163	0.908321757138039\\
0.59272904195948	0.908202253951516\\
0.594612852915566	0.908083192903198\\
0.596496752913483	0.907964571604024\\
0.598380740851372	0.907846387682065\\
0.600264815640282	0.907728638782377\\
0.602148976203996	0.907611322566842\\
0.604033221478866	0.90749443671403\\
0.605917550413648	0.907377978919044\\
0.607801961969333	0.907261946893375\\
0.609686455118992	0.907146338364764\\
0.611571028847618	0.907031151077051\\
0.613455682151963	0.906916382790042\\
0.615340414040394	0.906802031279361\\
0.617225223532734	0.90668809433632\\
0.619110109660115	0.906574569767774\\
0.620995071464836	0.906461455395993\\
0.622880108000212	0.906348749058523\\
0.624765218330432	0.906236448608053\\
0.626650401530424	0.906124551912288\\
0.628535656685712	0.906013056853814\\
0.63042098289228	0.905901961329971\\
0.63230637925644	0.905791263252729\\
0.634191844894699	0.905680960548555\\
0.636077378933627	0.905571051158293\\
0.63796298050973	0.905461533037039\\
0.639848648769323	0.90535240415402\\
0.641734382868407	0.905243662492469\\
0.643620181972543	0.90513530604951\\
0.645506045256733	0.905027332836035\\
0.647391971905302	0.904919740876588\\
0.649277961111775	0.904812528209252\\
0.651164012078768	0.904705692885527\\
0.653050124017869	0.904599232970221\\
0.654936296149528	0.904493146541336\\
0.656822527702942	0.904387431689956\\
0.658708817915952	0.904282086520134\\
0.660595166034929	0.904177109148789\\
0.662481571314672	0.904072497705588\\
0.664368033018301	0.903968250332847\\
0.666254550417153	0.903864365185421\\
0.668141122790684	0.903760840430599\\
0.670027749426365	0.903657674247998\\
0.671914429619581	0.903554864829465\\
0.673801162673542	0.903452410378967\\
0.675687947899176	0.903350309112498\\
0.677574784615043	0.903248559257974\\
0.679461672147237	0.903147159055134\\
0.681348609829293	0.903046106755444\\
0.6832355970021	0.902945400621995\\
0.685122633013808	0.902845038929414\\
0.68700971721974	0.902745019963763\\
0.688896848982307	0.902645342022446\\
0.69078402767092	0.902546003414115\\
0.692671252661904	0.90244700245858\\
0.694558523338418	0.902348337486713\\
0.69644583909037	0.90225000684036\\
0.698333199314336	0.902152008872253\\
0.700220603413481	0.902054341945915\\
0.702108050797477	0.901957004435578\\
0.703995540882428	0.901859994726092\\
0.705883073090792	0.901763311212839\\
0.707770646851306	0.901666952301649\\
0.709658261598906	0.901570916408712\\
0.711545916774662	0.901475201960498\\
0.713433611825697	0.901379807393672\\
0.71532134620512	0.901284731155009\\
0.717209119371953	0.901189971701317\\
0.719096930791061	0.901095527499352\\
0.720984779933083	0.901001397025741\\
0.722872666274367	0.900907578766902\\
0.724760589296897	0.900814071218964\\
0.72664854848823	0.90072087288769\\
0.728536543341434	0.9006279822884\\
0.730424573355015	0.900535397945897\\
0.732312638032861	0.900443118394387\\
0.734200736884177	0.900351142177406\\
0.73608886942342	0.900259467847747\\
0.737977035170241	0.900168093967387\\
0.739865233649426	0.90007701910741\\
0.74175346439083	0.89998624184794\\
0.743641726929326	0.899895760778067\\
0.745530020804741	0.899805574495776\\
0.747418345561803	0.899715681607878\\
0.749306700750082	0.899626080729939\\
0.751195085923933	0.899536770486215\\
0.753083500642446	0.899447749509579\\
0.754971944469385	0.899359016441457\\
0.756860416973141	0.899270569931758\\
0.758748917726673	0.899182408638811\\
0.76063744630746	0.8990945312293\\
0.762526002297449	0.899006936378193\\
0.764414585283001	0.898919622768685\\
0.766303194854844	0.898832589092128\\
0.768191830608022	0.898745834047973\\
0.770080492141847	0.898659356343703\\
0.771969179059848	0.898573154694775\\
0.773857890969728	0.898487227824555\\
0.775746627483311	0.898401574464258\\
0.777635388216502	0.898316193352889\\
0.779524172789234	0.898231083237182\\
0.78141298082543	0.898146242871542\\
0.783301811952951	0.898061671017984\\
0.785190665803557	0.897977366446076\\
0.787079542012863	0.897893327932882\\
0.788968440220293	0.897809554262905\\
0.790857360069038	0.89772604422803\\
0.792746301206018	0.897642796627465\\
0.794635263281836	0.897559810267692\\
0.796524245950738	0.897477083962405\\
0.798413248870574	0.897394616532459\\
0.800302271702758	0.897312406805816\\
0.802191314112225	0.897230453617491\\
0.804080375767398	0.897148755809497\\
0.805969456340144	0.897067312230797\\
0.80785855550574	0.896986121737246\\
0.809747672942833	0.896905183191544\\
0.811636808333404	0.896824495463181\\
0.813525961362731	0.896744057428389\\
0.815415131719354	0.896663867970092\\
0.817304319095035	0.896583925977854\\
0.81919352318473	0.896504230347831\\
0.821082743686549	0.89642477998272\\
0.822971980301721	0.896345573791713\\
0.824861232734565	0.89626661069045\\
0.82675050069245	0.896187889600965\\
0.828639783885767	0.896109409451647\\
0.830529082027895	0.896031169177185\\
0.832418394835165	0.895953167718531\\
0.834307722026834	0.895875404022842\\
0.836197063325049	0.895797877043448\\
0.838086418454817	0.895720585739794\\
0.839975787143976	0.895643529077404\\
0.841865169123161	0.895566706027832\\
0.843754564125777	0.895490115568622\\
0.845643971887968	0.895413756683259\\
0.847533392148588	0.895337628361129\\
0.84942282464917	0.895261729597478\\
0.851312269133902	0.895186059393364\\
0.853201725349594	0.89511061675562\\
0.855091193045651	0.89503540069681\\
0.856980671974049	0.894960410235187\\
0.858870161889303	0.894885644394653\\
0.860759662548441	0.894811102204718\\
0.86264917371098	0.894736782700461\\
0.864538695138896	0.894662684922486\\
0.866428226596601	0.894588807916887\\
0.868317767850917	0.894515150735206\\
0.870207318671047	0.894441712434395\\
0.872096878828553	0.894368492076776\\
0.873986448097334	0.894295488730004\\
0.875876026253593	0.89422270146703\\
0.87776561307582	0.894150129366061\\
0.879655208344768	0.894077771510523\\
0.881544811843424	0.894005626989026\\
0.883434423356989	0.893933694895325\\
0.885324042672856	0.893861974328285\\
0.887213669580585	0.893790464391844\\
0.889103303871881	0.893719164194978\\
0.89099294534057	0.893648072851665\\
0.892882593782583	0.89357718948085\\
0.894772248995924	0.893506513206411\\
0.89666191078066	0.893436043157122\\
0.898551578938888	0.893365778466621\\
0.900441253274725	0.893295718273376\\
0.902330933594278	0.893225861720647\\
0.90422061970563	0.893156207956461\\
0.906110311418814	0.893086756133571\\
0.908000008545799	0.893017505409425\\
0.909889710900465	0.892948454946137\\
0.911779418298585	0.892879603910451\\
0.913669130557806	0.892810951473708\\
0.915558847497631	0.89274249681182\\
0.917448568939395	0.892674239105231\\
0.919338294706254	0.892606177538891\\
0.921228024623158	0.892538311302223\\
0.923117758516839	0.892470639589093\\
0.92500749621579	0.892403161597778\\
0.926897237550248	0.892335876530937\\
0.928786982352175	0.892268783595583\\
0.930676730455242	0.892201882003047\\
0.932566481694809	0.892135170968955\\
0.934456235907913	0.892068649713197\\
0.936345992933245	0.892002317459895\\
0.938235752611136	0.891936173437377\\
0.940125514783543	0.891870216878149\\
0.942015279294026	0.891804447018864\\
0.943905045987738	0.891738863100297\\
0.945794814711407	0.891673464367313\\
0.947684585313319	0.891608250068844\\
0.949574357643302	0.891543219457859\\
0.951464131552714	0.891478371791337\\
0.953353906894423	0.89141370633024\\
0.955243683522796	0.891349222339487\\
0.957133461293679	0.891284919087928\\
0.959023240064388	0.891220795848314\\
0.96091301969369	0.891156851897276\\
0.962802800041791	0.891093086515294\\
0.96469258097032	0.891029498986678\\
0.966582362342315	0.890966088599535\\
0.968472144022209	0.890902854645747\\
0.970361925875818	0.890839796420949\\
0.972251707770324	0.890776913224498\\
0.974141489574265	0.890714204359454\\
0.976031271157518	0.890651669132551\\
0.977921052391288	0.890589306854177\\
0.979810833148093	0.890527116838345\\
0.981700613301755	0.890465098402673\\
0.983590392727382	0.890403250868359\\
0.985480171301357	0.890341573560158\\
0.987369948901329	0.890280065806356\\
0.989259725406195	0.890218726938752\\
0.991149500696091	0.890157556292629\\
0.993039274652379	0.890096553206737\\
0.994929047157635	0.890035717023264\\
0.996818818095636	0.889975047087821\\
0.998708587351352	0.889914542749412\\
1.00059835481093	0.889854203360419\\
1.00248812036168	0.889794028276575\\
1.00437788389208	0.889734016856945\\
1.00626764529173	0.889674168463902\\
1.00815740445139	0.889614482463109\\
1.01004716126292	0.889554958223494\\
1.01193691561931	0.889495595117232\\
1.01382666741463	0.889436392519723\\
1.01571641654405	0.88937734980957\\
1.01760616290383	0.88931846636856\\
1.01949590639128	0.889259741581644\\
1.02138564690477	0.889201174836914\\
1.02327538434373	0.889142765525586\\
1.02516511860862	0.889084513041977\\
1.02705484960093	0.88902641678349\\
1.02894457722317	0.888968476150589\\
1.03083430137884	0.888910690546781\\
1.03272402197246	0.888853059378599\\
1.03461373890953	0.888795582055582\\
1.03650345209653	0.888738257990251\\
1.03839316144091	0.8886810865981\\
1.04028286685108	0.888624067297566\\
1.0421725682364	0.888567199510019\\
1.04406226550717	0.888510482659739\\
1.04595195857462	0.8884539161739\\
1.04784164735092	0.88839749948255\\
1.04973133174913	0.888341232018593\\
1.05162101168324	0.888285113217773\\
1.05351068706811	0.888229142518654\\
1.05540035781953	0.888173319362604\\
1.05729002385413	0.888117643193776\\
1.05917968508943	0.888062113459091\\
1.06106934144381	0.888006729608224\\
1.06295899283651	0.88795149109358\\
1.06484863918762	0.887896397370282\\
1.06673828041806	0.887841447896155\\
1.06862791644957	0.887786642131705\\
1.07051754720475	0.887731979540107\\
1.07240717260698	0.887677459587182\\
1.07429679258046	0.88762308174139\\
1.0761864070502	0.887568845473804\\
1.07807601594199	0.887514750258101\\
1.07996561918241	0.887460795570543\\
1.08185521669881	0.88740698088996\\
1.08374480841932	0.887353305697738\\
1.08563439427282	0.887299769477798\\
1.08752397418897	0.887246371716585\\
1.08941354809816	0.887193111903052\\
1.09130311593152	0.887139989528643\\
1.09319267762092	0.887087004087276\\
1.09508223309896	0.887034155075334\\
1.09697178229895	0.886981441991645\\
1.09886132515493	0.886928864337466\\
1.10075086160163	0.886876421616475\\
1.1026403915745	0.886824113334749\\
1.10452991500967	0.886771939000754\\
1.10641943184395	0.886719898125328\\
1.10830894201485	0.886667990221668\\
1.11019844546055	0.886616214805315\\
1.1120879421199	0.886564571394142\\
1.11397743193239	0.886513059508337\\
1.11586691483819	0.88646167867039\\
1.11775639077811	0.88641042840508\\
1.11964585969362	0.886359308239461\\
1.1215353215268	0.886308317702848\\
1.12342477622037	0.886257456326803\\
1.1253142237177	0.886206723645123\\
1.12720366396275	0.886156119193826\\
1.12909309690011	0.886105642511137\\
1.13098252247497	0.886055293137476\\
1.13287194063313	0.886005070615443\\
1.13476135132098	0.885954974489809\\
1.13665075448551	0.885905004307497\\
1.13854015007429	0.885855159617577\\
1.14042953803548	0.885805439971246\\
1.1423189183178	0.885755844921819\\
1.14420829087055	0.885706374024717\\
1.1460976556436	0.885657026837453\\
1.14798701258736	0.88560780291962\\
1.14987636165282	0.885558701832881\\
1.15176570279149	0.885509723140952\\
1.15365503595546	0.885460866409594\\
1.15554436109733	0.8854121312066\\
1.15743367817024	0.885363517101783\\
1.15932298712788	0.885315023666964\\
1.16121228792442	0.88526665047596\\
1.1631015805146	0.885218397104574\\
1.16499086485364	0.885170263130581\\
1.16688014089728	0.885122248133717\\
1.16876940860177	0.88507435169567\\
1.17065866792387	0.885026573400066\\
1.1725479188208	0.88497891283246\\
1.17443716125032	0.88493136958032\\
1.17632639517063	0.884883943233025\\
1.17821562054044	0.884836633381843\\
1.18010483731894	0.884789439619929\\
1.18199404546578	0.884742361542309\\
1.18388324494109	0.884695398745872\\
1.18577243570546	0.884648550829357\\
1.18766161771993	0.884601817393344\\
1.18955079094602	0.884555198040244\\
1.19143995534569	0.884508692374286\\
1.19332911088135	0.884462300001508\\
1.19521825751584	0.884416020529749\\
1.19710739521246	0.884369853568633\\
1.19899652393493	0.884323798729564\\
1.20088564364743	0.884277855625716\\
1.20277475431454	0.884232023872017\\
1.20466385590126	0.884186303085147\\
1.20655294837305	0.884140692883521\\
1.20844203169573	0.884095192887284\\
1.21033110583558	0.884049802718299\\
1.21222017075927	0.884004522000139\\
1.21410922643387	0.883959350358073\\
1.21599827282687	0.883914287419063\\
1.21788730990614	0.883869332811749\\
1.21977633763994	0.883824486166442\\
1.22166535599695	0.883779747115114\\
1.2235543649462	0.883735115291389\\
1.22544336445714	0.883690590330533\\
1.22733235449956	0.883646171869447\\
1.22922133504367	0.883601859546652\\
1.23111030606003	0.88355765300229\\
1.23299926751955	0.883513551878105\\
1.23488821939355	0.883469555817439\\
1.23677716165369	0.883425664465222\\
1.23866609427198	0.883381877467964\\
1.2405550172208	0.883338194473745\\
1.24244393047289	0.883294615132207\\
1.24433283400132	0.883251139094546\\
1.24622172777953	0.883207766013501\\
1.24811061178129	0.883164495543349\\
1.24999948598071	0.883121327339891\\
1.25188835035224	0.883078261060452\\
1.25377720487067	0.883035296363864\\
1.25566604951111	0.882992432910463\\
1.25755488424901	0.882949670362079\\
1.25944370906014	0.882907008382026\\
1.2613325239206	0.882864446635098\\
1.2632213288068	0.882821984787559\\
1.26511012369547	0.882779622507132\\
1.26699890856365	0.882737359462997\\
1.2688876833887	0.882695195325776\\
1.27077644814829	0.88265312976753\\
1.27266520282038	0.882611162461753\\
1.27455394738325	0.882569293083356\\
1.27644268181547	0.882527521308667\\
1.27833140609591	0.882485846815422\\
1.28022012020373	0.882444269282753\\
1.2821088241184	0.882402788391186\\
1.28399751781966	0.882361403822629\\
1.28588620128754	0.882320115260368\\
1.28777487450236	0.882278922389056\\
1.28966353744472	0.88223782489471\\
1.2915521900955	0.8821968224647\\
1.29344083243585	0.882155914787743\\
1.29532946444721	0.882115101553895\\
1.29721808611126	0.882074382454545\\
1.29910669740999	0.882033757182408\\
1.30099529832563	0.881993225431516\\
1.30288388884067	0.881952786897213\\
1.3047724689379	0.881912441276148\\
1.30666103860031	0.881872188266265\\
1.30854959781121	0.881832027566799\\
1.31043814655412	0.881791958878269\\
1.31232668481283	0.881751981902471\\
1.31421521257137	0.881712096342468\\
1.31610372981404	0.88167230190259\\
1.31799223652537	0.881632598288419\\
1.31988073269013	0.881592985206789\\
1.32176921829335	0.881553462365777\\
1.32365769332027	0.881514029474696\\
1.32554615775639	0.881474686244088\\
1.32743461158745	0.88143543238572\\
1.3293230547994	0.881396267612575\\
1.33121148737844	0.881357191638845\\
1.33309990931099	0.881318204179928\\
1.3349883205837	0.88127930495242\\
1.33687672118344	0.881240493674107\\
1.33876511109732	0.88120177006396\\
1.34065349031263	0.881163133842132\\
1.34254185881693	0.881124584729945\\
1.34443021659796	0.881086122449889\\
1.34631856364369	0.881047746725616\\
1.34820689994231	0.881009457281931\\
1.35009522548219	0.880971253844788\\
1.35198354025195	0.880933136141283\\
1.35387184424039	0.880895103899649\\
1.35576013743652	0.88085715684925\\
1.35764841982955	0.880819294720574\\
1.35953669140891	0.880781517245228\\
1.36142495216421	0.880743824155932\\
1.36331320208527	0.880706215186516\\
1.3652014411621	0.880668690071907\\
1.36708966938491	0.880631248548131\\
1.36897788674409	0.880593890352305\\
1.37086609323023	0.880556615222628\\
1.37275428883411	0.880519422898381\\
1.3746424735467	0.880482313119915\\
1.37653064735916	0.880445285628652\\
1.37841881026281	0.880408340167076\\
1.38030696224918	0.880371476478727\\
1.38219510330996	0.880334694308198\\
1.38408323343705	0.880297993401128\\
1.38597135262248	0.880261373504195\\
1.3878594608585	0.880224834365116\\
1.38974755813751	0.880188375732636\\
1.3916356444521	0.880151997356525\\
1.39352371979501	0.880115698987574\\
1.39541178415917	0.880079480377589\\
1.39729983753766	0.880043341279384\\
1.39918787992375	0.880007281446778\\
1.40107591131085	0.879971300634591\\
1.40296393169254	0.879935398598633\\
1.40485194106259	0.879899575095709\\
1.40673993941489	0.879863829883603\\
1.40862792674351	0.879828162721082\\
1.41051590304267	0.879792573367886\\
1.41240386830677	0.879757061584723\\
1.41429182253033	0.879721627133267\\
1.41617976570805	0.879686269776151\\
1.41806769783476	0.879650989276965\\
1.41995561890547	0.879615785400246\\
1.42184352891532	0.879580657911479\\
1.4237314278596	0.879545606577086\\
1.42561931573373	0.87951063116443\\
1.42750719253332	0.8794757314418\\
1.42939505825409	0.879440907178414\\
1.4312829128919	0.879406158144413\\
1.43317075644277	0.879371484110854\\
1.43505858890286	0.879336884849706\\
1.43694641026845	0.879302360133847\\
1.43883422053597	0.87926790973706\\
1.44072201970198	0.879233533434025\\
1.4426098077632	0.879199231000318\\
1.44449758471645	0.879165002212405\\
1.4463853505587	0.879130846847639\\
1.44827310528706	0.879096764684253\\
1.45016084889874	0.879062755501358\\
1.45204858139111	0.879028819078937\\
1.45393630276165	0.878994955197844\\
1.45582401300799	0.878961163639794\\
1.45771171212785	0.878927444187364\\
1.45959940011911	0.878893796623986\\
1.46148707697974	0.878860220733946\\
1.46337474270786	0.878826716302374\\
1.4652623973017	0.878793283115245\\
1.46715004075961	0.878759920959374\\
1.46903767308006	0.878726629622409\\
1.47092529426164	0.878693408892831\\
1.47281290430304	0.878660258559946\\
1.4747005032031	0.878627178413885\\
1.47658809096074	0.878594168245597\\
1.47847566757502	0.878561227846845\\
1.48036323304509	0.878528357010204\\
1.48225078737022	0.878495555529056\\
1.4841383305498	0.878462823197587\\
1.48602586258332	0.878430159810779\\
1.48791338347039	0.878397565164413\\
1.4898008932107	0.87836503905506\\
1.49168839180407	0.878332581280078\\
1.49357587925043	0.878300191637611\\
1.49546335554979	0.87826786992658\\
1.49735082070229	0.878235615946685\\
1.49923827470816	0.878203429498398\\
1.50112571756772	0.878171310382959\\
1.50301314928141	0.878139258402374\\
1.50490056984977	0.878107273359411\\
1.50678797927341	0.878075355057594\\
1.50867537755309	0.878043503301203\\
1.51056276468961	0.878011717895269\\
1.5124501406839	0.877979998645568\\
1.51433750553698	0.877948345358622\\
1.51622485924996	0.877916757841691\\
1.51811220182405	0.877885235902772\\
1.51999953326054	0.877853779350596\\
1.52188685356083	0.877822387994621\\
1.52377416272639	0.877791061645034\\
1.52566146075879	0.877759800112742\\
1.5275487476597	0.877728603209373\\
1.52943602343085	0.877697470747269\\
1.5313232880741	0.877666402539486\\
1.53321054159135	0.877635398399788\\
1.53509778398462	0.877604458142643\\
1.53698501525599	0.877573581583226\\
1.53887223540765	0.877542768537405\\
1.54075944444186	0.877512018821748\\
1.54264664236096	0.877481332253514\\
1.54453382916737	0.877450708650651\\
1.54642100486359	0.877420147831793\\
1.54830816945223	0.877389649616256\\
1.55019532293594	0.877359213824038\\
1.55208246531746	0.877328840275809\\
1.55396959659962	0.877298528792917\\
1.55585671678532	0.877268279197376\\
1.55774382587754	0.877238091311869\\
1.55963092387933	0.877207964959742\\
1.56151801079381	0.877177899965003\\
1.56340508662419	0.877147896152315\\
1.56529215137375	0.877117953346998\\
1.56717920504582	0.877088071375022\\
1.56906624764384	0.877058250063006\\
1.5709532791713	0.877028489238214\\
1.57284029963176	0.876998788728554\\
1.57472730902885	0.876969148362571\\
1.57661430736627	0.876939567969449\\
1.57850129464781	0.876910047379004\\
1.58038827087729	0.876880586421683\\
1.58227523605863	0.876851184928561\\
1.58416219019581	0.876821842731337\\
1.58604913329285	0.876792559662333\\
1.58793606535388	0.876763335554491\\
1.58982298638306	0.876734170241367\\
1.59170989638462	0.876705063557132\\
1.59359679536287	0.876676015336567\\
1.59548368332217	0.87664702541506\\
1.59737056026695	0.876618093628607\\
1.59925742620169	0.876589219813803\\
1.60114428113094	0.876560403807844\\
1.60303112505931	0.876531645448523\\
1.60491795799147	0.876502944574226\\
1.60680477993214	0.876474301023932\\
1.60869159088613	0.876445714637208\\
1.61057839085826	0.876417185254206\\
1.61246517985344	0.876388712715662\\
1.61435195787664	0.876360296862894\\
1.61623872493287	0.876331937537795\\
1.61812548102721	0.876303634582837\\
1.62001222616478	0.876275387841062\\
1.62189896035076	0.876247197156084\\
1.62378568359041	0.876219062372083\\
1.62567239588899	0.876190983333807\\
1.62755909725187	0.876162959886562\\
1.62944578768444	0.876134991876219\\
1.63133246719215	0.876107079149202\\
1.6332191357805	0.876079221552494\\
1.63510579345505	0.876051418933628\\
1.63699244022139	0.876023671140687\\
1.63887907608518	0.875995978022303\\
1.64076570105212	0.875968339427651\\
1.64265231512797	0.875940755206451\\
1.64453891831853	0.875913225208961\\
1.64642551062965	0.875885749285979\\
1.64831209206721	0.875858327288837\\
1.65019866263718	0.875830959069399\\
1.65208522234553	0.875803644480063\\
1.6539717711983	0.875776383373751\\
1.65585830920158	0.875749175603914\\
1.65774483636149	0.875722021024526\\
1.65963135268421	0.875694919490082\\
1.66151785817595	0.875667870855595\\
1.66340435284297	0.875640874976596\\
1.66529083669158	0.875613931709129\\
1.66717730972812	0.875587040909752\\
1.66906377195898	0.875560202435532\\
1.67095022339059	0.875533416144042\\
1.67283666402944	0.875506681893361\\
1.67472309388202	0.875479999542074\\
1.67660951295489	0.875453368949263\\
1.67849592125466	0.87542678997451\\
1.68038231878794	0.875400262477894\\
1.68226870556143	0.875373786319989\\
1.68415508158183	0.875347361361858\\
1.68604144685589	0.875320987465057\\
1.6879278013904	0.875294664491629\\
1.6898141451922	0.875268392304102\\
1.69170047826814	0.875242170765489\\
1.69358680062512	0.875215999739283\\
1.6954731122701	0.875189879089456\\
1.69735941321003	0.875163808680459\\
1.69924570345194	0.875137788377216\\
1.70113198300286	0.875111818045127\\
1.70301825186988	0.875085897550061\\
1.70490451006012	0.875060026758355\\
1.70679075758072	0.875034205536816\\
1.70867699443886	0.875008433752712\\
1.71056322064177	0.874982711273778\\
1.7124494361967	0.874957037968207\\
1.71433564111092	0.874931413704653\\
1.71622183539175	0.874905838352224\\
1.71810801904655	0.874880311780486\\
1.71999419208268	0.874854833859456\\
1.72188035450756	0.874829404459604\\
1.72376650632862	0.874804023451846\\
1.72565264755335	0.874778690707548\\
1.72753877818925	0.87475340609852\\
1.72942489824384	0.874728169497015\\
1.73131100772468	0.874702980775729\\
1.73319710663937	0.874677839807796\\
1.73508319499553	0.874652746466787\\
1.7369692728008	0.87462770062671\\
1.73885534006286	0.874602702162008\\
1.74074139678942	0.874577750947553\\
1.7426274429882	0.874552846858649\\
1.74451347866697	0.87452798977103\\
1.74639950383351	0.874503179560852\\
1.74828551849563	0.8744784161047\\
1.75017152266117	0.87445369927958\\
1.75205751633801	0.874429028962919\\
1.75394349953401	0.874404405032565\\
1.75582947225711	0.87437982736678\\
1.75771543451525	0.874355295844245\\
1.75960138631638	0.874330810344054\\
1.7614873276685	0.874306370745713\\
1.76337325857963	0.874281976929138\\
1.76525917905779	0.874257628774655\\
1.76714508911106	0.874233326162997\\
1.76903098874752	0.8742090689753\\
1.77091687797527	0.874184857093107\\
1.77280275680245	0.874160690398359\\
1.7746886252372	0.874136568773401\\
1.77657448328771	0.874112492100975\\
1.77846033096217	0.874088460264218\\
1.78034616826879	0.874064473146666\\
1.78223199521583	0.874040530632244\\
1.78411781181153	0.874016632605273\\
1.78600361806419	0.873992778950462\\
1.7878894139821	0.873968969552909\\
1.78977519957359	0.873945204298098\\
1.79166097484699	0.873921483071901\\
1.79354673981067	0.87389780576057\\
1.79543249447302	0.873874172250743\\
1.79731823884243	0.873850582429436\\
1.79920397292732	0.873827036184044\\
1.80108969673613	0.873803533402341\\
1.80297541027731	0.873780073972475\\
1.80486111355936	0.87375665778297\\
1.80674680659074	0.87373328472272\\
1.80863248937998	0.873709954680994\\
1.81051816193561	0.873686667547428\\
1.81240382426617	0.873663423212025\\
1.81428947638021	0.873640221565158\\
1.81617511828633	0.873617062497562\\
1.81806074999312	0.873593945900337\\
1.81994637150919	0.873570871664944\\
1.82183198284316	0.873547839683207\\
1.82371758400369	0.873524849847305\\
1.82560317499942	0.873501902049779\\
1.82748875583905	0.873478996183522\\
1.82937432653124	0.873456132141784\\
1.83125988708472	0.873433309818169\\
1.8331454375082	0.87341052910663\\
1.83503097781042	0.873387789901473\\
1.83691650800011	0.87336509209735\\
1.83880202808606	0.873342435589263\\
1.84068753807702	0.873319820272559\\
1.8425730379818	0.87329724604293\\
1.8444585278092	0.873274712796409\\
1.84634400756803	0.873252220429374\\
1.84822947726712	0.873229768838542\\
1.85011493691532	0.873207357920968\\
1.85200038652148	0.873184987574046\\
1.85388582609447	0.873162657695506\\
1.85577125564317	0.873140368183411\\
1.85765667517647	0.873118118936162\\
1.85954208470327	0.873095909852487\\
1.8614274842325	0.873073740831448\\
1.86331287377307	0.873051611772435\\
1.86519825333394	0.873029522575168\\
1.86708362292403	0.873007473139691\\
1.86896898255233	0.872985463366376\\
1.87085433222779	0.872963493155918\\
1.8727396719594	0.872941562409335\\
1.87462500175615	0.872919671027967\\
1.87651032162704	0.872897818913473\\
1.87839563158109	0.872876005967833\\
1.88028093162731	0.872854232093343\\
1.88216622177475	0.872832497192616\\
1.88405150203243	0.87281080116858\\
1.88593677240941	0.872789143924478\\
1.88782203291476	0.872767525363865\\
1.88970728355753	0.872745945390605\\
1.8915925243468	0.872724403908876\\
1.89347775529167	0.872702900823162\\
1.89536297640122	0.872681436038257\\
1.89724818768456	0.87266000945926\\
1.89913338915079	0.872638620991575\\
1.90101858080904	0.87261727054091\\
1.90290376266844	0.872595958013276\\
1.90478893473811	0.872574683314987\\
1.9066740970272	0.872553446352656\\
1.90855924954485	0.872532247033194\\
1.91044439230023	0.872511085263813\\
1.9123295253025	0.872489960952019\\
1.91421464856082	0.872468874005615\\
1.91609976208437	0.872447824332698\\
1.91798486588234	0.872426811841659\\
1.91986995996391	0.87240583644118\\
1.92175504433828	0.872384898040236\\
1.92364011901466	0.872363996548089\\
1.92552518400225	0.872343131874292\\
1.92741023931025	0.872322303928684\\
1.92929528494791	0.872301512621392\\
1.93118032092443	0.872280757862827\\
1.93306534724905	0.872260039563685\\
1.934950363931	0.872239357634944\\
1.93683537097953	0.872218711987865\\
1.93872036840388	0.872198102533989\\
1.94060535621329	0.872177529185138\\
1.94249033441704	0.872156991853411\\
1.94437530302437	0.872136490451186\\
1.94626026204455	0.872116024891117\\
1.94814521148685	0.872095595086132\\
1.95003015136054	0.872075200949436\\
1.9519150816749	0.872054842394504\\
1.9538000024392	0.872034519335087\\
1.95568491366275	0.872014231685204\\
1.95756981535481	0.871993979359144\\
1.95945470752469	0.871973762271468\\
1.96133959018167	0.871953580337002\\
1.96322446333507	0.871933433470841\\
1.96510932699417	0.871913321588343\\
1.96699418116829	0.871893244605134\\
1.96887902586672	0.871873202437102\\
1.9707638610988	0.871853195000399\\
1.97264868687381	0.871833222211438\\
1.97453350320109	0.871813283986892\\
1.97641831008995	0.871793380243695\\
1.97830310754971	0.87177351089904\\
1.9801878955897	0.871753675870376\\
1.98207267421923	0.871733875075412\\
1.98395744344764	0.871714108432108\\
1.98584220328426	0.871694375858684\\
1.98772695373841	0.87167467727361\\
1.98961169481943	0.87165501259561\\
1.99149642653666	0.871635381743661\\
1.99338114889942	0.871615784636989\\
1.99526586191706	0.871596221195072\\
1.99715056559892	0.871576691337634\\
1.99903525995433	0.871557194984651\\
2.00091994499263	0.871537732056343\\
2.00280462072317	0.871518302473178\\
2.00468928715529	0.871498906155867\\
2.00657394429834	0.871479543025367\\
2.00845859216164	0.871460213002878\\
2.01034323075456	0.871440916009842\\
2.01222786008644	0.871421651967944\\
2.01411248016662	0.871402420799107\\
2.01599709100444	0.871383222425496\\
2.01788169260926	0.871364056769514\\
2.01976628499041	0.8713449237538\\
2.02165086815725	0.871325823301233\\
2.02353544211912	0.871306755334925\\
2.02542000688536	0.871287719778227\\
2.02730456246533	0.87126871655472\\
2.02918910886836	0.871249745588221\\
2.0310736461038	0.871230806802778\\
2.032958174181	0.871211900122673\\
2.0348426931093	0.871193025472415\\
2.03672720289803	0.871174182776747\\
2.03861170355655	0.871155371960637\\
2.04049619509419	0.871136592949285\\
2.04238067752029	0.871117845668115\\
2.0442651508442	0.871099130042779\\
2.04614961507525	0.871080445999154\\
2.04803407022277	0.871061793463343\\
2.0499185162961	0.871043172361671\\
2.05180295330458	0.871024582620688\\
2.05368738125753	0.871006024167163\\
2.05557180016429	0.870987496928091\\
2.05745621003419	0.870969000830684\\
2.05934061087656	0.870950535802374\\
2.06122500270071	0.870932101770813\\
2.06310938551598	0.870913698663872\\
2.06499375933168	0.870895326409637\\
2.06687812415714	0.870876984936411\\
2.06876248000167	0.870858674172714\\
2.07064682687459	0.87084039404728\\
2.07253116478521	0.870822144489056\\
2.07441549374284	0.870803925427204\\
2.0762998137568	0.870785736791097\\
2.07818412483638	0.870767578510322\\
2.08006842699089	0.870749450514674\\
2.08195272022964	0.870731352734159\\
2.08383700456191	0.870713285098994\\
2.08572127999701	0.870695247539602\\
2.08760554654423	0.870677239986616\\
2.08948980421286	0.870659262370875\\
2.09137405301218	0.870641314623424\\
2.09325829295148	0.870623396675513\\
2.09514252404004	0.870605508458599\\
2.09702674628714	0.870587649904342\\
2.09891095970205	0.870569820944602\\
2.10079516429404	0.870552021511447\\
2.10267936007239	0.870534251537143\\
2.10456354704635	0.870516510954158\\
2.10644772522518	0.87049879969516\\
2.10833189461816	0.870481117693017\\
2.11021605523452	0.870463464880796\\
2.11210020708351	0.87044584119176\\
2.1139843501744	0.870428246559374\\
2.11586848451641	0.870410680917293\\
2.11775261011878	0.870393144199374\\
2.11963672699076	0.870375636339667\\
2.12152083514158	0.870358157272414\\
2.12340493458045	0.870340706932054\\
2.1252890253166	0.870323285253219\\
2.12717310735926	0.87030589217073\\
2.12905718071764	0.870288527619605\\
2.13094124540094	0.870271191535048\\
2.13282530141838	0.870253883852455\\
2.13470934877916	0.870236604507413\\
2.13659338749247	0.870219353435696\\
2.1384774175675	0.870202130573267\\
2.14036143901345	0.870184935856277\\
2.14224545183951	0.870167769221063\\
2.14412945605484	0.870150630604147\\
2.14601345166863	0.870133519942241\\
2.14789743869004	0.870116437172236\\
2.14978141712824	0.870099382231211\\
2.1516653869924	0.870082355056427\\
2.15354934829166	0.870065355585329\\
2.15543330103518	0.870048383755543\\
2.1573172452321	0.870031439504877\\
2.15920118089157	0.87001452277132\\
2.16108510802272	0.86999763349304\\
2.16296902663468	0.869980771608388\\
2.16485293673658	0.869963937055889\\
2.16673683833754	0.86994712977425\\
2.16862073144667	0.869930349702354\\
2.17050461607308	0.869913596779262\\
2.17238849222589	0.86989687094421\\
2.17427235991418	0.869880172136611\\
2.17615621914706	0.869863500296054\\
2.17804006993361	0.869846855362299\\
2.17992391228291	0.869830237275284\\
2.18180774620404	0.869813645975116\\
2.18369157170608	0.869797081402079\\
2.18557538879809	0.869780543496627\\
2.18745919748913	0.869764032199384\\
2.18934299778825	0.869747547451146\\
2.19122678970451	0.869731089192881\\
2.19311057324695	0.869714657365724\\
2.19499434842461	0.869698251910979\\
2.19687811524651	0.869681872770121\\
2.19876187372169	0.86966551988479\\
2.20064562385917	0.869649193196795\\
2.20252936566795	0.86963289264811\\
2.20441309915705	0.869616618180876\\
2.20629682433547	0.869600369737399\\
2.20818054121221	0.869584147260152\\
2.21006424979625	0.869567950691768\\
2.21194795009658	0.869551779975047\\
2.21383164212217	0.869535635052952\\
2.215715325882	0.869519515868606\\
2.21759900138504	0.869503422365297\\
2.21948266864023	0.869487354486472\\
2.22136632765654	0.869471312175741\\
2.22324997844291	0.869455295376873\\
2.22513362100828	0.869439304033796\\
2.22701725536158	0.869423338090598\\
2.22890088151174	0.869407397491526\\
2.23078449946768	0.869391482180984\\
2.23266810923832	0.869375592103535\\
2.23455171083255	0.869359727203896\\
2.23643530425929	0.869343887426945\\
2.23831888952742	0.86932807271771\\
2.24020246664583	0.86931228302138\\
2.2420860356234	0.869296518283294\\
2.24396959646901	0.869280778448949\\
2.24585314919153	0.869265063463993\\
2.24773669379981	0.869249373274228\\
2.24962023030271	0.86923370782561\\
2.25150375870907	0.869218067064244\\
2.25338727902774	0.869202450936389\\
2.25527079126754	0.869186859388454\\
2.25715429543731	0.869171292366999\\
2.25903779154586	0.869155749818734\\
2.26092127960201	0.869140231690517\\
2.26280475961455	0.869124737929358\\
2.2646882315923	0.869109268482411\\
2.26657169554403	0.869093823296983\\
2.26845515147854	0.869078402320524\\
2.27033859940459	0.869063005500633\\
2.27222203933097	0.869047632785055\\
2.27410547126643	0.869032284121682\\
2.27598889521973	0.869016959458548\\
2.27787231119961	0.869001658743836\\
2.27975571921483	0.868986381925871\\
2.2816391192741	0.868971128953122\\
2.28352251138616	0.868955899774202\\
2.28540589555973	0.868940694337866\\
2.28728927180352	0.868925512593013\\
2.28917264012623	0.868910354488682\\
2.29105600053656	0.868895219974054\\
2.2929393530432	0.868880108998453\\
2.29482269765483	0.86886502151134\\
2.29670603438013	0.868849957462318\\
2.29858936322777	0.868834916801129\\
2.3004726842064	0.868819899477655\\
2.30235599732468	0.868804905441914\\
2.30423930259125	0.868789934644066\\
2.30612260001475	0.868774987034405\\
2.30800588960381	0.868760062563363\\
2.30988917136705	0.86874516118151\\
2.31177244531309	0.868730282839552\\
2.31365571145054	0.868715427488328\\
2.31553896978798	0.868700595078817\\
2.31742222033402	0.868685785562128\\
2.31930546309724	0.868670998889507\\
2.32118869808622	0.868656235012333\\
2.32307192530952	0.868641493882121\\
2.3249551447757	0.868626775450514\\
2.32683835649332	0.868612079669293\\
2.32872156047093	0.868597406490366\\
2.33060475671705	0.868582755865777\\
2.33248794524023	0.868568127747699\\
2.33437112604898	0.868553522088435\\
2.33625429915182	0.868538938840421\\
2.33813746455725	0.868524377956221\\
2.34002062227378	0.868509839388527\\
2.34190377230988	0.868495323090164\\
2.34378691467406	0.868480829014083\\
2.34567004937477	0.868466357113362\\
2.34755317642048	0.86845190734121\\
2.34943629581967	0.86843747965096\\
2.35131940758077	0.868423073996074\\
2.35320251171223	0.86840869033014\\
2.35508560822248	0.868394328606872\\
2.35696869711996	0.868379988780109\\
2.35885177841307	0.868365670803815\\
2.36073485211024	0.86835137463208\\
2.36261791821986	0.868337100219117\\
2.36450097675033	0.868322847519264\\
2.36638402771003	0.868308616486982\\
2.36826707110735	0.868294407076855\\
2.37015010695065	0.868280219243591\\
2.3720331352483	0.868266052942017\\
2.37391615600865	0.868251908127085\\
2.37579916924005	0.868237784753867\\
2.37768217495083	0.868223682777558\\
2.37956517314932	0.86820960215347\\
2.38144816384385	0.868195542837038\\
2.38333114704274	0.868181504783818\\
2.38521412275427	0.868167487949481\\
2.38709709098676	0.868153492289821\\
2.38898005174849	0.868139517760749\\
2.39086300504774	0.868125564318295\\
2.39274595089278	0.868111631918607\\
2.39462888929188	0.86809772051795\\
2.3965118202533	0.868083830072705\\
2.39839474378527	0.868069960539372\\
2.40027765989605	0.868056111874566\\
2.40216056859385	0.868042284035017\\
2.40404346988691	0.868028476977574\\
2.40592636378344	0.868014690659197\\
2.40780925029164	0.868000925036963\\
2.40969212941971	0.867987180068064\\
2.41157500117585	0.867973455709805\\
2.41345786556822	0.867959751919604\\
2.41534072260501	0.867946068654994\\
2.41722357229438	0.86793240587362\\
2.41910641464448	0.86791876353324\\
2.42098924966346	0.867905141591723\\
2.42287207735947	0.867891540007052\\
2.42475489774062	0.867877958737319\\
2.42663771081505	0.867864397740729\\
2.42852051659087	0.867850856975596\\
2.43040331507618	0.867837336400347\\
2.43228610627908	0.867823835973516\\
2.43416889020766	0.867810355653749\\
2.43605166686999	0.8677968953998\\
2.43793443627416	0.867783455170533\\
2.43981719842822	0.867770034924918\\
2.44169995334023	0.867756634622037\\
2.44358270101823	0.867743254221078\\
2.44546544147026	0.867729893681335\\
2.44734817470436	0.867716552962213\\
2.44923090072854	0.867703232023219\\
2.45111361955081	0.867689930823971\\
2.45299633117918	0.86767664932419\\
2.45487903562164	0.867663387483705\\
2.45676173288619	0.867650145262449\\
2.45864442298079	0.867636922620461\\
2.46052710591342	0.867623719517884\\
2.46240978169204	0.867610535914965\\
2.4642924503246	0.867597371772057\\
2.46617511181905	0.867584227049616\\
2.46805776618332	0.8675711017082\\
2.46994041342533	0.867557995708471\\
2.47182305355302	0.867544909011196\\
2.47370568657428	0.86753184157724\\
2.47558831249702	0.867518793367574\\
2.47747093132913	0.867505764343269\\
2.4793535430785	0.867492754465498\\
2.481236147753	0.867479763695534\\
2.48311874536049	0.867466791994752\\
2.48500133590884	0.867453839324628\\
2.4868839194059	0.867440905646736\\
2.48876649585951	0.867427990922752\\
2.49064906527749	0.86741509511445\\
2.49253162766768	0.867402218183705\\
2.49441418303789	0.867389360092488\\
2.49629673139593	0.867376520802872\\
2.49817927274959	0.867363700277026\\
2.50006180710667	0.867350898477217\\
2.50194433447494	0.867338115365811\\
2.50382685486218	0.867325350905269\\
2.50570936827615	0.867312605058152\\
2.50759187472462	0.867299877787115\\
2.50947437421531	0.867287169054911\\
2.51135686675598	0.867274478824389\\
2.51323935235435	0.867261807058493\\
2.51512183101814	0.867249153720263\\
2.51700430275506	0.867236518772833\\
2.51888676757283	0.867223902179434\\
2.52076922547912	0.867211303903391\\
2.52265167648164	0.867198723908121\\
2.52453412058805	0.867186162157137\\
2.52641655780603	0.867173618614047\\
2.52829898814324	0.867161093242549\\
2.53018141160732	0.867148586006437\\
2.53206382820592	0.867136096869596\\
2.53394623794668	0.867123625796004\\
2.53582864083722	0.867111172749731\\
2.53771103688516	0.86709873769494\\
2.53959342609811	0.867086320595885\\
2.54147580848366	0.867073921416911\\
2.54335818404941	0.867061540122453\\
2.54524055280294	0.867049176677039\\
2.54712291475183	0.867036831045287\\
2.54900526990363	0.867024503191904\\
2.55088761826592	0.867012193081687\\
2.55276995984623	0.866999900679524\\
2.5546522946521	0.866987625950391\\
2.55653462269107	0.866975368859355\\
2.55841694397065	0.866963129371568\\
2.56029925849837	0.866950907452275\\
2.56218156628172	0.866938703066806\\
2.56406386732821	0.866926516180581\\
2.56594616164531	0.866914346759106\\
2.56782844924052	0.866902194767975\\
2.56971073012129	0.86689006017287\\
2.57159300429509	0.866877942939559\\
2.57347527176938	0.866865843033896\\
2.57535753255159	0.866853760421823\\
2.57723978664917	0.866841695069368\\
2.57912203406954	0.866829646942642\\
2.58100427482012	0.866817616007844\\
2.58288650890833	0.866805602231259\\
2.58476873634155	0.866793605579254\\
2.58665095712718	0.866781626018284\\
2.58853317127262	0.866769663514886\\
2.59041537878523	0.866757718035682\\
2.59229757967238	0.866745789547378\\
2.59417977394142	0.866733878016765\\
2.59606196159972	0.866721983410715\\
2.5979441426546	0.866710105696185\\
2.59982631711341	0.866698244840213\\
2.60170848498346	0.866686400809923\\
2.60359064627207	0.866674573572517\\
2.60547280098655	0.866662763095283\\
2.6073549491342	0.866650969345589\\
2.6092370907223	0.866639192290885\\
2.61111922575814	0.866627431898702\\
2.61300135424898	0.866615688136653\\
2.6148834762021	0.866603960972431\\
2.61676559162474	0.86659225037381\\
2.61864770052415	0.866580556308645\\
2.62052980290758	0.86656887874487\\
2.62241189878224	0.8665572176505\\
2.62429398815537	0.866545572993628\\
2.62617607103417	0.866533944742429\\
2.62805814742584	0.866522332865155\\
2.62994021733759	0.866510737330138\\
2.6318222807766	0.866499158105788\\
2.63370433775004	0.866487595160594\\
2.63558638826509	0.866476048463123\\
2.6374684323289	0.866464517982019\\
2.63935046994863	0.866453003686006\\
2.64123250113142	0.866441505543884\\
2.64311452588442	0.866430023524529\\
2.64499654421473	0.866418557596897\\
2.64687855612949	0.866407107730019\\
2.6487605616358	0.866395673893002\\
2.65064256074076	0.866384256055031\\
2.65252455345146	0.866372854185365\\
2.654406539775	0.866361468253341\\
2.65628851971844	0.866350098228371\\
2.65817049328885	0.866338744079941\\
2.66005246049329	0.866327405777614\\
2.66193442133881	0.866316083291027\\
2.66381637583246	0.866304776589891\\
2.66569832398126	0.866293485643994\\
2.66758026579223	0.866282210423196\\
2.6694622012724	0.86627095089743\\
2.67134413042878	0.866259707036707\\
2.67322605326835	0.866248478811108\\
2.67510796979812	0.866237266190789\\
2.67698988002507	0.866226069145977\\
2.67887178395616	0.866214887646975\\
2.68075368159836	0.866203721664157\\
2.68263557295863	0.866192571167969\\
2.68451745804392	0.866181436128931\\
2.68639933686117	0.866170316517634\\
2.68828120941731	0.86615921230474\\
2.69016307571927	0.866148123460984\\
2.69204493577394	0.866137049957172\\
2.69392678958826	0.866125991764181\\
2.6958086371691	0.866114948852959\\
2.69769047852336	0.866103921194525\\
2.69957231365793	0.866092908759968\\
2.70145414257966	0.866081911520449\\
2.70333596529544	0.866070929447196\\
2.70521778181211	0.866059962511511\\
2.70709959213652	0.866049010684761\\
2.7089813962755	0.866038073938387\\
2.7108631942359	0.866027152243897\\
2.71274498602454	0.866016245572868\\
2.71462677164821	0.866005353896946\\
2.71650855111374	0.865994477187847\\
2.71839032442792	0.865983615417353\\
2.72027209159754	0.865972768557317\\
2.72215385262937	0.865961936579659\\
2.72403560753019	0.865951119456366\\
2.72591735630677	0.865940317159493\\
2.72779909896586	0.865929529661164\\
2.7296808355142	0.865918756933567\\
2.73156256595854	0.865907998948962\\
2.7334442903056	0.865897255679671\\
2.73532600856212	0.865886527098085\\
2.73720772073479	0.865875813176661\\
2.73908942683033	0.865865113887924\\
2.74097112685544	0.865854429204461\\
2.7428528208168	0.865843759098929\\
2.74473450872109	0.865833103544049\\
2.74661619057499	0.865822462512607\\
2.74849786638516	0.865811835977456\\
2.75037953615826	0.865801223911511\\
2.75226119990094	0.865790626287756\\
2.75414285761983	0.865780043079237\\
2.75602450932156	0.865769474259064\\
2.75790615501276	0.865758919800414\\
2.75978779470005	0.865748379676526\\
2.76166942839003	0.865737853860703\\
2.7635510560893	0.865727342326314\\
2.76543267780444	0.865716845046789\\
2.76731429354205	0.865706361995622\\
2.76919590330869	0.865695893146372\\
2.77107750711093	0.865685438472658\\
2.77295910495532	0.865674997948165\\
2.77484069684843	0.865664571546639\\
2.77672228279678	0.865654159241888\\
2.77860386280692	0.865643761007783\\
2.78048543688536	0.865633376818257\\
2.78236700503862	0.865623006647306\\
2.78424856727321	0.865612650468986\\
2.78613012359564	0.865602308257416\\
2.78801167401239	0.865591979986775\\
2.78989321852995	0.865581665631305\\
2.79177475715479	0.865571365165306\\
2.79365628989338	0.865561078563143\\
2.79553781675218	0.865550805799239\\
2.79741933773765	0.865540546848078\\
2.79930085285623	0.865530301684204\\
2.80118236211435	0.865520070282222\\
2.80306386551843	0.865509852616796\\
2.80494536307491	0.865499648662652\\
2.80682685479019	0.865489458394572\\
2.80870834067068	0.865479281787401\\
2.81058982072276	0.86546911881604\\
2.81247129495284	0.865458969455453\\
2.81435276336727	0.865448833680659\\
2.81623422597245	0.865438711466739\\
2.81811568277473	0.86542860278883\\
2.81999713378046	0.86541850762213\\
2.821878578996	0.865408425941893\\
2.82376001842768	0.865398357723432\\
2.82564145208184	0.865388302942118\\
2.82752287996479	0.86537826157338\\
2.82940430208285	0.865368233592705\\
2.83128571844234	0.865358218975635\\
2.83316712904955	0.865348217697772\\
2.83504853391076	0.865338229734775\\
2.83692993303227	0.865328255062358\\
2.83881132642036	0.865318293656293\\
2.84069271408128	0.865308345492408\\
2.8425740960213	0.865298410546589\\
2.84445547224666	0.865288488794777\\
2.84633684276363	0.865278580212969\\
2.84821820757842	0.865268684777218\\
2.85009956669727	0.865258802463634\\
2.8519809201264	0.865248933248381\\
2.85386226787202	0.865239077107679\\
2.85574360994033	0.865229234017805\\
2.85762494633754	0.865219403955089\\
2.85950627706983	0.865209586895916\\
2.86138760214338	0.865199782816728\\
2.86326892156436	0.86518999169402\\
2.86515023533895	0.865180213504341\\
2.8670315434733	0.865170448224296\\
2.86891284597355	0.865160695830543\\
2.87079414284585	0.865150956299795\\
2.87267543409634	0.865141229608819\\
2.87455671973114	0.865131515734435\\
2.87643799975637	0.865121814653516\\
2.87831927417814	0.86511212634299\\
2.88020054300255	0.865102450779839\\
2.8820818062357	0.865092787941095\\
2.88396306388368	0.865083137803845\\
2.88584431595256	0.86507350034523\\
2.88772556244842	0.865063875542442\\
2.88960680337733	0.865054263372725\\
2.89148803874533	0.865044663813378\\
2.89336926855848	0.86503507684175\\
2.89525049282282	0.865025502435243\\
2.89713171154439	0.865015940571311\\
2.8990129247292	0.865006391227459\\
2.90089413238328	0.864996854381244\\
2.90277533451264	0.864987330010277\\
2.90465653112328	0.864977818092216\\
2.9065377222212	0.864968318604773\\
2.90841890781238	0.864958831525712\\
2.9103000879028	0.864949356832845\\
2.91218126249844	0.864939894504036\\
2.91406243160526	0.864930444517202\\
2.91594359522922	0.864921006850308\\
2.91782475337627	0.864911581481369\\
2.91970590605234	0.864902168388451\\
2.92158705326338	0.864892767549672\\
2.92346819501531	0.864883378943197\\
2.92534933131405	0.864874002547243\\
2.92723046216551	0.864864638340075\\
2.9291115875756	0.864855286300009\\
2.9309927075502	0.864845946405409\\
2.93287382209522	0.864836618634689\\
2.93475493121653	0.864827302966313\\
2.93663603492001	0.864817999378793\\
2.93851713321152	0.86480870785069\\
2.94039822609692	0.864799428360613\\
2.94227931358207	0.864790160887221\\
2.9441603956728	0.864780905409221\\
2.94604147237496	0.864771661905367\\
2.94792254369437	0.864762430354463\\
2.94980360963685	0.864753210735361\\
2.95168467020822	0.864744003026959\\
2.95356572541429	0.864734807208205\\
2.95544677526085	0.864725623258093\\
2.95732781975369	0.864716451155665\\
2.95920885889861	0.864707290880011\\
2.96108989270137	0.864698142410268\\
2.96297092116775	0.86468900572562\\
2.96485194430351	0.864679880805297\\
2.9667329621144	0.864670767628578\\
2.96861397460617	0.864661666174787\\
2.97049498178456	0.864652576423294\\
2.97237598365531	0.864643498353519\\
2.97425698022414	0.864634431944924\\
2.97613797149676	0.86462537717702\\
2.97801895747889	0.864616334029363\\
2.97989993817622	0.864607302481555\\
2.98178091359447	0.864598282513244\\
2.98366188373931	0.864589274104124\\
2.98554284861643	0.864580277233935\\
2.98742380823149	0.864571291882461\\
2.98930476259018	0.864562318029532\\
2.99118571169813	0.864553355655024\\
2.99306665556102	0.864544404738856\\
2.99494759418448	0.864535465260995\\
2.99682852757416	0.864526537201451\\
2.99870945573567	0.864517620540278\\
3.00059037867465	0.864508715257577\\
3.00247129639671	0.86449982133349\\
3.00435220890746	0.864490938748207\\
3.0062331162125	0.86448206748196\\
3.00811401831742	0.864473207515025\\
3.00999491522782	0.864464358827724\\
3.01187580694926	0.86445552140042\\
3.01375669348734	0.864446695213523\\
3.0156375748476	0.864437880247483\\
3.01751845103561	0.864429076482796\\
3.01939932205692	0.864420283900002\\
3.02128018791708	0.864411502479682\\
3.02316104862162	0.86440273220246\\
3.02504190417606	0.864393973049007\\
3.02692275458594	0.864385225000032\\
3.02880359985677	0.864376488036289\\
3.03068443999406	0.864367762138576\\
3.03256527500331	0.864359047287731\\
3.03444610489001	0.864350343464636\\
3.03632692965965	0.864341650650215\\
3.03820774931771	0.864332968825434\\
3.04008856386967	0.864324297971302\\
3.04196937332098	0.864315638068868\\
3.04385017767712	0.864306989099225\\
3.04573097694352	0.864298351043506\\
3.04761177112564	0.864289723882888\\
3.04949256022892	0.864281107598588\\
3.05137334425878	0.864272502171863\\
3.05325412322064	0.864263907584014\\
3.05513489711994	0.864255323816382\\
3.05701566596206	0.86424675085035\\
3.05889642975243	0.864238188667339\\
3.06077718849643	0.864229637248814\\
3.06265794219946	0.86422109657628\\
3.06453869086689	0.864212566631283\\
3.0664194345041	0.864204047395408\\
3.06830017311646	0.864195538850281\\
3.07018090670933	0.86418704097757\\
3.07206163528806	0.86417855375898\\
3.07394235885801	0.86417007717626\\
3.07582307742451	0.864161611211197\\
3.07770379099289	0.864153155845616\\
3.07958449956849	0.864144711061385\\
3.08146520315662	0.86413627684041\\
3.0833459017626	0.864127853164638\\
3.08522659539173	0.864119440016053\\
3.08710728404931	0.86411103737668\\
3.08898796774064	0.864102645228583\\
3.09086864647099	0.864094263553866\\
3.09274932024565	0.86408589233467\\
3.0946299890699	0.864077531553177\\
3.09651065294899	0.864069181191607\\
3.09839131188818	0.864060841232218\\
3.10027196589273	0.864052511657308\\
3.10215261496788	0.864044192449212\\
3.10403325911887	0.864035883590306\\
3.10591389835093	0.864027585063\\
3.10779453266928	0.864019296849747\\
3.10967516207915	0.864011018933036\\
3.11155578658574	0.864002751295392\\
3.11343640619426	0.863994493919381\\
3.1153170209099	0.863986246787606\\
3.11719763073786	0.863978009882707\\
3.11907823568332	0.863969783187362\\
3.12095883575146	0.863961566684286\\
3.12283943094745	0.863953360356233\\
3.12472002127645	0.863945164185993\\
3.12660060674363	0.863936978156392\\
3.12848118735412	0.863928802250297\\
3.13036176311308	0.863920636450607\\
3.13224233402565	0.863912480740263\\
3.13412290009694	0.863904335102238\\
3.1360034613321	0.863896199519545\\
3.13788401773623	0.863888073975233\\
3.13976456931445	0.863879958452388\\
3.14164511607186	0.863871852934129\\
3.14352565801356	0.863863757403616\\
3.14540619514464	0.863855671844043\\
3.14728672747019	0.863847596238641\\
3.14916725499528	0.863839530570674\\
3.15104777772498	0.863831474823447\\
3.15292829566437	0.863823428980297\\
3.15480880881849	0.863815393024598\\
3.1566893171924	0.863807366939759\\
3.15856982079115	0.863799350709227\\
3.16045031961978	0.863791344316482\\
3.16233081368331	0.863783347745039\\
3.16421130298677	0.863775360978451\\
3.16609178753519	0.863767384000304\\
3.16797226733357	0.863759416794219\\
3.16985274238692	0.863751459343854\\
3.17173321270024	0.863743511632901\\
3.17361367827853	0.863735573645085\\
3.17549413912677	0.863727645364168\\
3.17737459524995	0.863719726773946\\
3.17925504665303	0.86371181785825\\
3.18113549334098	0.863703918600945\\
3.18301593531877	0.863696028985929\\
3.18489637259135	0.863688148997137\\
3.18677680516367	0.863680278618536\\
3.18865723304067	0.863672417834129\\
3.19053765622729	0.863664566627952\\
3.19241807472845	0.863656724984074\\
3.19429848854908	0.8636488928866\\
3.1961788976941	0.863641070319667\\
3.19805930216841	0.863633257267446\\
3.19993970197693	0.863625453714142\\
3.20182009712454	0.863617659643994\\
3.20370048761614	0.863609875041273\\
3.20558087345662	0.863602099890284\\
3.20746125465085	0.863594334175366\\
3.2093416312037	0.863586577880891\\
3.21122200312004	0.863578830991262\\
3.21310237040473	0.863571093490918\\
3.21498273306263	0.863563365364329\\
3.21686309109858	0.863555646595998\\
3.21874344451742	0.863547937170461\\
3.22062379332399	0.863540237072287\\
3.22250413752311	0.863532546286077\\
3.22438447711962	0.863524864796465\\
3.22626481211831	0.863517192588117\\
3.22814514252401	0.86350952964573\\
3.23002546834152	0.863501875954037\\
3.23190578957563	0.8634942314978\\
3.23378610623113	0.863486596261812\\
3.23566641831282	0.863478970230902\\
3.23754672582547	0.863471353389928\\
3.23942702877384	0.86346374572378\\
3.24130732716272	0.863456147217381\\
3.24318762099686	0.863448557855684\\
3.24506791028101	0.863440977623675\\
3.24694819501992	0.863433406506371\\
3.24882847521833	0.863425844488821\\
3.25070875088098	0.863418291556103\\
3.25258902201261	0.863410747693328\\
3.25446928861792	0.86340321288564\\
3.25634955070164	0.86339568711821\\
3.25822980826849	0.863388170376244\\
3.26011006132316	0.863380662644976\\
3.26199030987036	0.863373163909672\\
3.26387055391477	0.863365674155628\\
3.2657507934611	0.863358193368173\\
3.26763102851401	0.863350721532664\\
3.26951125907818	0.86334325863449\\
3.27139148515829	0.863335804659069\\
3.27327170675898	0.863328359591852\\
3.27515192388493	0.863320923418317\\
3.27703213654079	0.863313496123975\\
3.27891234473119	0.863306077694366\\
3.28079254846078	0.863298668115059\\
3.28267274773418	0.863291267371655\\
3.28455294255604	0.863283875449784\\
3.28643313293096	0.863276492335105\\
3.28831331886357	0.863269118013309\\
3.29019350035846	0.863261752470114\\
3.29207367742026	0.86325439569127\\
3.29395385005354	0.863247047662554\\
3.29583401826291	0.863239708369776\\
3.29771418205295	0.863232377798771\\
3.29959434142824	0.863225055935407\\
3.30147449639336	0.86321774276558\\
3.30335464695286	0.863210438275214\\
3.30523479311132	0.863203142450265\\
3.30711493487328	0.863195855276713\\
3.3089950722433	0.863188576740573\\
3.31087520522593	0.863181306827885\\
3.3127553338257	0.863174045524718\\
3.31463545804714	0.863166792817171\\
3.31651557789479	0.863159548691371\\
3.31839569337316	0.863152313133474\\
3.32027580448676	0.863145086129664\\
3.32215591124011	0.863137867666153\\
3.32403601363771	0.863130657729184\\
3.32591611168406	0.863123456305024\\
3.32779620538365	0.863116263379971\\
3.32967629474096	0.863109078940352\\
3.33155637976048	0.86310190297252\\
3.33343646044668	0.863094735462857\\
3.33531653680403	0.863087576397772\\
3.337196608837	0.863080425763703\\
3.33907667655003	0.863073283547116\\
3.34095673994759	0.863066149734502\\
3.34283679903411	0.863059024312385\\
3.34471685381404	0.863051907267311\\
3.34659690429182	0.863044798585856\\
3.34847695047186	0.863037698254624\\
3.3503569923586	0.863030606260246\\
3.35223702995645	0.863023522589379\\
3.35411706326982	0.86301644722871\\
3.35599709230313	0.86300938016495\\
3.35787711706076	0.863002321384838\\
3.35975713754712	0.862995270875143\\
3.36163715376659	0.862988228622656\\
3.36351716572356	0.8629811946142\\
3.36539717342241	0.86297416883662\\
3.3672771768675	0.862967151276792\\
3.3691571760632	0.862960141921616\\
3.37103717101388	0.862953140758019\\
3.37291716172389	0.862946147772957\\
3.37479714819758	0.862939162953408\\
3.3766771304393	0.86293218628638\\
3.37855710845337	0.862925217758908\\
3.38043708224415	0.862918257358049\\
3.38231705181594	0.862911305070891\\
3.38419701717308	0.862904360884546\\
3.38607697831989	0.862897424786151\\
3.38795693526066	0.862890496762871\\
3.38983688799971	0.862883576801896\\
3.39171683654135	0.862876664890444\\
3.39359678088985	0.862869761015754\\
3.39547672104951	0.862862865165097\\
3.39735665702462	0.862855977325764\\
3.39923658881945	0.862849097485077\\
3.40111651643827	0.862842225630378\\
3.40299643988535	0.862835361749039\\
3.40487635916496	0.862828505828456\\
3.40675627428134	0.86282165785605\\
3.40863618523876	0.862814817819268\\
3.41051609204144	0.862807985705582\\
3.41239599469364	0.862801161502489\\
3.41427589319959	0.862794345197511\\
3.41615578756351	0.862787536778195\\
3.41803567778963	0.862780736232115\\
3.41991556388217	0.862773943546868\\
3.42179544584534	0.862767158710076\\
3.42367532368334	0.862760381709387\\
3.42555519740039	0.862753612532472\\
3.42743506700067	0.862746851167029\\
3.42931493248837	0.862740097600779\\
3.43119479386768	0.862733351821468\\
3.43307465114279	0.862726613816866\\
3.43495450431786	0.86271988357477\\
3.43683435339707	0.862713161082998\\
3.43871419838458	0.862706446329394\\
3.44059403928455	0.862699739301827\\
3.44247387610113	0.86269303998819\\
3.44435370883847	0.862686348376398\\
3.44623353750071	0.862679664454394\\
3.448113362092	0.862672988210141\\
3.44999318261645	0.86266631963163\\
3.45187299907821	0.862659658706872\\
3.4537528114814	0.862653005423905\\
3.45563261983012	0.862646359770789\\
3.45751242412849	0.862639721735609\\
3.45939222438062	0.862633091306473\\
3.4612720205906	0.862626468471513\\
3.46315181276254	0.862619853218884\\
3.46503160090052	0.862613245536766\\
3.46691138500863	0.862606645413361\\
3.46879116509095	0.862600052836895\\
3.47067094115155	0.862593467795618\\
3.4725507131945	0.862586890277801\\
3.47443048122387	0.862580320271742\\
3.47631024524372	0.862573757765759\\
3.47819000525809	0.862567202748194\\
3.48006976127104	0.862560655207414\\
3.48194951328662	0.862554115131806\\
3.48382926130885	0.862547582509782\\
3.48570900534178	0.862541057329777\\
3.48758874538943	0.862534539580247\\
3.48946848145582	0.862528029249674\\
3.49134821354498	0.862521526326559\\
3.49322794166091	0.862515030799429\\
3.49510766580762	0.862508542656831\\
3.49698738598912	0.862502061887337\\
3.4988671022094	0.862495588479541\\
3.50074681447246	0.862489122422057\\
3.50262652278228	0.862482663703524\\
3.50450622714284	0.862476212312604\\
3.50638592755813	0.862469768237978\\
3.50826562403211	0.862463331468353\\
3.51014531656875	0.862456901992457\\
3.51202500517201	0.862450479799038\\
3.51390468984586	0.862444064876869\\
3.51578437059423	0.862437657214744\\
3.51766404742108	0.862431256801478\\
3.51954372033035	0.86242486362591\\
3.52142338932598	0.8624184776769\\
3.5233030544119	0.862412098943328\\
3.52518271559203	0.8624057274141\\
3.52706237287031	0.862399363078139\\
3.52894202625063	0.862393005924393\\
3.53082167573692	0.862386655941831\\
3.53270132133308	0.862380313119442\\
3.53458096304302	0.862373977446239\\
3.53646060087063	0.862367648911254\\
3.53834023481981	0.862361327503542\\
3.54021986489443	0.86235501321218\\
3.54209949109839	0.862348706026264\\
3.54397911343556	0.862342405934914\\
3.54585873190981	0.862336112927269\\
3.54773834652501	0.86232982699249\\
3.54961795728502	0.86232354811976\\
3.5514975641937	0.862317276298282\\
3.5533771672549	0.86231101151728\\
3.55525676647248	0.862304753766\\
3.55713636185027	0.862298503033707\\
3.55901595339211	0.86229225930969\\
3.56089554110183	0.862286022583256\\
3.56277512498326	0.862279792843733\\
3.56465470504024	0.862273570080472\\
3.56653428127656	0.862267354282842\\
3.56841385369606	0.862261145440235\\
3.57029342230253	0.862254943542062\\
3.57217298709978	0.862248748577754\\
3.57405254809162	0.862242560536765\\
3.57593210528183	0.862236379408566\\
3.5778116586742	0.862230205182652\\
3.57969120827253	0.862224037848537\\
3.58157075408058	0.862217877395753\\
3.58345029610214	0.862211723813855\\
3.58532983434098	0.862205577092418\\
3.58720936880085	0.862199437221037\\
3.58908889948553	0.862193304189325\\
3.59096842639877	0.862187177986917\\
3.59284794954432	0.862181058603469\\
3.59472746892593	0.862174946028655\\
3.59660698454733	0.86216884025217\\
3.59848649641227	0.862162741263727\\
3.60036600452448	0.862156649053062\\
3.60224550888769	0.862150563609929\\
3.60412500950562	0.862144484924101\\
3.60600450638198	0.862138412985371\\
3.6078839995205	0.862132347783554\\
3.60976348892488	0.862126289308481\\
3.61164297459882	0.862120237550006\\
3.61352245654603	0.862114192497999\\
3.6154019347702	0.862108154142352\\
3.61728140927502	0.862102122472976\\
3.61916088006417	0.8620960974798\\
3.62104034714135	0.862090079152774\\
3.62291981051022	0.862084067481867\\
3.62479927017445	0.862078062457067\\
3.62667872613771	0.862072064068379\\
3.62855817840367	0.862066072305832\\
3.63043762697599	0.862060087159469\\
3.63231707185831	0.862054108619355\\
3.63419651305428	0.862048136675574\\
3.63607595056756	0.862042171318227\\
3.63795538440177	0.862036212537436\\
3.63983481456056	0.862030260323342\\
3.64171424104755	0.862024314666101\\
3.64359366386637	0.862018375555894\\
3.64547308302065	0.862012442982915\\
3.64735249851399	0.862006516937379\\
3.64923191035001	0.862000597409522\\
3.65111131853231	0.861994684389594\\
3.65299072306451	0.861988777867868\\
3.6548701239502	0.861982877834631\\
3.65674952119297	0.861976984280193\\
3.65862891479641	0.861971097194879\\
3.66050830476411	0.861965216569035\\
3.66238769109965	0.861959342393023\\
3.66426707380661	0.861953474657225\\
3.66614645288854	0.861947613352041\\
3.66802582834903	0.861941758467888\\
3.66990520019164	0.861935909995202\\
3.67178456841992	0.861930067924438\\
3.67366393303742	0.861924232246068\\
3.6755432940477	0.861918402950583\\
3.6774226514543	0.86191258002849\\
3.67930200526076	0.861906763470316\\
3.68118135547061	0.861900953266606\\
3.6830607020874	0.861895149407921\\
3.68494004511464	0.861889351884842\\
3.68681938455585	0.861883560687966\\
3.68869872041456	0.861877775807909\\
3.69057805269428	0.861871997235305\\
3.69245738139851	0.861866224960804\\
3.69433670653077	0.861860458975075\\
3.69621602809455	0.861854699268804\\
3.69809534609335	0.861848945832694\\
3.69997466053066	0.861843198657469\\
3.70185397140997	0.861837457733865\\
3.70373327873476	0.86183172305264\\
3.70561258250851	0.861825994604567\\
3.7074918827347	0.861820272380436\\
3.70937117941678	0.861814556371058\\
3.71125047255824	0.861808846567256\\
3.71312976216253	0.861803142959874\\
3.7150090482331	0.861797445539772\\
3.71688833077342	0.861791754297827\\
3.71876760978692	0.861786069224933\\
3.72064688527706	0.861780390312002\\
3.72252615724727	0.861774717549962\\
3.72440542570098	0.86176905092976\\
3.72628469064164	0.861763390442356\\
3.72816395207265	0.861757736078731\\
3.73004320999746	0.861752087829881\\
3.73192246441947	0.861746445686819\\
3.7338017153421	0.861740809640575\\
3.73568096276877	0.861735179682195\\
3.73756020670286	0.861729555802743\\
3.7394394471478	0.861723937993299\\
3.74131868410696	0.861718326244961\\
3.74319791758375	0.86171272054884\\
3.74507714758156	0.861707120896068\\
3.74695637410376	0.86170152727779\\
3.74883559715375	0.861695939685171\\
3.75071481673488	0.861690358109389\\
3.75259403285055	0.86168478254164\\
3.75447324550411	0.861679212973136\\
3.75635245469892	0.861673649395107\\
3.75823166043835	0.861668091798798\\
3.76011086272575	0.861662540175469\\
3.76199006156447	0.861656994516398\\
3.76386925695786	0.861651454812879\\
3.76574844890927	0.861645921056222\\
3.76762763742201	0.861640393237753\\
3.76950682249945	0.861634871348814\\
3.77138600414489	0.861629355380763\\
3.77326518236167	0.861623845324975\\
3.77514435715312	0.86161834117284\\
3.77702352852254	0.861612842915764\\
3.77890269647325	0.861607350545169\\
3.78078186100856	0.861601864052494\\
3.78266102213178	0.861596383429191\\
3.7845401798462	0.861590908666731\\
3.78641933415512	0.861585439756599\\
3.78829848506185	0.861579976690297\\
3.79017763256965	0.861574519459341\\
3.79205677668183	0.861569068055264\\
3.79393591740166	0.861563622469613\\
3.79581505473241	0.861558182693953\\
3.79769418867736	0.861552748719864\\
3.79957331923978	0.861547320538939\\
3.80145244642294	0.86154189814279\\
3.80333157023008	0.861536481523042\\
3.80521069066448	0.861531070671336\\
3.80708980772937	0.86152566557933\\
3.80896892142801	0.861520266238694\\
3.81084803176365	0.861514872641117\\
3.81272713873952	0.861509484778301\\
3.81460624235886	0.861504102641964\\
3.81648534262491	0.861498726223838\\
3.81836443954088	0.861493355515673\\
3.82024353311001	0.861487990509231\\
3.82212262333551	0.861482631196291\\
3.82400171022061	0.861477277568647\\
3.8258807937685	0.861471929618108\\
3.82775987398241	0.861466587336496\\
3.82963895086553	0.861461250715651\\
3.83151802442107	0.861455919747426\\
3.83339709465223	0.86145059442369\\
3.83527616156218	0.861445274736326\\
3.83715522515413	0.861439960677232\\
3.83903428543126	0.861434652238322\\
3.84091334239675	0.861429349411523\\
3.84279239605377	0.861424052188777\\
3.8446714464055	0.861418760562043\\
3.84655049345511	0.861413474523292\\
3.84842953720576	0.86140819406451\\
3.85030857766061	0.861402919177699\\
3.85218761482282	0.861397649854874\\
3.85406664869555	0.861392386088067\\
3.85594567928194	0.861387127869322\\
3.85782470658513	0.861381875190698\\
3.85970373060828	0.861376628044269\\
3.86158275135451	0.861371386422123\\
3.86346176882697	0.861366150316364\\
3.86534078302878	0.861360919719107\\
3.86721979396306	0.861355694622485\\
3.86909880163295	0.861350475018643\\
3.87097780604155	0.861345260899741\\
3.87285680719199	0.861340052257952\\
3.87473580508737	0.861334849085466\\
3.8766147997308	0.861329651374485\\
3.87849379112539	0.861324459117226\\
3.88037277927423	0.861319272305918\\
3.88225176418042	0.861314090932808\\
3.88413074584705	0.861308914990153\\
3.88600972427721	0.861303744470227\\
3.88788869947399	0.861298579365317\\
3.88976767144045	0.861293419667723\\
3.89164664017969	0.86128826536976\\
3.89352560569477	0.861283116463757\\
3.89540456798876	0.861277972942056\\
3.89728352706473	0.861272834797013\\
3.89916248292574	0.861267702020999\\
3.90104143557485	0.861262574606398\\
3.9029203850151	0.861257452545606\\
3.90479933124956	0.861252335831037\\
3.90667827428126	0.861247224455113\\
3.90855721411326	0.861242118410274\\
3.91043615074858	0.861237017688973\\
3.91231508419027	0.861231922283675\\
3.91419401444136	0.861226832186859\\
3.91607294150486	0.861221747391019\\
3.91795186538382	0.861216667888661\\
3.91983078608125	0.861211593672304\\
3.92170970360016	0.861206524734483\\
3.92358861794357	0.861201461067743\\
3.92546752911449	0.861196402664646\\
3.92734643711593	0.861191349517764\\
3.92922534195088	0.861186301619684\\
3.93110424362234	0.861181258963007\\
3.93298314213331	0.861176221540345\\
3.93486203748679	0.861171189344325\\
3.93674092968575	0.861166162367588\\
3.93861981873319	0.861161140602786\\
3.94049870463207	0.861156124042585\\
3.94237758738539	0.861151112679664\\
3.9442564669961	0.861146106506716\\
3.94613534346719	0.861141105516447\\
3.94801421680161	0.861136109701574\\
3.94989308700233	0.861131119054828\\
3.95177195407231	0.861126133568956\\
3.9536508180145	0.861121153236713\\
3.95552967883185	0.86111617805087\\
3.95740853652731	0.861111208004211\\
3.95928739110383	0.861106243089531\\
3.96116624256434	0.861101283299639\\
3.96304509091178	0.861096328627356\\
3.96492393614909	0.861091379065518\\
3.96680277827919	0.861086434606972\\
3.96868161730501	0.861081495244577\\
3.97056045322948	0.861076560971206\\
3.97243928605551	0.861071631779745\\
3.97431811578601	0.861066707663091\\
3.9761969424239	0.861061788614155\\
3.97807576597209	0.86105687462586\\
3.97995458643349	0.861051965691142\\
3.98183340381099	0.861047061802948\\
3.98371221810749	0.861042162954241\\
3.98559102932588	0.861037269137993\\
3.98746983746906	0.861032380347189\\
3.98934864253992	0.861027496574829\\
3.99122744454133	0.861022617813921\\
3.99310624347619	0.86101774405749\\
3.99498503934735	0.861012875298571\\
3.99686383215771	0.861008011530211\\
3.99874262191012	0.861003152745469\\
4.00062140860745	0.860998298937419\\
4.00250019225258	0.860993450099144\\
4.00437897284834	0.86098860622374\\
4.00625775039761	0.860983767304318\\
4.00813652490324	0.860978933333997\\
4.01001529636807	0.860974104305911\\
4.01189406479494	0.860969280213204\\
4.01377283018671	0.860964461049035\\
4.0156515925462	0.860959646806571\\
4.01753035187626	0.860954837478996\\
4.01940910817972	0.860950033059501\\
4.0212878614594	0.860945233541293\\
4.02316661171813	0.860940438917588\\
4.02504535895873	0.860935649181616\\
4.02692410318402	0.860930864326618\\
4.02880284439682	0.860926084345847\\
4.03068158259993	0.860921309232567\\
4.03256031779616	0.860916538980055\\
4.03443904998832	0.8609117735816\\
4.0363177791792	0.860907013030502\\
4.03819650537162	0.860902257320073\\
4.04007522856835	0.860897506443636\\
4.04195394877219	0.860892760394527\\
4.04383266598594	0.860888019166094\\
4.04571138021236	0.860883282751694\\
4.04759009145425	0.860878551144698\\
4.04946879971438	0.860873824338488\\
4.05134750499553	0.860869102326457\\
4.05322620730046	0.860864385102012\\
4.05510490663195	0.860859672658567\\
4.05698360299276	0.860854964989552\\
4.05886229638564	0.860850262088405\\
4.06074098681336	0.860845563948579\\
4.06261967427867	0.860840870563535\\
4.06449835878432	0.860836181926747\\
4.06637704033306	0.8608314980317\\
4.06825571892763	0.860826818871892\\
4.07013439457078	0.860822144440829\\
4.07201306726524	0.860817474732032\\
4.07389173701374	0.86081280973903\\
4.07577040381902	0.860808149455366\\
4.07764906768381	0.860803493874592\\
4.07952772861082	0.860798842990273\\
4.08140638660279	0.860794196795985\\
4.08328504166242	0.860789555285313\\
4.08516369379244	0.860784918451855\\
4.08704234299555	0.860780286289221\\
4.08892098927446	0.860775658791029\\
4.09079963263188	0.860771035950912\\
4.09267827307051	0.860766417762512\\
4.09455691059305	0.86076180421948\\
4.09643554520219	0.860757195315482\\
4.09831417690062	0.860752591044193\\
4.10019280569104	0.860747991399298\\
4.10207143157613	0.860743396374494\\
4.10395005455857	0.860738805963489\\
4.10582867464103	0.860734220160002\\
4.10770729182621	0.860729638957763\\
4.10958590611676	0.860725062350512\\
4.11146451751537	0.860720490332\\
4.11334312602468	0.860715922895989\\
4.11522173164738	0.860711360036252\\
4.11710033438611	0.860706801746573\\
4.11897893424354	0.860702248020746\\
4.12085753122231	0.860697698852575\\
4.12273612532508	0.860693154235877\\
4.1246147165545	0.860688614164478\\
4.12649330491321	0.860684078632214\\
4.12837189040385	0.860679547632934\\
4.13025047302905	0.860675021160495\\
4.13212905279146	0.860670499208767\\
4.1340076296937	0.860665981771628\\
4.13588620373841	0.860661468842968\\
4.1377647749282	0.860656960416688\\
4.1396433432657	0.860652456486698\\
4.14152190875353	0.860647957046919\\
4.1434004713943	0.860643462091284\\
4.14527903119063	0.860638971613735\\
4.14715758814513	0.860634485608223\\
4.1490361422604	0.860630004068712\\
4.15091469353904	0.860625526989174\\
4.15279324198367	0.860621054363595\\
4.15467178759686	0.860616586185966\\
4.15655033038122	0.860612122450293\\
4.15842887033934	0.86060766315059\\
4.16030740747381	0.860603208280881\\
4.16218594178721	0.860598757835201\\
4.16406447328212	0.860594311807596\\
4.16594300196112	0.86058987019212\\
4.16782152782679	0.860585432982839\\
4.16970005088169	0.860581000173828\\
4.17157857112841	0.860576571759174\\
4.1734570885695	0.860572147732971\\
4.17533560320753	0.860567728089326\\
4.17721411504505	0.860563312822355\\
4.17909262408464	0.860558901926183\\
4.18097113032883	0.860554495394946\\
4.18284963378019	0.86055009322279\\
4.18472813444125	0.860545695403871\\
4.18660663231457	0.860541301932355\\
4.18848512740269	0.860536912802417\\
4.19036361970814	0.860532528008244\\
4.19224210923346	0.86052814754403\\
4.19412059598119	0.860523771403981\\
4.19599907995385	0.860519399582313\\
4.19787756115398	0.86051503207325\\
4.19975603958409	0.860510668871027\\
4.20163451524671	0.860506309969889\\
4.20351298814435	0.86050195536409\\
4.20539145827953	0.860497605047894\\
4.20726992565477	0.860493259015576\\
4.20914839027257	0.860488917261419\\
4.21102685213544	0.860484579779716\\
4.21290531124587	0.86048024656477\\
4.21478376760638	0.860475917610893\\
4.21666222121946	0.860471592912409\\
4.2185406720876	0.860467272463648\\
4.2204191202133	0.860462956258952\\
4.22229756559904	0.860458644292672\\
4.22417600824731	0.860454336559168\\
4.22605444816059	0.860450033052812\\
4.22793288534136	0.860445733767981\\
4.2298113197921	0.860441438699065\\
4.23168975151528	0.860437147840462\\
4.23356818051337	0.860432861186581\\
4.23544660678884	0.860428578731839\\
4.23732503034416	0.860424300470663\\
4.23920345118179	0.860420026397488\\
4.24108186930418	0.86041575650676\\
4.2429602847138	0.860411490792935\\
4.24483869741309	0.860407229250476\\
4.24671710740452	0.860402971873856\\
4.24859551469051	0.86039871865756\\
4.25047391927353	0.860394469596078\\
4.25235232115601	0.860390224683912\\
4.25423072034039	0.860385983915573\\
4.25610911682911	0.86038174728558\\
4.2579875106246	0.860377514788463\\
4.25986590172929	0.860373286418759\\
4.2617442901456	0.860369062171017\\
4.26362267587598	0.860364842039791\\
4.26550105892282	0.860360626019649\\
4.26737943928857	0.860356414105165\\
4.26925781697562	0.860352206290922\\
4.2711361919864	0.860348002571513\\
4.27301456432331	0.860343802941541\\
4.27489293398876	0.860339607395616\\
4.27677130098516	0.860335415928359\\
4.27864966531492	0.860331228534397\\
4.28052802698042	0.86032704520837\\
4.28240638598406	0.860322865944923\\
4.28428474232825	0.860318690738714\\
4.28616309601536	0.860314519584406\\
4.28804144704779	0.860310352476673\\
4.28991979542793	0.860306189410198\\
4.29179814115815	0.860302030379672\\
4.29367648424083	0.860297875379795\\
4.29555482467836	0.860293724405277\\
4.29743316247311	0.860289577450835\\
4.29931149762744	0.860285434511196\\
4.30118983014372	0.860281295581096\\
4.30306816002433	0.860277160655278\\
4.30494648727163	0.860273029728497\\
4.30682481188797	0.860268902795512\\
4.30870313387571	0.860264779851096\\
4.31058145323721	0.860260660890026\\
4.31245976997481	0.860256545907091\\
4.31433808409088	0.860252434897087\\
4.31621639558775	0.860248327854819\\
4.31809470446777	0.860244224775101\\
4.31997301073328	0.860240125652756\\
4.32185131438662	0.860236030482613\\
4.32372961543012	0.860231939259512\\
4.32560791386611	0.860227851978302\\
4.32748620969694	0.860223768633838\\
4.32936450292491	0.860219689220987\\
4.33124279355237	0.860215613734621\\
4.33312108158162	0.860211542169622\\
4.334999367015	0.860207474520881\\
4.33687764985481	0.860203410783297\\
4.33875593010337	0.860199350951777\\
4.34063420776299	0.860195295021237\\
4.34251248283599	0.8601912429866\\
4.34439075532465	0.860187194842801\\
4.3462690252313	0.860183150584779\\
4.34814729255823	0.860179110207483\\
4.35002555730773	0.860175073705872\\
4.3519038194821	0.860171041074911\\
4.35378207908364	0.860167012309574\\
4.35566033611463	0.860162987404844\\
4.35753859057737	0.860158966355712\\
4.35941684247413	0.860154949157176\\
4.36129509180719	0.860150935804244\\
4.36317333857884	0.86014692629193\\
4.36505158279136	0.860142920615259\\
4.36692982444701	0.860138918769263\\
4.36880806354807	0.860134920748981\\
4.37068630009681	0.860130926549462\\
4.37256453409549	0.860126936165761\\
4.37444276554637	0.860122949592943\\
4.37632099445173	0.860118966826081\\
4.3781992208138	0.860114987860255\\
4.38007744463486	0.860111012690553\\
4.38195566591715	0.860107041312072\\
4.38383388466292	0.860103073719917\\
4.38571210087442	0.860099109909201\\
4.3875903145539	0.860095149875044\\
4.3894685257036	0.860091193612575\\
4.39134673432575	0.86008724111693\\
4.3932249404226	0.860083292383255\\
4.39510314399638	0.860079347406701\\
4.39698134504932	0.86007540618243\\
4.39885954358365	0.860071468705608\\
4.4007377396016	0.860067534971414\\
4.4026159331054	0.86006360497503\\
4.40449412409726	0.86005967871165\\
4.4063723125794	0.860055756176472\\
4.40825049855404	0.860051837364704\\
4.4101286820234	0.860047922271563\\
4.41200686298968	0.86004401089227\\
4.4138850414551	0.860040103222058\\
4.41576321742186	0.860036199256164\\
4.41764139089216	0.860032298989837\\
4.41951956186821	0.860028402418329\\
4.42139773035221	0.860024509536903\\
4.42327589634634	0.860020620340828\\
4.42515405985282	0.860016734825383\\
4.42703222087381	0.860012852985852\\
4.42891037941153	0.860008974817528\\
4.43078853546814	0.860005100315711\\
4.43266668904585	0.86000122947571\\
4.43454484014681	0.85999736229284\\
4.43642298877322	0.859993498762424\\
4.43830113492726	0.859989638879793\\
4.44017927861109	0.859985782640286\\
4.44205741982688	0.859981930039248\\
4.44393555857681	0.859978081072033\\
4.44581369486304	0.859974235734003\\
4.44769182868773	0.859970394020525\\
4.44956996005305	0.859966555926975\\
4.45144808896115	0.859962721448738\\
4.45332621541419	0.859958890581203\\
4.45520433941433	0.85995506331977\\
4.45708246096371	0.859951239659845\\
4.45896058006449	0.85994741959684\\
4.4608386967188	0.859943603126176\\
4.4627168109288	0.859939790243282\\
4.46459492269663	0.859935980943593\\
4.46647303202442	0.859932175222552\\
4.46835113891431	0.859928373075608\\
4.47022924336844	0.85992457449822\\
4.47210734538893	0.859920779485852\\
4.47398544497792	0.859916988033976\\
4.47586354213753	0.859913200138072\\
4.47774163686989	0.859909415793626\\
4.47961972917712	0.859905634996133\\
4.48149781906134	0.859901857741094\\
4.48337590652466	0.859898084024017\\
4.4852539915692	0.859894313840418\\
4.48713207419708	0.859890547185819\\
4.48901015441039	0.859886784055751\\
4.49088823221126	0.859883024445751\\
4.49276630760178	0.859879268351363\\
4.49464438058405	0.85987551576814\\
4.49652245116019	0.859871766691639\\
4.49840051933227	0.859868021117426\\
4.50027858510241	0.859864279041075\\
4.50215664847269	0.859860540458166\\
4.50403470944521	0.859856805364284\\
4.50591276802204	0.859853073755026\\
4.50779082420529	0.859849345625992\\
4.50966887799702	0.859845620972789\\
4.51154692939933	0.859841899791035\\
4.51342497841428	0.85983818207635\\
4.51530302504397	0.859834467824364\\
4.51718106929045	0.859830757030714\\
4.51905911115581	0.859827049691043\\
4.52093715064211	0.859823345801\\
4.52281518775142	0.859819645356243\\
4.5246932224858	0.859815948352437\\
4.52657125484732	0.859812254785251\\
4.52844928483803	0.859808564650365\\
4.53032731246	0.859804877943463\\
4.53220533771527	0.859801194660236\\
4.53408336060591	0.859797514796383\\
4.53596138113395	0.85979383834761\\
4.53783939930146	0.859790165309628\\
4.53971741511047	0.859786495678158\\
4.54159542856302	0.859782829448923\\
4.54347343966117	0.859779166617659\\
4.54535144840695	0.859775507180103\\
4.54722945480239	0.859771851132002\\
4.54910745884954	0.859768198469109\\
4.55098546055041	0.859764549187184\\
4.55286345990705	0.859760903281993\\
4.55474145692147	0.859757260749309\\
4.55661945159571	0.859753621584912\\
4.55849744393178	0.85974998578459\\
4.56037543393171	0.859746353344134\\
4.56225342159752	0.859742724259345\\
4.56413140693122	0.85973909852603\\
4.56600938993482	0.859735476140002\\
4.56788737061034	0.85973185709708\\
4.56976534895978	0.859728241393092\\
4.57164332498515	0.859724629023869\\
4.57352129868846	0.859721019985253\\
4.57539927007171	0.859717414273089\\
4.5772772391369	0.859713811883229\\
4.57915520588602	0.859710212811534\\
4.58103317032108	0.859706617053869\\
4.58291113244406	0.859703024606106\\
4.58478909225695	0.859699435464125\\
4.58666704976175	0.859695849623811\\
4.58854500496044	0.859692267081056\\
4.59042295785501	0.859688687831758\\
4.59230090844743	0.859685111871823\\
4.59417885673969	0.859681539197161\\
4.59605680273377	0.85967796980369\\
4.59793474643163	0.859674403687335\\
4.59981268783526	0.859670840844026\\
4.60169062694662	0.8596672812697\\
4.60356856376769	0.859663724960301\\
4.60544649830042	0.859660171911777\\
4.60732443054678	0.859656622120087\\
4.60920236050874	0.859653075581192\\
4.61108028818826	0.85964953229106\\
4.61295821358729	0.859645992245667\\
4.61483613670778	0.859642455440995\\
4.6167140575517	0.859638921873032\\
4.61859197612098	0.859635391537771\\
4.62046989241759	0.859631864431213\\
4.62234780644347	0.859628340549364\\
4.62422571820056	0.859624819888238\\
4.6261036276908	0.859621302443853\\
4.62798153491613	0.859617788212236\\
4.6298594398785	0.859614277189417\\
4.63173734257983	0.859610769371435\\
4.63361524302207	0.859607264754333\\
4.63549314120714	0.859603763334162\\
4.63737103713696	0.859600265106978\\
4.63924893081348	0.859596770068844\\
4.64112682223861	0.859593278215828\\
4.64300471141428	0.859589789544005\\
4.6448825983424	0.859586304049457\\
4.6467604830249	0.85958282172827\\
4.64863836546369	0.859579342576538\\
4.65051624566069	0.859575866590359\\
4.6523941236178	0.85957239376584\\
4.65427199933694	0.859568924099091\\
4.65614987282001	0.859565457586231\\
4.65802774406893	0.859561994223382\\
4.65990561308559	0.859558534006675\\
4.66178347987189	0.859555076932245\\
4.66366134442974	0.859551622996233\\
4.66553920676103	0.859548172194788\\
4.66741706686765	0.859544724524062\\
4.66929492475151	0.859541279980216\\
4.67117278041448	0.859537838559415\\
4.67305063385846	0.859534400257831\\
4.67492848508534	0.859530965071641\\
4.676806334097	0.859527532997028\\
4.67868418089532	0.859524104030182\\
4.68056202548217	0.859520678167298\\
4.68243986785945	0.859517255404577\\
4.68431770802903	0.859513835738227\\
4.68619554599277	0.85951041916446\\
4.68807338175256	0.859507005679495\\
4.68995121531026	0.859503595279557\\
4.69182904666773	0.859500187960875\\
4.69370687582685	0.859496783719688\\
4.69558470278948	0.859493382552235\\
4.69746252755747	0.859489984454767\\
4.6993403501327	0.859486589423535\\
4.70121817051701	0.859483197454801\\
4.70309598871226	0.859479808544829\\
4.70497380472031	0.85947642268989\\
4.706851618543	0.859473039886261\\
4.70872943018218	0.859469660130225\\
4.71060723963971	0.85946628341807\\
4.71248504691742	0.859462909746091\\
4.71436285201716	0.859459539110586\\
4.71624065494077	0.859456171507862\\
4.71811845569009	0.85945280693423\\
4.71999625426695	0.859449445386006\\
4.72187405067319	0.859446086859513\\
4.72375184491065	0.859442731351079\\
4.72562963698115	0.859439378857039\\
4.72750742688652	0.859436029373732\\
4.72938521462859	0.859432682897502\\
4.73126300020918	0.8594293394247\\
4.73314078363012	0.859425998951684\\
4.73501856489322	0.859422661474814\\
4.73689634400031	0.859419326990459\\
4.73877412095319	0.859415995494991\\
4.7406518957537	0.85941266698479\\
4.74252966840363	0.859409341456239\\
4.7444074389048	0.859406018905728\\
4.74628520725901	0.859402699329653\\
4.74816297346808	0.859399382724414\\
4.75004073753381	0.859396069086418\\
4.75191849945799	0.859392758412076\\
4.75379625924244	0.859389450697807\\
4.75567401688895	0.859386145940032\\
4.75755177239931	0.859382844135181\\
4.75942952577532	0.859379545279687\\
4.76130727701877	0.859376249369989\\
4.76318502613146	0.859372956402532\\
4.76506277311517	0.859369666373766\\
4.76694051797168	0.859366379280147\\
4.76881826070279	0.859363095118135\\
4.77069600131028	0.859359813884197\\
4.77257373979592	0.859356535574806\\
4.77445147616149	0.859353260186437\\
4.77632921040878	0.859349987715573\\
4.77820694253955	0.859346718158703\\
4.78008467255558	0.859343451512319\\
4.78196240045865	0.859340187772921\\
4.78384012625051	0.859336926937011\\
4.78571784993294	0.8593336690011\\
4.7875955715077	0.859330413961702\\
4.78947329097655	0.859327161815336\\
4.79135100834126	0.859323912558529\\
4.79322872360358	0.859320666187809\\
4.79510643676528	0.859317422699714\\
4.7969841478281	0.859314182090784\\
4.79886185679381	0.859310944357565\\
4.80073956366415	0.859307709496609\\
4.80261726844087	0.859304477504473\\
4.80449497112572	0.859301248377717\\
4.80637267172044	0.859298022112911\\
4.80825037022678	0.859294798706626\\
4.81012806664649	0.859291578155439\\
4.8120057609813	0.859288360455933\\
4.81388345323294	0.859285145604697\\
4.81576114340317	0.859281933598323\\
4.8176388314937	0.859278724433409\\
4.81951651750627	0.85927551810656\\
4.82139420144262	0.859272314614384\\
4.82327188330447	0.859269113953494\\
4.82514956309355	0.859265916120509\\
4.82702724081158	0.859262721112053\\
4.82890491646029	0.859259528924755\\
4.83078259004139	0.859256339555249\\
4.83266026155661	0.859253153000175\\
4.83453793100766	0.859249969256177\\
4.83641559839627	0.859246788319903\\
4.83829326372413	0.85924361018801\\
4.84017092699297	0.859240434857155\\
4.8420485882045	0.859237262324003\\
4.84392624736042	0.859234092585225\\
4.84580390446243	0.859230925637494\\
4.84768155951225	0.85922776147749\\
4.84955921251157	0.859224600101898\\
4.8514368634621	0.859221441507407\\
4.85331451236553	0.859218285690711\\
4.85519215922356	0.859215132648511\\
4.85706980403788	0.85921198237751\\
4.8589474468102	0.859208834874418\\
4.8608250875422	0.85920569013595\\
4.86270272623556	0.859202548158825\\
4.86458036289198	0.859199408939767\\
4.86645799751314	0.859196272475504\\
4.86833563010073	0.859193138762772\\
4.87021326065643	0.859190007798309\\
4.87209088918192	0.859186879578859\\
4.87396851567887	0.859183754101171\\
4.87584614014897	0.859180631361999\\
4.87772376259389	0.8591775113581\\
4.8796013830153	0.859174394086238\\
4.88147900141487	0.859171279543182\\
4.88335661779427	0.859168167725704\\
4.88523423215517	0.859165058630582\\
4.88711184449924	0.8591619522546\\
4.88898945482813	0.859158848594544\\
4.89086706314352	0.859155747647206\\
4.89274466944705	0.859152649409385\\
4.8946222737404	0.859149553877881\\
4.89649987602521	0.859146461049502\\
4.89837747630314	0.859143370921059\\
4.90025507457585	0.859140283489367\\
4.90213267084498	0.859137198751249\\
4.90401026511218	0.859134116703529\\
4.90588785737911	0.859131037343038\\
4.90776544764741	0.859127960666612\\
4.90964303591872	0.859124886671089\\
4.91152062219468	0.859121815353316\\
4.91339820647694	0.85911874671014\\
4.91527578876713	0.859115680738416\\
4.91715336906689	0.859112617435003\\
4.91903094737786	0.859109556796763\\
4.92090852370167	0.859106498820566\\
4.92278609803995	0.859103443503283\\
4.92466367039433	0.859100390841792\\
4.92654124076645	0.859097340832975\\
4.92841880915791	0.859094293473718\\
4.93029637557036	0.859091248760912\\
4.93217394000541	0.859088206691455\\
4.93405150246469	0.859085167262245\\
4.9359290629498	0.859082130470187\\
4.93780662146238	0.859079096312193\\
4.93968417800404	0.859076064785175\\
4.94156173257639	0.859073035886052\\
4.94343928518105	0.859070009611749\\
4.94531683581962	0.859066985959192\\
4.94719438449371	0.859063964925314\\
4.94907193120493	0.859060946507053\\
4.95094947595489	0.859057930701349\\
4.95282701874519	0.859054917505149\\
4.95470455957743	0.859051906915403\\
4.95658209845322	0.859048898929067\\
4.95845963537414	0.8590458935431\\
4.96033717034181	0.859042890754466\\
4.9622147033578	0.859039890560134\\
4.96409223442373	0.859036892957076\\
4.96596976354117	0.859033897942271\\
4.96784729071172	0.859030905512699\\
4.96972481593696	0.859027915665349\\
4.97160233921849	0.85902492839721\\
4.97347986055789	0.859021943705278\\
4.97535737995673	0.859018961586553\\
4.97723489741661	0.859015982038038\\
4.9791124129391	0.859013005056743\\
4.98098992652579	0.85901003063968\\
4.98286743817824	0.859007058783866\\
4.98474494789803	0.859004089486324\\
4.98662245568673	0.85900112274408\\
4.98849996154592	0.858998158554164\\
4.99037746547717	0.858995196913611\\
4.99225496748204	0.85899223781946\\
4.99413246756209	0.858989281268756\\
4.99600996571891	0.858986327258545\\
4.99788746195404	0.85898337578588\\
4.99976495626905	0.858980426847818\\
5.00164244866549	0.85897748044142\\
5.00351993914493	0.858974536563751\\
5.00539742770893	0.85897159521188\\
5.00727491435903	0.858968656382881\\
5.00915239909679	0.858965720073833\\
5.01102988192376	0.858962786281818\\
5.01290736284149	0.858959855003923\\
5.01478484185153	0.858956926237238\\
5.01666231895542	0.858953999978859\\
5.01853979415471	0.858951076225885\\
5.02041726745095	0.85894815497542\\
5.02229473884567	0.858945236224572\\
5.02417220834041	0.858942319970453\\
5.02604967593671	0.858939406210179\\
5.02792714163611	0.858936494940872\\
5.02980460544014	0.858933586159655\\
5.03168206735034	0.858930679863659\\
5.03355952736823	0.858927776050015\\
5.03543698549535	0.858924874715862\\
5.03731444173323	0.858921975858341\\
5.03919189608339	0.858919079474598\\
5.04106934854735	0.858916185561783\\
5.04294679912665	0.85891329411705\\
5.04482424782279	0.858910405137556\\
5.04670169463731	0.858907518620465\\
5.04857913957172	0.858904634562943\\
5.05045658262754	0.85890175296216\\
5.05233402380627	0.858898873815292\\
5.05421146310945	0.858895997119516\\
5.05608890053857	0.858893122872016\\
5.05796633609515	0.858890251069979\\
5.0598437697807	0.858887381710597\\
5.06172120159672	0.858884514791063\\
5.06359863154472	0.858881650308578\\
5.06547605962621	0.858878788260345\\
5.06735348584268	0.858875928643571\\
5.06923091019564	0.858873071455468\\
5.07110833268659	0.858870216693251\\
5.07298575331702	0.85886736435414\\
5.07486317208843	0.858864514435358\\
5.07674058900232	0.858861666934133\\
5.07861800406017	0.858858821847696\\
5.08049541726349	0.858855979173284\\
5.08237282861375	0.858853138908136\\
5.08425023811245	0.858850301049494\\
5.08612764576107	0.858847465594608\\
5.0880050515611	0.858844632540729\\
5.08988245551402	0.858841801885111\\
5.09175985762132	0.858838973625016\\
5.09363725788447	0.858836147757705\\
5.09551465630496	0.858833324280447\\
5.09739205288425	0.858830503190514\\
5.09926944762383	0.858827684485179\\
5.10114684052517	0.858824868161723\\
5.10302423158974	0.858822054217429\\
5.10490162081901	0.858819242649584\\
5.10677900821446	0.858816433455479\\
5.10865639377755	0.858813626632409\\
5.11053377750974	0.858810822177673\\
5.1124111594125	0.858808020088573\\
5.1142885394873	0.858805220362417\\
5.1161659177356	0.858802422996514\\
5.11804329415885	0.858799627988179\\
5.11992066875851	0.85879683533473\\
5.12179804153605	0.85879404503349\\
5.12367541249291	0.858791257081783\\
5.12555278163056	0.858788471476941\\
5.12743014895044	0.858785688216296\\
5.129307514454	0.858782907297187\\
5.1311848781427	0.858780128716954\\
5.13306224001798	0.858777352472942\\
5.13493960008128	0.858774578562501\\
5.13681695833406	0.858771806982983\\
5.13869431477776	0.858769037731744\\
5.14057166941381	0.858766270806146\\
5.14244902224366	0.858763506203551\\
5.14432637326875	0.858760743921328\\
5.14620372249051	0.858757983956849\\
5.14808106991038	0.858755226307489\\
5.14995841552979	0.858752470970627\\
5.15183575935018	0.858749717943646\\
5.15371310137298	0.858746967223932\\
5.15559044159961	0.858744218808877\\
5.15746778003151	0.858741472695874\\
5.1593451166701	0.858738728882321\\
5.16122245151681	0.85873598736562\\
5.16309978457306	0.858733248143175\\
5.16497711584027	0.858730511212397\\
5.16685444531987	0.858727776570697\\
5.16873177301327	0.858725044215492\\
5.17060909892189	0.858722314144203\\
5.17248642304715	0.858719586354252\\
5.17436374539046	0.858716860843068\\
5.17624106595323	0.858714137608081\\
5.17811838473688	0.858711416646726\\
5.17999570174282	0.858708697956442\\
5.18187301697245	0.858705981534671\\
5.18375033042719	0.858703267378858\\
5.18562764210843	0.858700555486454\\
5.18750495201759	0.85869784585491\\
5.18938226015607	0.858695138481683\\
5.19125956652526	0.858692433364234\\
5.19313687112657	0.858689730500026\\
5.1950141739614	0.858687029886528\\
5.19689147503114	0.858684331521209\\
5.19876877433719	0.858681635401545\\
5.20064607188095	0.858678941525014\\
5.2025233676638	0.858676249889098\\
5.20440066168714	0.858673560491282\\
5.20627795395235	0.858670873329054\\
5.20815524446083	0.858668188399908\\
5.21003253321397	0.85866550570134\\
5.21190982021314	0.858662825230849\\
5.21378710545973	0.858660146985939\\
5.21566438895513	0.858657470964115\\
5.21754167070072	0.858654797162889\\
5.21941895069787	0.858652125579774\\
5.22129622894796	0.858649456212287\\
5.22317350545238	0.858646789057949\\
5.22505078021249	0.858644124114285\\
5.22692805322967	0.858641461378822\\
5.22880532450529	0.858638800849092\\
5.23068259404073	0.858636142522629\\
5.23255986183735	0.858633486396972\\
5.23443712789653	0.858630832469662\\
5.23631439221962	0.858628180738244\\
5.238191654808	0.858625531200268\\
5.24006891566302	0.858622883853286\\
5.24194617478605	0.858620238694852\\
5.24382343217846	0.858617595722527\\
5.2457006878416	0.858614954933872\\
5.24757794177683	0.858612316326454\\
5.24945519398551	0.858609679897841\\
5.25133244446898	0.858607045645607\\
5.25320969322862	0.858604413567328\\
5.25508694026577	0.858601783660584\\
5.25696418558177	0.858599155922957\\
5.25884142917799	0.858596530352033\\
5.26071867105576	0.858593906945404\\
5.26259591121645	0.858591285700662\\
5.26447314966138	0.858588666615403\\
5.26635038639191	0.858586049687227\\
5.26822762140938	0.858583434913739\\
5.27010485471513	0.858580822292544\\
5.27198208631051	0.858578211821253\\
5.27385931619684	0.858575603497478\\
5.27573654437547	0.858572997318838\\
5.27761377084773	0.858570393282951\\
5.27949099561496	0.858567791387442\\
5.2813682186785	0.858565191629937\\
5.28324544003966	0.858562594008066\\
5.28512265969979	0.858559998519464\\
5.28699987766022	0.858557405161765\\
5.28887709392226	0.858554813932611\\
5.29075430848725	0.858552224829645\\
5.29263152135652	0.858549637850514\\
5.29450873253138	0.858547052992867\\
5.29638594201316	0.858544470254358\\
5.29826314980317	0.858541889632643\\
5.30014035590275	0.858539311125382\\
5.3020175603132	0.858536734730238\\
5.30389476303585	0.858534160444878\\
5.30577196407201	0.858531588266971\\
5.30764916342299	0.85852901819419\\
5.3095263610901	0.858526450224211\\
5.31140355707466	0.858523884354713\\
5.31328075137799	0.858521320583379\\
5.31515794400137	0.858518758907895\\
5.31703513494614	0.85851619932595\\
5.31891232421358	0.858513641835235\\
5.320789511805	0.858511086433448\\
5.32266669772172	0.858508533118285\\
5.32454388196502	0.85850598188745\\
5.32642106453622	0.858503432738647\\
5.32829824543661	0.858500885669584\\
5.33017542466748	0.858498340677973\\
5.33205260223014	0.858495797761529\\
5.33392977812588	0.858493256917969\\
5.33580695235599	0.858490718145015\\
5.33768412492177	0.85848818144039\\
5.33956129582451	0.858485646801822\\
5.3414384650655	0.858483114227041\\
5.34331563264603	0.858480583713782\\
5.34519279856738	0.85847805525978\\
5.34706996283084	0.858475528862776\\
5.34894712543769	0.858473004520513\\
5.35082428638923	0.858470482230737\\
5.35270144568672	0.858467961991197\\
5.35457860333146	0.858465443799646\\
5.35645575932472	0.858462927653839\\
5.35833291366778	0.858460413551535\\
5.36021006636192	0.858457901490496\\
5.3620872174084	0.858455391468487\\
5.36396436680852	0.858452883483275\\
5.36584151456353	0.858450377532631\\
5.36771866067472	0.85844787361433\\
5.36959580514335	0.858445371726149\\
5.37147294797068	0.858442871865868\\
5.373350089158	0.858440374031271\\
5.37522722870656	0.858437878220144\\
5.37710436661763	0.858435384430276\\
5.37898150289248	0.858432892659459\\
5.38085863753236	0.858430402905491\\
5.38273577053854	0.858427915166168\\
5.38461290191228	0.858425429439293\\
5.38649003165483	0.85842294572267\\
5.38836715976746	0.858420464014107\\
5.39024428625142	0.858417984311415\\
5.39212141110797	0.858415506612408\\
5.39399853433835	0.858413030914902\\
5.39587565594383	0.858410557216716\\
5.39775277592565	0.858408085515675\\
5.39962989428506	0.858405615809603\\
5.40150701102332	0.858403148096329\\
5.40338412614166	0.858400682373685\\
5.40526123964133	0.858398218639506\\
5.40713835152359	0.858395756891629\\
5.40901546178967	0.858393297127895\\
5.41089257044081	0.858390839346148\\
5.41276967747825	0.858388383544234\\
5.41464678290325	0.858385929720003\\
5.41652388671703	0.858383477871307\\
5.41840098892082	0.858381027996001\\
5.42027808951588	0.858378580091945\\
5.42215518850343	0.858376134156999\\
5.4240322858847	0.858373690189028\\
5.42590938166093	0.858371248185898\\
5.42778647583335	0.85836880814548\\
5.42966356840318	0.858366370065648\\
5.43154065937166	0.858363933944276\\
5.43341774874002	0.858361499779244\\
5.43529483650947	0.858359067568433\\
5.43717192268125	0.858356637309728\\
5.43904900725657	0.858354209001017\\
5.44092609023667	0.85835178264019\\
5.44280317162275	0.858349358225141\\
5.44468025141605	0.858346935753765\\
5.44655732961778	0.858344515223961\\
5.44843440622915	0.858342096633632\\
5.45031148125138	0.858339679980682\\
5.45218855468569	0.858337265263019\\
5.45406562653329	0.858334852478554\\
5.45594269679539	0.858332441625199\\
5.45781976547321	0.858330032700872\\
5.45969683256795	0.85832762570349\\
5.46157389808083	0.858325220630977\\
5.46345096201304	0.858322817481256\\
5.4653280243658	0.858320416252255\\
5.46720508514031	0.858318016941905\\
5.46908214433778	0.858315619548139\\
5.47095920195941	0.858313224068892\\
5.47283625800639	0.858310830502104\\
5.47471331247993	0.858308438845716\\
5.47659036538123	0.858306049097672\\
5.47846741671149	0.858303661255921\\
5.4803444664719	0.858301275318411\\
5.48222151466366	0.858298891283095\\
5.48409856128796	0.85829650914793\\
5.48597560634599	0.858294128910873\\
5.48785264983895	0.858291750569886\\
5.48972969176803	0.858289374122931\\
5.49160673213441	0.858286999567977\\
5.49348377093929	0.858284626902993\\
5.49536080818385	0.85828225612595\\
5.49723784386928	0.858279887234823\\
5.49911487799676	0.85827752022759\\
5.50099191056748	0.858275155102232\\
5.50286894158261	0.858272791856731\\
5.50474597104334	0.858270430489074\\
5.50662299895085	0.858268070997249\\
5.50850002530631	0.858265713379247\\
5.51037705011091	0.858263357633062\\
5.51225407336582	0.858261003756691\\
5.51413109507222	0.858258651748134\\
5.51600811523128	0.858256301605392\\
5.51788513384416	0.858253953326471\\
5.51976215091205	0.858251606909378\\
5.52163916643612	0.858249262352122\\
5.52351618041753	0.858246919652718\\
5.52539319285745	0.858244578809181\\
5.52727020375705	0.858242239819528\\
5.52914721311749	0.858239902681781\\
5.53102422093994	0.858237567393964\\
5.53290122722556	0.858235233954102\\
5.53477823197552	0.858232902360225\\
5.53665523519097	0.858230572610365\\
5.53853223687307	0.858228244702554\\
5.54040923702299	0.858225918634832\\
5.54228623564188	0.858223594405236\\
5.5441632327309	0.858221272011809\\
5.5460402282912	0.858218951452596\\
5.54791722232393	0.858216632725645\\
5.54979421483026	0.858214315829004\\
5.55167120581133	0.858212000760728\\
5.55354819526828	0.858209687518872\\
5.55542518320229	0.858207376101492\\
5.55730216961448	0.85820506650665\\
5.559179154506	0.85820275873241\\
5.56105613787801	0.858200452776835\\
5.56293311973165	0.858198148637996\\
5.56481010006806	0.858195846313963\\
5.56668707888838	0.858193545802809\\
5.56856405619376	0.858191247102611\\
5.57044103198534	0.858188950211446\\
5.57231800626425	0.858186655127397\\
5.57419497903163	0.858184361848547\\
5.57607195028862	0.858182070372983\\
5.57794892003636	0.858179780698793\\
5.57982588827598	0.858177492824069\\
5.58170285500862	0.858175206746906\\
5.5835798202354	0.858172922465398\\
5.58545678395746	0.858170639977647\\
5.58733374617592	0.858168359281753\\
5.58921070689193	0.85816608037582\\
5.5910876661066	0.858163803257956\\
5.59296462382106	0.85816152792627\\
5.59484158003643	0.858159254378873\\
5.59671853475385	0.85815698261388\\
5.59859548797444	0.858154712629407\\
5.60047243969931	0.858152444423575\\
5.60234938992959	0.858150177994505\\
5.6042263386664	0.858147913340321\\
5.60610328591086	0.85814565045915\\
5.60798023166409	0.858143389349121\\
5.60985717592719	0.858141130008367\\
5.6117341187013	0.858138872435022\\
5.61361105998752	0.858136616627222\\
5.61548799978696	0.858134362583107\\
5.61736493810074	0.858132110300819\\
5.61924187492998	0.858129859778502\\
5.62111881027577	0.858127611014303\\
5.62299574413923	0.85812536400637\\
5.62487267652147	0.858123118752856\\
5.62674960742359	0.858120875251915\\
5.62862653684671	0.858118633501703\\
5.63050346479191	0.858116393500379\\
5.63238039126032	0.858114155246105\\
5.63425731625302	0.858111918737045\\
5.63613423977113	0.858109683971364\\
5.63801116181574	0.858107450947233\\
5.63988808238794	0.858105219662821\\
5.64176500148885	0.858102990116303\\
5.64364191911956	0.858100762305854\\
5.64551883528116	0.858098536229654\\
5.64739574997474	0.858096311885883\\
5.6492726632014	0.858094089272723\\
5.65114957496224	0.858091868388362\\
5.65302648525835	0.858089649230987\\
5.65490339409081	0.858087431798789\\
5.65678030146071	0.85808521608996\\
5.65865720736915	0.858083002102696\\
5.66053411181721	0.858080789835194\\
5.66241101480598	0.858078579285655\\
5.66428791633654	0.85807637045228\\
5.66616481640998	0.858074163333276\\
5.66804171502738	0.858071957926848\\
5.66991861218981	0.858069754231207\\
5.67179550789838	0.858067552244564\\
5.67367240215414	0.858065351965134\\
5.67554929495818	0.858063153391132\\
5.67742618631159	0.85806095652078\\
5.67930307621543	0.858058761352296\\
5.68117996467078	0.858056567883906\\
5.68305685167872	0.858054376113835\\
5.68493373724031	0.858052186040312\\
5.68681062135664	0.858049997661567\\
5.68868750402878	0.858047810975833\\
5.69056438525778	0.858045625981346\\
5.69244126504473	0.858043442676344\\
5.69431814339069	0.858041261059065\\
5.69619502029673	0.858039081127753\\
5.69807189576391	0.858036902880652\\
5.69994876979331	0.85803472631601\\
5.70182564238597	0.858032551432074\\
5.70370251354298	0.858030378227098\\
5.70557938326538	0.858028206699334\\
5.70745625155424	0.858026036847039\\
5.70933311841063	0.858023868668471\\
5.71120998383559	0.85802170216189\\
5.71308684783019	0.858019537325561\\
5.71496371039548	0.858017374157748\\
5.71684057153252	0.858015212656718\\
5.71871743124236	0.858013052820742\\
5.72059428952606	0.858010894648091\\
5.72247114638467	0.85800873813704\\
5.72434800181924	0.858006583285865\\
5.72622485583083	0.858004430092845\\
5.72810170842048	0.858002278556262\\
5.72997855958923	0.858000128674398\\
5.73185540933815	0.85799798044554\\
5.73373225766827	0.857995833867974\\
5.73560910458064	0.857993688939992\\
5.7374859500763	0.857991545659885\\
5.73936279415631	0.857989404025948\\
5.74123963682169	0.857987264036478\\
5.74311647807349	0.857985125689773\\
5.74499331791276	0.857982988984136\\
5.74687015634053	0.857980853917869\\
5.74874699335783	0.857978720489277\\
5.75062382896572	0.85797658869667\\
5.75250066316522	0.857974458538357\\
5.75437749595737	0.85797233001265\\
5.7562543273432	0.857970203117864\\
5.75813115732374	0.857968077852315\\
5.76000798590004	0.857965954214322\\
5.76188481307312	0.857963832202207\\
5.76376163884401	0.857961711814292\\
5.76563846321373	0.857959593048903\\
5.76751528618333	0.857957475904368\\
5.76939210775382	0.857955360379016\\
5.77126892792623	0.857953246471179\\
5.77314574670159	0.857951134179191\\
5.77502256408092	0.85794902350139\\
5.77689938006525	0.857946914436112\\
5.77877619465559	0.857944806981699\\
5.78065300785297	0.857942701136494\\
5.78252981965841	0.857940596898841\\
5.78440663007293	0.857938494267087\\
5.78628343909754	0.857936393239582\\
5.78816024673327	0.857934293814677\\
5.79003705298113	0.857932195990726\\
5.79191385784213	0.857930099766084\\
5.79379066131729	0.857928005139109\\
5.79566746340763	0.85792591210816\\
5.79754426411415	0.8579238206716\\
5.79942106343787	0.857921730827794\\
5.8012978613798	0.857919642575106\\
5.80317465794094	0.857917555911906\\
5.80505145312232	0.857915470836564\\
5.80692824692492	0.857913387347453\\
5.80880503934976	0.857911305442947\\
5.81068183039785	0.857909225121423\\
5.81255862007019	0.85790714638126\\
5.81443540836779	0.857905069220839\\
5.81631219529164	0.857902993638543\\
5.81818898084275	0.857900919632758\\
5.82006576502211	0.85789884720187\\
5.82194254783074	0.857896776344269\\
5.82381932926963	0.857894707058346\\
5.82569610933977	0.857892639342495\\
5.82757288804216	0.857890573195111\\
5.8294496653778	0.857888508614593\\
5.83132644134768	0.857886445599339\\
5.83320321595281	0.857884384147751\\
5.83507998919416	0.857882324258234\\
5.83695676107273	0.857880265929193\\
5.83883353158952	0.857878209159036\\
5.84071030074551	0.857876153946174\\
5.8425870685417	0.857874100289017\\
5.84446383497906	0.857872048185981\\
5.8463406000586	0.857869997635481\\
5.84821736378129	0.857867948635935\\
5.85009412614812	0.857865901185764\\
5.85197088716008	0.85786385528339\\
5.85384764681814	0.857861810927238\\
5.8557244051233	0.857859768115732\\
5.85760116207653	0.857857726847303\\
5.85947791767881	0.857855687120379\\
5.86135467193112	0.857853648933394\\
5.86323142483445	0.857851612284782\\
5.86510817638977	0.857849577172979\\
5.86698492659805	0.857847543596423\\
5.86886167546028	0.857845511553555\\
5.87073842297742	0.857843481042817\\
5.87261516915045	0.857841452062654\\
5.87449191398036	0.857839424611512\\
5.87636865746809	0.857837398687839\\
5.87824539961464	0.857835374290085\\
5.88012214042097	0.857833351416704\\
5.88199887988804	0.857831330066148\\
5.88387561801683	0.857829310236875\\
5.88575235480831	0.857827291927343\\
5.88762909026344	0.857825275136011\\
5.88950582438318	0.857823259861343\\
5.89138255716851	0.857821246101802\\
5.89325928862038	0.857819233855854\\
5.89513601873976	0.857817223121968\\
5.89701274752761	0.857815213898613\\
5.89888947498489	0.857813206184262\\
5.90076620111257	0.857811199977388\\
5.90264292591159	0.857809195276468\\
5.90451964938293	0.857807192079978\\
5.90639637152753	0.8578051903864\\
5.90827309234636	0.857803190194214\\
5.91014981184036	0.857801191501904\\
5.9120265300105	0.857799194307957\\
5.91390324685773	0.857797198610858\\
5.915779962383	0.857795204409099\\
5.91765667658726	0.857793211701169\\
5.91953338947147	0.857791220485563\\
5.92141010103657	0.857789230760776\\
5.92328681128351	0.857787242525304\\
5.92516352021324	0.857785255777647\\
5.92704022782672	0.857783270516306\\
5.92891693412488	0.857781286739784\\
5.93079363910867	0.857779304446585\\
5.93267034277903	0.857777323635215\\
5.93454704513692	0.857775344304185\\
5.93642374618326	0.857773366452003\\
5.93830044591902	0.857771390077182\\
5.94017714434511	0.857769415178237\\
5.9420538414625	0.857767441753684\\
5.94393053727211	0.85776546980204\\
5.94580723177488	0.857763499321825\\
5.94768392497176	0.857761530311562\\
5.94956061686367	0.857759562769773\\
5.95143730745156	0.857757596694985\\
5.95331399673636	0.857755632085724\\
5.95519068471899	0.85775366894052\\
5.95706737140041	0.857751707257904\\
5.95894405678153	0.857749747036409\\
5.96082074086329	0.85774778827457\\
5.96269742364663	0.857745830970923\\
5.96457410513246	0.857743875124007\\
5.96645078532172	0.857741920732362\\
5.96832746421533	0.85773996779453\\
5.97020414181423	0.857738016309056\\
5.97208081811934	0.857736066274486\\
5.97395749313158	0.857734117689366\\
5.97583416685188	0.857732170552246\\
5.97771083928116	0.857730224861679\\
5.97958751042035	0.857728280616217\\
5.98146418027036	0.857726337814415\\
5.98334084883212	0.857724396454831\\
5.98521751610655	0.857722456536022\\
5.98709418209456	0.857720518056549\\
5.98897084679707	0.857718581014975\\
5.99084751021501	0.857716645409863\\
5.99272417234929	0.857714711239781\\
5.99460083320081	0.857712778503294\\
5.99647749277051	0.857710847198974\\
5.99835415105929	0.857708917325391\\
6.00023080806806	0.857706988881118\\
6.00210746379774	0.857705061864731\\
6.00398411824924	0.857703136274805\\
6.00586077142347	0.857701212109921\\
6.00773742332133	0.857699289368657\\
6.00961407394374	0.857697368049596\\
6.01149072329161	0.857695448151322\\
6.01336737136584	0.857693529672421\\
6.01524401816733	0.85769161261148\\
6.017120663697	0.857689696967088\\
6.01899730795575	0.857687782737836\\
6.02087395094448	0.857685869922317\\
6.02275059266409	0.857683958519126\\
6.02462723311548	0.857682048526858\\
6.02650387229957	0.857680139944112\\
6.02838051021723	0.857678232769488\\
6.03025714686939	0.857676327001586\\
6.03213378225693	0.857674422639011\\
6.03401041638075	0.857672519680368\\
6.03588704924175	0.857670618124262\\
6.03776368084082	0.857668717969304\\
6.03964031117887	0.857666819214102\\
6.04151694025677	0.85766492185727\\
6.04339356807544	0.85766302589742\\
6.04527019463575	0.857661131333169\\
6.04714681993861	0.857659238163133\\
6.0490234439849	0.857657346385932\\
6.05090006677551	0.857655456000186\\
6.05277668831133	0.857653567004517\\
6.05465330859325	0.85765167939755\\
6.05652992762216	0.857649793177911\\
6.05840654539895	0.857647908344227\\
6.06028316192449	0.857646024895127\\
6.06215977719968	0.857644142829242\\
6.0640363912254	0.857642262145206\\
6.06591300400253	0.857640382841651\\
6.06778961553195	0.857638504917216\\
6.06966622581456	0.857636628370536\\
6.07154283485122	0.857634753200252\\
6.07341944264282	0.857632879405006\\
6.07529604919023	0.857631006983439\\
6.07717265449434	0.857629135934196\\
6.07904925855602	0.857627266255924\\
6.08092586137616	0.857625397947271\\
6.08280246295562	0.857623531006886\\
6.08467906329528	0.85762166543342\\
6.08655566239601	0.857619801225527\\
6.0884322602587	0.857617938381861\\
6.0903088568842	0.857616076901079\\
6.0921854522734	0.857614216781838\\
6.09406204642716	0.857612358022798\\
6.09593863934636	0.85761050062262\\
6.09781523103186	0.857608644579968\\
6.09969182148453	0.857606789893506\\
6.10156841070524	0.857604936561901\\
6.10344499869486	0.857603084583819\\
6.10532158545426	0.857601233957932\\
6.10719817098429	0.85759938468291\\
6.10907475528582	0.857597536757426\\
6.11095133835973	0.857595690180154\\
6.11282792020686	0.857593844949772\\
6.11470450082808	0.857592001064956\\
6.11658108022426	0.857590158524386\\
6.11845765839625	0.857588317326744\\
6.12033423534492	0.857586477470711\\
6.12221081107112	0.857584638954973\\
6.1240873855757	0.857582801778216\\
6.12596395885954	0.857580965939126\\
6.12784053092348	0.857579131436393\\
6.12971710176838	0.857577298268709\\
6.1315936713951	0.857575466434765\\
6.13347023980449	0.857573635933255\\
6.1353468069974	0.857571806762876\\
6.13722337297468	0.857569978922325\\
6.13909993773719	0.8575681524103\\
6.14097650128578	0.857566327225503\\
6.1428530636213	0.857564503366635\\
6.14472962474459	0.857562680832399\\
6.14660618465651	0.857560859621502\\
6.14848274335791	0.857559039732651\\
6.15035930084962	0.857557221164553\\
6.1522358571325	0.857555403915919\\
6.1541124122074	0.85755358798546\\
6.15598896607515	0.857551773371891\\
6.15786551873661	0.857549960073925\\
6.1597420701926	0.857548148090279\\
6.16161862044399	0.857546337419672\\
6.1634951694916	0.857544528060823\\
6.16537171733628	0.857542720012452\\
6.16724826397887	0.857540913273284\\
6.16912480942021	0.857539107842042\\
6.17100135366113	0.857537303717451\\
6.17287789670248	0.85753550089824\\
6.17475443854509	0.857533699383138\\
6.1766309791898	0.857531899170875\\
6.17850751863744	0.857530100260183\\
6.18038405688884	0.857528302649796\\
6.18226059394485	0.85752650633845\\
6.18413712980629	0.85752471132488\\
6.18601366447399	0.857522917607826\\
6.1878901979488	0.857521125186027\\
6.18976673023153	0.857519334058225\\
6.19164326132301	0.857517544223162\\
6.19351979122409	0.857515755679584\\
6.19539631993558	0.857513968426236\\
6.19727284745832	0.857512182461866\\
6.19914937379312	0.857510397785223\\
6.20102589894082	0.857508614395058\\
6.20290242290224	0.857506832290123\\
6.20477894567821	0.857505051469171\\
6.20665546726955	0.857503271930959\\
6.20853198767708	0.857501493674242\\
6.21040850690162	0.857499716697779\\
6.212285024944	0.85749794100033\\
6.21416154180504	0.857496166580655\\
6.21603805748556	0.857494393437519\\
6.21791457198637	0.857492621569685\\
6.2197910853083	0.857490850975919\\
6.22166759745216	0.857489081654989\\
6.22354410841877	0.857487313605663\\
6.22542061820895	0.857485546826712\\
6.22729712682351	0.857483781316907\\
6.22917363426326	0.857482017075022\\
6.23105014052903	0.857480254099832\\
6.23292664562162	0.857478492390113\\
6.23480314954184	0.857476731944643\\
6.23667965229052	0.857474972762202\\
6.23855615386845	0.857473214841569\\
6.24043265427645	0.857471458181528\\
6.24230915351533	0.857469702780862\\
6.2441856515859	0.857467948638357\\
6.24606214848896	0.857466195752798\\
6.24793864422533	0.857464444122975\\
6.24981513879581	0.857462693747678\\
6.2516916322012	0.857460944625696\\
6.25356812444231	0.857459196755823\\
6.25544461551995	0.857457450136854\\
6.25732110543491	0.857455704767583\\
6.259197594188	0.857453960646807\\
6.26107408178003	0.857452217773325\\
6.2629505682118	0.857450476145938\\
6.26482705348409	0.857448735763446\\
6.26670353759773	0.857446996624652\\
6.2685800205535	0.85744525872836\\
6.2704565023522	0.857443522073377\\
6.27233298299463	0.857441786658509\\
6.27420946248159	0.857440052482565\\
6.27608594081387	0.857438319544356\\
6.27796241799227	0.857436587842692\\
6.27983889401759	0.857434857376387\\
6.28171536889061	0.857433128144255\\
6.28359184261214	0.857431400145112\\
6.28546831518296	0.857429673377775\\
6.28734478660386	0.857427947841063\\
6.28922125687565	0.857426223533796\\
6.2910977259991	0.857424500454795\\
6.29297419397501	0.857422778602884\\
6.29485066080416	0.857421057976887\\
6.29672712648735	0.85741933857563\\
6.29860359102537	0.85741762039794\\
6.30048005441899	0.857415903442645\\
6.302356516669	0.857414187708577\\
6.3042329777762	0.857412473194565\\
6.30610943774137	0.857410759899444\\
6.30798589656528	0.857409047822047\\
6.30986235424873	0.85740733696121\\
6.31173881079249	0.85740562731577\\
6.31361526619735	0.857403918884566\\
6.31549172046408	0.857402211666438\\
6.31736817359348	0.857400505660227\\
6.31924462558631	0.857398800864775\\
6.32112107644336	0.857397097278927\\
6.32299752616541	0.857395394901529\\
6.32487397475323	0.857393693731426\\
6.3267504222076	0.857391993767468\\
6.3286268685293	0.857390295008504\\
6.3305033137191	0.857388597453385\\
6.33237975777778	0.857386901100964\\
6.3342562007061	0.857385205950093\\
6.33613264250485	0.85738351199963\\
6.3380090831748	0.857381819248429\\
6.33988552271672	0.857380127695349\\
6.34176196113138	0.857378437339249\\
6.34363839841954	0.857376748178991\\
6.34551483458199	0.857375060213435\\
6.34739126961949	0.857373373441445\\
6.3492677035328	0.857371687861887\\
6.3511441363227	0.857370003473625\\
6.35302056798996	0.857368320275528\\
6.35489699853533	0.857366638266465\\
6.35677342795959	0.857364957445305\\
6.3586498562635	0.857363277810921\\
6.36052628344782	0.857361599362184\\
6.36240270951332	0.857359922097971\\
6.36427913446076	0.857358246017154\\
6.3661555582909	0.857356571118613\\
6.36803198100451	0.857354897401225\\
6.36990840260234	0.85735322486387\\
6.37178482308516	0.857351553505428\\
6.37366124245372	0.857349883324783\\
6.37553766070879	0.857348214320817\\
6.37741407785111	0.857346546492416\\
6.37929049388146	0.857344879838466\\
6.38116690880058	0.857343214357854\\
6.38304332260924	0.85734155004947\\
6.38491973530818	0.857339886912203\\
6.38679614689816	0.857338224944946\\
6.38867255737994	0.857336564146592\\
6.39054896675427	0.857334904516034\\
6.3924253750219	0.857333246052168\\
6.39430178218358	0.857331588753892\\
6.39617818824007	0.857329932620102\\
6.39805459319212	0.8573282776497\\
6.39993099704047	0.857326623841585\\
6.40180739978587	0.85732497119466\\
6.40368380142908	0.857323319707828\\
6.40556020197084	0.857321669379994\\
6.40743660141189	0.857320020210064\\
6.40931299975299	0.857318372196946\\
6.41118939699489	0.857316725339547\\
6.41306579313831	0.857315079636779\\
6.41494218818402	0.857313435087552\\
6.41681858213275	0.857311791690778\\
6.41869497498525	0.857310149445373\\
6.42057136674226	0.85730850835025\\
6.42244775740453	0.857306868404326\\
6.42432414697278	0.857305229606519\\
6.42620053544777	0.857303591955747\\
6.42807692283023	0.857301955450932\\
6.4299533091209	0.857300320090995\\
6.43182969432053	0.857298685874858\\
6.43370607842984	0.857297052801446\\
6.43558246144958	0.857295420869684\\
6.43745884338048	0.857293790078498\\
6.43933522422329	0.857292160426817\\
6.44121160397872	0.85729053191357\\
6.44308798264752	0.857288904537687\\
6.44496436023043	0.8572872782981\\
6.44684073672817	0.857285653193742\\
6.44871711214147	0.857284029223547\\
6.45059348647108	0.857282406386451\\
6.45246985971772	0.857280784681391\\
6.45434623188212	0.857279164107304\\
6.45622260296501	0.85727754466313\\
6.45809897296712	0.857275926347809\\
6.45997534188919	0.857274309160284\\
6.46185170973193	0.857272693099497\\
6.46372807649607	0.857271078164393\\
6.46560444218235	0.857269464353916\\
6.46748080679148	0.857267851667015\\
6.4693571703242	0.857266240102637\\
6.47123353278122	0.857264629659731\\
6.47310989416328	0.857263020337248\\
6.47498625447109	0.857261412134139\\
6.47686261370538	0.857259805049358\\
6.47873897186687	0.857258199081858\\
6.48061532895628	0.857256594230596\\
6.48249168497433	0.857254990494528\\
6.48436803992175	0.857253387872611\\
6.48624439379925	0.857251786363806\\
6.48812074660755	0.857250185967071\\
6.48999709834737	0.85724858668137\\
6.49187344901942	0.857246988505665\\
6.49374979862443	0.857245391438919\\
6.49562614716311	0.857243795480098\\
6.49750249463618	0.857242200628169\\
6.49937884104435	0.857240606882099\\
6.50125518638834	0.857239014240857\\
6.50313153066885	0.857237422703413\\
6.50500787388661	0.857235832268739\\
6.50688421604232	0.857234242935807\\
6.5087605571367	0.85723265470359\\
6.51063689717046	0.857231067571065\\
6.5125132361443	0.857229481537206\\
6.51438957405895	0.857227896600992\\
6.5162659109151	0.8572263127614\\
6.51814224671347	0.857224730017411\\
6.52001858145477	0.857223148368006\\
6.5218949151397	0.857221567812166\\
6.52377124776897	0.857219988348875\\
6.52564757934329	0.857218409977117\\
6.52752390986336	0.857216832695879\\
6.52940023932988	0.857215256504147\\
6.53127656774356	0.857213681400909\\
6.53315289510511	0.857212107385154\\
6.53502922141523	0.857210534455873\\
6.53690554667462	0.857208962612058\\
6.53878187088398	0.857207391852702\\
6.54065819404401	0.857205822176797\\
6.54253451615542	0.85720425358334\\
6.5444108372189	0.857202686071327\\
6.54628715723515	0.857201119639755\\
6.54816347620487	0.857199554287623\\
6.55003979412877	0.857197990013931\\
6.55191611100753	0.85719642681768\\
6.55379242684186	0.857194864697873\\
6.55566874163245	0.857193303653511\\
6.55754505537999	0.857191743683601\\
6.55942136808518	0.857190184787147\\
6.56129767974873	0.857188626963157\\
6.56317399037131	0.857187070210638\\
6.56505029995362	0.8571855145286\\
6.56692660849636	0.857183959916052\\
6.56880291600022	0.857182406372007\\
6.57067922246588	0.857180853895478\\
6.57255552789404	0.857179302485477\\
6.5744318322854	0.857177752141019\\
6.57630813564063	0.857176202861122\\
6.57818443796043	0.857174654644801\\
6.58006073924549	0.857173107491076\\
6.58193703949649	0.857171561398965\\
6.58381333871412	0.85717001636749\\
6.58568963689907	0.857168472395672\\
6.58756593405202	0.857166929482534\\
6.58944223017367	0.8571653876271\\
6.59131852526468	0.857163846828395\\
6.59319481932575	0.857162307085446\\
6.59507111235757	0.85716076839728\\
6.5969474043608	0.857159230762926\\
6.59882369533615	0.857157694181413\\
6.60069998528428	0.857156158651771\\
6.60257627420588	0.857154624173034\\
6.60445256210162	0.857153090744234\\
6.6063288489722	0.857151558364405\\
6.60820513481829	0.857150027032583\\
6.61008141964056	0.857148496747804\\
6.6119577034397	0.857146967509105\\
6.61383398621638	0.857145439315526\\
6.61571026797128	0.857143912166105\\
6.61758654870507	0.857142386059885\\
6.61946282841844	0.857140860995906\\
6.62133910711205	0.857139336973212\\
6.62321538478658	0.857137813990847\\
6.62509166144271	0.857136292047856\\
6.6269679370811	0.857134771143287\\
6.62884421170244	0.857133251276185\\
6.63072048530739	0.857131732445601\\
6.63259675789662	0.857130214650583\\
6.63447302947081	0.857128697890183\\
6.63634930003062	0.857127182163453\\
6.63822556957673	0.857125667469444\\
6.6401018381098	0.857124153807213\\
6.64197810563051	0.857122641175812\\
6.64385437213952	0.8571211295743\\
6.6457306376375	0.857119619001734\\
6.64760690212511	0.857118109457171\\
6.64948316560302	0.857116600939671\\
6.6513594280719	0.857115093448296\\
6.65323568953241	0.857113586982106\\
6.65511194998522	0.857112081540164\\
6.656988209431	0.857110577121535\\
6.65886446787039	0.857109073725283\\
6.66074072530407	0.857107571350475\\
6.6626169817327	0.857106069996177\\
6.66449323715694	0.857104569661457\\
6.66636949157746	0.857103070345385\\
6.6682457449949	0.857101572047032\\
6.67012199740994	0.857100074765468\\
6.67199824882322	0.857098578499765\\
6.67387449923542	0.857097083248999\\
6.67575074864719	0.857095589012242\\
6.67762699705918	0.857094095788571\\
6.67950324447205	0.857092603577063\\
6.68137949088647	0.857091112376795\\
6.68325573630308	0.857089622186846\\
6.68513198072253	0.857088133006295\\
6.6870082241455	0.857086644834225\\
6.68888446657262	0.857085157669716\\
6.69076070800455	0.857083671511852\\
6.69263694844195	0.857082186359718\\
6.69451318788547	0.857080702212397\\
6.69638942633575	0.857079219068977\\
6.69826566379346	0.857077736928544\\
6.70014190025924	0.857076255790187\\
6.70201813573374	0.857074775652995\\
6.70389437021761	0.857073296516059\\
6.70577060371151	0.85707181837847\\
6.70764683621607	0.85707034123932\\
6.70952306773194	0.857068865097703\\
6.71139929825979	0.857067389952714\\
6.71327552780024	0.857065915803447\\
6.71515175635395	0.857064442649\\
6.71702798392156	0.857062970488471\\
6.71890421050372	0.857061499320957\\
6.72078043610107	0.857060029145559\\
6.72265666071426	0.857058559961378\\
6.72453288434392	0.857057091767514\\
6.72640910699071	0.857055624563072\\
6.72828532865526	0.857054158347154\\
6.73016154933822	0.857052693118865\\
6.73203776904023	0.857051228877312\\
6.73391398776192	0.857049765621601\\
6.73579020550395	0.85704830335084\\
6.73766642226694	0.857046842064138\\
6.73954263805153	0.857045381760605\\
6.74141885285837	0.857043922439352\\
6.7432950666881	0.857042464099491\\
6.74517127954134	0.857041006740134\\
6.74704749141874	0.857039550360396\\
6.74892370232093	0.857038094959392\\
6.75079991224855	0.857036640536237\\
6.75267612120223	0.857035187090049\\
6.75455232918261	0.857033734619945\\
6.75642853619033	0.857032283125046\\
6.758304742226	0.857030832604469\\
6.76018094729028	0.857029383057338\\
6.76205715138379	0.857027934482773\\
6.76393335450715	0.857026486879897\\
6.76580955666102	0.857025040247835\\
6.767685757846	0.857023594585712\\
6.76956195806274	0.857022149892653\\
6.77143815731186	0.857020706167786\\
6.773314355594	0.857019263410238\\
6.77519055290978	0.857017821619139\\
6.77706674925982	0.857016380793617\\
6.77894294464477	0.857014940932806\\
6.78081913906523	0.857013502035835\\
6.78269533252185	0.857012064101839\\
6.78457152501524	0.85701062712995\\
6.78644771654603	0.857009191119304\\
6.78832390711485	0.857007756069037\\
6.79020009672232	0.857006321978285\\
6.79207628536906	0.857004888846187\\
6.7939524730557	0.85700345667188\\
6.79582865978285	0.857002025454506\\
6.79770484555115	0.857000595193204\\
6.79958103036121	0.856999165887116\\
6.80145721421365	0.856997737535385\\
6.8033333971091	0.856996310137154\\
6.80520957904817	0.856994883691569\\
6.80708576003148	0.856993458197774\\
6.80896194005966	0.856992033654916\\
6.81083811913331	0.856990610062143\\
6.81271429725307	0.856989187418603\\
6.81459047441954	0.856987765723445\\
6.81646665063334	0.856986344975821\\
6.81834282589508	0.856984925174881\\
6.82021900020539	0.856983506319777\\
6.82209517356488	0.856982088409663\\
6.82397134597417	0.856980671443693\\
6.82584751743386	0.856979255421022\\
6.82772368794457	0.856977840340806\\
6.82959985750691	0.856976426202203\\
6.8314760261215	0.85697501300437\\
6.83335219378895	0.856973600746467\\
6.83522836050987	0.856972189427652\\
6.83710452628487	0.856970779047088\\
6.83898069111456	0.856969369603936\\
6.84085685499954	0.856967961097358\\
6.84273301794044	0.856966553526518\\
6.84460917993786	0.856965146890581\\
6.84648534099241	0.856963741188713\\
6.84836150110469	0.85696233642008\\
6.85023766027531	0.856960932583849\\
6.85211381850488	0.856959529679189\\
6.853989975794	0.856958127705269\\
6.85586613214329	0.856956726661259\\
6.85774228755334	0.856955326546331\\
6.85961844202476	0.856953927359656\\
6.86149459555815	0.856952529100408\\
6.86337074815412	0.856951131767761\\
6.86524689981328	0.856949735360889\\
6.86712305053621	0.856948339878968\\
6.86899920032354	0.856946945321175\\
6.87087534917584	0.856945551686688\\
6.87275149709374	0.856944158974686\\
6.87462764407783	0.856942767184347\\
6.8765037901287	0.856941376314852\\
6.87837993524697	0.856939986365383\\
6.88025607943321	0.856938597335121\\
6.88213222268805	0.856937209223251\\
6.88400836501207	0.856935822028955\\
6.88588450640587	0.85693443575142\\
6.88776064687004	0.85693305038983\\
6.88963678640519	0.856931665943374\\
6.89151292501192	0.856930282411237\\
6.8933890626908	0.85692889979261\\
6.89526519944245	0.856927518086682\\
6.89714133526745	0.856926137292642\\
6.8990174701664	0.856924757409683\\
6.90089360413989	0.856923378436997\\
6.90276973718852	0.856922000373777\\
6.90464586931287	0.856920623219217\\
6.90652200051354	0.856919246972511\\
6.90839813079113	0.856917871632857\\
6.91027426014621	0.85691649719945\\
6.91215038857939	0.856915123671488\\
6.91402651609124	0.856913751048169\\
6.91590264268237	0.856912379328694\\
6.91777876835336	0.856911008512263\\
6.91965489310479	0.856909638598076\\
6.92153101693726	0.856908269585336\\
6.92340713985136	0.856906901473246\\
6.92528326184766	0.856905534261009\\
6.92715938292676	0.856904167947831\\
6.92903550308924	0.856902802532917\\
6.93091162233569	0.856901438015474\\
6.93278774066669	0.856900074394709\\
6.93466385808283	0.85689871166983\\
6.93653997458469	0.856897349840047\\
6.93841609017285	0.85689598890457\\
6.9402922048479	0.856894628862609\\
6.94216831861042	0.856893269713376\\
6.94404443146099	0.856891911456085\\
6.94592054340019	0.856890554089949\\
6.9477966544286	0.856889197614181\\
6.94967276454681	0.856887842027998\\
6.95154887375539	0.856886487330616\\
6.95342498205493	0.856885133521252\\
6.95530108944599	0.856883780599123\\
6.95717719592917	0.856882428563448\\
6.95905330150503	0.856881077413448\\
6.96092940617416	0.856879727148342\\
6.96280550993713	0.856878377767353\\
6.96468161279453	0.856877029269701\\
6.96655771474691	0.856875681654611\\
6.96843381579487	0.856874334921306\\
6.97030991593897	0.856872989069012\\
6.97218601517979	0.856871644096953\\
6.97406211351791	0.856870300004357\\
6.97593821095389	0.856868956790451\\
6.97781430748832	0.856867614454462\\
6.97969040312176	0.856866272995621\\
6.98156649785478	0.856864932413157\\
6.98344259168796	0.856863592706301\\
6.98531868462187	0.856862253874285\\
6.98719477665708	0.856860915916341\\
6.98907086779415	0.856859578831702\\
6.99094695803367	0.856858242619603\\
6.99282304737619	0.856856907279279\\
6.99469913582229	0.856855572809965\\
6.99657522337254	0.856854239210899\\
6.99845131002749	0.856852906481318\\
7.00032739578773	0.856851574620461\\
7.00220348065381	0.856850243627566\\
7.00407956462631	0.856848913501874\\
7.00595564770579	0.856847584242625\\
7.00783172989281	0.856846255849063\\
7.00970781118794	0.856844928320428\\
7.01158389159174	0.856843601655965\\
7.01345997110478	0.856842275854918\\
7.01533604972762	0.856840950916532\\
7.01721212746083	0.856839626840052\\
7.01908820430496	0.856838303624727\\
7.02096428026058	0.856836981269803\\
7.02284035532825	0.856835659774529\\
7.02471642950854	0.856834339138153\\
7.02659250280199	0.856833019359927\\
7.02846857520918	0.856831700439101\\
7.03034464673066	0.856830382374926\\
7.03222071736699	0.856829065166656\\
7.03409678711873	0.856827748813544\\
7.03597285598644	0.856826433314843\\
7.03784892397068	0.856825118669809\\
7.03972499107199	0.856823804877698\\
7.04160105729095	0.856822491937767\\
7.04347712262811	0.856821179849272\\
7.04535318708402	0.856819868611472\\
7.04722925065923	0.856818558223627\\
7.04910531335431	0.856817248684995\\
7.05098137516981	0.856815939994839\\
7.05285743610628	0.856814632152419\\
7.05473349616428	0.856813325156998\\
7.05660955534435	0.856812019007838\\
7.05848561364706	0.856810713704204\\
7.06036167107295	0.856809409245361\\
7.06223772762257	0.856808105630574\\
7.06411378329648	0.85680680285911\\
7.06598983809523	0.856805500930235\\
7.06786589201937	0.856804199843218\\
7.06974194506944	0.856802899597327\\
7.071617997246	0.856801600191832\\
7.0734940485496	0.856800301626004\\
7.07537009898079	0.856799003899113\\
7.0772461485401	0.856797707010432\\
7.0791221972281	0.856796410959233\\
7.08099824504533	0.85679511574479\\
7.08287429199233	0.856793821366377\\
7.08475033806965	0.856792527823269\\
7.08662638327783	0.856791235114742\\
7.08850242761743	0.856789943240074\\
7.09037847108899	0.856788652198541\\
7.09225451369305	0.856787361989422\\
7.09413055543015	0.856786072611995\\
7.09600659630085	0.856784784065542\\
7.09788263630567	0.856783496349342\\
7.09975867544517	0.856782209462676\\
7.10163471371988	0.856780923404828\\
7.10351075113036	0.85677963817508\\
7.10538678767713	0.856778353772716\\
7.10726282336075	0.85677707019702\\
7.10913885818175	0.856775787447278\\
7.11101489214066	0.856774505522776\\
7.11289092523804	0.8567732244228\\
7.11476695747442	0.85677194414664\\
7.11664298885033	0.856770664693582\\
7.11851901936633	0.856769386062916\\
7.12039504902293	0.856768108253933\\
7.12227107782069	0.856766831265922\\
7.12414710576013	0.856765555098177\\
7.1260231328418	0.856764279749987\\
7.12789915906623	0.856763005220648\\
7.12977518443395	0.856761731509453\\
7.13165120894551	0.856760458615695\\
7.13352723260143	0.856759186538672\\
7.13540325540225	0.856757915277678\\
7.1372792773485	0.856756644832012\\
7.13915529844072	0.856755375200969\\
7.14103131867944	0.85675410638385\\
7.14290733806519	0.856752838379953\\
7.1447833565985	0.856751571188577\\
7.1466593742799	0.856750304809025\\
7.14853539110994	0.856749039240597\\
7.15041140708913	0.856747774482595\\
7.152287422218	0.856746510534322\\
7.15416343649709	0.856745247395083\\
7.15603944992693	0.856743985064182\\
7.15791546250804	0.856742723540923\\
7.15979147424096	0.856741462824613\\
7.1616674851262	0.856740202914559\\
7.1635434951643	0.856738943810068\\
7.16541950435579	0.856737685510449\\
7.16729551270119	0.85673642801501\\
7.16917152020103	0.856735171323061\\
7.17104752685583	0.856733915433912\\
7.17292353266612	0.856732660346876\\
7.17479953763242	0.856731406061263\\
7.17667554175527	0.856730152576387\\
7.17855154503517	0.856728899891561\\
7.18042754747266	0.856727648006099\\
7.18230354906826	0.856726396919316\\
7.1841795498225	0.856725146630528\\
7.18605554973589	0.856723897139051\\
7.18793154880895	0.856722648444203\\
7.18980754704221	0.8567214005453\\
7.19168354443619	0.856720153441663\\
7.19355954099141	0.85671890713261\\
7.19543553670839	0.856717661617462\\
7.19731153158765	0.856716416895538\\
7.1991875256297	0.856715172966161\\
7.20106351883507	0.856713929828654\\
7.20293951120428	0.856712687482338\\
7.20481550273785	0.856711445926538\\
7.20669149343628	0.856710205160579\\
7.2085674833001	0.856708965183785\\
7.21044347232983	0.856707725995482\\
7.21231946052598	0.856706487594997\\
7.21419544788906	0.856705249981658\\
7.2160714344196	0.856704013154792\\
7.21794742011811	0.856702777113729\\
7.2198234049851	0.856701541857797\\
7.22169938902109	0.856700307386328\\
7.22357537222658	0.856699073698652\\
7.2254513546021	0.8566978407941\\
7.22732733614816	0.856696608672006\\
7.22920331686526	0.856695377331702\\
7.23107929675393	0.856694146772523\\
7.23295527581467	0.856692916993802\\
7.234831254048	0.856691687994875\\
7.23670723145441	0.856690459775078\\
7.23858320803444	0.856689232333748\\
7.24045918378858	0.856688005670222\\
7.24233515871734	0.856686779783839\\
7.24421113282124	0.856685554673936\\
7.24608710610078	0.856684330339855\\
7.24796307855647	0.856683106780934\\
7.24983905018882	0.856681883996515\\
7.25171502099834	0.856680661985939\\
7.25359099098553	0.856679440748549\\
7.2554669601509	0.856678220283688\\
7.25734292849495	0.8566770005907\\
7.2592188960182	0.856675781668929\\
7.26109486272115	0.856674563517721\\
7.2629708286043	0.85667334613642\\
7.26484679366815	0.856672129524375\\
7.26672275791322	0.856670913680931\\
7.26859872134	0.856669698605437\\
7.270474683949	0.856668484297242\\
7.27235064574072	0.856667270755695\\
7.27422660671566	0.856666057980147\\
7.27610256687433	0.856664845969946\\
7.27797852621722	0.856663634724447\\
7.27985448474485	0.856662424242999\\
7.2817304424577	0.856661214524957\\
7.28360639935629	0.856660005569673\\
7.2854823554411	0.856658797376502\\
7.28735831071264	0.856657589944799\\
7.28923426517142	0.85665638327392\\
7.29111021881792	0.85665517736322\\
7.29298617165264	0.856653972212056\\
7.29486212367609	0.856652767819787\\
7.29673807488876	0.85665156418577\\
7.29861402529115	0.856650361309365\\
7.30048997488376	0.856649159189931\\
7.30236592366708	0.856647957826828\\
7.3042418716416	0.856646757219418\\
7.30611781880783	0.856645557367063\\
7.30799376516626	0.856644358269124\\
7.30986971071738	0.856643159924965\\
7.31174565546169	0.856641962333949\\
7.31362159939968	0.856640765495441\\
7.31549754253185	0.856639569408806\\
7.31737348485868	0.85663837407341\\
7.31924942638068	0.85663717948862\\
7.32112536709832	0.856635985653802\\
7.32300130701212	0.856634792568324\\
7.32487724612255	0.856633600231555\\
7.32675318443011	0.856632408642863\\
7.32862912193529	0.856631217801619\\
7.33050505863858	0.856630027707193\\
7.33238099454046	0.856628838358957\\
7.33425692964144	0.856627649756281\\
7.33613286394199	0.856626461898539\\
7.33800879744261	0.856625274785104\\
7.33988473014379	0.856624088415349\\
7.341760662046	0.856622902788649\\
7.34363659314975	0.856621717904379\\
7.34551252345552	0.856620533761915\\
7.34738845296379	0.856619350360633\\
7.34926438167505	0.856618167699911\\
7.35114030958979	0.856616985779125\\
7.35301623670849	0.856615804597656\\
7.35489216303164	0.85661462415488\\
7.35676808855972	0.85661344445018\\
7.35864401329322	0.856612265482934\\
7.36051993723261	0.856611087252524\\
7.3623958603784	0.856609909758331\\
7.36427178273105	0.856608732999738\\
7.36614770429105	0.856607556976129\\
7.36802362505888	0.856606381686886\\
7.36989954503503	0.856605207131393\\
7.37177546421998	0.856604033309037\\
7.3736513826142	0.856602860219202\\
7.37552730021818	0.856601687861274\\
7.3774032170324	0.856600516234642\\
7.37927913305735	0.856599345338691\\
7.38115504829349	0.856598175172811\\
7.38303096274131	0.85659700573639\\
7.38490687640128	0.856595837028817\\
7.38678278927389	0.856594669049484\\
7.38865870135962	0.856593501797779\\
7.39053461265894	0.856592335273096\\
7.39241052317233	0.856591169474826\\
7.39428643290026	0.856590004402362\\
7.39616234184322	0.856588840055097\\
7.39803825000168	0.856587676432424\\
7.39991415737611	0.85658651353374\\
7.40179006396699	0.856585351358438\\
7.4036659697748	0.856584189905915\\
7.40554187480001	0.856583029175567\\
7.40741777904309	0.856581869166791\\
7.40929368250452	0.856580709878986\\
7.41116958518478	0.85657955131155\\
7.41304548708433	0.85657839346388\\
7.41492138820364	0.856577236335379\\
7.4167972885432	0.856576079925444\\
7.41867318810347	0.856574924233479\\
7.42054908688493	0.856573769258883\\
7.42242498488804	0.85657261500106\\
7.42430088211328	0.856571461459412\\
7.42617677856111	0.856570308633343\\
7.42805267423201	0.856569156522256\\
7.42992856912645	0.856568005125556\\
7.4318044632449	0.856566854442649\\
7.43368035658782	0.856565704472941\\
7.43555624915568	0.856564555215838\\
7.43743214094896	0.856563406670747\\
7.43930803196811	0.856562258837076\\
7.44118392221362	0.856561111714234\\
7.44305981168594	0.856559965301629\\
7.44493570038553	0.856558819598672\\
7.44681158831288	0.856557674604772\\
7.44868747546844	0.856556530319341\\
7.45056336185268	0.856555386741789\\
7.45243924746607	0.85655424387153\\
7.45431513230906	0.856553101707975\\
7.45619101638213	0.856551960250539\\
7.45806689968574	0.856550819498635\\
7.45994278222035	0.856549679451677\\
};
\addplot [color=blue,solid,forget plot]
  table[row sep=crcr]{%
7.45994278222035	0.856549679451677\\
7.46181866398642	0.856548540109082\\
7.46369454498442	0.856547401470264\\
7.46557042521482	0.85654626353464\\
7.46744630467806	0.856545126301627\\
7.46932218337462	0.856543989770643\\
7.47119806130496	0.856542853941106\\
7.47307393846954	0.856541718812435\\
7.47494981486881	0.85654058438405\\
7.47682569050324	0.856539450655369\\
7.4787015653733	0.856538317625815\\
7.48057743947943	0.856537185294808\\
7.4824533128221	0.85653605366177\\
7.48432918540176	0.856534922726123\\
7.48620505721888	0.856533792487291\\
7.48808092827392	0.856532662944698\\
7.48995679856733	0.856531534097767\\
7.49183266809956	0.856530405945924\\
7.49370853687109	0.856529278488593\\
7.49558440488235	0.856528151725202\\
7.49746027213382	0.856527025655176\\
7.49933613862594	0.856525900277943\\
7.50121200435917	0.856524775592931\\
7.50308786933396	0.856523651599568\\
7.50496373355078	0.856522528297283\\
7.50683959701008	0.856521405685507\\
7.5087154597123	0.856520283763668\\
7.51059132165791	0.856519162531198\\
7.51246718284736	0.856518041987529\\
7.51434304328109	0.856516922132092\\
7.51621890295957	0.85651580296432\\
7.51809476188325	0.856514684483645\\
7.51997062005257	0.856513566689503\\
7.52184647746799	0.856512449581327\\
7.52372233412997	0.856511333158552\\
7.52559819003894	0.856510217420614\\
7.52747404519537	0.856509102366948\\
7.5293498995997	0.856507987996992\\
7.53122575325238	0.856506874310183\\
7.53310160615386	0.856505761305958\\
7.53497745830459	0.856504648983757\\
7.53685330970503	0.856503537343017\\
7.53872916035561	0.856502426383179\\
7.54060501025679	0.856501316103683\\
7.54248085940901	0.856500206503969\\
7.54435670781273	0.85649909758348\\
7.54623255546838	0.856497989341656\\
7.54810840237642	0.85649688177794\\
7.54998424853729	0.856495774891776\\
7.55186009395143	0.856494668682607\\
7.5537359386193	0.856493563149878\\
7.55561178254134	0.856492458293032\\
7.55748762571799	0.856491354111516\\
7.5593634681497	0.856490250604775\\
7.56123930983691	0.856489147772256\\
7.56311515078007	0.856488045613406\\
7.56499099097961	0.856486944127673\\
7.56686683043599	0.856485843314505\\
7.56874266914964	0.85648474317335\\
7.57061850712101	0.856483643703658\\
7.57249434435054	0.856482544904879\\
7.57437018083868	0.856481446776464\\
7.57624601658585	0.856480349317863\\
7.57812185159251	0.856479252528528\\
7.57999768585909	0.856478156407912\\
7.58187351938604	0.856477060955466\\
7.58374935217379	0.856475966170645\\
7.58562518422279	0.856474872052903\\
7.58750101553347	0.856473778601692\\
7.58937684610627	0.85647268581647\\
7.59125267594163	0.856471593696691\\
7.59312850504	0.856470502241812\\
7.5950043334018	0.856469411451288\\
7.59688016102747	0.856468321324578\\
7.59875598791746	0.856467231861139\\
7.6006318140722	0.85646614306043\\
7.60250763949212	0.856465054921909\\
7.60438346417766	0.856463967445036\\
7.60625928812927	0.856462880629271\\
7.60813511134736	0.856461794474074\\
7.61001093383238	0.856460708978907\\
7.61188675558477	0.856459624143232\\
7.61376257660495	0.85645853996651\\
7.61563839689336	0.856457456448204\\
7.61751421645044	0.856456373587778\\
7.61939003527662	0.856455291384696\\
7.62126585337233	0.856454209838422\\
7.623141670738	0.85645312894842\\
7.62501748737408	0.856452048714158\\
7.62689330328097	0.8564509691351\\
7.62876911845914	0.856449890210713\\
7.63064493290899	0.856448811940465\\
7.63252074663096	0.856447734323823\\
7.63439655962549	0.856446657360256\\
7.636272371893	0.856445581049231\\
7.63814818343393	0.85644450539022\\
7.6400239942487	0.856443430382691\\
7.64189980433775	0.856442356026115\\
7.64377561370149	0.856441282319963\\
7.64565142234037	0.856440209263707\\
7.6475272302548	0.856439136856818\\
7.64940303744523	0.856438065098769\\
7.65127884391207	0.856436993989034\\
7.65315464965575	0.856435923527086\\
7.65503045467669	0.856434853712399\\
7.65690625897534	0.856433784544448\\
7.6587820625521	0.856432716022708\\
7.66065786540742	0.856431648146656\\
7.66253366754171	0.856430580915767\\
7.66440946895539	0.856429514329518\\
7.6662852696489	0.856428448387387\\
7.66816106962266	0.856427383088852\\
7.67003686887709	0.856426318433391\\
7.67191266741262	0.856425254420484\\
7.67378846522967	0.856424191049609\\
7.67566426232866	0.856423128320248\\
7.67754005871002	0.85642206623188\\
7.67941585437417	0.856421004783986\\
7.68129164932153	0.85641994397605\\
7.68316744355253	0.856418883807551\\
7.68504323706758	0.856417824277975\\
7.6869190298671	0.856416765386802\\
7.68879482195153	0.856415707133518\\
7.69067061332127	0.856414649517607\\
7.69254640397675	0.856413592538554\\
7.69442219391839	0.856412536195843\\
7.69629798314661	0.856411480488961\\
7.69817377166182	0.856410425417394\\
7.70004955946445	0.856409370980629\\
7.70192534655492	0.856408317178154\\
7.70380113293364	0.856407264009457\\
7.70567691860103	0.856406211474025\\
7.70755270355751	0.856405159571349\\
7.7094284878035	0.856404108300917\\
7.71130427133942	0.856403057662219\\
7.71318005416567	0.856402007654747\\
7.71505583628268	0.856400958277991\\
7.71693161769086	0.856399909531443\\
7.71880739839063	0.856398861414594\\
7.72068317838241	0.856397813926939\\
7.7225589576666	0.856396767067969\\
7.72443473624363	0.856395720837178\\
7.72631051411391	0.856394675234061\\
7.72818629127785	0.856393630258112\\
7.73006206773587	0.856392585908826\\
7.73193784348838	0.856391542185699\\
7.73381361853579	0.856390499088228\\
7.73568939287852	0.856389456615909\\
7.73756516651697	0.856388414768238\\
7.73944093945157	0.856387373544715\\
7.74131671168272	0.856386332944837\\
7.74319248321083	0.856385292968103\\
7.74506825403632	0.856384253614013\\
7.74694402415959	0.856383214882065\\
7.74881979358106	0.856382176771761\\
7.75069556230114	0.8563811392826\\
7.75257133032024	0.856380102414085\\
7.75444709763876	0.856379066165717\\
7.75632286425712	0.856378030536999\\
7.75819863017572	0.856376995527432\\
7.76007439539498	0.856375961136521\\
7.7619501599153	0.85637492736377\\
7.76382592373709	0.856373894208682\\
7.76570168686076	0.856372861670762\\
7.76757744928671	0.856371829749516\\
7.76945321101536	0.856370798444449\\
7.7713289720471	0.856369767755068\\
7.77320473238236	0.85636873768088\\
7.77508049202152	0.856367708221392\\
7.776956250965	0.856366679376111\\
7.7788320092132	0.856365651144546\\
7.78070776676653	0.856364623526205\\
7.7825835236254	0.856363596520599\\
7.7844592797902	0.856362570127236\\
7.78633503526135	0.856361544345627\\
7.78821079003924	0.856360519175283\\
7.79008654412428	0.856359494615714\\
7.79196229751687	0.856358470666434\\
7.79383805021742	0.856357447326953\\
7.79571380222633	0.856356424596785\\
7.797589553544	0.856355402475442\\
7.79946530417083	0.856354380962439\\
7.80134105410723	0.85635336005729\\
7.8032168033536	0.856352339759509\\
7.80509255191033	0.856351320068611\\
7.80696829977783	0.856350300984113\\
7.8088440469565	0.85634928250553\\
7.81071979344674	0.856348264632379\\
7.81259553924895	0.856347247364176\\
7.81447128436352	0.85634623070044\\
7.81634702879087	0.856345214640689\\
7.81822277253138	0.856344199184441\\
7.82009851558545	0.856343184331215\\
7.82197425795349	0.856342170080531\\
7.8238499996359	0.856341156431908\\
7.82572574063306	0.856340143384868\\
7.82760148094538	0.856339130938931\\
7.82947722057325	0.856338119093618\\
7.83135295951707	0.856337107848452\\
7.83322869777725	0.856336097202955\\
7.83510443535416	0.85633508715665\\
7.83698017224822	0.856334077709059\\
7.83885590845981	0.856333068859708\\
7.84073164398933	0.85633206060812\\
7.84260737883717	0.85633105295382\\
7.84448311300374	0.856330045896333\\
7.84635884648942	0.856329039435185\\
7.84823457929461	0.856328033569903\\
7.85011031141971	0.856327028300012\\
7.85198604286509	0.85632602362504\\
7.85386177363117	0.856325019544515\\
7.85573750371833	0.856324016057965\\
7.85761323312697	0.856323013164918\\
7.85948896185747	0.856322010864903\\
7.86136468991023	0.85632100915745\\
7.86324041728564	0.856320008042089\\
7.86511614398409	0.856319007518351\\
7.86699187000597	0.856318007585765\\
7.86886759535168	0.856317008243865\\
7.8707433200216	0.85631600949218\\
7.87261904401613	0.856315011330245\\
7.87449476733565	0.85631401375759\\
7.87637048998055	0.856313016773751\\
7.87824621195123	0.85631202037826\\
7.88012193324807	0.856311024570652\\
7.88199765387146	0.856310029350461\\
7.88387337382178	0.856309034717222\\
7.88574909309944	0.856308040670472\\
7.88762481170481	0.856307047209745\\
7.88950052963828	0.856306054334579\\
7.89137624690024	0.856305062044511\\
7.89325196349107	0.856304070339077\\
7.89512767941117	0.856303079217815\\
7.89700339466092	0.856302088680265\\
7.8988791092407	0.856301098725965\\
7.9007548231509	0.856300109354454\\
7.90263053639191	0.856299120565271\\
7.90450624896411	0.856298132357958\\
7.90638196086789	0.856297144732054\\
7.90825767210362	0.856296157687101\\
7.9101333826717	0.856295171222639\\
7.91200909257251	0.856294185338212\\
7.91388480180643	0.856293200033362\\
7.91576051037385	0.856292215307631\\
7.91763621827514	0.856291231160563\\
7.9195119255107	0.856290247591702\\
7.9213876320809	0.856289264600591\\
7.92326333798612	0.856288282186776\\
7.92513904322676	0.856287300349802\\
7.92701474780318	0.856286319089214\\
7.92889045171577	0.856285338404558\\
7.93076615496491	0.856284358295382\\
7.93264185755099	0.856283378761231\\
7.93451755947438	0.856282399801654\\
7.93639326073546	0.856281421416197\\
7.93826896133461	0.85628044360441\\
7.94014466127221	0.856279466365842\\
7.94202036054865	0.85627848970004\\
7.9438960591643	0.856277513606556\\
7.94577175711953	0.856276538084939\\
7.94764745441473	0.856275563134739\\
7.94952315105028	0.856274588755508\\
7.95139884702655	0.856273614946797\\
7.95327454234392	0.856272641708158\\
7.95515023700277	0.856271669039143\\
7.95702593100348	0.856270696939304\\
7.95890162434641	0.856269725408196\\
7.96077731703195	0.856268754445372\\
7.96265300906048	0.856267784050385\\
7.96452870043237	0.856266814222791\\
7.96640439114799	0.856265844962144\\
7.96828008120772	0.856264876267999\\
7.97015577061194	0.856263908139913\\
7.97203145936102	0.856262940577442\\
7.97390714745533	0.856261973580142\\
7.97578283489525	0.856261007147571\\
7.97765852168115	0.856260041279286\\
7.97953420781341	0.856259075974845\\
7.98140989329239	0.856258111233806\\
7.98328557811848	0.856257147055729\\
7.98516126229204	0.856256183440173\\
7.98703694581345	0.856255220386697\\
7.98891262868308	0.856254257894862\\
7.9907883109013	0.856253295964228\\
7.99266399246848	0.856252334594357\\
7.99453967338499	0.856251373784809\\
7.99641535365121	0.856250413535148\\
7.9982910332675	0.856249453844934\\
8.00016671223424	0.856248494713731\\
8.0020423905518	0.856247536141102\\
8.00391806822054	0.856246578126611\\
8.00579374524083	0.856245620669822\\
8.00766942161305	0.856244663770298\\
8.00954509733756	0.856243707427606\\
8.01142077241474	0.856242751641311\\
8.01329644684494	0.856241796410977\\
8.01517212062854	0.856240841736172\\
8.01704779376591	0.856239887616462\\
8.01892346625742	0.856238934051415\\
8.02079913810342	0.856237981040596\\
8.02267480930429	0.856237028583576\\
8.0245504798604	0.856236076679921\\
8.02642614977211	0.8562351253292\\
8.02830181903979	0.856234174530983\\
8.0301774876638	0.85623322428484\\
8.03205315564451	0.856232274590339\\
8.03392882298229	0.856231325447053\\
8.03580448967749	0.856230376854551\\
8.03768015573049	0.856229428812404\\
8.03955582114165	0.856228481320185\\
8.04143148591133	0.856227534377466\\
8.0433071500399	0.856226587983819\\
8.04518281352772	0.856225642138817\\
8.04705847637516	0.856224696842033\\
8.04893413858257	0.856223752093042\\
8.05080980015032	0.856222807891417\\
8.05268546107878	0.856221864236733\\
8.0545611213683	0.856220921128565\\
8.05643678101925	0.856219978566489\\
8.05831244003199	0.856219036550079\\
8.06018809840688	0.856218095078914\\
8.06206375614429	0.856217154152569\\
8.06393941324456	0.856216213770621\\
8.06581506970807	0.856215273932648\\
8.06769072553518	0.856214334638228\\
8.06956638072623	0.856213395886939\\
8.07144203528161	0.856212457678359\\
8.07331768920166	0.856211520012069\\
8.07519334248674	0.856210582887648\\
8.07706899513721	0.856209646304675\\
8.07894464715344	0.856208710262731\\
8.08082029853577	0.856207774761397\\
8.08269594928457	0.856206839800254\\
8.0845715994002	0.856205905378884\\
8.08644724888302	0.856204971496868\\
8.08832289773337	0.856204038153789\\
8.09019854595162	0.85620310534923\\
8.09207419353813	0.856202173082774\\
8.09394984049325	0.856201241354005\\
8.09582548681734	0.856200310162506\\
8.09770113251076	0.856199379507863\\
8.09957677757385	0.85619844938966\\
8.10145242200698	0.856197519807483\\
8.1033280658105	0.856196590760916\\
8.10520370898477	0.856195662249547\\
8.10707935153014	0.856194734272961\\
8.10895499344697	0.856193806830745\\
8.1108306347356	0.856192879922488\\
8.1127062753964	0.856191953547775\\
8.11458191542972	0.856191027706196\\
8.11645755483591	0.856190102397339\\
8.11833319361532	0.856189177620793\\
8.12020883176831	0.856188253376147\\
8.12208446929523	0.856187329662991\\
8.12396010619644	0.856186406480914\\
8.12583574247227	0.856185483829508\\
8.1277113781231	0.856184561708363\\
8.12958701314926	0.85618364011707\\
8.13146264755111	0.856182719055222\\
8.133338281329	0.856181798522409\\
8.13521391448328	0.856180878518225\\
8.13708954701431	0.856179959042263\\
8.13896517892242	0.856179040094115\\
8.14084081020798	0.856178121673375\\
8.14271644087133	0.856177203779638\\
8.14459207091283	0.856176286412497\\
8.14646770033281	0.856175369571548\\
8.14834332913163	0.856174453256386\\
8.15021895730964	0.856173537466605\\
8.15209458486719	0.856172622201803\\
8.15397021180462	0.856171707461576\\
8.15584583812229	0.85617079324552\\
8.15772146382053	0.856169879553233\\
8.15959708889971	0.856168966384312\\
8.16147271336016	0.856168053738355\\
8.16334833720223	0.85616714161496\\
8.16522396042627	0.856166230013726\\
8.16709958303262	0.856165318934253\\
8.16897520502164	0.856164408376139\\
8.17085082639366	0.856163498338984\\
8.17272644714904	0.85616258882239\\
8.17460206728811	0.856161679825955\\
8.17647768681123	0.856160771349283\\
8.17835330571874	0.856159863391973\\
8.18022892401097	0.856158955953627\\
8.18210454168829	0.856158049033849\\
8.18398015875102	0.85615714263224\\
8.18585577519952	0.856156236748403\\
8.18773139103413	0.856155331381942\\
8.18960700625519	0.85615442653246\\
8.19148262086305	0.856153522199561\\
8.19335823485804	0.856152618382851\\
8.19523384824051	0.856151715081933\\
8.1971094610108	0.856150812296413\\
8.19898507316926	0.856149910025897\\
8.20086068471623	0.856149008269991\\
8.20273629565204	0.8561481070283\\
8.20461190597704	0.856147206300432\\
8.20648751569157	0.856146306085994\\
8.20836312479597	0.856145406384594\\
8.21023873329059	0.856144507195838\\
8.21211434117575	0.856143608519336\\
8.21398994845181	0.856142710354697\\
8.21586555511909	0.856141812701528\\
8.21774116117795	0.85614091555944\\
8.21961676662871	0.856140018928042\\
8.22149237147173	0.856139122806944\\
8.22336797570733	0.856138227195758\\
8.22524357933586	0.856137332094093\\
8.22711918235765	0.856136437501562\\
8.22899478477305	0.856135543417775\\
8.23087038658238	0.856134649842345\\
8.23274598778599	0.856133756774884\\
8.23462158838422	0.856132864215004\\
8.2364971883774	0.85613197216232\\
8.23837278776587	0.856131080616444\\
8.24024838654996	0.856130189576991\\
8.24212398473001	0.856129299043574\\
8.24399958230636	0.856128409015808\\
8.24587517927934	0.856127519493308\\
8.24775077564929	0.856126630475689\\
8.24962637141654	0.856125741962568\\
8.25150196658143	0.85612485395356\\
8.2533775611443	0.856123966448281\\
8.25525315510547	0.856123079446349\\
8.25712874846528	0.856122192947381\\
8.25900434122407	0.856121306950994\\
8.26087993338217	0.856120421456806\\
8.26275552493991	0.856119536464435\\
8.26463111589763	0.8561186519735\\
8.26650670625566	0.85611776798362\\
8.26838229601433	0.856116884494414\\
8.27025788517398	0.856116001505502\\
8.27213347373493	0.856115119016505\\
8.27400906169752	0.856114237027042\\
8.27588464906208	0.856113355536734\\
8.27776023582894	0.856112474545203\\
8.27963582199844	0.856111594052071\\
8.2815114075709	0.856110714056958\\
8.28338699254666	0.856109834559487\\
8.28526257692604	0.856108955559281\\
8.28713816070938	0.856108077055963\\
8.289013743897	0.856107199049157\\
8.29088932648924	0.856106321538485\\
8.29276490848643	0.856105444523572\\
8.29464048988889	0.856104568004043\\
8.29651607069696	0.856103691979521\\
8.29839165091095	0.856102816449633\\
8.30026723053121	0.856101941414004\\
8.30214280955806	0.856101066872259\\
8.30401838799183	0.856100192824025\\
8.30589396583285	0.856099319268928\\
8.30776954308143	0.856098446206596\\
8.30964511973792	0.856097573636655\\
8.31152069580264	0.856096701558733\\
8.31339627127591	0.856095829972458\\
8.31527184615806	0.856094958877459\\
8.31714742044942	0.856094088273364\\
8.31902299415032	0.856093218159801\\
8.32089856726107	0.856092348536402\\
8.32277413978201	0.856091479402794\\
8.32464971171347	0.856090610758609\\
8.32652528305576	0.856089742603477\\
8.32840085380921	0.856088874937028\\
8.33027642397415	0.856088007758894\\
8.33215199355089	0.856087141068706\\
8.33402756253978	0.856086274866096\\
8.33590313094112	0.856085409150696\\
8.33777869875525	0.85608454392214\\
8.33965426598248	0.856083679180058\\
8.34152983262315	0.856082814924086\\
8.34340539867756	0.856081951153856\\
8.34528096414606	0.856081087869002\\
8.34715652902895	0.856080225069159\\
8.34903209332656	0.856079362753961\\
8.35090765703922	0.856078500923044\\
8.35278322016724	0.856077639576042\\
8.35465878271095	0.856076778712591\\
8.35653434467066	0.856075918332328\\
8.35840990604671	0.856075058434888\\
8.3602854668394	0.856074199019908\\
8.36216102704907	0.856073340087025\\
8.36403658667603	0.856072481635878\\
8.3659121457206	0.856071623666102\\
8.3677877041831	0.856070766177337\\
8.36966326206385	0.856069909169221\\
8.37153881936318	0.856069052641392\\
8.37341437608139	0.856068196593491\\
8.37528993221882	0.856067341025155\\
8.37716548777577	0.856066485936025\\
8.37904104275257	0.856065631325741\\
8.38091659714954	0.856064777193943\\
8.38279215096699	0.856063923540273\\
8.38466770420524	0.856063070364371\\
8.38654325686461	0.856062217665879\\
8.38841880894542	0.856061365444438\\
8.39029436044798	0.856060513699692\\
8.39216991137261	0.856059662431281\\
8.39404546171963	0.85605881163885\\
8.39592101148936	0.85605796132204\\
8.3977965606821	0.856057111480497\\
8.39967210929819	0.856056262113863\\
8.40154765733792	0.856055413221782\\
8.40342320480162	0.8560545648039\\
8.40529875168961	0.856053716859861\\
8.40717429800219	0.85605286938931\\
8.40904984373969	0.856052022391893\\
8.41092538890242	0.856051175867255\\
8.41280093349069	0.856050329815043\\
8.41467647750482	0.856049484234903\\
8.41655202094512	0.856048639126482\\
8.4184275638119	0.856047794489427\\
8.42030310610548	0.856046950323386\\
8.42217864782618	0.856046106628007\\
8.4240541889743	0.856045263402937\\
8.42592972955015	0.856044420647825\\
8.42780526955406	0.85604357836232\\
8.42968080898633	0.856042736546071\\
8.43155634784727	0.856041895198728\\
8.4334318861372	0.85604105431994\\
8.43530742385643	0.856040213909358\\
8.43718296100527	0.856039373966631\\
8.43905849758403	0.856038534491412\\
8.44093403359302	0.85603769548335\\
8.44280956903256	0.856036856942098\\
8.44468510390295	0.856036018867306\\
8.4465606382045	0.856035181258628\\
8.44843617193753	0.856034344115716\\
8.45031170510235	0.856033507438222\\
8.45218723769925	0.8560326712258\\
8.45406276972857	0.856031835478102\\
8.45593830119059	0.856031000194783\\
8.45781383208563	0.856030165375497\\
8.45968936241401	0.856029331019897\\
8.46156489217603	0.85602849712764\\
8.46344042137199	0.856027663698379\\
8.46531595000221	0.85602683073177\\
8.46719147806699	0.856025998227469\\
8.46906700556664	0.856025166185131\\
8.47094253250147	0.856024334604414\\
8.47281805887179	0.856023503484973\\
8.4746935846779	0.856022672826466\\
8.47656910992011	0.85602184262855\\
8.47844463459873	0.856021012890883\\
8.48032015871406	0.856020183613122\\
8.48219568226641	0.856019354794925\\
8.48407120525608	0.856018526435952\\
8.48594672768338	0.856017698535861\\
8.48782224954862	0.856016871094311\\
8.4896977708521	0.856016044110961\\
8.49157329159412	0.856015217585473\\
8.493448811775	0.856014391517505\\
8.49532433139503	0.856013565906718\\
8.49719985045452	0.856012740752774\\
8.49907536895378	0.856011916055332\\
8.5009508868931	0.856011091814055\\
8.50282640427279	0.856010268028604\\
8.50470192109316	0.856009444698641\\
8.50657743735451	0.856008621823829\\
8.50845295305714	0.856007799403831\\
8.51032846820135	0.856006977438308\\
8.51220398278745	0.856006155926925\\
8.51407949681574	0.856005334869345\\
8.51595501028652	0.856004514265232\\
8.5178305232001	0.856003694114251\\
8.51970603555676	0.856002874416065\\
8.52158154735683	0.856002055170341\\
8.52345705860059	0.856001236376742\\
8.52533256928835	0.856000418034934\\
8.52720807942041	0.855999600144584\\
8.52908358899707	0.855998782705357\\
8.53095909801863	0.855997965716919\\
8.53283460648539	0.855997149178938\\
8.53471011439766	0.855996333091081\\
8.53658562175572	0.855995517453014\\
8.53846112855989	0.855994702264406\\
8.54033663481045	0.855993887524924\\
8.54221214050771	0.855993073234237\\
8.54408764565197	0.855992259392014\\
8.54596315024353	0.855991445997922\\
8.54783865428268	0.855990633051632\\
8.54971415776973	0.855989820552812\\
8.55158966070496	0.855989008501134\\
8.55346516308869	0.855988196896266\\
8.5553406649212	0.855987385737879\\
8.55721616620279	0.855986575025644\\
8.55909166693377	0.855985764759233\\
8.56096716711443	0.855984954938315\\
8.56284266674506	0.855984145562564\\
8.56471816582596	0.85598333663165\\
8.56659366435743	0.855982528145246\\
8.56846916233976	0.855981720103026\\
8.57034465977325	0.85598091250466\\
8.5722201566582	0.855980105349824\\
8.5740956529949	0.855979298638189\\
8.57597114878365	0.855978492369431\\
8.57784664402474	0.855977686543222\\
8.57972213871846	0.855976881159238\\
8.58159763286512	0.855976076217153\\
8.583473126465	0.855975271716641\\
8.5853486195184	0.855974467657379\\
8.58722411202561	0.855973664039042\\
8.58909960398693	0.855972860861306\\
8.59097509540265	0.855972058123846\\
8.59285058627307	0.85597125582634\\
8.59472607659848	0.855970453968463\\
8.59660156637916	0.855969652549894\\
8.59847705561542	0.855968851570309\\
8.60035254430755	0.855968051029387\\
8.60222803245583	0.855967250926804\\
8.60410352006056	0.85596645126224\\
8.60597900712204	0.855965652035373\\
8.60785449364055	0.855964853245882\\
8.60972997961638	0.855964054893446\\
8.61160546504984	0.855963256977743\\
8.6134809499412	0.855962459498455\\
8.61535643429075	0.855961662455262\\
8.6172319180988	0.855960865847842\\
8.61910740136563	0.855960069675878\\
8.62098288409153	0.855959273939049\\
8.62285836627679	0.855958478637038\\
8.62473384792169	0.855957683769525\\
8.62660932902654	0.855956889336193\\
8.62848480959162	0.855956095336723\\
8.63036028961722	0.855955301770798\\
8.63223576910362	0.855954508638101\\
8.63411124805112	0.855953715938314\\
8.63598672646	0.855952923671122\\
8.63786220433055	0.855952131836206\\
8.63973768166307	0.855951340433252\\
8.64161315845784	0.855950549461943\\
8.64348863471514	0.855949758921964\\
8.64536411043527	0.855948968812999\\
8.64723958561851	0.855948179134734\\
8.64911506026515	0.855947389886853\\
8.65099053437547	0.855946601069043\\
8.65286600794977	0.855945812680988\\
8.65474148098833	0.855945024722377\\
8.65661695349143	0.855944237192893\\
8.65849242545937	0.855943450092226\\
8.66036789689242	0.855942663420061\\
8.66224336779088	0.855941877176085\\
8.66411883815502	0.855941091359987\\
8.66599430798514	0.855940305971454\\
8.66786977728152	0.855939521010175\\
8.66974524604444	0.855938736475838\\
8.67162071427419	0.855937952368131\\
8.67349618197106	0.855937168686744\\
8.67537164913532	0.855936385431365\\
8.67724711576727	0.855935602601685\\
8.67912258186718	0.855934820197394\\
8.68099804743534	0.855934038218181\\
8.68287351247203	0.855933256663737\\
8.68474897697754	0.855932475533753\\
8.68662444095215	0.855931694827919\\
8.68849990439615	0.855930914545927\\
8.6903753673098	0.855930134687469\\
8.69225082969341	0.855929355252236\\
8.69412629154724	0.855928576239921\\
8.69600175287159	0.855927797650216\\
8.69787721366673	0.855927019482813\\
8.69975267393295	0.855926241737406\\
8.70162813367052	0.855925464413688\\
8.70350359287973	0.855924687511353\\
8.70537905156086	0.855923911030093\\
8.7072545097142	0.855923134969604\\
8.70912996734001	0.85592235932958\\
8.71100542443858	0.855921584109715\\
8.7128808810102	0.855920809309704\\
8.71475633705514	0.855920034929243\\
8.71663179257368	0.855919260968026\\
8.7185072475661	0.855918487425751\\
8.72038270203268	0.855917714302112\\
8.72225815597371	0.855916941596806\\
8.72413360938945	0.855916169309531\\
8.72600906228019	0.855915397439981\\
8.72788451464621	0.855914625987856\\
8.72975996648778	0.855913854952851\\
8.73163541780519	0.855913084334666\\
8.73351086859871	0.855912314132997\\
8.73538631886862	0.855911544347544\\
8.7372617686152	0.855910774978004\\
8.73913721783872	0.855910006024077\\
8.74101266653947	0.855909237485461\\
8.74288811471772	0.855908469361856\\
8.74476356237374	0.855907701652961\\
8.74663900950781	0.855906934358477\\
8.74851445612022	0.855906167478103\\
8.75038990221123	0.855905401011541\\
8.75226534778113	0.85590463495849\\
8.75414079283018	0.855903869318652\\
8.75601623735867	0.855903104091728\\
8.75789168136687	0.855902339277419\\
8.75976712485505	0.855901574875427\\
8.7616425678235	0.855900810885455\\
8.76351801027247	0.855900047307205\\
8.76539345220226	0.855899284140379\\
8.76726889361314	0.85589852138468\\
8.76914433450538	0.855897759039811\\
8.77101977487924	0.855896997105477\\
8.77289521473502	0.855896235581379\\
8.77477065407298	0.855895474467223\\
8.77664609289339	0.855894713762713\\
8.77852153119653	0.855893953467553\\
8.78039696898267	0.855893193581448\\
8.78227240625209	0.855892434104102\\
8.78414784300506	0.855891675035222\\
8.78602327924184	0.855890916374513\\
8.78789871496272	0.85589015812168\\
8.78977415016797	0.85588940027643\\
8.79164958485785	0.85588864283847\\
8.79352501903264	0.855887885807504\\
8.79540045269261	0.855887129183242\\
8.79727588583804	0.855886372965389\\
8.79915131846919	0.855885617153653\\
8.80102675058633	0.855884861747743\\
8.80290218218974	0.855884106747365\\
8.80477761327968	0.855883352152228\\
8.80665304385644	0.855882597962041\\
8.80852847392027	0.855881844176512\\
8.81040390347145	0.85588109079535\\
8.81227933251025	0.855880337818265\\
8.81415476103694	0.855879585244966\\
8.81603018905178	0.855878833075162\\
8.81790561655505	0.855878081308565\\
8.81978104354702	0.855877329944884\\
8.82165647002795	0.85587657898383\\
8.82353189599811	0.855875828425113\\
8.82540732145778	0.855875078268446\\
8.82728274640722	0.855874328513538\\
8.82915817084671	0.855873579160102\\
8.83103359477649	0.85587283020785\\
8.83290901819686	0.855872081656494\\
8.83478444110806	0.855871333505746\\
8.83665986351038	0.855870585755319\\
8.83853528540408	0.855869838404926\\
8.84041070678942	0.855869091454279\\
8.84228612766668	0.855868344903093\\
8.84416154803611	0.855867598751081\\
8.84603696789799	0.855866852997958\\
8.84791238725258	0.855866107643436\\
8.84978780610016	0.855865362687231\\
8.85166322444097	0.855864618129058\\
8.8535386422753	0.855863873968631\\
8.8554140596034	0.855863130205666\\
8.85728947642555	0.855862386839878\\
8.859164892742	0.855861643870983\\
8.86104030855303	0.855860901298697\\
8.86291572385889	0.855860159122737\\
8.86479113865986	0.855859417342818\\
8.86666655295619	0.855858675958657\\
8.86854196674815	0.855857934969972\\
8.87041738003601	0.85585719437648\\
8.87229279282003	0.855856454177898\\
8.87416820510047	0.855855714373944\\
8.8760436168776	0.855854974964337\\
8.87791902815168	0.855854235948794\\
8.87979443892297	0.855853497327034\\
8.88166984919174	0.855852759098776\\
8.88354525895825	0.855852021263739\\
8.88542066822276	0.855851283821642\\
8.88729607698554	0.855850546772204\\
8.88917148524684	0.855849810115146\\
8.89104689300694	0.855849073850188\\
8.89292230026608	0.85584833797705\\
8.89479770702454	0.855847602495452\\
8.89667311328257	0.855846867405116\\
8.89854851904044	0.855846132705761\\
8.90042392429841	0.855845398397111\\
8.90229932905674	0.855844664478886\\
8.90417473331568	0.855843930950807\\
8.90605013707551	0.855843197812598\\
8.90792554033648	0.85584246506398\\
8.90980094309885	0.855841732704676\\
8.91167634536288	0.855841000734409\\
8.91355174712884	0.855840269152902\\
8.91542714839697	0.855839537959878\\
8.91730254916755	0.85583880715506\\
8.91917794944083	0.855838076738173\\
8.92105334921707	0.855837346708941\\
8.92292874849652	0.855836617067087\\
8.92480414727946	0.855835887812336\\
8.92667954556613	0.855835158944414\\
8.9285549433568	0.855834430463045\\
8.93043034065172	0.855833702367954\\
8.93230573745116	0.855832974658867\\
8.93418113375536	0.85583224733551\\
8.93605652956459	0.855831520397608\\
8.93793192487911	0.855830793844888\\
8.93980731969917	0.855830067677077\\
8.94168271402503	0.855829341893901\\
8.94355810785695	0.855828616495087\\
8.94543350119518	0.855827891480362\\
8.94730889403999	0.855827166849454\\
8.94918428639162	0.855826442602091\\
8.95105967825034	0.855825718738001\\
8.95293506961639	0.85582499525691\\
8.95481046049005	0.85582427215855\\
8.95668585087155	0.855823549442647\\
8.95856124076117	0.85582282710893\\
8.96043663015915	0.85582210515713\\
8.96231201906575	0.855821383586975\\
8.96418740748122	0.855820662398195\\
8.96606279540582	0.85581994159052\\
8.96793818283981	0.855819221163679\\
8.96981356978343	0.855818501117404\\
8.97168895623695	0.855817781451425\\
8.97356434220062	0.855817062165473\\
8.97543972767469	0.855816343259279\\
8.97731511265941	0.855815624732574\\
8.97919049715505	0.85581490658509\\
8.98106588116184	0.855814188816558\\
8.98294126468006	0.855813471426711\\
8.98481664770994	0.855812754415281\\
8.98669203025175	0.855812037782\\
8.98856741230573	0.855811321526601\\
8.99044279387214	0.855810605648818\\
8.99231817495123	0.855809890148383\\
8.99419355554326	0.85580917502503\\
8.99606893564847	0.855808460278492\\
8.99794431526712	0.855807745908504\\
8.99981969439946	0.8558070319148\\
9.00169507304574	0.855806318297113\\
9.00357045120622	0.85580560505518\\
9.00544582888113	0.855804892188733\\
9.00732120607075	0.85580417969751\\
9.0091965827753	0.855803467581245\\
9.01107195899506	0.855802755839673\\
9.01294733473027	0.855802044472531\\
9.01482270998117	0.855801333479554\\
9.01669808474802	0.85580062286048\\
9.01857345903107	0.855799912615043\\
9.02044883283057	0.855799202742982\\
9.02232420614676	0.855798493244032\\
9.02419957897991	0.855797784117932\\
9.02607495133025	0.855797075364419\\
9.02795032319804	0.85579636698323\\
9.02982569458353	0.855795658974104\\
9.03170106548696	0.855794951336778\\
9.03357643590858	0.85579424407099\\
9.03545180584865	0.855793537176481\\
9.03732717530741	0.855792830652987\\
9.0392025442851	0.855792124500249\\
9.04107791278198	0.855791418718006\\
9.0429532807983	0.855790713305997\\
9.0448286483343	0.855790008263962\\
9.04670401539023	0.855789303591641\\
9.04857938196634	0.855788599288774\\
9.05045474806287	0.855787895355101\\
9.05233011368007	0.855787191790364\\
9.0542054788182	0.855786488594304\\
9.05608084347748	0.85578578576666\\
9.05795620765818	0.855785083307176\\
9.05983157136054	0.855784381215592\\
9.06170693458481	0.855783679491649\\
9.06358229733122	0.855782978135091\\
9.06545765960003	0.85578227714566\\
9.06733302139148	0.855781576523098\\
9.06920838270582	0.855780876267147\\
9.07108374354329	0.855780176377551\\
9.07295910390414	0.855779476854053\\
9.07483446378861	0.855778777696396\\
9.07670982319695	0.855778078904324\\
9.07858518212941	0.85577738047758\\
9.08046054058622	0.85577668241591\\
9.08233589856763	0.855775984719056\\
9.08421125607389	0.855775287386765\\
9.08608661310524	0.85577459041878\\
9.08796196966192	0.855773893814846\\
9.08983732574418	0.855773197574709\\
9.09171268135226	0.855772501698114\\
9.09358803648641	0.855771806184807\\
9.09546339114686	0.855771111034534\\
9.09733874533386	0.85577041624704\\
9.09921409904766	0.855769721822072\\
9.10108945228849	0.855769027759377\\
9.1029648050566	0.855768334058701\\
9.10484015735223	0.855767640719792\\
9.10671550917563	0.855766947742396\\
9.10859086052703	0.855766255126261\\
9.11046621140667	0.855765562871135\\
9.11234156181481	0.855764870976766\\
9.11421691175167	0.855764179442901\\
9.11609226121751	0.855763488269289\\
9.11796761021256	0.855762797455679\\
9.11984295873707	0.855762107001819\\
9.12171830679126	0.855761416907458\\
9.1235936543754	0.855760727172346\\
9.12546900148971	0.855760037796232\\
9.12734434813443	0.855759348778865\\
9.12921969430981	0.855758660119996\\
9.13109504001609	0.855757971819374\\
9.13297038525351	0.855757283876751\\
9.1348457300223	0.855756596291875\\
9.13672107432271	0.855755909064499\\
9.13859641815497	0.855755222194372\\
9.14047176151933	0.855754535681247\\
9.14234710441602	0.855753849524875\\
9.14422244684528	0.855753163725006\\
9.14609778880736	0.855752478281394\\
9.14797313030248	0.85575179319379\\
9.14984847133089	0.855751108461946\\
9.15172381189283	0.855750424085615\\
9.15359915198853	0.85574974006455\\
9.15547449161824	0.855749056398503\\
9.15734983078218	0.855748373087227\\
9.1592251694806	0.855747690130477\\
9.16110050771374	0.855747007528005\\
9.16297584548183	0.855746325279565\\
9.16485118278511	0.855745643384911\\
9.16672651962382	0.855744961843798\\
9.16860185599819	0.85574428065598\\
9.17047719190846	0.855743599821211\\
9.17235252735487	0.855742919339246\\
9.17422786233765	0.855742239209841\\
9.17610319685704	0.85574155943275\\
9.17797853091327	0.85574088000773\\
9.17985386450659	0.855740200934535\\
9.18172919763722	0.855739522212922\\
9.1836045303054	0.855738843842647\\
9.18547986251137	0.855738165823467\\
9.18735519425537	0.855737488155136\\
9.18923052553762	0.855736810837414\\
9.19110585635836	0.855736133870056\\
9.19298118671784	0.855735457252819\\
9.19485651661627	0.855734780985462\\
9.1967318460539	0.855734105067741\\
9.19860717503096	0.855733429499415\\
9.20048250354768	0.855732754280241\\
9.20235783160431	0.855732079409978\\
9.20423315920106	0.855731404888385\\
9.20610848633819	0.855730730715219\\
9.20798381301591	0.85573005689024\\
9.20985913923446	0.855729383413206\\
9.21173446499408	0.855728710283878\\
9.213609790295	0.855728037502014\\
9.21548511513745	0.855727365067375\\
9.21736043952167	0.85572669297972\\
9.21923576344788	0.855726021238809\\
9.22111108691632	0.855725349844403\\
9.22298640992722	0.855724678796262\\
9.22486173248082	0.855724008094147\\
9.22673705457734	0.855723337737818\\
9.22861237621702	0.855722667727038\\
9.23048769740008	0.855721998061568\\
9.23236301812677	0.855721328741168\\
9.23423833839731	0.855720659765601\\
9.23611365821193	0.855719991134629\\
9.23798897757086	0.855719322848014\\
9.23986429647434	0.855718654905518\\
9.24173961492259	0.855717987306905\\
9.24361493291585	0.855717320051936\\
9.24549025045434	0.855716653140375\\
9.2473655675383	0.855715986571985\\
9.24924088416796	0.855715320346529\\
9.25111620034354	0.855714654463772\\
9.25299151606527	0.855713988923476\\
9.25486683133339	0.855713323725407\\
9.25674214614813	0.855712658869327\\
9.25861746050971	0.855711994355003\\
9.26049277441836	0.855711330182198\\
9.26236808787431	0.855710666350677\\
9.26424340087779	0.855710002860205\\
9.26611871342903	0.855709339710548\\
9.26799402552826	0.855708676901471\\
9.2698693371757	0.855708014432739\\
9.27174464837159	0.855707352304119\\
9.27361995911615	0.855706690515377\\
9.27549526940961	0.855706029066278\\
9.27737057925219	0.85570536795659\\
9.27924588864413	0.855704707186079\\
9.28112119758566	0.855704046754512\\
9.28299650607699	0.855703386661656\\
9.28487181411836	0.855702726907278\\
9.28674712170999	0.855702067491146\\
9.28862242885211	0.855701408413027\\
9.29049773554495	0.85570074967269\\
9.29237304178874	0.855700091269902\\
9.29424834758369	0.855699433204432\\
9.29612365293004	0.855698775476049\\
9.29799895782801	0.85569811808452\\
9.29987426227783	0.855697461029615\\
9.30174956627972	0.855696804311103\\
9.30362486983392	0.855696147928752\\
9.30550017294063	0.855695491882334\\
9.3073754756001	0.855694836171617\\
9.30925077781254	0.855694180796371\\
9.31112607957818	0.855693525756367\\
9.31300138089724	0.855692871051373\\
9.31487668176996	0.855692216681162\\
9.31675198219655	0.855691562645504\\
9.31862728217723	0.855690908944169\\
9.32050258171224	0.855690255576929\\
9.3223778808018	0.855689602543555\\
9.32425317944612	0.855688949843818\\
9.32612847764544	0.85568829747749\\
9.32800377539998	0.855687645444343\\
9.32987907270996	0.855686993744149\\
9.3317543695756	0.85568634237668\\
9.33362966599713	0.855685691341708\\
9.33550496197477	0.855685040639007\\
9.33738025750874	0.855684390268349\\
9.33925555259927	0.855683740229507\\
9.34113084724658	0.855683090522254\\
9.34300614145089	0.855682441146364\\
9.34488143521243	0.85568179210161\\
9.34675672853141	0.855681143387766\\
9.34863202140806	0.855680495004606\\
9.35050731384259	0.855679846951905\\
9.35238260583524	0.855679199229436\\
9.35425789738622	0.855678551836975\\
9.35613318849575	0.855677904774295\\
9.35800847916406	0.855677258041173\\
9.35988376939136	0.855676611637382\\
9.36175905917789	0.855675965562699\\
9.36363434852385	0.855675319816899\\
9.36550963742946	0.855674674399758\\
9.36738492589496	0.855674029311052\\
9.36926021392056	0.855673384550556\\
9.37113550150647	0.855672740118048\\
9.37301078865293	0.855672096013303\\
9.37488607536014	0.855671452236098\\
9.37676136162834	0.855670808786211\\
9.37863664745773	0.855670165663418\\
9.38051193284854	0.855669522867496\\
9.38238721780099	0.855668880398224\\
9.3842625023153	0.855668238255378\\
9.38613778639168	0.855667596438736\\
9.38801307003036	0.855666954948077\\
9.38988835323154	0.855666313783178\\
9.39176363599547	0.855665672943819\\
9.39363891832234	0.855665032429778\\
9.39551420021238	0.855664392240833\\
9.3973894816658	0.855663752376763\\
9.39926476268284	0.855663112837348\\
9.40114004326369	0.855662473622367\\
9.40301532340859	0.855661834731599\\
9.40489060311774	0.855661196164825\\
9.40676588239137	0.855660557921823\\
9.40864116122969	0.855659920002375\\
9.41051643963292	0.855659282406261\\
9.41239171760128	0.85565864513326\\
9.41426699513498	0.855658008183154\\
9.41614227223425	0.855657371555724\\
9.41801754889929	0.85565673525075\\
9.41989282513032	0.855656099268014\\
9.42176810092756	0.855655463607297\\
9.42364337629123	0.855654828268381\\
9.42551865122154	0.855654193251047\\
9.42739392571871	0.855653558555078\\
9.42926919978295	0.855652924180255\\
9.43114447341448	0.855652290126362\\
9.43301974661351	0.85565165639318\\
9.43489501938027	0.855651022980493\\
9.43677029171495	0.855650389888082\\
9.43864556361779	0.855649757115732\\
9.44052083508899	0.855649124663225\\
9.44239610612878	0.855648492530345\\
9.44427137673735	0.855647860716876\\
9.44614664691493	0.855647229222601\\
9.44802191666174	0.855646598047304\\
9.44989718597798	0.85564596719077\\
9.45177245486388	0.855645336652783\\
9.45364772331963	0.855644706433127\\
9.45552299134547	0.855644076531588\\
9.45739825894159	0.855643446947949\\
9.45927352610823	0.855642817681997\\
9.46114879284558	0.855642188733517\\
9.46302405915386	0.855641560102293\\
9.46489932503328	0.855640931788112\\
9.46677459048407	0.855640303790759\\
9.46864985550642	0.855639676110021\\
9.47052512010056	0.855639048745683\\
9.47240038426669	0.855638421697533\\
9.47427564800502	0.855637794965355\\
9.47615091131578	0.855637168548939\\
9.47802617419917	0.855636542448069\\
9.4799014366554	0.855635916662533\\
9.48177669868469	0.85563529119212\\
9.48365196028724	0.855634666036615\\
9.48552722146328	0.855634041195806\\
9.487402482213	0.855633416669483\\
9.48927774253662	0.855632792457431\\
9.49115300243435	0.855632168559441\\
9.49302826190641	0.855631544975299\\
9.494903520953	0.855630921704795\\
9.49677877957433	0.855630298747717\\
9.49865403777062	0.855629676103854\\
9.50052929554207	0.855629053772995\\
9.5024045528889	0.85562843175493\\
9.50427980981131	0.855627810049447\\
9.50615506630952	0.855627188656337\\
9.50803032238373	0.855626567575389\\
9.50990557803416	0.855625946806393\\
9.51178083326101	0.85562532634914\\
9.51365608806449	0.855624706203419\\
9.51553134244482	0.855624086369022\\
9.51740659640219	0.855623466845738\\
9.51928184993683	0.855622847633359\\
9.52115710304894	0.855622228731675\\
9.52303235573872	0.855621610140478\\
9.5249076080064	0.85562099185956\\
9.52678285985216	0.855620373888711\\
9.52865811127623	0.855619756227724\\
9.53053336227881	0.85561913887639\\
9.53240861286011	0.855618521834502\\
9.53428386302034	0.855617905101851\\
9.5361591127597	0.85561728867823\\
9.5380343620784	0.855616672563433\\
9.53990961097666	0.85561605675725\\
9.54178485945467	0.855615441259476\\
9.54366010751265	0.855614826069904\\
9.5455353551508	0.855614211188327\\
9.54741060236932	0.855613596614538\\
9.54928584916843	0.855612982348331\\
9.55116109554834	0.855612368389499\\
9.55303634150924	0.855611754737838\\
9.55491158705134	0.855611141393141\\
9.55678683217486	0.855610528355202\\
9.55866207688	0.855609915623815\\
9.56053732116695	0.855609303198777\\
9.56241256503594	0.85560869107988\\
9.56428780848716	0.855608079266921\\
9.56616305152082	0.855607467759695\\
9.56803829413712	0.855606856557997\\
9.56991353633628	0.855606245661622\\
9.57178877811849	0.855605635070367\\
9.57366401948397	0.855605024784026\\
9.57553926043291	0.855604414802397\\
9.57741450096552	0.855603805125276\\
9.579289741082	0.855603195752458\\
9.58116498078256	0.85560258668374\\
9.58304022006741	0.85560197791892\\
9.58491545893675	0.855601369457794\\
9.58679069739077	0.855600761300159\\
9.5886659354297	0.855600153445813\\
9.59054117305372	0.855599545894552\\
9.59241641026305	0.855598938646175\\
9.59429164705788	0.855598331700479\\
9.59616688343842	0.855597725057262\\
9.59804211940488	0.855597118716322\\
9.59991735495746	0.855596512677459\\
9.60179259009635	0.855595906940469\\
9.60366782482177	0.855595301505151\\
9.60554305913391	0.855594696371305\\
9.60741829303299	0.855594091538729\\
9.60929352651919	0.855593487007223\\
9.61116875959272	0.855592882776586\\
9.61304399225379	0.855592278846617\\
9.6149192245026	0.855591675217115\\
9.61679445633935	0.855591071887881\\
9.61866968776424	0.855590468858715\\
9.62054491877747	0.855589866129417\\
9.62242014937924	0.855589263699786\\
9.62429537956976	0.855588661569624\\
9.62617060934923	0.855588059738731\\
9.62804583871784	0.855587458206908\\
9.62992106767581	0.855586856973956\\
9.63179629622332	0.855586256039675\\
9.63367152436059	0.855585655403868\\
9.63554675208781	0.855585055066336\\
9.63742197940518	0.855584455026879\\
9.6392972063129	0.855583855285301\\
9.64117243281117	0.855583255841403\\
9.6430476589002	0.855582656694988\\
9.64492288458018	0.855582057845857\\
9.64679810985131	0.855581459293812\\
9.64867333471379	0.855580861038657\\
9.65054855916783	0.855580263080195\\
9.65242378321362	0.855579665418228\\
9.65429900685135	0.855579068052559\\
9.65617423008124	0.855578470982991\\
9.65804945290348	0.855577874209329\\
9.65992467531826	0.855577277731376\\
9.66179989732579	0.855576681548935\\
9.66367511892627	0.85557608566181\\
9.66555034011989	0.855575490069806\\
9.66742556090685	0.855574894772727\\
9.66930078128735	0.855574299770376\\
9.6711760012616	0.855573705062559\\
9.67305122082978	0.855573110649081\\
9.67492643999209	0.855572516529746\\
9.67680165874873	0.85557192270436\\
9.67867687709991	0.855571329172727\\
9.68055209504581	0.855570735934654\\
9.68242731258664	0.855570142989944\\
9.6843025297226	0.855569550338406\\
9.68617774645387	0.855568957979843\\
9.68805296278065	0.855568365914063\\
9.68992817870316	0.855567774140871\\
9.69180339422157	0.855567182660074\\
9.69367860933608	0.855566591471478\\
9.69555382404691	0.85556600057489\\
9.69742903835423	0.855565409970117\\
9.69930425225824	0.855564819656966\\
9.70117946575915	0.855564229635244\\
9.70305467885715	0.855563639904758\\
9.70492989155243	0.855563050465316\\
9.70680510384519	0.855562461316726\\
9.70868031573562	0.855561872458795\\
9.71055552722393	0.855561283891331\\
9.7124307383103	0.855560695614142\\
9.71430594899494	0.855560107627037\\
9.71618115927803	0.855559519929824\\
9.71805636915977	0.855558932522311\\
9.71993157864036	0.855558345404308\\
9.72180678771998	0.855557758575624\\
9.72368199639885	0.855557172036067\\
9.72555720467714	0.855556585785446\\
9.72743241255506	0.855555999823571\\
9.7293076200328	0.855555414150252\\
9.73118282711055	0.855554828765299\\
9.73305803378851	0.85555424366852\\
9.73493324006687	0.855553658859727\\
9.73680844594582	0.855553074338729\\
9.73868365142556	0.855552490105336\\
9.74055885650628	0.85555190615936\\
9.74243406118817	0.855551322500611\\
9.74430926547144	0.855550739128899\\
9.74618446935626	0.855550156044036\\
9.74805967284284	0.855549573245832\\
9.74993487593137	0.8555489907341\\
9.75181007862203	0.855548408508649\\
9.75368528091503	0.855547826569293\\
9.75556048281055	0.855547244915843\\
9.75743568430879	0.85554666354811\\
9.75931088540993	0.855546082465907\\
9.76118608611418	0.855545501669046\\
9.76306128642173	0.855544921157338\\
9.76493648633276	0.855544340930598\\
9.76681168584746	0.855543760988637\\
9.76868688496603	0.855543181331269\\
9.77056208368866	0.855542601958306\\
9.77243728201555	0.855542022869561\\
9.77431247994688	0.855541444064847\\
9.77618767748284	0.855540865543979\\
9.77806287462362	0.85554028730677\\
9.77993807136942	0.855539709353033\\
9.78181326772043	0.855539131682582\\
9.78368846367683	0.855538554295232\\
9.78556365923882	0.855537977190797\\
9.78743885440659	0.85553740036909\\
9.78931404918033	0.855536823829927\\
9.79118924356022	0.855536247573122\\
9.79306443754647	0.85553567159849\\
9.79493963113925	0.855535095905847\\
9.79681482433876	0.855534520495006\\
9.79869001714519	0.855533945365783\\
9.80056520955872	0.855533370517995\\
9.80244040157955	0.855532795951456\\
9.80431559320786	0.855532221665982\\
9.80619078444385	0.855531647661389\\
9.80806597528771	0.855531073937493\\
9.80994116573962	0.85553050049411\\
9.81181635579977	0.855529927331057\\
9.81369154546834	0.85552935444815\\
9.81556673474554	0.855528781845206\\
9.81744192363155	0.855528209522041\\
9.81931711212655	0.855527637478473\\
9.82119230023073	0.855527065714318\\
9.82306748794429	0.855526494229394\\
9.8249426752674	0.855525923023519\\
9.82681786220027	0.855525352096509\\
9.82869304874307	0.855524781448182\\
9.83056823489599	0.855524211078357\\
9.83244342065922	0.855523640986851\\
9.83431860603295	0.855523071173482\\
9.83619379101736	0.855522501638069\\
9.83806897561265	0.855521932380431\\
9.83994415981899	0.855521363400384\\
9.84181934363659	0.85552079469775\\
9.84369452706561	0.855520226272346\\
9.84556971010625	0.855519658123991\\
9.8474448927587	0.855519090252505\\
9.84932007502314	0.855518522657706\\
9.85119525689976	0.855517955339415\\
9.85307043838874	0.855517388297451\\
9.85494561949027	0.855516821531634\\
9.85682080020455	0.855516255041784\\
9.85869598053174	0.85551568882772\\
9.86057116047204	0.855515122889263\\
9.86244634002563	0.855514557226234\\
9.86432151919271	0.855513991838452\\
9.86619669797344	0.855513426725739\\
9.86807187636803	0.855512861887916\\
9.86994705437665	0.855512297324802\\
9.87182223199949	0.85551173303622\\
9.87369740923674	0.855511169021991\\
9.87557258608857	0.855510605281935\\
9.87744776255517	0.855510041815875\\
9.87932293863674	0.855509478623633\\
9.88119811433345	0.855508915705029\\
9.88307328964548	0.855508353059886\\
9.88494846457303	0.855507790688027\\
9.88682363911627	0.855507228589273\\
9.88869881327539	0.855506666763447\\
9.89057398705057	0.855506105210371\\
9.89244916044199	0.855505543929868\\
9.89432433344985	0.855504982921761\\
9.89619950607432	0.855504422185874\\
9.89807467831559	0.855503861722028\\
9.89994985017384	0.855503301530047\\
9.90182502164925	0.855502741609755\\
9.903700192742	0.855502181960975\\
9.90557536345229	0.855501622583532\\
9.90745053378028	0.855501063477248\\
9.90932570372617	0.855500504641948\\
9.91120087329014	0.855499946077456\\
9.91307604247237	0.855499387783596\\
9.91495121127303	0.855498829760193\\
9.91682637969232	0.855498272007071\\
9.91870154773042	0.855497714524055\\
9.9205767153875	0.85549715731097\\
9.92245188266375	0.855496600367641\\
9.92432704955935	0.855496043693893\\
9.92620221607449	0.855495487289551\\
9.92807738220933	0.855494931154441\\
9.92995254796408	0.855494375288388\\
9.9318277133389	0.855493819691217\\
9.93370287833397	0.855493264362756\\
9.93557804294949	0.855492709302829\\
9.93745320718562	0.855492154511263\\
9.93932837104256	0.855491599987885\\
9.94120353452048	0.85549104573252\\
9.94307869761955	0.855490491744995\\
9.94495386033997	0.855489938025137\\
9.94682902268191	0.855489384572773\\
9.94870418464556	0.855488831387729\\
9.95057934623108	0.855488278469833\\
9.95245450743867	0.855487725818912\\
9.9543296682685	0.855487173434794\\
9.95620482872075	0.855486621317305\\
9.9580799887956	0.855486069466274\\
9.95995514849323	0.855485517881529\\
9.96183030781383	0.855484966562897\\
9.96370546675756	0.855484415510206\\
9.96558062532461	0.855483864723286\\
9.96745578351516	0.855483314201963\\
9.96933094132939	0.855482763946067\\
9.97120609876747	0.855482213955426\\
9.97308125582959	0.85548166422987\\
9.97495641251592	0.855481114769226\\
9.97683156882664	0.855480565573325\\
9.97870672476193	0.855480016641994\\
9.98058188032197	0.855479467975065\\
9.98245703550694	0.855478919572365\\
9.98433219031701	0.855478371433726\\
9.98620734475236	0.855477823558976\\
9.98808249881317	0.855477275947945\\
9.98995765249962	0.855476728600464\\
9.99183280581188	0.855476181516362\\
9.99370795875014	0.85547563469547\\
9.99558311131457	0.855475088137619\\
9.99745826350534	0.855474541842638\\
9.99933341532264	0.85547399581036\\
10.0012085667666	0.855473450040613\\
10.0030837178375	0.855472904533231\\
10.0049588685355	0.855472359288043\\
10.0068340188606	0.855471814304881\\
10.0087091688132	0.855471269583576\\
10.0105843183934	0.85547072512396\\
10.0124594676013	0.855470180925865\\
10.0143346164371	0.855469636989122\\
10.0162097649011	0.855469093313564\\
10.0180849129934	0.855468549899022\\
10.0199600607141	0.855468006745328\\
10.0218352080635	0.855467463852316\\
10.0237103550417	0.855466921219817\\
10.0255855016488	0.855466378847664\\
10.0274606478852	0.85546583673569\\
10.0293357937509	0.855465294883728\\
10.0312109392461	0.85546475329161\\
10.033086084371	0.855464211959171\\
10.0349612291258	0.855463670886243\\
10.0368363735106	0.855463130072659\\
10.0387115175256	0.855462589518254\\
10.040586661171	0.855462049222861\\
10.042461804447	0.855461509186313\\
10.0443369473538	0.855460969408445\\
10.0462120898914	0.855460429889091\\
10.0480872320602	0.855459890628086\\
10.0499623738602	0.855459351625262\\
10.0518375152916	0.855458812880456\\
10.0537126563547	0.855458274393501\\
10.0555877970495	0.855457736164233\\
10.0574629373763	0.855457198192486\\
10.0593380773353	0.855456660478095\\
10.0612132169265	0.855456123020896\\
10.0630883561503	0.855455585820724\\
10.0649634950067	0.855455048877414\\
10.0668386334959	0.855454512190801\\
10.0687137716181	0.855453975760722\\
10.0705889093735	0.855453439587013\\
10.0724640467622	0.855452903669508\\
10.0743391837845	0.855452368008045\\
10.0762143204404	0.85545183260246\\
10.0780894567302	0.855451297452589\\
10.0799645926541	0.855450762558268\\
10.0818397282121	0.855450227919334\\
10.0837148634046	0.855449693535624\\
10.0855899982316	0.855449159406974\\
10.0874651326933	0.855448625533222\\
10.0893402667899	0.855448091914206\\
10.0912154005216	0.855447558549761\\
10.0930905338885	0.855447025439726\\
10.0949656668908	0.855446492583937\\
10.0968407995288	0.855445959982234\\
10.0987159318024	0.855445427634453\\
10.100591063712	0.855444895540432\\
10.1024661952577	0.85544436370001\\
10.1043413264397	0.855443832113025\\
10.1062164572581	0.855443300779315\\
10.1080915877131	0.855442769698718\\
10.1099667178049	0.855442238871073\\
10.1118418475336	0.855441708296218\\
10.1137169768995	0.855441177973994\\
10.1155921059027	0.855440647904237\\
10.1174672345433	0.855440118086789\\
10.1193423628215	0.855439588521487\\
10.1212174907376	0.855439059208171\\
10.1230926182916	0.855438530146681\\
10.1249677454838	0.855438001336856\\
10.1268428723143	0.855437472778536\\
10.1287179987832	0.85543694447156\\
10.1305931248908	0.85543641641577\\
10.1324682506373	0.855435888611004\\
10.1343433760227	0.855435361057104\\
10.1362185010473	0.855434833753909\\
10.1380936257112	0.85543430670126\\
10.1399687500146	0.855433779898998\\
10.1418438739576	0.855433253346963\\
10.1437189975406	0.855432727044996\\
10.1455941207635	0.855432200992939\\
10.1474692436265	0.855431675190632\\
10.14934436613	0.855431149637917\\
10.1512194882739	0.855430624334635\\
10.1530946100585	0.855430099280627\\
10.1549697314839	0.855429574475735\\
10.1568448525504	0.855429049919802\\
10.158719973258	0.855428525612668\\
10.160595093607	0.855428001554176\\
10.1624702135975	0.855427477744168\\
10.1643453332297	0.855426954182487\\
10.1662204525038	0.855426430868974\\
10.1680955714198	0.855425907803472\\
10.1699706899781	0.855425384985824\\
10.1718458081787	0.855424862415873\\
10.1737209260218	0.855424340093462\\
10.1755960435076	0.855423818018432\\
10.1774711606363	0.855423296190629\\
10.179346277408	0.855422774609894\\
10.1812213938229	0.855422253276072\\
10.1830965098811	0.855421732189006\\
10.1849716255828	0.855421211348539\\
10.1868467409283	0.855420690754515\\
10.1887218559176	0.855420170406779\\
10.1905969705509	0.855419650305174\\
10.1924720848284	0.855419130449544\\
10.1943471987502	0.855418610839734\\
10.1962223123166	0.855418091475587\\
10.1980974255276	0.855417572356949\\
10.1999725383835	0.855417053483665\\
10.2018476508844	0.855416534855578\\
10.2037227630305	0.855416016472534\\
10.205597874822	0.855415498334378\\
10.2074729862589	0.855414980440954\\
10.2093480973415	0.855414462792109\\
10.21122320807	0.855413945387687\\
10.2130983184445	0.855413428227535\\
10.2149734284651	0.855412911311497\\
10.2168485381321	0.855412394639419\\
10.2187236474456	0.855411878211148\\
10.2205987564057	0.855411362026529\\
10.2224738650127	0.855410846085408\\
10.2243489732666	0.855410330387632\\
10.2262240811677	0.855409814933046\\
10.2280991887162	0.855409299721499\\
10.2299742959121	0.855408784752835\\
10.2318494027557	0.855408270026902\\
10.2337245092471	0.855407755543546\\
10.2355996153864	0.855407241302615\\
10.2374747211739	0.855406727303956\\
10.2393498266097	0.855406213547415\\
10.241224931694	0.85540570003284\\
10.2431000364269	0.855405186760079\\
10.2449751408086	0.85540467372898\\
10.2468502448393	0.855404160939389\\
10.248725348519	0.855403648391154\\
10.2506004518481	0.855403136084124\\
10.2524755548266	0.855402624018147\\
10.2543506574547	0.855402112193071\\
10.2562257597326	0.855401600608743\\
10.2581008616604	0.855401089265014\\
10.2599759632383	0.85540057816173\\
10.2618510644665	0.85540006729874\\
10.2637261653451	0.855399556675894\\
10.2656012658742	0.855399046293041\\
10.2674763660541	0.855398536150028\\
10.269351465885	0.855398026246706\\
10.2712265653669	0.855397516582924\\
10.2731016645	0.855397007158531\\
10.2749767632845	0.855396497973376\\
10.2768518617205	0.855395989027309\\
10.2787269598083	0.855395480320181\\
10.280602057548	0.855394971851839\\
10.2824771549397	0.855394463622136\\
10.2843522519835	0.85539395563092\\
10.2862273486798	0.855393447878042\\
10.2881024450286	0.855392940363352\\
10.28997754103	0.8553924330867\\
10.2918526366843	0.855391926047938\\
10.2937277319916	0.855391419246916\\
10.295602826952	0.855390912683484\\
10.2974779215658	0.855390406357494\\
10.299353015833	0.855389900268796\\
10.3012281097539	0.855389394417242\\
10.3031032033286	0.855388888802683\\
10.3049782965573	0.85538838342497\\
10.3068533894401	0.855387878283955\\
10.3087284819771	0.855387373379489\\
10.3106035741686	0.855386868711424\\
10.3124786660148	0.855386364279613\\
10.3143537575156	0.855385860083906\\
10.3162288486714	0.855385356124156\\
10.3181039394823	0.855384852400215\\
10.3199790299484	0.855384348911935\\
10.3218541200699	0.855383845659169\\
10.323729209847	0.85538334264177\\
10.3256042992798	0.855382839859589\\
10.3274793883684	0.855382337312481\\
10.3293544771131	0.855381835000297\\
10.331229565514	0.85538133292289\\
10.3331046535713	0.855380831080115\\
10.334979741285	0.855380329471823\\
10.3368548286554	0.855379828097869\\
10.3387299156827	0.855379326958105\\
10.3406050023669	0.855378826052386\\
10.3424800887083	0.855378325380565\\
10.3443551747069	0.855377824942495\\
10.3462302603631	0.855377324738032\\
10.3481053456768	0.855376824767028\\
10.3499804306483	0.855376325029338\\
10.3518555152778	0.855375825524816\\
10.3537305995653	0.855375326253317\\
10.3556056835111	0.855374827214695\\
10.3574807671153	0.855374328408805\\
10.359355850378	0.8553738298355\\
10.3612309332995	0.855373331494638\\
10.3631060158798	0.855372833386071\\
10.3649810981192	0.855372335509655\\
10.3668561800178	0.855371837865246\\
10.3687312615757	0.855371340452698\\
10.3706063427931	0.855370843271867\\
10.3724814236701	0.855370346322609\\
10.374356504207	0.855369849604779\\
10.3762315844039	0.855369353118233\\
10.3781066642608	0.855368856862827\\
10.3799817437781	0.855368360838417\\
10.3818568229558	0.855367865044858\\
10.3837319017941	0.855367369482007\\
10.3856069802931	0.855366874149721\\
10.3874820584531	0.855366379047855\\
10.3893571362741	0.855365884176266\\
10.3912322137563	0.855365389534811\\
10.3931072908999	0.855364895123347\\
10.394982367705	0.85536440094173\\
10.3968574441719	0.855363906989818\\
10.3987325203005	0.855363413267467\\
10.4006075960912	0.855362919774535\\
10.402482671544	0.855362426510879\\
10.4043577466591	0.855361933476356\\
10.4062328214366	0.855361440670824\\
10.4081078958768	0.855360948094141\\
10.4099829699797	0.855360455746164\\
10.4118580437456	0.855359963626751\\
10.4137331171745	0.855359471735761\\
10.4156081902667	0.855358980073051\\
10.4174832630222	0.855358488638479\\
10.4193583354413	0.855357997431904\\
10.4212334075241	0.855357506453184\\
10.4231084792708	0.855357015702178\\
10.4249835506814	0.855356525178744\\
10.4268586217562	0.855356034882741\\
10.4287336924953	0.855355544814028\\
10.4306087628988	0.855355054972464\\
10.432483832967	0.855354565357908\\
10.4343589027	0.855354075970219\\
10.4362339720978	0.855353586809257\\
10.4381090411608	0.85535309787488\\
10.439984109889	0.855352609166949\\
10.4418591782825	0.855352120685323\\
10.4437342463416	0.855351632429861\\
10.4456093140664	0.855351144400424\\
10.447484381457	0.855350656596872\\
10.4493594485136	0.855350169019064\\
10.4512345152363	0.855349681666861\\
10.4531095816254	0.855349194540123\\
10.4549846476809	0.85534870763871\\
10.456859713403	0.855348220962484\\
10.4587347787919	0.855347734511303\\
10.4606098438476	0.85534724828503\\
10.4624849085704	0.855346762283525\\
10.4643599729605	0.855346276506649\\
10.4662350370179	0.855345790954263\\
10.4681101007428	0.855345305626228\\
10.4699851641354	0.855344820522405\\
10.4718602271957	0.855344335642656\\
10.4737352899241	0.855343850986843\\
10.4756103523206	0.855343366554826\\
10.4774854143853	0.855342882346467\\
10.4793604761185	0.855342398361629\\
10.4812355375202	0.855341914600173\\
10.4831105985907	0.855341431061961\\
10.4849856593301	0.855340947746855\\
10.4868607197384	0.855340464654718\\
10.488735779816	0.855339981785411\\
10.4906108395628	0.855339499138798\\
10.4924858989792	0.855339016714741\\
10.4943609580652	0.855338534513101\\
10.4962360168209	0.855338052533744\\
10.4981110752466	0.85533757077653\\
10.4999861333424	0.855337089241322\\
10.5018611911084	0.855336607927985\\
10.5037362485447	0.855336126836382\\
10.5056113056516	0.855335645966374\\
10.5074863624292	0.855335165317827\\
10.5093614188776	0.855334684890603\\
10.511236474997	0.855334204684565\\
10.5131115307875	0.855333724699579\\
10.5149865862493	0.855333244935507\\
10.5168616413825	0.855332765392213\\
10.5187366961873	0.855332286069562\\
10.5206117506638	0.855331806967417\\
10.5224868048122	0.855331328085643\\
10.5243618586326	0.855330849424104\\
10.5262369121252	0.855330370982664\\
10.5281119652901	0.855329892761188\\
10.5299870181274	0.855329414759542\\
10.5318620706374	0.855328936977588\\
10.5337371228202	0.855328459415193\\
10.5356121746758	0.855327982072221\\
10.5374872262046	0.855327504948537\\
10.5393622774065	0.855327028044007\\
10.5412373282818	0.855326551358496\\
10.5431123788306	0.855326074891869\\
10.5449874290531	0.855325598643991\\
10.5468624789493	0.855325122614728\\
10.5487375285195	0.855324646803947\\
10.5506125777638	0.855324171211512\\
10.5524876266824	0.85532369583729\\
10.5543626752753	0.855323220681146\\
10.5562377235428	0.855322745742947\\
10.558112771485	0.855322271022559\\
10.559987819102	0.855321796519848\\
10.561862866394	0.85532132223468\\
10.5637379133611	0.855320848166923\\
10.5656129600035	0.855320374316442\\
10.5674880063214	0.855319900683105\\
10.5693630523148	0.855319427266778\\
10.5712380979839	0.855318954067328\\
10.5731131433289	0.855318481084622\\
10.5749881883498	0.855318008318528\\
10.576863233047	0.855317535768912\\
10.5787382774204	0.855317063435642\\
10.5806133214703	0.855316591318585\\
10.5824883651968	0.85531611941761\\
10.5843634086001	0.855315647732582\\
10.5862384516802	0.855315176263371\\
10.5881134944373	0.855314705009844\\
10.5899885368717	0.855314233971869\\
10.5918635789833	0.855313763149314\\
10.5937386207725	0.855313292542048\\
10.5956136622392	0.855312822149937\\
10.5974887033837	0.855312351972852\\
10.5993637442061	0.85531188201066\\
10.6012387847066	0.855311412263229\\
10.6031138248853	0.855310942730429\\
10.6049888647423	0.855310473412129\\
10.6068639042778	0.855310004308196\\
10.6087389434919	0.8553095354185\\
10.6106139823848	0.855309066742911\\
10.6124890209567	0.855308598281296\\
10.6143640592076	0.855308130033526\\
10.6162390971377	0.85530766199947\\
10.6181141347471	0.855307194178997\\
10.6199891720361	0.855306726571977\\
10.6218642090047	0.855306259178279\\
10.6237392456531	0.855305791997774\\
10.6256142819815	0.855305325030331\\
10.6274893179899	0.855304858275819\\
10.6293643536785	0.85530439173411\\
10.6312393890475	0.855303925405074\\
10.6331144240971	0.855303459288579\\
10.6349894588272	0.855302993384498\\
10.6368644932382	0.855302527692699\\
10.6387395273301	0.855302062213055\\
10.6406145611031	0.855301596945435\\
10.6424895945573	0.85530113188971\\
10.6443646276929	0.855300667045752\\
10.64623966051	0.855300202413431\\
10.6481146930088	0.855299737992617\\
10.6499897251894	0.855299273783183\\
10.6518647570519	0.855298809785\\
10.6537397885965	0.855298345997938\\
10.6556148198233	0.855297882421869\\
10.6574898507325	0.855297419056665\\
10.6593648813242	0.855296955902198\\
10.6612399115986	0.855296492958339\\
10.6631149415557	0.85529603022496\\
10.6649899711959	0.855295567701932\\
10.6668650005191	0.855295105389129\\
10.6687400295255	0.855294643286422\\
10.6706150582153	0.855294181393683\\
10.6724900865886	0.855293719710784\\
10.6743651146455	0.855293258237599\\
10.6762401423863	0.855292796973999\\
10.678115169811	0.855292335919858\\
10.6799901969197	0.855291875075047\\
10.6818652237127	0.85529141443944\\
10.6837402501901	0.855290954012909\\
10.6856152763519	0.855290493795328\\
10.6874903021984	0.85529003378657\\
10.6893653277297	0.855289573986507\\
10.6912403529459	0.855289114395014\\
10.6931153778472	0.855288655011963\\
10.6949904024336	0.855288195837228\\
10.6968654267055	0.855287736870682\\
10.6987404506628	0.8552872781122\\
10.7006154743058	0.855286819561655\\
10.7024904976345	0.855286361218921\\
10.7043655206491	0.855285903083872\\
10.7062405433498	0.855285445156382\\
10.7081155657367	0.855284987436324\\
10.7099905878099	0.855284529923574\\
10.7118656095696	0.855284072618006\\
10.7137406310159	0.855283615519494\\
10.715615652149	0.855283158627912\\
10.7174906729689	0.855282701943136\\
10.7193656934759	0.85528224546504\\
10.7212407136701	0.855281789193498\\
10.7231157335516	0.855281333128387\\
10.7249907531205	0.85528087726958\\
10.726865772377	0.855280421616953\\
10.7287407913213	0.85527996617038\\
10.7306158099534	0.855279510929739\\
10.7324908282736	0.855279055894903\\
10.7343658462819	0.855278601065748\\
10.7362408639785	0.85527814644215\\
10.7381158813635	0.855277692023985\\
10.7399908984371	0.855277237811128\\
10.7418659151994	0.855276783803455\\
10.7437409316505	0.855276330000842\\
10.7456159477906	0.855275876403166\\
10.7474909636199	0.855275423010301\\
10.7493659791384	0.855274969822125\\
10.7512409943463	0.855274516838514\\
10.7531160092437	0.855274064059345\\
10.7549910238309	0.855273611484493\\
10.7568660381078	0.855273159113835\\
10.7587410520747	0.855272706947249\\
10.7606160657317	0.855272254984611\\
10.7624910790789	0.855271803225797\\
10.7643660921165	0.855271351670685\\
10.7662411048446	0.855270900319152\\
10.7681161172633	0.855270449171076\\
10.7699911293729	0.855269998226332\\
10.7718661411733	0.8552695474848\\
10.7737411526648	0.855269096946355\\
10.7756161638475	0.855268646610876\\
10.7774911747215	0.855268196478241\\
10.779366185287	0.855267746548327\\
10.7812411955441	0.855267296821011\\
10.7831162054929	0.855266847296173\\
10.7849912151337	0.855266397973689\\
10.7868662244664	0.855265948853438\\
10.7887412334913	0.855265499935298\\
10.7906162422085	0.855265051219148\\
10.7924912506181	0.855264602704865\\
10.7943662587203	0.855264154392329\\
10.7962412665151	0.855263706281417\\
10.7981162740029	0.855263258372009\\
10.7999912811835	0.855262810663983\\
10.8018662880573	0.855262363157219\\
10.8037412946244	0.855261915851594\\
10.8056163008848	0.855261468746988\\
10.8074913068388	0.85526102184328\\
10.8093663124864	0.85526057514035\\
10.8112413178278	0.855260128638076\\
10.8131163228631	0.855259682336338\\
10.8149913275925	0.855259236235016\\
10.816866332016	0.855258790333988\\
10.8187413361339	0.855258344633135\\
10.8206163399463	0.855257899132337\\
10.8224913434533	0.855257453831473\\
10.824366346655	0.855257008730423\\
10.8262413495516	0.855256563829067\\
10.8281163521433	0.855256119127286\\
10.82999135443	0.855255674624959\\
10.8318663564121	0.855255230321967\\
10.8337413580895	0.85525478621819\\
10.8356163594625	0.855254342313509\\
10.8374913605312	0.855253898607803\\
10.8393663612957	0.855253455100955\\
10.8412413617562	0.855253011792844\\
10.8431163619127	0.855252568683352\\
10.8449913617655	0.855252125772359\\
10.8468663613146	0.855251683059746\\
10.8487413605603	0.855251240545395\\
10.8506163595025	0.855250798229186\\
10.8524913581415	0.855250356111002\\
10.8543663564774	0.855249914190722\\
10.8562413545104	0.855249472468229\\
10.8581163522405	0.855249030943405\\
10.8599913496679	0.85524858961613\\
10.8618663467927	0.855248148486287\\
10.8637413436151	0.855247707553758\\
10.8656163401352	0.855247266818423\\
10.8674913363531	0.855246826280166\\
10.869366332269	0.855246385938869\\
10.871241327883	0.855245945794413\\
10.8731163231952	0.85524550584668\\
10.8749913182058	0.855245066095554\\
10.8768663129149	0.855244626540917\\
10.8787413073226	0.85524418718265\\
10.880616301429	0.855243748020638\\
10.8824912952344	0.855243309054761\\
10.8843662887388	0.855242870284904\\
10.8862412819424	0.855242431710949\\
10.8881162748452	0.855241993332779\\
10.8899912674475	0.855241555150276\\
10.8918662597494	0.855241117163325\\
10.8937412517509	0.855240679371809\\
10.8956162434523	0.85524024177561\\
10.8974912348536	0.855239804374612\\
10.899366225955	0.855239367168699\\
10.9012412167566	0.855238930157754\\
10.9031162072586	0.85523849334166\\
10.9049911974611	0.855238056720303\\
10.9068661873642	0.855237620293564\\
10.908741176968	0.855237184061329\\
10.9106161662727	0.855236748023482\\
10.9124911552785	0.855236312179905\\
10.9143661439854	0.855235876530485\\
10.9162411323935	0.855235441075104\\
10.9181161205031	0.855235005813647\\
10.9199911083142	0.855234570745999\\
10.9218660958269	0.855234135872044\\
10.9237410830415	0.855233701191667\\
10.925616069958	0.855233266704752\\
10.9274910565766	0.855232832411184\\
10.9293660428973	0.855232398310849\\
10.9312410289204	0.855231964403631\\
10.933116014646	0.855231530689414\\
10.9349910000741	0.855231097168086\\
10.9368659852049	0.855230663839529\\
10.9387409700386	0.85523023070363\\
10.9406159545753	0.855229797760275\\
10.9424909388151	0.855229365009348\\
10.9443659227581	0.855228932450735\\
10.9462409064045	0.855228500084322\\
10.9481158897544	0.855228067909994\\
10.9499908728079	0.855227635927638\\
10.9518658555652	0.855227204137139\\
10.9537408380264	0.855226772538383\\
10.9556158201916	0.855226341131256\\
10.9574908020609	0.855225909915645\\
10.9593657836345	0.855225478891436\\
10.9612407649126	0.855225048058514\\
10.9631157458951	0.855224617416767\\
10.9649907265824	0.855224186966081\\
10.9668657069744	0.855223756706342\\
10.9687406870714	0.855223326637438\\
10.9706156668734	0.855222896759254\\
10.9724906463806	0.855222467071678\\
10.9743656255931	0.855222037574596\\
10.9762406045111	0.855221608267897\\
10.9781155831346	0.855221179151465\\
10.9799905614638	0.85522075022519\\
10.9818655394989	0.855220321488958\\
10.9837405172399	0.855219892942656\\
10.9856154946871	0.855219464586172\\
10.9874904718404	0.855219036419393\\
10.9893654487001	0.855218608442207\\
10.9912404252662	0.855218180654502\\
10.993115401539	0.855217753056165\\
10.9949903775185	0.855217325647084\\
10.9968653532048	0.855216898427148\\
10.9987403285981	0.855216471396243\\
11.0006153036986	0.855216044554259\\
11.0024902785063	0.855215617901083\\
11.0043652530213	0.855215191436603\\
11.0062402272439	0.855214765160708\\
11.0081152011741	0.855214339073287\\
11.009990174812	0.855213913174228\\
11.0118651481579	0.855213487463419\\
11.0137401212117	0.855213061940749\\
11.0156150939737	0.855212636606107\\
11.0174900664439	0.855212211459381\\
11.0193650386225	0.855211786500461\\
11.0212400105097	0.855211361729236\\
11.0231149821055	0.855210937145594\\
11.02498995341	0.855210512749425\\
11.0268649244235	0.855210088540618\\
11.028739895146	0.855209664519063\\
11.0306148655776	0.855209240684648\\
11.0324898357186	0.855208817037263\\
11.0343648055689	0.855208393576798\\
11.0362397751288	0.855207970303143\\
11.0381147443983	0.855207547216187\\
11.0399897133776	0.85520712431582\\
11.0418646820668	0.855206701601932\\
11.0437396504661	0.855206279074412\\
11.0456146185756	0.855205856733152\\
11.0474895863953	0.855205434578041\\
11.0493645539255	0.855205012608969\\
11.0512395211662	0.855204590825826\\
11.0531144881175	0.855204169228504\\
11.0549894547797	0.855203747816892\\
11.0568644211528	0.855203326590881\\
11.0587393872369	0.855202905550361\\
11.0606143530323	0.855202484695224\\
11.0624893185389	0.85520206402536\\
11.0643642837569	0.85520164354066\\
11.0662392486865	0.855201223241015\\
11.0681142133278	0.855200803126316\\
11.0699891776809	0.855200383196454\\
11.0718641417459	0.85519996345132\\
11.0737391055229	0.855199543890805\\
11.0756140690122	0.855199124514802\\
11.0774890322137	0.855198705323201\\
11.0793639951277	0.855198286315894\\
11.0812389577542	0.855197867492772\\
11.0831139200934	0.855197448853727\\
11.0849888821454	0.855197030398651\\
11.0868638439103	0.855196612127435\\
11.0887388053883	0.855196194039973\\
11.0906137665795	0.855195776136154\\
11.0924887274839	0.855195358415873\\
11.0943636881018	0.85519494087902\\
11.0962386484333	0.855194523525488\\
11.0981136084784	0.85519410635517\\
11.0999885682373	0.855193689367957\\
11.1018635277101	0.855193272563742\\
11.103738486897	0.855192855942419\\
11.105613445798	0.855192439503878\\
11.1074884044134	0.855192023248014\\
11.1093633627431	0.855191607174718\\
11.1112383207875	0.855191191283884\\
11.1131132785464	0.855190775575405\\
11.1149882360202	0.855190360049173\\
11.1168631932089	0.855189944705082\\
11.1187381501126	0.855189529543025\\
11.1206131067314	0.855189114562896\\
11.1224880630656	0.855188699764586\\
11.1243630191152	0.855188285147991\\
11.1262379748803	0.855187870713003\\
11.128112930361	0.855187456459516\\
11.1299878855575	0.855187042387424\\
11.1318628404699	0.85518662849662\\
11.1337377950984	0.855186214786999\\
11.135612749443	0.855185801258453\\
11.1374877035038	0.855185387910877\\
11.1393626572811	0.855184974744166\\
11.1412376107748	0.855184561758213\\
11.1431125639852	0.855184148952912\\
11.1449875169124	0.855183736328157\\
11.1468624695564	0.855183323883844\\
11.1487374219175	0.855182911619866\\
11.1506123739956	0.855182499536118\\
11.152487325791	0.855182087632494\\
11.1543622773038	0.85518167590889\\
11.1562372285341	0.8551812643652\\
11.158112179482	0.855180853001318\\
11.1599871301476	0.85518044181714\\
11.1618620805311	0.85518003081256\\
11.1637370306325	0.855179619987474\\
11.1656119804521	0.855179209341777\\
11.1674869299899	0.855178798875364\\
11.169361879246	0.85517838858813\\
11.1712368282206	0.85517797847997\\
11.1731117769138	0.855177568550781\\
11.1749867253257	0.855177158800457\\
11.1768616734564	0.855176749228894\\
11.1787366213061	0.855176339835988\\
11.1806115688748	0.855175930621634\\
11.1824865161627	0.855175521585728\\
11.18436146317	0.855175112728167\\
11.1862364098967	0.855174704048846\\
11.1881113563429	0.855174295547661\\
11.1899863025088	0.855173887224508\\
11.1918612483946	0.855173479079284\\
11.1937361940002	0.855173071111884\\
11.1956111393259	0.855172663322206\\
11.1974860843717	0.855172255710145\\
11.1993610291378	0.855171848275598\\
11.2012359736244	0.855171441018462\\
11.2031109178314	0.855171033938633\\
11.2049858617591	0.855170627036008\\
11.2068608054075	0.855170220310483\\
11.2087357487768	0.855169813761956\\
11.2106106918672	0.855169407390324\\
11.2124856346786	0.855169001195483\\
11.2143605772113	0.855168595177331\\
11.2162355194654	0.855168189335765\\
11.2181104614409	0.855167783670682\\
11.219985403138	0.855167378181979\\
11.2218603445569	0.855166972869554\\
11.2237352856976	0.855166567733305\\
11.2256102265602	0.855166162773128\\
11.227485167145	0.855165757988922\\
11.2293601074519	0.855165353380584\\
11.2312350474811	0.855164948948012\\
11.2331099872328	0.855164544691103\\
11.2349849267071	0.855164140609756\\
11.236859865904	0.855163736703869\\
11.2387348048237	0.85516333297334\\
11.2406097434663	0.855162929418067\\
11.242484681832	0.855162526037948\\
11.2443596199208	0.855162122832881\\
11.2462345577329	0.855161719802765\\
11.2481094952684	0.855161316947498\\
11.2499844325273	0.855160914266979\\
11.2518593695099	0.855160511761106\\
11.2537343062163	0.855160109429778\\
11.2556092426465	0.855159707272894\\
11.2574841788007	0.855159305290353\\
11.259359114679	0.855158903482052\\
11.2612340502815	0.855158501847893\\
11.2631089856083	0.855158100387772\\
11.2649839206596	0.85515769910159\\
11.2668588554355	0.855157297989246\\
11.268733789936	0.855156897050639\\
11.2706087241614	0.855156496285668\\
11.2724836581117	0.855156095694233\\
11.274358591787	0.855155695276232\\
11.2762335251875	0.855155295031567\\
11.2781084583133	0.855154894960135\\
11.2799833911645	0.855154495061838\\
11.2818583237412	0.855154095336574\\
11.2837332560435	0.855153695784244\\
11.2856081880715	0.855153296404748\\
11.2874831198255	0.855152897197984\\
11.2893580513054	0.855152498163855\\
11.2912329825114	0.855152099302259\\
11.2931079134436	0.855151700613097\\
11.2949828441022	0.855151302096268\\
11.2968577744872	0.855150903751675\\
11.2987327045988	0.855150505579216\\
11.300607634437	0.855150107578792\\
11.3024825640021	0.855149709750305\\
11.3043574932941	0.855149312093654\\
11.3062324223131	0.85514891460874\\
11.3081073510593	0.855148517295464\\
11.3099822795327	0.855148120153727\\
11.3118572077335	0.855147723183429\\
11.3137321356619	0.855147326384473\\
11.3156070633178	0.855146929756758\\
11.3174819907015	0.855146533300186\\
11.319356917813	0.855146137014659\\
11.3212318446525	0.855145740900076\\
11.32310677122	0.855145344956341\\
11.3249816975158	0.855144949183354\\
11.3268566235399	0.855144553581016\\
11.3287315492924	0.85514415814923\\
11.3306064747734	0.855143762887896\\
11.3324813999832	0.855143367796917\\
11.3343563249216	0.855142972876195\\
11.336231249589	0.855142578125631\\
11.3381061739854	0.855142183545127\\
11.3399810981109	0.855141789134585\\
11.3418560219657	0.855141394893907\\
11.3437309455498	0.855141000822995\\
11.3456058688633	0.855140606921752\\
11.3474807919065	0.85514021319008\\
11.3493557146793	0.855139819627881\\
11.351230637182	0.855139426235057\\
11.3531055594146	0.855139033011511\\
11.3549804813772	0.855138639957146\\
11.35685540307	0.855138247071864\\
11.3587303244931	0.855137854355568\\
11.3606052456465	0.855137461808161\\
11.3624801665305	0.855137069429545\\
11.3643550871451	0.855136677219624\\
11.3662300074904	0.8551362851783\\
11.3681049275665	0.855135893305476\\
11.3699798473736	0.855135501601056\\
11.3718547669118	0.855135110064942\\
11.3737296861812	0.855134718697039\\
11.3756046051819	0.855134327497248\\
11.3774795239139	0.855133936465475\\
11.3793544423776	0.855133545601621\\
11.3812293605728	0.855133154905592\\
11.3831042784999	0.855132764377289\\
11.3849791961588	0.855132374016618\\
11.3868541135497	0.855131983823481\\
11.3887290306727	0.855131593797782\\
11.3906039475279	0.855131203939426\\
11.3924788641154	0.855130814248317\\
11.3943537804354	0.855130424724358\\
11.3962286964879	0.855130035367453\\
11.3981036122731	0.855129646177506\\
11.3999785277911	0.855129257154423\\
11.4018534430419	0.855128868298107\\
11.4037283580258	0.855128479608462\\
11.4056032727428	0.855128091085393\\
11.407478187193	0.855127702728804\\
11.4093531013766	0.8551273145386\\
11.4112280152937	0.855126926514686\\
11.4131029289443	0.855126538656967\\
11.4149778423286	0.855126150965346\\
11.4168527554467	0.855125763439729\\
11.4187276682987	0.855125376080021\\
11.4206025808847	0.855124988886127\\
11.4224774932049	0.855124601857952\\
11.4243524052593	0.855124214995401\\
11.4262273170481	0.855123828298379\\
11.4281022285714	0.855123441766792\\
11.4299771398292	0.855123055400544\\
11.4318520508218	0.855122669199541\\
11.4337269615491	0.855122283163689\\
11.4356018720114	0.855121897292893\\
11.4374767822087	0.855121511587059\\
11.4393516921412	0.855121126046092\\
11.4412266018089	0.855120740669898\\
11.443101511212	0.855120355458383\\
11.4449764203506	0.855119970411453\\
11.4468513292247	0.855119585529013\\
11.4487262378346	0.85511920081097\\
11.4506011461803	0.85511881625723\\
11.4524760542619	0.855118431867699\\
11.4543509620796	0.855118047642283\\
11.4562258696334	0.855117663580888\\
11.4581007769235	0.855117279683421\\
11.4599756839499	0.855116895949788\\
11.4618505907129	0.855116512379896\\
11.4637254972124	0.85511612897365\\
11.4656004034487	0.855115745730958\\
11.4674753094217	0.855115362651727\\
11.4693502151317	0.855114979735862\\
11.4712251205788	0.855114596983272\\
11.473100025763	0.855114214393862\\
11.4749749306844	0.85511383196754\\
11.4768498353432	0.855113449704213\\
11.4787247397395	0.855113067603787\\
11.4805996438735	0.855112685666171\\
11.4824745477451	0.85511230389127\\
11.4843494513545	0.855111922278993\\
11.4862243547019	0.855111540829246\\
11.4880992577873	0.855111159541938\\
11.4899741606108	0.855110778416975\\
11.4918490631726	0.855110397454265\\
11.4937239654728	0.855110016653716\\
11.4955988675114	0.855109636015235\\
11.4974737692887	0.855109255538731\\
11.4993486708046	0.85510887522411\\
11.5012235720593	0.855108495071281\\
11.503098473053	0.855108115080152\\
11.5049733737857	0.85510773525063\\
11.5068482742575	0.855107355582624\\
11.5087231744686	0.855106976076042\\
11.510598074419	0.855106596730792\\
11.5124729741089	0.855106217546782\\
11.5143478735384	0.855105838523921\\
11.5162227727076	0.855105459662117\\
11.5180976716165	0.855105080961278\\
11.5199725702654	0.855104702421313\\
11.5218474686543	0.855104324042131\\
11.5237223667833	0.85510394582364\\
11.5255972646525	0.855103567765749\\
11.5274721622621	0.855103189868367\\
11.5293470596121	0.855102812131402\\
11.5312219567027	0.855102434554763\\
11.5330968535339	0.85510205713836\\
11.5349717501059	0.855101679882102\\
11.5368466464189	0.855101302785897\\
11.5387215424728	0.855100925849654\\
11.5405964382678	0.855100549073284\\
11.542471333804	0.855100172456695\\
11.5443462290815	0.855099795999797\\
11.5462211241005	0.855099419702498\\
11.548096018861	0.855099043564709\\
11.5499709133632	0.855098667586339\\
11.5518458076071	0.855098291767298\\
11.5537207015929	0.855097916107496\\
11.5555955953206	0.855097540606841\\
11.5574704887904	0.855097165265244\\
11.5593453820024	0.855096790082615\\
11.5612202749567	0.855096415058864\\
11.5630951676534	0.855096040193901\\
11.5649700600926	0.855095665487635\\
11.5668449522745	0.855095290939977\\
11.568719844199	0.855094916550838\\
11.5705947358664	0.855094542320126\\
11.5724696272767	0.855094168247754\\
11.5743445184301	0.855093794333631\\
11.5762194093267	0.855093420577667\\
11.5780942999665	0.855093046979773\\
11.5799691903496	0.85509267353986\\
11.5818440804763	0.855092300257838\\
11.5837189703465	0.855091927133618\\
11.5855938599605	0.855091554167111\\
11.5874687493182	0.855091181358228\\
11.5893436384198	0.855090808706879\\
11.5912185272655	0.855090436212975\\
11.5930934158552	0.855090063876428\\
11.5949683041892	0.855089691697148\\
11.5968431922676	0.855089319675047\\
11.5987180800904	0.855088947810036\\
11.6005929676577	0.855088576102026\\
11.6024678549697	0.855088204550929\\
11.6043427420264	0.855087833156655\\
11.606217628828	0.855087461919117\\
11.6080925153746	0.855087090838226\\
11.6099674016663	0.855086719913893\\
11.6118422877031	0.85508634914603\\
11.6137171734853	0.855085978534549\\
11.6155920590128	0.855085608079361\\
11.6174669442859	0.855085237780379\\
11.6193418293046	0.855084867637515\\
11.6212167140689	0.855084497650679\\
11.6230915985792	0.855084127819785\\
11.6249664828353	0.855083758144744\\
11.6268413668375	0.855083388625469\\
11.6287162505858	0.855083019261871\\
11.6305911340804	0.855082650053864\\
11.6324660173213	0.855082281001359\\
11.6343409003087	0.855081912104268\\
11.6362157830426	0.855081543362505\\
11.6380906655233	0.855081174775981\\
11.6399655477507	0.85508080634461\\
11.641840429725	0.855080438068303\\
11.6437153114462	0.855080069946974\\
11.6455901929146	0.855079701980536\\
11.6474650741301	0.855079334168901\\
11.649339955093	0.855078966511982\\
11.6512148358033	0.855078599009692\\
11.6530897162611	0.855078231661944\\
11.6549645964665	0.855077864468651\\
11.6568394764196	0.855077497429726\\
11.6587143561205	0.855077130545083\\
11.6605892355694	0.855076763814634\\
11.6624641147664	0.855076397238294\\
11.6643389937114	0.855076030815975\\
11.6662138724047	0.855075664547591\\
11.6680887508464	0.855075298433055\\
11.6699636290365	0.855074932472281\\
11.6718385069752	0.855074566665183\\
11.6737133846626	0.855074201011674\\
11.6755882620987	0.855073835511667\\
11.6774631392837	0.855073470165078\\
11.6793380162177	0.855073104971819\\
11.6812128929007	0.855072739931805\\
11.683087769333	0.855072375044949\\
11.6849626455145	0.855072010311166\\
11.6868375214454	0.855071645730369\\
11.6887123971259	0.855071281302474\\
11.6905872725559	0.855070917027393\\
11.6924621477356	0.855070552905042\\
11.6943370226652	0.855070188935334\\
11.6962118973446	0.855069825118185\\
11.6980867717741	0.855069461453508\\
11.6999616459537	0.855069097941219\\
11.7018365198836	0.855068734581231\\
11.7037113935637	0.855068371373459\\
11.7055862669943	0.855068008317819\\
11.7074611401754	0.855067645414224\\
11.7093360131072	0.85506728266259\\
11.7112108857897	0.855066920062832\\
11.7130857582231	0.855066557614864\\
11.7149606304074	0.855066195318602\\
11.7168355023427	0.85506583317396\\
11.7187103740292	0.855065471180854\\
11.720585245467	0.855065109339199\\
11.7224601166561	0.85506474764891\\
11.7243349875967	0.855064386109903\\
11.7262098582889	0.855064024722092\\
11.7280847287327	0.855063663485394\\
11.7299595989283	0.855063302399723\\
11.7318344688758	0.855062941464995\\
11.7337093385753	0.855062580681127\\
11.7355842080268	0.855062220048033\\
11.7374590772305	0.855061859565629\\
11.7393339461865	0.855061499233831\\
11.7412088148949	0.855061139052556\\
11.7430836833558	0.855060779021718\\
11.7449585515693	0.855060419141233\\
11.7468334195354	0.855060059411019\\
11.7487082872544	0.85505969983099\\
11.7505831547263	0.855059340401063\\
11.7524580219511	0.855058981121155\\
11.7543328889291	0.85505862199118\\
11.7562077556603	0.855058263011057\\
11.7580826221448	0.8550579041807\\
11.7599574883826	0.855057545500027\\
11.761832354374	0.855057186968955\\
11.763707220119	0.855056828587398\\
11.7655820856178	0.855056470355275\\
11.7674569508703	0.855056112272501\\
11.7693318158768	0.855055754338995\\
11.7712066806372	0.855055396554671\\
11.7730815451518	0.855055038919447\\
11.7749564094207	0.855054681433241\\
11.7768312734438	0.855054324095968\\
11.7787061372213	0.855053966907547\\
11.7805810007534	0.855053609867893\\
11.7824558640401	0.855053252976925\\
11.7843307270816	0.855052896234559\\
11.7862055898779	0.855052539640713\\
11.7880804524291	0.855052183195303\\
11.7899553147353	0.855051826898248\\
11.7918301767966	0.855051470749465\\
11.7937050386132	0.85505111474887\\
11.7955799001852	0.855050758896382\\
11.7974547615125	0.855050403191919\\
11.7993296225954	0.855050047635397\\
11.8012044834339	0.855049692226736\\
11.8030793440282	0.855049336965851\\
11.8049542043783	0.855048981852662\\
11.8068290644843	0.855048626887085\\
11.8087039243464	0.85504827206904\\
11.8105787839646	0.855047917398443\\
11.812453643339	0.855047562875214\\
11.8143285024698	0.85504720849927\\
11.816203361357	0.855046854270529\\
11.8180782200008	0.855046500188909\\
11.8199530784012	0.85504614625433\\
11.8218279365583	0.855045792466708\\
11.8237027944723	0.855045438825963\\
11.8255776521432	0.855045085332013\\
11.8274525095711	0.855044731984776\\
11.8293273667562	0.855044378784171\\
11.8312022236985	0.855044025730117\\
11.8330770803982	0.855043672822532\\
11.8349519368553	0.855043320061335\\
11.8368267930699	0.855042967446445\\
11.8387016490422	0.855042614977781\\
11.8405765047722	0.855042262655261\\
11.84245136026	0.855041910478805\\
11.8443262155058	0.855041558448331\\
11.8462010705095	0.855041206563758\\
11.8480759252715	0.855040854825007\\
11.8499507797916	0.855040503231995\\
11.8518256340701	0.855040151784642\\
11.853700488107	0.855039800482868\\
11.8555753419025	0.855039449326591\\
11.8574501954566	0.855039098315732\\
11.8593250487694	0.855038747450209\\
11.861199901841	0.855038396729942\\
11.8630747546716	0.855038046154851\\
11.8649496072611	0.855037695724855\\
11.8668244596098	0.855037345439874\\
11.8686993117177	0.855036995299828\\
11.870574163585	0.855036645304637\\
11.8724490152116	0.85503629545422\\
11.8743238665978	0.855035945748497\\
11.8761987177436	0.855035596187388\\
11.8780735686491	0.855035246770814\\
11.8799484193144	0.855034897498694\\
11.8818232697396	0.855034548370948\\
11.8836981199248	0.855034199387497\\
11.8855729698701	0.855033850548261\\
11.8874478195757	0.85503350185316\\
11.8893226690415	0.855033153302115\\
11.8911975182677	0.855032804895046\\
11.8930723672545	0.855032456631872\\
11.8949472160018	0.855032108512516\\
11.8968220645098	0.855031760536898\\
11.8986969127786	0.855031412704937\\
11.9005717608083	0.855031065016555\\
11.902446608599	0.855030717471673\\
11.9043214561508	0.855030370070211\\
11.9061963034637	0.85503002281209\\
11.908071150538	0.855029675697231\\
11.9099459973736	0.855029328725555\\
11.9118208439706	0.855028981896983\\
11.9136956903293	0.855028635211436\\
11.9155705364496	0.855028288668836\\
11.9174453823317	0.855027942269102\\
11.9193202279757	0.855027596012158\\
11.9211950733816	0.855027249897923\\
11.9230699185495	0.855026903926319\\
11.9249447634797	0.855026558097268\\
11.926819608172	0.855026212410691\\
11.9286944526267	0.855025866866509\\
11.9305692968439	0.855025521464645\\
11.9324441408236	0.855025176205019\\
11.934318984566	0.855024831087553\\
11.9361938280711	0.85502448611217\\
11.938068671339	0.85502414127879\\
11.9399435143698	0.855023796587336\\
11.9418183571637	0.855023452037729\\
11.9436931997207	0.855023107629892\\
11.9455680420409	0.855022763363747\\
11.9474428841244	0.855022419239215\\
11.9493177259714	0.855022075256218\\
11.9511925675818	0.855021731414679\\
11.9530674089559	0.85502138771452\\
11.9549422500936	0.855021044155663\\
11.9568170909952	0.855020700738031\\
11.9586919316606	0.855020357461545\\
11.96056677209	0.855020014326129\\
11.9624416122835	0.855019671331705\\
11.9643164522412	0.855019328478194\\
11.9661912919632	0.855018985765521\\
11.9680661314495	0.855018643193607\\
11.9699409707004	0.855018300762375\\
11.9718158097157	0.855017958471747\\
11.9736906484958	0.855017616321648\\
11.9755654870405	0.855017274311999\\
11.9774403253502	0.855016932442723\\
11.9793151634248	0.855016590713744\\
11.9811900012644	0.855016249124984\\
11.9830648388691	0.855015907676366\\
11.9849396762391	0.855015566367814\\
11.9868145133745	0.85501522519925\\
11.9886893502752	0.855014884170599\\
11.9905641869415	0.855014543281782\\
11.9924390233733	0.855014202532724\\
11.9943138595709	0.855013861923347\\
11.9961886955343	0.855013521453576\\
11.9980635312636	0.855013181123334\\
11.9999383667589	0.855012840932543\\
12.0018132020202	0.855012500881129\\
12.0036880370478	0.855012160969014\\
12.0055628718416	0.855011821196122\\
12.0074377064018	0.855011481562377\\
12.0093125407285	0.855011142067702\\
12.0111873748217	0.855010802712022\\
12.0130622086816	0.855010463495261\\
12.0149370423082	0.855010124417341\\
12.0168118757017	0.855009785478188\\
12.0186867088622	0.855009446677726\\
12.0205615417896	0.855009108015878\\
12.0224363744842	0.855008769492568\\
12.024311206946	0.855008431107722\\
12.0261860391751	0.855008092861262\\
12.0280608711717	0.855007754753114\\
12.0299357029357	0.855007416783201\\
12.0318105344674	0.855007078951449\\
12.0336853657667	0.855006741257782\\
12.0355601968338	0.855006403702123\\
12.0374350276689	0.855006066284399\\
12.0393098582718	0.855005729004532\\
12.0411846886429	0.855005391862449\\
12.0430595187821	0.855005054858074\\
12.0449343486896	0.855004717991331\\
12.0468091783654	0.855004381262145\\
12.0486840078097	0.855004044670442\\
12.0505588370225	0.855003708216146\\
12.0524336660039	0.855003371899182\\
12.0543084947541	0.855003035719475\\
12.0561833232731	0.855002699676951\\
12.058058151561	0.855002363771534\\
12.0599329796179	0.855002028003149\\
12.0618078074439	0.855001692371723\\
12.0636826350391	0.855001356877179\\
12.0655574624035	0.855001021519444\\
12.0674322895374	0.855000686298443\\
12.0693071164407	0.855000351214101\\
12.0711819431136	0.855000016266344\\
12.0730567695562	0.854999681455097\\
12.0749315957685	0.854999346780286\\
12.0768064217506	0.854999012241837\\
12.0786812475027	0.854998677839675\\
12.0805560730248	0.854998343573727\\
12.082430898317	0.854998009443917\\
12.0843057233794	0.854997675450171\\
12.0861805482122	0.854997341592416\\
12.0880553728153	0.854997007870578\\
12.089930197189	0.854996674284582\\
12.0918050213332	0.854996340834355\\
12.0936798452481	0.854996007519823\\
12.0955546689337	0.854995674340911\\
12.0974294923903	0.854995341297547\\
12.0993043156178	0.854995008389656\\
12.1011791386163	0.854994675617164\\
12.1030539613859	0.854994342979998\\
12.1049287839268	0.854994010478085\\
12.1068036062391	0.854993678111351\\
12.1086784283227	0.854993345879721\\
12.1105532501778	0.854993013783124\\
12.1124280718046	0.854992681821485\\
12.114302893203	0.854992349994732\\
12.1161777143732	0.85499201830279\\
12.1180525353153	0.854991686745587\\
12.1199273560293	0.854991355323049\\
12.1218021765154	0.854991024035103\\
12.1236769967737	0.854990692881677\\
12.1255518168042	0.854990361862697\\
12.127426636607	0.854990030978089\\
12.1293014561823	0.854989700227782\\
12.13117627553	0.854989369611703\\
12.1330510946504	0.854989039129777\\
12.1349259135435	0.854988708781934\\
12.1368007322093	0.854988378568099\\
12.1386755506481	0.8549880484882\\
12.1405503688598	0.854987718542165\\
12.1424251868446	0.854987388729921\\
12.1443000046025	0.854987059051395\\
12.1461748221337	0.854986729506516\\
12.1480496394382	0.854986400095209\\
12.1499244565161	0.854986070817404\\
12.1517992733675	0.854985741673027\\
12.1536740899926	0.854985412662007\\
12.1555489063914	0.85498508378427\\
12.1574237225639	0.854984755039746\\
12.1592985385103	0.854984426428361\\
12.1611733542307	0.854984097950044\\
12.1630481697252	0.854983769604723\\
12.1649229849938	0.854983441392325\\
12.1667978000367	0.854983113312779\\
12.1686726148539	0.854982785366012\\
12.1705474294455	0.854982457551954\\
12.1724222438116	0.854982129870531\\
12.1742970579524	0.854981802321673\\
12.1761718718678	0.854981474905307\\
12.178046685558	0.854981147621362\\
12.1799214990231	0.854980820469766\\
12.1817963122632	0.854980493450448\\
12.1836711252783	0.854980166563336\\
12.1855459380685	0.854979839808359\\
12.187420750634	0.854979513185445\\
12.1892955629748	0.854979186694523\\
12.1911703750911	0.854978860335522\\
12.1930451869828	0.854978534108369\\
12.1949199986502	0.854978208012995\\
12.1967948100932	0.854977882049328\\
12.198669621312	0.854977556217296\\
12.2005444323066	0.854977230516829\\
12.2024192430773	0.854976904947856\\
12.2042940536239	0.854976579510305\\
12.2061688639467	0.854976254204106\\
12.2080436740457	0.854975929029187\\
12.209918483921	0.854975603985479\\
12.2117932935728	0.854975279072909\\
12.213668103001	0.854974954291408\\
12.2155429122058	0.854974629640905\\
12.2174177211873	0.854974305121328\\
12.2192925299455	0.854973980732608\\
12.2211673384806	0.854973656474674\\
12.2230421467926	0.854973332347455\\
12.2249169548816	0.854973008350881\\
12.2267917627477	0.854972684484881\\
12.2286665703911	0.854972360749385\\
12.2305413778117	0.854972037144323\\
12.2324161850098	0.854971713669625\\
12.2342909919853	0.85497139032522\\
12.2361657987383	0.854971067111038\\
12.238040605269	0.854970744027008\\
12.2399154115775	0.854970421073062\\
12.2417902176637	0.854970098249128\\
12.2436650235279	0.854969775555138\\
12.2455398291701	0.85496945299102\\
12.2474146345903	0.854969130556705\\
12.2492894397888	0.854968808252123\\
12.2511642447655	0.854968486077204\\
12.2530390495206	0.854968164031879\\
12.2549138540541	0.854967842116078\\
12.2567886583661	0.854967520329731\\
12.2586634624568	0.854967198672768\\
12.2605382663262	0.854966877145121\\
12.2624130699743	0.854966555746719\\
12.2642878734014	0.854966234477493\\
12.2661626766074	0.854965913337373\\
12.2680374795925	0.854965592326291\\
12.2699122823567	0.854965271444177\\
12.2717870849002	0.854964950690961\\
12.273661887223	0.854964630066575\\
12.2755366893252	0.854964309570949\\
12.2774114912069	0.854963989204013\\
12.2792862928682	0.8549636689657\\
12.2811610943092	0.85496334885594\\
12.2830358955299	0.854963028874663\\
12.2849106965305	0.854962709021801\\
12.286785497311	0.854962389297285\\
12.2886602978715	0.854962069701047\\
12.2905350982122	0.854961750233016\\
12.2924098983331	0.854961430893125\\
12.2942846982342	0.854961111681305\\
12.2961594979157	0.854960792597487\\
12.2980342973777	0.854960473641602\\
12.2999090966202	0.854960154813583\\
12.3017838956434	0.854959836113359\\
12.3036586944473	0.854959517540864\\
12.3055334930319	0.854959199096027\\
12.3074082913975	0.854958880778782\\
12.3092830895441	0.854958562589059\\
12.3111578874718	0.854958244526791\\
12.3130326851806	0.854957926591908\\
12.3149074826706	0.854957608784344\\
12.316782279942	0.854957291104029\\
12.3186570769948	0.854956973550895\\
12.3205318738291	0.854956656124875\\
12.322406670445	0.8549563388259\\
12.3242814668425	0.854956021653903\\
12.3261562630218	0.854955704608815\\
12.328031058983	0.854955387690568\\
12.3299058547261	0.854955070899096\\
12.3317806502513	0.854954754234329\\
12.3336554455585	0.8549544376962\\
12.335530240648	0.854954121284641\\
12.3374050355197	0.854953804999585\\
12.3392798301738	0.854953488840964\\
12.3411546246104	0.854953172808711\\
12.3430294188294	0.854952856902758\\
12.3449042128312	0.854952541123037\\
12.3467790066156	0.854952225469481\\
12.3486538001828	0.854951909942022\\
12.3505285935329	0.854951594540594\\
12.352403386666	0.854951279265129\\
12.3542781795822	0.854950964115559\\
12.3561529722814	0.854950649091818\\
12.358027764764	0.854950334193838\\
12.3599025570298	0.854950019421552\\
12.361777349079	0.854949704774893\\
12.3636521409117	0.854949390253794\\
12.365526932528	0.854949075858188\\
12.3674017239279	0.854948761588008\\
12.3692765151116	0.854948447443187\\
12.3711513060791	0.854948133423659\\
12.3730260968305	0.854947819529356\\
12.3749008873659	0.854947505760211\\
12.3767756776854	0.854947192116158\\
12.3786504677891	0.854946878597131\\
12.380525257677	0.854946565203062\\
12.3824000473493	0.854946251933885\\
12.3842748368059	0.854945938789534\\
12.3861496260471	0.854945625769941\\
12.3880244150729	0.854945312875042\\
12.3898992038834	0.854945000104768\\
12.3917739924786	0.854944687459053\\
12.3936487808587	0.854944374937832\\
12.3955235690237	0.854944062541039\\
12.3973983569737	0.854943750268605\\
12.3992731447088	0.854943438120467\\
12.4011479322291	0.854943126096557\\
12.4030227195347	0.85494281419681\\
12.4048975066257	0.854942502421158\\
12.4067722935021	0.854942190769537\\
12.408647080164	0.85494187924188\\
12.4105218666115	0.854941567838122\\
12.4123966528447	0.854941256558196\\
12.4142714388637	0.854940945402037\\
12.4161462246686	0.854940634369578\\
12.4180210102594	0.854940323460755\\
12.4198957956362	0.854940012675501\\
12.4217705807992	0.85493970201375\\
12.4236453657483	0.854939391475438\\
12.4255201504838	0.854939081060498\\
12.4273949350056	0.854938770768865\\
12.4292697193139	0.854938460600473\\
12.4311445034087	0.854938150555258\\
12.4330192872901	0.854937840633153\\
12.4348940709582	0.854937530834093\\
12.4367688544131	0.854937221158013\\
12.4386436376549	0.854936911604848\\
12.4405184206837	0.854936602174532\\
12.4423932034994	0.854936292867\\
12.4442679861024	0.854935983682188\\
12.4461427684925	0.854935674620029\\
12.4480175506699	0.854935365680459\\
12.4498923326347	0.854935056863413\\
12.4517671143869	0.854934748168826\\
12.4536418959267	0.854934439596633\\
12.4555166772541	0.854934131146769\\
12.4573914583692	0.854933822819169\\
12.4592662392721	0.854933514613768\\
12.4611410199629	0.854933206530503\\
12.4630158004417	0.854932898569307\\
12.4648905807084	0.854932590730116\\
12.4667653607634	0.854932283012866\\
12.4686401406065	0.854931975417492\\
12.4705149202379	0.85493166794393\\
12.4723896996577	0.854931360592114\\
12.4742644788659	0.854931053361981\\
12.4761392578627	0.854930746253467\\
12.4780140366481	0.854930439266506\\
12.4798888152222	0.854930132401034\\
12.4817635935851	0.854929825656988\\
12.4836383717369	0.854929519034302\\
12.4855131496776	0.854929212532914\\
12.4873879274074	0.854928906152757\\
12.4892627049262	0.85492859989377\\
12.4911374822343	0.854928293755886\\
12.4930122593317	0.854927987739043\\
12.4948870362184	0.854927681843176\\
12.4967618128946	0.854927376068222\\
12.4986365893603	0.854927070414116\\
12.5005113656157	0.854926764880794\\
12.5023861416607	0.854926459468194\\
12.5042609174955	0.85492615417625\\
12.5061356931202	0.8549258490049\\
12.5080104685348	0.854925543954079\\
12.5098852437394	0.854925239023725\\
12.5117600187342	0.854924934213772\\
12.5136347935191	0.854924629524158\\
12.5155095680943	0.85492432495482\\
12.5173843424599	0.854924020505693\\
12.5192591166159	0.854923716176715\\
12.5211338905624	0.854923411967822\\
12.5230086642995	0.85492310787895\\
12.5248834378273	0.854922803910036\\
12.5267582111459	0.854922500061017\\
12.5286329842553	0.85492219633183\\
12.5305077571556	0.854921892722412\\
12.532382529847	0.854921589232699\\
12.5342573023294	0.854921285862628\\
12.536132074603	0.854920982612136\\
12.5380068466679	0.854920679481161\\
12.5398816185241	0.854920376469639\\
12.5417563901717	0.854920073577507\\
12.5436311616108	0.854919770804702\\
12.5455059328415	0.854919468151163\\
12.5473807038638	0.854919165616824\\
12.5492554746779	0.854918863201625\\
12.5511302452839	0.854918560905502\\
12.5530050156817	0.854918258728392\\
12.5548797858715	0.854917956670234\\
12.5567545558534	0.854917654730963\\
12.5586293256274	0.854917352910518\\
12.5605040951936	0.854917051208837\\
12.5623788645522	0.854916749625856\\
12.5642536337031	0.854916448161513\\
12.5661284026466	0.854916146815746\\
12.5680031713825	0.854915845588493\\
12.5698779399111	0.854915544479691\\
12.5717527082325	0.854915243489277\\
12.5736274763466	0.854914942617191\\
12.5755022442536	0.854914641863368\\
12.5773770119535	0.854914341227748\\
12.5792517794465	0.854914040710269\\
12.5811265467326	0.854913740310867\\
12.5830013138119	0.854913440029481\\
12.5848760806845	0.85491313986605\\
12.5867508473504	0.85491283982051\\
12.5886256138098	0.854912539892801\\
12.5905003800627	0.85491224008286\\
12.5923751461092	0.854911940390626\\
12.5942499119494	0.854911640816036\\
12.5961246775833	0.85491134135903\\
12.5979994430111	0.854911042019544\\
12.5998742082328	0.854910742797519\\
12.6017489732485	0.854910443692891\\
12.6036237380582	0.854910144705599\\
12.6054985026622	0.854909845835583\\
12.6073732670604	0.85490954708278\\
12.6092480312529	0.854909248447129\\
12.6111227952398	0.854908949928568\\
12.6129975590212	0.854908651527036\\
12.6148723225971	0.854908353242472\\
12.6167470859677	0.854908055074815\\
12.618621849133	0.854907757024002\\
12.6204966120931	0.854907459089974\\
12.6223713748481	0.854907161272668\\
12.624246137398	0.854906863572024\\
12.626120899743	0.854906565987981\\
12.6279956618831	0.854906268520477\\
12.6298704238184	0.854905971169451\\
12.631745185549	0.854905673934843\\
12.6336199470749	0.854905376816592\\
12.6354947083963	0.854905079814636\\
12.6373694695131	0.854904782928915\\
12.6392442304256	0.854904486159367\\
12.6411189911338	0.854904189505933\\
12.6429937516377	0.854903892968552\\
12.6448685119374	0.854903596547162\\
12.646743272033	0.854903300241703\\
12.6486180319247	0.854903004052115\\
12.6504927916124	0.854902707978336\\
12.6523675510962	0.854902412020307\\
12.6542423103763	0.854902116177966\\
12.6561170694527	0.854901820451254\\
12.6579918283255	0.85490152484011\\
12.6598665869947	0.854901229344473\\
12.6617413454605	0.854900933964284\\
12.6636161037229	0.854900638699481\\
12.665490861782	0.854900343550005\\
12.6673656196379	0.854900048515796\\
12.6692403772906	0.854899753596792\\
12.6711151347403	0.854899458792935\\
12.672989891987	0.854899164104164\\
12.6748646490307	0.854898869530418\\
12.6767394058717	0.854898575071638\\
12.6786141625099	0.854898280727765\\
12.6804889189454	0.854897986498737\\
12.6823636751784	0.854897692384495\\
12.6842384312088	0.854897398384979\\
12.6861131870368	0.85489710450013\\
12.6879879426624	0.854896810729887\\
12.6898626980858	0.854896517074191\\
12.6917374533069	0.854896223532982\\
12.6936122083259	0.8548959301062\\
12.6954869631429	0.854895636793786\\
12.6973617177579	0.85489534359568\\
12.699236472171	0.854895050511823\\
12.7011112263823	0.854894757542155\\
12.7029859803919	0.854894464686616\\
12.7048607341998	0.854894171945147\\
12.7067354878061	0.854893879317689\\
12.708610241211	0.854893586804183\\
12.7104849944144	0.854893294404568\\
12.7123597474165	0.854893002118786\\
12.7142345002173	0.854892709946778\\
12.7161092528169	0.854892417888484\\
12.7179840052154	0.854892125943845\\
12.7198587574128	0.854891834112802\\
12.7217335094093	0.854891542395296\\
12.723608261205	0.854891250791267\\
12.7254830127998	0.854890959300658\\
12.7273577641939	0.854890667923408\\
12.7292325153874	0.854890376659459\\
12.7311072663802	0.854890085508752\\
12.7329820171726	0.854889794471228\\
12.7348567677646	0.854889503546829\\
12.7367315181562	0.854889212735495\\
12.7386062683476	0.854888922037167\\
12.7404810183388	0.854888631451788\\
12.7423557681299	0.854888340979298\\
12.7442305177209	0.854888050619639\\
12.746105267112	0.854887760372752\\
12.7479800163033	0.854887470238578\\
12.7498547652947	0.85488718021706\\
12.7517295140864	0.854886890308138\\
12.7536042626784	0.854886600511754\\
12.7554790110709	0.85488631082785\\
12.7573537592639	0.854886021256367\\
12.7592285072575	0.854885731797247\\
12.7611032550518	0.854885442450432\\
12.7629780026467	0.854885153215863\\
12.7648527500425	0.854884864093483\\
12.7667274972392	0.854884575083233\\
12.7686022442369	0.854884286185054\\
12.7704769910356	0.85488399739889\\
12.7723517376354	0.854883708724681\\
12.7742264840364	0.85488342016237\\
12.7761012302387	0.854883131711899\\
12.7779759762423	0.854882843373209\\
12.7798507220474	0.854882555146244\\
12.781725467654	0.854882267030944\\
12.7836002130621	0.854881979027253\\
12.7854749582719	0.854881691135111\\
12.7873497032834	0.854881403354463\\
12.7892244480968	0.854881115685249\\
12.791099192712	0.854880828127413\\
12.7929739371292	0.854880540680896\\
12.7948486813484	0.854880253345641\\
12.7967234253697	0.85487996612159\\
12.7985981691932	0.854879679008686\\
12.800472912819	0.854879392006871\\
12.8023476562471	0.854879105116089\\
12.8042223994776	0.85487881833628\\
12.8060971425106	0.854878531667389\\
12.8079718853462	0.854878245109357\\
12.8098466279844	0.854877958662128\\
12.8117213704254	0.854877672325643\\
12.8135961126691	0.854877386099847\\
12.8154708547157	0.854877099984681\\
12.8173455965653	0.854876813980088\\
12.8192203382178	0.854876528086012\\
12.8210950796735	0.854876242302395\\
12.8229698209323	0.85487595662918\\
12.8248445619944	0.85487567106631\\
12.8267193028598	0.854875385613729\\
12.8285940435286	0.854875100271378\\
12.8304687840008	0.854874815039202\\
12.8323435242766	0.854874529917144\\
12.834218264356	0.854874244905146\\
12.8360930042392	0.854873960003151\\
12.8379677439261	0.854873675211104\\
12.8398424834168	0.854873390528947\\
12.8417172227115	0.854873105956624\\
12.8435919618101	0.854872821494078\\
12.8454667007129	0.854872537141252\\
12.8473414394197	0.85487225289809\\
12.8492161779308	0.854871968764535\\
12.8510909162462	0.85487168474053\\
12.852965654366	0.85487140082602\\
12.8548403922902	0.854871117020948\\
12.8567151300189	0.854870833325257\\
12.8585898675523	0.854870549738891\\
12.8604646048903	0.854870266261794\\
12.862339342033	0.85486998289391\\
12.8642140789806	0.854869699635181\\
12.866088815733	0.854869416485553\\
12.8679635522904	0.854869133444969\\
12.8698382886529	0.854868850513372\\
12.8717130248205	0.854868567690706\\
12.8735877607932	0.854868284976916\\
12.8754624965713	0.854868002371946\\
12.8773372321547	0.854867719875739\\
12.8792119675434	0.854867437488239\\
12.8810867027377	0.854867155209392\\
12.8829614377376	0.854866873039139\\
12.884836172543	0.854866590977427\\
12.8867109071542	0.854866309024198\\
12.8885856415712	0.854866027179398\\
12.890460375794	0.854865745442971\\
12.8923351098228	0.85486546381486\\
12.8942098436575	0.85486518229501\\
12.8960845772984	0.854864900883366\\
12.8979593107454	0.854864619579871\\
12.8998340439986	0.854864338384471\\
12.9017087770581	0.85486405729711\\
12.9035835099241	0.854863776317731\\
12.9054582425964	0.854863495446281\\
12.9073329750753	0.854863214682703\\
12.9092077073608	0.854862934026942\\
12.9110824394529	0.854862653478943\\
12.9129571713518	0.85486237303865\\
12.9148319030575	0.854862092706008\\
12.9167066345701	0.854861812480962\\
12.9185813658897	0.854861532363457\\
12.9204560970163	0.854861252353436\\
12.92233082795	0.854860972450847\\
12.9242055586909	0.854860692655632\\
12.926080289239	0.854860412967738\\
12.9279550195945	0.854860133387108\\
12.9298297497574	0.854859853913689\\
12.9317044797278	0.854859574547425\\
12.9335792095057	0.854859295288261\\
12.9354539390912	0.854859016136142\\
12.9373286684845	0.854858737091014\\
12.9392033976855	0.854858458152821\\
12.9410781266943	0.85485817932151\\
12.9429528555111	0.854857900597024\\
12.9448275841358	0.85485762197931\\
12.9467023125687	0.854857343468312\\
12.9485770408096	0.854857065063977\\
12.9504517688588	0.854856786766249\\
12.9523264967162	0.854856508575073\\
12.954201224382	0.854856230490397\\
12.9560759518562	0.854855952512164\\
12.9579506791389	0.85485567464032\\
12.9598254062302	0.854855396874811\\
12.9617001331301	0.854855119215583\\
12.9635748598388	0.854854841662581\\
12.9654495863562	0.854854564215751\\
12.9673243126825	0.854854286875038\\
12.9691990388178	0.854854009640389\\
12.971073764762	0.854853732511749\\
12.9729484905153	0.854853455489063\\
12.9748232160778	0.854853178572278\\
12.9766979414495	0.85485290176134\\
12.9785726666305	0.854852625056195\\
12.9804473916208	0.854852348456787\\
12.9823221164206	0.854852071963065\\
12.9841968410299	0.854851795574972\\
12.9860715654488	0.854851519292456\\
12.9879462896773	0.854851243115463\\
12.9898210137156	0.854850967043939\\
12.9916957375637	0.854850691077829\\
12.9935704612216	0.85485041521708\\
12.9954451846895	0.854850139461639\\
12.9973199079674	0.854849863811451\\
12.9991946310554	0.854849588266463\\
13.0010693539536	0.854849312826621\\
13.002944076662	0.854849037491872\\
13.0048187991807	0.854848762262162\\
13.0066935215097	0.854848487137437\\
13.0085682436492	0.854848212117643\\
13.0104429655993	0.854847937202728\\
13.0123176873599	0.854847662392638\\
13.0141924089312	0.854847387687319\\
13.0160671303132	0.854847113086719\\
13.017941851506	0.854846838590783\\
13.0198165725097	0.854846564199458\\
13.0216912933244	0.854846289912691\\
13.02356601395	0.854846015730429\\
13.0254407343868	0.854845741652619\\
13.0273154546347	0.854845467679207\\
13.0291901746939	0.85484519381014\\
13.0310648945643	0.854844920045365\\
13.0329396142461	0.854844646384829\\
13.0348143337394	0.854844372828478\\
13.0366890530442	0.854844099376261\\
13.0385637721606	0.854843826028124\\
13.0404384910887	0.854843552784013\\
13.0423132098284	0.854843279643877\\
13.04418792838	0.854843006607661\\
13.0460626467435	0.854842733675314\\
13.0479373649189	0.854842460846783\\
13.0498120829063	0.854842188122013\\
13.0516868007058	0.854841915500954\\
13.0535615183174	0.854841642983552\\
13.0554362357413	0.854841370569754\\
13.0573109529775	0.854841098259509\\
13.0591856700261	0.854840826052762\\
13.061060386887	0.854840553949462\\
13.0629351035606	0.854840281949556\\
13.0648098200466	0.854840010052992\\
13.0666845363454	0.854839738259716\\
13.0685592524568	0.854839466569677\\
13.0704339683811	0.854839194982823\\
13.0723086841182	0.8548389234991\\
13.0741833996682	0.854838652118456\\
13.0760581150313	0.85483838084084\\
13.0779328302074	0.854838109666199\\
13.0798075451967	0.85483783859448\\
13.0816822599992	0.854837567625631\\
13.083556974615	0.854837296759601\\
13.0854316890441	0.854837025996336\\
13.0873064032866	0.854836755335786\\
13.0891811173427	0.854836484777897\\
13.0910558312123	0.854836214322618\\
13.0929305448956	0.854835943969896\\
13.0948052583925	0.854835673719681\\
13.0966799717033	0.854835403571919\\
13.0985546848278	0.854835133526559\\
13.1004293977663	0.854834863583549\\
13.1023041105188	0.854834593742837\\
13.1041788230854	0.854834324004371\\
13.106053535466	0.8548340543681\\
13.1079282476609	0.854833784833972\\
13.10980295967	0.854833515401935\\
13.1116776714935	0.854833246071937\\
13.1135523831314	0.854832976843926\\
13.1154270945837	0.854832707717852\\
13.1173018058506	0.854832438693662\\
13.1191765169321	0.854832169771305\\
13.1210512278283	0.854831900950729\\
13.1229259385392	0.854831632231883\\
13.124800649065	0.854831363614715\\
13.1266753594056	0.854831095099175\\
13.1285500695612	0.85483082668521\\
13.1304247795319	0.854830558372769\\
13.1322994893176	0.854830290161801\\
13.1341741989185	0.854830022052255\\
13.1360489083347	0.854829754044078\\
13.1379236175662	0.854829486137221\\
13.139798326613	0.854829218331632\\
13.1416730354753	0.85482895062726\\
13.1435477441531	0.854828683024053\\
13.1454224526465	0.85482841552196\\
13.1472971609555	0.854828148120931\\
13.1491718690803	0.854827880820914\\
13.1510465770208	0.854827613621858\\
13.1529212847772	0.854827346523713\\
13.1547959923496	0.854827079526427\\
13.1566706997379	0.85482681262995\\
13.1585454069423	0.85482654583423\\
13.1604201139629	0.854826279139217\\
13.1622948207996	0.85482601254486\\
13.1641695274527	0.854825746051108\\
13.166044233922	0.854825479657911\\
13.1679189402078	0.854825213365217\\
13.1697936463101	0.854824947172976\\
13.1716683522289	0.854824681081138\\
13.1735430579643	0.854824415089651\\
13.1754177635164	0.854824149198465\\
13.1772924688853	0.85482388340753\\
13.179167174071	0.854823617716795\\
13.1810418790736	0.85482335212621\\
13.1829165838931	0.854823086635723\\
13.1847912885297	0.854822821245286\\
13.1866659929834	0.854822555954846\\
13.1885406972542	0.854822290764354\\
13.1904154013423	0.85482202567376\\
13.1922901052476	0.854821760683014\\
13.1941648089704	0.854821495792064\\
13.1960395125105	0.85482123100086\\
13.1979142158682	0.854820966309354\\
13.1997889190435	0.854820701717494\\
13.2016636220364	0.854820437225229\\
13.203538324847	0.854820172832512\\
13.2054130274754	0.85481990853929\\
13.2072877299216	0.854819644345514\\
13.2091624321857	0.854819380251133\\
13.2110371342679	0.854819116256099\\
13.2129118361681	0.854818852360361\\
13.2147865378864	0.854818588563869\\
13.2166612394228	0.854818324866573\\
13.2185359407776	0.854818061268423\\
13.2204106419506	0.85481779776937\\
13.2222853429421	0.854817534369364\\
13.224160043752	0.854817271068354\\
13.2260347443804	0.854817007866292\\
13.2279094448274	0.854816744763127\\
13.2297841450931	0.85481648175881\\
13.2316588451775	0.854816218853291\\
13.2335335450807	0.85481595604652\\
13.2354082448028	0.854815693338449\\
13.2372829443438	0.854815430729027\\
13.2391576437038	0.854815168218205\\
13.2410323428828	0.854814905805933\\
13.242907041881	0.854814643492162\\
13.2447817406984	0.854814381276843\\
13.246656439335	0.854814119159925\\
13.248531137791	0.854813857141361\\
13.2504058360664	0.8548135952211\\
13.2522805341613	0.854813333399093\\
13.2541552320757	0.854813071675291\\
13.2560299298097	0.854812810049645\\
13.2579046273633	0.854812548522105\\
13.2597793247367	0.854812287092623\\
13.2616540219299	0.854812025761148\\
13.263528718943	0.854811764527632\\
13.265403415776	0.854811503392027\\
13.267278112429	0.854811242354282\\
13.2691528089021	0.854810981414349\\
13.2710275051953	0.854810720572179\\
13.2729022013087	0.854810459827723\\
13.2747768972424	0.854810199180931\\
13.2766515929965	0.854809938631756\\
13.2785262885709	0.854809678180148\\
13.2804009839658	0.854809417826058\\
13.2822756791813	0.854809157569438\\
13.2841503742174	0.854808897410239\\
13.2860250690741	0.854808637348411\\
13.2878997637516	0.854808377383907\\
13.2897744582498	0.854808117516678\\
13.291649152569	0.854807857746674\\
13.2935238467091	0.854807598073848\\
13.2953985406701	0.85480733849815\\
13.2972732344523	0.854807079019533\\
13.2991479280556	0.854806819637947\\
13.3010226214801	0.854806560353345\\
13.3028973147258	0.854806301165677\\
13.3047720077929	0.854806042074895\\
13.3066467006815	0.854805783080951\\
13.3085213933914	0.854805524183797\\
13.3103960859229	0.854805265383384\\
13.312270778276	0.854805006679663\\
13.3141454704508	0.854804748072587\\
13.3160201624473	0.854804489562107\\
13.3178948542656	0.854804231148175\\
13.3197695459058	0.854803972830743\\
13.3216442373679	0.854803714609763\\
13.323518928652	0.854803456485186\\
13.3253936197581	0.854803198456964\\
13.3272683106864	0.854802940525049\\
13.3291430014369	0.854802682689394\\
13.3310176920096	0.85480242494995\\
13.3328923824047	0.85480216730667\\
13.3347670726222	0.854801909759504\\
13.3366417626621	0.854801652308406\\
13.3385164525245	0.854801394953328\\
13.3403911422095	0.854801137694221\\
13.3422658317172	0.854800880531038\\
13.3441405210476	0.85480062346373\\
13.3460152102008	0.854800366492251\\
13.3478898991768	0.854800109616552\\
13.3497645879758	0.854799852836586\\
13.3516392765977	0.854799596152305\\
13.3535139650427	0.854799339563661\\
13.3553886533108	0.854799083070607\\
13.357263341402	0.854798826673095\\
13.3591380293166	0.854798570371078\\
13.3610127170544	0.854798314164507\\
13.3628874046156	0.854798058053336\\
13.3647620920002	0.854797802037517\\
13.3666367792083	0.854797546117003\\
13.36851146624	0.854797290291745\\
13.3703861530954	0.854797034561698\\
13.3722608397744	0.854796778926812\\
13.3741355262772	0.854796523387042\\
13.3760102126039	0.85479626794234\\
13.3778848987544	0.854796012592658\\
13.3797595847289	0.854795757337949\\
13.3816342705274	0.854795502178166\\
13.38350895615	0.854795247113262\\
13.3853836415967	0.854794992143189\\
13.3872583268677	0.854794737267901\\
13.389133011963	0.854794482487351\\
13.3910076968826	0.854794227801491\\
13.3928823816266	0.854793973210274\\
13.3947570661951	0.854793718713653\\
13.3966317505881	0.854793464311582\\
13.3985064348058	0.854793210004013\\
13.4003811188481	0.854792955790899\\
13.4022558027151	0.854792701672194\\
13.404130486407	0.854792447647851\\
13.4060051699237	0.854792193717823\\
13.4078798532654	0.854791939882062\\
13.409754536432	0.854791686140523\\
13.4116292194237	0.854791432493159\\
13.4135039022405	0.854791178939922\\
13.4153785848826	0.854790925480766\\
13.4172532673498	0.854790672115645\\
13.4191279496424	0.854790418844511\\
13.4210026317604	0.854790165667319\\
13.4228773137038	0.854789912584022\\
13.4247519954727	0.854789659594572\\
13.4266266770672	0.854789406698924\\
13.4285013584873	0.854789153897031\\
13.4303760397331	0.854788901188847\\
13.4322507208047	0.854788648574325\\
13.4341254017021	0.854788396053419\\
13.4360000824255	0.854788143626082\\
13.4378747629747	0.854787891292268\\
13.43974944335	0.854787639051931\\
13.4416241235514	0.854787386905024\\
13.4434988035789	0.854787134851502\\
13.4453734834327	0.854786882891318\\
13.4472481631127	0.854786631024425\\
13.4491228426191	0.854786379250778\\
13.4509975219518	0.854786127570331\\
13.4528722011111	0.854785875983037\\
13.4547468800968	0.85478562448885\\
13.4566215589092	0.854785373087724\\
13.4584962375482	0.854785121779614\\
13.460370916014	0.854784870564473\\
13.4622455943065	0.854784619442255\\
13.4641202724259	0.854784368412914\\
13.4659949503722	0.854784117476405\\
13.4678696281454	0.854783866632681\\
13.4697443057457	0.854783615881697\\
13.4716189831731	0.854783365223407\\
13.4734936604277	0.854783114657764\\
13.4753683375095	0.854782864184724\\
13.4772430144186	0.85478261380424\\
13.4791176911551	0.854782363516267\\
13.480992367719	0.85478211332076\\
13.4828670441104	0.854781863217671\\
13.4847417203293	0.854781613206956\\
13.4866163963758	0.85478136328857\\
13.48849107225	0.854781113462466\\
13.4903657479519	0.854780863728599\\
13.4922404234817	0.854780614086924\\
13.4941150988392	0.854780364537395\\
13.4959897740248	0.854780115079966\\
13.4978644490383	0.854779865714593\\
13.4997391238798	0.854779616441229\\
13.5016137985495	0.85477936725983\\
13.5034884730474	0.85477911817035\\
13.5053631473734	0.854778869172743\\
13.5072378215278	0.854778620266965\\
13.5091124955106	0.85477837145297\\
13.5109871693218	0.854778122730713\\
13.5128618429615	0.854777874100148\\
13.5147365164297	0.854777625561232\\
13.5166111897266	0.854777377113917\\
13.5184858628521	0.85477712875816\\
13.5203605358064	0.854776880493915\\
13.5222352085894	0.854776632321137\\
13.5241098812014	0.854776384239781\\
13.5259845536422	0.854776136249802\\
13.5278592259121	0.854775888351155\\
13.529733898011	0.854775640543795\\
13.5316085699391	0.854775392827677\\
13.5334832416963	0.854775145202757\\
13.5353579132827	0.854774897668989\\
13.5372325846985	0.854774650226328\\
13.5391072559436	0.85477440287473\\
13.5409819270182	0.85477415561415\\
13.5428565979222	0.854773908444543\\
13.5447312686558	0.854773661365865\\
13.546605939219	0.85477341437807\\
13.5484806096119	0.854773167481114\\
13.5503552798346	0.854772920674953\\
13.552229949887	0.854772673959541\\
13.5541046197693	0.854772427334834\\
13.5559792894815	0.854772180800788\\
13.5578539590237	0.854771934357358\\
13.559728628396	0.8547716880045\\
13.5616032975984	0.854771441742168\\
13.5634779666309	0.854771195570319\\
13.5653526354937	0.854770949488908\\
13.5672273041868	0.854770703497891\\
13.5691019727102	0.854770457597223\\
13.5709766410641	0.85477021178686\\
13.5728513092484	0.854769966066758\\
13.5747259772633	0.854769720436872\\
13.5766006451088	0.854769474897158\\
13.5784753127849	0.854769229447572\\
13.5803499802918	0.854768984088069\\
13.5822246476295	0.854768738818606\\
13.5840993147981	0.854768493639139\\
13.5859739817975	0.854768248549622\\
13.5878486486279	0.854768003550012\\
13.5897233152894	0.854767758640265\\
13.591597981782	0.854767513820337\\
13.5934726481057	0.854767269090184\\
13.5953473142606	0.854767024449762\\
13.5972219802469	0.854766779899026\\
13.5990966460644	0.854766535437934\\
13.6009713117134	0.85476629106644\\
13.6028459771939	0.854766046784502\\
13.6047206425058	0.854765802592075\\
13.6065953076494	0.854765558489115\\
13.6084699726246	0.854765314475579\\
13.6103446374315	0.854765070551422\\
13.6122193020702	0.854764826716602\\
13.6140939665408	0.854764582971074\\
13.6159686308432	0.854764339314795\\
13.6178432949776	0.85476409574772\\
13.619717958944	0.854763852269807\\
13.6215926227425	0.854763608881012\\
13.6234672863731	0.85476336558129\\
13.6253419498359	0.854763122370599\\
13.627216613131	0.854762879248895\\
13.6290912762584	0.854762636216135\\
13.6309659392182	0.854762393272274\\
13.6328406020104	0.85476215041727\\
13.6347152646352	0.854761907651079\\
13.6365899270925	0.854761664973657\\
13.6384645893824	0.854761422384962\\
13.640339251505	0.854761179884949\\
13.6422139134604	0.854760937473577\\
13.6440885752485	0.8547606951508\\
13.6459632368696	0.854760452916577\\
13.6478378983236	0.854760210770863\\
13.6497125596105	0.854759968713615\\
13.6515872207305	0.854759726744791\\
13.6534618816836	0.854759484864347\\
13.6553365424699	0.85475924307224\\
13.6572112030895	0.854759001368427\\
13.6590858635423	0.854758759752865\\
13.6609605238284	0.854758518225511\\
13.662835183948	0.854758276786321\\
13.6647098439011	0.854758035435252\\
13.6665845036877	0.854757794172263\\
13.6684591633079	0.854757552997309\\
13.6703338227617	0.854757311910348\\
13.6722084820492	0.854757070911336\\
13.6740831411706	0.854756830000232\\
13.6759578001257	0.854756589176992\\
13.6778324589148	0.854756348441573\\
13.6797071175378	0.854756107793933\\
13.6815817759948	0.854755867234028\\
13.6834564342859	0.854755626761816\\
13.6853310924112	0.854755386377255\\
13.6872057503706	0.854755146080301\\
13.6890804081643	0.854754905870912\\
13.6909550657923	0.854754665749045\\
13.6928297232546	0.854754425714657\\
13.6947043805514	0.854754185767707\\
13.6965790376827	0.854753945908151\\
13.6984536946486	0.854753706135947\\
13.7003283514491	0.854753466451052\\
13.7022030080842	0.854753226853425\\
13.7040776645541	0.854752987343021\\
13.7059523208588	0.8547527479198\\
13.7078269769983	0.854752508583718\\
13.7097016329727	0.854752269334733\\
13.7115762887821	0.854752030172804\\
13.7134509444266	0.854751791097886\\
13.7153255999061	0.854751552109939\\
13.7172002552208	0.85475131320892\\
13.7190749103707	0.854751074394786\\
13.7209495653558	0.854750835667495\\
13.7228242201763	0.854750597027006\\
13.7246988748322	0.854750358473276\\
13.7265735293235	0.854750120006262\\
13.7284481836504	0.854749881625924\\
13.7303228378128	0.854749643332218\\
13.7321974918108	0.854749405125102\\
13.7340721456445	0.854749167004535\\
13.735946799314	0.854748928970474\\
13.7378214528192	0.854748691022878\\
13.7396961061603	0.854748453161704\\
13.7415707593374	0.854748215386911\\
13.7434454123504	0.854747977698456\\
13.7453200651995	0.854747740096299\\
13.7471947178846	0.854747502580395\\
13.7490693704059	0.854747265150705\\
13.7509440227635	0.854747027807186\\
13.7528186749573	0.854746790549796\\
13.7546933269874	0.854746553378494\\
13.756567978854	0.854746316293238\\
13.758442630557	0.854746079293985\\
13.7603172820965	0.854745842380696\\
13.7621919334726	0.854745605553326\\
13.7640665846853	0.854745368811836\\
13.7659412357347	0.854745132156184\\
13.7678158866208	0.854744895586327\\
13.7696905373437	0.854744659102224\\
13.7715651879036	0.854744422703835\\
13.7734398383003	0.854744186391116\\
13.775314488534	0.854743950164027\\
13.7771891386047	0.854743714022527\\
13.7790637885126	0.854743477966573\\
13.7809384382576	0.854743241996125\\
13.7828130878398	0.85474300611114\\
13.7846877372593	0.854742770311579\\
13.7865623865162	0.854742534597398\\
13.7884370356104	0.854742298968558\\
13.7903116845421	0.854742063425016\\
13.7921863333112	0.854741827966732\\
13.794060981918	0.854741592593664\\
13.7959356303624	0.85474135730577\\
13.7978102786444	0.854741122103011\\
13.7996849267642	0.854740886985344\\
13.8015595747218	0.854740651952729\\
13.8034342225173	0.854740417005123\\
13.8053088701506	0.854740182142487\\
13.807183517622	0.85473994736478\\
13.8090581649313	0.854739712671959\\
13.8109328120788	0.854739478063984\\
13.8128074590644	0.854739243540815\\
13.8146821058882	0.85473900910241\\
13.8165567525503	0.854738774748727\\
13.8184313990507	0.854738540479727\\
13.8203060453895	0.854738306295369\\
13.8221806915667	0.854738072195611\\
13.8240553375824	0.854737838180412\\
13.8259299834367	0.854737604249733\\
13.8278046291296	0.854737370403531\\
13.8296792746611	0.854737136641767\\
13.8315539200314	0.854736902964399\\
13.8334285652405	0.854736669371387\\
13.8353032102884	0.85473643586269\\
13.8371778551752	0.854736202438268\\
13.839052499901	0.854735969098079\\
13.8409271444658	0.854735735842083\\
13.8428017888696	0.85473550267024\\
13.8446764331126	0.854735269582509\\
13.8465510771948	0.85473503657885\\
13.8484257211162	0.854734803659221\\
13.8503003648769	0.854734570823582\\
13.852175008477	0.854734338071894\\
13.8540496519165	0.854734105404115\\
13.8559242951955	0.854733872820205\\
13.857798938314	0.854733640320123\\
13.8596735812722	0.85473340790383\\
13.8615482240699	0.854733175571285\\
13.8634228667074	0.854732943322447\\
13.8652975091846	0.854732711157277\\
13.8671721515017	0.854732479075733\\
13.8690467936586	0.854732247077777\\
13.8709214356554	0.854732015163367\\
13.8727960774923	0.854731783332462\\
13.8746707191692	0.854731551585025\\
13.8765453606862	0.854731319921013\\
13.8784200020434	0.854731088340386\\
13.8802946432408	0.854730856843106\\
13.8821692842784	0.854730625429131\\
13.8840439251564	0.854730394098421\\
13.8859185658748	0.854730162850937\\
13.8877932064337	0.854729931686639\\
13.889667846833	0.854729700605486\\
13.8915424870729	0.854729469607438\\
13.8934171271535	0.854729238692457\\
13.8952917670747	0.8547290078605\\
13.8971664068366	0.85472877711153\\
13.8990410464393	0.854728546445505\\
13.9009156858829	0.854728315862386\\
13.9027903251674	0.854728085362134\\
13.9046649642928	0.854727854944708\\
13.9065396032593	0.854727624610068\\
13.9084142420668	0.854727394358176\\
13.9102888807155	0.85472716418899\\
13.9121635192053	0.854726934102472\\
13.9140381575364	0.854726704098582\\
13.9159127957088	0.85472647417728\\
13.9177874337226	0.854726244338527\\
13.9196620715777	0.854726014582282\\
13.9215367092744	0.854725784908506\\
13.9234113468125	0.854725555317161\\
13.9252859841923	0.854725325808205\\
13.9271606214136	0.8547250963816\\
13.9290352584767	0.854724867037306\\
13.9309098953815	0.854724637775284\\
13.9327845321282	0.854724408595494\\
13.9346591687167	0.854724179497898\\
13.9365338051471	0.854723950482454\\
13.9384084414195	0.854723721549125\\
13.9402830775339	0.85472349269787\\
13.9421577134904	0.854723263928651\\
13.9440323492891	0.854723035241428\\
13.9459069849299	0.854722806636162\\
13.9477816204131	0.854722578112813\\
13.9496562557385	0.854722349671343\\
13.9515308909063	0.854722121311712\\
13.9534055259165	0.85472189303388\\
13.9552801607692	0.85472166483781\\
13.9571547954645	0.854721436723461\\
13.9590294300023	0.854721208690795\\
13.9609040643828	0.854720980739773\\
13.962778698606	0.854720752870354\\
13.964653332672	0.854720525082502\\
13.9665279665808	0.854720297376175\\
13.9684026003324	0.854720069751336\\
13.970277233927	0.854719842207945\\
13.9721518673646	0.854719614745964\\
13.9740265006452	0.854719387365353\\
13.9759011337689	0.854719160066073\\
13.9777757667357	0.854718932848087\\
13.9796503995458	0.854718705711354\\
13.9815250321991	0.854718478655837\\
13.9833996646957	0.854718251681495\\
13.9852742970358	0.854718024788291\\
13.9871489292192	0.854717797976186\\
13.9890235612461	0.854717571245141\\
13.9908981931166	0.854717344595117\\
13.9927728248307	0.854717118026075\\
13.9946474563884	0.854716891537977\\
13.9965220877898	0.854716665130785\\
13.998396719035	0.854716438804459\\
14.000271350124	0.854716212558961\\
14.0021459810569	0.854715986394253\\
14.0040206118337	0.854715760310295\\
14.0058952424545	0.85471553430705\\
14.0077698729194	0.854715308384478\\
14.0096445032283	0.854715082542542\\
14.0115191333814	0.854714856781203\\
14.0133937633787	0.854714631100422\\
14.0152683932202	0.854714405500162\\
14.0171430229061	0.854714179980383\\
14.0190176524363	0.854713954541047\\
14.0208922818109	0.854713729182116\\
14.0227669110301	0.854713503903552\\
14.0246415400938	0.854713278705315\\
14.026516169002	0.854713053587369\\
14.0283907977549	0.854712828549675\\
14.0302654263525	0.854712603592194\\
14.0321400547949	0.854712378714889\\
14.0340146830821	0.854712153917721\\
14.0358893112141	0.854711929200651\\
14.0377639391911	0.854711704563643\\
14.039638567013	0.854711480006657\\
14.04151319468	0.854711255529656\\
14.043387822192	0.854711031132601\\
14.0452624495492	0.854710806815455\\
14.0471370767516	0.854710582578179\\
14.0490117037992	0.854710358420736\\
14.0508863306922	0.854710134343087\\
14.0527609574305	0.854709910345195\\
14.0546355840142	0.854709686427021\\
14.0565102104433	0.854709462588528\\
14.058384836718	0.854709238829678\\
14.0602594628383	0.854709015150432\\
14.0621340888042	0.854708791550754\\
14.0640087146158	0.854708568030605\\
14.0658833402731	0.854708344589947\\
14.0677579657763	0.854708121228743\\
14.0696325911252	0.854707897946955\\
14.0715072163201	0.854707674744545\\
14.0733818413609	0.854707451621475\\
14.0752564662478	0.854707228577708\\
14.0771310909807	0.854707005613207\\
14.0790057155597	0.854706782727932\\
14.0808803399849	0.854706559921848\\
14.0827549642564	0.854706337194915\\
14.0846295883741	0.854706114547098\\
14.0865042123381	0.854705891978357\\
14.0883788361486	0.854705669488656\\
14.0902534598055	0.854705447077957\\
14.0921280833089	0.854705224746223\\
14.0940027066588	0.854705002493415\\
14.0958773298554	0.854704780319498\\
14.0977519528986	0.854704558224432\\
14.0996265757885	0.854704336208182\\
14.1015011985252	0.854704114270708\\
14.1033758211088	0.854703892411975\\
14.1052504435392	0.854703670631945\\
14.1071250658165	0.85470344893058\\
14.1089996879408	0.854703227307843\\
14.1108743099122	0.854703005763697\\
14.1127489317306	0.854702784298105\\
14.1146235533962	0.854702562911029\\
14.116498174909	0.854702341602433\\
14.1183727962691	0.854702120372278\\
14.1202474174765	0.854701899220529\\
14.1221220385312	0.854701678147147\\
14.1239966594333	0.854701457152096\\
14.125871280183	0.854701236235338\\
14.1277459007801	0.854701015396837\\
14.1296205212248	0.854700794636555\\
14.1314951415172	0.854700573954456\\
14.1333697616572	0.854700353350502\\
14.135244381645	0.854700132824656\\
14.1371190014805	0.854699912376882\\
14.1389936211639	0.854699692007143\\
14.1408682406952	0.854699471715401\\
14.1427428600745	0.85469925150162\\
14.1446174793017	0.854699031365763\\
14.146492098377	0.854698811307793\\
14.1483667173005	0.854698591327673\\
14.150241336072	0.854698371425366\\
14.1521159546919	0.854698151600837\\
14.1539905731599	0.854697931854046\\
14.1558651914763	0.854697712184959\\
14.1577398096411	0.854697492593539\\
14.1596144276543	0.854697273079748\\
14.161489045516	0.85469705364355\\
14.1633636632263	0.854696834284909\\
14.1652382807851	0.854696615003787\\
14.1671128981926	0.854696395800148\\
14.1689875154487	0.854696176673956\\
14.1708621325536	0.854695957625174\\
14.1727367495074	0.854695738653765\\
14.1746113663099	0.854695519759694\\
14.1764859829614	0.854695300942922\\
14.1783605994619	0.854695082203415\\
14.1802352158113	0.854694863541135\\
14.1821098320098	0.854694644956046\\
14.1839844480575	0.854694426448112\\
14.1858590639543	0.854694208017296\\
14.1877336797004	0.854693989663561\\
14.1896082952957	0.854693771386873\\
14.1914829107403	0.854693553187193\\
14.1933575260344	0.854693335064487\\
14.1952321411779	0.854693117018716\\
14.1971067561708	0.854692899049847\\
14.1989813710133	0.854692681157841\\
14.2008559857054	0.854692463342663\\
14.2027306002472	0.854692245604277\\
14.2046052146387	0.854692027942646\\
14.2064798288799	0.854691810357735\\
14.2083544429709	0.854691592849507\\
14.2102290569118	0.854691375417926\\
14.2121036707025	0.854691158062956\\
14.2139782843433	0.85469094078456\\
14.2158528978341	0.854690723582704\\
14.2177275111749	0.85469050645735\\
14.2196021243658	0.854690289408464\\
14.221476737407	0.854690072436008\\
14.2233513502983	0.854689855539947\\
14.2252259630399	0.854689638720245\\
14.2271005756319	0.854689421976865\\
14.2289751880742	0.854689205309773\\
14.230849800367	0.854688988718933\\
14.2327244125103	0.854688772204307\\
14.2345990245041	0.854688555765861\\
14.2364736363485	0.854688339403559\\
14.2383482480435	0.854688123117365\\
14.2402228595892	0.854687906907242\\
14.2420974709857	0.854687690773156\\
14.243972082233	0.854687474715071\\
14.2458466933311	0.854687258732951\\
14.2477213042801	0.85468704282676\\
14.2495959150801	0.854686826996462\\
14.2514705257311	0.854686611242023\\
14.2533451362332	0.854686395563405\\
14.2552197465864	0.854686179960574\\
14.2570943567907	0.854685964433495\\
14.2589689668462	0.854685748982131\\
14.2608435767531	0.854685533606447\\
14.2627181865112	0.854685318306407\\
14.2645927961207	0.854685103081976\\
14.2664674055816	0.854684887933119\\
14.268342014894	0.8546846728598\\
14.270216624058	0.854684457861983\\
14.2720912330735	0.854684242939634\\
14.2739658419406	0.854684028092716\\
14.2758404506595	0.854683813321194\\
14.27771505923	0.854683598625034\\
14.2795896676523	0.854683384004199\\
14.2814642759265	0.854683169458655\\
14.2833388840526	0.854682954988365\\
14.2852134920306	0.854682740593296\\
14.2870880998606	0.854682526273411\\
14.2889627075426	0.854682312028675\\
14.2908373150768	0.854682097859054\\
14.2927119224631	0.854681883764511\\
14.2945865297015	0.854681669745012\\
14.2964611367923	0.854681455800522\\
14.2983357437353	0.854681241931005\\
14.3002103505307	0.854681028136426\\
14.3020849571784	0.854680814416751\\
14.3039595636787	0.854680600771944\\
14.3058341700314	0.85468038720197\\
14.3077087762367	0.854680173706794\\
14.3095833822946	0.854679960286381\\
14.3114579882052	0.854679746940697\\
14.3133325939685	0.854679533669706\\
14.3152071995845	0.854679320473372\\
14.3170818050534	0.854679107351662\\
14.3189564103751	0.854678894304541\\
14.3208310155497	0.854678681331973\\
14.3227056205773	0.854678468433923\\
14.3245802254579	0.854678255610358\\
14.3264548301916	0.854678042861241\\
14.3283294347784	0.854677830186538\\
14.3302040392184	0.854677617586215\\
14.3320786435116	0.854677405060237\\
14.333953247658	0.854677192608568\\
14.3358278516578	0.854676980231175\\
14.337702455511	0.854676767928022\\
14.3395770592176	0.854676555699075\\
14.3414516627776	0.854676343544299\\
14.3433262661912	0.854676131463659\\
14.3452008694584	0.854675919457122\\
14.3470754725792	0.854675707524652\\
14.3489500755537	0.854675495666214\\
14.3508246783819	0.854675283881774\\
14.3526992810639	0.854675072171299\\
14.3545738835997	0.854674860534752\\
14.3564484859894	0.8546746489721\\
14.358323088233	0.854674437483308\\
14.3601976903306	0.854674226068341\\
14.3620722922823	0.854674014727166\\
14.363946894088	0.854673803459748\\
14.3658214957479	0.854673592266052\\
14.367696097262	0.854673381146044\\
14.3695706986302	0.85467317009969\\
14.3714452998528	0.854672959126955\\
14.3733199009297	0.854672748227805\\
14.375194501861	0.854672537402206\\
14.3770691026468	0.854672326650123\\
14.378943703287	0.854672115971522\\
14.3808183037818	0.85467190536637\\
14.3826929041311	0.854671694834631\\
14.3845675043351	0.854671484376272\\
14.3864421043938	0.854671273991258\\
14.3883167043072	0.854671063679555\\
14.3901913040754	0.854670853441129\\
14.3920659036985	0.854670643275947\\
14.3939405031764	0.854670433183973\\
14.3958151025093	0.854670223165174\\
14.3976897016972	0.854670013219516\\
14.3995643007401	0.854669803346964\\
14.4014388996381	0.854669593547485\\
14.4033134983913	0.854669383821045\\
14.4051880969996	0.85466917416761\\
14.4070626954632	0.854668964587145\\
14.4089372937821	0.854668755079617\\
14.4108118919563	0.854668545644992\\
14.4126864899859	0.854668336283236\\
14.4145610878709	0.854668126994315\\
14.4164356856115	0.854667917778195\\
14.4183102832075	0.854667708634843\\
14.4201848806592	0.854667499564224\\
14.4220594779665	0.854667290566305\\
14.4239340751295	0.854667081641052\\
14.4258086721483	0.854666872788432\\
14.4276832690228	0.85466666400841\\
14.4295578657531	0.854666455300952\\
14.4314324623394	0.854666246666026\\
14.4333070587816	0.854666038103597\\
14.4351816550798	0.854665829613632\\
14.437056251234	0.854665621196097\\
14.4389308472443	0.854665412850959\\
14.4408054431108	0.854665204578183\\
14.4426800388334	0.854664996377737\\
14.4445546344123	0.854664788249586\\
14.4464292298475	0.854664580193698\\
14.448303825139	0.854664372210038\\
14.4501784202869	0.854664164298573\\
14.4520530152912	0.854663956459271\\
14.453927610152	0.854663748692096\\
14.4558022048693	0.854663540997016\\
14.4576767994433	0.854663333373997\\
14.4595513938738	0.854663125823007\\
14.4614259881611	0.85466291834401\\
14.463300582305	0.854662710936975\\
14.4651751763058	0.854662503601868\\
14.4670497701634	0.854662296338655\\
14.4689243638778	0.854662089147304\\
14.4707989574492	0.85466188202778\\
14.4726735508776	0.854661674980051\\
14.474548144163	0.854661468004083\\
14.4764227373055	0.854661261099843\\
14.4782973303051	0.854661054267298\\
14.4801719231618	0.854660847506415\\
14.4820465158758	0.854660640817161\\
14.4839211084471	0.854660434199502\\
14.4857957008757	0.854660227653405\\
14.4876702931617	0.854660021178837\\
14.4895448853051	0.854659814775765\\
14.4914194773059	0.854659608444156\\
14.4932940691643	0.854659402183977\\
14.4951686608802	0.854659195995194\\
14.4970432524538	0.854658989877776\\
14.498917843885	0.854658783831688\\
14.500792435174	0.854658577856898\\
14.5026670263207	0.854658371953373\\
14.5045416173252	0.85465816612108\\
14.5064162081876	0.854657960359986\\
14.5082907989078	0.854657754670058\\
14.5101653894861	0.854657549051263\\
14.5120399799223	0.854657343503568\\
14.5139145702166	0.854657138026941\\
14.515789160369	0.854656932621348\\
14.5176637503796	0.854656727286757\\
14.5195383402484	0.854656522023135\\
14.5214129299754	0.85465631683045\\
14.5232875195607	0.854656111708668\\
14.5251621090044	0.854655906657756\\
14.5270366983064	0.854655701677683\\
14.5289112874669	0.854655496768415\\
14.5307858764859	0.854655291929919\\
14.5326604653635	0.854655087162164\\
14.5345350540996	0.854654882465116\\
14.5364096426944	0.854654677838742\\
14.5382842311478	0.854654473283011\\
14.54015881946	0.854654268797889\\
14.542033407631	0.854654064383344\\
14.5439079956608	0.854653860039344\\
14.5457825835495	0.854653655765855\\
14.5476571712971	0.854653451562846\\
14.5495317589037	0.854653247430284\\
14.5514063463693	0.854653043368136\\
14.553280933694	0.85465283937637\\
14.5551555208779	0.854652635454953\\
14.5570301079208	0.854652431603854\\
14.5589046948231	0.854652227823039\\
14.5607792815845	0.854652024112476\\
14.5626538682053	0.854651820472134\\
14.5645284546855	0.854651616901979\\
14.5664030410251	0.854651413401979\\
14.5682776272241	0.854651209972102\\
14.5701522132826	0.854651006612316\\
14.5720267992007	0.854650803322589\\
14.5739013849784	0.854650600102887\\
14.5757759706158	0.85465039695318\\
14.5776505561128	0.854650193873434\\
14.5795251414696	0.854649990863618\\
14.5813997266862	0.854649787923699\\
14.5832743117627	0.854649585053645\\
14.585148896699	0.854649382253424\\
14.5870234814953	0.854649179523004\\
14.5888980661515	0.854648976862353\\
14.5907726506678	0.854648774271439\\
14.5926472350442	0.85464857175023\\
14.5945218192807	0.854648369298693\\
14.5963964033774	0.854648166916797\\
14.5982709873343	0.854647964604509\\
14.6001455711515	0.854647762361798\\
14.602020154829	0.854647560188632\\
14.6038947383669	0.854647358084978\\
14.6057693217652	0.854647156050805\\
14.607643905024	0.854646954086081\\
14.6095184881432	0.854646752190774\\
14.6113930711231	0.854646550364851\\
14.6132676539635	0.854646348608283\\
14.6151422366646	0.854646146921035\\
14.6170168192264	0.854645945303077\\
14.618891401649	0.854645743754377\\
14.6207659839323	0.854645542274902\\
14.6226405660765	0.854645340864622\\
14.6245151480816	0.854645139523504\\
14.6263897299476	0.854644938251517\\
14.6282643116746	0.854644737048629\\
14.6301388932627	0.854644535914808\\
14.6320134747118	0.854644334850023\\
14.6338880560221	0.854644133854241\\
14.6357626371935	0.854643932927432\\
14.6376372182262	0.854643732069563\\
14.6395117991201	0.854643531280604\\
14.6413863798754	0.854643330560522\\
14.643260960492	0.854643129909286\\
14.6451355409701	0.854642929326864\\
14.6470101213096	0.854642728813225\\
14.6488847015106	0.854642528368338\\
14.6507592815731	0.85464232799217\\
14.6526338614973	0.85464212768469\\
14.6545084412831	0.854641927445867\\
14.6563830209306	0.85464172727567\\
14.6582576004399	0.854641527174067\\
14.6601321798109	0.854641327141026\\
14.6620067590438	0.854641127176516\\
14.6638813381386	0.854640927280507\\
14.6657559170953	0.854640727452965\\
14.6676304959139	0.854640527693861\\
14.6695050745946	0.854640328003163\\
14.6713796531374	0.854640128380839\\
14.6732542315423	0.854639928826858\\
14.6751288098094	0.85463972934119\\
14.6770033879386	0.854639529923802\\
14.6788779659302	0.854639330574663\\
14.680752543784	0.854639131293743\\
14.6826271215002	0.85463893208101\\
14.6845016990788	0.854638732936433\\
14.6863762765198	0.85463853385998\\
14.6882508538233	0.854638334851622\\
14.6901254309894	0.854638135911325\\
14.6920000080181	0.85463793703906\\
14.6938745849094	0.854637738234796\\
14.6957491616633	0.8546375394985\\
14.69762373828	0.854637340830143\\
14.6994983147595	0.854637142229693\\
14.7013728911018	0.854636943697119\\
14.703247467307	0.854636745232391\\
14.705122043375	0.854636546835476\\
14.7069966193061	0.854636348506345\\
14.7088711951001	0.854636150244966\\
14.7107457707572	0.854635952051308\\
14.7126203462774	0.854635753925341\\
14.7144949216607	0.854635555867033\\
14.7163694969072	0.854635357876354\\
14.718244072017	0.854635159953272\\
14.7201186469901	0.854634962097758\\
14.7219932218264	0.85463476430978\\
14.7238677965262	0.854634566589307\\
14.7257423710894	0.854634368936309\\
14.727616945516	0.854634171350755\\
14.7294915198062	0.854633973832613\\
14.7313660939599	0.854633776381854\\
14.7332406679773	0.854633578998447\\
14.7351152418583	0.854633381682361\\
14.736989815603	0.854633184433564\\
14.7388643892114	0.854632987252028\\
14.7407389626836	0.85463279013772\\
14.7426135360197	0.854632593090611\\
14.7444881092197	0.854632396110669\\
14.7463626822836	0.854632199197864\\
14.7482372552114	0.854632002352166\\
14.7501118280033	0.854631805573544\\
14.7519864006593	0.854631608861968\\
14.7538609731794	0.854631412217406\\
14.7557355455636	0.854631215639828\\
14.7576101178121	0.854631019129205\\
14.7594846899248	0.854630822685505\\
14.7613592619018	0.854630626308698\\
14.7632338337431	0.854630429998754\\
14.7651084054489	0.854630233755642\\
14.7669829770191	0.854630037579331\\
14.7688575484537	0.854629841469792\\
14.7707321197529	0.854629645426994\\
14.7726066909167	0.854629449450906\\
14.774481261945	0.854629253541499\\
14.7763558328381	0.854629057698742\\
14.7782304035959	0.854628861922605\\
14.7801049742184	0.854628666213057\\
14.7819795447057	0.854628470570068\\
14.7838541150579	0.854628274993608\\
14.7857286852749	0.854628079483647\\
14.7876032553569	0.854627884040154\\
14.7894778253039	0.854627688663099\\
14.791352395116	0.854627493352453\\
14.7932269647931	0.854627298108185\\
14.7951015343353	0.854627102930264\\
14.7969761037427	0.854626907818662\\
14.7988506730153	0.854626712773346\\
14.8007252421531	0.854626517794289\\
14.8025998111563	0.854626322881458\\
14.8044743800248	0.854626128034825\\
14.8063489487587	0.85462593325436\\
14.8082235173581	0.854625738540031\\
14.8100980858229	0.85462554389181\\
14.8119726541533	0.854625349309667\\
14.8138472223492	0.85462515479357\\
14.8157217904108	0.854624960343491\\
14.817596358338	0.8546247659594\\
14.819470926131	0.854624571641266\\
14.8213454937897	0.85462437738906\\
14.8232200613142	0.854624183202751\\
14.8250946287045	0.85462398908231\\
14.8269691959608	0.854623795027708\\
14.828843763083	0.854623601038913\\
14.8307183300711	0.854623407115897\\
14.8325928969253	0.85462321325863\\
14.8344674636456	0.854623019467081\\
14.836342030232	0.854622825741221\\
14.8382165966846	0.854622632081021\\
14.8400911630033	0.85462243848645\\
14.8419657291884	0.854622244957479\\
14.8438402952397	0.854622051494078\\
14.8457148611573	0.854621858096217\\
14.8475894269414	0.854621664763867\\
14.8494639925919	0.854621471496999\\
14.8513385581088	0.854621278295582\\
14.8532131234923	0.854621085159586\\
14.8550876887424	0.854620892088983\\
14.8569622538591	0.854620699083743\\
14.8588368188424	0.854620506143836\\
14.8607113836924	0.854620313269232\\
14.8625859484092	0.854620120459903\\
14.8644605129928	0.854619927715818\\
14.8663350774432	0.854619735036948\\
14.8682096417605	0.854619542423263\\
14.8700842059447	0.854619349874735\\
14.8719587699959	0.854619157391333\\
14.8738333339141	0.854618964973029\\
14.8757078976994	0.854618772619792\\
14.8775824613518	0.854618580331594\\
14.8794570248713	0.854618388108404\\
14.881331588258	0.854618195950195\\
14.883206151512	0.854618003856935\\
14.8850807146333	0.854617811828597\\
14.8869552776218	0.85461761986515\\
14.8888298404778	0.854617427966566\\
14.8907044032012	0.854617236132815\\
14.892578965792	0.854617044363867\\
14.8944535282503	0.854616852659694\\
14.8963280905762	0.854616661020267\\
14.8982026527697	0.854616469445555\\
14.9000772148308	0.854616277935531\\
14.9019517767596	0.854616086490164\\
14.9038263385562	0.854615895109426\\
14.9057009002205	0.854615703793287\\
14.9075754617526	0.854615512541718\\
14.9094500231526	0.854615321354691\\
14.9113245844205	0.854615130232176\\
14.9131991455564	0.854614939174143\\
14.9150737065602	0.854614748180565\\
14.9169482674321	0.854614557251411\\
14.918822828172	0.854614366386653\\
14.9206973887801	0.854614175586263\\
14.9225719492564	0.854613984850209\\
14.9244465096009	0.854613794178465\\
14.9263210698136	0.854613603571001\\
14.9281956298947	0.854613413027787\\
14.9300701898441	0.854613222548795\\
14.9319447496618	0.854613032133997\\
14.9338193093481	0.854612841783362\\
14.9356938689028	0.854612651496863\\
14.937568428326	0.85461246127447\\
14.9394429876178	0.854612271116155\\
14.9413175467782	0.854612081021888\\
14.9431921058072	0.854611890991642\\
14.945066664705	0.854611701025386\\
14.9469412234715	0.854611511123092\\
14.9488157821068	0.854611321284732\\
14.9506903406109	0.854611131510277\\
14.9525648989839	0.854610941799698\\
14.9544394572258	0.854610752152966\\
14.9563140153367	0.854610562570053\\
14.9581885733166	0.854610373050929\\
14.9600631311655	0.854610183595567\\
};
\addplot [color=blue,solid,forget plot]
  table[row sep=crcr]{%
14.9600631311655	0.854610183595567\\
14.9619376888836	0.854609994203937\\
14.9638122464707	0.854609804876011\\
14.9656868039271	0.85460961561176\\
14.9675613612527	0.854609426411156\\
14.9694359184475	0.85460923727417\\
14.9713104755117	0.854609048200774\\
14.9731850324452	0.854608859190938\\
14.9750595892481	0.854608670244635\\
14.9769341459205	0.854608481361836\\
14.9788087024623	0.854608292542511\\
14.9806832588737	0.854608103786634\\
14.9825578151547	0.854607915094175\\
14.9844323713052	0.854607726465106\\
14.9863069273255	0.854607537899398\\
14.9881814832154	0.854607349397023\\
14.9900560389751	0.854607160957953\\
14.9919305946046	0.854606972582159\\
14.9938051501039	0.854606784269613\\
14.9956797054731	0.854606596020286\\
14.9975542607122	0.85460640783415\\
14.9994288158213	0.854606219711177\\
15.0013033708004	0.854606031651338\\
15.0031779256495	0.854605843654605\\
15.0050524803688	0.85460565572095\\
15.0069270349582	0.854605467850344\\
15.0088015894177	0.85460528004276\\
15.0106761437475	0.854605092298169\\
15.0125506979476	0.854604904616542\\
15.0144252520179	0.854604716997852\\
15.0162998059587	0.85460452944207\\
15.0181743597698	0.854604341949169\\
15.0200489134513	0.854604154519119\\
15.0219234670034	0.854603967151893\\
15.023798020426	0.854603779847464\\
15.0256725737191	0.854603592605801\\
15.0275471268829	0.854603405426878\\
15.0294216799173	0.854603218310667\\
15.0312962328224	0.854603031257138\\
15.0331707855983	0.854602844266266\\
15.0350453382449	0.85460265733802\\
15.0369198907624	0.854602470472373\\
15.0387944431507	0.854602283669298\\
15.04066899541	0.854602096928766\\
15.0425435475402	0.85460191025075\\
15.0444180995414	0.85460172363522\\
15.0462926514137	0.85460153708215\\
15.048167203157	0.854601350591511\\
15.0500417547715	0.854601164163276\\
15.0519163062572	0.854600977797416\\
15.053790857614	0.854600791493904\\
15.0556654088421	0.854600605252712\\
15.0575399599416	0.854600419073812\\
15.0594145109123	0.854600232957176\\
15.0612890617545	0.854600046902777\\
15.0631636124681	0.854599860910585\\
15.0650381630531	0.854599674980575\\
15.0669127135097	0.854599489112718\\
15.0687872638378	0.854599303306986\\
15.0706618140375	0.854599117563351\\
15.0725363641089	0.854598931881786\\
15.074410914052	0.854598746262263\\
15.0762854638667	0.854598560704755\\
15.0781600135533	0.854598375209233\\
15.0800345631116	0.85459818977567\\
15.0819091125419	0.854598004404038\\
15.083783661844	0.85459781909431\\
15.085658211018	0.854597633846458\\
15.0875327600641	0.854597448660455\\
15.0894073089821	0.854597263536273\\
15.0912818577723	0.854597078473884\\
15.0931564064345	0.85459689347326\\
15.0950309549689	0.854596708534375\\
15.0969055033755	0.854596523657201\\
15.0987800516543	0.854596338841709\\
15.1006545998054	0.854596154087874\\
15.1025291478288	0.854595969395667\\
15.1044036957246	0.85459578476506\\
15.1062782434928	0.854595600196027\\
15.1081527911334	0.854595415688539\\
15.1100273386466	0.85459523124257\\
15.1119018860322	0.854595046858092\\
15.1137764332905	0.854594862535078\\
15.1156509804213	0.8545946782735\\
15.1175255274248	0.854594494073331\\
15.1194000743011	0.854594309934544\\
15.12127462105	0.85459412585711\\
15.1231491676718	0.854593941841004\\
15.1250237141664	0.854593757886197\\
15.1268982605338	0.854593573992663\\
15.1287728067742	0.854593390160374\\
15.1306473528875	0.854593206389302\\
15.1325218988738	0.854593022679422\\
15.1343964447331	0.854592839030704\\
15.1362709904656	0.854592655443123\\
15.1381455360711	0.854592471916651\\
15.1400200815498	0.854592288451261\\
15.1418946269017	0.854592105046925\\
15.1437691721269	0.854591921703617\\
15.1456437172254	0.854591738421309\\
15.1475182621972	0.854591555199975\\
15.1493928070423	0.854591372039586\\
15.1512673517609	0.854591188940117\\
15.1531418963529	0.85459100590154\\
15.1550164408185	0.854590822923827\\
15.1568909851575	0.854590640006953\\
15.1587655293702	0.854590457150889\\
15.1606400734565	0.85459027435561\\
15.1625146174164	0.854590091621087\\
15.1643891612501	0.854589908947294\\
15.1662637049575	0.854589726334204\\
15.1681382485387	0.85458954378179\\
15.1700127919937	0.854589361290026\\
15.1718873353226	0.854589178858883\\
15.1737618785255	0.854588996488336\\
15.1756364216022	0.854588814178356\\
15.177510964553	0.854588631928919\\
15.1793855073778	0.854588449739996\\
15.1812600500767	0.854588267611561\\
15.1831345926498	0.854588085543586\\
15.185009135097	0.854587903536046\\
15.1868836774184	0.854587721588914\\
15.188758219614	0.854587539702161\\
15.1906327616839	0.854587357875763\\
15.1925073036282	0.854587176109691\\
15.1943818454469	0.85458699440392\\
15.1962563871399	0.854586812758423\\
15.1981309287074	0.854586631173172\\
15.2000054701494	0.854586449648141\\
15.201880011466	0.854586268183304\\
15.2037545526571	0.854586086778633\\
15.2056290937229	0.854585905434103\\
15.2075036346633	0.854585724149686\\
15.2093781754784	0.854585542925357\\
15.2112527161682	0.854585361761087\\
15.2131272567329	0.854585180656851\\
15.2150017971724	0.854584999612622\\
15.2168763374867	0.854584818628374\\
15.2187508776759	0.85458463770408\\
15.2206254177401	0.854584456839713\\
15.2224999576793	0.854584276035247\\
15.2243744974936	0.854584095290656\\
15.2262490371829	0.854583914605913\\
15.2281235767473	0.854583733980991\\
15.2299981161869	0.854583553415864\\
15.2318726555016	0.854583372910505\\
15.2337471946916	0.854583192464889\\
15.2356217337569	0.854583012078989\\
15.2374962726975	0.854582831752778\\
15.2393708115135	0.854582651486229\\
15.2412453502049	0.854582471279318\\
15.2431198887717	0.854582291132016\\
15.244994427214	0.854582111044299\\
15.2468689655318	0.854581931016139\\
15.2487435037252	0.85458175104751\\
15.2506180417942	0.854581571138386\\
15.2524925797388	0.854581391288741\\
15.2543671175592	0.854581211498548\\
15.2562416552552	0.854581031767781\\
15.2581161928271	0.854580852096414\\
15.2599907302747	0.854580672484421\\
15.2618652675982	0.854580492931775\\
15.2637398047976	0.854580313438451\\
15.2656143418729	0.854580134004421\\
15.2674888788242	0.854579954629661\\
15.2693634156515	0.854579775314143\\
15.2712379523549	0.854579596057841\\
15.2731124889344	0.854579416860731\\
15.27498702539	0.854579237722784\\
15.2768615617218	0.854579058643976\\
15.2787360979298	0.85457887962428\\
15.280610634014	0.85457870066367\\
15.2824851699746	0.85457852176212\\
15.2843597058115	0.854578342919604\\
15.2862342415247	0.854578164136096\\
15.2881087771144	0.85457798541157\\
15.2899833125806	0.854577806746\\
15.2918578479232	0.85457762813936\\
15.2937323831425	0.854577449591624\\
15.2956069182382	0.854577271102767\\
15.2974814532106	0.854577092672761\\
15.2993559880597	0.854576914301582\\
15.3012305227855	0.854576735989202\\
15.303105057388	0.854576557735598\\
15.3049795918673	0.854576379540742\\
15.3068541262234	0.854576201404608\\
15.3087286604564	0.854576023327171\\
15.3106031945663	0.854575845308406\\
15.3124777285532	0.854575667348285\\
15.314352262417	0.854575489446784\\
15.3162267961579	0.854575311603876\\
15.3181013297758	0.854575133819536\\
15.3199758632708	0.854574956093739\\
15.321850396643	0.854574778426457\\
15.3237249298924	0.854574600817666\\
15.325599463019	0.854574423267339\\
15.3274739960228	0.854574245775452\\
15.329348528904	0.854574068341978\\
15.3312230616625	0.854573890966892\\
15.3330975942984	0.854573713650168\\
15.3349721268118	0.85457353639178\\
15.3368466592026	0.854573359191703\\
15.3387211914709	0.854573182049911\\
15.3405957236168	0.854573004966378\\
15.3424702556402	0.85457282794108\\
15.3443447875413	0.854572650973989\\
15.34621931932	0.854572474065082\\
15.3480938509765	0.854572297214331\\
15.3499683825107	0.854572120421713\\
15.3518429139227	0.8545719436872\\
15.3537174452125	0.854571767010768\\
15.3555919763802	0.854571590392391\\
15.3574665074258	0.854571413832044\\
15.3593410383493	0.854571237329701\\
15.3612155691509	0.854571060885336\\
15.3630900998304	0.854570884498925\\
15.364964630388	0.854570708170441\\
15.3668391608238	0.85457053189986\\
15.3687136911376	0.854570355687156\\
15.3705882213297	0.854570179532304\\
15.3724627514	0.854570003435278\\
15.3743372813486	0.854569827396053\\
15.3762118111754	0.854569651414603\\
15.3780863408806	0.854569475490904\\
15.3799608704642	0.854569299624929\\
15.3818353999262	0.854569123816654\\
15.3837099292667	0.854568948066053\\
15.3855844584857	0.854568772373102\\
15.3874589875832	0.854568596737774\\
15.3893335165593	0.854568421160044\\
15.3912080454141	0.854568245639888\\
15.3930825741475	0.85456807017728\\
15.3949571027596	0.854567894772195\\
15.3968316312504	0.854567719424607\\
15.39870615962	0.854567544134492\\
15.4005806878685	0.854567368901824\\
15.4024552159958	0.854567193726579\\
15.404329744002	0.85456701860873\\
15.4062042718871	0.854566843548253\\
15.4080787996512	0.854566668545123\\
15.4099533272944	0.854566493599315\\
15.4118278548166	0.854566318710803\\
15.4137023822178	0.854566143879563\\
15.4155769094983	0.85456596910557\\
15.4174514366579	0.854565794388798\\
15.4193259636967	0.854565619729222\\
15.4212004906147	0.854565445126818\\
15.4230750174121	0.85456527058156\\
15.4249495440888	0.854565096093423\\
15.4268240706448	0.854564921662383\\
15.4286985970803	0.854564747288414\\
15.4305731233952	0.854564572971492\\
15.4324476495896	0.854564398711592\\
15.4343221756636	0.854564224508688\\
15.4361967016171	0.854564050362756\\
15.4380712274502	0.854563876273771\\
15.4399457531629	0.854563702241708\\
15.4418202787554	0.854563528266542\\
15.4436948042275	0.854563354348248\\
15.4455693295794	0.854563180486802\\
15.4474438548112	0.854563006682179\\
15.4493183799227	0.854562832934353\\
15.4511929049142	0.854562659243301\\
15.4530674297855	0.854562485608996\\
15.4549419545369	0.854562312031415\\
15.4568164791682	0.854562138510533\\
15.4586910036795	0.854561965046325\\
15.460565528071	0.854561791638766\\
15.4624400523425	0.854561618287832\\
15.4643145764942	0.854561444993497\\
15.4661891005261	0.854561271755738\\
15.4680636244382	0.854561098574529\\
15.4699381482306	0.854560925449846\\
15.4718126719033	0.854560752381664\\
15.4736871954563	0.854560579369959\\
15.4755617188897	0.854560406414706\\
15.4774362422036	0.85456023351588\\
15.4793107653979	0.854560060673456\\
15.4811852884727	0.854559887887411\\
15.483059811428	0.854559715157719\\
15.484934334264	0.854559542484357\\
15.4868088569805	0.854559369867299\\
15.4886833795777	0.85455919730652\\
15.4905579020556	0.854559024801998\\
15.4924324244143	0.854558852353706\\
15.4943069466537	0.854558679961621\\
15.4961814687739	0.854558507625717\\
15.498055990775	0.854558335345972\\
15.4999305126569	0.854558163122359\\
15.5018050344198	0.854557990954855\\
15.5036795560636	0.854557818843435\\
15.5055540775885	0.854557646788075\\
15.5074285989944	0.854557474788751\\
15.5093031202813	0.854557302845437\\
15.5111776414494	0.85455713095811\\
15.5130521624987	0.854556959126746\\
15.5149266834291	0.854556787351319\\
15.5168012042408	0.854556615631806\\
15.5186757249338	0.854556443968183\\
15.520550245508	0.854556272360424\\
15.5224247659637	0.854556100808507\\
15.5242992863007	0.854555929312406\\
15.5261738065191	0.854555757872097\\
15.528048326619	0.854555586487556\\
15.5299228466004	0.854555415158758\\
15.5317973664633	0.854555243885681\\
15.5336718862078	0.854555072668298\\
15.535546405834	0.854554901506587\\
15.5374209253418	0.854554730400523\\
15.5392954447313	0.854554559350081\\
15.5411699640025	0.854554388355238\\
15.5430444831555	0.85455421741597\\
15.5449190021903	0.854554046532252\\
15.5467935211069	0.85455387570406\\
15.5486680399054	0.85455370493137\\
15.5505425585859	0.854553534214158\\
15.5524170771483	0.854553363552401\\
15.5542915955927	0.854553192946073\\
15.5561661139192	0.854553022395151\\
15.5580406321277	0.854552851899611\\
15.5599151502184	0.854552681459428\\
15.5617896681912	0.85455251107458\\
15.5636641860462	0.854552340745041\\
15.5655387037834	0.854552170470788\\
15.5674132214029	0.854552000251797\\
15.5692877389047	0.854551830088044\\
15.5711622562888	0.854551659979504\\
15.5730367735553	0.854551489926155\\
15.5749112907042	0.854551319927972\\
15.5767858077356	0.854551149984931\\
15.5786603246495	0.854550980097008\\
15.5805348414459	0.854550810264179\\
15.5824093581249	0.854550640486421\\
15.5842838746864	0.85455047076371\\
15.5861583911307	0.854550301096022\\
15.5880329074576	0.854550131483332\\
15.5899074236672	0.854549961925618\\
15.5917819397596	0.854549792422855\\
15.5936564557348	0.85454962297502\\
15.5955309715928	0.854549453582088\\
15.5974054873337	0.854549284244037\\
15.5992800029575	0.854549114960842\\
15.6011545184642	0.854548945732479\\
15.603029033854	0.854548776558925\\
15.6049035491267	0.854548607440156\\
15.6067780642825	0.854548438376149\\
15.6086525793215	0.85454826936688\\
15.6105270942435	0.854548100412324\\
15.6124016090487	0.854547931512459\\
15.6142761237372	0.854547762667261\\
15.6161506383088	0.854547593876705\\
15.6180251527638	0.85454742514077\\
15.6198996671021	0.85454725645943\\
15.6217741813237	0.854547087832662\\
15.6236486954288	0.854546919260443\\
15.6255232094173	0.854546750742749\\
15.6273977232892	0.854546582279557\\
15.6292722370447	0.854546413870843\\
15.6311467506837	0.854546245516583\\
15.6330212642063	0.854546077216754\\
15.6348957776125	0.854545908971333\\
15.6367702909024	0.854545740780295\\
15.638644804076	0.854545572643619\\
15.6405193171333	0.854545404561279\\
15.6423938300744	0.854545236533252\\
15.6442683428993	0.854545068559516\\
15.646142855608	0.854544900640046\\
15.6480173682006	0.85454473277482\\
15.6498918806772	0.854544564963813\\
15.6517663930377	0.854544397207003\\
15.6536409052822	0.854544229504366\\
15.6555154174107	0.854544061855879\\
15.6573899294233	0.854543894261518\\
15.65926444132	0.85454372672126\\
15.6611389531009	0.854543559235081\\
15.6630134647659	0.854543391802959\\
15.6648879763152	0.85454322442487\\
15.6667624877487	0.854543057100791\\
15.6686369990665	0.854542889830698\\
15.6705115102687	0.854542722614569\\
15.6723860213552	0.854542555452379\\
15.6742605323261	0.854542388344106\\
15.6761350431814	0.854542221289726\\
15.6780095539212	0.854542054289217\\
15.6798840645456	0.854541887342555\\
15.6817585750545	0.854541720449716\\
15.683633085448	0.854541553610679\\
15.6855075957261	0.854541386825418\\
15.6873821058889	0.854541220093912\\
15.6892566159364	0.854541053416138\\
15.6911311258686	0.854540886792071\\
15.6930056356856	0.854540720221689\\
15.6948801453874	0.85454055370497\\
15.696754654974	0.854540387241888\\
15.6986291644455	0.854540220832423\\
15.700503673802	0.85454005447655\\
15.7023781830434	0.854539888174247\\
15.7042526921698	0.85453972192549\\
15.7061272011813	0.854539555730257\\
15.7080017100778	0.854539389588524\\
15.7098762188594	0.854539223500269\\
15.7117507275262	0.854539057465468\\
15.7136252360781	0.854538891484098\\
15.7154997445153	0.854538725556137\\
15.7173742528377	0.854538559681562\\
15.7192487610454	0.854538393860349\\
15.7211232691384	0.854538228092476\\
15.7229977771168	0.854538062377919\\
15.7248722849806	0.854537896716657\\
15.7267467927298	0.854537731108665\\
15.7286213003645	0.854537565553921\\
15.7304958078847	0.854537400052403\\
15.7323703152904	0.854537234604087\\
15.7342448225818	0.85453706920895\\
15.7361193297587	0.85453690386697\\
15.7379938368213	0.854536738578124\\
15.7398683437696	0.854536573342389\\
15.7417428506037	0.854536408159743\\
15.7436173573235	0.854536243030161\\
15.745491863929	0.854536077953623\\
15.7473663704205	0.854535912930104\\
15.7492408767978	0.854535747959582\\
15.751115383061	0.854535583042035\\
15.7529898892102	0.85453541817744\\
15.7548643952453	0.854535253365773\\
15.7567389011665	0.854535088607013\\
15.7586134069737	0.854534923901136\\
15.760487912667	0.854534759248121\\
15.7623624182465	0.854534594647943\\
15.7642369237121	0.854534430100581\\
15.7661114290639	0.854534265606013\\
15.7679859343019	0.854534101164214\\
15.7698604394263	0.854533936775164\\
15.7717349444369	0.854533772438838\\
15.7736094493339	0.854533608155215\\
15.7754839541173	0.854533443924272\\
15.777358458787	0.854533279745986\\
15.7792329633433	0.854533115620336\\
15.781107467786	0.854532951547297\\
15.7829819721153	0.854532787526848\\
15.7848564763311	0.854532623558967\\
15.7867309804335	0.85453245964363\\
15.7886054844226	0.854532295780816\\
15.7904799882983	0.854532131970501\\
15.7923544920608	0.854531968212664\\
15.7942289957099	0.854531804507282\\
15.7961034992459	0.854531640854332\\
15.7979780026687	0.854531477253792\\
15.7998525059783	0.85453131370564\\
15.8017270091748	0.854531150209852\\
15.8036015122583	0.854530986766408\\
15.8054760152287	0.854530823375284\\
15.8073505180861	0.854530660036458\\
15.8092250208305	0.854530496749907\\
15.811099523462	0.85453033351561\\
15.8129740259806	0.854530170333544\\
15.8148485283864	0.854530007203686\\
15.8167230306793	0.854529844126015\\
15.8185975328594	0.854529681100508\\
15.8204720349268	0.854529518127142\\
15.8223465368815	0.854529355205896\\
15.8242210387235	0.854529192336747\\
15.8260955404528	0.854529029519673\\
15.8279700420696	0.854528866754652\\
15.8298445435737	0.854528704041661\\
15.8317190449653	0.854528541380679\\
15.8335935462445	0.854528378771683\\
15.8354680474111	0.85452821621465\\
15.8373425484653	0.854528053709559\\
15.8392170494072	0.854527891256388\\
15.8410915502366	0.854527728855113\\
15.8429660509538	0.854527566505714\\
15.8448405515587	0.854527404208169\\
15.8467150520513	0.854527241962454\\
15.8485895524317	0.854527079768548\\
15.8504640526999	0.854526917626428\\
15.8523385528559	0.854526755536074\\
15.8542130528999	0.854526593497462\\
15.8560875528318	0.85452643151057\\
15.8579620526516	0.854526269575377\\
15.8598365523595	0.85452610769186\\
15.8617110519553	0.854525945859998\\
15.8635855514393	0.854525784079768\\
15.8654600508113	0.854525622351148\\
15.8673345500715	0.854525460674117\\
15.8692090492199	0.854525299048652\\
15.8710835482565	0.854525137474732\\
15.8729580471813	0.854524975952334\\
15.8748325459944	0.854524814481436\\
15.8767070446958	0.854524653062017\\
15.8785815432856	0.854524491694055\\
15.8804560417637	0.854524330377527\\
15.8823305401303	0.854524169112413\\
15.8842050383853	0.854524007898689\\
15.8860795365288	0.854523846736334\\
15.8879540345609	0.854523685625327\\
15.8898285324815	0.854523524565645\\
15.8917030302907	0.854523363557266\\
15.8935775279886	0.854523202600169\\
15.8954520255751	0.854523041694332\\
15.8973265230504	0.854522880839734\\
15.8992010204143	0.854522720036351\\
15.9010755176671	0.854522559284163\\
15.9029500148086	0.854522398583147\\
15.904824511839	0.854522237933283\\
15.9066990087583	0.854522077334548\\
15.9085735055665	0.85452191678692\\
15.9104480022636	0.854521756290378\\
15.9123224988498	0.8545215958449\\
15.9141969953249	0.854521435450465\\
15.9160714916892	0.85452127510705\\
15.9179459879425	0.854521114814634\\
15.9198204840849	0.854520954573195\\
15.9216949801165	0.854520794382712\\
15.9235694760373	0.854520634243163\\
15.9254439718473	0.854520474154526\\
15.9273184675466	0.854520314116781\\
15.9291929631352	0.854520154129904\\
15.9310674586131	0.854519994193875\\
15.9329419539804	0.854519834308671\\
15.9348164492371	0.854519674474272\\
15.9366909443832	0.854519514690656\\
15.9385654394189	0.854519354957802\\
15.940439934344	0.854519195275687\\
15.9423144291587	0.85451903564429\\
15.9441889238629	0.854518876063589\\
15.9460634184568	0.854518716533564\\
15.9479379129404	0.854518557054193\\
15.9498124073136	0.854518397625454\\
15.9516869015765	0.854518238247325\\
15.9535613957292	0.854518078919786\\
15.9554358897717	0.854517919642815\\
15.957310383704	0.85451776041639\\
15.9591848775261	0.85451760124049\\
15.9610593712382	0.854517442115093\\
15.9629338648402	0.854517283040179\\
15.9648083583321	0.854517124015725\\
15.9666828517141	0.854516965041711\\
15.9685573449861	0.854516806118114\\
15.9704318381481	0.854516647244915\\
15.9723063312003	0.85451648842209\\
15.9741808241426	0.85451632964962\\
15.976055316975	0.854516170927482\\
15.9779298096977	0.854516012255655\\
15.9798043023106	0.854515853634119\\
15.9816787948138	0.854515695062851\\
15.9835532872073	0.85451553654183\\
15.9854277794911	0.854515378071036\\
15.9873022716654	0.854515219650447\\
15.98917676373	0.854515061280041\\
15.9910512556851	0.854514902959798\\
15.9929257475306	0.854514744689696\\
15.9948002392667	0.854514586469714\\
15.9966747308934	0.85451442829983\\
15.9985492224106	0.854514270180025\\
16.0004237138184	0.854514112110275\\
16.0022982051169	0.854513954090561\\
16.0041726963061	0.854513796120861\\
16.006047187386	0.854513638201154\\
16.0079216783567	0.854513480331419\\
16.0097961692181	0.854513322511635\\
16.0116706599704	0.85451316474178\\
16.0135451506135	0.854513007021834\\
16.0154196411476	0.854512849351775\\
16.0172941315726	0.854512691731582\\
16.0191686218885	0.854512534161235\\
16.0210431120954	0.854512376640711\\
16.0229176021934	0.854512219169991\\
16.0247920921824	0.854512061749054\\
16.0266665820626	0.854511904377877\\
16.0285410718338	0.85451174705644\\
16.0304155614963	0.854511589784723\\
16.0322900510499	0.854511432562703\\
16.0341645404948	0.854511275390361\\
16.0360390298309	0.854511118267674\\
16.0379135190584	0.854510961194623\\
16.0397880081772	0.854510804171187\\
16.0416624971874	0.854510647197343\\
16.043536986089	0.854510490273072\\
16.045411474882	0.854510333398353\\
16.0472859635665	0.854510176573164\\
16.0491604521425	0.854510019797485\\
16.05103494061	0.854509863071294\\
16.0529094289692	0.854509706394572\\
16.0547839172199	0.854509549767297\\
16.0566584053623	0.854509393189448\\
16.0585328933964	0.854509236661004\\
16.0604073813222	0.854509080181945\\
16.0622818691397	0.85450892375225\\
16.0641563568491	0.854508767371898\\
16.0660308444502	0.854508611040868\\
16.0679053319432	0.854508454759139\\
16.0697798193281	0.854508298526691\\
16.0716543066048	0.854508142343503\\
16.0735287937736	0.854507986209554\\
16.0754032808343	0.854507830124824\\
16.0772777677871	0.854507674089291\\
16.0791522546319	0.854507518102936\\
16.0810267413688	0.854507362165736\\
16.0829012279978	0.854507206277672\\
16.084775714519	0.854507050438723\\
16.0866502009323	0.854506894648869\\
16.0885246872379	0.854506738908087\\
16.0903991734357	0.854506583216359\\
16.0922736595259	0.854506427573663\\
16.0941481455083	0.854506271979979\\
16.0960226313831	0.854506116435286\\
16.0978971171503	0.854505960939563\\
16.0997716028099	0.85450580549279\\
16.101646088362	0.854505650094947\\
16.1035205738066	0.854505494746012\\
16.1053950591437	0.854505339445965\\
16.1072695443734	0.854505184194786\\
16.1091440294956	0.854505028992454\\
16.1110185145105	0.854504873838948\\
16.112892999418	0.854504718734249\\
16.1147674842183	0.854504563678335\\
16.1166419689112	0.854504408671186\\
16.118516453497	0.854504253712781\\
16.1203909379755	0.854504098803101\\
16.1222654223468	0.854503943942124\\
16.1241399066111	0.85450378912983\\
16.1260143907682	0.854503634366199\\
16.1278888748182	0.85450347965121\\
16.1297633587612	0.854503324984843\\
16.1316378425972	0.854503170367078\\
16.1335123263262	0.854503015797893\\
16.1353868099483	0.854502861277269\\
16.1372612934635	0.854502706805186\\
16.1391357768718	0.854502552381622\\
16.1410102601732	0.854502398006558\\
16.1428847433679	0.854502243679973\\
16.1447592264558	0.854502089401847\\
16.1466337094369	0.854501935172159\\
16.1485081923114	0.85450178099089\\
16.1503826750792	0.854501626858018\\
16.1522571577403	0.854501472773525\\
16.1541316402949	0.854501318737388\\
16.1560061227428	0.854501164749589\\
16.1578806050843	0.854501010810106\\
16.1597550873192	0.85450085691892\\
16.1616295694476	0.85450070307601\\
16.1635040514696	0.854500549281357\\
16.1653785333853	0.854500395534939\\
16.1672530151945	0.854500241836737\\
16.1691274968974	0.85450008818673\\
16.171001978494	0.854499934584899\\
16.1728764599843	0.854499781031222\\
16.1747509413684	0.854499627525681\\
16.1766254226463	0.854499474068255\\
16.178499903818	0.854499320658923\\
16.1803743848835	0.854499167297665\\
16.182248865843	0.854499013984462\\
16.1841233466964	0.854498860719294\\
16.1859978274437	0.854498707502139\\
16.1878723080851	0.854498554332979\\
16.1897467886204	0.854498401211792\\
16.1916212690498	0.85449824813856\\
16.1934957493733	0.854498095113261\\
16.195370229591	0.854497942135876\\
16.1972447097028	0.854497789206385\\
16.1991191897088	0.854497636324768\\
16.200993669609	0.854497483491005\\
16.2028681494034	0.854497330705075\\
16.2047426290922	0.854497177966959\\
16.2066171086752	0.854497025276636\\
16.2084915881527	0.854496872634088\\
16.2103660675245	0.854496720039293\\
16.2122405467907	0.854496567492232\\
16.2141150259514	0.854496414992886\\
16.2159895050066	0.854496262541233\\
16.2178639839562	0.854496110137254\\
16.2197384628005	0.854495957780929\\
16.2216129415393	0.854495805472239\\
16.2234874201728	0.854495653211163\\
16.2253618987009	0.854495500997681\\
16.2272363771236	0.854495348831774\\
16.2291108554411	0.854495196713421\\
16.2309853336534	0.854495044642604\\
16.2328598117604	0.854494892619301\\
16.2347342897622	0.854494740643494\\
16.2366087676589	0.854494588715162\\
16.2384832454505	0.854494436834286\\
16.2403577231369	0.854494285000846\\
16.2422322007183	0.854494133214821\\
16.2441066781947	0.854493981476193\\
16.2459811555661	0.854493829784941\\
16.2478556328326	0.854493678141046\\
16.2497301099941	0.854493526544488\\
16.2516045870507	0.854493374995247\\
16.2534790640025	0.854493223493304\\
16.2553535408494	0.854493072038638\\
16.2572280175916	0.854492920631231\\
16.2591024942289	0.854492769271062\\
16.2609769707616	0.854492617958111\\
16.2628514471895	0.85449246669236\\
16.2647259235128	0.854492315473787\\
16.2666003997315	0.854492164302375\\
16.2684748758455	0.854492013178103\\
16.270349351855	0.854491862100951\\
16.2722238277599	0.854491711070899\\
16.2740983035604	0.854491560087929\\
16.2759727792563	0.854491409152021\\
16.2778472548479	0.854491258263154\\
16.279721730335	0.85449110742131\\
16.2815962057178	0.854490956626469\\
16.2834706809962	0.854490805878611\\
16.2853451561703	0.854490655177716\\
16.2872196312401	0.854490504523766\\
16.2890941062057	0.854490353916741\\
16.290968581067	0.85449020335662\\
16.2928430558242	0.854490052843385\\
16.2947175304773	0.854489902377016\\
16.2965920050262	0.854489751957494\\
16.298466479471	0.854489601584799\\
16.3003409538118	0.854489451258912\\
16.3022154280486	0.854489300979813\\
16.3040899021814	0.854489150747482\\
16.3059643762102	0.854489000561901\\
16.3078388501351	0.85448885042305\\
16.3097133239561	0.854488700330909\\
16.3115877976733	0.854488550285459\\
16.3134622712866	0.854488400286681\\
16.3153367447962	0.854488250334555\\
16.3172112182019	0.854488100429062\\
16.319085691504	0.854487950570182\\
16.3209601647023	0.854487800757897\\
16.322834637797	0.854487650992186\\
16.3247091107881	0.854487501273031\\
16.3265835836755	0.854487351600412\\
16.3284580564594	0.85448720197431\\
16.3303325291398	0.854487052394705\\
16.3322070017166	0.854486902861579\\
16.33408147419	0.854486753374911\\
16.3359559465599	0.854486603934683\\
16.3378304188264	0.854486454540876\\
16.3397048909895	0.854486305193469\\
16.3415793630493	0.854486155892444\\
16.3434538350058	0.854486006637782\\
16.345328306859	0.854485857429464\\
16.3472027786089	0.854485708267469\\
16.3490772502557	0.854485559151779\\
16.3509517217992	0.854485410082376\\
16.3528261932396	0.854485261059238\\
16.3547006645769	0.854485112082348\\
16.356575135811	0.854484963151687\\
16.3584496069422	0.854484814267234\\
16.3603240779702	0.854484665428971\\
16.3621985488953	0.854484516636879\\
16.3640730197175	0.854484367890939\\
16.3659474904367	0.854484219191131\\
16.367821961053	0.854484070537437\\
16.3696964315664	0.854483921929836\\
16.371570901977	0.854483773368311\\
16.3734453722848	0.854483624852843\\
16.3753198424898	0.854483476383411\\
16.377194312592	0.854483327959997\\
16.3790687825916	0.854483179582582\\
16.3809432524885	0.854483031251147\\
16.3828177222827	0.854482882965673\\
16.3846921919743	0.854482734726141\\
16.3865666615634	0.854482586532532\\
16.3884411310499	0.854482438384826\\
16.3903156004338	0.854482290283006\\
16.3921900697153	0.854482142227051\\
16.3940645388943	0.854481994216943\\
16.3959390079709	0.854481846252663\\
16.3978134769451	0.854481698334192\\
16.399687945817	0.854481550461511\\
16.4015624145865	0.854481402634602\\
16.4034368832537	0.854481254853444\\
16.4053113518186	0.85448110711802\\
16.4071858202813	0.85448095942831\\
16.4090602886418	0.854480811784295\\
16.4109347569002	0.854480664185957\\
16.4128092250563	0.854480516633277\\
16.4146836931104	0.854480369126236\\
16.4165581610624	0.854480221664815\\
16.4184326289123	0.854480074248995\\
16.4203070966603	0.854479926878757\\
16.4221815643062	0.854479779554082\\
16.4240560318502	0.854479632274953\\
16.4259304992923	0.854479485041349\\
16.4278049666324	0.854479337853252\\
16.4296794338707	0.854479190710644\\
16.4315539010072	0.854479043613505\\
16.4334283680419	0.854478896561817\\
16.4353028349748	0.854478749555561\\
16.437177301806	0.854478602594718\\
16.4390517685355	0.85447845567927\\
16.4409262351633	0.854478308809197\\
16.4428007016895	0.854478161984481\\
16.4446751681141	0.854478015205104\\
16.446549634437	0.854477868471047\\
16.4484241006585	0.85447772178229\\
16.4502985667784	0.854477575138816\\
16.4521730327968	0.854477428540606\\
16.4540474987138	0.85447728198764\\
16.4559219645294	0.854477135479901\\
16.4577964302435	0.854476989017369\\
16.4596708958563	0.854476842600026\\
16.4615453613678	0.854476696227854\\
16.463419826778	0.854476549900834\\
16.4652942920869	0.854476403618946\\
16.4671687572946	0.854476257382174\\
16.469043222401	0.854476111190497\\
16.4709176874063	0.854475965043898\\
16.4727921523105	0.854475818942358\\
16.4746666171135	0.854475672885858\\
16.4765410818155	0.854475526874379\\
16.4784155464164	0.854475380907905\\
16.4802900109162	0.854475234986414\\
16.4821644753151	0.85447508910989\\
16.4840389396131	0.854474943278314\\
16.4859134038101	0.854474797491667\\
16.4877878679062	0.854474651749931\\
16.4896623319015	0.854474506053087\\
16.4915367957959	0.854474360401116\\
16.4934112595895	0.854474214794001\\
16.4952857232824	0.854474069231723\\
16.4971601868745	0.854473923714263\\
16.4990346503658	0.854473778241603\\
16.5009091137566	0.854473632813725\\
16.5027835770466	0.85447348743061\\
16.5046580402361	0.85447334209224\\
16.5065325033249	0.854473196798596\\
16.5084069663132	0.85447305154966\\
16.510281429201	0.854472906345413\\
16.5121558919883	0.854472761185838\\
16.5140303546751	0.854472616070916\\
16.5159048172615	0.854472471000628\\
16.5177792797475	0.854472325974956\\
16.5196537421331	0.854472180993883\\
16.5215282044184	0.854472036057389\\
16.5234026666033	0.854471891165456\\
16.525277128688	0.854471746318066\\
16.5271515906725	0.8544716015152\\
16.5290260525567	0.854471456756842\\
16.5309005143407	0.854471312042971\\
16.5327749760246	0.85447116737357\\
16.5346494376083	0.854471022748621\\
16.5365238990919	0.854470878168105\\
16.5383983604755	0.854470733632004\\
16.5402728217591	0.8544705891403\\
16.5421472829426	0.854470444692974\\
16.5440217440262	0.85447030029001\\
16.5458962050098	0.854470155931387\\
16.5477706658935	0.854470011617089\\
16.5496451266773	0.854469867347096\\
16.5515195873613	0.854469723121391\\
16.5533940479455	0.854469578939956\\
16.5552685084298	0.854469434802773\\
16.5571429688144	0.854469290709822\\
16.5590174290993	0.854469146661087\\
16.5608918892845	0.854469002656549\\
16.56276634937	0.85446885869619\\
16.5646408093559	0.854468714779991\\
16.5665152692421	0.854468570907936\\
16.5683897290288	0.854468427080005\\
16.5702641887159	0.854468283296181\\
16.5721386483035	0.854468139556445\\
16.5740131077917	0.854467995860779\\
16.5758875671803	0.854467852209166\\
16.5777620264696	0.854467708601588\\
16.5796364856594	0.854467565038026\\
16.5815109447499	0.854467421518462\\
16.5833854037411	0.854467278042878\\
16.5852598626329	0.854467134611257\\
16.5871343214255	0.854466991223581\\
16.5890087801188	0.85446684787983\\
16.5908832387129	0.854466704579988\\
16.5927576972079	0.854466561324036\\
16.5946321556036	0.854466418111957\\
16.5965066139003	0.854466274943733\\
16.5983810720978	0.854466131819345\\
16.6002555301963	0.854465988738775\\
16.6021299881958	0.854465845702006\\
16.6040044460962	0.854465702709021\\
16.6058789038977	0.8544655597598\\
16.6077533616003	0.854465416854326\\
16.6096278192039	0.854465273992581\\
16.6115022767087	0.854465131174548\\
16.6133767341145	0.854464988400208\\
16.6152511914216	0.854464845669544\\
16.6171256486299	0.854464702982537\\
16.6190001057394	0.854464560339171\\
16.6208745627502	0.854464417739426\\
16.6227490196623	0.854464275183286\\
16.6246234764757	0.854464132670732\\
16.6264979331905	0.854463990201747\\
16.6283723898066	0.854463847776312\\
16.6302468463242	0.854463705394411\\
16.6321213027432	0.854463563056025\\
16.6339957590637	0.854463420761136\\
16.6358702152857	0.854463278509728\\
16.6377446714093	0.854463136301781\\
16.6396191274344	0.854462994137279\\
16.6414935833611	0.854462852016203\\
16.6433680391894	0.854462709938536\\
16.6452424949194	0.854462567904261\\
16.6471169505511	0.854462425913359\\
16.6489914060845	0.854462283965812\\
16.6508658615196	0.854462142061604\\
16.6527403168566	0.854462000200716\\
16.6546147720953	0.854461858383131\\
16.6564892272359	0.854461716608831\\
16.6583636822783	0.854461574877798\\
16.6602381372226	0.854461433190015\\
16.6621125920689	0.854461291545465\\
16.6639870468171	0.854461149944129\\
16.6658615014673	0.85446100838599\\
16.6677359560195	0.85446086687103\\
16.6696104104737	0.854460725399232\\
16.67148486483	0.854460583970579\\
16.6733593190885	0.854460442585052\\
16.675233773249	0.854460301242634\\
16.6771082273118	0.854460159943307\\
16.6789826812767	0.854460018687055\\
16.6808571351438	0.854459877473859\\
16.6827315889132	0.854459736303702\\
16.6846060425849	0.854459595176566\\
16.6864804961589	0.854459454092434\\
16.6883549496352	0.854459313051289\\
16.6902294030139	0.854459172053112\\
16.692103856295	0.854459031097887\\
16.6939783094785	0.854458890185595\\
16.6958527625645	0.854458749316221\\
16.697727215553	0.854458608489745\\
16.699601668444	0.85445846770615\\
16.7014761212375	0.85445832696542\\
16.7033505739337	0.854458186267537\\
16.7052250265324	0.854458045612482\\
16.7070994790338	0.85445790500024\\
16.7089739314378	0.854457764430792\\
16.7108483837445	0.854457623904121\\
16.712722835954	0.85445748342021\\
16.7145972880662	0.854457342979041\\
16.7164717400812	0.854457202580597\\
16.718346191999	0.85445706222486\\
16.7202206438196	0.854456921911814\\
16.7220950955431	0.85445678164144\\
16.7239695471695	0.854456641413722\\
16.7258439986989	0.854456501228643\\
16.7277184501312	0.854456361086184\\
16.7295929014665	0.854456220986329\\
16.7314673527048	0.85445608092906\\
16.7333418038462	0.85445594091436\\
16.7352162548906	0.854455800942212\\
16.7370907058382	0.854455661012598\\
16.7389651566889	0.854455521125501\\
16.7408396074427	0.854455381280905\\
16.7427140580998	0.854455241478791\\
16.74458850866	0.854455101719142\\
16.7464629591235	0.854454962001942\\
16.7483374094903	0.854454822327173\\
16.7502118597605	0.854454682694817\\
16.7520863099339	0.854454543104858\\
16.7539607600107	0.854454403557279\\
16.755835209991	0.854454264052062\\
16.7577096598746	0.854454124589189\\
16.7595841096617	0.854453985168645\\
16.7614585593523	0.854453845790412\\
16.7633330089464	0.854453706454472\\
16.7652074584441	0.854453567160808\\
16.7670819078453	0.854453427909404\\
16.7689563571501	0.854453288700242\\
16.7708308063586	0.854453149533305\\
16.7727052554707	0.854453010408575\\
16.7745797044865	0.854452871326037\\
16.7764541534061	0.854452732285673\\
16.7783286022294	0.854452593287465\\
16.7802030509564	0.854452454331397\\
16.7820774995873	0.854452315417451\\
16.783951948122	0.854452176545611\\
16.7858263965605	0.85445203771586\\
16.787700844903	0.85445189892818\\
16.7895752931494	0.854451760182554\\
16.7914497412997	0.854451621478966\\
16.793324189354	0.854451482817398\\
16.7951986373123	0.854451344197833\\
16.7970730851747	0.854451205620255\\
16.7989475329411	0.854451067084646\\
16.8008219806116	0.854450928590989\\
16.8026964281863	0.854450790139269\\
16.8045708756651	0.854450651729466\\
16.8064453230481	0.854450513361565\\
16.8083197703352	0.854450375035549\\
16.8101942175267	0.8544502367514\\
16.8120686646224	0.854450098509102\\
16.8139431116224	0.854449960308638\\
16.8158175585267	0.854449822149991\\
16.8176920053354	0.854449684033144\\
16.8195664520484	0.85444954595808\\
16.8214408986659	0.854449407924782\\
16.8233153451878	0.854449269933234\\
16.8251897916142	0.854449131983418\\
16.8270642379451	0.854448994075318\\
16.8289386841805	0.854448856208916\\
16.8308131303205	0.854448718384197\\
16.832687576365	0.854448580601143\\
16.8345620223142	0.854448442859737\\
16.836436468168	0.854448305159962\\
16.8383109139265	0.854448167501803\\
16.8401853595897	0.854448029885241\\
16.8420598051576	0.85444789231026\\
16.8439342506302	0.854447754776844\\
16.8458086960077	0.854447617284975\\
16.84768314129	0.854447479834637\\
16.8495575864771	0.854447342425813\\
16.8514320315691	0.854447205058486\\
16.8533064765659	0.85444706773264\\
16.8551809214677	0.854446930448258\\
16.8570553662745	0.854446793205323\\
16.8589298109863	0.854446656003818\\
16.860804255603	0.854446518843727\\
16.8626787001248	0.854446381725033\\
16.8645531445517	0.85444624464772\\
16.8664275888837	0.85444610761177\\
16.8683020331208	0.854445970617167\\
16.8701764772631	0.854445833663894\\
16.8720509213105	0.854445696751935\\
16.8739253652632	0.854445559881273\\
16.8757998091211	0.854445423051892\\
16.8776742528843	0.854445286263774\\
16.8795486965527	0.854445149516903\\
16.8814231401265	0.854445012811263\\
16.8832975836057	0.854444876146837\\
16.8851720269902	0.854444739523608\\
16.8870464702802	0.854444602941559\\
16.8889209134756	0.854444466400675\\
16.8907953565764	0.854444329900939\\
16.8926697995828	0.854444193442334\\
16.8945442424947	0.854444057024843\\
16.8964186853121	0.85444392064845\\
16.8982931280352	0.854443784313138\\
16.9001675706638	0.854443648018892\\
16.9020420131981	0.854443511765693\\
16.903916455638	0.854443375553527\\
16.9057908979836	0.854443239382376\\
16.907665340235	0.854443103252224\\
16.9095397823921	0.854442967163054\\
16.911414224455	0.85444283111485\\
16.9132886664237	0.854442695107595\\
16.9151631082982	0.854442559141273\\
16.9170375500786	0.854442423215868\\
16.9189119917649	0.854442287331363\\
16.9207864333571	0.854442151487742\\
16.9226608748553	0.854442015684988\\
16.9245353162594	0.854441879923085\\
16.9264097575696	0.854441744202015\\
16.9282841987858	0.854441608521764\\
16.930158639908	0.854441472882315\\
16.9320330809364	0.854441337283651\\
16.9339075218708	0.854441201725755\\
16.9357819627114	0.854441066208612\\
16.9376564034582	0.854440930732205\\
16.9395308441111	0.854440795296518\\
16.9414052846703	0.854440659901534\\
16.9432797251358	0.854440524547237\\
16.9451541655075	0.854440389233611\\
16.9470286057856	0.854440253960639\\
16.94890304597	0.854440118728306\\
16.9507774860608	0.854439983536594\\
16.952651926058	0.854439848385487\\
16.9545263659616	0.85443971327497\\
16.9564008057716	0.854439578205026\\
16.9582752454881	0.854439443175638\\
16.9601496851112	0.854439308186791\\
16.9620241246408	0.854439173238468\\
16.9638985640769	0.854439038330652\\
16.9657730034197	0.854438903463329\\
16.967647442669	0.854438768636481\\
16.969521881825	0.854438633850092\\
16.9713963208877	0.854438499104145\\
16.9732707598571	0.854438364398626\\
16.9751451987332	0.854438229733517\\
16.9770196375161	0.854438095108803\\
16.9788940762058	0.854437960524467\\
16.9807685148023	0.854437825980493\\
16.9826429533056	0.854437691476864\\
16.9845173917158	0.854437557013566\\
16.9863918300329	0.854437422590581\\
16.988266268257	0.854437288207893\\
16.990140706388	0.854437153865487\\
16.9920151444259	0.854437019563346\\
16.9938895823709	0.854436885301453\\
16.995764020223	0.854436751079794\\
16.997638457982	0.854436616898352\\
16.9995128956482	0.85443648275711\\
17.0013873332215	0.854436348656053\\
17.003261770702	0.854436214595164\\
17.0051362080896	0.854436080574428\\
17.0070106453845	0.854435946593829\\
17.0088850825865	0.854435812653349\\
17.0107595196959	0.854435678752975\\
17.0126339567125	0.854435544892688\\
17.0145083936364	0.854435411072474\\
17.0163828304677	0.854435277292316\\
17.0182572672063	0.854435143552198\\
17.0201317038524	0.854435009852104\\
17.0220061404058	0.854434876192019\\
17.0238805768667	0.854434742571926\\
17.0257550132351	0.854434608991809\\
17.0276294495111	0.854434475451653\\
17.0295038856945	0.854434341951441\\
17.0313783217855	0.854434208491157\\
17.0332527577841	0.854434075070786\\
17.0351271936903	0.854433941690311\\
17.0370016295042	0.854433808349716\\
17.0388760652257	0.854433675048987\\
17.040750500855	0.854433541788106\\
17.0426249363919	0.854433408567057\\
17.0444993718366	0.854433275385826\\
17.0463738071891	0.854433142244396\\
17.0482482424495	0.854433009142751\\
17.0501226776176	0.854432876080875\\
17.0519971126936	0.854432743058753\\
17.0538715476775	0.854432610076368\\
17.0557459825693	0.854432477133705\\
17.0576204173691	0.854432344230747\\
17.0594948520769	0.85443221136748\\
17.0613692866926	0.854432078543886\\
17.0632437212164	0.854431945759951\\
17.0651181556482	0.854431813015659\\
17.0669925899882	0.854431680310993\\
17.0688670242362	0.854431547645938\\
17.0707414583923	0.854431415020479\\
17.0726158924567	0.854431282434598\\
17.0744903264292	0.854431149888282\\
17.07636476031	0.854431017381513\\
17.0782391940989	0.854430884914276\\
17.0801136277962	0.854430752486555\\
17.0819880614018	0.854430620098335\\
17.0838624949157	0.8544304877496\\
17.0857369283379	0.854430355440334\\
17.0876113616686	0.854430223170521\\
17.0894857949076	0.854430090940146\\
17.0913602280551	0.854429958749193\\
17.0932346611111	0.854429826597646\\
17.0951090940755	0.85442969448549\\
17.0969835269485	0.854429562412709\\
17.09885795973	0.854429430379286\\
17.1007323924201	0.854429298385208\\
17.1026068250187	0.854429166430457\\
17.1044812575261	0.854429034515019\\
17.106355689942	0.854428902638877\\
17.1082301222667	0.854428770802016\\
17.1101045545	0.854428639004421\\
17.1119789866421	0.854428507246075\\
17.113853418693	0.854428375526963\\
17.1157278506526	0.85442824384707\\
17.1176022825211	0.85442811220638\\
17.1194767142984	0.854427980604877\\
17.1213511459845	0.854427849042546\\
17.1232255775796	0.854427717519371\\
17.1251000090836	0.854427586035337\\
17.1269744404966	0.854427454590428\\
17.1288488718185	0.854427323184628\\
17.1307233030494	0.854427191817923\\
17.1325977341894	0.854427060490295\\
17.1344721652384	0.854426929201731\\
17.1363465961965	0.854426797952214\\
17.1382210270638	0.854426666741729\\
17.1400954578401	0.854426535570261\\
17.1419698885256	0.854426404437793\\
17.1438443191204	0.854426273344311\\
17.1457187496243	0.854426142289799\\
17.1475931800375	0.854426011274241\\
17.14946761036	0.854425880297622\\
17.1513420405918	0.854425749359927\\
17.1532164707329	0.85442561846114\\
17.1550909007833	0.854425487601246\\
17.1569653307431	0.854425356780229\\
17.1588397606124	0.854425225998073\\
17.1607141903911	0.854425095254765\\
17.1625886200792	0.854424964550287\\
17.1644630496768	0.854424833884624\\
17.166337479184	0.854424703257762\\
17.1682119086007	0.854424572669685\\
17.1700863379269	0.854424442120377\\
17.1719607671628	0.854424311609823\\
17.1738351963083	0.854424181138008\\
17.1757096253634	0.854424050704916\\
17.1775840543282	0.854423920310532\\
17.1794584832027	0.854423789954841\\
17.1813329119869	0.854423659637827\\
17.1832073406809	0.854423529359475\\
17.1850817692847	0.85442339911977\\
17.1869561977983	0.854423268918696\\
17.1888306262217	0.854423138756238\\
17.1907050545549	0.854423008632381\\
17.1925794827981	0.854422878547109\\
17.1944539109512	0.854422748500408\\
17.1963283390142	0.854422618492261\\
17.1982027669872	0.854422488522655\\
17.2000771948702	0.854422358591572\\
17.2019516226632	0.854422228698999\\
17.2038260503662	0.85442209884492\\
17.2057004779794	0.854421969029319\\
17.2075749055026	0.854421839252182\\
17.209449332936	0.854421709513493\\
17.2113237602795	0.854421579813237\\
17.2131981875332	0.854421450151399\\
17.2150726146971	0.854421320527963\\
17.2169470417712	0.854421190942915\\
17.2188214687556	0.85442106139624\\
17.2206958956503	0.854420931887921\\
17.2225703224553	0.854420802417944\\
17.2244447491706	0.854420672986294\\
17.2263191757963	0.854420543592955\\
17.2281936023324	0.854420414237913\\
17.2300680287789	0.854420284921153\\
17.2319424551358	0.854420155642658\\
17.2338168814033	0.854420026402415\\
17.2356913075812	0.854419897200407\\
17.2375657336696	0.85441976803662\\
17.2394401596686	0.854419638911039\\
17.2413145855782	0.854419509823649\\
17.2431890113984	0.854419380774435\\
17.2450634371292	0.85441925176338\\
17.2469378627706	0.854419122790472\\
17.2488122883228	0.854418993855693\\
17.2506867137856	0.85441886495903\\
17.2525611391592	0.854418736100467\\
17.2544355644436	0.85441860727999\\
17.2563099896387	0.854418478497582\\
17.2581844147446	0.85441834975323\\
17.2600588397614	0.854418221046918\\
17.261933264689	0.854418092378631\\
17.2638076895276	0.854417963748354\\
17.265682114277	0.854417835156072\\
17.2675565389374	0.854417706601771\\
17.2694309635088	0.854417578085434\\
17.2713053879912	0.854417449607048\\
17.2731798123846	0.854417321166597\\
17.275054236689	0.854417192764066\\
17.2769286609045	0.854417064399441\\
17.2788030850311	0.854416936072706\\
17.2806775090689	0.854416807783847\\
17.2825519330178	0.854416679532848\\
17.2844263568778	0.854416551319694\\
17.2863007806491	0.854416423144371\\
17.2881752043316	0.854416295006864\\
17.2900496279254	0.854416166907158\\
17.2919240514304	0.854416038845238\\
17.2937984748468	0.854415910821089\\
17.2956728981745	0.854415782834697\\
17.2975473214135	0.854415654886045\\
17.299421744564	0.85441552697512\\
17.3012961676258	0.854415399101907\\
17.3031705905991	0.854415271266391\\
17.3050450134839	0.854415143468556\\
17.3069194362801	0.854415015708389\\
17.3087938589879	0.854414887985873\\
17.3106682816072	0.854414760300996\\
17.3125427041381	0.854414632653741\\
17.3144171265806	0.854414505044094\\
17.3162915489346	0.854414377472039\\
17.3181659712004	0.854414249937564\\
17.3200403933778	0.854414122440651\\
17.3219148154669	0.854413994981288\\
17.3237892374678	0.854413867559458\\
17.3256636593803	0.854413740175147\\
17.3275380812047	0.854413612828341\\
17.3294125029409	0.854413485519025\\
17.3312869245889	0.854413358247184\\
17.3331613461487	0.854413231012802\\
17.3350357676204	0.854413103815867\\
17.3369101890041	0.854412976656362\\
17.3387846102996	0.854412849534273\\
17.3406590315072	0.854412722449585\\
17.3425334526267	0.854412595402285\\
17.3444078736582	0.854412468392356\\
17.3462822946017	0.854412341419784\\
17.3481567154573	0.854412214484555\\
17.350031136225	0.854412087586654\\
17.3519055569048	0.854411960726067\\
17.3537799774968	0.854411833902778\\
17.3556543980009	0.854411707116773\\
17.3575288184172	0.854411580368037\\
17.3594032387457	0.854411453656556\\
17.3612776589864	0.854411326982315\\
17.3631520791394	0.8544112003453\\
17.3650264992047	0.854411073745496\\
17.3669009191823	0.854410947182888\\
17.3687753390723	0.854410820657461\\
17.3706497588746	0.854410694169202\\
17.3725241785893	0.854410567718095\\
17.3743985982165	0.854410441304127\\
17.3762730177561	0.854410314927281\\
17.3781474372081	0.854410188587545\\
17.3800218565727	0.854410062284902\\
17.3818962758497	0.85440993601934\\
17.3837706950394	0.854409809790843\\
17.3856451141416	0.854409683599396\\
17.3875195331564	0.854409557444985\\
17.3893939520838	0.854409431327597\\
17.3912683709238	0.854409305247215\\
17.3931427896766	0.854409179203826\\
17.395017208342	0.854409053197415\\
17.3968916269202	0.854408927227968\\
17.3987660454111	0.854408801295469\\
17.4006404638148	0.854408675399906\\
17.4025148821313	0.854408549541263\\
17.4043893003606	0.854408423719525\\
17.4062637185028	0.854408297934679\\
17.4081381365578	0.85440817218671\\
17.4100125545258	0.854408046475603\\
17.4118869724067	0.854407920801344\\
17.4137613902005	0.854407795163919\\
17.4156358079073	0.854407669563312\\
17.4175102255271	0.854407543999511\\
17.41938464306	0.8544074184725\\
17.4212590605059	0.854407292982265\\
17.4231334778649	0.854407167528792\\
17.425007895137	0.854407042112065\\
17.4268823123222	0.854406916732072\\
17.4287567294206	0.854406791388797\\
17.4306311464322	0.854406666082226\\
17.432505563357	0.854406540812344\\
17.434379980195	0.854406415579139\\
17.4362543969462	0.854406290382594\\
17.4381288136108	0.854406165222696\\
17.4400032301887	0.85440604009943\\
17.4418776466799	0.854405915012782\\
17.4437520630844	0.854405789962739\\
17.4456264794024	0.854405664949284\\
17.4475008956337	0.854405539972405\\
17.4493753117785	0.854405415032087\\
17.4512497278367	0.854405290128315\\
17.4531241438085	0.854405165261075\\
17.4549985596937	0.854405040430354\\
17.4568729754925	0.854404915636136\\
17.4587473912048	0.854404790878408\\
17.4606218068308	0.854404666157156\\
17.4624962223703	0.854404541472364\\
17.4643706378235	0.854404416824019\\
17.4662450531903	0.854404292212107\\
17.4681194684708	0.854404167636613\\
17.4699938836651	0.854404043097523\\
17.471868298773	0.854403918594823\\
17.4737427137948	0.854403794128499\\
17.4756171287303	0.854403669698536\\
17.4774915435796	0.854403545304921\\
17.4793659583428	0.854403420947639\\
17.4812403730198	0.854403296626676\\
17.4831147876107	0.854403172342018\\
17.4849892021156	0.85440304809365\\
17.4868636165343	0.854402923881559\\
17.488738030867	0.854402799705731\\
17.4906124451138	0.854402675566151\\
17.4924868592745	0.854402551462804\\
17.4943612733493	0.854402427395678\\
17.4962356873381	0.854402303364758\\
17.498110101241	0.85440217937003\\
17.499984515058	0.854402055411479\\
17.5018589287892	0.854401931489092\\
17.5037333424345	0.854401807602854\\
17.505607755994	0.854401683752752\\
17.5074821694677	0.854401559938771\\
17.5093565828557	0.854401436160897\\
17.5112309961579	0.854401312419117\\
17.5131054093744	0.854401188713415\\
17.5149798225052	0.854401065043779\\
17.5168542355504	0.854400941410194\\
17.5187286485099	0.854400817812646\\
17.5206030613838	0.854400694251121\\
17.5224774741721	0.854400570725605\\
17.5243518868749	0.854400447236084\\
17.5262262994921	0.854400323782544\\
17.5281007120238	0.854400200364971\\
17.52997512447	0.85440007698335\\
17.5318495368307	0.854399953637669\\
17.533723949106	0.854399830327913\\
17.5355983612959	0.854399707054068\\
17.5374727734003	0.85439958381612\\
17.5393471854195	0.854399460614055\\
17.5412215973532	0.854399337447859\\
17.5430960092017	0.854399214317518\\
17.5449704209649	0.854399091223019\\
17.5468448326428	0.854398968164347\\
17.5487192442354	0.854398845141489\\
17.5505936557429	0.85439872215443\\
17.5524680671651	0.854398599203156\\
17.5543424785022	0.854398476287655\\
17.5562168897541	0.854398353407911\\
17.5580913009209	0.854398230563911\\
17.5599657120027	0.854398107755641\\
17.5618401229993	0.854397984983087\\
17.5637145339109	0.854397862246235\\
17.5655889447375	0.854397739545072\\
17.5674633554791	0.854397616879583\\
17.5693377661357	0.854397494249755\\
17.5712121767074	0.854397371655574\\
17.5730865871941	0.854397249097026\\
17.5749609975959	0.854397126574097\\
17.5768354079129	0.854397004086773\\
17.578709818145	0.854396881635041\\
17.5805842282923	0.854396759218887\\
17.5824586383548	0.854396636838297\\
17.5843330483325	0.854396514493257\\
17.5862074582255	0.854396392183753\\
17.5880818680337	0.854396269909771\\
17.5899562777573	0.854396147671299\\
17.5918306873962	0.854396025468321\\
17.5937050969504	0.854395903300825\\
17.5955795064199	0.854395781168796\\
17.5974539158049	0.854395659072221\\
17.5993283251053	0.854395537011086\\
17.6012027343212	0.854395414985377\\
17.6030771434525	0.854395292995081\\
17.6049515524993	0.854395171040184\\
17.6068259614616	0.854395049120671\\
17.6087003703395	0.85439492723653\\
17.6105747791329	0.854394805387747\\
17.6124491878419	0.854394683574308\\
17.6143235964666	0.854394561796199\\
17.6161980050068	0.854394440053406\\
17.6180724134628	0.854394318345917\\
17.6199468218344	0.854394196673717\\
17.6218212301217	0.854394075036792\\
17.6236956383248	0.854393953435129\\
17.6255700464437	0.854393831868715\\
17.6274444544783	0.854393710337535\\
17.6293188624287	0.854393588841577\\
17.631193270295	0.854393467380825\\
17.6330676780771	0.854393345955268\\
17.6349420857751	0.85439322456489\\
17.6368164933891	0.854393103209679\\
17.6386909009189	0.854392981889621\\
17.6405653083647	0.854392860604703\\
17.6424397157265	0.85439273935491\\
17.6443141230043	0.854392618140229\\
17.6461885301981	0.854392496960647\\
17.648062937308	0.85439237581615\\
17.6499373443339	0.854392254706724\\
17.651811751276	0.854392133632356\\
17.6536861581341	0.854392012593033\\
17.6555605649084	0.85439189158874\\
17.6574349715989	0.854391770619464\\
17.6593093782056	0.854391649685193\\
17.6611837847285	0.854391528785911\\
17.6630581911677	0.854391407921606\\
17.6649325975231	0.854391287092264\\
17.6668070037948	0.854391166297872\\
17.6686814099829	0.854391045538416\\
17.6705558160873	0.854390924813883\\
17.672430222108	0.854390804124259\\
17.6743046280451	0.854390683469531\\
17.6761790338987	0.854390562849684\\
17.6780534396687	0.854390442264707\\
17.6799278453551	0.854390321714585\\
17.6818022509581	0.854390201199305\\
17.6836766564775	0.854390080718853\\
17.6855510619135	0.854389960273216\\
17.687425467266	0.85438983986238\\
17.6892998725351	0.854389719486333\\
17.6911742777209	0.85438959914506\\
17.6930486828232	0.854389478838549\\
17.6949230878422	0.854389358566785\\
17.6967974927779	0.854389238329756\\
17.6986718976303	0.854389118127448\\
17.7005463023994	0.854388997959848\\
17.7024207070853	0.854388877826942\\
17.7042951116879	0.854388757728717\\
17.7061695162074	0.854388637665159\\
17.7080439206436	0.854388517636255\\
17.7099183249967	0.854388397641993\\
17.7117927292667	0.854388277682358\\
17.7136671334536	0.854388157757337\\
17.7155415375573	0.854388037866916\\
17.7174159415781	0.854387918011083\\
17.7192903455158	0.854387798189825\\
17.7211647493704	0.854387678403127\\
17.7230391531421	0.854387558650976\\
17.7249135568309	0.85438743893336\\
17.7267879604366	0.854387319250265\\
17.7286623639595	0.854387199601677\\
17.7305367673995	0.854387079987584\\
17.7324111707566	0.854386960407972\\
17.7342855740309	0.854386840862827\\
17.7361599772223	0.854386721352137\\
17.738034380331	0.854386601875888\\
17.7399087833568	0.854386482434068\\
17.7417831863	0.854386363026662\\
17.7436575891604	0.854386243653657\\
17.745531991938	0.854386124315041\\
17.7474063946331	0.854386005010799\\
17.7492807972454	0.85438588574092\\
17.7511551997752	0.854385766505389\\
17.7530296022223	0.854385647304194\\
17.7549040045868	0.85438552813732\\
17.7567784068688	0.854385409004756\\
17.7586528090682	0.854385289906488\\
17.7605272111852	0.854385170842502\\
17.7624016132196	0.854385051812786\\
17.7642760151716	0.854384932817326\\
17.7661504170411	0.854384813856109\\
17.7680248188282	0.854384694929122\\
17.7698992205329	0.854384576036351\\
17.7717736221553	0.854384457177785\\
17.7736480236953	0.854384338353409\\
17.775522425153	0.85438421956321\\
17.7773968265283	0.854384100807175\\
17.7792712278214	0.854383982085292\\
17.7811456290323	0.854383863397547\\
17.7830200301609	0.854383744743926\\
17.7848944312073	0.854383626124418\\
17.7867688321715	0.854383507539008\\
17.7886432330536	0.854383388987684\\
17.7905176338535	0.854383270470432\\
17.7923920345713	0.85438315198724\\
17.794266435207	0.854383033538094\\
17.7961408357607	0.854382915122982\\
17.7980152362323	0.854382796741889\\
17.7998896366219	0.854382678394804\\
17.8017640369295	0.854382560081713\\
17.8036384371552	0.854382441802604\\
17.8055128372989	0.854382323557462\\
17.8073872373606	0.854382205346276\\
17.8092616373405	0.854382087169031\\
17.8111360372385	0.854381969025716\\
17.8130104370546	0.854381850916316\\
17.8148848367889	0.85438173284082\\
17.8167592364414	0.854381614799214\\
17.8186336360121	0.854381496791485\\
17.820508035501	0.854381378817619\\
17.8223824349082	0.854381260877605\\
17.8242568342337	0.854381142971429\\
17.8261312334775	0.854381025099079\\
17.8280056326396	0.85438090726054\\
17.8298800317201	0.854380789455801\\
17.831754430719	0.854380671684848\\
17.8336288296363	0.854380553947668\\
17.835503228472	0.854380436244249\\
17.8373776272261	0.854380318574577\\
17.8392520258987	0.85438020093864\\
17.8411264244898	0.854380083336425\\
17.8430008229995	0.854379965767918\\
17.8448752214276	0.854379848233107\\
17.8467496197744	0.854379730731979\\
17.8486240180397	0.854379613264522\\
17.8504984162236	0.854379495830721\\
17.8523728143262	0.854379378430565\\
17.8542472123474	0.85437926106404\\
17.8561216102873	0.854379143731134\\
17.8579960081459	0.854379026431833\\
17.8598704059233	0.854378909166126\\
17.8617448036193	0.854378791933998\\
17.8636192012342	0.854378674735438\\
17.8654935987679	0.854378557570432\\
17.8673679962203	0.854378440438967\\
17.8692423935917	0.854378323341031\\
17.8711167908819	0.854378206276612\\
17.8729911880909	0.854378089245695\\
17.8748655852189	0.854377972248268\\
17.8767399822659	0.854377855284319\\
17.8786143792317	0.854377738353835\\
17.8804887761166	0.854377621456802\\
17.8823631729205	0.854377504593209\\
17.8842375696434	0.854377387763042\\
17.8861119662853	0.854377270966288\\
17.8879863628463	0.854377154202936\\
17.8898607593265	0.854377037472971\\
17.8917351557257	0.854376920776382\\
17.8936095520441	0.854376804113155\\
17.8954839482816	0.854376687483279\\
17.8973583444384	0.854376570886739\\
17.8992327405143	0.854376454323524\\
17.9011071365095	0.85437633779362\\
17.9029815324239	0.854376221297015\\
17.9048559282577	0.854376104833697\\
17.9067303240107	0.854375988403652\\
17.908604719683	0.854375872006868\\
17.9104791152748	0.854375755643332\\
17.9123535107858	0.854375639313032\\
17.9142279062163	0.854375523015955\\
17.9161023015662	0.854375406752087\\
17.9179766968355	0.854375290521417\\
17.9198510920243	0.854375174323932\\
17.9217254871326	0.854375058159619\\
17.9235998821604	0.854374942028466\\
17.9254742771077	0.85437482593046\\
17.9273486719746	0.854374709865588\\
17.929223066761	0.854374593833837\\
17.9310974614671	0.854374477835195\\
17.9329718560928	0.85437436186965\\
17.9348462506381	0.854374245937189\\
17.9367206451031	0.854374130037799\\
17.9385950394878	0.854374014171467\\
17.9404694337922	0.854373898338181\\
17.9423438280164	0.854373782537929\\
17.9442182221603	0.854373666770698\\
17.946092616224	0.854373551036475\\
17.9479670102075	0.854373435335247\\
17.9498414041108	0.854373319667003\\
17.951715797934	0.854373204031729\\
17.953590191677	0.854373088429413\\
17.95546458534	0.854372972860043\\
17.9573389789229	0.854372857323606\\
17.9592133724257	0.854372741820089\\
17.9610877658485	0.85437262634948\\
17.9629621591912	0.854372510911766\\
17.964836552454	0.854372395506936\\
17.9667109456368	0.854372280134975\\
17.9685853387397	0.854372164795873\\
17.9704597317626	0.854372049489615\\
17.9723341247057	0.854371934216191\\
17.9742085175689	0.854371818975587\\
17.9760829103522	0.854371703767791\\
17.9779573030557	0.85437158859279\\
17.9798316956794	0.854371473450573\\
17.9817060882232	0.854371358341126\\
17.9835804806874	0.854371243264436\\
17.9854548730718	0.854371128220493\\
17.9873292653765	0.854371013209283\\
17.9892036576014	0.854370898230793\\
17.9910780497468	0.854370783285012\\
17.9929524418124	0.854370668371926\\
17.9948268337984	0.854370553491524\\
17.9967012257049	0.854370438643793\\
17.9985756175317	0.854370323828721\\
18.000450009279	0.854370209046295\\
18.0023244009467	0.854370094296502\\
18.004198792535	0.854369979579331\\
18.0060731840437	0.85436986489477\\
18.007947575473	0.854369750242805\\
18.0098219668228	0.854369635623424\\
18.0116963580932	0.854369521036615\\
18.0135707492842	0.854369406482366\\
18.0154451403958	0.854369291960664\\
18.0173195314281	0.854369177471497\\
18.019193922381	0.854369063014852\\
18.0210683132546	0.854368948590718\\
18.022942704049	0.854368834199081\\
18.024817094764	0.854368719839931\\
18.0266914853998	0.854368605513253\\
18.0285658759564	0.854368491219036\\
18.0304402664338	0.854368376957268\\
18.032314656832	0.854368262727936\\
18.0341890471511	0.854368148531029\\
18.036063437391	0.854368034366533\\
18.0379378275518	0.854367920234436\\
18.0398122176335	0.854367806134727\\
18.0416866076362	0.854367692067392\\
18.0435609975598	0.85436757803242\\
18.0454353874044	0.854367464029799\\
18.04730977717	0.854367350059516\\
18.0491841668566	0.854367236121558\\
18.0510585564642	0.854367122215914\\
18.052932945993	0.854367008342572\\
18.0548073354428	0.854366894501519\\
18.0566817248137	0.854366780692743\\
18.0585561141058	0.854366666916231\\
18.060430503319	0.854366553171972\\
18.0623048924534	0.854366439459953\\
18.064179281509	0.854366325780162\\
18.0660536704858	0.854366212132588\\
18.0679280593839	0.854366098517216\\
18.0698024482032	0.854365984934037\\
18.0716768369438	0.854365871383037\\
18.0735512256058	0.854365757864204\\
18.075425614189	0.854365644377526\\
18.0773000026937	0.854365530922991\\
18.0791743911197	0.854365417500586\\
18.0810487794671	0.8543653041103\\
18.0829231677359	0.85436519075212\\
18.0847975559262	0.854365077426035\\
18.0866719440379	0.854364964132032\\
18.0885463320711	0.854364850870099\\
18.0904207200259	0.854364737640223\\
18.0922951079022	0.854364624442394\\
18.0941694957	0.854364511276598\\
18.0960438834194	0.854364398142824\\
18.0979182710604	0.854364285041059\\
18.099792658623	0.854364171971291\\
18.1016670461073	0.854364058933509\\
18.1035414335133	0.8543639459277\\
18.1054158208409	0.854363832953852\\
18.1072902080902	0.854363720011953\\
18.1091645952613	0.854363607101991\\
18.1110389823541	0.854363494223953\\
18.1129133693687	0.854363381377829\\
18.1147877563051	0.854363268563606\\
18.1166621431633	0.854363155781271\\
18.1185365299434	0.854363043030813\\
18.1204109166453	0.85436293031222\\
18.1222853032691	0.85436281762548\\
18.1241596898149	0.85436270497058\\
18.1260340762825	0.854362592347508\\
18.1279084626721	0.854362479756254\\
18.1297828489837	0.854362367196804\\
18.1316572352173	0.854362254669147\\
18.1335316213729	0.85436214217327\\
18.1354060074505	0.854362029709162\\
18.1372803934502	0.854361917276811\\
18.139154779372	0.854361804876205\\
18.1410291652159	0.854361692507331\\
18.1429035509819	0.854361580170178\\
18.14477793667	0.854361467864734\\
18.1466523222803	0.854361355590986\\
18.1485267078129	0.854361243348924\\
18.1504010932676	0.854361131138535\\
18.1522754786446	0.854361018959806\\
18.1541498639438	0.854360906812727\\
18.1560242491653	0.854360794697285\\
18.1578986343091	0.854360682613469\\
18.1597730193753	0.854360570561265\\
18.1616474043638	0.854360458540664\\
18.1635217892746	0.854360346551651\\
18.1653961741079	0.854360234594217\\
18.1672705588635	0.854360122668348\\
18.1691449435416	0.854360010774034\\
18.1710193281422	0.854359898911261\\
18.1728937126652	0.854359787080018\\
18.1747680971107	0.854359675280294\\
18.1766424814787	0.854359563512076\\
18.1785168657693	0.854359451775353\\
18.1803912499824	0.854359340070112\\
18.1822656341182	0.854359228396343\\
18.1841400181765	0.854359116754032\\
18.1860144021575	0.854359005143169\\
18.1878887860611	0.854358893563741\\
18.1897631698874	0.854358782015736\\
18.1916375536363	0.854358670499143\\
18.193511937308	0.854358559013951\\
18.1953863209025	0.854358447560146\\
18.1972607044197	0.854358336137718\\
18.1991350878596	0.854358224746654\\
18.2010094712224	0.854358113386943\\
18.202883854508	0.854358002058573\\
18.2047582377164	0.854357890761532\\
18.2066326208477	0.854357779495808\\
18.2085070039019	0.854357668261391\\
18.210381386879	0.854357557058267\\
18.2122557697791	0.854357445886425\\
18.2141301526021	0.854357334745854\\
18.216004535348	0.854357223636541\\
18.217878918017	0.854357112558475\\
18.219753300609	0.854357001511644\\
18.221627683124	0.854356890496037\\
18.2235020655621	0.854356779511641\\
18.2253764479232	0.854356668558446\\
18.2272508302075	0.854356557636439\\
18.2291252124149	0.854356446745608\\
18.2309995945454	0.854356335885942\\
18.2328739765991	0.85435622505743\\
18.234748358576	0.854356114260059\\
18.2366227404761	0.854356003493817\\
18.2384971222994	0.854355892758694\\
18.240371504046	0.854355782054678\\
18.2422458857159	0.854355671381756\\
18.2441202673091	0.854355560739917\\
18.2459946488256	0.85435545012915\\
18.2478690302654	0.854355339549442\\
18.2497434116286	0.854355229000783\\
18.2516177929151	0.85435511848316\\
18.2534921741251	0.854355007996563\\
18.2553665552585	0.854354897540978\\
18.2572409363154	0.854354787116395\\
18.2591153172957	0.854354676722803\\
18.2609896981995	0.854354566360188\\
18.2628640790268	0.854354456028541\\
18.2647384597777	0.854354345727848\\
18.2666128404521	0.8543542354581\\
18.26848722105	0.854354125219283\\
18.2703616015716	0.854354015011387\\
18.2722359820168	0.854353904834399\\
18.2741103623856	0.854353794688309\\
18.2759847426781	0.854353684573105\\
18.2778591228943	0.854353574488775\\
18.2797335030342	0.854353464435307\\
18.2816078830978	0.854353354412691\\
18.2834822630851	0.854353244420914\\
18.2853566429963	0.854353134459964\\
18.2872310228312	0.854353024529832\\
18.2891054025899	0.854352914630504\\
18.2909797822724	0.85435280476197\\
18.2928541618788	0.854352694924217\\
18.2947285414091	0.854352585117235\\
18.2966029208633	0.854352475341011\\
18.2984773002414	0.854352365595535\\
18.3003516795434	0.854352255880795\\
18.3022260587693	0.854352146196779\\
18.3041004379193	0.854352036543476\\
18.3059748169933	0.854351926920874\\
18.3078491959912	0.854351817328962\\
18.3097235749133	0.854351707767728\\
18.3115979537594	0.854351598237162\\
18.3134723325295	0.854351488737251\\
18.3153467112238	0.854351379267984\\
18.3172210898422	0.854351269829349\\
18.3190954683848	0.854351160421336\\
18.3209698468515	0.854351051043932\\
18.3228442252424	0.854350941697126\\
18.3247186035575	0.854350832380908\\
18.3265929817969	0.854350723095264\\
18.3284673599605	0.854350613840185\\
18.3303417380484	0.854350504615658\\
18.3322161160606	0.854350395421673\\
18.3340904939971	0.854350286258217\\
18.3359648718579	0.854350177125279\\
18.3378392496431	0.854350068022849\\
18.3397136273527	0.854349958950914\\
18.3415880049867	0.854349849909463\\
18.3434623825451	0.854349740898485\\
18.345336760028	0.854349631917969\\
18.3472111374353	0.854349522967902\\
18.3490855147671	0.854349414048275\\
18.3509598920234	0.854349305159074\\
18.3528342692042	0.85434919630029\\
18.3547086463096	0.85434908747191\\
18.3565830233395	0.854348978673924\\
18.3584574002941	0.85434886990632\\
18.3603317771733	0.854348761169086\\
18.362206153977	0.854348652462212\\
18.3640805307055	0.854348543785686\\
18.3659549073586	0.854348435139496\\
18.3678292839364	0.854348326523632\\
18.3697036604389	0.854348217938082\\
18.3715780368662	0.854348109382835\\
18.3734524132182	0.854348000857879\\
18.375326789495	0.854347892363204\\
18.3772011656966	0.854347783898798\\
18.3790755418231	0.854347675464649\\
18.3809499178743	0.854347567060747\\
18.3828242938505	0.854347458687079\\
18.3846986697515	0.854347350343636\\
18.3865730455774	0.854347242030405\\
18.3884474213282	0.854347133747376\\
18.390321797004	0.854347025494537\\
18.3921961726048	0.854346917271877\\
18.3940705481305	0.854346809079384\\
18.3959449235813	0.854346700917048\\
18.397819298957	0.854346592784857\\
18.3996936742579	0.854346484682801\\
18.4015680494838	0.854346376610867\\
18.4034424246348	0.854346268569044\\
18.4053167997109	0.854346160557322\\
18.4071911747121	0.854346052575689\\
18.4090655496385	0.854345944624134\\
18.4109399244901	0.854345836702646\\
18.4128142992668	0.854345728811213\\
18.4146886739688	0.854345620949825\\
18.416563048596	0.85434551311847\\
18.4184374231485	0.854345405317138\\
18.4203117976262	0.854345297545816\\
18.4221861720293	0.854345189804494\\
18.4240605463577	0.854345082093161\\
18.4259349206114	0.854344974411805\\
18.4278092947904	0.854344866760415\\
18.4296836688949	0.854344759138981\\
18.4315580429247	0.854344651547491\\
18.43343241688	0.854344543985934\\
18.4353067907607	0.854344436454299\\
18.4371811645669	0.854344328952574\\
18.4390555382986	0.854344221480749\\
18.4409299119558	0.854344114038813\\
18.4428042855385	0.854344006626754\\
18.4446786590467	0.854343899244561\\
18.4465530324806	0.854343791892224\\
18.44842740584	0.854343684569731\\
18.450301779125	0.854343577277071\\
18.4521761523356	0.854343470014233\\
18.4540505254719	0.854343362781206\\
18.4559248985339	0.854343255577978\\
18.4577992715215	0.85434314840454\\
18.4596736444349	0.854343041260879\\
18.461548017274	0.854342934146985\\
18.4634223900389	0.854342827062847\\
18.4652967627295	0.854342720008454\\
18.4671711353459	0.854342612983794\\
18.4690455078881	0.854342505988856\\
18.4709198803562	0.854342399023631\\
18.4727942527501	0.854342292088105\\
18.4746686250699	0.85434218518227\\
18.4765429973156	0.854342078306112\\
18.4784173694872	0.854341971459623\\
18.4802917415848	0.85434186464279\\
18.4821661136083	0.854341757855602\\
18.4840404855578	0.854341651098049\\
18.4859148574333	0.85434154437012\\
18.4877892292348	0.854341437671803\\
18.4896636009623	0.854341331003088\\
18.4915379726159	0.854341224363964\\
18.4934123441956	0.854341117754419\\
18.4952867157014	0.854341011174443\\
18.4971610871333	0.854340904624025\\
18.4990354584913	0.854340798103154\\
18.5009098297755	0.854340691611818\\
18.5027842009859	0.854340585150008\\
18.5046585721225	0.854340478717712\\
18.5065329431853	0.854340372314918\\
18.5084073141744	0.854340265941617\\
18.5102816850897	0.854340159597798\\
18.5121560559313	0.854340053283449\\
18.5140304266992	0.854339946998559\\
18.5159047973934	0.854339840743118\\
18.517779168014	0.854339734517114\\
18.519653538561	0.854339628320538\\
18.5215279090343	0.854339522153377\\
18.523402279434	0.854339416015621\\
18.5252766497602	0.85433930990726\\
18.5271510200128	0.854339203828282\\
18.5290253901919	0.854339097778676\\
18.5308997602974	0.854338991758432\\
18.5327741303295	0.854338885767538\\
18.5346485002881	0.854338779805985\\
18.5365228701732	0.85433867387376\\
18.538397239985	0.854338567970854\\
18.5402716097233	0.854338462097255\\
18.5421459793882	0.854338356252953\\
18.5440203489797	0.854338250437936\\
18.5458947184979	0.854338144652194\\
18.5477690879427	0.854338038895716\\
18.5496434573143	0.854337933168492\\
18.5515178266125	0.85433782747051\\
18.5533921958375	0.854337721801759\\
18.5552665649893	0.85433761616223\\
18.5571409340678	0.854337510551911\\
18.5590153030731	0.85433740497079\\
18.5608896720052	0.854337299418859\\
18.5627640408641	0.854337193896105\\
18.5646384096499	0.854337088402518\\
18.5665127783625	0.854336982938087\\
18.5683871470021	0.854336877502802\\
18.5702615155686	0.854336772096652\\
18.5721358840619	0.854336666719625\\
18.5740102524823	0.854336561371712\\
18.5758846208296	0.854336456052901\\
18.5777589891039	0.854336350763182\\
18.5796333573052	0.854336245502544\\
18.5815077254336	0.854336140270976\\
18.583382093489	0.854336035068468\\
18.5852564614714	0.854335929895008\\
18.587130829381	0.854335824750587\\
18.5890051972177	0.854335719635193\\
18.5908795649815	0.854335614548816\\
18.5927539326724	0.854335509491445\\
18.5946283002906	0.854335404463069\\
18.5965026678359	0.854335299463678\\
18.5983770353084	0.854335194493261\\
18.6002514027082	0.854335089551807\\
18.6021257700352	0.854334984639306\\
18.6040001372895	0.854334879755747\\
18.605874504471	0.854334774901119\\
18.6077488715799	0.854334670075412\\
18.6096232386162	0.854334565278614\\
18.6114976055797	0.854334460510716\\
18.6133719724707	0.854334355771707\\
18.615246339289	0.854334251061576\\
18.6171207060348	0.854334146380313\\
18.6189950727079	0.854334041727906\\
18.6208694393086	0.854333937104346\\
18.6227438058367	0.854333832509621\\
18.6246181722923	0.854333727943721\\
18.6264925386754	0.854333623406635\\
18.628366904986	0.854333518898353\\
18.6302412712242	0.854333414418865\\
18.6321156373899	0.854333309968159\\
18.6339900034833	0.854333205546225\\
18.6358643695042	0.854333101153052\\
18.6377387354528	0.85433299678863\\
18.6396131013291	0.854332892452949\\
18.641487467133	0.854332788145997\\
18.6433618328645	0.854332683867765\\
18.6452361985239	0.854332579618241\\
18.6471105641109	0.854332475397414\\
18.6489849296257	0.854332371205276\\
18.6508592950682	0.854332267041814\\
18.6527336604386	0.854332162907019\\
18.6546080257367	0.85433205880088\\
18.6564823909627	0.854331954723386\\
18.6583567561165	0.854331850674526\\
18.6602311211982	0.854331746654292\\
18.6621054862078	0.85433164266267\\
18.6639798511453	0.854331538699653\\
18.6658542160107	0.854331434765228\\
18.667728580804	0.854331330859385\\
18.6696029455254	0.854331226982114\\
18.6714773101747	0.854331123133404\\
18.673351674752	0.854331019313245\\
18.6752260392574	0.854330915521627\\
18.6771004036907	0.854330811758538\\
18.6789747680522	0.854330708023968\\
18.6808491323417	0.854330604317908\\
18.6827234965594	0.854330500640346\\
18.6845978607052	0.854330396991272\\
18.6864722247791	0.854330293370676\\
18.6883465887811	0.854330189778546\\
18.6902209527114	0.854330086214874\\
18.6920953165699	0.854329982679647\\
18.6939696803565	0.854329879172857\\
18.6958440440715	0.854329775694491\\
18.6977184077147	0.854329672244541\\
18.6995927712861	0.854329568822995\\
18.7014671347859	0.854329465429843\\
18.703341498214	0.854329362065075\\
18.7052158615704	0.85432925872868\\
18.7070902248552	0.854329155420648\\
};
\addplot [color=red,solid,forget plot]
  table[row sep=crcr]{%
0	-1\\
0.000953612251194968	-0.999991569056121\\
0.00190798729678	-0.999963601161843\\
0.00286396019577596	-0.999916173293784\\
0.0038223562787305	-0.999849420394275\\
0.00478398574829361	-0.999763533861643\\
0.0057496385504451	-0.99965875939445\\
0.00672007960576098	-0.999535394245381\\
0.00769604447755355	-0.999393783954496\\
0.00867823553891636	-0.999234318642431\\
0.00966731868445496	-0.999057428951598\\
0.0106639206156271	-0.9988635817273\\
0.0116686267119624	-0.998653275531109\\
0.0126819794847081	-0.998427036076018\\
0.0137044775952622	-0.998185411667346\\
0.014736575408559	-0.997928968725522\\
0.0157786830416588	-0.997658287457446\\
0.0168311668603032	-0.997373957732574\\
0.0178943503711121	-0.997076575208879\\
0.0189685154542912	-0.996766737742879\\
0.0200539038809719	-0.996445042107388\\
0.0211507190603203	-0.996112081030925\\
0.0222591279640143	-0.995768440564039\\
0.0233792631792637	-0.995414697770303\\
0.0245112250459113	-0.995051418733452\\
0.0256550838380166	-0.994679156867197\\
0.0268108819554154	-0.994298451510375\\
0.0279786360958673	-0.99390982678745\\
0.0291583393833613	-0.993513790712613\\
0.0303499634328282	-0.993110834514872\\
0.0315534603358096	-0.992701432161321\\
0.0327687645555044	-0.992286040056183\\
0.0339957947230267	-0.99186509689403\\
0.0352344553296577	-0.991439023646738\\
0.0364846383123741	-0.991008223665093\\
0.0377462245320113	-0.990573082877498\\
0.0390190851451084	-0.990133970069768\\
0.0403030828718157	-0.989691237231613\\
0.0415980731632802	-0.989245219956958\\
0.0429039052726805	-0.98879623788673\\
0.044220423234625	-0.988344595184161\\
0.0455474667579683	-0.987890581033938\\
0.0468848720372929	-0.987434470157766\\
0.0482324724883683	-0.986976523339946\\
0.0495900994128629	-0.98651698795759\\
0.0509575825974728	-0.986056098510929\\
0.0523347508524641	-0.985594077149959\\
0.0537214324944103	-0.985131134194338\\
0.0551174557776738	-0.98466746864402\\
0.0565226492789171	-0.984203268678633\\
0.0579368422386703	-0.983738712144056\\
0.0593598648637108	-0.98327396702498\\
0.060791548593747	-0.982809191902609\\
0.0622317263356417	-0.982344536396883\\
0.0636802326681619	-0.981880141592852\\
0.0651369040200065	-0.981416140451011\\
0.066601578823637	-0.980952658201549\\
0.0680740976472316	-0.980489812722622\\
0.0695543033068806	-0.980027714902813\\
0.0710420409609634	-0.979566468988088\\
0.0725371581884741	-0.979106172913558\\
0.0740395050529106	-0.978646918620451\\
0.0755489341531933	-0.978188792358722\\
0.0770653006629513	-0.97773187497574\\
0.0785884623593899	-0.977276242191551\\
0.0801182796428433	-0.976821964861173\\
0.0816546155480142	-0.976369109224434\\
0.0831973357478098	-0.975917737143837\\
0.0847463085505986	-0.975467906330936\\
0.0863014048916339	-0.97501967056171\\
0.0878624983193236	-0.974573079881413\\
0.0894294649769562	-0.974128180799347\\
0.0910021835804397	-0.973685016474027\\
0.092580535392554	-0.97324362688916\\
0.0941644041941709	-0.972804049020873\\
0.0957536762528503	-0.97236631699659\\
0.0973482402891837	-0.971930462245957\\
0.0989479874412172	-0.971496513644193\\
0.100552811227255	-0.971064497648223\\
0.102162607507317	-0.970634438425961\\
0.103777274443486	-0.970206357979052\\
0.105396712459376	-0.969780276259413\\
0.107020824198905	-0.969356211279867\\
0.108649514484559	-0.968934179219167\\
0.110282690275297	-0.968514194521683\\
0.111920260624243	-0.96809626999203\\
0.113562136636292	-0.967680416884871\\
0.11520823142573	-0.967266644990151\\
0.116858460073987	-0.966854962713989\\
0.118512739587591	-0.966445377155438\\
0.120170988856418	-0.966037894179336\\
0.121833128612288	-0.965632518485425\\
0.123499081387987	-0.965229253673952\\
0.12516877147675	-0.964828102307902\\
0.126842124892257	-0.964429065972059\\
0.128519069329185	-0.964032145329033\\
0.130199534124343	-0.963637340172433\\
0.131883450218418	-0.963244649477297\\
0.133570750118359	-0.962854071447955\\
0.135261367860419	-0.962465603563423\\
0.136955238973864	-0.962079242620478\\
0.138652300445369	-0.961694984774514\\
0.140352490684108	-0.961312825578304\\
0.142055749487534	-0.960932760018763\\
0.143762018007875	-0.960554782551828\\
0.145471238719322	-0.960178887135538\\
0.147183355385936	-0.959805067261404\\
0.148898313030239	-0.959433315984174\\
0.150616057902523	-0.959063625950049\\
0.152336537450844	-0.95869598942345\\
0.154059700291704	-0.958330398312393\\
0.155785496181417	-0.95796684419256\\
0.157513875988152	-0.957605318330112\\
0.159244791664635	-0.957245811703321\\
0.160978196221513	-0.956888315023083\\
0.162714043701357	-0.956532818752353\\
0.16445228915331	-0.956179313124574\\
0.166192888608347	-0.955827788161139\\
0.167935799055163	-0.955478233687942\\
0.169680978416652	-0.955130639351056\\
0.17142838552698	-0.954784994631594\\
0.173177980109241	-0.954441288859778\\
0.174929722753671	-0.954099511228272\\
0.176683574896424	-0.953759650804807\\
0.178439498798886	-0.953421696544129\\
0.180197457527526	-0.953085637299323\\
0.181957414934258	-0.952751461832518\\
0.183719335637314	-0.952419158825028\\
0.185483185002611	-0.952088716886938\\
0.187248929125601	-0.951760124566185\\
0.189016534813593	-0.951433370357127\\
0.190785969568535	-0.951108442708666\\
0.192557201570241	-0.950785330031909\\
0.194330199660063	-0.950464020707415\\
0.196104933324982	-0.950144503092041\\
0.197881372682119	-0.949826765525408\\
0.199659488463647	-0.949510796335996\\
0.201439252002103	-0.949196583846911\\
0.203220635216078	-0.948884116381309\\
0.205003610596287	-0.948573382267521\\
0.206788151191999	-0.948264369843881\\
0.208574230597822	-0.94795706746327\\
0.210361822940838	-0.947651463497397\\
0.212150902868068	-0.947347546340828\\
0.213941445534267	-0.947045304414774\\
0.215733426590037	-0.94674472617065\\
0.217526822170251	-0.946445800093421\\
0.219321608882775	-0.946148514704736\\
0.221117763797485	-0.945852858565872\\
0.222915264435571	-0.945558820280494\\
0.224714088759116	-0.945266388497231\\
0.226514215160944	-0.94497555191209\\
0.228315622454732	-0.944686299270713\\
0.230118289865379	-0.944398619370478\\
0.231922197019621	-0.944112501062461\\
0.233727323936885	-0.943827933253256\\
0.235533651020384	-0.943544904906676\\
0.237341159048436	-0.943263405045319\\
0.239149829166002	-0.942983422752028\\
0.240959642876453	-0.942704947171234\\
0.242770582033528	-0.942427967510198\\
0.244582628833514	-0.942152473040151\\
0.24639576580761	-0.941878453097344\\
0.2482099758145	-0.941605897083998\\
0.250025242033095	-0.941334794469186\\
0.251841547955475	-0.941065134789615\\
0.253658877379999	-0.940796907650349\\
0.255477214404587	-0.94053010272545\\
0.257296543420181	-0.940264709758552\\
0.259116849104352	-0.940000718563368\\
0.260938116415083	-0.939738119024145\\
0.262760330584697	-0.939476901096048\\
0.264583477113936	-0.939217054805497\\
0.266407541766191	-0.938958570250447\\
0.268232510561869	-0.938701437600627\\
0.270058369772903	-0.938445647097719\\
0.271885105917392	-0.938191189055505\\
0.273712705754377	-0.937938053859964\\
0.275541156278742	-0.937686231969335\\
0.277370444716237	-0.93743571391414\\
0.279200558518625	-0.937186490297171\\
0.281031485358948	-0.936938551793447\\
0.282863213126898	-0.936691889150135\\
0.284695729924316	-0.936446493186446\\
0.286529024060777	-0.9362023547935\\
0.288363084049305	-0.935959464934167\\
0.290197898602166	-0.935717814642881\\
0.292033456626783	-0.935477395025433\\
0.293869747221731	-0.935238197258739\\
0.295706759672837	-0.935000212590594\\
0.29754448344937	-0.934763432339398\\
0.299382908200314	-0.93452784789387\\
0.301222023750738	-0.934293450712746\\
0.303061820098246	-0.934060232324454\\
0.30490228740951	-0.933828184326782\\
0.306743416016883	-0.93359729838653\\
0.308585196415091	-0.933367566239147\\
0.310427619258006	-0.933138979688355\\
0.312270675355485	-0.932911530605771\\
0.314114355670287	-0.932685210930506\\
0.315958651315059	-0.932460012668761\\
0.317803553549393	-0.932235927893419\\
0.31964905377695	-0.932012948743617\\
0.321495143542641	-0.931791067424323\\
0.323341814529887	-0.931570276205896\\
0.325189058557924	-0.931350567423644\\
0.327036867579184	-0.931131933477383\\
0.328885233676726	-0.930914366830976\\
0.330734149061721	-0.93069786001188\\
0.332583606071007	-0.930482405610684\\
0.334433597164682	-0.930267996280645\\
0.336284114923763	-0.930054624737216\\
0.338135152047889	-0.929842283757579\\
0.33998670135308	-0.929630966180168\\
0.341838755769539	-0.929420664904196\\
0.343691308339509	-0.929211372889173\\
0.345544352215173	-0.929003083154434\\
0.347397880656598	-0.928795788778651\\
0.34925188702973	-0.928589482899358\\
0.351106364804425	-0.928384158712466\\
0.352961307552526	-0.928179809471785\\
0.354816708945984	-0.927976428488538\\
0.356672562755013	-0.927774009130881\\
0.358528862846291	-0.927572544823426\\
0.360385603181194	-0.927372029046753\\
0.362242777814071	-0.927172455336937\\
0.364100380890556	-0.926973817285066\\
0.365958406645911	-0.926776108536766\\
0.367816849403407	-0.926579322791723\\
0.369675703572744	-0.926383453803209\\
0.371534963648494	-0.926188495377613\\
0.373394624208582	-0.925994441373965\\
0.375254679912804	-0.925801285703468\\
0.377115125501361	-0.925609022329037\\
0.37897595579344	-0.925417645264826\\
0.380837165685813	-0.925227148575772\\
0.382698750151471	-0.925037526377134\\
0.38456070423828	-0.924848772834031\\
0.386423023067672	-0.924660882160995\\
0.388285701833356	-0.924473848621513\\
0.390148735800059	-0.92428766652758\\
0.392012120302291	-0.92410233023925\\
0.393875850743138	-0.923917834164198\\
0.395739922593073	-0.923734172757273\\
0.397604331388797	-0.923551340520063\\
0.3994690727321	-0.923369332000461\\
0.401334142288749	-0.923188141792231\\
0.403199535787389	-0.923007764534581\\
0.405065249018478	-0.922828194911739\\
0.40693127783323	-0.922649427652528\\
0.408797618142595	-0.922471457529949\\
0.410664265916239	-0.922294279360765\\
0.412531217181565	-0.922117888005087\\
0.414398468022736	-0.921942278365967\\
0.416266014579727	-0.921767445388991\\
0.418133853047393	-0.921593384061878\\
0.420001979674554	-0.921420089414078\\
0.421870390763098	-0.921247556516379\\
0.423739082667103	-0.921075780480514\\
0.425608051791974	-0.920904756458771\\
0.427477294593596	-0.920734479643607\\
0.429346807577507	-0.920564945267268\\
0.431216587298085	-0.920396148601409\\
0.433086630357744	-0.920228084956716\\
0.434956933406159	-0.920060749682539\\
0.436827493139491	-0.919894138166517\\
0.438698306299638	-0.919728245834218\\
0.440569369673493	-0.919563068148775\\
0.442440680092219	-0.919398600610526\\
0.444312234430538	-0.919234838756659\\
0.446184029606031	-0.919071778160863\\
0.448056062578452	-0.918909414432974\\
0.449928330349059	-0.918747743218635\\
0.451800829959947	-0.918586760198949\\
0.453673558493406	-0.918426461090145\\
0.45554651307128	-0.918266841643238\\
0.457419690854347	-0.918107897643698\\
0.459293089041702	-0.917949624911124\\
0.461166704870157	-0.917792019298912\\
0.463040535613647	-0.917635076693937\\
0.464914578582657	-0.917478793016233\\
0.466788831123644	-0.917323164218675\\
0.468663290618482	-0.917168186286665\\
0.470537954483914	-0.917013855237824\\
0.472412820171009	-0.916860167121685\\
0.474287885164634	-0.916707118019385\\
0.476163146982935	-0.91655470404337\\
0.478038603176823	-0.916402921337092\\
0.479914251329476	-0.916251766074716\\
0.481790089055844	-0.916101234460829\\
0.483666114002162	-0.915951322730149\\
0.485542323845481	-0.915802027147238\\
0.487418716293196	-0.915653344006223\\
0.489295289082586	-0.915505269630513\\
0.491172039980369	-0.915357800372518\\
0.493048966782251	-0.915210932613382\\
0.494926067312497	-0.915064662762705\\
0.496803339423501	-0.914918987258276\\
0.498680780995364	-0.914773902565807\\
0.500558389935482	-0.914629405178669\\
0.502436164178144	-0.91448549161763\\
0.504314101684125	-0.914342158430599\\
0.506192200440301	-0.91419940219237\\
0.50807045845926	-0.914057219504367\\
0.509948873778925	-0.913915606994396\\
0.51182744446218	-0.913774561316397\\
0.513706168596506	-0.913634079150198\\
0.515585044293618	-0.913494157201272\\
0.517464069689116	-0.913354792200499\\
0.519343242942133	-0.913215980903927\\
0.521222562234995	-0.913077720092536\\
0.523102025772883	-0.912940006572008\\
0.524981631783508	-0.912802837172493\\
0.526861378516779	-0.912666208748385\\
0.528741264244488	-0.912530118178095\\
0.530621287259993	-0.912394562363826\\
0.532501445877912	-0.912259538231354\\
0.534381738433817	-0.91212504272981\\
0.536262163283934	-0.911991072831463\\
0.538142718804853	-0.911857625531505\\
0.540023403393236	-0.911724697847842\\
0.541904215465531	-0.91159228682088\\
0.543785153457697	-0.911460389513322\\
0.545666215824924	-0.911329003009961\\
0.547547401041368	-0.911198124417476\\
0.549428707599877	-0.911067750864232\\
0.551310134011736	-0.910937879500081\\
0.553191678806407	-0.910808507496166\\
0.555073340531272	-0.910679632044725\\
0.55695511775139	-0.910551250358898\\
0.558837009049245	-0.910423359672538\\
0.560719013024511	-0.910295957240019\\
0.562601128293807	-0.910169040336055\\
0.564483353490469	-0.910042606255508\\
0.566365687264318	-0.90991665231321\\
0.568248128281432	-0.909791175843781\\
0.570130675223925	-0.90966617420145\\
0.572013326789726	-0.909541644759874\\
0.573896081692366	-0.909417584911969\\
0.575778938660759	-0.909293992069731\\
0.577661896439002	-0.909170863664067\\
0.579544953786161	-0.909048197144624\\
0.581428109476073	-0.908925989979621\\
0.583311362297144	-0.908804239655681\\
0.585194711052153	-0.90868294367767\\
0.587078154558059	-0.908562099568528\\
0.588961691645812	-0.908441704869113\\
0.590845321160163	-0.908321757138039\\
0.59272904195948	-0.908202253951516\\
0.594612852915566	-0.908083192903198\\
0.596496752913483	-0.907964571604024\\
0.598380740851372	-0.907846387682065\\
0.600264815640282	-0.907728638782377\\
0.602148976203996	-0.907611322566842\\
0.604033221478866	-0.90749443671403\\
0.605917550413648	-0.907377978919044\\
0.607801961969333	-0.907261946893375\\
0.609686455118992	-0.907146338364764\\
0.611571028847618	-0.907031151077051\\
0.613455682151963	-0.906916382790042\\
0.615340414040394	-0.906802031279361\\
0.617225223532734	-0.90668809433632\\
0.619110109660115	-0.906574569767774\\
0.620995071464836	-0.906461455395993\\
0.622880108000212	-0.906348749058523\\
0.624765218330432	-0.906236448608053\\
0.626650401530424	-0.906124551912288\\
0.628535656685712	-0.906013056853814\\
0.63042098289228	-0.905901961329971\\
0.63230637925644	-0.905791263252729\\
0.634191844894699	-0.905680960548555\\
0.636077378933627	-0.905571051158293\\
0.63796298050973	-0.905461533037039\\
0.639848648769323	-0.90535240415402\\
0.641734382868407	-0.905243662492469\\
0.643620181972543	-0.90513530604951\\
0.645506045256733	-0.905027332836035\\
0.647391971905302	-0.904919740876588\\
0.649277961111775	-0.904812528209252\\
0.651164012078768	-0.904705692885527\\
0.653050124017869	-0.904599232970221\\
0.654936296149528	-0.904493146541336\\
0.656822527702942	-0.904387431689956\\
0.658708817915952	-0.904282086520134\\
0.660595166034929	-0.904177109148789\\
0.662481571314672	-0.904072497705588\\
0.664368033018301	-0.903968250332847\\
0.666254550417153	-0.903864365185421\\
0.668141122790684	-0.903760840430599\\
0.670027749426365	-0.903657674247998\\
0.671914429619581	-0.903554864829465\\
0.673801162673542	-0.903452410378967\\
0.675687947899176	-0.903350309112498\\
0.677574784615043	-0.903248559257974\\
0.679461672147237	-0.903147159055134\\
0.681348609829293	-0.903046106755444\\
0.6832355970021	-0.902945400621995\\
0.685122633013808	-0.902845038929414\\
0.68700971721974	-0.902745019963763\\
0.688896848982307	-0.902645342022446\\
0.69078402767092	-0.902546003414115\\
0.692671252661904	-0.90244700245858\\
0.694558523338418	-0.902348337486713\\
0.69644583909037	-0.90225000684036\\
0.698333199314336	-0.902152008872253\\
0.700220603413481	-0.902054341945915\\
0.702108050797477	-0.901957004435578\\
0.703995540882428	-0.901859994726092\\
0.705883073090792	-0.901763311212839\\
0.707770646851306	-0.901666952301649\\
0.709658261598906	-0.901570916408712\\
0.711545916774662	-0.901475201960498\\
0.713433611825697	-0.901379807393672\\
0.71532134620512	-0.901284731155009\\
0.717209119371953	-0.901189971701317\\
0.719096930791061	-0.901095527499352\\
0.720984779933083	-0.901001397025741\\
0.722872666274367	-0.900907578766902\\
0.724760589296897	-0.900814071218964\\
0.72664854848823	-0.90072087288769\\
0.728536543341434	-0.9006279822884\\
0.730424573355015	-0.900535397945897\\
0.732312638032861	-0.900443118394387\\
0.734200736884177	-0.900351142177406\\
0.73608886942342	-0.900259467847747\\
0.737977035170241	-0.900168093967387\\
0.739865233649426	-0.90007701910741\\
0.74175346439083	-0.89998624184794\\
0.743641726929326	-0.899895760778067\\
0.745530020804741	-0.899805574495776\\
0.747418345561803	-0.899715681607878\\
0.749306700750082	-0.899626080729939\\
0.751195085923933	-0.899536770486215\\
0.753083500642446	-0.899447749509579\\
0.754971944469385	-0.899359016441457\\
0.756860416973141	-0.899270569931758\\
0.758748917726673	-0.899182408638811\\
0.76063744630746	-0.8990945312293\\
0.762526002297449	-0.899006936378193\\
0.764414585283001	-0.898919622768685\\
0.766303194854844	-0.898832589092128\\
0.768191830608022	-0.898745834047973\\
0.770080492141847	-0.898659356343703\\
0.771969179059848	-0.898573154694775\\
0.773857890969728	-0.898487227824555\\
0.775746627483311	-0.898401574464258\\
0.777635388216502	-0.898316193352889\\
0.779524172789234	-0.898231083237182\\
0.78141298082543	-0.898146242871542\\
0.783301811952951	-0.898061671017984\\
0.785190665803557	-0.897977366446076\\
0.787079542012863	-0.897893327932882\\
0.788968440220293	-0.897809554262905\\
0.790857360069038	-0.89772604422803\\
0.792746301206018	-0.897642796627465\\
0.794635263281836	-0.897559810267692\\
0.796524245950738	-0.897477083962405\\
0.798413248870574	-0.897394616532459\\
0.800302271702758	-0.897312406805816\\
0.802191314112225	-0.897230453617491\\
0.804080375767398	-0.897148755809497\\
0.805969456340144	-0.897067312230797\\
0.80785855550574	-0.896986121737246\\
0.809747672942833	-0.896905183191544\\
0.811636808333404	-0.896824495463181\\
0.813525961362731	-0.896744057428389\\
0.815415131719354	-0.896663867970092\\
0.817304319095035	-0.896583925977854\\
0.81919352318473	-0.896504230347831\\
0.821082743686549	-0.89642477998272\\
0.822971980301721	-0.896345573791713\\
0.824861232734565	-0.89626661069045\\
0.82675050069245	-0.896187889600965\\
0.828639783885767	-0.896109409451647\\
0.830529082027895	-0.896031169177185\\
0.832418394835165	-0.895953167718531\\
0.834307722026834	-0.895875404022842\\
0.836197063325049	-0.895797877043448\\
0.838086418454817	-0.895720585739794\\
0.839975787143976	-0.895643529077404\\
0.841865169123161	-0.895566706027832\\
0.843754564125777	-0.895490115568622\\
0.845643971887968	-0.895413756683259\\
0.847533392148588	-0.895337628361129\\
0.84942282464917	-0.895261729597478\\
0.851312269133902	-0.895186059393364\\
0.853201725349594	-0.89511061675562\\
0.855091193045651	-0.89503540069681\\
0.856980671974049	-0.894960410235187\\
0.858870161889303	-0.894885644394653\\
0.860759662548441	-0.894811102204718\\
0.86264917371098	-0.894736782700461\\
0.864538695138896	-0.894662684922486\\
0.866428226596601	-0.894588807916887\\
0.868317767850917	-0.894515150735206\\
0.870207318671047	-0.894441712434395\\
0.872096878828553	-0.894368492076776\\
0.873986448097334	-0.894295488730004\\
0.875876026253593	-0.89422270146703\\
0.87776561307582	-0.894150129366061\\
0.879655208344768	-0.894077771510523\\
0.881544811843424	-0.894005626989026\\
0.883434423356989	-0.893933694895325\\
0.885324042672856	-0.893861974328285\\
0.887213669580585	-0.893790464391844\\
0.889103303871881	-0.893719164194978\\
0.89099294534057	-0.893648072851665\\
0.892882593782583	-0.89357718948085\\
0.894772248995924	-0.893506513206411\\
0.89666191078066	-0.893436043157122\\
0.898551578938888	-0.893365778466621\\
0.900441253274725	-0.893295718273376\\
0.902330933594278	-0.893225861720647\\
0.90422061970563	-0.893156207956461\\
0.906110311418814	-0.893086756133571\\
0.908000008545799	-0.893017505409425\\
0.909889710900465	-0.892948454946137\\
0.911779418298585	-0.892879603910451\\
0.913669130557806	-0.892810951473708\\
0.915558847497631	-0.89274249681182\\
0.917448568939395	-0.892674239105231\\
0.919338294706254	-0.892606177538891\\
0.921228024623158	-0.892538311302223\\
0.923117758516839	-0.892470639589093\\
0.92500749621579	-0.892403161597778\\
0.926897237550248	-0.892335876530937\\
0.928786982352175	-0.892268783595583\\
0.930676730455242	-0.892201882003047\\
0.932566481694809	-0.892135170968955\\
0.934456235907913	-0.892068649713197\\
0.936345992933245	-0.892002317459895\\
0.938235752611136	-0.891936173437377\\
0.940125514783543	-0.891870216878149\\
0.942015279294026	-0.891804447018864\\
0.943905045987738	-0.891738863100297\\
0.945794814711407	-0.891673464367313\\
0.947684585313319	-0.891608250068844\\
0.949574357643302	-0.891543219457859\\
0.951464131552714	-0.891478371791337\\
0.953353906894423	-0.89141370633024\\
0.955243683522796	-0.891349222339487\\
0.957133461293679	-0.891284919087928\\
0.959023240064388	-0.891220795848314\\
0.96091301969369	-0.891156851897276\\
0.962802800041791	-0.891093086515294\\
0.96469258097032	-0.891029498986678\\
0.966582362342315	-0.890966088599535\\
0.968472144022209	-0.890902854645747\\
0.970361925875818	-0.890839796420949\\
0.972251707770324	-0.890776913224498\\
0.974141489574265	-0.890714204359454\\
0.976031271157518	-0.890651669132551\\
0.977921052391288	-0.890589306854177\\
0.979810833148093	-0.890527116838345\\
0.981700613301755	-0.890465098402673\\
0.983590392727382	-0.890403250868359\\
0.985480171301357	-0.890341573560158\\
0.987369948901329	-0.890280065806356\\
0.989259725406195	-0.890218726938752\\
0.991149500696091	-0.890157556292629\\
0.993039274652379	-0.890096553206737\\
0.994929047157635	-0.890035717023264\\
0.996818818095636	-0.889975047087821\\
0.998708587351352	-0.889914542749412\\
1.00059835481093	-0.889854203360419\\
1.00248812036168	-0.889794028276575\\
1.00437788389208	-0.889734016856945\\
1.00626764529173	-0.889674168463902\\
1.00815740445139	-0.889614482463109\\
1.01004716126292	-0.889554958223494\\
1.01193691561931	-0.889495595117232\\
1.01382666741463	-0.889436392519723\\
1.01571641654405	-0.88937734980957\\
1.01760616290383	-0.88931846636856\\
1.01949590639128	-0.889259741581644\\
1.02138564690477	-0.889201174836914\\
1.02327538434373	-0.889142765525586\\
1.02516511860862	-0.889084513041977\\
1.02705484960093	-0.88902641678349\\
1.02894457722317	-0.888968476150589\\
1.03083430137884	-0.888910690546781\\
1.03272402197246	-0.888853059378599\\
1.03461373890953	-0.888795582055582\\
1.03650345209653	-0.888738257990251\\
1.03839316144091	-0.8886810865981\\
1.04028286685108	-0.888624067297566\\
1.0421725682364	-0.888567199510019\\
1.04406226550717	-0.888510482659739\\
1.04595195857462	-0.8884539161739\\
1.04784164735092	-0.88839749948255\\
1.04973133174913	-0.888341232018593\\
1.05162101168324	-0.888285113217773\\
1.05351068706811	-0.888229142518654\\
1.05540035781953	-0.888173319362604\\
1.05729002385413	-0.888117643193776\\
1.05917968508943	-0.888062113459091\\
1.06106934144381	-0.888006729608224\\
1.06295899283651	-0.88795149109358\\
1.06484863918762	-0.887896397370282\\
1.06673828041806	-0.887841447896155\\
1.06862791644957	-0.887786642131705\\
1.07051754720475	-0.887731979540107\\
1.07240717260698	-0.887677459587182\\
1.07429679258046	-0.88762308174139\\
1.0761864070502	-0.887568845473804\\
1.07807601594199	-0.887514750258101\\
1.07996561918241	-0.887460795570543\\
1.08185521669881	-0.88740698088996\\
1.08374480841932	-0.887353305697738\\
1.08563439427282	-0.887299769477798\\
1.08752397418897	-0.887246371716585\\
1.08941354809816	-0.887193111903052\\
1.09130311593152	-0.887139989528643\\
1.09319267762092	-0.887087004087276\\
1.09508223309896	-0.887034155075334\\
1.09697178229895	-0.886981441991645\\
1.09886132515493	-0.886928864337466\\
1.10075086160163	-0.886876421616475\\
1.1026403915745	-0.886824113334749\\
1.10452991500967	-0.886771939000754\\
1.10641943184395	-0.886719898125328\\
1.10830894201485	-0.886667990221668\\
1.11019844546055	-0.886616214805315\\
1.1120879421199	-0.886564571394142\\
1.11397743193239	-0.886513059508337\\
1.11586691483819	-0.88646167867039\\
1.11775639077811	-0.88641042840508\\
1.11964585969362	-0.886359308239461\\
1.1215353215268	-0.886308317702848\\
1.12342477622037	-0.886257456326803\\
1.1253142237177	-0.886206723645123\\
1.12720366396275	-0.886156119193826\\
1.12909309690011	-0.886105642511137\\
1.13098252247497	-0.886055293137476\\
1.13287194063313	-0.886005070615443\\
1.13476135132098	-0.885954974489809\\
1.13665075448551	-0.885905004307497\\
1.13854015007429	-0.885855159617577\\
1.14042953803548	-0.885805439971246\\
1.1423189183178	-0.885755844921819\\
1.14420829087055	-0.885706374024717\\
1.1460976556436	-0.885657026837453\\
1.14798701258736	-0.88560780291962\\
1.14987636165282	-0.885558701832881\\
1.15176570279149	-0.885509723140952\\
1.15365503595546	-0.885460866409594\\
1.15554436109733	-0.8854121312066\\
1.15743367817024	-0.885363517101783\\
1.15932298712788	-0.885315023666964\\
1.16121228792442	-0.88526665047596\\
1.1631015805146	-0.885218397104574\\
1.16499086485364	-0.885170263130581\\
1.16688014089728	-0.885122248133717\\
1.16876940860177	-0.88507435169567\\
1.17065866792387	-0.885026573400066\\
1.1725479188208	-0.88497891283246\\
1.17443716125032	-0.88493136958032\\
1.17632639517063	-0.884883943233025\\
1.17821562054044	-0.884836633381843\\
1.18010483731894	-0.884789439619929\\
1.18199404546578	-0.884742361542309\\
1.18388324494109	-0.884695398745872\\
1.18577243570546	-0.884648550829357\\
1.18766161771993	-0.884601817393344\\
1.18955079094602	-0.884555198040244\\
1.19143995534569	-0.884508692374286\\
1.19332911088135	-0.884462300001508\\
1.19521825751584	-0.884416020529749\\
1.19710739521246	-0.884369853568633\\
1.19899652393493	-0.884323798729564\\
1.20088564364743	-0.884277855625716\\
1.20277475431454	-0.884232023872017\\
1.20466385590126	-0.884186303085147\\
1.20655294837305	-0.884140692883521\\
1.20844203169573	-0.884095192887284\\
1.21033110583558	-0.884049802718299\\
1.21222017075927	-0.884004522000139\\
1.21410922643387	-0.883959350358073\\
1.21599827282687	-0.883914287419063\\
1.21788730990614	-0.883869332811749\\
1.21977633763994	-0.883824486166442\\
1.22166535599695	-0.883779747115114\\
1.2235543649462	-0.883735115291389\\
1.22544336445714	-0.883690590330533\\
1.22733235449956	-0.883646171869447\\
1.22922133504367	-0.883601859546652\\
1.23111030606003	-0.88355765300229\\
1.23299926751955	-0.883513551878105\\
1.23488821939355	-0.883469555817439\\
1.23677716165369	-0.883425664465222\\
1.23866609427198	-0.883381877467964\\
1.2405550172208	-0.883338194473745\\
1.24244393047289	-0.883294615132207\\
1.24433283400132	-0.883251139094546\\
1.24622172777953	-0.883207766013501\\
1.24811061178129	-0.883164495543349\\
1.24999948598071	-0.883121327339891\\
1.25188835035224	-0.883078261060452\\
1.25377720487067	-0.883035296363864\\
1.25566604951111	-0.882992432910463\\
1.25755488424901	-0.882949670362079\\
1.25944370906014	-0.882907008382026\\
1.2613325239206	-0.882864446635098\\
1.2632213288068	-0.882821984787559\\
1.26511012369547	-0.882779622507132\\
1.26699890856365	-0.882737359462997\\
1.2688876833887	-0.882695195325776\\
1.27077644814829	-0.88265312976753\\
1.27266520282038	-0.882611162461753\\
1.27455394738325	-0.882569293083356\\
1.27644268181547	-0.882527521308667\\
1.27833140609591	-0.882485846815422\\
1.28022012020373	-0.882444269282753\\
1.2821088241184	-0.882402788391186\\
1.28399751781966	-0.882361403822629\\
1.28588620128754	-0.882320115260368\\
1.28777487450236	-0.882278922389056\\
1.28966353744472	-0.88223782489471\\
1.2915521900955	-0.8821968224647\\
1.29344083243585	-0.882155914787743\\
1.29532946444721	-0.882115101553895\\
1.29721808611126	-0.882074382454545\\
1.29910669740999	-0.882033757182408\\
1.30099529832563	-0.881993225431516\\
1.30288388884067	-0.881952786897213\\
1.3047724689379	-0.881912441276148\\
1.30666103860031	-0.881872188266265\\
1.30854959781121	-0.881832027566799\\
1.31043814655412	-0.881791958878269\\
1.31232668481283	-0.881751981902471\\
1.31421521257137	-0.881712096342468\\
1.31610372981404	-0.88167230190259\\
1.31799223652537	-0.881632598288419\\
1.31988073269013	-0.881592985206789\\
1.32176921829335	-0.881553462365777\\
1.32365769332027	-0.881514029474696\\
1.32554615775639	-0.881474686244088\\
1.32743461158745	-0.88143543238572\\
1.3293230547994	-0.881396267612575\\
1.33121148737844	-0.881357191638845\\
1.33309990931099	-0.881318204179928\\
1.3349883205837	-0.88127930495242\\
1.33687672118344	-0.881240493674107\\
1.33876511109732	-0.88120177006396\\
1.34065349031263	-0.881163133842132\\
1.34254185881693	-0.881124584729945\\
1.34443021659796	-0.881086122449889\\
1.34631856364369	-0.881047746725616\\
1.34820689994231	-0.881009457281931\\
1.35009522548219	-0.880971253844788\\
1.35198354025195	-0.880933136141283\\
1.35387184424039	-0.880895103899649\\
1.35576013743652	-0.88085715684925\\
1.35764841982955	-0.880819294720574\\
1.35953669140891	-0.880781517245228\\
1.36142495216421	-0.880743824155932\\
1.36331320208527	-0.880706215186516\\
1.3652014411621	-0.880668690071907\\
1.36708966938491	-0.880631248548131\\
1.36897788674409	-0.880593890352305\\
1.37086609323023	-0.880556615222628\\
1.37275428883411	-0.880519422898381\\
1.3746424735467	-0.880482313119915\\
1.37653064735916	-0.880445285628652\\
1.37841881026281	-0.880408340167076\\
1.38030696224918	-0.880371476478727\\
1.38219510330996	-0.880334694308198\\
1.38408323343705	-0.880297993401128\\
1.38597135262248	-0.880261373504195\\
1.3878594608585	-0.880224834365116\\
1.38974755813751	-0.880188375732636\\
1.3916356444521	-0.880151997356525\\
1.39352371979501	-0.880115698987574\\
1.39541178415917	-0.880079480377589\\
1.39729983753766	-0.880043341279384\\
1.39918787992375	-0.880007281446778\\
1.40107591131085	-0.879971300634591\\
1.40296393169254	-0.879935398598633\\
1.40485194106259	-0.879899575095709\\
1.40673993941489	-0.879863829883603\\
1.40862792674351	-0.879828162721082\\
1.41051590304267	-0.879792573367886\\
1.41240386830677	-0.879757061584723\\
1.41429182253033	-0.879721627133267\\
1.41617976570805	-0.879686269776151\\
1.41806769783476	-0.879650989276965\\
1.41995561890547	-0.879615785400246\\
1.42184352891532	-0.879580657911479\\
1.4237314278596	-0.879545606577086\\
1.42561931573373	-0.87951063116443\\
1.42750719253332	-0.8794757314418\\
1.42939505825409	-0.879440907178414\\
1.4312829128919	-0.879406158144413\\
1.43317075644277	-0.879371484110854\\
1.43505858890286	-0.879336884849706\\
1.43694641026845	-0.879302360133847\\
1.43883422053597	-0.87926790973706\\
1.44072201970198	-0.879233533434025\\
1.4426098077632	-0.879199231000318\\
1.44449758471645	-0.879165002212405\\
1.4463853505587	-0.879130846847639\\
1.44827310528706	-0.879096764684253\\
1.45016084889874	-0.879062755501358\\
1.45204858139111	-0.879028819078937\\
1.45393630276165	-0.878994955197844\\
1.45582401300799	-0.878961163639794\\
1.45771171212785	-0.878927444187364\\
1.45959940011911	-0.878893796623986\\
1.46148707697974	-0.878860220733946\\
1.46337474270786	-0.878826716302374\\
1.4652623973017	-0.878793283115245\\
1.46715004075961	-0.878759920959374\\
1.46903767308006	-0.878726629622409\\
1.47092529426164	-0.878693408892831\\
1.47281290430304	-0.878660258559946\\
1.4747005032031	-0.878627178413885\\
1.47658809096074	-0.878594168245597\\
1.47847566757502	-0.878561227846845\\
1.48036323304509	-0.878528357010204\\
1.48225078737022	-0.878495555529056\\
1.4841383305498	-0.878462823197587\\
1.48602586258332	-0.878430159810779\\
1.48791338347039	-0.878397565164413\\
1.4898008932107	-0.87836503905506\\
1.49168839180407	-0.878332581280078\\
1.49357587925043	-0.878300191637611\\
1.49546335554979	-0.87826786992658\\
1.49735082070229	-0.878235615946685\\
1.49923827470816	-0.878203429498398\\
1.50112571756772	-0.878171310382959\\
1.50301314928141	-0.878139258402374\\
1.50490056984977	-0.878107273359411\\
1.50678797927341	-0.878075355057594\\
1.50867537755309	-0.878043503301203\\
1.51056276468961	-0.878011717895269\\
1.5124501406839	-0.877979998645568\\
1.51433750553698	-0.877948345358622\\
1.51622485924996	-0.877916757841691\\
1.51811220182405	-0.877885235902772\\
1.51999953326054	-0.877853779350596\\
1.52188685356083	-0.877822387994621\\
1.52377416272639	-0.877791061645034\\
1.52566146075879	-0.877759800112742\\
1.5275487476597	-0.877728603209373\\
1.52943602343085	-0.877697470747269\\
1.5313232880741	-0.877666402539486\\
1.53321054159135	-0.877635398399788\\
1.53509778398462	-0.877604458142643\\
1.53698501525599	-0.877573581583226\\
1.53887223540765	-0.877542768537405\\
1.54075944444186	-0.877512018821748\\
1.54264664236096	-0.877481332253514\\
1.54453382916737	-0.877450708650651\\
1.54642100486359	-0.877420147831793\\
1.54830816945223	-0.877389649616256\\
1.55019532293594	-0.877359213824038\\
1.55208246531746	-0.877328840275809\\
1.55396959659962	-0.877298528792917\\
1.55585671678532	-0.877268279197376\\
1.55774382587754	-0.877238091311869\\
1.55963092387933	-0.877207964959742\\
1.56151801079381	-0.877177899965003\\
1.56340508662419	-0.877147896152315\\
1.56529215137375	-0.877117953346998\\
1.56717920504582	-0.877088071375022\\
1.56906624764384	-0.877058250063006\\
1.5709532791713	-0.877028489238214\\
1.57284029963176	-0.876998788728554\\
1.57472730902885	-0.876969148362571\\
1.57661430736627	-0.876939567969449\\
1.57850129464781	-0.876910047379004\\
1.58038827087729	-0.876880586421683\\
1.58227523605863	-0.876851184928561\\
1.58416219019581	-0.876821842731337\\
1.58604913329285	-0.876792559662333\\
1.58793606535388	-0.876763335554491\\
1.58982298638306	-0.876734170241367\\
1.59170989638462	-0.876705063557132\\
1.59359679536287	-0.876676015336567\\
1.59548368332217	-0.87664702541506\\
1.59737056026695	-0.876618093628607\\
1.59925742620169	-0.876589219813803\\
1.60114428113094	-0.876560403807844\\
1.60303112505931	-0.876531645448523\\
1.60491795799147	-0.876502944574226\\
1.60680477993214	-0.876474301023932\\
1.60869159088613	-0.876445714637208\\
1.61057839085826	-0.876417185254206\\
1.61246517985344	-0.876388712715662\\
1.61435195787664	-0.876360296862894\\
1.61623872493287	-0.876331937537795\\
1.61812548102721	-0.876303634582837\\
1.62001222616478	-0.876275387841062\\
1.62189896035076	-0.876247197156084\\
1.62378568359041	-0.876219062372083\\
1.62567239588899	-0.876190983333807\\
1.62755909725187	-0.876162959886562\\
1.62944578768444	-0.876134991876219\\
1.63133246719215	-0.876107079149202\\
1.6332191357805	-0.876079221552494\\
1.63510579345505	-0.876051418933628\\
1.63699244022139	-0.876023671140687\\
1.63887907608518	-0.875995978022303\\
1.64076570105212	-0.875968339427651\\
1.64265231512797	-0.875940755206451\\
1.64453891831853	-0.875913225208961\\
1.64642551062965	-0.875885749285979\\
1.64831209206721	-0.875858327288837\\
1.65019866263718	-0.875830959069399\\
1.65208522234553	-0.875803644480063\\
1.6539717711983	-0.875776383373751\\
1.65585830920158	-0.875749175603914\\
1.65774483636149	-0.875722021024526\\
1.65963135268421	-0.875694919490082\\
1.66151785817595	-0.875667870855595\\
1.66340435284297	-0.875640874976596\\
1.66529083669158	-0.875613931709129\\
1.66717730972812	-0.875587040909752\\
1.66906377195898	-0.875560202435532\\
1.67095022339059	-0.875533416144042\\
1.67283666402944	-0.875506681893361\\
1.67472309388202	-0.875479999542074\\
1.67660951295489	-0.875453368949263\\
1.67849592125466	-0.87542678997451\\
1.68038231878794	-0.875400262477894\\
1.68226870556143	-0.875373786319989\\
1.68415508158183	-0.875347361361858\\
1.68604144685589	-0.875320987465057\\
1.6879278013904	-0.875294664491629\\
1.6898141451922	-0.875268392304102\\
1.69170047826814	-0.875242170765489\\
1.69358680062512	-0.875215999739283\\
1.6954731122701	-0.875189879089456\\
1.69735941321003	-0.875163808680459\\
1.69924570345194	-0.875137788377216\\
1.70113198300286	-0.875111818045127\\
1.70301825186988	-0.875085897550061\\
1.70490451006012	-0.875060026758355\\
1.70679075758072	-0.875034205536816\\
1.70867699443886	-0.875008433752712\\
1.71056322064177	-0.874982711273778\\
1.7124494361967	-0.874957037968207\\
1.71433564111092	-0.874931413704653\\
1.71622183539175	-0.874905838352224\\
1.71810801904655	-0.874880311780486\\
1.71999419208268	-0.874854833859456\\
1.72188035450756	-0.874829404459604\\
1.72376650632862	-0.874804023451846\\
1.72565264755335	-0.874778690707548\\
1.72753877818925	-0.87475340609852\\
1.72942489824384	-0.874728169497015\\
1.73131100772468	-0.874702980775729\\
1.73319710663937	-0.874677839807796\\
1.73508319499553	-0.874652746466787\\
1.7369692728008	-0.87462770062671\\
1.73885534006286	-0.874602702162008\\
1.74074139678942	-0.874577750947553\\
1.7426274429882	-0.874552846858649\\
1.74451347866697	-0.87452798977103\\
1.74639950383351	-0.874503179560852\\
1.74828551849563	-0.8744784161047\\
1.75017152266117	-0.87445369927958\\
1.75205751633801	-0.874429028962919\\
1.75394349953401	-0.874404405032565\\
1.75582947225711	-0.87437982736678\\
1.75771543451525	-0.874355295844245\\
1.75960138631638	-0.874330810344054\\
1.7614873276685	-0.874306370745713\\
1.76337325857963	-0.874281976929138\\
1.76525917905779	-0.874257628774655\\
1.76714508911106	-0.874233326162997\\
1.76903098874752	-0.8742090689753\\
1.77091687797527	-0.874184857093107\\
1.77280275680245	-0.874160690398359\\
1.7746886252372	-0.874136568773401\\
1.77657448328771	-0.874112492100975\\
1.77846033096217	-0.874088460264218\\
1.78034616826879	-0.874064473146666\\
1.78223199521583	-0.874040530632244\\
1.78411781181153	-0.874016632605273\\
1.78600361806419	-0.873992778950462\\
1.7878894139821	-0.873968969552909\\
1.78977519957359	-0.873945204298098\\
1.79166097484699	-0.873921483071901\\
1.79354673981067	-0.87389780576057\\
1.79543249447302	-0.873874172250743\\
1.79731823884243	-0.873850582429436\\
1.79920397292732	-0.873827036184044\\
1.80108969673613	-0.873803533402341\\
1.80297541027731	-0.873780073972475\\
1.80486111355936	-0.87375665778297\\
1.80674680659074	-0.87373328472272\\
1.80863248937998	-0.873709954680994\\
1.81051816193561	-0.873686667547428\\
1.81240382426617	-0.873663423212025\\
1.81428947638021	-0.873640221565158\\
1.81617511828633	-0.873617062497562\\
1.81806074999312	-0.873593945900337\\
1.81994637150919	-0.873570871664944\\
1.82183198284316	-0.873547839683207\\
1.82371758400369	-0.873524849847305\\
1.82560317499942	-0.873501902049779\\
1.82748875583905	-0.873478996183522\\
1.82937432653124	-0.873456132141784\\
1.83125988708472	-0.873433309818169\\
1.8331454375082	-0.87341052910663\\
1.83503097781042	-0.873387789901473\\
1.83691650800011	-0.87336509209735\\
1.83880202808606	-0.873342435589263\\
1.84068753807702	-0.873319820272559\\
1.8425730379818	-0.87329724604293\\
1.8444585278092	-0.873274712796409\\
1.84634400756803	-0.873252220429374\\
1.84822947726712	-0.873229768838542\\
1.85011493691532	-0.873207357920968\\
1.85200038652148	-0.873184987574046\\
1.85388582609447	-0.873162657695506\\
1.85577125564317	-0.873140368183411\\
1.85765667517647	-0.873118118936162\\
1.85954208470327	-0.873095909852487\\
1.8614274842325	-0.873073740831448\\
1.86331287377307	-0.873051611772435\\
1.86519825333394	-0.873029522575168\\
1.86708362292403	-0.873007473139691\\
1.86896898255233	-0.872985463366376\\
1.87085433222779	-0.872963493155918\\
1.8727396719594	-0.872941562409335\\
1.87462500175615	-0.872919671027967\\
1.87651032162704	-0.872897818913473\\
1.87839563158109	-0.872876005967833\\
1.88028093162731	-0.872854232093343\\
1.88216622177475	-0.872832497192616\\
1.88405150203243	-0.87281080116858\\
1.88593677240941	-0.872789143924478\\
1.88782203291476	-0.872767525363865\\
1.88970728355753	-0.872745945390605\\
1.8915925243468	-0.872724403908876\\
1.89347775529167	-0.872702900823162\\
1.89536297640122	-0.872681436038257\\
1.89724818768456	-0.87266000945926\\
1.89913338915079	-0.872638620991575\\
1.90101858080904	-0.87261727054091\\
1.90290376266844	-0.872595958013276\\
1.90478893473811	-0.872574683314987\\
1.9066740970272	-0.872553446352656\\
1.90855924954485	-0.872532247033194\\
1.91044439230023	-0.872511085263813\\
1.9123295253025	-0.872489960952019\\
1.91421464856082	-0.872468874005615\\
1.91609976208437	-0.872447824332698\\
1.91798486588234	-0.872426811841659\\
1.91986995996391	-0.87240583644118\\
1.92175504433828	-0.872384898040236\\
1.92364011901466	-0.872363996548089\\
1.92552518400225	-0.872343131874292\\
1.92741023931025	-0.872322303928684\\
1.92929528494791	-0.872301512621392\\
1.93118032092443	-0.872280757862827\\
1.93306534724905	-0.872260039563685\\
1.934950363931	-0.872239357634944\\
1.93683537097953	-0.872218711987865\\
1.93872036840388	-0.872198102533989\\
1.94060535621329	-0.872177529185138\\
1.94249033441704	-0.872156991853411\\
1.94437530302437	-0.872136490451186\\
1.94626026204455	-0.872116024891117\\
1.94814521148685	-0.872095595086132\\
1.95003015136054	-0.872075200949436\\
1.9519150816749	-0.872054842394504\\
1.9538000024392	-0.872034519335087\\
1.95568491366275	-0.872014231685204\\
1.95756981535481	-0.871993979359144\\
1.95945470752469	-0.871973762271468\\
1.96133959018167	-0.871953580337002\\
1.96322446333507	-0.871933433470841\\
1.96510932699417	-0.871913321588343\\
1.96699418116829	-0.871893244605134\\
1.96887902586672	-0.871873202437102\\
1.9707638610988	-0.871853195000399\\
1.97264868687381	-0.871833222211438\\
1.97453350320109	-0.871813283986892\\
1.97641831008995	-0.871793380243695\\
1.97830310754971	-0.87177351089904\\
1.9801878955897	-0.871753675870376\\
1.98207267421923	-0.871733875075412\\
1.98395744344764	-0.871714108432108\\
1.98584220328426	-0.871694375858684\\
1.98772695373841	-0.87167467727361\\
1.98961169481943	-0.87165501259561\\
1.99149642653666	-0.871635381743661\\
1.99338114889942	-0.871615784636989\\
1.99526586191706	-0.871596221195072\\
1.99715056559892	-0.871576691337634\\
1.99903525995433	-0.871557194984651\\
2.00091994499263	-0.871537732056343\\
2.00280462072317	-0.871518302473178\\
2.00468928715529	-0.871498906155867\\
2.00657394429834	-0.871479543025367\\
2.00845859216164	-0.871460213002878\\
2.01034323075456	-0.871440916009842\\
2.01222786008644	-0.871421651967944\\
2.01411248016662	-0.871402420799107\\
2.01599709100444	-0.871383222425496\\
2.01788169260926	-0.871364056769514\\
2.01976628499041	-0.8713449237538\\
2.02165086815725	-0.871325823301233\\
2.02353544211912	-0.871306755334925\\
2.02542000688536	-0.871287719778227\\
2.02730456246533	-0.87126871655472\\
2.02918910886836	-0.871249745588221\\
2.0310736461038	-0.871230806802778\\
2.032958174181	-0.871211900122673\\
2.0348426931093	-0.871193025472415\\
2.03672720289803	-0.871174182776747\\
2.03861170355655	-0.871155371960637\\
2.04049619509419	-0.871136592949285\\
2.04238067752029	-0.871117845668115\\
2.0442651508442	-0.871099130042779\\
2.04614961507525	-0.871080445999154\\
2.04803407022277	-0.871061793463343\\
2.0499185162961	-0.871043172361671\\
2.05180295330458	-0.871024582620688\\
2.05368738125753	-0.871006024167163\\
2.05557180016429	-0.870987496928091\\
2.05745621003419	-0.870969000830684\\
2.05934061087656	-0.870950535802374\\
2.06122500270071	-0.870932101770813\\
2.06310938551598	-0.870913698663872\\
2.06499375933168	-0.870895326409637\\
2.06687812415714	-0.870876984936411\\
2.06876248000167	-0.870858674172714\\
2.07064682687459	-0.87084039404728\\
2.07253116478521	-0.870822144489056\\
2.07441549374284	-0.870803925427204\\
2.0762998137568	-0.870785736791097\\
2.07818412483638	-0.870767578510322\\
2.08006842699089	-0.870749450514674\\
2.08195272022964	-0.870731352734159\\
2.08383700456191	-0.870713285098994\\
2.08572127999701	-0.870695247539602\\
2.08760554654423	-0.870677239986616\\
2.08948980421286	-0.870659262370875\\
2.09137405301218	-0.870641314623424\\
2.09325829295148	-0.870623396675513\\
2.09514252404004	-0.870605508458599\\
2.09702674628714	-0.870587649904342\\
2.09891095970205	-0.870569820944602\\
2.10079516429404	-0.870552021511447\\
2.10267936007239	-0.870534251537143\\
2.10456354704635	-0.870516510954158\\
2.10644772522518	-0.87049879969516\\
2.10833189461816	-0.870481117693017\\
2.11021605523452	-0.870463464880796\\
2.11210020708351	-0.87044584119176\\
2.1139843501744	-0.870428246559374\\
2.11586848451641	-0.870410680917293\\
2.11775261011878	-0.870393144199374\\
2.11963672699076	-0.870375636339667\\
2.12152083514158	-0.870358157272414\\
2.12340493458045	-0.870340706932054\\
2.1252890253166	-0.870323285253219\\
2.12717310735926	-0.87030589217073\\
2.12905718071764	-0.870288527619605\\
2.13094124540094	-0.870271191535048\\
2.13282530141838	-0.870253883852455\\
2.13470934877916	-0.870236604507413\\
2.13659338749247	-0.870219353435696\\
2.1384774175675	-0.870202130573267\\
2.14036143901345	-0.870184935856277\\
2.14224545183951	-0.870167769221063\\
2.14412945605484	-0.870150630604147\\
2.14601345166863	-0.870133519942241\\
2.14789743869004	-0.870116437172236\\
2.14978141712824	-0.870099382231211\\
2.1516653869924	-0.870082355056427\\
2.15354934829166	-0.870065355585329\\
2.15543330103518	-0.870048383755543\\
2.1573172452321	-0.870031439504877\\
2.15920118089157	-0.87001452277132\\
2.16108510802272	-0.86999763349304\\
2.16296902663468	-0.869980771608388\\
2.16485293673658	-0.869963937055889\\
2.16673683833754	-0.86994712977425\\
2.16862073144667	-0.869930349702354\\
2.17050461607308	-0.869913596779262\\
2.17238849222589	-0.86989687094421\\
2.17427235991418	-0.869880172136611\\
2.17615621914706	-0.869863500296054\\
2.17804006993361	-0.869846855362299\\
2.17992391228291	-0.869830237275284\\
2.18180774620404	-0.869813645975116\\
2.18369157170608	-0.869797081402079\\
2.18557538879809	-0.869780543496627\\
2.18745919748913	-0.869764032199384\\
2.18934299778825	-0.869747547451146\\
2.19122678970451	-0.869731089192881\\
2.19311057324695	-0.869714657365724\\
2.19499434842461	-0.869698251910979\\
2.19687811524651	-0.869681872770121\\
2.19876187372169	-0.86966551988479\\
2.20064562385917	-0.869649193196795\\
2.20252936566795	-0.86963289264811\\
2.20441309915705	-0.869616618180876\\
2.20629682433547	-0.869600369737399\\
2.20818054121221	-0.869584147260152\\
2.21006424979625	-0.869567950691768\\
2.21194795009658	-0.869551779975047\\
2.21383164212217	-0.869535635052952\\
2.215715325882	-0.869519515868606\\
2.21759900138504	-0.869503422365297\\
2.21948266864023	-0.869487354486472\\
2.22136632765654	-0.869471312175741\\
2.22324997844291	-0.869455295376873\\
2.22513362100828	-0.869439304033796\\
2.22701725536158	-0.869423338090598\\
2.22890088151174	-0.869407397491526\\
2.23078449946768	-0.869391482180984\\
2.23266810923832	-0.869375592103535\\
2.23455171083255	-0.869359727203896\\
2.23643530425929	-0.869343887426945\\
2.23831888952742	-0.86932807271771\\
2.24020246664583	-0.86931228302138\\
2.2420860356234	-0.869296518283294\\
2.24396959646901	-0.869280778448949\\
2.24585314919153	-0.869265063463993\\
2.24773669379981	-0.869249373274228\\
2.24962023030271	-0.86923370782561\\
2.25150375870907	-0.869218067064244\\
2.25338727902774	-0.869202450936389\\
2.25527079126754	-0.869186859388454\\
2.25715429543731	-0.869171292366999\\
2.25903779154586	-0.869155749818734\\
2.26092127960201	-0.869140231690517\\
2.26280475961455	-0.869124737929358\\
2.2646882315923	-0.869109268482411\\
2.26657169554403	-0.869093823296983\\
2.26845515147854	-0.869078402320524\\
2.27033859940459	-0.869063005500633\\
2.27222203933097	-0.869047632785055\\
2.27410547126643	-0.869032284121682\\
2.27598889521973	-0.869016959458548\\
2.27787231119961	-0.869001658743836\\
2.27975571921483	-0.868986381925871\\
2.2816391192741	-0.868971128953122\\
2.28352251138616	-0.868955899774202\\
2.28540589555973	-0.868940694337866\\
2.28728927180352	-0.868925512593013\\
2.28917264012623	-0.868910354488682\\
2.29105600053656	-0.868895219974054\\
2.2929393530432	-0.868880108998453\\
2.29482269765483	-0.86886502151134\\
2.29670603438013	-0.868849957462318\\
2.29858936322777	-0.868834916801129\\
2.3004726842064	-0.868819899477655\\
2.30235599732468	-0.868804905441914\\
2.30423930259125	-0.868789934644066\\
2.30612260001475	-0.868774987034405\\
2.30800588960381	-0.868760062563363\\
2.30988917136705	-0.86874516118151\\
2.31177244531309	-0.868730282839552\\
2.31365571145054	-0.868715427488328\\
2.31553896978798	-0.868700595078817\\
2.31742222033402	-0.868685785562128\\
2.31930546309724	-0.868670998889507\\
2.32118869808622	-0.868656235012333\\
2.32307192530952	-0.868641493882121\\
2.3249551447757	-0.868626775450514\\
2.32683835649332	-0.868612079669293\\
2.32872156047093	-0.868597406490366\\
2.33060475671705	-0.868582755865777\\
2.33248794524023	-0.868568127747699\\
2.33437112604898	-0.868553522088435\\
2.33625429915182	-0.868538938840421\\
2.33813746455725	-0.868524377956221\\
2.34002062227378	-0.868509839388527\\
2.34190377230988	-0.868495323090164\\
2.34378691467406	-0.868480829014083\\
2.34567004937477	-0.868466357113362\\
2.34755317642048	-0.86845190734121\\
2.34943629581967	-0.86843747965096\\
2.35131940758077	-0.868423073996074\\
2.35320251171223	-0.86840869033014\\
2.35508560822248	-0.868394328606872\\
2.35696869711996	-0.868379988780109\\
2.35885177841307	-0.868365670803815\\
2.36073485211024	-0.86835137463208\\
2.36261791821986	-0.868337100219117\\
2.36450097675033	-0.868322847519264\\
2.36638402771003	-0.868308616486982\\
2.36826707110735	-0.868294407076855\\
2.37015010695065	-0.868280219243591\\
2.3720331352483	-0.868266052942017\\
2.37391615600865	-0.868251908127085\\
2.37579916924005	-0.868237784753867\\
2.37768217495083	-0.868223682777558\\
2.37956517314932	-0.86820960215347\\
2.38144816384385	-0.868195542837038\\
2.38333114704274	-0.868181504783818\\
2.38521412275427	-0.868167487949481\\
2.38709709098676	-0.868153492289821\\
2.38898005174849	-0.868139517760749\\
2.39086300504774	-0.868125564318295\\
2.39274595089278	-0.868111631918607\\
2.39462888929188	-0.86809772051795\\
2.3965118202533	-0.868083830072705\\
2.39839474378527	-0.868069960539372\\
2.40027765989605	-0.868056111874566\\
2.40216056859385	-0.868042284035017\\
2.40404346988691	-0.868028476977574\\
2.40592636378344	-0.868014690659197\\
2.40780925029164	-0.868000925036963\\
2.40969212941971	-0.867987180068064\\
2.41157500117585	-0.867973455709805\\
2.41345786556822	-0.867959751919604\\
2.41534072260501	-0.867946068654994\\
2.41722357229438	-0.86793240587362\\
2.41910641464448	-0.86791876353324\\
2.42098924966346	-0.867905141591723\\
2.42287207735947	-0.867891540007052\\
2.42475489774062	-0.867877958737319\\
2.42663771081505	-0.867864397740729\\
2.42852051659087	-0.867850856975596\\
2.43040331507618	-0.867837336400347\\
2.43228610627908	-0.867823835973516\\
2.43416889020766	-0.867810355653749\\
2.43605166686999	-0.8677968953998\\
2.43793443627416	-0.867783455170533\\
2.43981719842822	-0.867770034924918\\
2.44169995334023	-0.867756634622037\\
2.44358270101823	-0.867743254221078\\
2.44546544147026	-0.867729893681335\\
2.44734817470436	-0.867716552962213\\
2.44923090072854	-0.867703232023219\\
2.45111361955081	-0.867689930823971\\
2.45299633117918	-0.86767664932419\\
2.45487903562164	-0.867663387483705\\
2.45676173288619	-0.867650145262449\\
2.45864442298079	-0.867636922620461\\
2.46052710591342	-0.867623719517884\\
2.46240978169204	-0.867610535914965\\
2.4642924503246	-0.867597371772057\\
2.46617511181905	-0.867584227049616\\
2.46805776618332	-0.8675711017082\\
2.46994041342533	-0.867557995708471\\
2.47182305355302	-0.867544909011196\\
2.47370568657428	-0.86753184157724\\
2.47558831249702	-0.867518793367574\\
2.47747093132913	-0.867505764343269\\
2.4793535430785	-0.867492754465498\\
2.481236147753	-0.867479763695534\\
2.48311874536049	-0.867466791994752\\
2.48500133590884	-0.867453839324628\\
2.4868839194059	-0.867440905646736\\
2.48876649585951	-0.867427990922752\\
2.49064906527749	-0.86741509511445\\
2.49253162766768	-0.867402218183705\\
2.49441418303789	-0.867389360092488\\
2.49629673139593	-0.867376520802872\\
2.49817927274959	-0.867363700277026\\
2.50006180710667	-0.867350898477217\\
2.50194433447494	-0.867338115365811\\
2.50382685486218	-0.867325350905269\\
2.50570936827615	-0.867312605058152\\
2.50759187472462	-0.867299877787115\\
2.50947437421531	-0.867287169054911\\
2.51135686675598	-0.867274478824389\\
2.51323935235435	-0.867261807058493\\
2.51512183101814	-0.867249153720263\\
2.51700430275506	-0.867236518772833\\
2.51888676757283	-0.867223902179434\\
2.52076922547912	-0.867211303903391\\
2.52265167648164	-0.867198723908121\\
2.52453412058805	-0.867186162157137\\
2.52641655780603	-0.867173618614047\\
2.52829898814324	-0.867161093242549\\
2.53018141160732	-0.867148586006437\\
2.53206382820592	-0.867136096869596\\
2.53394623794668	-0.867123625796004\\
2.53582864083722	-0.867111172749731\\
2.53771103688516	-0.86709873769494\\
2.53959342609811	-0.867086320595885\\
2.54147580848366	-0.867073921416911\\
2.54335818404941	-0.867061540122453\\
2.54524055280294	-0.867049176677039\\
2.54712291475183	-0.867036831045287\\
2.54900526990363	-0.867024503191904\\
2.55088761826592	-0.867012193081687\\
2.55276995984623	-0.866999900679524\\
2.5546522946521	-0.866987625950391\\
2.55653462269107	-0.866975368859355\\
2.55841694397065	-0.866963129371568\\
2.56029925849837	-0.866950907452275\\
2.56218156628172	-0.866938703066806\\
2.56406386732821	-0.866926516180581\\
2.56594616164531	-0.866914346759106\\
2.56782844924052	-0.866902194767975\\
2.56971073012129	-0.86689006017287\\
2.57159300429509	-0.866877942939559\\
2.57347527176938	-0.866865843033896\\
2.57535753255159	-0.866853760421823\\
2.57723978664917	-0.866841695069368\\
2.57912203406954	-0.866829646942642\\
2.58100427482012	-0.866817616007844\\
2.58288650890833	-0.866805602231259\\
2.58476873634155	-0.866793605579254\\
2.58665095712718	-0.866781626018284\\
2.58853317127262	-0.866769663514886\\
2.59041537878523	-0.866757718035682\\
2.59229757967238	-0.866745789547378\\
2.59417977394142	-0.866733878016765\\
2.59606196159972	-0.866721983410715\\
2.5979441426546	-0.866710105696185\\
2.59982631711341	-0.866698244840213\\
2.60170848498346	-0.866686400809923\\
2.60359064627207	-0.866674573572517\\
2.60547280098655	-0.866662763095283\\
2.6073549491342	-0.866650969345589\\
2.6092370907223	-0.866639192290885\\
2.61111922575814	-0.866627431898702\\
2.61300135424898	-0.866615688136653\\
2.6148834762021	-0.866603960972431\\
2.61676559162474	-0.86659225037381\\
2.61864770052415	-0.866580556308645\\
2.62052980290758	-0.86656887874487\\
2.62241189878224	-0.8665572176505\\
2.62429398815537	-0.866545572993628\\
2.62617607103417	-0.866533944742429\\
2.62805814742584	-0.866522332865155\\
2.62994021733759	-0.866510737330138\\
2.6318222807766	-0.866499158105788\\
2.63370433775004	-0.866487595160594\\
2.63558638826509	-0.866476048463123\\
2.6374684323289	-0.866464517982019\\
2.63935046994863	-0.866453003686006\\
2.64123250113142	-0.866441505543884\\
2.64311452588442	-0.866430023524529\\
2.64499654421473	-0.866418557596897\\
2.64687855612949	-0.866407107730019\\
2.6487605616358	-0.866395673893002\\
2.65064256074076	-0.866384256055031\\
2.65252455345146	-0.866372854185365\\
2.654406539775	-0.866361468253341\\
2.65628851971844	-0.866350098228371\\
2.65817049328885	-0.866338744079941\\
2.66005246049329	-0.866327405777614\\
2.66193442133881	-0.866316083291027\\
2.66381637583246	-0.866304776589891\\
2.66569832398126	-0.866293485643994\\
2.66758026579223	-0.866282210423196\\
2.6694622012724	-0.86627095089743\\
2.67134413042878	-0.866259707036707\\
2.67322605326835	-0.866248478811108\\
2.67510796979812	-0.866237266190789\\
2.67698988002507	-0.866226069145977\\
2.67887178395616	-0.866214887646975\\
2.68075368159836	-0.866203721664157\\
2.68263557295863	-0.866192571167969\\
2.68451745804392	-0.866181436128931\\
2.68639933686117	-0.866170316517634\\
2.68828120941731	-0.86615921230474\\
2.69016307571927	-0.866148123460984\\
2.69204493577394	-0.866137049957172\\
2.69392678958826	-0.866125991764181\\
2.6958086371691	-0.866114948852959\\
2.69769047852336	-0.866103921194525\\
2.69957231365793	-0.866092908759968\\
2.70145414257966	-0.866081911520449\\
2.70333596529544	-0.866070929447196\\
2.70521778181211	-0.866059962511511\\
2.70709959213652	-0.866049010684761\\
2.7089813962755	-0.866038073938387\\
2.7108631942359	-0.866027152243897\\
2.71274498602454	-0.866016245572868\\
2.71462677164821	-0.866005353896946\\
2.71650855111374	-0.865994477187847\\
2.71839032442792	-0.865983615417353\\
2.72027209159754	-0.865972768557317\\
2.72215385262937	-0.865961936579659\\
2.72403560753019	-0.865951119456366\\
2.72591735630677	-0.865940317159493\\
2.72779909896586	-0.865929529661164\\
2.7296808355142	-0.865918756933567\\
2.73156256595854	-0.865907998948962\\
2.7334442903056	-0.865897255679671\\
2.73532600856212	-0.865886527098085\\
2.73720772073479	-0.865875813176661\\
2.73908942683033	-0.865865113887924\\
2.74097112685544	-0.865854429204461\\
2.7428528208168	-0.865843759098929\\
2.74473450872109	-0.865833103544049\\
2.74661619057499	-0.865822462512607\\
2.74849786638516	-0.865811835977456\\
2.75037953615826	-0.865801223911511\\
2.75226119990094	-0.865790626287756\\
2.75414285761983	-0.865780043079237\\
2.75602450932156	-0.865769474259064\\
2.75790615501276	-0.865758919800414\\
2.75978779470005	-0.865748379676526\\
2.76166942839003	-0.865737853860703\\
2.7635510560893	-0.865727342326314\\
2.76543267780444	-0.865716845046789\\
2.76731429354205	-0.865706361995622\\
2.76919590330869	-0.865695893146372\\
2.77107750711093	-0.865685438472658\\
2.77295910495532	-0.865674997948165\\
2.77484069684843	-0.865664571546639\\
2.77672228279678	-0.865654159241888\\
2.77860386280692	-0.865643761007783\\
2.78048543688536	-0.865633376818257\\
2.78236700503862	-0.865623006647306\\
2.78424856727321	-0.865612650468986\\
2.78613012359564	-0.865602308257416\\
2.78801167401239	-0.865591979986775\\
2.78989321852995	-0.865581665631305\\
2.79177475715479	-0.865571365165306\\
2.79365628989338	-0.865561078563143\\
2.79553781675218	-0.865550805799239\\
2.79741933773765	-0.865540546848078\\
2.79930085285623	-0.865530301684204\\
2.80118236211435	-0.865520070282222\\
2.80306386551843	-0.865509852616796\\
2.80494536307491	-0.865499648662652\\
2.80682685479019	-0.865489458394572\\
2.80870834067068	-0.865479281787401\\
2.81058982072276	-0.86546911881604\\
2.81247129495284	-0.865458969455453\\
2.81435276336727	-0.865448833680659\\
2.81623422597245	-0.865438711466739\\
2.81811568277473	-0.86542860278883\\
2.81999713378046	-0.86541850762213\\
2.821878578996	-0.865408425941893\\
2.82376001842768	-0.865398357723432\\
2.82564145208184	-0.865388302942118\\
2.82752287996479	-0.86537826157338\\
2.82940430208285	-0.865368233592705\\
2.83128571844234	-0.865358218975635\\
2.83316712904955	-0.865348217697772\\
2.83504853391076	-0.865338229734775\\
2.83692993303227	-0.865328255062358\\
2.83881132642036	-0.865318293656293\\
2.84069271408128	-0.865308345492408\\
2.8425740960213	-0.865298410546589\\
2.84445547224666	-0.865288488794777\\
2.84633684276363	-0.865278580212969\\
2.84821820757842	-0.865268684777218\\
2.85009956669727	-0.865258802463634\\
2.8519809201264	-0.865248933248381\\
2.85386226787202	-0.865239077107679\\
2.85574360994033	-0.865229234017805\\
2.85762494633754	-0.865219403955089\\
2.85950627706983	-0.865209586895916\\
2.86138760214338	-0.865199782816728\\
2.86326892156436	-0.86518999169402\\
2.86515023533895	-0.865180213504341\\
2.8670315434733	-0.865170448224296\\
2.86891284597355	-0.865160695830543\\
2.87079414284585	-0.865150956299795\\
2.87267543409634	-0.865141229608819\\
2.87455671973114	-0.865131515734435\\
2.87643799975637	-0.865121814653516\\
2.87831927417814	-0.86511212634299\\
2.88020054300255	-0.865102450779839\\
2.8820818062357	-0.865092787941095\\
2.88396306388368	-0.865083137803845\\
2.88584431595256	-0.86507350034523\\
2.88772556244842	-0.865063875542442\\
2.88960680337733	-0.865054263372725\\
2.89148803874533	-0.865044663813378\\
2.89336926855848	-0.86503507684175\\
2.89525049282282	-0.865025502435243\\
2.89713171154439	-0.865015940571311\\
2.8990129247292	-0.865006391227459\\
2.90089413238328	-0.864996854381244\\
2.90277533451264	-0.864987330010277\\
2.90465653112328	-0.864977818092216\\
2.9065377222212	-0.864968318604773\\
2.90841890781238	-0.864958831525712\\
2.9103000879028	-0.864949356832845\\
2.91218126249844	-0.864939894504036\\
2.91406243160526	-0.864930444517202\\
2.91594359522922	-0.864921006850308\\
2.91782475337627	-0.864911581481369\\
2.91970590605234	-0.864902168388451\\
2.92158705326338	-0.864892767549672\\
2.92346819501531	-0.864883378943197\\
2.92534933131405	-0.864874002547243\\
2.92723046216551	-0.864864638340075\\
2.9291115875756	-0.864855286300009\\
2.9309927075502	-0.864845946405409\\
2.93287382209522	-0.864836618634689\\
2.93475493121653	-0.864827302966313\\
2.93663603492001	-0.864817999378793\\
2.93851713321152	-0.86480870785069\\
2.94039822609692	-0.864799428360613\\
2.94227931358207	-0.864790160887221\\
2.9441603956728	-0.864780905409221\\
2.94604147237496	-0.864771661905367\\
2.94792254369437	-0.864762430354463\\
2.94980360963685	-0.864753210735361\\
2.95168467020822	-0.864744003026959\\
2.95356572541429	-0.864734807208205\\
2.95544677526085	-0.864725623258093\\
2.95732781975369	-0.864716451155665\\
2.95920885889861	-0.864707290880011\\
2.96108989270137	-0.864698142410268\\
2.96297092116775	-0.86468900572562\\
2.96485194430351	-0.864679880805297\\
2.9667329621144	-0.864670767628578\\
2.96861397460617	-0.864661666174787\\
2.97049498178456	-0.864652576423294\\
2.97237598365531	-0.864643498353519\\
2.97425698022414	-0.864634431944924\\
2.97613797149676	-0.86462537717702\\
2.97801895747889	-0.864616334029363\\
2.97989993817622	-0.864607302481555\\
2.98178091359447	-0.864598282513244\\
2.98366188373931	-0.864589274104124\\
2.98554284861643	-0.864580277233935\\
2.98742380823149	-0.864571291882461\\
2.98930476259018	-0.864562318029532\\
2.99118571169813	-0.864553355655024\\
2.99306665556102	-0.864544404738856\\
2.99494759418448	-0.864535465260995\\
2.99682852757416	-0.864526537201451\\
2.99870945573567	-0.864517620540278\\
3.00059037867465	-0.864508715257577\\
3.00247129639671	-0.86449982133349\\
3.00435220890746	-0.864490938748207\\
3.0062331162125	-0.86448206748196\\
3.00811401831742	-0.864473207515025\\
3.00999491522782	-0.864464358827724\\
3.01187580694926	-0.86445552140042\\
3.01375669348734	-0.864446695213523\\
3.0156375748476	-0.864437880247483\\
3.01751845103561	-0.864429076482796\\
3.01939932205692	-0.864420283900002\\
3.02128018791708	-0.864411502479682\\
3.02316104862162	-0.86440273220246\\
3.02504190417606	-0.864393973049007\\
3.02692275458594	-0.864385225000032\\
3.02880359985677	-0.864376488036289\\
3.03068443999406	-0.864367762138576\\
3.03256527500331	-0.864359047287731\\
3.03444610489001	-0.864350343464636\\
3.03632692965965	-0.864341650650215\\
3.03820774931771	-0.864332968825434\\
3.04008856386967	-0.864324297971302\\
3.04196937332098	-0.864315638068868\\
3.04385017767712	-0.864306989099225\\
3.04573097694352	-0.864298351043506\\
3.04761177112564	-0.864289723882888\\
3.04949256022892	-0.864281107598588\\
3.05137334425878	-0.864272502171863\\
3.05325412322064	-0.864263907584014\\
3.05513489711994	-0.864255323816382\\
3.05701566596206	-0.86424675085035\\
3.05889642975243	-0.864238188667339\\
3.06077718849643	-0.864229637248814\\
3.06265794219946	-0.86422109657628\\
3.06453869086689	-0.864212566631283\\
3.0664194345041	-0.864204047395408\\
3.06830017311646	-0.864195538850281\\
3.07018090670933	-0.86418704097757\\
3.07206163528806	-0.86417855375898\\
3.07394235885801	-0.86417007717626\\
3.07582307742451	-0.864161611211197\\
3.07770379099289	-0.864153155845616\\
3.07958449956849	-0.864144711061385\\
3.08146520315662	-0.86413627684041\\
3.0833459017626	-0.864127853164638\\
3.08522659539173	-0.864119440016053\\
3.08710728404931	-0.86411103737668\\
3.08898796774064	-0.864102645228583\\
3.09086864647099	-0.864094263553866\\
3.09274932024565	-0.86408589233467\\
3.0946299890699	-0.864077531553177\\
3.09651065294899	-0.864069181191607\\
3.09839131188818	-0.864060841232218\\
3.10027196589273	-0.864052511657308\\
3.10215261496788	-0.864044192449212\\
3.10403325911887	-0.864035883590306\\
3.10591389835093	-0.864027585063\\
3.10779453266928	-0.864019296849747\\
3.10967516207915	-0.864011018933036\\
3.11155578658574	-0.864002751295392\\
3.11343640619426	-0.863994493919381\\
3.1153170209099	-0.863986246787606\\
3.11719763073786	-0.863978009882707\\
3.11907823568332	-0.863969783187362\\
3.12095883575146	-0.863961566684286\\
3.12283943094745	-0.863953360356233\\
3.12472002127645	-0.863945164185993\\
3.12660060674363	-0.863936978156392\\
3.12848118735412	-0.863928802250297\\
3.13036176311308	-0.863920636450607\\
3.13224233402565	-0.863912480740263\\
3.13412290009694	-0.863904335102238\\
3.1360034613321	-0.863896199519545\\
3.13788401773623	-0.863888073975233\\
3.13976456931445	-0.863879958452388\\
3.14164511607186	-0.863871852934129\\
3.14352565801356	-0.863863757403616\\
3.14540619514464	-0.863855671844043\\
3.14728672747019	-0.863847596238641\\
3.14916725499528	-0.863839530570674\\
3.15104777772498	-0.863831474823447\\
3.15292829566437	-0.863823428980297\\
3.15480880881849	-0.863815393024598\\
3.1566893171924	-0.863807366939759\\
3.15856982079115	-0.863799350709227\\
3.16045031961978	-0.863791344316482\\
3.16233081368331	-0.863783347745039\\
3.16421130298677	-0.863775360978451\\
3.16609178753519	-0.863767384000304\\
3.16797226733357	-0.863759416794219\\
3.16985274238692	-0.863751459343854\\
3.17173321270024	-0.863743511632901\\
3.17361367827853	-0.863735573645085\\
3.17549413912677	-0.863727645364168\\
3.17737459524995	-0.863719726773946\\
3.17925504665303	-0.86371181785825\\
3.18113549334098	-0.863703918600945\\
3.18301593531877	-0.863696028985929\\
3.18489637259135	-0.863688148997137\\
3.18677680516367	-0.863680278618536\\
3.18865723304067	-0.863672417834129\\
3.19053765622729	-0.863664566627952\\
3.19241807472845	-0.863656724984074\\
3.19429848854908	-0.8636488928866\\
3.1961788976941	-0.863641070319667\\
3.19805930216841	-0.863633257267446\\
3.19993970197693	-0.863625453714142\\
3.20182009712454	-0.863617659643994\\
3.20370048761614	-0.863609875041273\\
3.20558087345662	-0.863602099890284\\
3.20746125465085	-0.863594334175366\\
3.2093416312037	-0.863586577880891\\
3.21122200312004	-0.863578830991262\\
3.21310237040473	-0.863571093490918\\
3.21498273306263	-0.863563365364329\\
3.21686309109858	-0.863555646595998\\
3.21874344451742	-0.863547937170461\\
3.22062379332399	-0.863540237072287\\
3.22250413752311	-0.863532546286077\\
3.22438447711962	-0.863524864796465\\
3.22626481211831	-0.863517192588117\\
3.22814514252401	-0.86350952964573\\
3.23002546834152	-0.863501875954037\\
3.23190578957563	-0.8634942314978\\
3.23378610623113	-0.863486596261812\\
3.23566641831282	-0.863478970230902\\
3.23754672582547	-0.863471353389928\\
3.23942702877384	-0.86346374572378\\
3.24130732716272	-0.863456147217381\\
3.24318762099686	-0.863448557855684\\
3.24506791028101	-0.863440977623675\\
3.24694819501992	-0.863433406506371\\
3.24882847521833	-0.863425844488821\\
3.25070875088098	-0.863418291556103\\
3.25258902201261	-0.863410747693328\\
3.25446928861792	-0.86340321288564\\
3.25634955070164	-0.86339568711821\\
3.25822980826849	-0.863388170376244\\
3.26011006132316	-0.863380662644976\\
3.26199030987036	-0.863373163909672\\
3.26387055391477	-0.863365674155628\\
3.2657507934611	-0.863358193368173\\
3.26763102851401	-0.863350721532664\\
3.26951125907818	-0.86334325863449\\
3.27139148515829	-0.863335804659069\\
3.27327170675898	-0.863328359591852\\
3.27515192388493	-0.863320923418317\\
3.27703213654079	-0.863313496123975\\
3.27891234473119	-0.863306077694366\\
3.28079254846078	-0.863298668115059\\
3.28267274773418	-0.863291267371655\\
3.28455294255604	-0.863283875449784\\
3.28643313293096	-0.863276492335105\\
3.28831331886357	-0.863269118013309\\
3.29019350035846	-0.863261752470114\\
3.29207367742026	-0.86325439569127\\
3.29395385005354	-0.863247047662554\\
3.29583401826291	-0.863239708369776\\
3.29771418205295	-0.863232377798771\\
3.29959434142824	-0.863225055935407\\
3.30147449639336	-0.86321774276558\\
3.30335464695286	-0.863210438275214\\
3.30523479311132	-0.863203142450265\\
3.30711493487328	-0.863195855276713\\
3.3089950722433	-0.863188576740573\\
3.31087520522593	-0.863181306827885\\
3.3127553338257	-0.863174045524718\\
3.31463545804714	-0.863166792817171\\
3.31651557789479	-0.863159548691371\\
3.31839569337316	-0.863152313133474\\
3.32027580448676	-0.863145086129664\\
3.32215591124011	-0.863137867666153\\
3.32403601363771	-0.863130657729184\\
3.32591611168406	-0.863123456305024\\
3.32779620538365	-0.863116263379971\\
3.32967629474096	-0.863109078940352\\
3.33155637976048	-0.86310190297252\\
3.33343646044668	-0.863094735462857\\
3.33531653680403	-0.863087576397772\\
3.337196608837	-0.863080425763703\\
3.33907667655003	-0.863073283547116\\
3.34095673994759	-0.863066149734502\\
3.34283679903411	-0.863059024312385\\
3.34471685381404	-0.863051907267311\\
3.34659690429182	-0.863044798585856\\
3.34847695047186	-0.863037698254624\\
3.3503569923586	-0.863030606260246\\
3.35223702995645	-0.863023522589379\\
3.35411706326982	-0.86301644722871\\
3.35599709230313	-0.86300938016495\\
3.35787711706076	-0.863002321384838\\
3.35975713754712	-0.862995270875143\\
3.36163715376659	-0.862988228622656\\
3.36351716572356	-0.8629811946142\\
3.36539717342241	-0.86297416883662\\
3.3672771768675	-0.862967151276792\\
3.3691571760632	-0.862960141921616\\
3.37103717101388	-0.862953140758019\\
3.37291716172389	-0.862946147772957\\
3.37479714819758	-0.862939162953408\\
3.3766771304393	-0.86293218628638\\
3.37855710845337	-0.862925217758908\\
3.38043708224415	-0.862918257358049\\
3.38231705181594	-0.862911305070891\\
3.38419701717308	-0.862904360884546\\
3.38607697831989	-0.862897424786151\\
3.38795693526066	-0.862890496762871\\
3.38983688799971	-0.862883576801896\\
3.39171683654135	-0.862876664890444\\
3.39359678088985	-0.862869761015754\\
3.39547672104951	-0.862862865165097\\
3.39735665702462	-0.862855977325764\\
3.39923658881945	-0.862849097485077\\
3.40111651643827	-0.862842225630378\\
3.40299643988535	-0.862835361749039\\
3.40487635916496	-0.862828505828456\\
3.40675627428134	-0.86282165785605\\
3.40863618523876	-0.862814817819268\\
3.41051609204144	-0.862807985705582\\
3.41239599469364	-0.862801161502489\\
3.41427589319959	-0.862794345197511\\
3.41615578756351	-0.862787536778195\\
3.41803567778963	-0.862780736232115\\
3.41991556388217	-0.862773943546868\\
3.42179544584534	-0.862767158710076\\
3.42367532368334	-0.862760381709387\\
3.42555519740039	-0.862753612532472\\
3.42743506700067	-0.862746851167029\\
3.42931493248837	-0.862740097600779\\
3.43119479386768	-0.862733351821468\\
3.43307465114279	-0.862726613816866\\
3.43495450431786	-0.86271988357477\\
3.43683435339707	-0.862713161082998\\
3.43871419838458	-0.862706446329394\\
3.44059403928455	-0.862699739301827\\
3.44247387610113	-0.86269303998819\\
3.44435370883847	-0.862686348376398\\
3.44623353750071	-0.862679664454394\\
3.448113362092	-0.862672988210141\\
3.44999318261645	-0.86266631963163\\
3.45187299907821	-0.862659658706872\\
3.4537528114814	-0.862653005423905\\
3.45563261983012	-0.862646359770789\\
3.45751242412849	-0.862639721735609\\
3.45939222438062	-0.862633091306473\\
3.4612720205906	-0.862626468471513\\
3.46315181276254	-0.862619853218884\\
3.46503160090052	-0.862613245536766\\
3.46691138500863	-0.862606645413361\\
3.46879116509095	-0.862600052836895\\
3.47067094115155	-0.862593467795618\\
3.4725507131945	-0.862586890277801\\
3.47443048122387	-0.862580320271742\\
3.47631024524372	-0.862573757765759\\
3.47819000525809	-0.862567202748194\\
3.48006976127104	-0.862560655207414\\
3.48194951328662	-0.862554115131806\\
3.48382926130885	-0.862547582509782\\
3.48570900534178	-0.862541057329777\\
3.48758874538943	-0.862534539580247\\
3.48946848145582	-0.862528029249674\\
3.49134821354498	-0.862521526326559\\
3.49322794166091	-0.862515030799429\\
3.49510766580762	-0.862508542656831\\
3.49698738598912	-0.862502061887337\\
3.4988671022094	-0.862495588479541\\
3.50074681447246	-0.862489122422057\\
3.50262652278228	-0.862482663703524\\
3.50450622714284	-0.862476212312604\\
3.50638592755813	-0.862469768237978\\
3.50826562403211	-0.862463331468353\\
3.51014531656875	-0.862456901992457\\
3.51202500517201	-0.862450479799038\\
3.51390468984586	-0.862444064876869\\
3.51578437059423	-0.862437657214744\\
3.51766404742108	-0.862431256801478\\
3.51954372033035	-0.86242486362591\\
3.52142338932598	-0.8624184776769\\
3.5233030544119	-0.862412098943328\\
3.52518271559203	-0.8624057274141\\
3.52706237287031	-0.862399363078139\\
3.52894202625063	-0.862393005924393\\
3.53082167573692	-0.862386655941831\\
3.53270132133308	-0.862380313119442\\
3.53458096304302	-0.862373977446239\\
3.53646060087063	-0.862367648911254\\
3.53834023481981	-0.862361327503542\\
3.54021986489443	-0.86235501321218\\
3.54209949109839	-0.862348706026264\\
3.54397911343556	-0.862342405934914\\
3.54585873190981	-0.862336112927269\\
3.54773834652501	-0.86232982699249\\
3.54961795728502	-0.86232354811976\\
3.5514975641937	-0.862317276298282\\
3.5533771672549	-0.86231101151728\\
3.55525676647248	-0.862304753766\\
3.55713636185027	-0.862298503033707\\
3.55901595339211	-0.86229225930969\\
3.56089554110183	-0.862286022583256\\
3.56277512498326	-0.862279792843733\\
3.56465470504024	-0.862273570080472\\
3.56653428127656	-0.862267354282842\\
3.56841385369606	-0.862261145440235\\
3.57029342230253	-0.862254943542062\\
3.57217298709978	-0.862248748577754\\
3.57405254809162	-0.862242560536765\\
3.57593210528183	-0.862236379408566\\
3.5778116586742	-0.862230205182652\\
3.57969120827253	-0.862224037848537\\
3.58157075408058	-0.862217877395753\\
3.58345029610214	-0.862211723813855\\
3.58532983434098	-0.862205577092418\\
3.58720936880085	-0.862199437221037\\
3.58908889948553	-0.862193304189325\\
3.59096842639877	-0.862187177986917\\
3.59284794954432	-0.862181058603469\\
3.59472746892593	-0.862174946028655\\
3.59660698454733	-0.86216884025217\\
3.59848649641227	-0.862162741263727\\
3.60036600452448	-0.862156649053062\\
3.60224550888769	-0.862150563609929\\
3.60412500950562	-0.862144484924101\\
3.60600450638198	-0.862138412985371\\
3.6078839995205	-0.862132347783554\\
3.60976348892488	-0.862126289308481\\
3.61164297459882	-0.862120237550006\\
3.61352245654603	-0.862114192497999\\
3.6154019347702	-0.862108154142352\\
3.61728140927502	-0.862102122472976\\
3.61916088006417	-0.8620960974798\\
3.62104034714135	-0.862090079152774\\
3.62291981051022	-0.862084067481867\\
3.62479927017445	-0.862078062457067\\
3.62667872613771	-0.862072064068379\\
3.62855817840367	-0.862066072305832\\
3.63043762697599	-0.862060087159469\\
3.63231707185831	-0.862054108619355\\
3.63419651305428	-0.862048136675574\\
3.63607595056756	-0.862042171318227\\
3.63795538440177	-0.862036212537436\\
3.63983481456056	-0.862030260323342\\
3.64171424104755	-0.862024314666101\\
3.64359366386637	-0.862018375555894\\
3.64547308302065	-0.862012442982915\\
3.64735249851399	-0.862006516937379\\
3.64923191035001	-0.862000597409522\\
3.65111131853231	-0.861994684389594\\
3.65299072306451	-0.861988777867868\\
3.6548701239502	-0.861982877834631\\
3.65674952119297	-0.861976984280193\\
3.65862891479641	-0.861971097194879\\
3.66050830476411	-0.861965216569035\\
3.66238769109965	-0.861959342393023\\
3.66426707380661	-0.861953474657225\\
3.66614645288854	-0.861947613352041\\
3.66802582834903	-0.861941758467888\\
3.66990520019164	-0.861935909995202\\
3.67178456841992	-0.861930067924438\\
3.67366393303742	-0.861924232246068\\
3.6755432940477	-0.861918402950583\\
3.6774226514543	-0.86191258002849\\
3.67930200526076	-0.861906763470316\\
3.68118135547061	-0.861900953266606\\
3.6830607020874	-0.861895149407921\\
3.68494004511464	-0.861889351884842\\
3.68681938455585	-0.861883560687966\\
3.68869872041456	-0.861877775807909\\
3.69057805269428	-0.861871997235305\\
3.69245738139851	-0.861866224960804\\
3.69433670653077	-0.861860458975075\\
3.69621602809455	-0.861854699268804\\
3.69809534609335	-0.861848945832694\\
3.69997466053066	-0.861843198657469\\
3.70185397140997	-0.861837457733865\\
3.70373327873476	-0.86183172305264\\
3.70561258250851	-0.861825994604567\\
3.7074918827347	-0.861820272380436\\
3.70937117941678	-0.861814556371058\\
3.71125047255824	-0.861808846567256\\
3.71312976216253	-0.861803142959874\\
3.7150090482331	-0.861797445539772\\
3.71688833077342	-0.861791754297827\\
3.71876760978692	-0.861786069224933\\
3.72064688527706	-0.861780390312002\\
3.72252615724727	-0.861774717549962\\
3.72440542570098	-0.86176905092976\\
3.72628469064164	-0.861763390442356\\
3.72816395207265	-0.861757736078731\\
3.73004320999746	-0.861752087829881\\
3.73192246441947	-0.861746445686819\\
3.7338017153421	-0.861740809640575\\
3.73568096276877	-0.861735179682195\\
3.73756020670286	-0.861729555802743\\
3.7394394471478	-0.861723937993299\\
3.74131868410696	-0.861718326244961\\
3.74319791758375	-0.86171272054884\\
3.74507714758156	-0.861707120896068\\
3.74695637410376	-0.86170152727779\\
3.74883559715375	-0.861695939685171\\
3.75071481673488	-0.861690358109389\\
3.75259403285055	-0.86168478254164\\
3.75447324550411	-0.861679212973136\\
3.75635245469892	-0.861673649395107\\
3.75823166043835	-0.861668091798798\\
3.76011086272575	-0.861662540175469\\
3.76199006156447	-0.861656994516398\\
3.76386925695786	-0.861651454812879\\
3.76574844890927	-0.861645921056222\\
3.76762763742201	-0.861640393237753\\
3.76950682249945	-0.861634871348814\\
3.77138600414489	-0.861629355380763\\
3.77326518236167	-0.861623845324975\\
3.77514435715312	-0.86161834117284\\
3.77702352852254	-0.861612842915764\\
3.77890269647325	-0.861607350545169\\
3.78078186100856	-0.861601864052494\\
3.78266102213178	-0.861596383429191\\
3.7845401798462	-0.861590908666731\\
3.78641933415512	-0.861585439756599\\
3.78829848506185	-0.861579976690297\\
3.79017763256965	-0.861574519459341\\
3.79205677668183	-0.861569068055264\\
3.79393591740166	-0.861563622469613\\
3.79581505473241	-0.861558182693953\\
3.79769418867736	-0.861552748719864\\
3.79957331923978	-0.861547320538939\\
3.80145244642294	-0.86154189814279\\
3.80333157023008	-0.861536481523042\\
3.80521069066448	-0.861531070671336\\
3.80708980772937	-0.86152566557933\\
3.80896892142801	-0.861520266238694\\
3.81084803176365	-0.861514872641117\\
3.81272713873952	-0.861509484778301\\
3.81460624235886	-0.861504102641964\\
3.81648534262491	-0.861498726223838\\
3.81836443954088	-0.861493355515673\\
3.82024353311001	-0.861487990509231\\
3.82212262333551	-0.861482631196291\\
3.82400171022061	-0.861477277568647\\
3.8258807937685	-0.861471929618108\\
3.82775987398241	-0.861466587336496\\
3.82963895086553	-0.861461250715651\\
3.83151802442107	-0.861455919747426\\
3.83339709465223	-0.86145059442369\\
3.83527616156218	-0.861445274736326\\
3.83715522515413	-0.861439960677232\\
3.83903428543126	-0.861434652238322\\
3.84091334239675	-0.861429349411523\\
3.84279239605377	-0.861424052188777\\
3.8446714464055	-0.861418760562043\\
3.84655049345511	-0.861413474523292\\
3.84842953720576	-0.86140819406451\\
3.85030857766061	-0.861402919177699\\
3.85218761482282	-0.861397649854874\\
3.85406664869555	-0.861392386088067\\
3.85594567928194	-0.861387127869322\\
3.85782470658513	-0.861381875190698\\
3.85970373060828	-0.861376628044269\\
3.86158275135451	-0.861371386422123\\
3.86346176882697	-0.861366150316364\\
3.86534078302878	-0.861360919719107\\
3.86721979396306	-0.861355694622485\\
3.86909880163295	-0.861350475018643\\
3.87097780604155	-0.861345260899741\\
3.87285680719199	-0.861340052257952\\
3.87473580508737	-0.861334849085466\\
3.8766147997308	-0.861329651374485\\
3.87849379112539	-0.861324459117226\\
3.88037277927423	-0.861319272305918\\
3.88225176418042	-0.861314090932808\\
3.88413074584705	-0.861308914990153\\
3.88600972427721	-0.861303744470227\\
3.88788869947399	-0.861298579365317\\
3.88976767144045	-0.861293419667723\\
3.89164664017969	-0.86128826536976\\
3.89352560569477	-0.861283116463757\\
3.89540456798876	-0.861277972942056\\
3.89728352706473	-0.861272834797013\\
3.89916248292574	-0.861267702020999\\
3.90104143557485	-0.861262574606398\\
3.9029203850151	-0.861257452545606\\
3.90479933124956	-0.861252335831037\\
3.90667827428126	-0.861247224455113\\
3.90855721411326	-0.861242118410274\\
3.91043615074858	-0.861237017688973\\
3.91231508419027	-0.861231922283675\\
3.91419401444136	-0.861226832186859\\
3.91607294150486	-0.861221747391019\\
3.91795186538382	-0.861216667888661\\
3.91983078608125	-0.861211593672304\\
3.92170970360016	-0.861206524734483\\
3.92358861794357	-0.861201461067743\\
3.92546752911449	-0.861196402664646\\
3.92734643711593	-0.861191349517764\\
3.92922534195088	-0.861186301619684\\
3.93110424362234	-0.861181258963007\\
3.93298314213331	-0.861176221540345\\
3.93486203748679	-0.861171189344325\\
3.93674092968575	-0.861166162367588\\
3.93861981873319	-0.861161140602786\\
3.94049870463207	-0.861156124042585\\
3.94237758738539	-0.861151112679664\\
3.9442564669961	-0.861146106506716\\
3.94613534346719	-0.861141105516447\\
3.94801421680161	-0.861136109701574\\
3.94989308700233	-0.861131119054828\\
3.95177195407231	-0.861126133568956\\
3.9536508180145	-0.861121153236713\\
3.95552967883185	-0.86111617805087\\
3.95740853652731	-0.861111208004211\\
3.95928739110383	-0.861106243089531\\
3.96116624256434	-0.861101283299639\\
3.96304509091178	-0.861096328627356\\
3.96492393614909	-0.861091379065518\\
3.96680277827919	-0.861086434606972\\
3.96868161730501	-0.861081495244577\\
3.97056045322948	-0.861076560971206\\
3.97243928605551	-0.861071631779745\\
3.97431811578601	-0.861066707663091\\
3.9761969424239	-0.861061788614155\\
3.97807576597209	-0.86105687462586\\
3.97995458643349	-0.861051965691142\\
3.98183340381099	-0.861047061802948\\
3.98371221810749	-0.861042162954241\\
3.98559102932588	-0.861037269137993\\
3.98746983746906	-0.861032380347189\\
3.98934864253992	-0.861027496574829\\
3.99122744454133	-0.861022617813921\\
3.99310624347619	-0.86101774405749\\
3.99498503934735	-0.861012875298571\\
3.99686383215771	-0.861008011530211\\
3.99874262191012	-0.861003152745469\\
4.00062140860745	-0.860998298937419\\
4.00250019225258	-0.860993450099144\\
4.00437897284834	-0.86098860622374\\
4.00625775039761	-0.860983767304318\\
4.00813652490324	-0.860978933333997\\
4.01001529636807	-0.860974104305911\\
4.01189406479494	-0.860969280213204\\
4.01377283018671	-0.860964461049035\\
4.0156515925462	-0.860959646806571\\
4.01753035187626	-0.860954837478996\\
4.01940910817972	-0.860950033059501\\
4.0212878614594	-0.860945233541293\\
4.02316661171813	-0.860940438917588\\
4.02504535895873	-0.860935649181616\\
4.02692410318402	-0.860930864326618\\
4.02880284439682	-0.860926084345847\\
4.03068158259993	-0.860921309232567\\
4.03256031779616	-0.860916538980055\\
4.03443904998832	-0.8609117735816\\
4.0363177791792	-0.860907013030502\\
4.03819650537162	-0.860902257320073\\
4.04007522856835	-0.860897506443636\\
4.04195394877219	-0.860892760394527\\
4.04383266598594	-0.860888019166094\\
4.04571138021236	-0.860883282751694\\
4.04759009145425	-0.860878551144698\\
4.04946879971438	-0.860873824338488\\
4.05134750499553	-0.860869102326457\\
4.05322620730046	-0.860864385102012\\
4.05510490663195	-0.860859672658567\\
4.05698360299276	-0.860854964989552\\
4.05886229638564	-0.860850262088405\\
4.06074098681336	-0.860845563948579\\
4.06261967427867	-0.860840870563535\\
4.06449835878432	-0.860836181926747\\
4.06637704033306	-0.8608314980317\\
4.06825571892763	-0.860826818871892\\
4.07013439457078	-0.860822144440829\\
4.07201306726524	-0.860817474732032\\
4.07389173701374	-0.86081280973903\\
4.07577040381902	-0.860808149455366\\
4.07764906768381	-0.860803493874592\\
4.07952772861082	-0.860798842990273\\
4.08140638660279	-0.860794196795985\\
4.08328504166242	-0.860789555285313\\
4.08516369379244	-0.860784918451855\\
4.08704234299555	-0.860780286289221\\
4.08892098927446	-0.860775658791029\\
4.09079963263188	-0.860771035950912\\
4.09267827307051	-0.860766417762512\\
4.09455691059305	-0.86076180421948\\
4.09643554520219	-0.860757195315482\\
4.09831417690062	-0.860752591044193\\
4.10019280569104	-0.860747991399298\\
4.10207143157613	-0.860743396374494\\
4.10395005455857	-0.860738805963489\\
4.10582867464103	-0.860734220160002\\
4.10770729182621	-0.860729638957763\\
4.10958590611676	-0.860725062350512\\
4.11146451751537	-0.860720490332\\
4.11334312602468	-0.860715922895989\\
4.11522173164738	-0.860711360036252\\
4.11710033438611	-0.860706801746573\\
4.11897893424354	-0.860702248020746\\
4.12085753122231	-0.860697698852575\\
4.12273612532508	-0.860693154235877\\
4.1246147165545	-0.860688614164478\\
4.12649330491321	-0.860684078632214\\
4.12837189040385	-0.860679547632934\\
4.13025047302905	-0.860675021160495\\
4.13212905279146	-0.860670499208767\\
4.1340076296937	-0.860665981771628\\
4.13588620373841	-0.860661468842968\\
4.1377647749282	-0.860656960416688\\
4.1396433432657	-0.860652456486698\\
4.14152190875353	-0.860647957046919\\
4.1434004713943	-0.860643462091284\\
4.14527903119063	-0.860638971613735\\
4.14715758814513	-0.860634485608223\\
4.1490361422604	-0.860630004068712\\
4.15091469353904	-0.860625526989174\\
4.15279324198367	-0.860621054363595\\
4.15467178759686	-0.860616586185966\\
4.15655033038122	-0.860612122450293\\
4.15842887033934	-0.86060766315059\\
4.16030740747381	-0.860603208280881\\
4.16218594178721	-0.860598757835201\\
4.16406447328212	-0.860594311807596\\
4.16594300196112	-0.86058987019212\\
4.16782152782679	-0.860585432982839\\
4.16970005088169	-0.860581000173828\\
4.17157857112841	-0.860576571759174\\
4.1734570885695	-0.860572147732971\\
4.17533560320753	-0.860567728089326\\
4.17721411504505	-0.860563312822355\\
4.17909262408464	-0.860558901926183\\
4.18097113032883	-0.860554495394946\\
4.18284963378019	-0.86055009322279\\
4.18472813444125	-0.860545695403871\\
4.18660663231457	-0.860541301932355\\
4.18848512740269	-0.860536912802417\\
4.19036361970814	-0.860532528008244\\
4.19224210923346	-0.86052814754403\\
4.19412059598119	-0.860523771403981\\
4.19599907995385	-0.860519399582313\\
4.19787756115398	-0.86051503207325\\
4.19975603958409	-0.860510668871027\\
4.20163451524671	-0.860506309969889\\
4.20351298814435	-0.86050195536409\\
4.20539145827953	-0.860497605047894\\
4.20726992565477	-0.860493259015576\\
4.20914839027257	-0.860488917261419\\
4.21102685213544	-0.860484579779716\\
4.21290531124587	-0.86048024656477\\
4.21478376760638	-0.860475917610893\\
4.21666222121946	-0.860471592912409\\
4.2185406720876	-0.860467272463648\\
4.2204191202133	-0.860462956258952\\
4.22229756559904	-0.860458644292672\\
4.22417600824731	-0.860454336559168\\
4.22605444816059	-0.860450033052812\\
4.22793288534136	-0.860445733767981\\
4.2298113197921	-0.860441438699065\\
4.23168975151528	-0.860437147840462\\
4.23356818051337	-0.860432861186581\\
4.23544660678884	-0.860428578731839\\
4.23732503034416	-0.860424300470663\\
4.23920345118179	-0.860420026397488\\
4.24108186930418	-0.86041575650676\\
4.2429602847138	-0.860411490792935\\
4.24483869741309	-0.860407229250476\\
4.24671710740452	-0.860402971873856\\
4.24859551469051	-0.86039871865756\\
4.25047391927353	-0.860394469596078\\
4.25235232115601	-0.860390224683912\\
4.25423072034039	-0.860385983915573\\
4.25610911682911	-0.86038174728558\\
4.2579875106246	-0.860377514788463\\
4.25986590172929	-0.860373286418759\\
4.2617442901456	-0.860369062171017\\
4.26362267587598	-0.860364842039791\\
4.26550105892282	-0.860360626019649\\
4.26737943928857	-0.860356414105165\\
4.26925781697562	-0.860352206290922\\
4.2711361919864	-0.860348002571513\\
4.27301456432331	-0.860343802941541\\
4.27489293398876	-0.860339607395616\\
4.27677130098516	-0.860335415928359\\
4.27864966531492	-0.860331228534397\\
4.28052802698042	-0.86032704520837\\
4.28240638598406	-0.860322865944923\\
4.28428474232825	-0.860318690738714\\
4.28616309601536	-0.860314519584406\\
4.28804144704779	-0.860310352476673\\
4.28991979542793	-0.860306189410198\\
4.29179814115815	-0.860302030379672\\
4.29367648424083	-0.860297875379795\\
4.29555482467836	-0.860293724405277\\
4.29743316247311	-0.860289577450835\\
4.29931149762744	-0.860285434511196\\
4.30118983014372	-0.860281295581096\\
4.30306816002433	-0.860277160655278\\
4.30494648727163	-0.860273029728497\\
4.30682481188797	-0.860268902795512\\
4.30870313387571	-0.860264779851096\\
4.31058145323721	-0.860260660890026\\
4.31245976997481	-0.860256545907091\\
4.31433808409088	-0.860252434897087\\
4.31621639558775	-0.860248327854819\\
4.31809470446777	-0.860244224775101\\
4.31997301073328	-0.860240125652756\\
4.32185131438662	-0.860236030482613\\
4.32372961543012	-0.860231939259512\\
4.32560791386611	-0.860227851978302\\
4.32748620969694	-0.860223768633838\\
4.32936450292491	-0.860219689220987\\
4.33124279355237	-0.860215613734621\\
4.33312108158162	-0.860211542169622\\
4.334999367015	-0.860207474520881\\
4.33687764985481	-0.860203410783297\\
4.33875593010337	-0.860199350951777\\
4.34063420776299	-0.860195295021237\\
4.34251248283599	-0.8601912429866\\
4.34439075532465	-0.860187194842801\\
4.3462690252313	-0.860183150584779\\
4.34814729255823	-0.860179110207483\\
4.35002555730773	-0.860175073705872\\
4.3519038194821	-0.860171041074911\\
4.35378207908364	-0.860167012309574\\
4.35566033611463	-0.860162987404844\\
4.35753859057737	-0.860158966355712\\
4.35941684247413	-0.860154949157176\\
4.36129509180719	-0.860150935804244\\
4.36317333857884	-0.86014692629193\\
4.36505158279136	-0.860142920615259\\
4.36692982444701	-0.860138918769263\\
4.36880806354807	-0.860134920748981\\
4.37068630009681	-0.860130926549462\\
4.37256453409549	-0.860126936165761\\
4.37444276554637	-0.860122949592943\\
4.37632099445173	-0.860118966826081\\
4.3781992208138	-0.860114987860255\\
4.38007744463486	-0.860111012690553\\
4.38195566591715	-0.860107041312072\\
4.38383388466292	-0.860103073719917\\
4.38571210087442	-0.860099109909201\\
4.3875903145539	-0.860095149875044\\
4.3894685257036	-0.860091193612575\\
4.39134673432575	-0.86008724111693\\
4.3932249404226	-0.860083292383255\\
4.39510314399638	-0.860079347406701\\
4.39698134504932	-0.86007540618243\\
4.39885954358365	-0.860071468705608\\
4.4007377396016	-0.860067534971414\\
4.4026159331054	-0.86006360497503\\
4.40449412409726	-0.86005967871165\\
4.4063723125794	-0.860055756176472\\
4.40825049855404	-0.860051837364704\\
4.4101286820234	-0.860047922271563\\
4.41200686298968	-0.86004401089227\\
4.4138850414551	-0.860040103222058\\
4.41576321742186	-0.860036199256164\\
4.41764139089216	-0.860032298989837\\
4.41951956186821	-0.860028402418329\\
4.42139773035221	-0.860024509536903\\
4.42327589634634	-0.860020620340828\\
4.42515405985282	-0.860016734825383\\
4.42703222087381	-0.860012852985852\\
4.42891037941153	-0.860008974817528\\
4.43078853546814	-0.860005100315711\\
4.43266668904585	-0.86000122947571\\
4.43454484014681	-0.85999736229284\\
4.43642298877322	-0.859993498762424\\
4.43830113492726	-0.859989638879793\\
4.44017927861109	-0.859985782640286\\
4.44205741982688	-0.859981930039248\\
4.44393555857681	-0.859978081072033\\
4.44581369486304	-0.859974235734003\\
4.44769182868773	-0.859970394020525\\
4.44956996005305	-0.859966555926975\\
4.45144808896115	-0.859962721448738\\
4.45332621541419	-0.859958890581203\\
4.45520433941433	-0.85995506331977\\
4.45708246096371	-0.859951239659845\\
4.45896058006449	-0.85994741959684\\
4.4608386967188	-0.859943603126176\\
4.4627168109288	-0.859939790243282\\
4.46459492269663	-0.859935980943593\\
4.46647303202442	-0.859932175222552\\
4.46835113891431	-0.859928373075608\\
4.47022924336844	-0.85992457449822\\
4.47210734538893	-0.859920779485852\\
4.47398544497792	-0.859916988033976\\
4.47586354213753	-0.859913200138072\\
4.47774163686989	-0.859909415793626\\
4.47961972917712	-0.859905634996133\\
4.48149781906134	-0.859901857741094\\
4.48337590652466	-0.859898084024017\\
4.4852539915692	-0.859894313840418\\
4.48713207419708	-0.859890547185819\\
4.48901015441039	-0.859886784055751\\
4.49088823221126	-0.859883024445751\\
4.49276630760178	-0.859879268351363\\
4.49464438058405	-0.85987551576814\\
4.49652245116019	-0.859871766691639\\
4.49840051933227	-0.859868021117426\\
4.50027858510241	-0.859864279041075\\
4.50215664847269	-0.859860540458166\\
4.50403470944521	-0.859856805364284\\
4.50591276802204	-0.859853073755026\\
4.50779082420529	-0.859849345625992\\
4.50966887799702	-0.859845620972789\\
4.51154692939933	-0.859841899791035\\
4.51342497841428	-0.85983818207635\\
4.51530302504397	-0.859834467824364\\
4.51718106929045	-0.859830757030714\\
4.51905911115581	-0.859827049691043\\
4.52093715064211	-0.859823345801\\
4.52281518775142	-0.859819645356243\\
4.5246932224858	-0.859815948352437\\
4.52657125484732	-0.859812254785251\\
4.52844928483803	-0.859808564650365\\
4.53032731246	-0.859804877943463\\
4.53220533771527	-0.859801194660236\\
4.53408336060591	-0.859797514796383\\
4.53596138113395	-0.85979383834761\\
4.53783939930146	-0.859790165309628\\
4.53971741511047	-0.859786495678158\\
4.54159542856302	-0.859782829448923\\
4.54347343966117	-0.859779166617659\\
4.54535144840695	-0.859775507180103\\
4.54722945480239	-0.859771851132002\\
4.54910745884954	-0.859768198469109\\
4.55098546055041	-0.859764549187184\\
4.55286345990705	-0.859760903281993\\
4.55474145692147	-0.859757260749309\\
4.55661945159571	-0.859753621584912\\
4.55849744393178	-0.85974998578459\\
4.56037543393171	-0.859746353344134\\
4.56225342159752	-0.859742724259345\\
4.56413140693122	-0.85973909852603\\
4.56600938993482	-0.859735476140002\\
4.56788737061034	-0.85973185709708\\
4.56976534895978	-0.859728241393092\\
4.57164332498515	-0.859724629023869\\
4.57352129868846	-0.859721019985253\\
4.57539927007171	-0.859717414273089\\
4.5772772391369	-0.859713811883229\\
4.57915520588602	-0.859710212811534\\
4.58103317032108	-0.859706617053869\\
4.58291113244406	-0.859703024606106\\
4.58478909225695	-0.859699435464125\\
4.58666704976175	-0.859695849623811\\
4.58854500496044	-0.859692267081056\\
4.59042295785501	-0.859688687831758\\
4.59230090844743	-0.859685111871823\\
4.59417885673969	-0.859681539197161\\
4.59605680273377	-0.85967796980369\\
4.59793474643163	-0.859674403687335\\
4.59981268783526	-0.859670840844026\\
4.60169062694662	-0.8596672812697\\
4.60356856376769	-0.859663724960301\\
4.60544649830042	-0.859660171911777\\
4.60732443054678	-0.859656622120087\\
4.60920236050874	-0.859653075581192\\
4.61108028818826	-0.85964953229106\\
4.61295821358729	-0.859645992245667\\
4.61483613670778	-0.859642455440995\\
4.6167140575517	-0.859638921873032\\
4.61859197612098	-0.859635391537771\\
4.62046989241759	-0.859631864431213\\
4.62234780644347	-0.859628340549364\\
4.62422571820056	-0.859624819888238\\
4.6261036276908	-0.859621302443853\\
4.62798153491613	-0.859617788212236\\
4.6298594398785	-0.859614277189417\\
4.63173734257983	-0.859610769371435\\
4.63361524302207	-0.859607264754333\\
4.63549314120714	-0.859603763334162\\
4.63737103713696	-0.859600265106978\\
4.63924893081348	-0.859596770068844\\
4.64112682223861	-0.859593278215828\\
4.64300471141428	-0.859589789544005\\
4.6448825983424	-0.859586304049457\\
4.6467604830249	-0.85958282172827\\
4.64863836546369	-0.859579342576538\\
4.65051624566069	-0.859575866590359\\
4.6523941236178	-0.85957239376584\\
4.65427199933694	-0.859568924099091\\
4.65614987282001	-0.859565457586231\\
4.65802774406893	-0.859561994223382\\
4.65990561308559	-0.859558534006675\\
4.66178347987189	-0.859555076932245\\
4.66366134442974	-0.859551622996233\\
4.66553920676103	-0.859548172194788\\
4.66741706686765	-0.859544724524062\\
4.66929492475151	-0.859541279980216\\
4.67117278041448	-0.859537838559415\\
4.67305063385846	-0.859534400257831\\
4.67492848508534	-0.859530965071641\\
4.676806334097	-0.859527532997028\\
4.67868418089532	-0.859524104030182\\
4.68056202548217	-0.859520678167298\\
4.68243986785945	-0.859517255404577\\
4.68431770802903	-0.859513835738227\\
4.68619554599277	-0.85951041916446\\
4.68807338175256	-0.859507005679495\\
4.68995121531026	-0.859503595279557\\
4.69182904666773	-0.859500187960875\\
4.69370687582685	-0.859496783719688\\
4.69558470278948	-0.859493382552235\\
4.69746252755747	-0.859489984454767\\
4.6993403501327	-0.859486589423535\\
4.70121817051701	-0.859483197454801\\
4.70309598871226	-0.859479808544829\\
4.70497380472031	-0.85947642268989\\
4.706851618543	-0.859473039886261\\
4.70872943018218	-0.859469660130225\\
4.71060723963971	-0.85946628341807\\
4.71248504691742	-0.859462909746091\\
4.71436285201716	-0.859459539110586\\
4.71624065494077	-0.859456171507862\\
4.71811845569009	-0.85945280693423\\
4.71999625426695	-0.859449445386006\\
4.72187405067319	-0.859446086859513\\
4.72375184491065	-0.859442731351079\\
4.72562963698115	-0.859439378857039\\
4.72750742688652	-0.859436029373732\\
4.72938521462859	-0.859432682897502\\
4.73126300020918	-0.8594293394247\\
4.73314078363012	-0.859425998951684\\
4.73501856489322	-0.859422661474814\\
4.73689634400031	-0.859419326990459\\
4.73877412095319	-0.859415995494991\\
4.7406518957537	-0.85941266698479\\
4.74252966840363	-0.859409341456239\\
4.7444074389048	-0.859406018905728\\
4.74628520725901	-0.859402699329653\\
4.74816297346808	-0.859399382724414\\
4.75004073753381	-0.859396069086418\\
4.75191849945799	-0.859392758412076\\
4.75379625924244	-0.859389450697807\\
4.75567401688895	-0.859386145940032\\
4.75755177239931	-0.859382844135181\\
4.75942952577532	-0.859379545279687\\
4.76130727701877	-0.859376249369989\\
4.76318502613146	-0.859372956402532\\
4.76506277311517	-0.859369666373766\\
4.76694051797168	-0.859366379280147\\
4.76881826070279	-0.859363095118135\\
4.77069600131028	-0.859359813884197\\
4.77257373979592	-0.859356535574806\\
4.77445147616149	-0.859353260186437\\
4.77632921040878	-0.859349987715573\\
4.77820694253955	-0.859346718158703\\
4.78008467255558	-0.859343451512319\\
4.78196240045865	-0.859340187772921\\
4.78384012625051	-0.859336926937011\\
4.78571784993294	-0.8593336690011\\
4.7875955715077	-0.859330413961702\\
4.78947329097655	-0.859327161815336\\
4.79135100834126	-0.859323912558529\\
4.79322872360358	-0.859320666187809\\
4.79510643676528	-0.859317422699714\\
4.7969841478281	-0.859314182090784\\
4.79886185679381	-0.859310944357565\\
4.80073956366415	-0.859307709496609\\
4.80261726844087	-0.859304477504473\\
4.80449497112572	-0.859301248377717\\
4.80637267172044	-0.859298022112911\\
4.80825037022678	-0.859294798706626\\
4.81012806664649	-0.859291578155439\\
4.8120057609813	-0.859288360455933\\
4.81388345323294	-0.859285145604697\\
4.81576114340317	-0.859281933598323\\
4.8176388314937	-0.859278724433409\\
4.81951651750627	-0.85927551810656\\
4.82139420144262	-0.859272314614384\\
4.82327188330447	-0.859269113953494\\
4.82514956309355	-0.859265916120509\\
4.82702724081158	-0.859262721112053\\
4.82890491646029	-0.859259528924755\\
4.83078259004139	-0.859256339555249\\
4.83266026155661	-0.859253153000175\\
4.83453793100766	-0.859249969256177\\
4.83641559839627	-0.859246788319903\\
4.83829326372413	-0.85924361018801\\
4.84017092699297	-0.859240434857155\\
4.8420485882045	-0.859237262324003\\
4.84392624736042	-0.859234092585225\\
4.84580390446243	-0.859230925637494\\
4.84768155951225	-0.85922776147749\\
4.84955921251157	-0.859224600101898\\
4.8514368634621	-0.859221441507407\\
4.85331451236553	-0.859218285690711\\
4.85519215922356	-0.859215132648511\\
4.85706980403788	-0.85921198237751\\
4.8589474468102	-0.859208834874418\\
4.8608250875422	-0.85920569013595\\
4.86270272623556	-0.859202548158825\\
4.86458036289198	-0.859199408939767\\
4.86645799751314	-0.859196272475504\\
4.86833563010073	-0.859193138762772\\
4.87021326065643	-0.859190007798309\\
4.87209088918192	-0.859186879578859\\
4.87396851567887	-0.859183754101171\\
4.87584614014897	-0.859180631361999\\
4.87772376259389	-0.8591775113581\\
4.8796013830153	-0.859174394086238\\
4.88147900141487	-0.859171279543182\\
4.88335661779427	-0.859168167725704\\
4.88523423215517	-0.859165058630582\\
4.88711184449924	-0.8591619522546\\
4.88898945482813	-0.859158848594544\\
4.89086706314352	-0.859155747647206\\
4.89274466944705	-0.859152649409385\\
4.8946222737404	-0.859149553877881\\
4.89649987602521	-0.859146461049502\\
4.89837747630314	-0.859143370921059\\
4.90025507457585	-0.859140283489367\\
4.90213267084498	-0.859137198751249\\
4.90401026511218	-0.859134116703529\\
4.90588785737911	-0.859131037343038\\
4.90776544764741	-0.859127960666612\\
4.90964303591872	-0.859124886671089\\
4.91152062219468	-0.859121815353316\\
4.91339820647694	-0.85911874671014\\
4.91527578876713	-0.859115680738416\\
4.91715336906689	-0.859112617435003\\
4.91903094737786	-0.859109556796763\\
4.92090852370167	-0.859106498820566\\
4.92278609803995	-0.859103443503283\\
4.92466367039433	-0.859100390841792\\
4.92654124076645	-0.859097340832975\\
4.92841880915791	-0.859094293473718\\
4.93029637557036	-0.859091248760912\\
4.93217394000541	-0.859088206691455\\
4.93405150246469	-0.859085167262245\\
4.9359290629498	-0.859082130470187\\
4.93780662146238	-0.859079096312193\\
4.93968417800404	-0.859076064785175\\
4.94156173257639	-0.859073035886052\\
4.94343928518105	-0.859070009611749\\
4.94531683581962	-0.859066985959192\\
4.94719438449371	-0.859063964925314\\
4.94907193120493	-0.859060946507053\\
4.95094947595489	-0.859057930701349\\
4.95282701874519	-0.859054917505149\\
4.95470455957743	-0.859051906915403\\
4.95658209845322	-0.859048898929067\\
4.95845963537414	-0.8590458935431\\
4.96033717034181	-0.859042890754466\\
4.9622147033578	-0.859039890560134\\
4.96409223442373	-0.859036892957076\\
4.96596976354117	-0.859033897942271\\
4.96784729071172	-0.859030905512699\\
4.96972481593696	-0.859027915665349\\
4.97160233921849	-0.85902492839721\\
4.97347986055789	-0.859021943705278\\
4.97535737995673	-0.859018961586553\\
4.97723489741661	-0.859015982038038\\
4.9791124129391	-0.859013005056743\\
4.98098992652579	-0.85901003063968\\
4.98286743817824	-0.859007058783866\\
4.98474494789803	-0.859004089486324\\
4.98662245568673	-0.85900112274408\\
4.98849996154592	-0.858998158554164\\
4.99037746547717	-0.858995196913611\\
4.99225496748204	-0.85899223781946\\
4.99413246756209	-0.858989281268756\\
4.99600996571891	-0.858986327258545\\
4.99788746195404	-0.85898337578588\\
4.99976495626905	-0.858980426847818\\
5.00164244866549	-0.85897748044142\\
5.00351993914493	-0.858974536563751\\
5.00539742770893	-0.85897159521188\\
5.00727491435903	-0.858968656382881\\
5.00915239909679	-0.858965720073833\\
5.01102988192376	-0.858962786281818\\
5.01290736284149	-0.858959855003923\\
5.01478484185153	-0.858956926237238\\
5.01666231895542	-0.858953999978859\\
5.01853979415471	-0.858951076225885\\
5.02041726745095	-0.85894815497542\\
5.02229473884567	-0.858945236224572\\
5.02417220834041	-0.858942319970453\\
5.02604967593671	-0.858939406210179\\
5.02792714163611	-0.858936494940872\\
5.02980460544014	-0.858933586159655\\
5.03168206735034	-0.858930679863659\\
5.03355952736823	-0.858927776050015\\
5.03543698549535	-0.858924874715862\\
5.03731444173323	-0.858921975858341\\
5.03919189608339	-0.858919079474598\\
5.04106934854735	-0.858916185561783\\
5.04294679912665	-0.85891329411705\\
5.04482424782279	-0.858910405137556\\
5.04670169463731	-0.858907518620465\\
5.04857913957172	-0.858904634562943\\
5.05045658262754	-0.85890175296216\\
5.05233402380627	-0.858898873815292\\
5.05421146310945	-0.858895997119516\\
5.05608890053857	-0.858893122872016\\
5.05796633609515	-0.858890251069979\\
5.0598437697807	-0.858887381710597\\
5.06172120159672	-0.858884514791063\\
5.06359863154472	-0.858881650308578\\
5.06547605962621	-0.858878788260345\\
5.06735348584268	-0.858875928643571\\
5.06923091019564	-0.858873071455468\\
5.07110833268659	-0.858870216693251\\
5.07298575331702	-0.85886736435414\\
5.07486317208843	-0.858864514435358\\
5.07674058900232	-0.858861666934133\\
5.07861800406017	-0.858858821847696\\
5.08049541726349	-0.858855979173284\\
5.08237282861375	-0.858853138908136\\
5.08425023811245	-0.858850301049494\\
5.08612764576107	-0.858847465594608\\
5.0880050515611	-0.858844632540729\\
5.08988245551402	-0.858841801885111\\
5.09175985762132	-0.858838973625016\\
5.09363725788447	-0.858836147757705\\
5.09551465630496	-0.858833324280447\\
5.09739205288425	-0.858830503190514\\
5.09926944762383	-0.858827684485179\\
5.10114684052517	-0.858824868161723\\
5.10302423158974	-0.858822054217429\\
5.10490162081901	-0.858819242649584\\
5.10677900821446	-0.858816433455479\\
5.10865639377755	-0.858813626632409\\
5.11053377750974	-0.858810822177673\\
5.1124111594125	-0.858808020088573\\
5.1142885394873	-0.858805220362417\\
5.1161659177356	-0.858802422996514\\
5.11804329415885	-0.858799627988179\\
5.11992066875851	-0.85879683533473\\
5.12179804153605	-0.85879404503349\\
5.12367541249291	-0.858791257081783\\
5.12555278163056	-0.858788471476941\\
5.12743014895044	-0.858785688216296\\
5.129307514454	-0.858782907297187\\
5.1311848781427	-0.858780128716954\\
5.13306224001798	-0.858777352472942\\
5.13493960008128	-0.858774578562501\\
5.13681695833406	-0.858771806982983\\
5.13869431477776	-0.858769037731744\\
5.14057166941381	-0.858766270806146\\
5.14244902224366	-0.858763506203551\\
5.14432637326875	-0.858760743921328\\
5.14620372249051	-0.858757983956849\\
5.14808106991038	-0.858755226307489\\
5.14995841552979	-0.858752470970627\\
5.15183575935018	-0.858749717943646\\
5.15371310137298	-0.858746967223932\\
5.15559044159961	-0.858744218808877\\
5.15746778003151	-0.858741472695874\\
5.1593451166701	-0.858738728882321\\
5.16122245151681	-0.85873598736562\\
5.16309978457306	-0.858733248143175\\
5.16497711584027	-0.858730511212397\\
5.16685444531987	-0.858727776570697\\
5.16873177301327	-0.858725044215492\\
5.17060909892189	-0.858722314144203\\
5.17248642304715	-0.858719586354252\\
5.17436374539046	-0.858716860843068\\
5.17624106595323	-0.858714137608081\\
5.17811838473688	-0.858711416646726\\
5.17999570174282	-0.858708697956442\\
5.18187301697245	-0.858705981534671\\
5.18375033042719	-0.858703267378858\\
5.18562764210843	-0.858700555486454\\
5.18750495201759	-0.85869784585491\\
5.18938226015607	-0.858695138481683\\
5.19125956652526	-0.858692433364234\\
5.19313687112657	-0.858689730500026\\
5.1950141739614	-0.858687029886528\\
5.19689147503114	-0.858684331521209\\
5.19876877433719	-0.858681635401545\\
5.20064607188095	-0.858678941525014\\
5.2025233676638	-0.858676249889098\\
5.20440066168714	-0.858673560491282\\
5.20627795395235	-0.858670873329054\\
5.20815524446083	-0.858668188399908\\
5.21003253321397	-0.85866550570134\\
5.21190982021314	-0.858662825230849\\
5.21378710545973	-0.858660146985939\\
5.21566438895513	-0.858657470964115\\
5.21754167070072	-0.858654797162889\\
5.21941895069787	-0.858652125579774\\
5.22129622894796	-0.858649456212287\\
5.22317350545238	-0.858646789057949\\
5.22505078021249	-0.858644124114285\\
5.22692805322967	-0.858641461378822\\
5.22880532450529	-0.858638800849092\\
5.23068259404073	-0.858636142522629\\
5.23255986183735	-0.858633486396972\\
5.23443712789653	-0.858630832469662\\
5.23631439221962	-0.858628180738244\\
5.238191654808	-0.858625531200268\\
5.24006891566302	-0.858622883853286\\
5.24194617478605	-0.858620238694852\\
5.24382343217846	-0.858617595722527\\
5.2457006878416	-0.858614954933872\\
5.24757794177683	-0.858612316326454\\
5.24945519398551	-0.858609679897841\\
5.25133244446898	-0.858607045645607\\
5.25320969322862	-0.858604413567328\\
5.25508694026577	-0.858601783660584\\
5.25696418558177	-0.858599155922957\\
5.25884142917799	-0.858596530352033\\
5.26071867105576	-0.858593906945404\\
5.26259591121645	-0.858591285700662\\
5.26447314966138	-0.858588666615403\\
5.26635038639191	-0.858586049687227\\
5.26822762140938	-0.858583434913739\\
5.27010485471513	-0.858580822292544\\
5.27198208631051	-0.858578211821253\\
5.27385931619684	-0.858575603497478\\
5.27573654437547	-0.858572997318838\\
5.27761377084773	-0.858570393282951\\
5.27949099561496	-0.858567791387442\\
5.2813682186785	-0.858565191629937\\
5.28324544003966	-0.858562594008066\\
5.28512265969979	-0.858559998519464\\
5.28699987766022	-0.858557405161765\\
5.28887709392226	-0.858554813932611\\
5.29075430848725	-0.858552224829645\\
5.29263152135652	-0.858549637850514\\
5.29450873253138	-0.858547052992867\\
5.29638594201316	-0.858544470254358\\
5.29826314980317	-0.858541889632643\\
5.30014035590275	-0.858539311125382\\
5.3020175603132	-0.858536734730238\\
5.30389476303585	-0.858534160444878\\
5.30577196407201	-0.858531588266971\\
5.30764916342299	-0.85852901819419\\
5.3095263610901	-0.858526450224211\\
5.31140355707466	-0.858523884354713\\
5.31328075137799	-0.858521320583379\\
5.31515794400137	-0.858518758907895\\
5.31703513494614	-0.85851619932595\\
5.31891232421358	-0.858513641835235\\
5.320789511805	-0.858511086433448\\
5.32266669772172	-0.858508533118285\\
5.32454388196502	-0.85850598188745\\
5.32642106453622	-0.858503432738647\\
5.32829824543661	-0.858500885669584\\
5.33017542466748	-0.858498340677973\\
5.33205260223014	-0.858495797761529\\
5.33392977812588	-0.858493256917969\\
5.33580695235599	-0.858490718145015\\
5.33768412492177	-0.85848818144039\\
5.33956129582451	-0.858485646801822\\
5.3414384650655	-0.858483114227041\\
5.34331563264603	-0.858480583713782\\
5.34519279856738	-0.85847805525978\\
5.34706996283084	-0.858475528862776\\
5.34894712543769	-0.858473004520513\\
5.35082428638923	-0.858470482230737\\
5.35270144568672	-0.858467961991197\\
5.35457860333146	-0.858465443799646\\
5.35645575932472	-0.858462927653839\\
5.35833291366778	-0.858460413551535\\
5.36021006636192	-0.858457901490496\\
5.3620872174084	-0.858455391468487\\
5.36396436680852	-0.858452883483275\\
5.36584151456353	-0.858450377532631\\
5.36771866067472	-0.85844787361433\\
5.36959580514335	-0.858445371726149\\
5.37147294797068	-0.858442871865868\\
5.373350089158	-0.858440374031271\\
5.37522722870656	-0.858437878220144\\
5.37710436661763	-0.858435384430276\\
5.37898150289248	-0.858432892659459\\
5.38085863753236	-0.858430402905491\\
5.38273577053854	-0.858427915166168\\
5.38461290191228	-0.858425429439293\\
5.38649003165483	-0.85842294572267\\
5.38836715976746	-0.858420464014107\\
5.39024428625142	-0.858417984311415\\
5.39212141110797	-0.858415506612408\\
5.39399853433835	-0.858413030914902\\
5.39587565594383	-0.858410557216716\\
5.39775277592565	-0.858408085515675\\
5.39962989428506	-0.858405615809603\\
5.40150701102332	-0.858403148096329\\
5.40338412614166	-0.858400682373685\\
5.40526123964133	-0.858398218639506\\
5.40713835152359	-0.858395756891629\\
5.40901546178967	-0.858393297127895\\
5.41089257044081	-0.858390839346148\\
5.41276967747825	-0.858388383544234\\
5.41464678290325	-0.858385929720003\\
5.41652388671703	-0.858383477871307\\
5.41840098892082	-0.858381027996001\\
5.42027808951588	-0.858378580091945\\
5.42215518850343	-0.858376134156999\\
5.4240322858847	-0.858373690189028\\
5.42590938166093	-0.858371248185898\\
5.42778647583335	-0.85836880814548\\
5.42966356840318	-0.858366370065648\\
5.43154065937166	-0.858363933944276\\
5.43341774874002	-0.858361499779244\\
5.43529483650947	-0.858359067568433\\
5.43717192268125	-0.858356637309728\\
5.43904900725657	-0.858354209001017\\
5.44092609023667	-0.85835178264019\\
5.44280317162275	-0.858349358225141\\
5.44468025141605	-0.858346935753765\\
5.44655732961778	-0.858344515223961\\
5.44843440622915	-0.858342096633632\\
5.45031148125138	-0.858339679980682\\
5.45218855468569	-0.858337265263019\\
5.45406562653329	-0.858334852478554\\
5.45594269679539	-0.858332441625199\\
5.45781976547321	-0.858330032700872\\
5.45969683256795	-0.85832762570349\\
5.46157389808083	-0.858325220630977\\
5.46345096201304	-0.858322817481256\\
5.4653280243658	-0.858320416252255\\
5.46720508514031	-0.858318016941905\\
5.46908214433778	-0.858315619548139\\
5.47095920195941	-0.858313224068892\\
5.47283625800639	-0.858310830502104\\
5.47471331247993	-0.858308438845716\\
5.47659036538123	-0.858306049097672\\
5.47846741671149	-0.858303661255921\\
5.4803444664719	-0.858301275318411\\
5.48222151466366	-0.858298891283095\\
5.48409856128796	-0.85829650914793\\
5.48597560634599	-0.858294128910873\\
5.48785264983895	-0.858291750569886\\
5.48972969176803	-0.858289374122931\\
5.49160673213441	-0.858286999567977\\
5.49348377093929	-0.858284626902993\\
5.49536080818385	-0.85828225612595\\
5.49723784386928	-0.858279887234823\\
5.49911487799676	-0.85827752022759\\
5.50099191056748	-0.858275155102232\\
5.50286894158261	-0.858272791856731\\
5.50474597104334	-0.858270430489074\\
5.50662299895085	-0.858268070997249\\
5.50850002530631	-0.858265713379247\\
5.51037705011091	-0.858263357633062\\
5.51225407336582	-0.858261003756691\\
5.51413109507222	-0.858258651748134\\
5.51600811523128	-0.858256301605392\\
5.51788513384416	-0.858253953326471\\
5.51976215091205	-0.858251606909378\\
5.52163916643612	-0.858249262352122\\
5.52351618041753	-0.858246919652718\\
5.52539319285745	-0.858244578809181\\
5.52727020375705	-0.858242239819528\\
5.52914721311749	-0.858239902681781\\
5.53102422093994	-0.858237567393964\\
5.53290122722556	-0.858235233954102\\
5.53477823197552	-0.858232902360225\\
5.53665523519097	-0.858230572610365\\
5.53853223687307	-0.858228244702554\\
5.54040923702299	-0.858225918634832\\
5.54228623564188	-0.858223594405236\\
5.5441632327309	-0.858221272011809\\
5.5460402282912	-0.858218951452596\\
5.54791722232393	-0.858216632725645\\
5.54979421483026	-0.858214315829004\\
5.55167120581133	-0.858212000760728\\
5.55354819526828	-0.858209687518872\\
5.55542518320229	-0.858207376101492\\
5.55730216961448	-0.85820506650665\\
5.559179154506	-0.85820275873241\\
5.56105613787801	-0.858200452776835\\
5.56293311973165	-0.858198148637996\\
5.56481010006806	-0.858195846313963\\
5.56668707888838	-0.858193545802809\\
5.56856405619376	-0.858191247102611\\
5.57044103198534	-0.858188950211446\\
5.57231800626425	-0.858186655127397\\
5.57419497903163	-0.858184361848547\\
5.57607195028862	-0.858182070372983\\
5.57794892003636	-0.858179780698793\\
5.57982588827598	-0.858177492824069\\
5.58170285500862	-0.858175206746906\\
5.5835798202354	-0.858172922465398\\
5.58545678395746	-0.858170639977647\\
5.58733374617592	-0.858168359281753\\
5.58921070689193	-0.85816608037582\\
5.5910876661066	-0.858163803257956\\
5.59296462382106	-0.85816152792627\\
5.59484158003643	-0.858159254378873\\
5.59671853475385	-0.85815698261388\\
5.59859548797444	-0.858154712629407\\
5.60047243969931	-0.858152444423575\\
5.60234938992959	-0.858150177994505\\
5.6042263386664	-0.858147913340321\\
5.60610328591086	-0.85814565045915\\
5.60798023166409	-0.858143389349121\\
5.60985717592719	-0.858141130008367\\
5.6117341187013	-0.858138872435022\\
5.61361105998752	-0.858136616627222\\
5.61548799978696	-0.858134362583107\\
5.61736493810074	-0.858132110300819\\
5.61924187492998	-0.858129859778502\\
5.62111881027577	-0.858127611014303\\
5.62299574413923	-0.85812536400637\\
5.62487267652147	-0.858123118752856\\
5.62674960742359	-0.858120875251915\\
5.62862653684671	-0.858118633501703\\
5.63050346479191	-0.858116393500379\\
5.63238039126032	-0.858114155246105\\
5.63425731625302	-0.858111918737045\\
5.63613423977113	-0.858109683971364\\
5.63801116181574	-0.858107450947233\\
5.63988808238794	-0.858105219662821\\
5.64176500148885	-0.858102990116303\\
5.64364191911956	-0.858100762305854\\
5.64551883528116	-0.858098536229654\\
5.64739574997474	-0.858096311885883\\
5.6492726632014	-0.858094089272723\\
5.65114957496224	-0.858091868388362\\
5.65302648525835	-0.858089649230987\\
5.65490339409081	-0.858087431798789\\
5.65678030146071	-0.85808521608996\\
5.65865720736915	-0.858083002102696\\
5.66053411181721	-0.858080789835194\\
5.66241101480598	-0.858078579285655\\
5.66428791633654	-0.85807637045228\\
5.66616481640998	-0.858074163333276\\
5.66804171502738	-0.858071957926848\\
5.66991861218981	-0.858069754231207\\
5.67179550789838	-0.858067552244564\\
5.67367240215414	-0.858065351965134\\
5.67554929495818	-0.858063153391132\\
5.67742618631159	-0.85806095652078\\
5.67930307621543	-0.858058761352296\\
5.68117996467078	-0.858056567883906\\
5.68305685167872	-0.858054376113835\\
5.68493373724031	-0.858052186040312\\
5.68681062135664	-0.858049997661567\\
5.68868750402878	-0.858047810975833\\
5.69056438525778	-0.858045625981346\\
5.69244126504473	-0.858043442676344\\
5.69431814339069	-0.858041261059065\\
5.69619502029673	-0.858039081127753\\
5.69807189576391	-0.858036902880652\\
5.69994876979331	-0.85803472631601\\
5.70182564238597	-0.858032551432074\\
5.70370251354298	-0.858030378227098\\
5.70557938326538	-0.858028206699334\\
5.70745625155424	-0.858026036847039\\
5.70933311841063	-0.858023868668471\\
5.71120998383559	-0.85802170216189\\
5.71308684783019	-0.858019537325561\\
5.71496371039548	-0.858017374157748\\
5.71684057153252	-0.858015212656718\\
5.71871743124236	-0.858013052820742\\
5.72059428952606	-0.858010894648091\\
5.72247114638467	-0.85800873813704\\
5.72434800181924	-0.858006583285865\\
5.72622485583083	-0.858004430092845\\
5.72810170842048	-0.858002278556262\\
5.72997855958923	-0.858000128674398\\
5.73185540933815	-0.85799798044554\\
5.73373225766827	-0.857995833867974\\
5.73560910458064	-0.857993688939992\\
5.7374859500763	-0.857991545659885\\
5.73936279415631	-0.857989404025948\\
5.74123963682169	-0.857987264036478\\
5.74311647807349	-0.857985125689773\\
5.74499331791276	-0.857982988984136\\
5.74687015634053	-0.857980853917869\\
5.74874699335783	-0.857978720489277\\
5.75062382896572	-0.85797658869667\\
5.75250066316522	-0.857974458538357\\
5.75437749595737	-0.85797233001265\\
5.7562543273432	-0.857970203117864\\
5.75813115732374	-0.857968077852315\\
5.76000798590004	-0.857965954214322\\
5.76188481307312	-0.857963832202207\\
5.76376163884401	-0.857961711814292\\
5.76563846321373	-0.857959593048903\\
5.76751528618333	-0.857957475904368\\
5.76939210775382	-0.857955360379016\\
5.77126892792623	-0.857953246471179\\
5.77314574670159	-0.857951134179191\\
5.77502256408092	-0.85794902350139\\
5.77689938006525	-0.857946914436112\\
5.77877619465559	-0.857944806981699\\
5.78065300785297	-0.857942701136494\\
5.78252981965841	-0.857940596898841\\
5.78440663007293	-0.857938494267087\\
5.78628343909754	-0.857936393239582\\
5.78816024673327	-0.857934293814677\\
5.79003705298113	-0.857932195990726\\
5.79191385784213	-0.857930099766084\\
5.79379066131729	-0.857928005139109\\
5.79566746340763	-0.85792591210816\\
5.79754426411415	-0.8579238206716\\
5.79942106343787	-0.857921730827794\\
5.8012978613798	-0.857919642575106\\
5.80317465794094	-0.857917555911906\\
5.80505145312232	-0.857915470836564\\
5.80692824692492	-0.857913387347453\\
5.80880503934976	-0.857911305442947\\
5.81068183039785	-0.857909225121423\\
5.81255862007019	-0.85790714638126\\
5.81443540836779	-0.857905069220839\\
5.81631219529164	-0.857902993638543\\
5.81818898084275	-0.857900919632758\\
5.82006576502211	-0.85789884720187\\
5.82194254783074	-0.857896776344269\\
5.82381932926963	-0.857894707058346\\
5.82569610933977	-0.857892639342495\\
5.82757288804216	-0.857890573195111\\
5.8294496653778	-0.857888508614593\\
5.83132644134768	-0.857886445599339\\
5.83320321595281	-0.857884384147751\\
5.83507998919416	-0.857882324258234\\
5.83695676107273	-0.857880265929193\\
5.83883353158952	-0.857878209159036\\
5.84071030074551	-0.857876153946174\\
5.8425870685417	-0.857874100289017\\
5.84446383497906	-0.857872048185981\\
5.8463406000586	-0.857869997635481\\
5.84821736378129	-0.857867948635935\\
5.85009412614812	-0.857865901185764\\
5.85197088716008	-0.85786385528339\\
5.85384764681814	-0.857861810927238\\
5.8557244051233	-0.857859768115732\\
5.85760116207653	-0.857857726847303\\
5.85947791767881	-0.857855687120379\\
5.86135467193112	-0.857853648933394\\
5.86323142483445	-0.857851612284782\\
5.86510817638977	-0.857849577172979\\
5.86698492659805	-0.857847543596423\\
5.86886167546028	-0.857845511553555\\
5.87073842297742	-0.857843481042817\\
5.87261516915045	-0.857841452062654\\
5.87449191398036	-0.857839424611512\\
5.87636865746809	-0.857837398687839\\
5.87824539961464	-0.857835374290085\\
5.88012214042097	-0.857833351416704\\
5.88199887988804	-0.857831330066148\\
5.88387561801683	-0.857829310236875\\
5.88575235480831	-0.857827291927343\\
5.88762909026344	-0.857825275136011\\
5.88950582438318	-0.857823259861343\\
5.89138255716851	-0.857821246101802\\
5.89325928862038	-0.857819233855854\\
5.89513601873976	-0.857817223121968\\
5.89701274752761	-0.857815213898613\\
5.89888947498489	-0.857813206184262\\
5.90076620111257	-0.857811199977388\\
5.90264292591159	-0.857809195276468\\
5.90451964938293	-0.857807192079978\\
5.90639637152753	-0.8578051903864\\
5.90827309234636	-0.857803190194214\\
5.91014981184036	-0.857801191501904\\
5.9120265300105	-0.857799194307957\\
5.91390324685773	-0.857797198610858\\
5.915779962383	-0.857795204409099\\
5.91765667658726	-0.857793211701169\\
5.91953338947147	-0.857791220485563\\
5.92141010103657	-0.857789230760776\\
5.92328681128351	-0.857787242525304\\
5.92516352021324	-0.857785255777647\\
5.92704022782672	-0.857783270516306\\
5.92891693412488	-0.857781286739784\\
5.93079363910867	-0.857779304446585\\
5.93267034277903	-0.857777323635215\\
5.93454704513692	-0.857775344304185\\
5.93642374618326	-0.857773366452003\\
5.93830044591902	-0.857771390077182\\
5.94017714434511	-0.857769415178237\\
5.9420538414625	-0.857767441753684\\
5.94393053727211	-0.85776546980204\\
5.94580723177488	-0.857763499321825\\
5.94768392497176	-0.857761530311562\\
5.94956061686367	-0.857759562769773\\
5.95143730745156	-0.857757596694985\\
5.95331399673636	-0.857755632085724\\
5.95519068471899	-0.85775366894052\\
5.95706737140041	-0.857751707257904\\
5.95894405678153	-0.857749747036409\\
5.96082074086329	-0.85774778827457\\
5.96269742364663	-0.857745830970923\\
5.96457410513246	-0.857743875124007\\
5.96645078532172	-0.857741920732362\\
5.96832746421533	-0.85773996779453\\
5.97020414181423	-0.857738016309056\\
5.97208081811934	-0.857736066274486\\
5.97395749313158	-0.857734117689366\\
5.97583416685188	-0.857732170552246\\
5.97771083928116	-0.857730224861679\\
5.97958751042035	-0.857728280616217\\
5.98146418027036	-0.857726337814415\\
5.98334084883212	-0.857724396454831\\
5.98521751610655	-0.857722456536022\\
5.98709418209456	-0.857720518056549\\
5.98897084679707	-0.857718581014975\\
5.99084751021501	-0.857716645409863\\
5.99272417234929	-0.857714711239781\\
5.99460083320081	-0.857712778503294\\
5.99647749277051	-0.857710847198974\\
5.99835415105929	-0.857708917325391\\
6.00023080806806	-0.857706988881118\\
6.00210746379774	-0.857705061864731\\
6.00398411824924	-0.857703136274805\\
6.00586077142347	-0.857701212109921\\
6.00773742332133	-0.857699289368657\\
6.00961407394374	-0.857697368049596\\
6.01149072329161	-0.857695448151322\\
6.01336737136584	-0.857693529672421\\
6.01524401816733	-0.85769161261148\\
6.017120663697	-0.857689696967088\\
6.01899730795575	-0.857687782737836\\
6.02087395094448	-0.857685869922317\\
6.02275059266409	-0.857683958519126\\
6.02462723311548	-0.857682048526858\\
6.02650387229957	-0.857680139944112\\
6.02838051021723	-0.857678232769488\\
6.03025714686939	-0.857676327001586\\
6.03213378225693	-0.857674422639011\\
6.03401041638075	-0.857672519680368\\
6.03588704924175	-0.857670618124262\\
6.03776368084082	-0.857668717969304\\
6.03964031117887	-0.857666819214102\\
6.04151694025677	-0.85766492185727\\
6.04339356807544	-0.85766302589742\\
6.04527019463575	-0.857661131333169\\
6.04714681993861	-0.857659238163133\\
6.0490234439849	-0.857657346385932\\
6.05090006677551	-0.857655456000186\\
6.05277668831133	-0.857653567004517\\
6.05465330859325	-0.85765167939755\\
6.05652992762216	-0.857649793177911\\
6.05840654539895	-0.857647908344227\\
6.06028316192449	-0.857646024895127\\
6.06215977719968	-0.857644142829242\\
6.0640363912254	-0.857642262145206\\
6.06591300400253	-0.857640382841651\\
6.06778961553195	-0.857638504917216\\
6.06966622581456	-0.857636628370536\\
6.07154283485122	-0.857634753200252\\
6.07341944264282	-0.857632879405006\\
6.07529604919023	-0.857631006983439\\
6.07717265449434	-0.857629135934196\\
6.07904925855602	-0.857627266255924\\
6.08092586137616	-0.857625397947271\\
6.08280246295562	-0.857623531006886\\
6.08467906329528	-0.85762166543342\\
6.08655566239601	-0.857619801225527\\
6.0884322602587	-0.857617938381861\\
6.0903088568842	-0.857616076901079\\
6.0921854522734	-0.857614216781838\\
6.09406204642716	-0.857612358022798\\
6.09593863934636	-0.85761050062262\\
6.09781523103186	-0.857608644579968\\
6.09969182148453	-0.857606789893506\\
6.10156841070524	-0.857604936561901\\
6.10344499869486	-0.857603084583819\\
6.10532158545426	-0.857601233957932\\
6.10719817098429	-0.85759938468291\\
6.10907475528582	-0.857597536757426\\
6.11095133835973	-0.857595690180154\\
6.11282792020686	-0.857593844949772\\
6.11470450082808	-0.857592001064956\\
6.11658108022426	-0.857590158524386\\
6.11845765839625	-0.857588317326744\\
6.12033423534492	-0.857586477470711\\
6.12221081107112	-0.857584638954973\\
6.1240873855757	-0.857582801778216\\
6.12596395885954	-0.857580965939126\\
6.12784053092348	-0.857579131436393\\
6.12971710176838	-0.857577298268709\\
6.1315936713951	-0.857575466434765\\
6.13347023980449	-0.857573635933255\\
6.1353468069974	-0.857571806762876\\
6.13722337297468	-0.857569978922325\\
6.13909993773719	-0.8575681524103\\
6.14097650128578	-0.857566327225503\\
6.1428530636213	-0.857564503366635\\
6.14472962474459	-0.857562680832399\\
6.14660618465651	-0.857560859621502\\
6.14848274335791	-0.857559039732651\\
6.15035930084962	-0.857557221164553\\
6.1522358571325	-0.857555403915919\\
6.1541124122074	-0.85755358798546\\
6.15598896607515	-0.857551773371891\\
6.15786551873661	-0.857549960073925\\
6.1597420701926	-0.857548148090279\\
6.16161862044399	-0.857546337419672\\
6.1634951694916	-0.857544528060823\\
6.16537171733628	-0.857542720012452\\
6.16724826397887	-0.857540913273284\\
6.16912480942021	-0.857539107842042\\
6.17100135366113	-0.857537303717451\\
6.17287789670248	-0.85753550089824\\
6.17475443854509	-0.857533699383138\\
6.1766309791898	-0.857531899170875\\
6.17850751863744	-0.857530100260183\\
6.18038405688884	-0.857528302649796\\
6.18226059394485	-0.85752650633845\\
6.18413712980629	-0.85752471132488\\
6.18601366447399	-0.857522917607826\\
6.1878901979488	-0.857521125186027\\
6.18976673023153	-0.857519334058225\\
6.19164326132301	-0.857517544223162\\
6.19351979122409	-0.857515755679584\\
6.19539631993558	-0.857513968426236\\
6.19727284745832	-0.857512182461866\\
6.19914937379312	-0.857510397785223\\
6.20102589894082	-0.857508614395058\\
6.20290242290224	-0.857506832290123\\
6.20477894567821	-0.857505051469171\\
6.20665546726955	-0.857503271930959\\
6.20853198767708	-0.857501493674242\\
6.21040850690162	-0.857499716697779\\
6.212285024944	-0.85749794100033\\
6.21416154180504	-0.857496166580655\\
6.21603805748556	-0.857494393437519\\
6.21791457198637	-0.857492621569685\\
6.2197910853083	-0.857490850975919\\
6.22166759745216	-0.857489081654989\\
6.22354410841877	-0.857487313605663\\
6.22542061820895	-0.857485546826712\\
6.22729712682351	-0.857483781316907\\
6.22917363426326	-0.857482017075022\\
6.23105014052903	-0.857480254099832\\
6.23292664562162	-0.857478492390113\\
6.23480314954184	-0.857476731944643\\
6.23667965229052	-0.857474972762202\\
6.23855615386845	-0.857473214841569\\
6.24043265427645	-0.857471458181528\\
6.24230915351533	-0.857469702780862\\
6.2441856515859	-0.857467948638357\\
6.24606214848896	-0.857466195752798\\
6.24793864422533	-0.857464444122975\\
6.24981513879581	-0.857462693747678\\
6.2516916322012	-0.857460944625696\\
6.25356812444231	-0.857459196755823\\
6.25544461551995	-0.857457450136854\\
6.25732110543491	-0.857455704767583\\
6.259197594188	-0.857453960646807\\
6.26107408178003	-0.857452217773325\\
6.2629505682118	-0.857450476145938\\
6.26482705348409	-0.857448735763446\\
6.26670353759773	-0.857446996624652\\
6.2685800205535	-0.85744525872836\\
6.2704565023522	-0.857443522073377\\
6.27233298299463	-0.857441786658509\\
6.27420946248159	-0.857440052482565\\
6.27608594081387	-0.857438319544356\\
6.27796241799227	-0.857436587842692\\
6.27983889401759	-0.857434857376387\\
6.28171536889061	-0.857433128144255\\
6.28359184261214	-0.857431400145112\\
6.28546831518296	-0.857429673377775\\
6.28734478660386	-0.857427947841063\\
6.28922125687565	-0.857426223533796\\
6.2910977259991	-0.857424500454795\\
6.29297419397501	-0.857422778602884\\
6.29485066080416	-0.857421057976887\\
6.29672712648735	-0.85741933857563\\
6.29860359102537	-0.85741762039794\\
6.30048005441899	-0.857415903442645\\
6.302356516669	-0.857414187708577\\
6.3042329777762	-0.857412473194565\\
6.30610943774137	-0.857410759899444\\
6.30798589656528	-0.857409047822047\\
6.30986235424873	-0.85740733696121\\
6.31173881079249	-0.85740562731577\\
6.31361526619735	-0.857403918884566\\
6.31549172046408	-0.857402211666438\\
6.31736817359348	-0.857400505660227\\
6.31924462558631	-0.857398800864775\\
6.32112107644336	-0.857397097278927\\
6.32299752616541	-0.857395394901529\\
6.32487397475323	-0.857393693731426\\
6.3267504222076	-0.857391993767468\\
6.3286268685293	-0.857390295008504\\
6.3305033137191	-0.857388597453385\\
6.33237975777778	-0.857386901100964\\
6.3342562007061	-0.857385205950093\\
6.33613264250485	-0.85738351199963\\
6.3380090831748	-0.857381819248429\\
6.33988552271672	-0.857380127695349\\
6.34176196113138	-0.857378437339249\\
6.34363839841954	-0.857376748178991\\
6.34551483458199	-0.857375060213435\\
6.34739126961949	-0.857373373441445\\
6.3492677035328	-0.857371687861887\\
6.3511441363227	-0.857370003473625\\
6.35302056798996	-0.857368320275528\\
6.35489699853533	-0.857366638266465\\
6.35677342795959	-0.857364957445305\\
6.3586498562635	-0.857363277810921\\
6.36052628344782	-0.857361599362184\\
6.36240270951332	-0.857359922097971\\
6.36427913446076	-0.857358246017154\\
6.3661555582909	-0.857356571118613\\
6.36803198100451	-0.857354897401225\\
6.36990840260234	-0.85735322486387\\
6.37178482308516	-0.857351553505428\\
6.37366124245372	-0.857349883324783\\
6.37553766070879	-0.857348214320817\\
6.37741407785111	-0.857346546492416\\
6.37929049388146	-0.857344879838466\\
6.38116690880058	-0.857343214357854\\
6.38304332260924	-0.85734155004947\\
6.38491973530818	-0.857339886912203\\
6.38679614689816	-0.857338224944946\\
6.38867255737994	-0.857336564146592\\
6.39054896675427	-0.857334904516034\\
6.3924253750219	-0.857333246052168\\
6.39430178218358	-0.857331588753892\\
6.39617818824007	-0.857329932620102\\
6.39805459319212	-0.8573282776497\\
6.39993099704047	-0.857326623841585\\
6.40180739978587	-0.85732497119466\\
6.40368380142908	-0.857323319707828\\
6.40556020197084	-0.857321669379994\\
6.40743660141189	-0.857320020210064\\
6.40931299975299	-0.857318372196946\\
6.41118939699489	-0.857316725339547\\
6.41306579313831	-0.857315079636779\\
6.41494218818402	-0.857313435087552\\
6.41681858213275	-0.857311791690778\\
6.41869497498525	-0.857310149445373\\
6.42057136674226	-0.85730850835025\\
6.42244775740453	-0.857306868404326\\
6.42432414697278	-0.857305229606519\\
6.42620053544777	-0.857303591955747\\
6.42807692283023	-0.857301955450932\\
6.4299533091209	-0.857300320090995\\
6.43182969432053	-0.857298685874858\\
6.43370607842984	-0.857297052801446\\
6.43558246144958	-0.857295420869684\\
6.43745884338048	-0.857293790078498\\
6.43933522422329	-0.857292160426817\\
6.44121160397872	-0.85729053191357\\
6.44308798264752	-0.857288904537687\\
6.44496436023043	-0.8572872782981\\
6.44684073672817	-0.857285653193742\\
6.44871711214147	-0.857284029223547\\
6.45059348647108	-0.857282406386451\\
6.45246985971772	-0.857280784681391\\
6.45434623188212	-0.857279164107304\\
6.45622260296501	-0.85727754466313\\
6.45809897296712	-0.857275926347809\\
6.45997534188919	-0.857274309160284\\
6.46185170973193	-0.857272693099497\\
6.46372807649607	-0.857271078164393\\
6.46560444218235	-0.857269464353916\\
6.46748080679148	-0.857267851667015\\
6.4693571703242	-0.857266240102637\\
6.47123353278122	-0.857264629659731\\
6.47310989416328	-0.857263020337248\\
6.47498625447109	-0.857261412134139\\
6.47686261370538	-0.857259805049358\\
6.47873897186687	-0.857258199081858\\
6.48061532895628	-0.857256594230596\\
6.48249168497433	-0.857254990494528\\
6.48436803992175	-0.857253387872611\\
6.48624439379925	-0.857251786363806\\
6.48812074660755	-0.857250185967071\\
6.48999709834737	-0.85724858668137\\
6.49187344901942	-0.857246988505665\\
6.49374979862443	-0.857245391438919\\
6.49562614716311	-0.857243795480098\\
6.49750249463618	-0.857242200628169\\
6.49937884104435	-0.857240606882099\\
6.50125518638834	-0.857239014240857\\
6.50313153066885	-0.857237422703413\\
6.50500787388661	-0.857235832268739\\
6.50688421604232	-0.857234242935807\\
6.5087605571367	-0.85723265470359\\
6.51063689717046	-0.857231067571065\\
6.5125132361443	-0.857229481537206\\
6.51438957405895	-0.857227896600992\\
6.5162659109151	-0.8572263127614\\
6.51814224671347	-0.857224730017411\\
6.52001858145477	-0.857223148368006\\
6.5218949151397	-0.857221567812166\\
6.52377124776897	-0.857219988348875\\
6.52564757934329	-0.857218409977117\\
6.52752390986336	-0.857216832695879\\
6.52940023932988	-0.857215256504147\\
6.53127656774356	-0.857213681400909\\
6.53315289510511	-0.857212107385154\\
6.53502922141523	-0.857210534455873\\
6.53690554667462	-0.857208962612058\\
6.53878187088398	-0.857207391852702\\
6.54065819404401	-0.857205822176797\\
6.54253451615542	-0.85720425358334\\
6.5444108372189	-0.857202686071327\\
6.54628715723515	-0.857201119639755\\
6.54816347620487	-0.857199554287623\\
6.55003979412877	-0.857197990013931\\
6.55191611100753	-0.85719642681768\\
6.55379242684186	-0.857194864697873\\
6.55566874163245	-0.857193303653511\\
6.55754505537999	-0.857191743683601\\
6.55942136808518	-0.857190184787147\\
6.56129767974873	-0.857188626963157\\
6.56317399037131	-0.857187070210638\\
6.56505029995362	-0.8571855145286\\
6.56692660849636	-0.857183959916052\\
6.56880291600022	-0.857182406372007\\
6.57067922246588	-0.857180853895478\\
6.57255552789404	-0.857179302485477\\
6.5744318322854	-0.857177752141019\\
6.57630813564063	-0.857176202861122\\
6.57818443796043	-0.857174654644801\\
6.58006073924549	-0.857173107491076\\
6.58193703949649	-0.857171561398965\\
6.58381333871412	-0.85717001636749\\
6.58568963689907	-0.857168472395672\\
6.58756593405202	-0.857166929482534\\
6.58944223017367	-0.8571653876271\\
6.59131852526468	-0.857163846828395\\
6.59319481932575	-0.857162307085446\\
6.59507111235757	-0.85716076839728\\
6.5969474043608	-0.857159230762926\\
6.59882369533615	-0.857157694181413\\
6.60069998528428	-0.857156158651771\\
6.60257627420588	-0.857154624173034\\
6.60445256210162	-0.857153090744234\\
6.6063288489722	-0.857151558364405\\
6.60820513481829	-0.857150027032583\\
6.61008141964056	-0.857148496747804\\
6.6119577034397	-0.857146967509105\\
6.61383398621638	-0.857145439315526\\
6.61571026797128	-0.857143912166105\\
6.61758654870507	-0.857142386059885\\
6.61946282841844	-0.857140860995906\\
6.62133910711205	-0.857139336973212\\
6.62321538478658	-0.857137813990847\\
6.62509166144271	-0.857136292047856\\
6.6269679370811	-0.857134771143287\\
6.62884421170244	-0.857133251276185\\
6.63072048530739	-0.857131732445601\\
6.63259675789662	-0.857130214650583\\
6.63447302947081	-0.857128697890183\\
6.63634930003062	-0.857127182163453\\
6.63822556957673	-0.857125667469444\\
6.6401018381098	-0.857124153807213\\
6.64197810563051	-0.857122641175812\\
6.64385437213952	-0.8571211295743\\
6.6457306376375	-0.857119619001734\\
6.64760690212511	-0.857118109457171\\
6.64948316560302	-0.857116600939671\\
6.6513594280719	-0.857115093448296\\
6.65323568953241	-0.857113586982106\\
6.65511194998522	-0.857112081540164\\
6.656988209431	-0.857110577121535\\
6.65886446787039	-0.857109073725283\\
6.66074072530407	-0.857107571350475\\
6.6626169817327	-0.857106069996177\\
6.66449323715694	-0.857104569661457\\
6.66636949157746	-0.857103070345385\\
6.6682457449949	-0.857101572047032\\
6.67012199740994	-0.857100074765468\\
6.67199824882322	-0.857098578499765\\
6.67387449923542	-0.857097083248999\\
6.67575074864719	-0.857095589012242\\
6.67762699705918	-0.857094095788571\\
6.67950324447205	-0.857092603577063\\
6.68137949088647	-0.857091112376795\\
6.68325573630308	-0.857089622186846\\
6.68513198072253	-0.857088133006295\\
6.6870082241455	-0.857086644834225\\
6.68888446657262	-0.857085157669716\\
6.69076070800455	-0.857083671511852\\
6.69263694844195	-0.857082186359718\\
6.69451318788547	-0.857080702212397\\
6.69638942633575	-0.857079219068977\\
6.69826566379346	-0.857077736928544\\
6.70014190025924	-0.857076255790187\\
6.70201813573374	-0.857074775652995\\
6.70389437021761	-0.857073296516059\\
6.70577060371151	-0.85707181837847\\
6.70764683621607	-0.85707034123932\\
6.70952306773194	-0.857068865097703\\
6.71139929825979	-0.857067389952714\\
6.71327552780024	-0.857065915803447\\
6.71515175635395	-0.857064442649\\
6.71702798392156	-0.857062970488471\\
6.71890421050372	-0.857061499320957\\
6.72078043610107	-0.857060029145559\\
6.72265666071426	-0.857058559961378\\
6.72453288434392	-0.857057091767514\\
6.72640910699071	-0.857055624563072\\
6.72828532865526	-0.857054158347154\\
6.73016154933822	-0.857052693118865\\
6.73203776904023	-0.857051228877312\\
6.73391398776192	-0.857049765621601\\
6.73579020550395	-0.85704830335084\\
6.73766642226694	-0.857046842064138\\
6.73954263805153	-0.857045381760605\\
6.74141885285837	-0.857043922439352\\
6.7432950666881	-0.857042464099491\\
6.74517127954134	-0.857041006740134\\
6.74704749141874	-0.857039550360396\\
6.74892370232093	-0.857038094959392\\
6.75079991224855	-0.857036640536237\\
6.75267612120223	-0.857035187090049\\
6.75455232918261	-0.857033734619945\\
6.75642853619033	-0.857032283125046\\
6.758304742226	-0.857030832604469\\
6.76018094729028	-0.857029383057338\\
6.76205715138379	-0.857027934482773\\
6.76393335450715	-0.857026486879897\\
6.76580955666102	-0.857025040247835\\
6.767685757846	-0.857023594585712\\
6.76956195806274	-0.857022149892653\\
6.77143815731186	-0.857020706167786\\
6.773314355594	-0.857019263410238\\
6.77519055290978	-0.857017821619139\\
6.77706674925982	-0.857016380793617\\
6.77894294464477	-0.857014940932806\\
6.78081913906523	-0.857013502035835\\
6.78269533252185	-0.857012064101839\\
6.78457152501524	-0.85701062712995\\
6.78644771654603	-0.857009191119304\\
6.78832390711485	-0.857007756069037\\
6.79020009672232	-0.857006321978285\\
6.79207628536906	-0.857004888846187\\
6.7939524730557	-0.85700345667188\\
6.79582865978285	-0.857002025454506\\
6.79770484555115	-0.857000595193204\\
6.79958103036121	-0.856999165887116\\
6.80145721421365	-0.856997737535385\\
6.8033333971091	-0.856996310137154\\
6.80520957904817	-0.856994883691569\\
6.80708576003148	-0.856993458197774\\
6.80896194005966	-0.856992033654916\\
6.81083811913331	-0.856990610062143\\
6.81271429725307	-0.856989187418603\\
6.81459047441954	-0.856987765723445\\
6.81646665063334	-0.856986344975821\\
6.81834282589508	-0.856984925174881\\
6.82021900020539	-0.856983506319777\\
6.82209517356488	-0.856982088409663\\
6.82397134597417	-0.856980671443693\\
6.82584751743386	-0.856979255421022\\
6.82772368794457	-0.856977840340806\\
6.82959985750691	-0.856976426202203\\
6.8314760261215	-0.85697501300437\\
6.83335219378895	-0.856973600746467\\
6.83522836050987	-0.856972189427652\\
6.83710452628487	-0.856970779047088\\
6.83898069111456	-0.856969369603936\\
6.84085685499954	-0.856967961097358\\
6.84273301794044	-0.856966553526518\\
6.84460917993786	-0.856965146890581\\
6.84648534099241	-0.856963741188713\\
6.84836150110469	-0.85696233642008\\
6.85023766027531	-0.856960932583849\\
6.85211381850488	-0.856959529679189\\
6.853989975794	-0.856958127705269\\
6.85586613214329	-0.856956726661259\\
6.85774228755334	-0.856955326546331\\
6.85961844202476	-0.856953927359656\\
6.86149459555815	-0.856952529100408\\
6.86337074815412	-0.856951131767761\\
6.86524689981328	-0.856949735360889\\
6.86712305053621	-0.856948339878968\\
6.86899920032354	-0.856946945321175\\
6.87087534917584	-0.856945551686688\\
6.87275149709374	-0.856944158974686\\
6.87462764407783	-0.856942767184347\\
6.8765037901287	-0.856941376314852\\
6.87837993524697	-0.856939986365383\\
6.88025607943321	-0.856938597335121\\
6.88213222268805	-0.856937209223251\\
6.88400836501207	-0.856935822028955\\
6.88588450640587	-0.85693443575142\\
6.88776064687004	-0.85693305038983\\
6.88963678640519	-0.856931665943374\\
6.89151292501192	-0.856930282411237\\
6.8933890626908	-0.85692889979261\\
6.89526519944245	-0.856927518086682\\
6.89714133526745	-0.856926137292642\\
6.8990174701664	-0.856924757409683\\
6.90089360413989	-0.856923378436997\\
6.90276973718852	-0.856922000373777\\
6.90464586931287	-0.856920623219217\\
6.90652200051354	-0.856919246972511\\
6.90839813079113	-0.856917871632857\\
6.91027426014621	-0.85691649719945\\
6.91215038857939	-0.856915123671488\\
6.91402651609124	-0.856913751048169\\
6.91590264268237	-0.856912379328694\\
6.91777876835336	-0.856911008512263\\
6.91965489310479	-0.856909638598076\\
6.92153101693726	-0.856908269585336\\
6.92340713985136	-0.856906901473246\\
6.92528326184766	-0.856905534261009\\
6.92715938292676	-0.856904167947831\\
6.92903550308924	-0.856902802532917\\
6.93091162233569	-0.856901438015474\\
6.93278774066669	-0.856900074394709\\
6.93466385808283	-0.85689871166983\\
6.93653997458469	-0.856897349840047\\
6.93841609017285	-0.85689598890457\\
6.9402922048479	-0.856894628862609\\
6.94216831861042	-0.856893269713376\\
6.94404443146099	-0.856891911456085\\
6.94592054340019	-0.856890554089949\\
6.9477966544286	-0.856889197614181\\
6.94967276454681	-0.856887842027998\\
6.95154887375539	-0.856886487330616\\
6.95342498205493	-0.856885133521252\\
6.95530108944599	-0.856883780599123\\
6.95717719592917	-0.856882428563448\\
6.95905330150503	-0.856881077413448\\
6.96092940617416	-0.856879727148342\\
6.96280550993713	-0.856878377767353\\
6.96468161279453	-0.856877029269701\\
6.96655771474691	-0.856875681654611\\
6.96843381579487	-0.856874334921306\\
6.97030991593897	-0.856872989069012\\
6.97218601517979	-0.856871644096953\\
6.97406211351791	-0.856870300004357\\
6.97593821095389	-0.856868956790451\\
6.97781430748832	-0.856867614454462\\
6.97969040312176	-0.856866272995621\\
6.98156649785478	-0.856864932413157\\
6.98344259168796	-0.856863592706301\\
6.98531868462187	-0.856862253874285\\
6.98719477665708	-0.856860915916341\\
6.98907086779415	-0.856859578831702\\
6.99094695803367	-0.856858242619603\\
6.99282304737619	-0.856856907279279\\
6.99469913582229	-0.856855572809965\\
6.99657522337254	-0.856854239210899\\
6.99845131002749	-0.856852906481318\\
7.00032739578773	-0.856851574620461\\
7.00220348065381	-0.856850243627566\\
7.00407956462631	-0.856848913501874\\
7.00595564770579	-0.856847584242625\\
7.00783172989281	-0.856846255849063\\
7.00970781118794	-0.856844928320428\\
7.01158389159174	-0.856843601655965\\
7.01345997110478	-0.856842275854918\\
7.01533604972762	-0.856840950916532\\
7.01721212746083	-0.856839626840052\\
7.01908820430496	-0.856838303624727\\
7.02096428026058	-0.856836981269803\\
7.02284035532825	-0.856835659774529\\
7.02471642950854	-0.856834339138153\\
7.02659250280199	-0.856833019359927\\
7.02846857520918	-0.856831700439101\\
7.03034464673066	-0.856830382374926\\
7.03222071736699	-0.856829065166656\\
7.03409678711873	-0.856827748813544\\
7.03597285598644	-0.856826433314843\\
7.03784892397068	-0.856825118669809\\
7.03972499107199	-0.856823804877698\\
7.04160105729095	-0.856822491937767\\
7.04347712262811	-0.856821179849272\\
7.04535318708402	-0.856819868611472\\
7.04722925065923	-0.856818558223627\\
7.04910531335431	-0.856817248684995\\
7.05098137516981	-0.856815939994839\\
7.05285743610628	-0.856814632152419\\
7.05473349616428	-0.856813325156998\\
7.05660955534435	-0.856812019007838\\
7.05848561364706	-0.856810713704204\\
7.06036167107295	-0.856809409245361\\
7.06223772762257	-0.856808105630574\\
7.06411378329648	-0.85680680285911\\
7.06598983809523	-0.856805500930235\\
7.06786589201937	-0.856804199843218\\
7.06974194506944	-0.856802899597327\\
7.071617997246	-0.856801600191832\\
7.0734940485496	-0.856800301626004\\
7.07537009898079	-0.856799003899113\\
7.0772461485401	-0.856797707010432\\
7.0791221972281	-0.856796410959233\\
7.08099824504533	-0.85679511574479\\
7.08287429199233	-0.856793821366377\\
7.08475033806965	-0.856792527823269\\
7.08662638327783	-0.856791235114742\\
7.08850242761743	-0.856789943240074\\
7.09037847108899	-0.856788652198541\\
7.09225451369305	-0.856787361989422\\
7.09413055543015	-0.856786072611995\\
7.09600659630085	-0.856784784065542\\
7.09788263630567	-0.856783496349342\\
7.09975867544517	-0.856782209462676\\
7.10163471371988	-0.856780923404828\\
7.10351075113036	-0.85677963817508\\
7.10538678767713	-0.856778353772716\\
7.10726282336075	-0.85677707019702\\
7.10913885818175	-0.856775787447278\\
7.11101489214066	-0.856774505522776\\
7.11289092523804	-0.8567732244228\\
7.11476695747442	-0.85677194414664\\
7.11664298885033	-0.856770664693582\\
7.11851901936633	-0.856769386062916\\
7.12039504902293	-0.856768108253933\\
7.12227107782069	-0.856766831265922\\
7.12414710576013	-0.856765555098177\\
7.1260231328418	-0.856764279749987\\
7.12789915906623	-0.856763005220648\\
7.12977518443395	-0.856761731509453\\
7.13165120894551	-0.856760458615695\\
7.13352723260143	-0.856759186538672\\
7.13540325540225	-0.856757915277678\\
7.1372792773485	-0.856756644832012\\
7.13915529844072	-0.856755375200969\\
7.14103131867944	-0.85675410638385\\
7.14290733806519	-0.856752838379953\\
7.1447833565985	-0.856751571188577\\
7.1466593742799	-0.856750304809025\\
7.14853539110994	-0.856749039240597\\
7.15041140708913	-0.856747774482595\\
7.152287422218	-0.856746510534322\\
7.15416343649709	-0.856745247395083\\
7.15603944992693	-0.856743985064182\\
7.15791546250804	-0.856742723540923\\
7.15979147424096	-0.856741462824613\\
7.1616674851262	-0.856740202914559\\
7.1635434951643	-0.856738943810068\\
7.16541950435579	-0.856737685510449\\
7.16729551270119	-0.85673642801501\\
7.16917152020103	-0.856735171323061\\
7.17104752685583	-0.856733915433912\\
7.17292353266612	-0.856732660346876\\
7.17479953763242	-0.856731406061263\\
7.17667554175527	-0.856730152576387\\
7.17855154503517	-0.856728899891561\\
7.18042754747266	-0.856727648006099\\
7.18230354906826	-0.856726396919316\\
7.1841795498225	-0.856725146630528\\
7.18605554973589	-0.856723897139051\\
7.18793154880895	-0.856722648444203\\
7.18980754704221	-0.8567214005453\\
7.19168354443619	-0.856720153441663\\
7.19355954099141	-0.85671890713261\\
7.19543553670839	-0.856717661617462\\
7.19731153158765	-0.856716416895538\\
7.1991875256297	-0.856715172966161\\
7.20106351883507	-0.856713929828654\\
7.20293951120428	-0.856712687482338\\
7.20481550273785	-0.856711445926538\\
7.20669149343628	-0.856710205160579\\
7.2085674833001	-0.856708965183785\\
7.21044347232983	-0.856707725995482\\
7.21231946052598	-0.856706487594997\\
7.21419544788906	-0.856705249981658\\
7.2160714344196	-0.856704013154792\\
7.21794742011811	-0.856702777113729\\
7.2198234049851	-0.856701541857797\\
7.22169938902109	-0.856700307386328\\
7.22357537222658	-0.856699073698652\\
7.2254513546021	-0.8566978407941\\
7.22732733614816	-0.856696608672006\\
7.22920331686526	-0.856695377331702\\
7.23107929675393	-0.856694146772523\\
7.23295527581467	-0.856692916993802\\
7.234831254048	-0.856691687994875\\
7.23670723145441	-0.856690459775078\\
7.23858320803444	-0.856689232333748\\
7.24045918378858	-0.856688005670222\\
7.24233515871734	-0.856686779783839\\
7.24421113282124	-0.856685554673936\\
7.24608710610078	-0.856684330339855\\
7.24796307855647	-0.856683106780934\\
7.24983905018882	-0.856681883996515\\
7.25171502099834	-0.856680661985939\\
7.25359099098553	-0.856679440748549\\
7.2554669601509	-0.856678220283688\\
7.25734292849495	-0.8566770005907\\
7.2592188960182	-0.856675781668929\\
7.26109486272115	-0.856674563517721\\
7.2629708286043	-0.85667334613642\\
7.26484679366815	-0.856672129524375\\
7.26672275791322	-0.856670913680931\\
7.26859872134	-0.856669698605437\\
7.270474683949	-0.856668484297242\\
7.27235064574072	-0.856667270755695\\
7.27422660671566	-0.856666057980147\\
7.27610256687433	-0.856664845969946\\
7.27797852621722	-0.856663634724447\\
7.27985448474485	-0.856662424242999\\
7.2817304424577	-0.856661214524957\\
7.28360639935629	-0.856660005569673\\
7.2854823554411	-0.856658797376502\\
7.28735831071264	-0.856657589944799\\
7.28923426517142	-0.85665638327392\\
7.29111021881792	-0.85665517736322\\
7.29298617165264	-0.856653972212056\\
7.29486212367609	-0.856652767819787\\
7.29673807488876	-0.85665156418577\\
7.29861402529115	-0.856650361309365\\
7.30048997488376	-0.856649159189931\\
7.30236592366708	-0.856647957826828\\
7.3042418716416	-0.856646757219418\\
7.30611781880783	-0.856645557367063\\
7.30799376516626	-0.856644358269124\\
7.30986971071738	-0.856643159924965\\
7.31174565546169	-0.856641962333949\\
7.31362159939968	-0.856640765495441\\
7.31549754253185	-0.856639569408806\\
7.31737348485868	-0.85663837407341\\
7.31924942638068	-0.85663717948862\\
7.32112536709832	-0.856635985653802\\
7.32300130701212	-0.856634792568324\\
7.32487724612255	-0.856633600231555\\
7.32675318443011	-0.856632408642863\\
7.32862912193529	-0.856631217801619\\
7.33050505863858	-0.856630027707193\\
7.33238099454046	-0.856628838358957\\
7.33425692964144	-0.856627649756281\\
7.33613286394199	-0.856626461898539\\
7.33800879744261	-0.856625274785104\\
7.33988473014379	-0.856624088415349\\
7.341760662046	-0.856622902788649\\
7.34363659314975	-0.856621717904379\\
7.34551252345552	-0.856620533761915\\
7.34738845296379	-0.856619350360633\\
7.34926438167505	-0.856618167699911\\
7.35114030958979	-0.856616985779125\\
7.35301623670849	-0.856615804597656\\
7.35489216303164	-0.85661462415488\\
7.35676808855972	-0.85661344445018\\
7.35864401329322	-0.856612265482934\\
7.36051993723261	-0.856611087252524\\
7.3623958603784	-0.856609909758331\\
7.36427178273105	-0.856608732999738\\
7.36614770429105	-0.856607556976129\\
7.36802362505888	-0.856606381686886\\
7.36989954503503	-0.856605207131393\\
7.37177546421998	-0.856604033309037\\
7.3736513826142	-0.856602860219202\\
7.37552730021818	-0.856601687861274\\
7.3774032170324	-0.856600516234642\\
7.37927913305735	-0.856599345338691\\
7.38115504829349	-0.856598175172811\\
7.38303096274131	-0.85659700573639\\
7.38490687640128	-0.856595837028817\\
7.38678278927389	-0.856594669049484\\
7.38865870135962	-0.856593501797779\\
7.39053461265894	-0.856592335273096\\
7.39241052317233	-0.856591169474826\\
7.39428643290026	-0.856590004402362\\
7.39616234184322	-0.856588840055097\\
7.39803825000168	-0.856587676432424\\
7.39991415737611	-0.85658651353374\\
7.40179006396699	-0.856585351358438\\
7.4036659697748	-0.856584189905915\\
7.40554187480001	-0.856583029175567\\
7.40741777904309	-0.856581869166791\\
7.40929368250452	-0.856580709878986\\
7.41116958518478	-0.85657955131155\\
7.41304548708433	-0.85657839346388\\
7.41492138820364	-0.856577236335379\\
7.4167972885432	-0.856576079925444\\
7.41867318810347	-0.856574924233479\\
7.42054908688493	-0.856573769258883\\
7.42242498488804	-0.85657261500106\\
7.42430088211328	-0.856571461459412\\
7.42617677856111	-0.856570308633343\\
7.42805267423201	-0.856569156522256\\
7.42992856912645	-0.856568005125556\\
7.4318044632449	-0.856566854442649\\
7.43368035658782	-0.856565704472941\\
7.43555624915568	-0.856564555215838\\
7.43743214094896	-0.856563406670747\\
7.43930803196811	-0.856562258837076\\
7.44118392221362	-0.856561111714234\\
7.44305981168594	-0.856559965301629\\
7.44493570038553	-0.856558819598672\\
7.44681158831288	-0.856557674604772\\
7.44868747546844	-0.856556530319341\\
7.45056336185268	-0.856555386741789\\
7.45243924746607	-0.85655424387153\\
7.45431513230906	-0.856553101707975\\
7.45619101638213	-0.856551960250539\\
7.45806689968574	-0.856550819498635\\
7.45994278222035	-0.856549679451677\\
};
\addplot [color=red,solid,forget plot]
  table[row sep=crcr]{%
7.45994278222035	-0.856549679451677\\
7.46181866398642	-0.856548540109082\\
7.46369454498442	-0.856547401470264\\
7.46557042521482	-0.85654626353464\\
7.46744630467806	-0.856545126301627\\
7.46932218337462	-0.856543989770643\\
7.47119806130496	-0.856542853941106\\
7.47307393846954	-0.856541718812435\\
7.47494981486881	-0.85654058438405\\
7.47682569050324	-0.856539450655369\\
7.4787015653733	-0.856538317625815\\
7.48057743947943	-0.856537185294808\\
7.4824533128221	-0.85653605366177\\
7.48432918540176	-0.856534922726123\\
7.48620505721888	-0.856533792487291\\
7.48808092827392	-0.856532662944698\\
7.48995679856733	-0.856531534097767\\
7.49183266809956	-0.856530405945924\\
7.49370853687109	-0.856529278488593\\
7.49558440488235	-0.856528151725202\\
7.49746027213382	-0.856527025655176\\
7.49933613862594	-0.856525900277943\\
7.50121200435917	-0.856524775592931\\
7.50308786933396	-0.856523651599568\\
7.50496373355078	-0.856522528297283\\
7.50683959701008	-0.856521405685507\\
7.5087154597123	-0.856520283763668\\
7.51059132165791	-0.856519162531198\\
7.51246718284736	-0.856518041987529\\
7.51434304328109	-0.856516922132092\\
7.51621890295957	-0.85651580296432\\
7.51809476188325	-0.856514684483645\\
7.51997062005257	-0.856513566689503\\
7.52184647746799	-0.856512449581327\\
7.52372233412997	-0.856511333158552\\
7.52559819003894	-0.856510217420614\\
7.52747404519537	-0.856509102366948\\
7.5293498995997	-0.856507987996992\\
7.53122575325238	-0.856506874310183\\
7.53310160615386	-0.856505761305958\\
7.53497745830459	-0.856504648983757\\
7.53685330970503	-0.856503537343017\\
7.53872916035561	-0.856502426383179\\
7.54060501025679	-0.856501316103683\\
7.54248085940901	-0.856500206503969\\
7.54435670781273	-0.85649909758348\\
7.54623255546838	-0.856497989341656\\
7.54810840237642	-0.85649688177794\\
7.54998424853729	-0.856495774891776\\
7.55186009395143	-0.856494668682607\\
7.5537359386193	-0.856493563149878\\
7.55561178254134	-0.856492458293032\\
7.55748762571799	-0.856491354111516\\
7.5593634681497	-0.856490250604775\\
7.56123930983691	-0.856489147772256\\
7.56311515078007	-0.856488045613406\\
7.56499099097961	-0.856486944127673\\
7.56686683043599	-0.856485843314505\\
7.56874266914964	-0.85648474317335\\
7.57061850712101	-0.856483643703658\\
7.57249434435054	-0.856482544904879\\
7.57437018083868	-0.856481446776464\\
7.57624601658585	-0.856480349317863\\
7.57812185159251	-0.856479252528528\\
7.57999768585909	-0.856478156407912\\
7.58187351938604	-0.856477060955466\\
7.58374935217379	-0.856475966170645\\
7.58562518422279	-0.856474872052903\\
7.58750101553347	-0.856473778601692\\
7.58937684610627	-0.85647268581647\\
7.59125267594163	-0.856471593696691\\
7.59312850504	-0.856470502241812\\
7.5950043334018	-0.856469411451288\\
7.59688016102747	-0.856468321324578\\
7.59875598791746	-0.856467231861139\\
7.6006318140722	-0.85646614306043\\
7.60250763949212	-0.856465054921909\\
7.60438346417766	-0.856463967445036\\
7.60625928812927	-0.856462880629271\\
7.60813511134736	-0.856461794474074\\
7.61001093383238	-0.856460708978907\\
7.61188675558477	-0.856459624143232\\
7.61376257660495	-0.85645853996651\\
7.61563839689336	-0.856457456448204\\
7.61751421645044	-0.856456373587778\\
7.61939003527662	-0.856455291384696\\
7.62126585337233	-0.856454209838422\\
7.623141670738	-0.85645312894842\\
7.62501748737408	-0.856452048714158\\
7.62689330328097	-0.8564509691351\\
7.62876911845914	-0.856449890210713\\
7.63064493290899	-0.856448811940465\\
7.63252074663096	-0.856447734323823\\
7.63439655962549	-0.856446657360256\\
7.636272371893	-0.856445581049231\\
7.63814818343393	-0.85644450539022\\
7.6400239942487	-0.856443430382691\\
7.64189980433775	-0.856442356026115\\
7.64377561370149	-0.856441282319963\\
7.64565142234037	-0.856440209263707\\
7.6475272302548	-0.856439136856818\\
7.64940303744523	-0.856438065098769\\
7.65127884391207	-0.856436993989034\\
7.65315464965575	-0.856435923527086\\
7.65503045467669	-0.856434853712399\\
7.65690625897534	-0.856433784544448\\
7.6587820625521	-0.856432716022708\\
7.66065786540742	-0.856431648146656\\
7.66253366754171	-0.856430580915767\\
7.66440946895539	-0.856429514329518\\
7.6662852696489	-0.856428448387387\\
7.66816106962266	-0.856427383088852\\
7.67003686887709	-0.856426318433391\\
7.67191266741262	-0.856425254420484\\
7.67378846522967	-0.856424191049609\\
7.67566426232866	-0.856423128320248\\
7.67754005871002	-0.85642206623188\\
7.67941585437417	-0.856421004783986\\
7.68129164932153	-0.85641994397605\\
7.68316744355253	-0.856418883807551\\
7.68504323706758	-0.856417824277975\\
7.6869190298671	-0.856416765386802\\
7.68879482195153	-0.856415707133518\\
7.69067061332127	-0.856414649517607\\
7.69254640397675	-0.856413592538554\\
7.69442219391839	-0.856412536195843\\
7.69629798314661	-0.856411480488961\\
7.69817377166182	-0.856410425417394\\
7.70004955946445	-0.856409370980629\\
7.70192534655492	-0.856408317178154\\
7.70380113293364	-0.856407264009457\\
7.70567691860103	-0.856406211474025\\
7.70755270355751	-0.856405159571349\\
7.7094284878035	-0.856404108300917\\
7.71130427133942	-0.856403057662219\\
7.71318005416567	-0.856402007654747\\
7.71505583628268	-0.856400958277991\\
7.71693161769086	-0.856399909531443\\
7.71880739839063	-0.856398861414594\\
7.72068317838241	-0.856397813926939\\
7.7225589576666	-0.856396767067969\\
7.72443473624363	-0.856395720837178\\
7.72631051411391	-0.856394675234061\\
7.72818629127785	-0.856393630258112\\
7.73006206773587	-0.856392585908826\\
7.73193784348838	-0.856391542185699\\
7.73381361853579	-0.856390499088228\\
7.73568939287852	-0.856389456615909\\
7.73756516651697	-0.856388414768238\\
7.73944093945157	-0.856387373544715\\
7.74131671168272	-0.856386332944837\\
7.74319248321083	-0.856385292968103\\
7.74506825403632	-0.856384253614013\\
7.74694402415959	-0.856383214882065\\
7.74881979358106	-0.856382176771761\\
7.75069556230114	-0.8563811392826\\
7.75257133032024	-0.856380102414085\\
7.75444709763876	-0.856379066165717\\
7.75632286425712	-0.856378030536999\\
7.75819863017572	-0.856376995527432\\
7.76007439539498	-0.856375961136521\\
7.7619501599153	-0.85637492736377\\
7.76382592373709	-0.856373894208682\\
7.76570168686076	-0.856372861670762\\
7.76757744928671	-0.856371829749516\\
7.76945321101536	-0.856370798444449\\
7.7713289720471	-0.856369767755068\\
7.77320473238236	-0.85636873768088\\
7.77508049202152	-0.856367708221392\\
7.776956250965	-0.856366679376111\\
7.7788320092132	-0.856365651144546\\
7.78070776676653	-0.856364623526205\\
7.7825835236254	-0.856363596520599\\
7.7844592797902	-0.856362570127236\\
7.78633503526135	-0.856361544345627\\
7.78821079003924	-0.856360519175283\\
7.79008654412428	-0.856359494615714\\
7.79196229751687	-0.856358470666434\\
7.79383805021742	-0.856357447326953\\
7.79571380222633	-0.856356424596785\\
7.797589553544	-0.856355402475442\\
7.79946530417083	-0.856354380962439\\
7.80134105410723	-0.85635336005729\\
7.8032168033536	-0.856352339759509\\
7.80509255191033	-0.856351320068611\\
7.80696829977783	-0.856350300984113\\
7.8088440469565	-0.85634928250553\\
7.81071979344674	-0.856348264632379\\
7.81259553924895	-0.856347247364176\\
7.81447128436352	-0.85634623070044\\
7.81634702879087	-0.856345214640689\\
7.81822277253138	-0.856344199184441\\
7.82009851558545	-0.856343184331215\\
7.82197425795349	-0.856342170080531\\
7.8238499996359	-0.856341156431908\\
7.82572574063306	-0.856340143384868\\
7.82760148094538	-0.856339130938931\\
7.82947722057325	-0.856338119093618\\
7.83135295951707	-0.856337107848452\\
7.83322869777725	-0.856336097202955\\
7.83510443535416	-0.85633508715665\\
7.83698017224822	-0.856334077709059\\
7.83885590845981	-0.856333068859708\\
7.84073164398933	-0.85633206060812\\
7.84260737883717	-0.85633105295382\\
7.84448311300374	-0.856330045896333\\
7.84635884648942	-0.856329039435185\\
7.84823457929461	-0.856328033569903\\
7.85011031141971	-0.856327028300012\\
7.85198604286509	-0.85632602362504\\
7.85386177363117	-0.856325019544515\\
7.85573750371833	-0.856324016057965\\
7.85761323312697	-0.856323013164918\\
7.85948896185747	-0.856322010864903\\
7.86136468991023	-0.85632100915745\\
7.86324041728564	-0.856320008042089\\
7.86511614398409	-0.856319007518351\\
7.86699187000597	-0.856318007585765\\
7.86886759535168	-0.856317008243865\\
7.8707433200216	-0.85631600949218\\
7.87261904401613	-0.856315011330245\\
7.87449476733565	-0.85631401375759\\
7.87637048998055	-0.856313016773751\\
7.87824621195123	-0.85631202037826\\
7.88012193324807	-0.856311024570652\\
7.88199765387146	-0.856310029350461\\
7.88387337382178	-0.856309034717222\\
7.88574909309944	-0.856308040670472\\
7.88762481170481	-0.856307047209745\\
7.88950052963828	-0.856306054334579\\
7.89137624690024	-0.856305062044511\\
7.89325196349107	-0.856304070339077\\
7.89512767941117	-0.856303079217815\\
7.89700339466092	-0.856302088680265\\
7.8988791092407	-0.856301098725965\\
7.9007548231509	-0.856300109354454\\
7.90263053639191	-0.856299120565271\\
7.90450624896411	-0.856298132357958\\
7.90638196086789	-0.856297144732054\\
7.90825767210362	-0.856296157687101\\
7.9101333826717	-0.856295171222639\\
7.91200909257251	-0.856294185338212\\
7.91388480180643	-0.856293200033362\\
7.91576051037385	-0.856292215307631\\
7.91763621827514	-0.856291231160563\\
7.9195119255107	-0.856290247591702\\
7.9213876320809	-0.856289264600591\\
7.92326333798612	-0.856288282186776\\
7.92513904322676	-0.856287300349802\\
7.92701474780318	-0.856286319089214\\
7.92889045171577	-0.856285338404558\\
7.93076615496491	-0.856284358295382\\
7.93264185755099	-0.856283378761231\\
7.93451755947438	-0.856282399801654\\
7.93639326073546	-0.856281421416197\\
7.93826896133461	-0.85628044360441\\
7.94014466127221	-0.856279466365842\\
7.94202036054865	-0.85627848970004\\
7.9438960591643	-0.856277513606556\\
7.94577175711953	-0.856276538084939\\
7.94764745441473	-0.856275563134739\\
7.94952315105028	-0.856274588755508\\
7.95139884702655	-0.856273614946797\\
7.95327454234392	-0.856272641708158\\
7.95515023700277	-0.856271669039143\\
7.95702593100348	-0.856270696939304\\
7.95890162434641	-0.856269725408196\\
7.96077731703195	-0.856268754445372\\
7.96265300906048	-0.856267784050385\\
7.96452870043237	-0.856266814222791\\
7.96640439114799	-0.856265844962144\\
7.96828008120772	-0.856264876267999\\
7.97015577061194	-0.856263908139913\\
7.97203145936102	-0.856262940577442\\
7.97390714745533	-0.856261973580142\\
7.97578283489525	-0.856261007147571\\
7.97765852168115	-0.856260041279286\\
7.97953420781341	-0.856259075974845\\
7.98140989329239	-0.856258111233806\\
7.98328557811848	-0.856257147055729\\
7.98516126229204	-0.856256183440173\\
7.98703694581345	-0.856255220386697\\
7.98891262868308	-0.856254257894862\\
7.9907883109013	-0.856253295964228\\
7.99266399246848	-0.856252334594357\\
7.99453967338499	-0.856251373784809\\
7.99641535365121	-0.856250413535148\\
7.9982910332675	-0.856249453844934\\
8.00016671223424	-0.856248494713731\\
8.0020423905518	-0.856247536141102\\
8.00391806822054	-0.856246578126611\\
8.00579374524083	-0.856245620669822\\
8.00766942161305	-0.856244663770298\\
8.00954509733756	-0.856243707427606\\
8.01142077241474	-0.856242751641311\\
8.01329644684494	-0.856241796410977\\
8.01517212062854	-0.856240841736172\\
8.01704779376591	-0.856239887616462\\
8.01892346625742	-0.856238934051415\\
8.02079913810342	-0.856237981040596\\
8.02267480930429	-0.856237028583576\\
8.0245504798604	-0.856236076679921\\
8.02642614977211	-0.8562351253292\\
8.02830181903979	-0.856234174530983\\
8.0301774876638	-0.85623322428484\\
8.03205315564451	-0.856232274590339\\
8.03392882298229	-0.856231325447053\\
8.03580448967749	-0.856230376854551\\
8.03768015573049	-0.856229428812404\\
8.03955582114165	-0.856228481320185\\
8.04143148591133	-0.856227534377466\\
8.0433071500399	-0.856226587983819\\
8.04518281352772	-0.856225642138817\\
8.04705847637516	-0.856224696842033\\
8.04893413858257	-0.856223752093042\\
8.05080980015032	-0.856222807891417\\
8.05268546107878	-0.856221864236733\\
8.0545611213683	-0.856220921128565\\
8.05643678101925	-0.856219978566489\\
8.05831244003199	-0.856219036550079\\
8.06018809840688	-0.856218095078914\\
8.06206375614429	-0.856217154152569\\
8.06393941324456	-0.856216213770621\\
8.06581506970807	-0.856215273932648\\
8.06769072553518	-0.856214334638228\\
8.06956638072623	-0.856213395886939\\
8.07144203528161	-0.856212457678359\\
8.07331768920166	-0.856211520012069\\
8.07519334248674	-0.856210582887648\\
8.07706899513721	-0.856209646304675\\
8.07894464715344	-0.856208710262731\\
8.08082029853577	-0.856207774761397\\
8.08269594928457	-0.856206839800254\\
8.0845715994002	-0.856205905378884\\
8.08644724888302	-0.856204971496868\\
8.08832289773337	-0.856204038153789\\
8.09019854595162	-0.85620310534923\\
8.09207419353813	-0.856202173082774\\
8.09394984049325	-0.856201241354005\\
8.09582548681734	-0.856200310162506\\
8.09770113251076	-0.856199379507863\\
8.09957677757385	-0.85619844938966\\
8.10145242200698	-0.856197519807483\\
8.1033280658105	-0.856196590760916\\
8.10520370898477	-0.856195662249547\\
8.10707935153014	-0.856194734272961\\
8.10895499344697	-0.856193806830745\\
8.1108306347356	-0.856192879922488\\
8.1127062753964	-0.856191953547775\\
8.11458191542972	-0.856191027706196\\
8.11645755483591	-0.856190102397339\\
8.11833319361532	-0.856189177620793\\
8.12020883176831	-0.856188253376147\\
8.12208446929523	-0.856187329662991\\
8.12396010619644	-0.856186406480914\\
8.12583574247227	-0.856185483829508\\
8.1277113781231	-0.856184561708363\\
8.12958701314926	-0.85618364011707\\
8.13146264755111	-0.856182719055222\\
8.133338281329	-0.856181798522409\\
8.13521391448328	-0.856180878518225\\
8.13708954701431	-0.856179959042263\\
8.13896517892242	-0.856179040094115\\
8.14084081020798	-0.856178121673375\\
8.14271644087133	-0.856177203779638\\
8.14459207091283	-0.856176286412497\\
8.14646770033281	-0.856175369571548\\
8.14834332913163	-0.856174453256386\\
8.15021895730964	-0.856173537466605\\
8.15209458486719	-0.856172622201803\\
8.15397021180462	-0.856171707461576\\
8.15584583812229	-0.85617079324552\\
8.15772146382053	-0.856169879553233\\
8.15959708889971	-0.856168966384312\\
8.16147271336016	-0.856168053738355\\
8.16334833720223	-0.85616714161496\\
8.16522396042627	-0.856166230013726\\
8.16709958303262	-0.856165318934253\\
8.16897520502164	-0.856164408376139\\
8.17085082639366	-0.856163498338984\\
8.17272644714904	-0.85616258882239\\
8.17460206728811	-0.856161679825955\\
8.17647768681123	-0.856160771349283\\
8.17835330571874	-0.856159863391973\\
8.18022892401097	-0.856158955953627\\
8.18210454168829	-0.856158049033849\\
8.18398015875102	-0.85615714263224\\
8.18585577519952	-0.856156236748403\\
8.18773139103413	-0.856155331381942\\
8.18960700625519	-0.85615442653246\\
8.19148262086305	-0.856153522199561\\
8.19335823485804	-0.856152618382851\\
8.19523384824051	-0.856151715081933\\
8.1971094610108	-0.856150812296413\\
8.19898507316926	-0.856149910025897\\
8.20086068471623	-0.856149008269991\\
8.20273629565204	-0.8561481070283\\
8.20461190597704	-0.856147206300432\\
8.20648751569157	-0.856146306085994\\
8.20836312479597	-0.856145406384594\\
8.21023873329059	-0.856144507195838\\
8.21211434117575	-0.856143608519336\\
8.21398994845181	-0.856142710354697\\
8.21586555511909	-0.856141812701528\\
8.21774116117795	-0.85614091555944\\
8.21961676662871	-0.856140018928042\\
8.22149237147173	-0.856139122806944\\
8.22336797570733	-0.856138227195758\\
8.22524357933586	-0.856137332094093\\
8.22711918235765	-0.856136437501562\\
8.22899478477305	-0.856135543417775\\
8.23087038658238	-0.856134649842345\\
8.23274598778599	-0.856133756774884\\
8.23462158838422	-0.856132864215004\\
8.2364971883774	-0.85613197216232\\
8.23837278776587	-0.856131080616444\\
8.24024838654996	-0.856130189576991\\
8.24212398473001	-0.856129299043574\\
8.24399958230636	-0.856128409015808\\
8.24587517927934	-0.856127519493308\\
8.24775077564929	-0.856126630475689\\
8.24962637141654	-0.856125741962568\\
8.25150196658143	-0.85612485395356\\
8.2533775611443	-0.856123966448281\\
8.25525315510547	-0.856123079446349\\
8.25712874846528	-0.856122192947381\\
8.25900434122407	-0.856121306950994\\
8.26087993338217	-0.856120421456806\\
8.26275552493991	-0.856119536464435\\
8.26463111589763	-0.8561186519735\\
8.26650670625566	-0.85611776798362\\
8.26838229601433	-0.856116884494414\\
8.27025788517398	-0.856116001505502\\
8.27213347373493	-0.856115119016505\\
8.27400906169752	-0.856114237027042\\
8.27588464906208	-0.856113355536734\\
8.27776023582894	-0.856112474545203\\
8.27963582199844	-0.856111594052071\\
8.2815114075709	-0.856110714056958\\
8.28338699254666	-0.856109834559487\\
8.28526257692604	-0.856108955559281\\
8.28713816070938	-0.856108077055963\\
8.289013743897	-0.856107199049157\\
8.29088932648924	-0.856106321538485\\
8.29276490848643	-0.856105444523572\\
8.29464048988889	-0.856104568004043\\
8.29651607069696	-0.856103691979521\\
8.29839165091095	-0.856102816449633\\
8.30026723053121	-0.856101941414004\\
8.30214280955806	-0.856101066872259\\
8.30401838799183	-0.856100192824025\\
8.30589396583285	-0.856099319268928\\
8.30776954308143	-0.856098446206596\\
8.30964511973792	-0.856097573636655\\
8.31152069580264	-0.856096701558733\\
8.31339627127591	-0.856095829972458\\
8.31527184615806	-0.856094958877459\\
8.31714742044942	-0.856094088273364\\
8.31902299415032	-0.856093218159801\\
8.32089856726107	-0.856092348536402\\
8.32277413978201	-0.856091479402794\\
8.32464971171347	-0.856090610758609\\
8.32652528305576	-0.856089742603477\\
8.32840085380921	-0.856088874937028\\
8.33027642397415	-0.856088007758894\\
8.33215199355089	-0.856087141068706\\
8.33402756253978	-0.856086274866096\\
8.33590313094112	-0.856085409150696\\
8.33777869875525	-0.85608454392214\\
8.33965426598248	-0.856083679180058\\
8.34152983262315	-0.856082814924086\\
8.34340539867756	-0.856081951153856\\
8.34528096414606	-0.856081087869002\\
8.34715652902895	-0.856080225069159\\
8.34903209332656	-0.856079362753961\\
8.35090765703922	-0.856078500923044\\
8.35278322016724	-0.856077639576042\\
8.35465878271095	-0.856076778712591\\
8.35653434467066	-0.856075918332328\\
8.35840990604671	-0.856075058434888\\
8.3602854668394	-0.856074199019908\\
8.36216102704907	-0.856073340087025\\
8.36403658667603	-0.856072481635878\\
8.3659121457206	-0.856071623666102\\
8.3677877041831	-0.856070766177337\\
8.36966326206385	-0.856069909169221\\
8.37153881936318	-0.856069052641392\\
8.37341437608139	-0.856068196593491\\
8.37528993221882	-0.856067341025155\\
8.37716548777577	-0.856066485936025\\
8.37904104275257	-0.856065631325741\\
8.38091659714954	-0.856064777193943\\
8.38279215096699	-0.856063923540273\\
8.38466770420524	-0.856063070364371\\
8.38654325686461	-0.856062217665879\\
8.38841880894542	-0.856061365444438\\
8.39029436044798	-0.856060513699692\\
8.39216991137261	-0.856059662431281\\
8.39404546171963	-0.85605881163885\\
8.39592101148936	-0.85605796132204\\
8.3977965606821	-0.856057111480497\\
8.39967210929819	-0.856056262113863\\
8.40154765733792	-0.856055413221782\\
8.40342320480162	-0.8560545648039\\
8.40529875168961	-0.856053716859861\\
8.40717429800219	-0.85605286938931\\
8.40904984373969	-0.856052022391893\\
8.41092538890242	-0.856051175867255\\
8.41280093349069	-0.856050329815043\\
8.41467647750482	-0.856049484234903\\
8.41655202094512	-0.856048639126482\\
8.4184275638119	-0.856047794489427\\
8.42030310610548	-0.856046950323386\\
8.42217864782618	-0.856046106628007\\
8.4240541889743	-0.856045263402937\\
8.42592972955015	-0.856044420647825\\
8.42780526955406	-0.85604357836232\\
8.42968080898633	-0.856042736546071\\
8.43155634784727	-0.856041895198728\\
8.4334318861372	-0.85604105431994\\
8.43530742385643	-0.856040213909358\\
8.43718296100527	-0.856039373966631\\
8.43905849758403	-0.856038534491412\\
8.44093403359302	-0.85603769548335\\
8.44280956903256	-0.856036856942098\\
8.44468510390295	-0.856036018867306\\
8.4465606382045	-0.856035181258628\\
8.44843617193753	-0.856034344115716\\
8.45031170510235	-0.856033507438222\\
8.45218723769925	-0.8560326712258\\
8.45406276972857	-0.856031835478102\\
8.45593830119059	-0.856031000194783\\
8.45781383208563	-0.856030165375497\\
8.45968936241401	-0.856029331019897\\
8.46156489217603	-0.85602849712764\\
8.46344042137199	-0.856027663698379\\
8.46531595000221	-0.85602683073177\\
8.46719147806699	-0.856025998227469\\
8.46906700556664	-0.856025166185131\\
8.47094253250147	-0.856024334604414\\
8.47281805887179	-0.856023503484973\\
8.4746935846779	-0.856022672826466\\
8.47656910992011	-0.85602184262855\\
8.47844463459873	-0.856021012890883\\
8.48032015871406	-0.856020183613122\\
8.48219568226641	-0.856019354794925\\
8.48407120525608	-0.856018526435952\\
8.48594672768338	-0.856017698535861\\
8.48782224954862	-0.856016871094311\\
8.4896977708521	-0.856016044110961\\
8.49157329159412	-0.856015217585473\\
8.493448811775	-0.856014391517505\\
8.49532433139503	-0.856013565906718\\
8.49719985045452	-0.856012740752774\\
8.49907536895378	-0.856011916055332\\
8.5009508868931	-0.856011091814055\\
8.50282640427279	-0.856010268028604\\
8.50470192109316	-0.856009444698641\\
8.50657743735451	-0.856008621823829\\
8.50845295305714	-0.856007799403831\\
8.51032846820135	-0.856006977438308\\
8.51220398278745	-0.856006155926925\\
8.51407949681574	-0.856005334869345\\
8.51595501028652	-0.856004514265232\\
8.5178305232001	-0.856003694114251\\
8.51970603555676	-0.856002874416065\\
8.52158154735683	-0.856002055170341\\
8.52345705860059	-0.856001236376742\\
8.52533256928835	-0.856000418034934\\
8.52720807942041	-0.855999600144584\\
8.52908358899707	-0.855998782705357\\
8.53095909801863	-0.855997965716919\\
8.53283460648539	-0.855997149178938\\
8.53471011439766	-0.855996333091081\\
8.53658562175572	-0.855995517453014\\
8.53846112855989	-0.855994702264406\\
8.54033663481045	-0.855993887524924\\
8.54221214050771	-0.855993073234237\\
8.54408764565197	-0.855992259392014\\
8.54596315024353	-0.855991445997922\\
8.54783865428268	-0.855990633051632\\
8.54971415776973	-0.855989820552812\\
8.55158966070496	-0.855989008501134\\
8.55346516308869	-0.855988196896266\\
8.5553406649212	-0.855987385737879\\
8.55721616620279	-0.855986575025644\\
8.55909166693377	-0.855985764759233\\
8.56096716711443	-0.855984954938315\\
8.56284266674506	-0.855984145562564\\
8.56471816582596	-0.85598333663165\\
8.56659366435743	-0.855982528145246\\
8.56846916233976	-0.855981720103026\\
8.57034465977325	-0.85598091250466\\
8.5722201566582	-0.855980105349824\\
8.5740956529949	-0.855979298638189\\
8.57597114878365	-0.855978492369431\\
8.57784664402474	-0.855977686543222\\
8.57972213871846	-0.855976881159238\\
8.58159763286512	-0.855976076217153\\
8.583473126465	-0.855975271716641\\
8.5853486195184	-0.855974467657379\\
8.58722411202561	-0.855973664039042\\
8.58909960398693	-0.855972860861306\\
8.59097509540265	-0.855972058123846\\
8.59285058627307	-0.85597125582634\\
8.59472607659848	-0.855970453968463\\
8.59660156637916	-0.855969652549894\\
8.59847705561542	-0.855968851570309\\
8.60035254430755	-0.855968051029387\\
8.60222803245583	-0.855967250926804\\
8.60410352006056	-0.85596645126224\\
8.60597900712204	-0.855965652035373\\
8.60785449364055	-0.855964853245882\\
8.60972997961638	-0.855964054893446\\
8.61160546504984	-0.855963256977743\\
8.6134809499412	-0.855962459498455\\
8.61535643429075	-0.855961662455262\\
8.6172319180988	-0.855960865847842\\
8.61910740136563	-0.855960069675878\\
8.62098288409153	-0.855959273939049\\
8.62285836627679	-0.855958478637038\\
8.62473384792169	-0.855957683769525\\
8.62660932902654	-0.855956889336193\\
8.62848480959162	-0.855956095336723\\
8.63036028961722	-0.855955301770798\\
8.63223576910362	-0.855954508638101\\
8.63411124805112	-0.855953715938314\\
8.63598672646	-0.855952923671122\\
8.63786220433055	-0.855952131836206\\
8.63973768166307	-0.855951340433252\\
8.64161315845784	-0.855950549461943\\
8.64348863471514	-0.855949758921964\\
8.64536411043527	-0.855948968812999\\
8.64723958561851	-0.855948179134734\\
8.64911506026515	-0.855947389886853\\
8.65099053437547	-0.855946601069043\\
8.65286600794977	-0.855945812680988\\
8.65474148098833	-0.855945024722377\\
8.65661695349143	-0.855944237192893\\
8.65849242545937	-0.855943450092226\\
8.66036789689242	-0.855942663420061\\
8.66224336779088	-0.855941877176085\\
8.66411883815502	-0.855941091359987\\
8.66599430798514	-0.855940305971454\\
8.66786977728152	-0.855939521010175\\
8.66974524604444	-0.855938736475838\\
8.67162071427419	-0.855937952368131\\
8.67349618197106	-0.855937168686744\\
8.67537164913532	-0.855936385431365\\
8.67724711576727	-0.855935602601685\\
8.67912258186718	-0.855934820197394\\
8.68099804743534	-0.855934038218181\\
8.68287351247203	-0.855933256663737\\
8.68474897697754	-0.855932475533753\\
8.68662444095215	-0.855931694827919\\
8.68849990439615	-0.855930914545927\\
8.6903753673098	-0.855930134687469\\
8.69225082969341	-0.855929355252236\\
8.69412629154724	-0.855928576239921\\
8.69600175287159	-0.855927797650216\\
8.69787721366673	-0.855927019482813\\
8.69975267393295	-0.855926241737406\\
8.70162813367052	-0.855925464413688\\
8.70350359287973	-0.855924687511353\\
8.70537905156086	-0.855923911030093\\
8.7072545097142	-0.855923134969604\\
8.70912996734001	-0.85592235932958\\
8.71100542443858	-0.855921584109715\\
8.7128808810102	-0.855920809309704\\
8.71475633705514	-0.855920034929243\\
8.71663179257368	-0.855919260968026\\
8.7185072475661	-0.855918487425751\\
8.72038270203268	-0.855917714302112\\
8.72225815597371	-0.855916941596806\\
8.72413360938945	-0.855916169309531\\
8.72600906228019	-0.855915397439981\\
8.72788451464621	-0.855914625987856\\
8.72975996648778	-0.855913854952851\\
8.73163541780519	-0.855913084334666\\
8.73351086859871	-0.855912314132997\\
8.73538631886862	-0.855911544347544\\
8.7372617686152	-0.855910774978004\\
8.73913721783872	-0.855910006024077\\
8.74101266653947	-0.855909237485461\\
8.74288811471772	-0.855908469361856\\
8.74476356237374	-0.855907701652961\\
8.74663900950781	-0.855906934358477\\
8.74851445612022	-0.855906167478103\\
8.75038990221123	-0.855905401011541\\
8.75226534778113	-0.85590463495849\\
8.75414079283018	-0.855903869318652\\
8.75601623735867	-0.855903104091728\\
8.75789168136687	-0.855902339277419\\
8.75976712485505	-0.855901574875427\\
8.7616425678235	-0.855900810885455\\
8.76351801027247	-0.855900047307205\\
8.76539345220226	-0.855899284140379\\
8.76726889361314	-0.85589852138468\\
8.76914433450538	-0.855897759039811\\
8.77101977487924	-0.855896997105477\\
8.77289521473502	-0.855896235581379\\
8.77477065407298	-0.855895474467223\\
8.77664609289339	-0.855894713762713\\
8.77852153119653	-0.855893953467553\\
8.78039696898267	-0.855893193581448\\
8.78227240625209	-0.855892434104102\\
8.78414784300506	-0.855891675035222\\
8.78602327924184	-0.855890916374513\\
8.78789871496272	-0.85589015812168\\
8.78977415016797	-0.85588940027643\\
8.79164958485785	-0.85588864283847\\
8.79352501903264	-0.855887885807504\\
8.79540045269261	-0.855887129183242\\
8.79727588583804	-0.855886372965389\\
8.79915131846919	-0.855885617153653\\
8.80102675058633	-0.855884861747743\\
8.80290218218974	-0.855884106747365\\
8.80477761327968	-0.855883352152228\\
8.80665304385644	-0.855882597962041\\
8.80852847392027	-0.855881844176512\\
8.81040390347145	-0.85588109079535\\
8.81227933251025	-0.855880337818265\\
8.81415476103694	-0.855879585244966\\
8.81603018905178	-0.855878833075162\\
8.81790561655505	-0.855878081308565\\
8.81978104354702	-0.855877329944884\\
8.82165647002795	-0.85587657898383\\
8.82353189599811	-0.855875828425113\\
8.82540732145778	-0.855875078268446\\
8.82728274640722	-0.855874328513538\\
8.82915817084671	-0.855873579160102\\
8.83103359477649	-0.85587283020785\\
8.83290901819686	-0.855872081656494\\
8.83478444110806	-0.855871333505746\\
8.83665986351038	-0.855870585755319\\
8.83853528540408	-0.855869838404926\\
8.84041070678942	-0.855869091454279\\
8.84228612766668	-0.855868344903093\\
8.84416154803611	-0.855867598751081\\
8.84603696789799	-0.855866852997958\\
8.84791238725258	-0.855866107643436\\
8.84978780610016	-0.855865362687231\\
8.85166322444097	-0.855864618129058\\
8.8535386422753	-0.855863873968631\\
8.8554140596034	-0.855863130205666\\
8.85728947642555	-0.855862386839878\\
8.859164892742	-0.855861643870983\\
8.86104030855303	-0.855860901298697\\
8.86291572385889	-0.855860159122737\\
8.86479113865986	-0.855859417342818\\
8.86666655295619	-0.855858675958657\\
8.86854196674815	-0.855857934969972\\
8.87041738003601	-0.85585719437648\\
8.87229279282003	-0.855856454177898\\
8.87416820510047	-0.855855714373944\\
8.8760436168776	-0.855854974964337\\
8.87791902815168	-0.855854235948794\\
8.87979443892297	-0.855853497327034\\
8.88166984919174	-0.855852759098776\\
8.88354525895825	-0.855852021263739\\
8.88542066822276	-0.855851283821642\\
8.88729607698554	-0.855850546772204\\
8.88917148524684	-0.855849810115146\\
8.89104689300694	-0.855849073850188\\
8.89292230026608	-0.85584833797705\\
8.89479770702454	-0.855847602495452\\
8.89667311328257	-0.855846867405116\\
8.89854851904044	-0.855846132705761\\
8.90042392429841	-0.855845398397111\\
8.90229932905674	-0.855844664478886\\
8.90417473331568	-0.855843930950807\\
8.90605013707551	-0.855843197812598\\
8.90792554033648	-0.85584246506398\\
8.90980094309885	-0.855841732704676\\
8.91167634536288	-0.855841000734409\\
8.91355174712884	-0.855840269152902\\
8.91542714839697	-0.855839537959878\\
8.91730254916755	-0.85583880715506\\
8.91917794944083	-0.855838076738173\\
8.92105334921707	-0.855837346708941\\
8.92292874849652	-0.855836617067087\\
8.92480414727946	-0.855835887812336\\
8.92667954556613	-0.855835158944414\\
8.9285549433568	-0.855834430463045\\
8.93043034065172	-0.855833702367954\\
8.93230573745116	-0.855832974658867\\
8.93418113375536	-0.85583224733551\\
8.93605652956459	-0.855831520397608\\
8.93793192487911	-0.855830793844888\\
8.93980731969917	-0.855830067677077\\
8.94168271402503	-0.855829341893901\\
8.94355810785695	-0.855828616495087\\
8.94543350119518	-0.855827891480362\\
8.94730889403999	-0.855827166849454\\
8.94918428639162	-0.855826442602091\\
8.95105967825034	-0.855825718738001\\
8.95293506961639	-0.85582499525691\\
8.95481046049005	-0.85582427215855\\
8.95668585087155	-0.855823549442647\\
8.95856124076117	-0.85582282710893\\
8.96043663015915	-0.85582210515713\\
8.96231201906575	-0.855821383586975\\
8.96418740748122	-0.855820662398195\\
8.96606279540582	-0.85581994159052\\
8.96793818283981	-0.855819221163679\\
8.96981356978343	-0.855818501117404\\
8.97168895623695	-0.855817781451425\\
8.97356434220062	-0.855817062165473\\
8.97543972767469	-0.855816343259279\\
8.97731511265941	-0.855815624732574\\
8.97919049715505	-0.85581490658509\\
8.98106588116184	-0.855814188816558\\
8.98294126468006	-0.855813471426711\\
8.98481664770994	-0.855812754415281\\
8.98669203025175	-0.855812037782\\
8.98856741230573	-0.855811321526601\\
8.99044279387214	-0.855810605648818\\
8.99231817495123	-0.855809890148383\\
8.99419355554326	-0.85580917502503\\
8.99606893564847	-0.855808460278492\\
8.99794431526712	-0.855807745908504\\
8.99981969439946	-0.8558070319148\\
9.00169507304574	-0.855806318297113\\
9.00357045120622	-0.85580560505518\\
9.00544582888113	-0.855804892188733\\
9.00732120607075	-0.85580417969751\\
9.0091965827753	-0.855803467581245\\
9.01107195899506	-0.855802755839673\\
9.01294733473027	-0.855802044472531\\
9.01482270998117	-0.855801333479554\\
9.01669808474802	-0.85580062286048\\
9.01857345903107	-0.855799912615043\\
9.02044883283057	-0.855799202742982\\
9.02232420614676	-0.855798493244032\\
9.02419957897991	-0.855797784117932\\
9.02607495133025	-0.855797075364419\\
9.02795032319804	-0.85579636698323\\
9.02982569458353	-0.855795658974104\\
9.03170106548696	-0.855794951336778\\
9.03357643590858	-0.85579424407099\\
9.03545180584865	-0.855793537176481\\
9.03732717530741	-0.855792830652987\\
9.0392025442851	-0.855792124500249\\
9.04107791278198	-0.855791418718006\\
9.0429532807983	-0.855790713305997\\
9.0448286483343	-0.855790008263962\\
9.04670401539023	-0.855789303591641\\
9.04857938196634	-0.855788599288774\\
9.05045474806287	-0.855787895355101\\
9.05233011368007	-0.855787191790364\\
9.0542054788182	-0.855786488594304\\
9.05608084347748	-0.85578578576666\\
9.05795620765818	-0.855785083307176\\
9.05983157136054	-0.855784381215592\\
9.06170693458481	-0.855783679491649\\
9.06358229733122	-0.855782978135091\\
9.06545765960003	-0.85578227714566\\
9.06733302139148	-0.855781576523098\\
9.06920838270582	-0.855780876267147\\
9.07108374354329	-0.855780176377551\\
9.07295910390414	-0.855779476854053\\
9.07483446378861	-0.855778777696396\\
9.07670982319695	-0.855778078904324\\
9.07858518212941	-0.85577738047758\\
9.08046054058622	-0.85577668241591\\
9.08233589856763	-0.855775984719056\\
9.08421125607389	-0.855775287386765\\
9.08608661310524	-0.85577459041878\\
9.08796196966192	-0.855773893814846\\
9.08983732574418	-0.855773197574709\\
9.09171268135226	-0.855772501698114\\
9.09358803648641	-0.855771806184807\\
9.09546339114686	-0.855771111034534\\
9.09733874533386	-0.85577041624704\\
9.09921409904766	-0.855769721822072\\
9.10108945228849	-0.855769027759377\\
9.1029648050566	-0.855768334058701\\
9.10484015735223	-0.855767640719792\\
9.10671550917563	-0.855766947742396\\
9.10859086052703	-0.855766255126261\\
9.11046621140667	-0.855765562871135\\
9.11234156181481	-0.855764870976766\\
9.11421691175167	-0.855764179442901\\
9.11609226121751	-0.855763488269289\\
9.11796761021256	-0.855762797455679\\
9.11984295873707	-0.855762107001819\\
9.12171830679126	-0.855761416907458\\
9.1235936543754	-0.855760727172346\\
9.12546900148971	-0.855760037796232\\
9.12734434813443	-0.855759348778865\\
9.12921969430981	-0.855758660119996\\
9.13109504001609	-0.855757971819374\\
9.13297038525351	-0.855757283876751\\
9.1348457300223	-0.855756596291875\\
9.13672107432271	-0.855755909064499\\
9.13859641815497	-0.855755222194372\\
9.14047176151933	-0.855754535681247\\
9.14234710441602	-0.855753849524875\\
9.14422244684528	-0.855753163725006\\
9.14609778880736	-0.855752478281394\\
9.14797313030248	-0.85575179319379\\
9.14984847133089	-0.855751108461946\\
9.15172381189283	-0.855750424085615\\
9.15359915198853	-0.85574974006455\\
9.15547449161824	-0.855749056398503\\
9.15734983078218	-0.855748373087227\\
9.1592251694806	-0.855747690130477\\
9.16110050771374	-0.855747007528005\\
9.16297584548183	-0.855746325279565\\
9.16485118278511	-0.855745643384911\\
9.16672651962382	-0.855744961843798\\
9.16860185599819	-0.85574428065598\\
9.17047719190846	-0.855743599821211\\
9.17235252735487	-0.855742919339246\\
9.17422786233765	-0.855742239209841\\
9.17610319685704	-0.85574155943275\\
9.17797853091327	-0.85574088000773\\
9.17985386450659	-0.855740200934535\\
9.18172919763722	-0.855739522212922\\
9.1836045303054	-0.855738843842647\\
9.18547986251137	-0.855738165823467\\
9.18735519425537	-0.855737488155136\\
9.18923052553762	-0.855736810837414\\
9.19110585635836	-0.855736133870056\\
9.19298118671784	-0.855735457252819\\
9.19485651661627	-0.855734780985462\\
9.1967318460539	-0.855734105067741\\
9.19860717503096	-0.855733429499415\\
9.20048250354768	-0.855732754280241\\
9.20235783160431	-0.855732079409978\\
9.20423315920106	-0.855731404888385\\
9.20610848633819	-0.855730730715219\\
9.20798381301591	-0.85573005689024\\
9.20985913923446	-0.855729383413206\\
9.21173446499408	-0.855728710283878\\
9.213609790295	-0.855728037502014\\
9.21548511513745	-0.855727365067375\\
9.21736043952167	-0.85572669297972\\
9.21923576344788	-0.855726021238809\\
9.22111108691632	-0.855725349844403\\
9.22298640992722	-0.855724678796262\\
9.22486173248082	-0.855724008094147\\
9.22673705457734	-0.855723337737818\\
9.22861237621702	-0.855722667727038\\
9.23048769740008	-0.855721998061568\\
9.23236301812677	-0.855721328741168\\
9.23423833839731	-0.855720659765601\\
9.23611365821193	-0.855719991134629\\
9.23798897757086	-0.855719322848014\\
9.23986429647434	-0.855718654905518\\
9.24173961492259	-0.855717987306905\\
9.24361493291585	-0.855717320051936\\
9.24549025045434	-0.855716653140375\\
9.2473655675383	-0.855715986571985\\
9.24924088416796	-0.855715320346529\\
9.25111620034354	-0.855714654463772\\
9.25299151606527	-0.855713988923476\\
9.25486683133339	-0.855713323725407\\
9.25674214614813	-0.855712658869327\\
9.25861746050971	-0.855711994355003\\
9.26049277441836	-0.855711330182198\\
9.26236808787431	-0.855710666350677\\
9.26424340087779	-0.855710002860205\\
9.26611871342903	-0.855709339710548\\
9.26799402552826	-0.855708676901471\\
9.2698693371757	-0.855708014432739\\
9.27174464837159	-0.855707352304119\\
9.27361995911615	-0.855706690515377\\
9.27549526940961	-0.855706029066278\\
9.27737057925219	-0.85570536795659\\
9.27924588864413	-0.855704707186079\\
9.28112119758566	-0.855704046754512\\
9.28299650607699	-0.855703386661656\\
9.28487181411836	-0.855702726907278\\
9.28674712170999	-0.855702067491146\\
9.28862242885211	-0.855701408413027\\
9.29049773554495	-0.85570074967269\\
9.29237304178874	-0.855700091269902\\
9.29424834758369	-0.855699433204432\\
9.29612365293004	-0.855698775476049\\
9.29799895782801	-0.85569811808452\\
9.29987426227783	-0.855697461029615\\
9.30174956627972	-0.855696804311103\\
9.30362486983392	-0.855696147928752\\
9.30550017294063	-0.855695491882334\\
9.3073754756001	-0.855694836171617\\
9.30925077781254	-0.855694180796371\\
9.31112607957818	-0.855693525756367\\
9.31300138089724	-0.855692871051373\\
9.31487668176996	-0.855692216681162\\
9.31675198219655	-0.855691562645504\\
9.31862728217723	-0.855690908944169\\
9.32050258171224	-0.855690255576929\\
9.3223778808018	-0.855689602543555\\
9.32425317944612	-0.855688949843818\\
9.32612847764544	-0.85568829747749\\
9.32800377539998	-0.855687645444343\\
9.32987907270996	-0.855686993744149\\
9.3317543695756	-0.85568634237668\\
9.33362966599713	-0.855685691341708\\
9.33550496197477	-0.855685040639007\\
9.33738025750874	-0.855684390268349\\
9.33925555259927	-0.855683740229507\\
9.34113084724658	-0.855683090522254\\
9.34300614145089	-0.855682441146364\\
9.34488143521243	-0.85568179210161\\
9.34675672853141	-0.855681143387766\\
9.34863202140806	-0.855680495004606\\
9.35050731384259	-0.855679846951905\\
9.35238260583524	-0.855679199229436\\
9.35425789738622	-0.855678551836975\\
9.35613318849575	-0.855677904774295\\
9.35800847916406	-0.855677258041173\\
9.35988376939136	-0.855676611637382\\
9.36175905917789	-0.855675965562699\\
9.36363434852385	-0.855675319816899\\
9.36550963742946	-0.855674674399758\\
9.36738492589496	-0.855674029311052\\
9.36926021392056	-0.855673384550556\\
9.37113550150647	-0.855672740118048\\
9.37301078865293	-0.855672096013303\\
9.37488607536014	-0.855671452236098\\
9.37676136162834	-0.855670808786211\\
9.37863664745773	-0.855670165663418\\
9.38051193284854	-0.855669522867496\\
9.38238721780099	-0.855668880398224\\
9.3842625023153	-0.855668238255378\\
9.38613778639168	-0.855667596438736\\
9.38801307003036	-0.855666954948077\\
9.38988835323154	-0.855666313783178\\
9.39176363599547	-0.855665672943819\\
9.39363891832234	-0.855665032429778\\
9.39551420021238	-0.855664392240833\\
9.3973894816658	-0.855663752376763\\
9.39926476268284	-0.855663112837348\\
9.40114004326369	-0.855662473622367\\
9.40301532340859	-0.855661834731599\\
9.40489060311774	-0.855661196164825\\
9.40676588239137	-0.855660557921823\\
9.40864116122969	-0.855659920002375\\
9.41051643963292	-0.855659282406261\\
9.41239171760128	-0.85565864513326\\
9.41426699513498	-0.855658008183154\\
9.41614227223425	-0.855657371555724\\
9.41801754889929	-0.85565673525075\\
9.41989282513032	-0.855656099268014\\
9.42176810092756	-0.855655463607297\\
9.42364337629123	-0.855654828268381\\
9.42551865122154	-0.855654193251047\\
9.42739392571871	-0.855653558555078\\
9.42926919978295	-0.855652924180255\\
9.43114447341448	-0.855652290126362\\
9.43301974661351	-0.85565165639318\\
9.43489501938027	-0.855651022980493\\
9.43677029171495	-0.855650389888082\\
9.43864556361779	-0.855649757115732\\
9.44052083508899	-0.855649124663225\\
9.44239610612878	-0.855648492530345\\
9.44427137673735	-0.855647860716876\\
9.44614664691493	-0.855647229222601\\
9.44802191666174	-0.855646598047304\\
9.44989718597798	-0.85564596719077\\
9.45177245486388	-0.855645336652783\\
9.45364772331963	-0.855644706433127\\
9.45552299134547	-0.855644076531588\\
9.45739825894159	-0.855643446947949\\
9.45927352610823	-0.855642817681997\\
9.46114879284558	-0.855642188733517\\
9.46302405915386	-0.855641560102293\\
9.46489932503328	-0.855640931788112\\
9.46677459048407	-0.855640303790759\\
9.46864985550642	-0.855639676110021\\
9.47052512010056	-0.855639048745683\\
9.47240038426669	-0.855638421697533\\
9.47427564800502	-0.855637794965355\\
9.47615091131578	-0.855637168548939\\
9.47802617419917	-0.855636542448069\\
9.4799014366554	-0.855635916662533\\
9.48177669868469	-0.85563529119212\\
9.48365196028724	-0.855634666036615\\
9.48552722146328	-0.855634041195806\\
9.487402482213	-0.855633416669483\\
9.48927774253662	-0.855632792457431\\
9.49115300243435	-0.855632168559441\\
9.49302826190641	-0.855631544975299\\
9.494903520953	-0.855630921704795\\
9.49677877957433	-0.855630298747717\\
9.49865403777062	-0.855629676103854\\
9.50052929554207	-0.855629053772995\\
9.5024045528889	-0.85562843175493\\
9.50427980981131	-0.855627810049447\\
9.50615506630952	-0.855627188656337\\
9.50803032238373	-0.855626567575389\\
9.50990557803416	-0.855625946806393\\
9.51178083326101	-0.85562532634914\\
9.51365608806449	-0.855624706203419\\
9.51553134244482	-0.855624086369022\\
9.51740659640219	-0.855623466845738\\
9.51928184993683	-0.855622847633359\\
9.52115710304894	-0.855622228731675\\
9.52303235573872	-0.855621610140478\\
9.5249076080064	-0.85562099185956\\
9.52678285985216	-0.855620373888711\\
9.52865811127623	-0.855619756227724\\
9.53053336227881	-0.85561913887639\\
9.53240861286011	-0.855618521834502\\
9.53428386302034	-0.855617905101851\\
9.5361591127597	-0.85561728867823\\
9.5380343620784	-0.855616672563433\\
9.53990961097666	-0.85561605675725\\
9.54178485945467	-0.855615441259476\\
9.54366010751265	-0.855614826069904\\
9.5455353551508	-0.855614211188327\\
9.54741060236932	-0.855613596614538\\
9.54928584916843	-0.855612982348331\\
9.55116109554834	-0.855612368389499\\
9.55303634150924	-0.855611754737838\\
9.55491158705134	-0.855611141393141\\
9.55678683217486	-0.855610528355202\\
9.55866207688	-0.855609915623815\\
9.56053732116695	-0.855609303198777\\
9.56241256503594	-0.85560869107988\\
9.56428780848716	-0.855608079266921\\
9.56616305152082	-0.855607467759695\\
9.56803829413712	-0.855606856557997\\
9.56991353633628	-0.855606245661622\\
9.57178877811849	-0.855605635070367\\
9.57366401948397	-0.855605024784026\\
9.57553926043291	-0.855604414802397\\
9.57741450096552	-0.855603805125276\\
9.579289741082	-0.855603195752458\\
9.58116498078256	-0.85560258668374\\
9.58304022006741	-0.85560197791892\\
9.58491545893675	-0.855601369457794\\
9.58679069739077	-0.855600761300159\\
9.5886659354297	-0.855600153445813\\
9.59054117305372	-0.855599545894552\\
9.59241641026305	-0.855598938646175\\
9.59429164705788	-0.855598331700479\\
9.59616688343842	-0.855597725057262\\
9.59804211940488	-0.855597118716322\\
9.59991735495746	-0.855596512677459\\
9.60179259009635	-0.855595906940469\\
9.60366782482177	-0.855595301505151\\
9.60554305913391	-0.855594696371305\\
9.60741829303299	-0.855594091538729\\
9.60929352651919	-0.855593487007223\\
9.61116875959272	-0.855592882776586\\
9.61304399225379	-0.855592278846617\\
9.6149192245026	-0.855591675217115\\
9.61679445633935	-0.855591071887881\\
9.61866968776424	-0.855590468858715\\
9.62054491877747	-0.855589866129417\\
9.62242014937924	-0.855589263699786\\
9.62429537956976	-0.855588661569624\\
9.62617060934923	-0.855588059738731\\
9.62804583871784	-0.855587458206908\\
9.62992106767581	-0.855586856973956\\
9.63179629622332	-0.855586256039675\\
9.63367152436059	-0.855585655403868\\
9.63554675208781	-0.855585055066336\\
9.63742197940518	-0.855584455026879\\
9.6392972063129	-0.855583855285301\\
9.64117243281117	-0.855583255841403\\
9.6430476589002	-0.855582656694988\\
9.64492288458018	-0.855582057845857\\
9.64679810985131	-0.855581459293812\\
9.64867333471379	-0.855580861038657\\
9.65054855916783	-0.855580263080195\\
9.65242378321362	-0.855579665418228\\
9.65429900685135	-0.855579068052559\\
9.65617423008124	-0.855578470982991\\
9.65804945290348	-0.855577874209329\\
9.65992467531826	-0.855577277731376\\
9.66179989732579	-0.855576681548935\\
9.66367511892627	-0.85557608566181\\
9.66555034011989	-0.855575490069806\\
9.66742556090685	-0.855574894772727\\
9.66930078128735	-0.855574299770376\\
9.6711760012616	-0.855573705062559\\
9.67305122082978	-0.855573110649081\\
9.67492643999209	-0.855572516529746\\
9.67680165874873	-0.85557192270436\\
9.67867687709991	-0.855571329172727\\
9.68055209504581	-0.855570735934654\\
9.68242731258664	-0.855570142989944\\
9.6843025297226	-0.855569550338406\\
9.68617774645387	-0.855568957979843\\
9.68805296278065	-0.855568365914063\\
9.68992817870316	-0.855567774140871\\
9.69180339422157	-0.855567182660074\\
9.69367860933608	-0.855566591471478\\
9.69555382404691	-0.85556600057489\\
9.69742903835423	-0.855565409970117\\
9.69930425225824	-0.855564819656966\\
9.70117946575915	-0.855564229635244\\
9.70305467885715	-0.855563639904758\\
9.70492989155243	-0.855563050465316\\
9.70680510384519	-0.855562461316726\\
9.70868031573562	-0.855561872458795\\
9.71055552722393	-0.855561283891331\\
9.7124307383103	-0.855560695614142\\
9.71430594899494	-0.855560107627037\\
9.71618115927803	-0.855559519929824\\
9.71805636915977	-0.855558932522311\\
9.71993157864036	-0.855558345404308\\
9.72180678771998	-0.855557758575624\\
9.72368199639885	-0.855557172036067\\
9.72555720467714	-0.855556585785446\\
9.72743241255506	-0.855555999823571\\
9.7293076200328	-0.855555414150252\\
9.73118282711055	-0.855554828765299\\
9.73305803378851	-0.85555424366852\\
9.73493324006687	-0.855553658859727\\
9.73680844594582	-0.855553074338729\\
9.73868365142556	-0.855552490105336\\
9.74055885650628	-0.85555190615936\\
9.74243406118817	-0.855551322500611\\
9.74430926547144	-0.855550739128899\\
9.74618446935626	-0.855550156044036\\
9.74805967284284	-0.855549573245832\\
9.74993487593137	-0.8555489907341\\
9.75181007862203	-0.855548408508649\\
9.75368528091503	-0.855547826569293\\
9.75556048281055	-0.855547244915843\\
9.75743568430879	-0.85554666354811\\
9.75931088540993	-0.855546082465907\\
9.76118608611418	-0.855545501669046\\
9.76306128642173	-0.855544921157338\\
9.76493648633276	-0.855544340930598\\
9.76681168584746	-0.855543760988637\\
9.76868688496603	-0.855543181331269\\
9.77056208368866	-0.855542601958306\\
9.77243728201555	-0.855542022869561\\
9.77431247994688	-0.855541444064847\\
9.77618767748284	-0.855540865543979\\
9.77806287462362	-0.85554028730677\\
9.77993807136942	-0.855539709353033\\
9.78181326772043	-0.855539131682582\\
9.78368846367683	-0.855538554295232\\
9.78556365923882	-0.855537977190797\\
9.78743885440659	-0.85553740036909\\
9.78931404918033	-0.855536823829927\\
9.79118924356022	-0.855536247573122\\
9.79306443754647	-0.85553567159849\\
9.79493963113925	-0.855535095905847\\
9.79681482433876	-0.855534520495006\\
9.79869001714519	-0.855533945365783\\
9.80056520955872	-0.855533370517995\\
9.80244040157955	-0.855532795951456\\
9.80431559320786	-0.855532221665982\\
9.80619078444385	-0.855531647661389\\
9.80806597528771	-0.855531073937493\\
9.80994116573962	-0.85553050049411\\
9.81181635579977	-0.855529927331057\\
9.81369154546834	-0.85552935444815\\
9.81556673474554	-0.855528781845206\\
9.81744192363155	-0.855528209522041\\
9.81931711212655	-0.855527637478473\\
9.82119230023073	-0.855527065714318\\
9.82306748794429	-0.855526494229394\\
9.8249426752674	-0.855525923023519\\
9.82681786220027	-0.855525352096509\\
9.82869304874307	-0.855524781448182\\
9.83056823489599	-0.855524211078357\\
9.83244342065922	-0.855523640986851\\
9.83431860603295	-0.855523071173482\\
9.83619379101736	-0.855522501638069\\
9.83806897561265	-0.855521932380431\\
9.83994415981899	-0.855521363400384\\
9.84181934363659	-0.85552079469775\\
9.84369452706561	-0.855520226272346\\
9.84556971010625	-0.855519658123991\\
9.8474448927587	-0.855519090252505\\
9.84932007502314	-0.855518522657706\\
9.85119525689976	-0.855517955339415\\
9.85307043838874	-0.855517388297451\\
9.85494561949027	-0.855516821531634\\
9.85682080020455	-0.855516255041784\\
9.85869598053174	-0.85551568882772\\
9.86057116047204	-0.855515122889263\\
9.86244634002563	-0.855514557226234\\
9.86432151919271	-0.855513991838452\\
9.86619669797344	-0.855513426725739\\
9.86807187636803	-0.855512861887916\\
9.86994705437665	-0.855512297324802\\
9.87182223199949	-0.85551173303622\\
9.87369740923674	-0.855511169021991\\
9.87557258608857	-0.855510605281935\\
9.87744776255517	-0.855510041815875\\
9.87932293863674	-0.855509478623633\\
9.88119811433345	-0.855508915705029\\
9.88307328964548	-0.855508353059886\\
9.88494846457303	-0.855507790688027\\
9.88682363911627	-0.855507228589273\\
9.88869881327539	-0.855506666763447\\
9.89057398705057	-0.855506105210371\\
9.89244916044199	-0.855505543929868\\
9.89432433344985	-0.855504982921761\\
9.89619950607432	-0.855504422185874\\
9.89807467831559	-0.855503861722028\\
9.89994985017384	-0.855503301530047\\
9.90182502164925	-0.855502741609755\\
9.903700192742	-0.855502181960975\\
9.90557536345229	-0.855501622583532\\
9.90745053378028	-0.855501063477248\\
9.90932570372617	-0.855500504641948\\
9.91120087329014	-0.855499946077456\\
9.91307604247237	-0.855499387783596\\
9.91495121127303	-0.855498829760193\\
9.91682637969232	-0.855498272007071\\
9.91870154773042	-0.855497714524055\\
9.9205767153875	-0.85549715731097\\
9.92245188266375	-0.855496600367641\\
9.92432704955935	-0.855496043693893\\
9.92620221607449	-0.855495487289551\\
9.92807738220933	-0.855494931154441\\
9.92995254796408	-0.855494375288388\\
9.9318277133389	-0.855493819691217\\
9.93370287833397	-0.855493264362756\\
9.93557804294949	-0.855492709302829\\
9.93745320718562	-0.855492154511263\\
9.93932837104256	-0.855491599987885\\
9.94120353452048	-0.85549104573252\\
9.94307869761955	-0.855490491744995\\
9.94495386033997	-0.855489938025137\\
9.94682902268191	-0.855489384572773\\
9.94870418464556	-0.855488831387729\\
9.95057934623108	-0.855488278469833\\
9.95245450743867	-0.855487725818912\\
9.9543296682685	-0.855487173434794\\
9.95620482872075	-0.855486621317305\\
9.9580799887956	-0.855486069466274\\
9.95995514849323	-0.855485517881529\\
9.96183030781383	-0.855484966562897\\
9.96370546675756	-0.855484415510206\\
9.96558062532461	-0.855483864723286\\
9.96745578351516	-0.855483314201963\\
9.96933094132939	-0.855482763946067\\
9.97120609876747	-0.855482213955426\\
9.97308125582959	-0.85548166422987\\
9.97495641251592	-0.855481114769226\\
9.97683156882664	-0.855480565573325\\
9.97870672476193	-0.855480016641994\\
9.98058188032197	-0.855479467975065\\
9.98245703550694	-0.855478919572365\\
9.98433219031701	-0.855478371433726\\
9.98620734475236	-0.855477823558976\\
9.98808249881317	-0.855477275947945\\
9.98995765249962	-0.855476728600464\\
9.99183280581188	-0.855476181516362\\
9.99370795875014	-0.85547563469547\\
9.99558311131457	-0.855475088137619\\
9.99745826350534	-0.855474541842638\\
9.99933341532264	-0.85547399581036\\
10.0012085667666	-0.855473450040613\\
10.0030837178375	-0.855472904533231\\
10.0049588685355	-0.855472359288043\\
10.0068340188606	-0.855471814304881\\
10.0087091688132	-0.855471269583576\\
10.0105843183934	-0.85547072512396\\
10.0124594676013	-0.855470180925865\\
10.0143346164371	-0.855469636989122\\
10.0162097649011	-0.855469093313564\\
10.0180849129934	-0.855468549899022\\
10.0199600607141	-0.855468006745328\\
10.0218352080635	-0.855467463852316\\
10.0237103550417	-0.855466921219817\\
10.0255855016488	-0.855466378847664\\
10.0274606478852	-0.85546583673569\\
10.0293357937509	-0.855465294883728\\
10.0312109392461	-0.85546475329161\\
10.033086084371	-0.855464211959171\\
10.0349612291258	-0.855463670886243\\
10.0368363735106	-0.855463130072659\\
10.0387115175256	-0.855462589518254\\
10.040586661171	-0.855462049222861\\
10.042461804447	-0.855461509186313\\
10.0443369473538	-0.855460969408445\\
10.0462120898914	-0.855460429889091\\
10.0480872320602	-0.855459890628086\\
10.0499623738602	-0.855459351625262\\
10.0518375152916	-0.855458812880456\\
10.0537126563547	-0.855458274393501\\
10.0555877970495	-0.855457736164233\\
10.0574629373763	-0.855457198192486\\
10.0593380773353	-0.855456660478095\\
10.0612132169265	-0.855456123020896\\
10.0630883561503	-0.855455585820724\\
10.0649634950067	-0.855455048877414\\
10.0668386334959	-0.855454512190801\\
10.0687137716181	-0.855453975760722\\
10.0705889093735	-0.855453439587013\\
10.0724640467622	-0.855452903669508\\
10.0743391837845	-0.855452368008045\\
10.0762143204404	-0.85545183260246\\
10.0780894567302	-0.855451297452589\\
10.0799645926541	-0.855450762558268\\
10.0818397282121	-0.855450227919334\\
10.0837148634046	-0.855449693535624\\
10.0855899982316	-0.855449159406974\\
10.0874651326933	-0.855448625533222\\
10.0893402667899	-0.855448091914206\\
10.0912154005216	-0.855447558549761\\
10.0930905338885	-0.855447025439726\\
10.0949656668908	-0.855446492583937\\
10.0968407995288	-0.855445959982234\\
10.0987159318024	-0.855445427634453\\
10.100591063712	-0.855444895540432\\
10.1024661952577	-0.85544436370001\\
10.1043413264397	-0.855443832113025\\
10.1062164572581	-0.855443300779315\\
10.1080915877131	-0.855442769698718\\
10.1099667178049	-0.855442238871073\\
10.1118418475336	-0.855441708296218\\
10.1137169768995	-0.855441177973994\\
10.1155921059027	-0.855440647904237\\
10.1174672345433	-0.855440118086789\\
10.1193423628215	-0.855439588521487\\
10.1212174907376	-0.855439059208171\\
10.1230926182916	-0.855438530146681\\
10.1249677454838	-0.855438001336856\\
10.1268428723143	-0.855437472778536\\
10.1287179987832	-0.85543694447156\\
10.1305931248908	-0.85543641641577\\
10.1324682506373	-0.855435888611004\\
10.1343433760227	-0.855435361057104\\
10.1362185010473	-0.855434833753909\\
10.1380936257112	-0.85543430670126\\
10.1399687500146	-0.855433779898998\\
10.1418438739576	-0.855433253346963\\
10.1437189975406	-0.855432727044996\\
10.1455941207635	-0.855432200992939\\
10.1474692436265	-0.855431675190632\\
10.14934436613	-0.855431149637917\\
10.1512194882739	-0.855430624334635\\
10.1530946100585	-0.855430099280627\\
10.1549697314839	-0.855429574475735\\
10.1568448525504	-0.855429049919802\\
10.158719973258	-0.855428525612668\\
10.160595093607	-0.855428001554176\\
10.1624702135975	-0.855427477744168\\
10.1643453332297	-0.855426954182487\\
10.1662204525038	-0.855426430868974\\
10.1680955714198	-0.855425907803472\\
10.1699706899781	-0.855425384985824\\
10.1718458081787	-0.855424862415873\\
10.1737209260218	-0.855424340093462\\
10.1755960435076	-0.855423818018432\\
10.1774711606363	-0.855423296190629\\
10.179346277408	-0.855422774609894\\
10.1812213938229	-0.855422253276072\\
10.1830965098811	-0.855421732189006\\
10.1849716255828	-0.855421211348539\\
10.1868467409283	-0.855420690754515\\
10.1887218559176	-0.855420170406779\\
10.1905969705509	-0.855419650305174\\
10.1924720848284	-0.855419130449544\\
10.1943471987502	-0.855418610839734\\
10.1962223123166	-0.855418091475587\\
10.1980974255276	-0.855417572356949\\
10.1999725383835	-0.855417053483665\\
10.2018476508844	-0.855416534855578\\
10.2037227630305	-0.855416016472534\\
10.205597874822	-0.855415498334378\\
10.2074729862589	-0.855414980440954\\
10.2093480973415	-0.855414462792109\\
10.21122320807	-0.855413945387687\\
10.2130983184445	-0.855413428227535\\
10.2149734284651	-0.855412911311497\\
10.2168485381321	-0.855412394639419\\
10.2187236474456	-0.855411878211148\\
10.2205987564057	-0.855411362026529\\
10.2224738650127	-0.855410846085408\\
10.2243489732666	-0.855410330387632\\
10.2262240811677	-0.855409814933046\\
10.2280991887162	-0.855409299721499\\
10.2299742959121	-0.855408784752835\\
10.2318494027557	-0.855408270026902\\
10.2337245092471	-0.855407755543546\\
10.2355996153864	-0.855407241302615\\
10.2374747211739	-0.855406727303956\\
10.2393498266097	-0.855406213547415\\
10.241224931694	-0.85540570003284\\
10.2431000364269	-0.855405186760079\\
10.2449751408086	-0.85540467372898\\
10.2468502448393	-0.855404160939389\\
10.248725348519	-0.855403648391154\\
10.2506004518481	-0.855403136084124\\
10.2524755548266	-0.855402624018147\\
10.2543506574547	-0.855402112193071\\
10.2562257597326	-0.855401600608743\\
10.2581008616604	-0.855401089265014\\
10.2599759632383	-0.85540057816173\\
10.2618510644665	-0.85540006729874\\
10.2637261653451	-0.855399556675894\\
10.2656012658742	-0.855399046293041\\
10.2674763660541	-0.855398536150028\\
10.269351465885	-0.855398026246706\\
10.2712265653669	-0.855397516582924\\
10.2731016645	-0.855397007158531\\
10.2749767632845	-0.855396497973376\\
10.2768518617205	-0.855395989027309\\
10.2787269598083	-0.855395480320181\\
10.280602057548	-0.855394971851839\\
10.2824771549397	-0.855394463622136\\
10.2843522519835	-0.85539395563092\\
10.2862273486798	-0.855393447878042\\
10.2881024450286	-0.855392940363352\\
10.28997754103	-0.8553924330867\\
10.2918526366843	-0.855391926047938\\
10.2937277319916	-0.855391419246916\\
10.295602826952	-0.855390912683484\\
10.2974779215658	-0.855390406357494\\
10.299353015833	-0.855389900268796\\
10.3012281097539	-0.855389394417242\\
10.3031032033286	-0.855388888802683\\
10.3049782965573	-0.85538838342497\\
10.3068533894401	-0.855387878283955\\
10.3087284819771	-0.855387373379489\\
10.3106035741686	-0.855386868711424\\
10.3124786660148	-0.855386364279613\\
10.3143537575156	-0.855385860083906\\
10.3162288486714	-0.855385356124156\\
10.3181039394823	-0.855384852400215\\
10.3199790299484	-0.855384348911935\\
10.3218541200699	-0.855383845659169\\
10.323729209847	-0.85538334264177\\
10.3256042992798	-0.855382839859589\\
10.3274793883684	-0.855382337312481\\
10.3293544771131	-0.855381835000297\\
10.331229565514	-0.85538133292289\\
10.3331046535713	-0.855380831080115\\
10.334979741285	-0.855380329471823\\
10.3368548286554	-0.855379828097869\\
10.3387299156827	-0.855379326958105\\
10.3406050023669	-0.855378826052386\\
10.3424800887083	-0.855378325380565\\
10.3443551747069	-0.855377824942495\\
10.3462302603631	-0.855377324738032\\
10.3481053456768	-0.855376824767028\\
10.3499804306483	-0.855376325029338\\
10.3518555152778	-0.855375825524816\\
10.3537305995653	-0.855375326253317\\
10.3556056835111	-0.855374827214695\\
10.3574807671153	-0.855374328408805\\
10.359355850378	-0.8553738298355\\
10.3612309332995	-0.855373331494638\\
10.3631060158798	-0.855372833386071\\
10.3649810981192	-0.855372335509655\\
10.3668561800178	-0.855371837865246\\
10.3687312615757	-0.855371340452698\\
10.3706063427931	-0.855370843271867\\
10.3724814236701	-0.855370346322609\\
10.374356504207	-0.855369849604779\\
10.3762315844039	-0.855369353118233\\
10.3781066642608	-0.855368856862827\\
10.3799817437781	-0.855368360838417\\
10.3818568229558	-0.855367865044858\\
10.3837319017941	-0.855367369482007\\
10.3856069802931	-0.855366874149721\\
10.3874820584531	-0.855366379047855\\
10.3893571362741	-0.855365884176266\\
10.3912322137563	-0.855365389534811\\
10.3931072908999	-0.855364895123347\\
10.394982367705	-0.85536440094173\\
10.3968574441719	-0.855363906989818\\
10.3987325203005	-0.855363413267467\\
10.4006075960912	-0.855362919774535\\
10.402482671544	-0.855362426510879\\
10.4043577466591	-0.855361933476356\\
10.4062328214366	-0.855361440670824\\
10.4081078958768	-0.855360948094141\\
10.4099829699797	-0.855360455746164\\
10.4118580437456	-0.855359963626751\\
10.4137331171745	-0.855359471735761\\
10.4156081902667	-0.855358980073051\\
10.4174832630222	-0.855358488638479\\
10.4193583354413	-0.855357997431904\\
10.4212334075241	-0.855357506453184\\
10.4231084792708	-0.855357015702178\\
10.4249835506814	-0.855356525178744\\
10.4268586217562	-0.855356034882741\\
10.4287336924953	-0.855355544814028\\
10.4306087628988	-0.855355054972464\\
10.432483832967	-0.855354565357908\\
10.4343589027	-0.855354075970219\\
10.4362339720978	-0.855353586809257\\
10.4381090411608	-0.85535309787488\\
10.439984109889	-0.855352609166949\\
10.4418591782825	-0.855352120685323\\
10.4437342463416	-0.855351632429861\\
10.4456093140664	-0.855351144400424\\
10.447484381457	-0.855350656596872\\
10.4493594485136	-0.855350169019064\\
10.4512345152363	-0.855349681666861\\
10.4531095816254	-0.855349194540123\\
10.4549846476809	-0.85534870763871\\
10.456859713403	-0.855348220962484\\
10.4587347787919	-0.855347734511303\\
10.4606098438476	-0.85534724828503\\
10.4624849085704	-0.855346762283525\\
10.4643599729605	-0.855346276506649\\
10.4662350370179	-0.855345790954263\\
10.4681101007428	-0.855345305626228\\
10.4699851641354	-0.855344820522405\\
10.4718602271957	-0.855344335642656\\
10.4737352899241	-0.855343850986843\\
10.4756103523206	-0.855343366554826\\
10.4774854143853	-0.855342882346467\\
10.4793604761185	-0.855342398361629\\
10.4812355375202	-0.855341914600173\\
10.4831105985907	-0.855341431061961\\
10.4849856593301	-0.855340947746855\\
10.4868607197384	-0.855340464654718\\
10.488735779816	-0.855339981785411\\
10.4906108395628	-0.855339499138798\\
10.4924858989792	-0.855339016714741\\
10.4943609580652	-0.855338534513101\\
10.4962360168209	-0.855338052533744\\
10.4981110752466	-0.85533757077653\\
10.4999861333424	-0.855337089241322\\
10.5018611911084	-0.855336607927985\\
10.5037362485447	-0.855336126836382\\
10.5056113056516	-0.855335645966374\\
10.5074863624292	-0.855335165317827\\
10.5093614188776	-0.855334684890603\\
10.511236474997	-0.855334204684565\\
10.5131115307875	-0.855333724699579\\
10.5149865862493	-0.855333244935507\\
10.5168616413825	-0.855332765392213\\
10.5187366961873	-0.855332286069562\\
10.5206117506638	-0.855331806967417\\
10.5224868048122	-0.855331328085643\\
10.5243618586326	-0.855330849424104\\
10.5262369121252	-0.855330370982664\\
10.5281119652901	-0.855329892761188\\
10.5299870181274	-0.855329414759542\\
10.5318620706374	-0.855328936977588\\
10.5337371228202	-0.855328459415193\\
10.5356121746758	-0.855327982072221\\
10.5374872262046	-0.855327504948537\\
10.5393622774065	-0.855327028044007\\
10.5412373282818	-0.855326551358496\\
10.5431123788306	-0.855326074891869\\
10.5449874290531	-0.855325598643991\\
10.5468624789493	-0.855325122614728\\
10.5487375285195	-0.855324646803947\\
10.5506125777638	-0.855324171211512\\
10.5524876266824	-0.85532369583729\\
10.5543626752753	-0.855323220681146\\
10.5562377235428	-0.855322745742947\\
10.558112771485	-0.855322271022559\\
10.559987819102	-0.855321796519848\\
10.561862866394	-0.85532132223468\\
10.5637379133611	-0.855320848166923\\
10.5656129600035	-0.855320374316442\\
10.5674880063214	-0.855319900683105\\
10.5693630523148	-0.855319427266778\\
10.5712380979839	-0.855318954067328\\
10.5731131433289	-0.855318481084622\\
10.5749881883498	-0.855318008318528\\
10.576863233047	-0.855317535768912\\
10.5787382774204	-0.855317063435642\\
10.5806133214703	-0.855316591318585\\
10.5824883651968	-0.85531611941761\\
10.5843634086001	-0.855315647732582\\
10.5862384516802	-0.855315176263371\\
10.5881134944373	-0.855314705009844\\
10.5899885368717	-0.855314233971869\\
10.5918635789833	-0.855313763149314\\
10.5937386207725	-0.855313292542048\\
10.5956136622392	-0.855312822149937\\
10.5974887033837	-0.855312351972852\\
10.5993637442061	-0.85531188201066\\
10.6012387847066	-0.855311412263229\\
10.6031138248853	-0.855310942730429\\
10.6049888647423	-0.855310473412129\\
10.6068639042778	-0.855310004308196\\
10.6087389434919	-0.8553095354185\\
10.6106139823848	-0.855309066742911\\
10.6124890209567	-0.855308598281296\\
10.6143640592076	-0.855308130033526\\
10.6162390971377	-0.85530766199947\\
10.6181141347471	-0.855307194178997\\
10.6199891720361	-0.855306726571977\\
10.6218642090047	-0.855306259178279\\
10.6237392456531	-0.855305791997774\\
10.6256142819815	-0.855305325030331\\
10.6274893179899	-0.855304858275819\\
10.6293643536785	-0.85530439173411\\
10.6312393890475	-0.855303925405074\\
10.6331144240971	-0.855303459288579\\
10.6349894588272	-0.855302993384498\\
10.6368644932382	-0.855302527692699\\
10.6387395273301	-0.855302062213055\\
10.6406145611031	-0.855301596945435\\
10.6424895945573	-0.85530113188971\\
10.6443646276929	-0.855300667045752\\
10.64623966051	-0.855300202413431\\
10.6481146930088	-0.855299737992617\\
10.6499897251894	-0.855299273783183\\
10.6518647570519	-0.855298809785\\
10.6537397885965	-0.855298345997938\\
10.6556148198233	-0.855297882421869\\
10.6574898507325	-0.855297419056665\\
10.6593648813242	-0.855296955902198\\
10.6612399115986	-0.855296492958339\\
10.6631149415557	-0.85529603022496\\
10.6649899711959	-0.855295567701932\\
10.6668650005191	-0.855295105389129\\
10.6687400295255	-0.855294643286422\\
10.6706150582153	-0.855294181393683\\
10.6724900865886	-0.855293719710784\\
10.6743651146455	-0.855293258237599\\
10.6762401423863	-0.855292796973999\\
10.678115169811	-0.855292335919858\\
10.6799901969197	-0.855291875075047\\
10.6818652237127	-0.85529141443944\\
10.6837402501901	-0.855290954012909\\
10.6856152763519	-0.855290493795328\\
10.6874903021984	-0.85529003378657\\
10.6893653277297	-0.855289573986507\\
10.6912403529459	-0.855289114395014\\
10.6931153778472	-0.855288655011963\\
10.6949904024336	-0.855288195837228\\
10.6968654267055	-0.855287736870682\\
10.6987404506628	-0.8552872781122\\
10.7006154743058	-0.855286819561655\\
10.7024904976345	-0.855286361218921\\
10.7043655206491	-0.855285903083872\\
10.7062405433498	-0.855285445156382\\
10.7081155657367	-0.855284987436324\\
10.7099905878099	-0.855284529923574\\
10.7118656095696	-0.855284072618006\\
10.7137406310159	-0.855283615519494\\
10.715615652149	-0.855283158627912\\
10.7174906729689	-0.855282701943136\\
10.7193656934759	-0.85528224546504\\
10.7212407136701	-0.855281789193498\\
10.7231157335516	-0.855281333128387\\
10.7249907531205	-0.85528087726958\\
10.726865772377	-0.855280421616953\\
10.7287407913213	-0.85527996617038\\
10.7306158099534	-0.855279510929739\\
10.7324908282736	-0.855279055894903\\
10.7343658462819	-0.855278601065748\\
10.7362408639785	-0.85527814644215\\
10.7381158813635	-0.855277692023985\\
10.7399908984371	-0.855277237811128\\
10.7418659151994	-0.855276783803455\\
10.7437409316505	-0.855276330000842\\
10.7456159477906	-0.855275876403166\\
10.7474909636199	-0.855275423010301\\
10.7493659791384	-0.855274969822125\\
10.7512409943463	-0.855274516838514\\
10.7531160092437	-0.855274064059345\\
10.7549910238309	-0.855273611484493\\
10.7568660381078	-0.855273159113835\\
10.7587410520747	-0.855272706947249\\
10.7606160657317	-0.855272254984611\\
10.7624910790789	-0.855271803225797\\
10.7643660921165	-0.855271351670685\\
10.7662411048446	-0.855270900319152\\
10.7681161172633	-0.855270449171076\\
10.7699911293729	-0.855269998226332\\
10.7718661411733	-0.8552695474848\\
10.7737411526648	-0.855269096946355\\
10.7756161638475	-0.855268646610876\\
10.7774911747215	-0.855268196478241\\
10.779366185287	-0.855267746548327\\
10.7812411955441	-0.855267296821011\\
10.7831162054929	-0.855266847296173\\
10.7849912151337	-0.855266397973689\\
10.7868662244664	-0.855265948853438\\
10.7887412334913	-0.855265499935298\\
10.7906162422085	-0.855265051219148\\
10.7924912506181	-0.855264602704865\\
10.7943662587203	-0.855264154392329\\
10.7962412665151	-0.855263706281417\\
10.7981162740029	-0.855263258372009\\
10.7999912811835	-0.855262810663983\\
10.8018662880573	-0.855262363157219\\
10.8037412946244	-0.855261915851594\\
10.8056163008848	-0.855261468746988\\
10.8074913068388	-0.85526102184328\\
10.8093663124864	-0.85526057514035\\
10.8112413178278	-0.855260128638076\\
10.8131163228631	-0.855259682336338\\
10.8149913275925	-0.855259236235016\\
10.816866332016	-0.855258790333988\\
10.8187413361339	-0.855258344633135\\
10.8206163399463	-0.855257899132337\\
10.8224913434533	-0.855257453831473\\
10.824366346655	-0.855257008730423\\
10.8262413495516	-0.855256563829067\\
10.8281163521433	-0.855256119127286\\
10.82999135443	-0.855255674624959\\
10.8318663564121	-0.855255230321967\\
10.8337413580895	-0.85525478621819\\
10.8356163594625	-0.855254342313509\\
10.8374913605312	-0.855253898607803\\
10.8393663612957	-0.855253455100955\\
10.8412413617562	-0.855253011792844\\
10.8431163619127	-0.855252568683352\\
10.8449913617655	-0.855252125772359\\
10.8468663613146	-0.855251683059746\\
10.8487413605603	-0.855251240545395\\
10.8506163595025	-0.855250798229186\\
10.8524913581415	-0.855250356111002\\
10.8543663564774	-0.855249914190722\\
10.8562413545104	-0.855249472468229\\
10.8581163522405	-0.855249030943405\\
10.8599913496679	-0.85524858961613\\
10.8618663467927	-0.855248148486287\\
10.8637413436151	-0.855247707553758\\
10.8656163401352	-0.855247266818423\\
10.8674913363531	-0.855246826280166\\
10.869366332269	-0.855246385938869\\
10.871241327883	-0.855245945794413\\
10.8731163231952	-0.85524550584668\\
10.8749913182058	-0.855245066095554\\
10.8768663129149	-0.855244626540917\\
10.8787413073226	-0.85524418718265\\
10.880616301429	-0.855243748020638\\
10.8824912952344	-0.855243309054761\\
10.8843662887388	-0.855242870284904\\
10.8862412819424	-0.855242431710949\\
10.8881162748452	-0.855241993332779\\
10.8899912674475	-0.855241555150276\\
10.8918662597494	-0.855241117163325\\
10.8937412517509	-0.855240679371809\\
10.8956162434523	-0.85524024177561\\
10.8974912348536	-0.855239804374612\\
10.899366225955	-0.855239367168699\\
10.9012412167566	-0.855238930157754\\
10.9031162072586	-0.85523849334166\\
10.9049911974611	-0.855238056720303\\
10.9068661873642	-0.855237620293564\\
10.908741176968	-0.855237184061329\\
10.9106161662727	-0.855236748023482\\
10.9124911552785	-0.855236312179905\\
10.9143661439854	-0.855235876530485\\
10.9162411323935	-0.855235441075104\\
10.9181161205031	-0.855235005813647\\
10.9199911083142	-0.855234570745999\\
10.9218660958269	-0.855234135872044\\
10.9237410830415	-0.855233701191667\\
10.925616069958	-0.855233266704752\\
10.9274910565766	-0.855232832411184\\
10.9293660428973	-0.855232398310849\\
10.9312410289204	-0.855231964403631\\
10.933116014646	-0.855231530689414\\
10.9349910000741	-0.855231097168086\\
10.9368659852049	-0.855230663839529\\
10.9387409700386	-0.85523023070363\\
10.9406159545753	-0.855229797760275\\
10.9424909388151	-0.855229365009348\\
10.9443659227581	-0.855228932450735\\
10.9462409064045	-0.855228500084322\\
10.9481158897544	-0.855228067909994\\
10.9499908728079	-0.855227635927638\\
10.9518658555652	-0.855227204137139\\
10.9537408380264	-0.855226772538383\\
10.9556158201916	-0.855226341131256\\
10.9574908020609	-0.855225909915645\\
10.9593657836345	-0.855225478891436\\
10.9612407649126	-0.855225048058514\\
10.9631157458951	-0.855224617416767\\
10.9649907265824	-0.855224186966081\\
10.9668657069744	-0.855223756706342\\
10.9687406870714	-0.855223326637438\\
10.9706156668734	-0.855222896759254\\
10.9724906463806	-0.855222467071678\\
10.9743656255931	-0.855222037574596\\
10.9762406045111	-0.855221608267897\\
10.9781155831346	-0.855221179151465\\
10.9799905614638	-0.85522075022519\\
10.9818655394989	-0.855220321488958\\
10.9837405172399	-0.855219892942656\\
10.9856154946871	-0.855219464586172\\
10.9874904718404	-0.855219036419393\\
10.9893654487001	-0.855218608442207\\
10.9912404252662	-0.855218180654502\\
10.993115401539	-0.855217753056165\\
10.9949903775185	-0.855217325647084\\
10.9968653532048	-0.855216898427148\\
10.9987403285981	-0.855216471396243\\
11.0006153036986	-0.855216044554259\\
11.0024902785063	-0.855215617901083\\
11.0043652530213	-0.855215191436603\\
11.0062402272439	-0.855214765160708\\
11.0081152011741	-0.855214339073287\\
11.009990174812	-0.855213913174228\\
11.0118651481579	-0.855213487463419\\
11.0137401212117	-0.855213061940749\\
11.0156150939737	-0.855212636606107\\
11.0174900664439	-0.855212211459381\\
11.0193650386225	-0.855211786500461\\
11.0212400105097	-0.855211361729236\\
11.0231149821055	-0.855210937145594\\
11.02498995341	-0.855210512749425\\
11.0268649244235	-0.855210088540618\\
11.028739895146	-0.855209664519063\\
11.0306148655776	-0.855209240684648\\
11.0324898357186	-0.855208817037263\\
11.0343648055689	-0.855208393576798\\
11.0362397751288	-0.855207970303143\\
11.0381147443983	-0.855207547216187\\
11.0399897133776	-0.85520712431582\\
11.0418646820668	-0.855206701601932\\
11.0437396504661	-0.855206279074412\\
11.0456146185756	-0.855205856733152\\
11.0474895863953	-0.855205434578041\\
11.0493645539255	-0.855205012608969\\
11.0512395211662	-0.855204590825826\\
11.0531144881175	-0.855204169228504\\
11.0549894547797	-0.855203747816892\\
11.0568644211528	-0.855203326590881\\
11.0587393872369	-0.855202905550361\\
11.0606143530323	-0.855202484695224\\
11.0624893185389	-0.85520206402536\\
11.0643642837569	-0.85520164354066\\
11.0662392486865	-0.855201223241015\\
11.0681142133278	-0.855200803126316\\
11.0699891776809	-0.855200383196454\\
11.0718641417459	-0.85519996345132\\
11.0737391055229	-0.855199543890805\\
11.0756140690122	-0.855199124514802\\
11.0774890322137	-0.855198705323201\\
11.0793639951277	-0.855198286315894\\
11.0812389577542	-0.855197867492772\\
11.0831139200934	-0.855197448853727\\
11.0849888821454	-0.855197030398651\\
11.0868638439103	-0.855196612127435\\
11.0887388053883	-0.855196194039973\\
11.0906137665795	-0.855195776136154\\
11.0924887274839	-0.855195358415873\\
11.0943636881018	-0.85519494087902\\
11.0962386484333	-0.855194523525488\\
11.0981136084784	-0.85519410635517\\
11.0999885682373	-0.855193689367957\\
11.1018635277101	-0.855193272563742\\
11.103738486897	-0.855192855942419\\
11.105613445798	-0.855192439503878\\
11.1074884044134	-0.855192023248014\\
11.1093633627431	-0.855191607174718\\
11.1112383207875	-0.855191191283884\\
11.1131132785464	-0.855190775575405\\
11.1149882360202	-0.855190360049173\\
11.1168631932089	-0.855189944705082\\
11.1187381501126	-0.855189529543025\\
11.1206131067314	-0.855189114562896\\
11.1224880630656	-0.855188699764586\\
11.1243630191152	-0.855188285147991\\
11.1262379748803	-0.855187870713003\\
11.128112930361	-0.855187456459516\\
11.1299878855575	-0.855187042387424\\
11.1318628404699	-0.85518662849662\\
11.1337377950984	-0.855186214786999\\
11.135612749443	-0.855185801258453\\
11.1374877035038	-0.855185387910877\\
11.1393626572811	-0.855184974744166\\
11.1412376107748	-0.855184561758213\\
11.1431125639852	-0.855184148952912\\
11.1449875169124	-0.855183736328157\\
11.1468624695564	-0.855183323883844\\
11.1487374219175	-0.855182911619866\\
11.1506123739956	-0.855182499536118\\
11.152487325791	-0.855182087632494\\
11.1543622773038	-0.85518167590889\\
11.1562372285341	-0.8551812643652\\
11.158112179482	-0.855180853001318\\
11.1599871301476	-0.85518044181714\\
11.1618620805311	-0.85518003081256\\
11.1637370306325	-0.855179619987474\\
11.1656119804521	-0.855179209341777\\
11.1674869299899	-0.855178798875364\\
11.169361879246	-0.85517838858813\\
11.1712368282206	-0.85517797847997\\
11.1731117769138	-0.855177568550781\\
11.1749867253257	-0.855177158800457\\
11.1768616734564	-0.855176749228894\\
11.1787366213061	-0.855176339835988\\
11.1806115688748	-0.855175930621634\\
11.1824865161627	-0.855175521585728\\
11.18436146317	-0.855175112728167\\
11.1862364098967	-0.855174704048846\\
11.1881113563429	-0.855174295547661\\
11.1899863025088	-0.855173887224508\\
11.1918612483946	-0.855173479079284\\
11.1937361940002	-0.855173071111884\\
11.1956111393259	-0.855172663322206\\
11.1974860843717	-0.855172255710145\\
11.1993610291378	-0.855171848275598\\
11.2012359736244	-0.855171441018462\\
11.2031109178314	-0.855171033938633\\
11.2049858617591	-0.855170627036008\\
11.2068608054075	-0.855170220310483\\
11.2087357487768	-0.855169813761956\\
11.2106106918672	-0.855169407390324\\
11.2124856346786	-0.855169001195483\\
11.2143605772113	-0.855168595177331\\
11.2162355194654	-0.855168189335765\\
11.2181104614409	-0.855167783670682\\
11.219985403138	-0.855167378181979\\
11.2218603445569	-0.855166972869554\\
11.2237352856976	-0.855166567733305\\
11.2256102265602	-0.855166162773128\\
11.227485167145	-0.855165757988922\\
11.2293601074519	-0.855165353380584\\
11.2312350474811	-0.855164948948012\\
11.2331099872328	-0.855164544691103\\
11.2349849267071	-0.855164140609756\\
11.236859865904	-0.855163736703869\\
11.2387348048237	-0.85516333297334\\
11.2406097434663	-0.855162929418067\\
11.242484681832	-0.855162526037948\\
11.2443596199208	-0.855162122832881\\
11.2462345577329	-0.855161719802765\\
11.2481094952684	-0.855161316947498\\
11.2499844325273	-0.855160914266979\\
11.2518593695099	-0.855160511761106\\
11.2537343062163	-0.855160109429778\\
11.2556092426465	-0.855159707272894\\
11.2574841788007	-0.855159305290353\\
11.259359114679	-0.855158903482052\\
11.2612340502815	-0.855158501847893\\
11.2631089856083	-0.855158100387772\\
11.2649839206596	-0.85515769910159\\
11.2668588554355	-0.855157297989246\\
11.268733789936	-0.855156897050639\\
11.2706087241614	-0.855156496285668\\
11.2724836581117	-0.855156095694233\\
11.274358591787	-0.855155695276232\\
11.2762335251875	-0.855155295031567\\
11.2781084583133	-0.855154894960135\\
11.2799833911645	-0.855154495061838\\
11.2818583237412	-0.855154095336574\\
11.2837332560435	-0.855153695784244\\
11.2856081880715	-0.855153296404748\\
11.2874831198255	-0.855152897197984\\
11.2893580513054	-0.855152498163855\\
11.2912329825114	-0.855152099302259\\
11.2931079134436	-0.855151700613097\\
11.2949828441022	-0.855151302096268\\
11.2968577744872	-0.855150903751675\\
11.2987327045988	-0.855150505579216\\
11.300607634437	-0.855150107578792\\
11.3024825640021	-0.855149709750305\\
11.3043574932941	-0.855149312093654\\
11.3062324223131	-0.85514891460874\\
11.3081073510593	-0.855148517295464\\
11.3099822795327	-0.855148120153727\\
11.3118572077335	-0.855147723183429\\
11.3137321356619	-0.855147326384473\\
11.3156070633178	-0.855146929756758\\
11.3174819907015	-0.855146533300186\\
11.319356917813	-0.855146137014659\\
11.3212318446525	-0.855145740900076\\
11.32310677122	-0.855145344956341\\
11.3249816975158	-0.855144949183354\\
11.3268566235399	-0.855144553581016\\
11.3287315492924	-0.85514415814923\\
11.3306064747734	-0.855143762887896\\
11.3324813999832	-0.855143367796917\\
11.3343563249216	-0.855142972876195\\
11.336231249589	-0.855142578125631\\
11.3381061739854	-0.855142183545127\\
11.3399810981109	-0.855141789134585\\
11.3418560219657	-0.855141394893907\\
11.3437309455498	-0.855141000822995\\
11.3456058688633	-0.855140606921752\\
11.3474807919065	-0.85514021319008\\
11.3493557146793	-0.855139819627881\\
11.351230637182	-0.855139426235057\\
11.3531055594146	-0.855139033011511\\
11.3549804813772	-0.855138639957146\\
11.35685540307	-0.855138247071864\\
11.3587303244931	-0.855137854355568\\
11.3606052456465	-0.855137461808161\\
11.3624801665305	-0.855137069429545\\
11.3643550871451	-0.855136677219624\\
11.3662300074904	-0.8551362851783\\
11.3681049275665	-0.855135893305476\\
11.3699798473736	-0.855135501601056\\
11.3718547669118	-0.855135110064942\\
11.3737296861812	-0.855134718697039\\
11.3756046051819	-0.855134327497248\\
11.3774795239139	-0.855133936465475\\
11.3793544423776	-0.855133545601621\\
11.3812293605728	-0.855133154905592\\
11.3831042784999	-0.855132764377289\\
11.3849791961588	-0.855132374016618\\
11.3868541135497	-0.855131983823481\\
11.3887290306727	-0.855131593797782\\
11.3906039475279	-0.855131203939426\\
11.3924788641154	-0.855130814248317\\
11.3943537804354	-0.855130424724358\\
11.3962286964879	-0.855130035367453\\
11.3981036122731	-0.855129646177506\\
11.3999785277911	-0.855129257154423\\
11.4018534430419	-0.855128868298107\\
11.4037283580258	-0.855128479608462\\
11.4056032727428	-0.855128091085393\\
11.407478187193	-0.855127702728804\\
11.4093531013766	-0.8551273145386\\
11.4112280152937	-0.855126926514686\\
11.4131029289443	-0.855126538656967\\
11.4149778423286	-0.855126150965346\\
11.4168527554467	-0.855125763439729\\
11.4187276682987	-0.855125376080021\\
11.4206025808847	-0.855124988886127\\
11.4224774932049	-0.855124601857952\\
11.4243524052593	-0.855124214995401\\
11.4262273170481	-0.855123828298379\\
11.4281022285714	-0.855123441766792\\
11.4299771398292	-0.855123055400544\\
11.4318520508218	-0.855122669199541\\
11.4337269615491	-0.855122283163689\\
11.4356018720114	-0.855121897292893\\
11.4374767822087	-0.855121511587059\\
11.4393516921412	-0.855121126046092\\
11.4412266018089	-0.855120740669898\\
11.443101511212	-0.855120355458383\\
11.4449764203506	-0.855119970411453\\
11.4468513292247	-0.855119585529013\\
11.4487262378346	-0.85511920081097\\
11.4506011461803	-0.85511881625723\\
11.4524760542619	-0.855118431867699\\
11.4543509620796	-0.855118047642283\\
11.4562258696334	-0.855117663580888\\
11.4581007769235	-0.855117279683421\\
11.4599756839499	-0.855116895949788\\
11.4618505907129	-0.855116512379896\\
11.4637254972124	-0.85511612897365\\
11.4656004034487	-0.855115745730958\\
11.4674753094217	-0.855115362651727\\
11.4693502151317	-0.855114979735862\\
11.4712251205788	-0.855114596983272\\
11.473100025763	-0.855114214393862\\
11.4749749306844	-0.85511383196754\\
11.4768498353432	-0.855113449704213\\
11.4787247397395	-0.855113067603787\\
11.4805996438735	-0.855112685666171\\
11.4824745477451	-0.85511230389127\\
11.4843494513545	-0.855111922278993\\
11.4862243547019	-0.855111540829246\\
11.4880992577873	-0.855111159541938\\
11.4899741606108	-0.855110778416975\\
11.4918490631726	-0.855110397454265\\
11.4937239654728	-0.855110016653716\\
11.4955988675114	-0.855109636015235\\
11.4974737692887	-0.855109255538731\\
11.4993486708046	-0.85510887522411\\
11.5012235720593	-0.855108495071281\\
11.503098473053	-0.855108115080152\\
11.5049733737857	-0.85510773525063\\
11.5068482742575	-0.855107355582624\\
11.5087231744686	-0.855106976076042\\
11.510598074419	-0.855106596730792\\
11.5124729741089	-0.855106217546782\\
11.5143478735384	-0.855105838523921\\
11.5162227727076	-0.855105459662117\\
11.5180976716165	-0.855105080961278\\
11.5199725702654	-0.855104702421313\\
11.5218474686543	-0.855104324042131\\
11.5237223667833	-0.85510394582364\\
11.5255972646525	-0.855103567765749\\
11.5274721622621	-0.855103189868367\\
11.5293470596121	-0.855102812131402\\
11.5312219567027	-0.855102434554763\\
11.5330968535339	-0.85510205713836\\
11.5349717501059	-0.855101679882102\\
11.5368466464189	-0.855101302785897\\
11.5387215424728	-0.855100925849654\\
11.5405964382678	-0.855100549073284\\
11.542471333804	-0.855100172456695\\
11.5443462290815	-0.855099795999797\\
11.5462211241005	-0.855099419702498\\
11.548096018861	-0.855099043564709\\
11.5499709133632	-0.855098667586339\\
11.5518458076071	-0.855098291767298\\
11.5537207015929	-0.855097916107496\\
11.5555955953206	-0.855097540606841\\
11.5574704887904	-0.855097165265244\\
11.5593453820024	-0.855096790082615\\
11.5612202749567	-0.855096415058864\\
11.5630951676534	-0.855096040193901\\
11.5649700600926	-0.855095665487635\\
11.5668449522745	-0.855095290939977\\
11.568719844199	-0.855094916550838\\
11.5705947358664	-0.855094542320126\\
11.5724696272767	-0.855094168247754\\
11.5743445184301	-0.855093794333631\\
11.5762194093267	-0.855093420577667\\
11.5780942999665	-0.855093046979773\\
11.5799691903496	-0.85509267353986\\
11.5818440804763	-0.855092300257838\\
11.5837189703465	-0.855091927133618\\
11.5855938599605	-0.855091554167111\\
11.5874687493182	-0.855091181358228\\
11.5893436384198	-0.855090808706879\\
11.5912185272655	-0.855090436212975\\
11.5930934158552	-0.855090063876428\\
11.5949683041892	-0.855089691697148\\
11.5968431922676	-0.855089319675047\\
11.5987180800904	-0.855088947810036\\
11.6005929676577	-0.855088576102026\\
11.6024678549697	-0.855088204550929\\
11.6043427420264	-0.855087833156655\\
11.606217628828	-0.855087461919117\\
11.6080925153746	-0.855087090838226\\
11.6099674016663	-0.855086719913893\\
11.6118422877031	-0.85508634914603\\
11.6137171734853	-0.855085978534549\\
11.6155920590128	-0.855085608079361\\
11.6174669442859	-0.855085237780379\\
11.6193418293046	-0.855084867637515\\
11.6212167140689	-0.855084497650679\\
11.6230915985792	-0.855084127819785\\
11.6249664828353	-0.855083758144744\\
11.6268413668375	-0.855083388625469\\
11.6287162505858	-0.855083019261871\\
11.6305911340804	-0.855082650053864\\
11.6324660173213	-0.855082281001359\\
11.6343409003087	-0.855081912104268\\
11.6362157830426	-0.855081543362505\\
11.6380906655233	-0.855081174775981\\
11.6399655477507	-0.85508080634461\\
11.641840429725	-0.855080438068303\\
11.6437153114462	-0.855080069946974\\
11.6455901929146	-0.855079701980536\\
11.6474650741301	-0.855079334168901\\
11.649339955093	-0.855078966511982\\
11.6512148358033	-0.855078599009692\\
11.6530897162611	-0.855078231661944\\
11.6549645964665	-0.855077864468651\\
11.6568394764196	-0.855077497429726\\
11.6587143561205	-0.855077130545083\\
11.6605892355694	-0.855076763814634\\
11.6624641147664	-0.855076397238294\\
11.6643389937114	-0.855076030815975\\
11.6662138724047	-0.855075664547591\\
11.6680887508464	-0.855075298433055\\
11.6699636290365	-0.855074932472281\\
11.6718385069752	-0.855074566665183\\
11.6737133846626	-0.855074201011674\\
11.6755882620987	-0.855073835511667\\
11.6774631392837	-0.855073470165078\\
11.6793380162177	-0.855073104971819\\
11.6812128929007	-0.855072739931805\\
11.683087769333	-0.855072375044949\\
11.6849626455145	-0.855072010311166\\
11.6868375214454	-0.855071645730369\\
11.6887123971259	-0.855071281302474\\
11.6905872725559	-0.855070917027393\\
11.6924621477356	-0.855070552905042\\
11.6943370226652	-0.855070188935334\\
11.6962118973446	-0.855069825118185\\
11.6980867717741	-0.855069461453508\\
11.6999616459537	-0.855069097941219\\
11.7018365198836	-0.855068734581231\\
11.7037113935637	-0.855068371373459\\
11.7055862669943	-0.855068008317819\\
11.7074611401754	-0.855067645414224\\
11.7093360131072	-0.85506728266259\\
11.7112108857897	-0.855066920062832\\
11.7130857582231	-0.855066557614864\\
11.7149606304074	-0.855066195318602\\
11.7168355023427	-0.85506583317396\\
11.7187103740292	-0.855065471180854\\
11.720585245467	-0.855065109339199\\
11.7224601166561	-0.85506474764891\\
11.7243349875967	-0.855064386109903\\
11.7262098582889	-0.855064024722092\\
11.7280847287327	-0.855063663485394\\
11.7299595989283	-0.855063302399723\\
11.7318344688758	-0.855062941464995\\
11.7337093385753	-0.855062580681127\\
11.7355842080268	-0.855062220048033\\
11.7374590772305	-0.855061859565629\\
11.7393339461865	-0.855061499233831\\
11.7412088148949	-0.855061139052556\\
11.7430836833558	-0.855060779021718\\
11.7449585515693	-0.855060419141233\\
11.7468334195354	-0.855060059411019\\
11.7487082872544	-0.85505969983099\\
11.7505831547263	-0.855059340401063\\
11.7524580219511	-0.855058981121155\\
11.7543328889291	-0.85505862199118\\
11.7562077556603	-0.855058263011057\\
11.7580826221448	-0.8550579041807\\
11.7599574883826	-0.855057545500027\\
11.761832354374	-0.855057186968955\\
11.763707220119	-0.855056828587398\\
11.7655820856178	-0.855056470355275\\
11.7674569508703	-0.855056112272501\\
11.7693318158768	-0.855055754338995\\
11.7712066806372	-0.855055396554671\\
11.7730815451518	-0.855055038919447\\
11.7749564094207	-0.855054681433241\\
11.7768312734438	-0.855054324095968\\
11.7787061372213	-0.855053966907547\\
11.7805810007534	-0.855053609867893\\
11.7824558640401	-0.855053252976925\\
11.7843307270816	-0.855052896234559\\
11.7862055898779	-0.855052539640713\\
11.7880804524291	-0.855052183195303\\
11.7899553147353	-0.855051826898248\\
11.7918301767966	-0.855051470749465\\
11.7937050386132	-0.85505111474887\\
11.7955799001852	-0.855050758896382\\
11.7974547615125	-0.855050403191919\\
11.7993296225954	-0.855050047635397\\
11.8012044834339	-0.855049692226736\\
11.8030793440282	-0.855049336965851\\
11.8049542043783	-0.855048981852662\\
11.8068290644843	-0.855048626887085\\
11.8087039243464	-0.85504827206904\\
11.8105787839646	-0.855047917398443\\
11.812453643339	-0.855047562875214\\
11.8143285024698	-0.85504720849927\\
11.816203361357	-0.855046854270529\\
11.8180782200008	-0.855046500188909\\
11.8199530784012	-0.85504614625433\\
11.8218279365583	-0.855045792466708\\
11.8237027944723	-0.855045438825963\\
11.8255776521432	-0.855045085332013\\
11.8274525095711	-0.855044731984776\\
11.8293273667562	-0.855044378784171\\
11.8312022236985	-0.855044025730117\\
11.8330770803982	-0.855043672822532\\
11.8349519368553	-0.855043320061335\\
11.8368267930699	-0.855042967446445\\
11.8387016490422	-0.855042614977781\\
11.8405765047722	-0.855042262655261\\
11.84245136026	-0.855041910478805\\
11.8443262155058	-0.855041558448331\\
11.8462010705095	-0.855041206563758\\
11.8480759252715	-0.855040854825007\\
11.8499507797916	-0.855040503231995\\
11.8518256340701	-0.855040151784642\\
11.853700488107	-0.855039800482868\\
11.8555753419025	-0.855039449326591\\
11.8574501954566	-0.855039098315732\\
11.8593250487694	-0.855038747450209\\
11.861199901841	-0.855038396729942\\
11.8630747546716	-0.855038046154851\\
11.8649496072611	-0.855037695724855\\
11.8668244596098	-0.855037345439874\\
11.8686993117177	-0.855036995299828\\
11.870574163585	-0.855036645304637\\
11.8724490152116	-0.85503629545422\\
11.8743238665978	-0.855035945748497\\
11.8761987177436	-0.855035596187388\\
11.8780735686491	-0.855035246770814\\
11.8799484193144	-0.855034897498694\\
11.8818232697396	-0.855034548370948\\
11.8836981199248	-0.855034199387497\\
11.8855729698701	-0.855033850548261\\
11.8874478195757	-0.85503350185316\\
11.8893226690415	-0.855033153302115\\
11.8911975182677	-0.855032804895046\\
11.8930723672545	-0.855032456631872\\
11.8949472160018	-0.855032108512516\\
11.8968220645098	-0.855031760536898\\
11.8986969127786	-0.855031412704937\\
11.9005717608083	-0.855031065016555\\
11.902446608599	-0.855030717471673\\
11.9043214561508	-0.855030370070211\\
11.9061963034637	-0.85503002281209\\
11.908071150538	-0.855029675697231\\
11.9099459973736	-0.855029328725555\\
11.9118208439706	-0.855028981896983\\
11.9136956903293	-0.855028635211436\\
11.9155705364496	-0.855028288668836\\
11.9174453823317	-0.855027942269102\\
11.9193202279757	-0.855027596012158\\
11.9211950733816	-0.855027249897923\\
11.9230699185495	-0.855026903926319\\
11.9249447634797	-0.855026558097268\\
11.926819608172	-0.855026212410691\\
11.9286944526267	-0.855025866866509\\
11.9305692968439	-0.855025521464645\\
11.9324441408236	-0.855025176205019\\
11.934318984566	-0.855024831087553\\
11.9361938280711	-0.85502448611217\\
11.938068671339	-0.85502414127879\\
11.9399435143698	-0.855023796587336\\
11.9418183571637	-0.855023452037729\\
11.9436931997207	-0.855023107629892\\
11.9455680420409	-0.855022763363747\\
11.9474428841244	-0.855022419239215\\
11.9493177259714	-0.855022075256218\\
11.9511925675818	-0.855021731414679\\
11.9530674089559	-0.85502138771452\\
11.9549422500936	-0.855021044155663\\
11.9568170909952	-0.855020700738031\\
11.9586919316606	-0.855020357461545\\
11.96056677209	-0.855020014326129\\
11.9624416122835	-0.855019671331705\\
11.9643164522412	-0.855019328478194\\
11.9661912919632	-0.855018985765521\\
11.9680661314495	-0.855018643193607\\
11.9699409707004	-0.855018300762375\\
11.9718158097157	-0.855017958471747\\
11.9736906484958	-0.855017616321648\\
11.9755654870405	-0.855017274311999\\
11.9774403253502	-0.855016932442723\\
11.9793151634248	-0.855016590713744\\
11.9811900012644	-0.855016249124984\\
11.9830648388691	-0.855015907676366\\
11.9849396762391	-0.855015566367814\\
11.9868145133745	-0.85501522519925\\
11.9886893502752	-0.855014884170599\\
11.9905641869415	-0.855014543281782\\
11.9924390233733	-0.855014202532724\\
11.9943138595709	-0.855013861923347\\
11.9961886955343	-0.855013521453576\\
11.9980635312636	-0.855013181123334\\
11.9999383667589	-0.855012840932543\\
12.0018132020202	-0.855012500881129\\
12.0036880370478	-0.855012160969014\\
12.0055628718416	-0.855011821196122\\
12.0074377064018	-0.855011481562377\\
12.0093125407285	-0.855011142067702\\
12.0111873748217	-0.855010802712022\\
12.0130622086816	-0.855010463495261\\
12.0149370423082	-0.855010124417341\\
12.0168118757017	-0.855009785478188\\
12.0186867088622	-0.855009446677726\\
12.0205615417896	-0.855009108015878\\
12.0224363744842	-0.855008769492568\\
12.024311206946	-0.855008431107722\\
12.0261860391751	-0.855008092861262\\
12.0280608711717	-0.855007754753114\\
12.0299357029357	-0.855007416783201\\
12.0318105344674	-0.855007078951449\\
12.0336853657667	-0.855006741257782\\
12.0355601968338	-0.855006403702123\\
12.0374350276689	-0.855006066284399\\
12.0393098582718	-0.855005729004532\\
12.0411846886429	-0.855005391862449\\
12.0430595187821	-0.855005054858074\\
12.0449343486896	-0.855004717991331\\
12.0468091783654	-0.855004381262145\\
12.0486840078097	-0.855004044670442\\
12.0505588370225	-0.855003708216146\\
12.0524336660039	-0.855003371899182\\
12.0543084947541	-0.855003035719475\\
12.0561833232731	-0.855002699676951\\
12.058058151561	-0.855002363771534\\
12.0599329796179	-0.855002028003149\\
12.0618078074439	-0.855001692371723\\
12.0636826350391	-0.855001356877179\\
12.0655574624035	-0.855001021519444\\
12.0674322895374	-0.855000686298443\\
12.0693071164407	-0.855000351214101\\
12.0711819431136	-0.855000016266344\\
12.0730567695562	-0.854999681455097\\
12.0749315957685	-0.854999346780286\\
12.0768064217506	-0.854999012241837\\
12.0786812475027	-0.854998677839675\\
12.0805560730248	-0.854998343573727\\
12.082430898317	-0.854998009443917\\
12.0843057233794	-0.854997675450171\\
12.0861805482122	-0.854997341592416\\
12.0880553728153	-0.854997007870578\\
12.089930197189	-0.854996674284582\\
12.0918050213332	-0.854996340834355\\
12.0936798452481	-0.854996007519823\\
12.0955546689337	-0.854995674340911\\
12.0974294923903	-0.854995341297547\\
12.0993043156178	-0.854995008389656\\
12.1011791386163	-0.854994675617164\\
12.1030539613859	-0.854994342979998\\
12.1049287839268	-0.854994010478085\\
12.1068036062391	-0.854993678111351\\
12.1086784283227	-0.854993345879721\\
12.1105532501778	-0.854993013783124\\
12.1124280718046	-0.854992681821485\\
12.114302893203	-0.854992349994732\\
12.1161777143732	-0.85499201830279\\
12.1180525353153	-0.854991686745587\\
12.1199273560293	-0.854991355323049\\
12.1218021765154	-0.854991024035103\\
12.1236769967737	-0.854990692881677\\
12.1255518168042	-0.854990361862697\\
12.127426636607	-0.854990030978089\\
12.1293014561823	-0.854989700227782\\
12.13117627553	-0.854989369611703\\
12.1330510946504	-0.854989039129777\\
12.1349259135435	-0.854988708781934\\
12.1368007322093	-0.854988378568099\\
12.1386755506481	-0.8549880484882\\
12.1405503688598	-0.854987718542165\\
12.1424251868446	-0.854987388729921\\
12.1443000046025	-0.854987059051395\\
12.1461748221337	-0.854986729506516\\
12.1480496394382	-0.854986400095209\\
12.1499244565161	-0.854986070817404\\
12.1517992733675	-0.854985741673027\\
12.1536740899926	-0.854985412662007\\
12.1555489063914	-0.85498508378427\\
12.1574237225639	-0.854984755039746\\
12.1592985385103	-0.854984426428361\\
12.1611733542307	-0.854984097950044\\
12.1630481697252	-0.854983769604723\\
12.1649229849938	-0.854983441392325\\
12.1667978000367	-0.854983113312779\\
12.1686726148539	-0.854982785366012\\
12.1705474294455	-0.854982457551954\\
12.1724222438116	-0.854982129870531\\
12.1742970579524	-0.854981802321673\\
12.1761718718678	-0.854981474905307\\
12.178046685558	-0.854981147621362\\
12.1799214990231	-0.854980820469766\\
12.1817963122632	-0.854980493450448\\
12.1836711252783	-0.854980166563336\\
12.1855459380685	-0.854979839808359\\
12.187420750634	-0.854979513185445\\
12.1892955629748	-0.854979186694523\\
12.1911703750911	-0.854978860335522\\
12.1930451869828	-0.854978534108369\\
12.1949199986502	-0.854978208012995\\
12.1967948100932	-0.854977882049328\\
12.198669621312	-0.854977556217296\\
12.2005444323066	-0.854977230516829\\
12.2024192430773	-0.854976904947856\\
12.2042940536239	-0.854976579510305\\
12.2061688639467	-0.854976254204106\\
12.2080436740457	-0.854975929029187\\
12.209918483921	-0.854975603985479\\
12.2117932935728	-0.854975279072909\\
12.213668103001	-0.854974954291408\\
12.2155429122058	-0.854974629640905\\
12.2174177211873	-0.854974305121328\\
12.2192925299455	-0.854973980732608\\
12.2211673384806	-0.854973656474674\\
12.2230421467926	-0.854973332347455\\
12.2249169548816	-0.854973008350881\\
12.2267917627477	-0.854972684484881\\
12.2286665703911	-0.854972360749385\\
12.2305413778117	-0.854972037144323\\
12.2324161850098	-0.854971713669625\\
12.2342909919853	-0.85497139032522\\
12.2361657987383	-0.854971067111038\\
12.238040605269	-0.854970744027008\\
12.2399154115775	-0.854970421073062\\
12.2417902176637	-0.854970098249128\\
12.2436650235279	-0.854969775555138\\
12.2455398291701	-0.85496945299102\\
12.2474146345903	-0.854969130556705\\
12.2492894397888	-0.854968808252123\\
12.2511642447655	-0.854968486077204\\
12.2530390495206	-0.854968164031879\\
12.2549138540541	-0.854967842116078\\
12.2567886583661	-0.854967520329731\\
12.2586634624568	-0.854967198672768\\
12.2605382663262	-0.854966877145121\\
12.2624130699743	-0.854966555746719\\
12.2642878734014	-0.854966234477493\\
12.2661626766074	-0.854965913337373\\
12.2680374795925	-0.854965592326291\\
12.2699122823567	-0.854965271444177\\
12.2717870849002	-0.854964950690961\\
12.273661887223	-0.854964630066575\\
12.2755366893252	-0.854964309570949\\
12.2774114912069	-0.854963989204013\\
12.2792862928682	-0.8549636689657\\
12.2811610943092	-0.85496334885594\\
12.2830358955299	-0.854963028874663\\
12.2849106965305	-0.854962709021801\\
12.286785497311	-0.854962389297285\\
12.2886602978715	-0.854962069701047\\
12.2905350982122	-0.854961750233016\\
12.2924098983331	-0.854961430893125\\
12.2942846982342	-0.854961111681305\\
12.2961594979157	-0.854960792597487\\
12.2980342973777	-0.854960473641602\\
12.2999090966202	-0.854960154813583\\
12.3017838956434	-0.854959836113359\\
12.3036586944473	-0.854959517540864\\
12.3055334930319	-0.854959199096027\\
12.3074082913975	-0.854958880778782\\
12.3092830895441	-0.854958562589059\\
12.3111578874718	-0.854958244526791\\
12.3130326851806	-0.854957926591908\\
12.3149074826706	-0.854957608784344\\
12.316782279942	-0.854957291104029\\
12.3186570769948	-0.854956973550895\\
12.3205318738291	-0.854956656124875\\
12.322406670445	-0.8549563388259\\
12.3242814668425	-0.854956021653903\\
12.3261562630218	-0.854955704608815\\
12.328031058983	-0.854955387690568\\
12.3299058547261	-0.854955070899096\\
12.3317806502513	-0.854954754234329\\
12.3336554455585	-0.8549544376962\\
12.335530240648	-0.854954121284641\\
12.3374050355197	-0.854953804999585\\
12.3392798301738	-0.854953488840964\\
12.3411546246104	-0.854953172808711\\
12.3430294188294	-0.854952856902758\\
12.3449042128312	-0.854952541123037\\
12.3467790066156	-0.854952225469481\\
12.3486538001828	-0.854951909942022\\
12.3505285935329	-0.854951594540594\\
12.352403386666	-0.854951279265129\\
12.3542781795822	-0.854950964115559\\
12.3561529722814	-0.854950649091818\\
12.358027764764	-0.854950334193838\\
12.3599025570298	-0.854950019421552\\
12.361777349079	-0.854949704774893\\
12.3636521409117	-0.854949390253794\\
12.365526932528	-0.854949075858188\\
12.3674017239279	-0.854948761588008\\
12.3692765151116	-0.854948447443187\\
12.3711513060791	-0.854948133423659\\
12.3730260968305	-0.854947819529356\\
12.3749008873659	-0.854947505760211\\
12.3767756776854	-0.854947192116158\\
12.3786504677891	-0.854946878597131\\
12.380525257677	-0.854946565203062\\
12.3824000473493	-0.854946251933885\\
12.3842748368059	-0.854945938789534\\
12.3861496260471	-0.854945625769941\\
12.3880244150729	-0.854945312875042\\
12.3898992038834	-0.854945000104768\\
12.3917739924786	-0.854944687459053\\
12.3936487808587	-0.854944374937832\\
12.3955235690237	-0.854944062541039\\
12.3973983569737	-0.854943750268605\\
12.3992731447088	-0.854943438120467\\
12.4011479322291	-0.854943126096557\\
12.4030227195347	-0.85494281419681\\
12.4048975066257	-0.854942502421158\\
12.4067722935021	-0.854942190769537\\
12.408647080164	-0.85494187924188\\
12.4105218666115	-0.854941567838122\\
12.4123966528447	-0.854941256558196\\
12.4142714388637	-0.854940945402037\\
12.4161462246686	-0.854940634369578\\
12.4180210102594	-0.854940323460755\\
12.4198957956362	-0.854940012675501\\
12.4217705807992	-0.85493970201375\\
12.4236453657483	-0.854939391475438\\
12.4255201504838	-0.854939081060498\\
12.4273949350056	-0.854938770768865\\
12.4292697193139	-0.854938460600473\\
12.4311445034087	-0.854938150555258\\
12.4330192872901	-0.854937840633153\\
12.4348940709582	-0.854937530834093\\
12.4367688544131	-0.854937221158013\\
12.4386436376549	-0.854936911604848\\
12.4405184206837	-0.854936602174532\\
12.4423932034994	-0.854936292867\\
12.4442679861024	-0.854935983682188\\
12.4461427684925	-0.854935674620029\\
12.4480175506699	-0.854935365680459\\
12.4498923326347	-0.854935056863413\\
12.4517671143869	-0.854934748168826\\
12.4536418959267	-0.854934439596633\\
12.4555166772541	-0.854934131146769\\
12.4573914583692	-0.854933822819169\\
12.4592662392721	-0.854933514613768\\
12.4611410199629	-0.854933206530503\\
12.4630158004417	-0.854932898569307\\
12.4648905807084	-0.854932590730116\\
12.4667653607634	-0.854932283012866\\
12.4686401406065	-0.854931975417492\\
12.4705149202379	-0.85493166794393\\
12.4723896996577	-0.854931360592114\\
12.4742644788659	-0.854931053361981\\
12.4761392578627	-0.854930746253467\\
12.4780140366481	-0.854930439266506\\
12.4798888152222	-0.854930132401034\\
12.4817635935851	-0.854929825656988\\
12.4836383717369	-0.854929519034302\\
12.4855131496776	-0.854929212532914\\
12.4873879274074	-0.854928906152757\\
12.4892627049262	-0.85492859989377\\
12.4911374822343	-0.854928293755886\\
12.4930122593317	-0.854927987739043\\
12.4948870362184	-0.854927681843176\\
12.4967618128946	-0.854927376068222\\
12.4986365893603	-0.854927070414116\\
12.5005113656157	-0.854926764880794\\
12.5023861416607	-0.854926459468194\\
12.5042609174955	-0.85492615417625\\
12.5061356931202	-0.8549258490049\\
12.5080104685348	-0.854925543954079\\
12.5098852437394	-0.854925239023725\\
12.5117600187342	-0.854924934213772\\
12.5136347935191	-0.854924629524158\\
12.5155095680943	-0.85492432495482\\
12.5173843424599	-0.854924020505693\\
12.5192591166159	-0.854923716176715\\
12.5211338905624	-0.854923411967822\\
12.5230086642995	-0.85492310787895\\
12.5248834378273	-0.854922803910036\\
12.5267582111459	-0.854922500061017\\
12.5286329842553	-0.85492219633183\\
12.5305077571556	-0.854921892722412\\
12.532382529847	-0.854921589232699\\
12.5342573023294	-0.854921285862628\\
12.536132074603	-0.854920982612136\\
12.5380068466679	-0.854920679481161\\
12.5398816185241	-0.854920376469639\\
12.5417563901717	-0.854920073577507\\
12.5436311616108	-0.854919770804702\\
12.5455059328415	-0.854919468151163\\
12.5473807038638	-0.854919165616824\\
12.5492554746779	-0.854918863201625\\
12.5511302452839	-0.854918560905502\\
12.5530050156817	-0.854918258728392\\
12.5548797858715	-0.854917956670234\\
12.5567545558534	-0.854917654730963\\
12.5586293256274	-0.854917352910518\\
12.5605040951936	-0.854917051208837\\
12.5623788645522	-0.854916749625856\\
12.5642536337031	-0.854916448161513\\
12.5661284026466	-0.854916146815746\\
12.5680031713825	-0.854915845588493\\
12.5698779399111	-0.854915544479691\\
12.5717527082325	-0.854915243489277\\
12.5736274763466	-0.854914942617191\\
12.5755022442536	-0.854914641863368\\
12.5773770119535	-0.854914341227748\\
12.5792517794465	-0.854914040710269\\
12.5811265467326	-0.854913740310867\\
12.5830013138119	-0.854913440029481\\
12.5848760806845	-0.85491313986605\\
12.5867508473504	-0.85491283982051\\
12.5886256138098	-0.854912539892801\\
12.5905003800627	-0.85491224008286\\
12.5923751461092	-0.854911940390626\\
12.5942499119494	-0.854911640816036\\
12.5961246775833	-0.85491134135903\\
12.5979994430111	-0.854911042019544\\
12.5998742082328	-0.854910742797519\\
12.6017489732485	-0.854910443692891\\
12.6036237380582	-0.854910144705599\\
12.6054985026622	-0.854909845835583\\
12.6073732670604	-0.85490954708278\\
12.6092480312529	-0.854909248447129\\
12.6111227952398	-0.854908949928568\\
12.6129975590212	-0.854908651527036\\
12.6148723225971	-0.854908353242472\\
12.6167470859677	-0.854908055074815\\
12.618621849133	-0.854907757024002\\
12.6204966120931	-0.854907459089974\\
12.6223713748481	-0.854907161272668\\
12.624246137398	-0.854906863572024\\
12.626120899743	-0.854906565987981\\
12.6279956618831	-0.854906268520477\\
12.6298704238184	-0.854905971169451\\
12.631745185549	-0.854905673934843\\
12.6336199470749	-0.854905376816592\\
12.6354947083963	-0.854905079814636\\
12.6373694695131	-0.854904782928915\\
12.6392442304256	-0.854904486159367\\
12.6411189911338	-0.854904189505933\\
12.6429937516377	-0.854903892968552\\
12.6448685119374	-0.854903596547162\\
12.646743272033	-0.854903300241703\\
12.6486180319247	-0.854903004052115\\
12.6504927916124	-0.854902707978336\\
12.6523675510962	-0.854902412020307\\
12.6542423103763	-0.854902116177966\\
12.6561170694527	-0.854901820451254\\
12.6579918283255	-0.85490152484011\\
12.6598665869947	-0.854901229344473\\
12.6617413454605	-0.854900933964284\\
12.6636161037229	-0.854900638699481\\
12.665490861782	-0.854900343550005\\
12.6673656196379	-0.854900048515796\\
12.6692403772906	-0.854899753596792\\
12.6711151347403	-0.854899458792935\\
12.672989891987	-0.854899164104164\\
12.6748646490307	-0.854898869530418\\
12.6767394058717	-0.854898575071638\\
12.6786141625099	-0.854898280727765\\
12.6804889189454	-0.854897986498737\\
12.6823636751784	-0.854897692384495\\
12.6842384312088	-0.854897398384979\\
12.6861131870368	-0.85489710450013\\
12.6879879426624	-0.854896810729887\\
12.6898626980858	-0.854896517074191\\
12.6917374533069	-0.854896223532982\\
12.6936122083259	-0.8548959301062\\
12.6954869631429	-0.854895636793786\\
12.6973617177579	-0.85489534359568\\
12.699236472171	-0.854895050511823\\
12.7011112263823	-0.854894757542155\\
12.7029859803919	-0.854894464686616\\
12.7048607341998	-0.854894171945147\\
12.7067354878061	-0.854893879317689\\
12.708610241211	-0.854893586804183\\
12.7104849944144	-0.854893294404568\\
12.7123597474165	-0.854893002118786\\
12.7142345002173	-0.854892709946778\\
12.7161092528169	-0.854892417888484\\
12.7179840052154	-0.854892125943845\\
12.7198587574128	-0.854891834112802\\
12.7217335094093	-0.854891542395296\\
12.723608261205	-0.854891250791267\\
12.7254830127998	-0.854890959300658\\
12.7273577641939	-0.854890667923408\\
12.7292325153874	-0.854890376659459\\
12.7311072663802	-0.854890085508752\\
12.7329820171726	-0.854889794471228\\
12.7348567677646	-0.854889503546829\\
12.7367315181562	-0.854889212735495\\
12.7386062683476	-0.854888922037167\\
12.7404810183388	-0.854888631451788\\
12.7423557681299	-0.854888340979298\\
12.7442305177209	-0.854888050619639\\
12.746105267112	-0.854887760372752\\
12.7479800163033	-0.854887470238578\\
12.7498547652947	-0.85488718021706\\
12.7517295140864	-0.854886890308138\\
12.7536042626784	-0.854886600511754\\
12.7554790110709	-0.85488631082785\\
12.7573537592639	-0.854886021256367\\
12.7592285072575	-0.854885731797247\\
12.7611032550518	-0.854885442450432\\
12.7629780026467	-0.854885153215863\\
12.7648527500425	-0.854884864093483\\
12.7667274972392	-0.854884575083233\\
12.7686022442369	-0.854884286185054\\
12.7704769910356	-0.85488399739889\\
12.7723517376354	-0.854883708724681\\
12.7742264840364	-0.85488342016237\\
12.7761012302387	-0.854883131711899\\
12.7779759762423	-0.854882843373209\\
12.7798507220474	-0.854882555146244\\
12.781725467654	-0.854882267030944\\
12.7836002130621	-0.854881979027253\\
12.7854749582719	-0.854881691135111\\
12.7873497032834	-0.854881403354463\\
12.7892244480968	-0.854881115685249\\
12.791099192712	-0.854880828127413\\
12.7929739371292	-0.854880540680896\\
12.7948486813484	-0.854880253345641\\
12.7967234253697	-0.85487996612159\\
12.7985981691932	-0.854879679008686\\
12.800472912819	-0.854879392006871\\
12.8023476562471	-0.854879105116089\\
12.8042223994776	-0.85487881833628\\
12.8060971425106	-0.854878531667389\\
12.8079718853462	-0.854878245109357\\
12.8098466279844	-0.854877958662128\\
12.8117213704254	-0.854877672325643\\
12.8135961126691	-0.854877386099847\\
12.8154708547157	-0.854877099984681\\
12.8173455965653	-0.854876813980088\\
12.8192203382178	-0.854876528086012\\
12.8210950796735	-0.854876242302395\\
12.8229698209323	-0.85487595662918\\
12.8248445619944	-0.85487567106631\\
12.8267193028598	-0.854875385613729\\
12.8285940435286	-0.854875100271378\\
12.8304687840008	-0.854874815039202\\
12.8323435242766	-0.854874529917144\\
12.834218264356	-0.854874244905146\\
12.8360930042392	-0.854873960003151\\
12.8379677439261	-0.854873675211104\\
12.8398424834168	-0.854873390528947\\
12.8417172227115	-0.854873105956624\\
12.8435919618101	-0.854872821494078\\
12.8454667007129	-0.854872537141252\\
12.8473414394197	-0.85487225289809\\
12.8492161779308	-0.854871968764535\\
12.8510909162462	-0.85487168474053\\
12.852965654366	-0.85487140082602\\
12.8548403922902	-0.854871117020948\\
12.8567151300189	-0.854870833325257\\
12.8585898675523	-0.854870549738891\\
12.8604646048903	-0.854870266261794\\
12.862339342033	-0.85486998289391\\
12.8642140789806	-0.854869699635181\\
12.866088815733	-0.854869416485553\\
12.8679635522904	-0.854869133444969\\
12.8698382886529	-0.854868850513372\\
12.8717130248205	-0.854868567690706\\
12.8735877607932	-0.854868284976916\\
12.8754624965713	-0.854868002371946\\
12.8773372321547	-0.854867719875739\\
12.8792119675434	-0.854867437488239\\
12.8810867027377	-0.854867155209392\\
12.8829614377376	-0.854866873039139\\
12.884836172543	-0.854866590977427\\
12.8867109071542	-0.854866309024198\\
12.8885856415712	-0.854866027179398\\
12.890460375794	-0.854865745442971\\
12.8923351098228	-0.85486546381486\\
12.8942098436575	-0.85486518229501\\
12.8960845772984	-0.854864900883366\\
12.8979593107454	-0.854864619579871\\
12.8998340439986	-0.854864338384471\\
12.9017087770581	-0.85486405729711\\
12.9035835099241	-0.854863776317731\\
12.9054582425964	-0.854863495446281\\
12.9073329750753	-0.854863214682703\\
12.9092077073608	-0.854862934026942\\
12.9110824394529	-0.854862653478943\\
12.9129571713518	-0.85486237303865\\
12.9148319030575	-0.854862092706008\\
12.9167066345701	-0.854861812480962\\
12.9185813658897	-0.854861532363457\\
12.9204560970163	-0.854861252353436\\
12.92233082795	-0.854860972450847\\
12.9242055586909	-0.854860692655632\\
12.926080289239	-0.854860412967738\\
12.9279550195945	-0.854860133387108\\
12.9298297497574	-0.854859853913689\\
12.9317044797278	-0.854859574547425\\
12.9335792095057	-0.854859295288261\\
12.9354539390912	-0.854859016136142\\
12.9373286684845	-0.854858737091014\\
12.9392033976855	-0.854858458152821\\
12.9410781266943	-0.85485817932151\\
12.9429528555111	-0.854857900597024\\
12.9448275841358	-0.85485762197931\\
12.9467023125687	-0.854857343468312\\
12.9485770408096	-0.854857065063977\\
12.9504517688588	-0.854856786766249\\
12.9523264967162	-0.854856508575073\\
12.954201224382	-0.854856230490397\\
12.9560759518562	-0.854855952512164\\
12.9579506791389	-0.85485567464032\\
12.9598254062302	-0.854855396874811\\
12.9617001331301	-0.854855119215583\\
12.9635748598388	-0.854854841662581\\
12.9654495863562	-0.854854564215751\\
12.9673243126825	-0.854854286875038\\
12.9691990388178	-0.854854009640389\\
12.971073764762	-0.854853732511749\\
12.9729484905153	-0.854853455489063\\
12.9748232160778	-0.854853178572278\\
12.9766979414495	-0.85485290176134\\
12.9785726666305	-0.854852625056195\\
12.9804473916208	-0.854852348456787\\
12.9823221164206	-0.854852071963065\\
12.9841968410299	-0.854851795574972\\
12.9860715654488	-0.854851519292456\\
12.9879462896773	-0.854851243115463\\
12.9898210137156	-0.854850967043939\\
12.9916957375637	-0.854850691077829\\
12.9935704612216	-0.85485041521708\\
12.9954451846895	-0.854850139461639\\
12.9973199079674	-0.854849863811451\\
12.9991946310554	-0.854849588266463\\
13.0010693539536	-0.854849312826621\\
13.002944076662	-0.854849037491872\\
13.0048187991807	-0.854848762262162\\
13.0066935215097	-0.854848487137437\\
13.0085682436492	-0.854848212117643\\
13.0104429655993	-0.854847937202728\\
13.0123176873599	-0.854847662392638\\
13.0141924089312	-0.854847387687319\\
13.0160671303132	-0.854847113086719\\
13.017941851506	-0.854846838590783\\
13.0198165725097	-0.854846564199458\\
13.0216912933244	-0.854846289912691\\
13.02356601395	-0.854846015730429\\
13.0254407343868	-0.854845741652619\\
13.0273154546347	-0.854845467679207\\
13.0291901746939	-0.85484519381014\\
13.0310648945643	-0.854844920045365\\
13.0329396142461	-0.854844646384829\\
13.0348143337394	-0.854844372828478\\
13.0366890530442	-0.854844099376261\\
13.0385637721606	-0.854843826028124\\
13.0404384910887	-0.854843552784013\\
13.0423132098284	-0.854843279643877\\
13.04418792838	-0.854843006607661\\
13.0460626467435	-0.854842733675314\\
13.0479373649189	-0.854842460846783\\
13.0498120829063	-0.854842188122013\\
13.0516868007058	-0.854841915500954\\
13.0535615183174	-0.854841642983552\\
13.0554362357413	-0.854841370569754\\
13.0573109529775	-0.854841098259509\\
13.0591856700261	-0.854840826052762\\
13.061060386887	-0.854840553949462\\
13.0629351035606	-0.854840281949556\\
13.0648098200466	-0.854840010052992\\
13.0666845363454	-0.854839738259716\\
13.0685592524568	-0.854839466569677\\
13.0704339683811	-0.854839194982823\\
13.0723086841182	-0.8548389234991\\
13.0741833996682	-0.854838652118456\\
13.0760581150313	-0.85483838084084\\
13.0779328302074	-0.854838109666199\\
13.0798075451967	-0.85483783859448\\
13.0816822599992	-0.854837567625631\\
13.083556974615	-0.854837296759601\\
13.0854316890441	-0.854837025996336\\
13.0873064032866	-0.854836755335786\\
13.0891811173427	-0.854836484777897\\
13.0910558312123	-0.854836214322618\\
13.0929305448956	-0.854835943969896\\
13.0948052583925	-0.854835673719681\\
13.0966799717033	-0.854835403571919\\
13.0985546848278	-0.854835133526559\\
13.1004293977663	-0.854834863583549\\
13.1023041105188	-0.854834593742837\\
13.1041788230854	-0.854834324004371\\
13.106053535466	-0.8548340543681\\
13.1079282476609	-0.854833784833972\\
13.10980295967	-0.854833515401935\\
13.1116776714935	-0.854833246071937\\
13.1135523831314	-0.854832976843926\\
13.1154270945837	-0.854832707717852\\
13.1173018058506	-0.854832438693662\\
13.1191765169321	-0.854832169771305\\
13.1210512278283	-0.854831900950729\\
13.1229259385392	-0.854831632231883\\
13.124800649065	-0.854831363614715\\
13.1266753594056	-0.854831095099175\\
13.1285500695612	-0.85483082668521\\
13.1304247795319	-0.854830558372769\\
13.1322994893176	-0.854830290161801\\
13.1341741989185	-0.854830022052255\\
13.1360489083347	-0.854829754044078\\
13.1379236175662	-0.854829486137221\\
13.139798326613	-0.854829218331632\\
13.1416730354753	-0.85482895062726\\
13.1435477441531	-0.854828683024053\\
13.1454224526465	-0.85482841552196\\
13.1472971609555	-0.854828148120931\\
13.1491718690803	-0.854827880820914\\
13.1510465770208	-0.854827613621858\\
13.1529212847772	-0.854827346523713\\
13.1547959923496	-0.854827079526427\\
13.1566706997379	-0.85482681262995\\
13.1585454069423	-0.85482654583423\\
13.1604201139629	-0.854826279139217\\
13.1622948207996	-0.85482601254486\\
13.1641695274527	-0.854825746051108\\
13.166044233922	-0.854825479657911\\
13.1679189402078	-0.854825213365217\\
13.1697936463101	-0.854824947172976\\
13.1716683522289	-0.854824681081138\\
13.1735430579643	-0.854824415089651\\
13.1754177635164	-0.854824149198465\\
13.1772924688853	-0.85482388340753\\
13.179167174071	-0.854823617716795\\
13.1810418790736	-0.85482335212621\\
13.1829165838931	-0.854823086635723\\
13.1847912885297	-0.854822821245286\\
13.1866659929834	-0.854822555954846\\
13.1885406972542	-0.854822290764354\\
13.1904154013423	-0.85482202567376\\
13.1922901052476	-0.854821760683014\\
13.1941648089704	-0.854821495792064\\
13.1960395125105	-0.85482123100086\\
13.1979142158682	-0.854820966309354\\
13.1997889190435	-0.854820701717494\\
13.2016636220364	-0.854820437225229\\
13.203538324847	-0.854820172832512\\
13.2054130274754	-0.85481990853929\\
13.2072877299216	-0.854819644345514\\
13.2091624321857	-0.854819380251133\\
13.2110371342679	-0.854819116256099\\
13.2129118361681	-0.854818852360361\\
13.2147865378864	-0.854818588563869\\
13.2166612394228	-0.854818324866573\\
13.2185359407776	-0.854818061268423\\
13.2204106419506	-0.85481779776937\\
13.2222853429421	-0.854817534369364\\
13.224160043752	-0.854817271068354\\
13.2260347443804	-0.854817007866292\\
13.2279094448274	-0.854816744763127\\
13.2297841450931	-0.85481648175881\\
13.2316588451775	-0.854816218853291\\
13.2335335450807	-0.85481595604652\\
13.2354082448028	-0.854815693338449\\
13.2372829443438	-0.854815430729027\\
13.2391576437038	-0.854815168218205\\
13.2410323428828	-0.854814905805933\\
13.242907041881	-0.854814643492162\\
13.2447817406984	-0.854814381276843\\
13.246656439335	-0.854814119159925\\
13.248531137791	-0.854813857141361\\
13.2504058360664	-0.8548135952211\\
13.2522805341613	-0.854813333399093\\
13.2541552320757	-0.854813071675291\\
13.2560299298097	-0.854812810049645\\
13.2579046273633	-0.854812548522105\\
13.2597793247367	-0.854812287092623\\
13.2616540219299	-0.854812025761148\\
13.263528718943	-0.854811764527632\\
13.265403415776	-0.854811503392027\\
13.267278112429	-0.854811242354282\\
13.2691528089021	-0.854810981414349\\
13.2710275051953	-0.854810720572179\\
13.2729022013087	-0.854810459827723\\
13.2747768972424	-0.854810199180931\\
13.2766515929965	-0.854809938631756\\
13.2785262885709	-0.854809678180148\\
13.2804009839658	-0.854809417826058\\
13.2822756791813	-0.854809157569438\\
13.2841503742174	-0.854808897410239\\
13.2860250690741	-0.854808637348411\\
13.2878997637516	-0.854808377383907\\
13.2897744582498	-0.854808117516678\\
13.291649152569	-0.854807857746674\\
13.2935238467091	-0.854807598073848\\
13.2953985406701	-0.85480733849815\\
13.2972732344523	-0.854807079019533\\
13.2991479280556	-0.854806819637947\\
13.3010226214801	-0.854806560353345\\
13.3028973147258	-0.854806301165677\\
13.3047720077929	-0.854806042074895\\
13.3066467006815	-0.854805783080951\\
13.3085213933914	-0.854805524183797\\
13.3103960859229	-0.854805265383384\\
13.312270778276	-0.854805006679663\\
13.3141454704508	-0.854804748072587\\
13.3160201624473	-0.854804489562107\\
13.3178948542656	-0.854804231148175\\
13.3197695459058	-0.854803972830743\\
13.3216442373679	-0.854803714609763\\
13.323518928652	-0.854803456485186\\
13.3253936197581	-0.854803198456964\\
13.3272683106864	-0.854802940525049\\
13.3291430014369	-0.854802682689394\\
13.3310176920096	-0.85480242494995\\
13.3328923824047	-0.85480216730667\\
13.3347670726222	-0.854801909759504\\
13.3366417626621	-0.854801652308406\\
13.3385164525245	-0.854801394953328\\
13.3403911422095	-0.854801137694221\\
13.3422658317172	-0.854800880531038\\
13.3441405210476	-0.85480062346373\\
13.3460152102008	-0.854800366492251\\
13.3478898991768	-0.854800109616552\\
13.3497645879758	-0.854799852836586\\
13.3516392765977	-0.854799596152305\\
13.3535139650427	-0.854799339563661\\
13.3553886533108	-0.854799083070607\\
13.357263341402	-0.854798826673095\\
13.3591380293166	-0.854798570371078\\
13.3610127170544	-0.854798314164507\\
13.3628874046156	-0.854798058053336\\
13.3647620920002	-0.854797802037517\\
13.3666367792083	-0.854797546117003\\
13.36851146624	-0.854797290291745\\
13.3703861530954	-0.854797034561698\\
13.3722608397744	-0.854796778926812\\
13.3741355262772	-0.854796523387042\\
13.3760102126039	-0.85479626794234\\
13.3778848987544	-0.854796012592658\\
13.3797595847289	-0.854795757337949\\
13.3816342705274	-0.854795502178166\\
13.38350895615	-0.854795247113262\\
13.3853836415967	-0.854794992143189\\
13.3872583268677	-0.854794737267901\\
13.389133011963	-0.854794482487351\\
13.3910076968826	-0.854794227801491\\
13.3928823816266	-0.854793973210274\\
13.3947570661951	-0.854793718713653\\
13.3966317505881	-0.854793464311582\\
13.3985064348058	-0.854793210004013\\
13.4003811188481	-0.854792955790899\\
13.4022558027151	-0.854792701672194\\
13.404130486407	-0.854792447647851\\
13.4060051699237	-0.854792193717823\\
13.4078798532654	-0.854791939882062\\
13.409754536432	-0.854791686140523\\
13.4116292194237	-0.854791432493159\\
13.4135039022405	-0.854791178939922\\
13.4153785848826	-0.854790925480766\\
13.4172532673498	-0.854790672115645\\
13.4191279496424	-0.854790418844511\\
13.4210026317604	-0.854790165667319\\
13.4228773137038	-0.854789912584022\\
13.4247519954727	-0.854789659594572\\
13.4266266770672	-0.854789406698924\\
13.4285013584873	-0.854789153897031\\
13.4303760397331	-0.854788901188847\\
13.4322507208047	-0.854788648574325\\
13.4341254017021	-0.854788396053419\\
13.4360000824255	-0.854788143626082\\
13.4378747629747	-0.854787891292268\\
13.43974944335	-0.854787639051931\\
13.4416241235514	-0.854787386905024\\
13.4434988035789	-0.854787134851502\\
13.4453734834327	-0.854786882891318\\
13.4472481631127	-0.854786631024425\\
13.4491228426191	-0.854786379250778\\
13.4509975219518	-0.854786127570331\\
13.4528722011111	-0.854785875983037\\
13.4547468800968	-0.85478562448885\\
13.4566215589092	-0.854785373087724\\
13.4584962375482	-0.854785121779614\\
13.460370916014	-0.854784870564473\\
13.4622455943065	-0.854784619442255\\
13.4641202724259	-0.854784368412914\\
13.4659949503722	-0.854784117476405\\
13.4678696281454	-0.854783866632681\\
13.4697443057457	-0.854783615881697\\
13.4716189831731	-0.854783365223407\\
13.4734936604277	-0.854783114657764\\
13.4753683375095	-0.854782864184724\\
13.4772430144186	-0.85478261380424\\
13.4791176911551	-0.854782363516267\\
13.480992367719	-0.85478211332076\\
13.4828670441104	-0.854781863217671\\
13.4847417203293	-0.854781613206956\\
13.4866163963758	-0.85478136328857\\
13.48849107225	-0.854781113462466\\
13.4903657479519	-0.854780863728599\\
13.4922404234817	-0.854780614086924\\
13.4941150988392	-0.854780364537395\\
13.4959897740248	-0.854780115079966\\
13.4978644490383	-0.854779865714593\\
13.4997391238798	-0.854779616441229\\
13.5016137985495	-0.85477936725983\\
13.5034884730474	-0.85477911817035\\
13.5053631473734	-0.854778869172743\\
13.5072378215278	-0.854778620266965\\
13.5091124955106	-0.85477837145297\\
13.5109871693218	-0.854778122730713\\
13.5128618429615	-0.854777874100148\\
13.5147365164297	-0.854777625561232\\
13.5166111897266	-0.854777377113917\\
13.5184858628521	-0.85477712875816\\
13.5203605358064	-0.854776880493915\\
13.5222352085894	-0.854776632321137\\
13.5241098812014	-0.854776384239781\\
13.5259845536422	-0.854776136249802\\
13.5278592259121	-0.854775888351155\\
13.529733898011	-0.854775640543795\\
13.5316085699391	-0.854775392827677\\
13.5334832416963	-0.854775145202757\\
13.5353579132827	-0.854774897668989\\
13.5372325846985	-0.854774650226328\\
13.5391072559436	-0.85477440287473\\
13.5409819270182	-0.85477415561415\\
13.5428565979222	-0.854773908444543\\
13.5447312686558	-0.854773661365865\\
13.546605939219	-0.85477341437807\\
13.5484806096119	-0.854773167481114\\
13.5503552798346	-0.854772920674953\\
13.552229949887	-0.854772673959541\\
13.5541046197693	-0.854772427334834\\
13.5559792894815	-0.854772180800788\\
13.5578539590237	-0.854771934357358\\
13.559728628396	-0.8547716880045\\
13.5616032975984	-0.854771441742168\\
13.5634779666309	-0.854771195570319\\
13.5653526354937	-0.854770949488908\\
13.5672273041868	-0.854770703497891\\
13.5691019727102	-0.854770457597223\\
13.5709766410641	-0.85477021178686\\
13.5728513092484	-0.854769966066758\\
13.5747259772633	-0.854769720436872\\
13.5766006451088	-0.854769474897158\\
13.5784753127849	-0.854769229447572\\
13.5803499802918	-0.854768984088069\\
13.5822246476295	-0.854768738818606\\
13.5840993147981	-0.854768493639139\\
13.5859739817975	-0.854768248549622\\
13.5878486486279	-0.854768003550012\\
13.5897233152894	-0.854767758640265\\
13.591597981782	-0.854767513820337\\
13.5934726481057	-0.854767269090184\\
13.5953473142606	-0.854767024449762\\
13.5972219802469	-0.854766779899026\\
13.5990966460644	-0.854766535437934\\
13.6009713117134	-0.85476629106644\\
13.6028459771939	-0.854766046784502\\
13.6047206425058	-0.854765802592075\\
13.6065953076494	-0.854765558489115\\
13.6084699726246	-0.854765314475579\\
13.6103446374315	-0.854765070551422\\
13.6122193020702	-0.854764826716602\\
13.6140939665408	-0.854764582971074\\
13.6159686308432	-0.854764339314795\\
13.6178432949776	-0.85476409574772\\
13.619717958944	-0.854763852269807\\
13.6215926227425	-0.854763608881012\\
13.6234672863731	-0.85476336558129\\
13.6253419498359	-0.854763122370599\\
13.627216613131	-0.854762879248895\\
13.6290912762584	-0.854762636216135\\
13.6309659392182	-0.854762393272274\\
13.6328406020104	-0.85476215041727\\
13.6347152646352	-0.854761907651079\\
13.6365899270925	-0.854761664973657\\
13.6384645893824	-0.854761422384962\\
13.640339251505	-0.854761179884949\\
13.6422139134604	-0.854760937473577\\
13.6440885752485	-0.8547606951508\\
13.6459632368696	-0.854760452916577\\
13.6478378983236	-0.854760210770863\\
13.6497125596105	-0.854759968713615\\
13.6515872207305	-0.854759726744791\\
13.6534618816836	-0.854759484864347\\
13.6553365424699	-0.85475924307224\\
13.6572112030895	-0.854759001368427\\
13.6590858635423	-0.854758759752865\\
13.6609605238284	-0.854758518225511\\
13.662835183948	-0.854758276786321\\
13.6647098439011	-0.854758035435252\\
13.6665845036877	-0.854757794172263\\
13.6684591633079	-0.854757552997309\\
13.6703338227617	-0.854757311910348\\
13.6722084820492	-0.854757070911336\\
13.6740831411706	-0.854756830000232\\
13.6759578001257	-0.854756589176992\\
13.6778324589148	-0.854756348441573\\
13.6797071175378	-0.854756107793933\\
13.6815817759948	-0.854755867234028\\
13.6834564342859	-0.854755626761816\\
13.6853310924112	-0.854755386377255\\
13.6872057503706	-0.854755146080301\\
13.6890804081643	-0.854754905870912\\
13.6909550657923	-0.854754665749045\\
13.6928297232546	-0.854754425714657\\
13.6947043805514	-0.854754185767707\\
13.6965790376827	-0.854753945908151\\
13.6984536946486	-0.854753706135947\\
13.7003283514491	-0.854753466451052\\
13.7022030080842	-0.854753226853425\\
13.7040776645541	-0.854752987343021\\
13.7059523208588	-0.8547527479198\\
13.7078269769983	-0.854752508583718\\
13.7097016329727	-0.854752269334733\\
13.7115762887821	-0.854752030172804\\
13.7134509444266	-0.854751791097886\\
13.7153255999061	-0.854751552109939\\
13.7172002552208	-0.85475131320892\\
13.7190749103707	-0.854751074394786\\
13.7209495653558	-0.854750835667495\\
13.7228242201763	-0.854750597027006\\
13.7246988748322	-0.854750358473276\\
13.7265735293235	-0.854750120006262\\
13.7284481836504	-0.854749881625924\\
13.7303228378128	-0.854749643332218\\
13.7321974918108	-0.854749405125102\\
13.7340721456445	-0.854749167004535\\
13.735946799314	-0.854748928970474\\
13.7378214528192	-0.854748691022878\\
13.7396961061603	-0.854748453161704\\
13.7415707593374	-0.854748215386911\\
13.7434454123504	-0.854747977698456\\
13.7453200651995	-0.854747740096299\\
13.7471947178846	-0.854747502580395\\
13.7490693704059	-0.854747265150705\\
13.7509440227635	-0.854747027807186\\
13.7528186749573	-0.854746790549796\\
13.7546933269874	-0.854746553378494\\
13.756567978854	-0.854746316293238\\
13.758442630557	-0.854746079293985\\
13.7603172820965	-0.854745842380696\\
13.7621919334726	-0.854745605553326\\
13.7640665846853	-0.854745368811836\\
13.7659412357347	-0.854745132156184\\
13.7678158866208	-0.854744895586327\\
13.7696905373437	-0.854744659102224\\
13.7715651879036	-0.854744422703835\\
13.7734398383003	-0.854744186391116\\
13.775314488534	-0.854743950164027\\
13.7771891386047	-0.854743714022527\\
13.7790637885126	-0.854743477966573\\
13.7809384382576	-0.854743241996125\\
13.7828130878398	-0.85474300611114\\
13.7846877372593	-0.854742770311579\\
13.7865623865162	-0.854742534597398\\
13.7884370356104	-0.854742298968558\\
13.7903116845421	-0.854742063425016\\
13.7921863333112	-0.854741827966732\\
13.794060981918	-0.854741592593664\\
13.7959356303624	-0.85474135730577\\
13.7978102786444	-0.854741122103011\\
13.7996849267642	-0.854740886985344\\
13.8015595747218	-0.854740651952729\\
13.8034342225173	-0.854740417005123\\
13.8053088701506	-0.854740182142487\\
13.807183517622	-0.85473994736478\\
13.8090581649313	-0.854739712671959\\
13.8109328120788	-0.854739478063984\\
13.8128074590644	-0.854739243540815\\
13.8146821058882	-0.85473900910241\\
13.8165567525503	-0.854738774748727\\
13.8184313990507	-0.854738540479727\\
13.8203060453895	-0.854738306295369\\
13.8221806915667	-0.854738072195611\\
13.8240553375824	-0.854737838180412\\
13.8259299834367	-0.854737604249733\\
13.8278046291296	-0.854737370403531\\
13.8296792746611	-0.854737136641767\\
13.8315539200314	-0.854736902964399\\
13.8334285652405	-0.854736669371387\\
13.8353032102884	-0.85473643586269\\
13.8371778551752	-0.854736202438268\\
13.839052499901	-0.854735969098079\\
13.8409271444658	-0.854735735842083\\
13.8428017888696	-0.85473550267024\\
13.8446764331126	-0.854735269582509\\
13.8465510771948	-0.85473503657885\\
13.8484257211162	-0.854734803659221\\
13.8503003648769	-0.854734570823582\\
13.852175008477	-0.854734338071894\\
13.8540496519165	-0.854734105404115\\
13.8559242951955	-0.854733872820205\\
13.857798938314	-0.854733640320123\\
13.8596735812722	-0.85473340790383\\
13.8615482240699	-0.854733175571285\\
13.8634228667074	-0.854732943322447\\
13.8652975091846	-0.854732711157277\\
13.8671721515017	-0.854732479075733\\
13.8690467936586	-0.854732247077777\\
13.8709214356554	-0.854732015163367\\
13.8727960774923	-0.854731783332462\\
13.8746707191692	-0.854731551585025\\
13.8765453606862	-0.854731319921013\\
13.8784200020434	-0.854731088340386\\
13.8802946432408	-0.854730856843106\\
13.8821692842784	-0.854730625429131\\
13.8840439251564	-0.854730394098421\\
13.8859185658748	-0.854730162850937\\
13.8877932064337	-0.854729931686639\\
13.889667846833	-0.854729700605486\\
13.8915424870729	-0.854729469607438\\
13.8934171271535	-0.854729238692457\\
13.8952917670747	-0.8547290078605\\
13.8971664068366	-0.85472877711153\\
13.8990410464393	-0.854728546445505\\
13.9009156858829	-0.854728315862386\\
13.9027903251674	-0.854728085362134\\
13.9046649642928	-0.854727854944708\\
13.9065396032593	-0.854727624610068\\
13.9084142420668	-0.854727394358176\\
13.9102888807155	-0.85472716418899\\
13.9121635192053	-0.854726934102472\\
13.9140381575364	-0.854726704098582\\
13.9159127957088	-0.85472647417728\\
13.9177874337226	-0.854726244338527\\
13.9196620715777	-0.854726014582282\\
13.9215367092744	-0.854725784908506\\
13.9234113468125	-0.854725555317161\\
13.9252859841923	-0.854725325808205\\
13.9271606214136	-0.8547250963816\\
13.9290352584767	-0.854724867037306\\
13.9309098953815	-0.854724637775284\\
13.9327845321282	-0.854724408595494\\
13.9346591687167	-0.854724179497898\\
13.9365338051471	-0.854723950482454\\
13.9384084414195	-0.854723721549125\\
13.9402830775339	-0.85472349269787\\
13.9421577134904	-0.854723263928651\\
13.9440323492891	-0.854723035241428\\
13.9459069849299	-0.854722806636162\\
13.9477816204131	-0.854722578112813\\
13.9496562557385	-0.854722349671343\\
13.9515308909063	-0.854722121311712\\
13.9534055259165	-0.85472189303388\\
13.9552801607692	-0.85472166483781\\
13.9571547954645	-0.854721436723461\\
13.9590294300023	-0.854721208690795\\
13.9609040643828	-0.854720980739773\\
13.962778698606	-0.854720752870354\\
13.964653332672	-0.854720525082502\\
13.9665279665808	-0.854720297376175\\
13.9684026003324	-0.854720069751336\\
13.970277233927	-0.854719842207945\\
13.9721518673646	-0.854719614745964\\
13.9740265006452	-0.854719387365353\\
13.9759011337689	-0.854719160066073\\
13.9777757667357	-0.854718932848087\\
13.9796503995458	-0.854718705711354\\
13.9815250321991	-0.854718478655837\\
13.9833996646957	-0.854718251681495\\
13.9852742970358	-0.854718024788291\\
13.9871489292192	-0.854717797976186\\
13.9890235612461	-0.854717571245141\\
13.9908981931166	-0.854717344595117\\
13.9927728248307	-0.854717118026075\\
13.9946474563884	-0.854716891537977\\
13.9965220877898	-0.854716665130785\\
13.998396719035	-0.854716438804459\\
14.000271350124	-0.854716212558961\\
14.0021459810569	-0.854715986394253\\
14.0040206118337	-0.854715760310295\\
14.0058952424545	-0.85471553430705\\
14.0077698729194	-0.854715308384478\\
14.0096445032283	-0.854715082542542\\
14.0115191333814	-0.854714856781203\\
14.0133937633787	-0.854714631100422\\
14.0152683932202	-0.854714405500162\\
14.0171430229061	-0.854714179980383\\
14.0190176524363	-0.854713954541047\\
14.0208922818109	-0.854713729182116\\
14.0227669110301	-0.854713503903552\\
14.0246415400938	-0.854713278705315\\
14.026516169002	-0.854713053587369\\
14.0283907977549	-0.854712828549675\\
14.0302654263525	-0.854712603592194\\
14.0321400547949	-0.854712378714889\\
14.0340146830821	-0.854712153917721\\
14.0358893112141	-0.854711929200651\\
14.0377639391911	-0.854711704563643\\
14.039638567013	-0.854711480006657\\
14.04151319468	-0.854711255529656\\
14.043387822192	-0.854711031132601\\
14.0452624495492	-0.854710806815455\\
14.0471370767516	-0.854710582578179\\
14.0490117037992	-0.854710358420736\\
14.0508863306922	-0.854710134343087\\
14.0527609574305	-0.854709910345195\\
14.0546355840142	-0.854709686427021\\
14.0565102104433	-0.854709462588528\\
14.058384836718	-0.854709238829678\\
14.0602594628383	-0.854709015150432\\
14.0621340888042	-0.854708791550754\\
14.0640087146158	-0.854708568030605\\
14.0658833402731	-0.854708344589947\\
14.0677579657763	-0.854708121228743\\
14.0696325911252	-0.854707897946955\\
14.0715072163201	-0.854707674744545\\
14.0733818413609	-0.854707451621475\\
14.0752564662478	-0.854707228577708\\
14.0771310909807	-0.854707005613207\\
14.0790057155597	-0.854706782727932\\
14.0808803399849	-0.854706559921848\\
14.0827549642564	-0.854706337194915\\
14.0846295883741	-0.854706114547098\\
14.0865042123381	-0.854705891978357\\
14.0883788361486	-0.854705669488656\\
14.0902534598055	-0.854705447077957\\
14.0921280833089	-0.854705224746223\\
14.0940027066588	-0.854705002493415\\
14.0958773298554	-0.854704780319498\\
14.0977519528986	-0.854704558224432\\
14.0996265757885	-0.854704336208182\\
14.1015011985252	-0.854704114270708\\
14.1033758211088	-0.854703892411975\\
14.1052504435392	-0.854703670631945\\
14.1071250658165	-0.85470344893058\\
14.1089996879408	-0.854703227307843\\
14.1108743099122	-0.854703005763697\\
14.1127489317306	-0.854702784298105\\
14.1146235533962	-0.854702562911029\\
14.116498174909	-0.854702341602433\\
14.1183727962691	-0.854702120372278\\
14.1202474174765	-0.854701899220529\\
14.1221220385312	-0.854701678147147\\
14.1239966594333	-0.854701457152096\\
14.125871280183	-0.854701236235338\\
14.1277459007801	-0.854701015396837\\
14.1296205212248	-0.854700794636555\\
14.1314951415172	-0.854700573954456\\
14.1333697616572	-0.854700353350502\\
14.135244381645	-0.854700132824656\\
14.1371190014805	-0.854699912376882\\
14.1389936211639	-0.854699692007143\\
14.1408682406952	-0.854699471715401\\
14.1427428600745	-0.85469925150162\\
14.1446174793017	-0.854699031365763\\
14.146492098377	-0.854698811307793\\
14.1483667173005	-0.854698591327673\\
14.150241336072	-0.854698371425366\\
14.1521159546919	-0.854698151600837\\
14.1539905731599	-0.854697931854046\\
14.1558651914763	-0.854697712184959\\
14.1577398096411	-0.854697492593539\\
14.1596144276543	-0.854697273079748\\
14.161489045516	-0.85469705364355\\
14.1633636632263	-0.854696834284909\\
14.1652382807851	-0.854696615003787\\
14.1671128981926	-0.854696395800148\\
14.1689875154487	-0.854696176673956\\
14.1708621325536	-0.854695957625174\\
14.1727367495074	-0.854695738653765\\
14.1746113663099	-0.854695519759694\\
14.1764859829614	-0.854695300942922\\
14.1783605994619	-0.854695082203415\\
14.1802352158113	-0.854694863541135\\
14.1821098320098	-0.854694644956046\\
14.1839844480575	-0.854694426448112\\
14.1858590639543	-0.854694208017296\\
14.1877336797004	-0.854693989663561\\
14.1896082952957	-0.854693771386873\\
14.1914829107403	-0.854693553187193\\
14.1933575260344	-0.854693335064487\\
14.1952321411779	-0.854693117018716\\
14.1971067561708	-0.854692899049847\\
14.1989813710133	-0.854692681157841\\
14.2008559857054	-0.854692463342663\\
14.2027306002472	-0.854692245604277\\
14.2046052146387	-0.854692027942646\\
14.2064798288799	-0.854691810357735\\
14.2083544429709	-0.854691592849507\\
14.2102290569118	-0.854691375417926\\
14.2121036707025	-0.854691158062956\\
14.2139782843433	-0.85469094078456\\
14.2158528978341	-0.854690723582704\\
14.2177275111749	-0.85469050645735\\
14.2196021243658	-0.854690289408464\\
14.221476737407	-0.854690072436008\\
14.2233513502983	-0.854689855539947\\
14.2252259630399	-0.854689638720245\\
14.2271005756319	-0.854689421976865\\
14.2289751880742	-0.854689205309773\\
14.230849800367	-0.854688988718933\\
14.2327244125103	-0.854688772204307\\
14.2345990245041	-0.854688555765861\\
14.2364736363485	-0.854688339403559\\
14.2383482480435	-0.854688123117365\\
14.2402228595892	-0.854687906907242\\
14.2420974709857	-0.854687690773156\\
14.243972082233	-0.854687474715071\\
14.2458466933311	-0.854687258732951\\
14.2477213042801	-0.85468704282676\\
14.2495959150801	-0.854686826996462\\
14.2514705257311	-0.854686611242023\\
14.2533451362332	-0.854686395563405\\
14.2552197465864	-0.854686179960574\\
14.2570943567907	-0.854685964433495\\
14.2589689668462	-0.854685748982131\\
14.2608435767531	-0.854685533606447\\
14.2627181865112	-0.854685318306407\\
14.2645927961207	-0.854685103081976\\
14.2664674055816	-0.854684887933119\\
14.268342014894	-0.8546846728598\\
14.270216624058	-0.854684457861983\\
14.2720912330735	-0.854684242939634\\
14.2739658419406	-0.854684028092716\\
14.2758404506595	-0.854683813321194\\
14.27771505923	-0.854683598625034\\
14.2795896676523	-0.854683384004199\\
14.2814642759265	-0.854683169458655\\
14.2833388840526	-0.854682954988365\\
14.2852134920306	-0.854682740593296\\
14.2870880998606	-0.854682526273411\\
14.2889627075426	-0.854682312028675\\
14.2908373150768	-0.854682097859054\\
14.2927119224631	-0.854681883764511\\
14.2945865297015	-0.854681669745012\\
14.2964611367923	-0.854681455800522\\
14.2983357437353	-0.854681241931005\\
14.3002103505307	-0.854681028136426\\
14.3020849571784	-0.854680814416751\\
14.3039595636787	-0.854680600771944\\
14.3058341700314	-0.85468038720197\\
14.3077087762367	-0.854680173706794\\
14.3095833822946	-0.854679960286381\\
14.3114579882052	-0.854679746940697\\
14.3133325939685	-0.854679533669706\\
14.3152071995845	-0.854679320473372\\
14.3170818050534	-0.854679107351662\\
14.3189564103751	-0.854678894304541\\
14.3208310155497	-0.854678681331973\\
14.3227056205773	-0.854678468433923\\
14.3245802254579	-0.854678255610358\\
14.3264548301916	-0.854678042861241\\
14.3283294347784	-0.854677830186538\\
14.3302040392184	-0.854677617586215\\
14.3320786435116	-0.854677405060237\\
14.333953247658	-0.854677192608568\\
14.3358278516578	-0.854676980231175\\
14.337702455511	-0.854676767928022\\
14.3395770592176	-0.854676555699075\\
14.3414516627776	-0.854676343544299\\
14.3433262661912	-0.854676131463659\\
14.3452008694584	-0.854675919457122\\
14.3470754725792	-0.854675707524652\\
14.3489500755537	-0.854675495666214\\
14.3508246783819	-0.854675283881774\\
14.3526992810639	-0.854675072171299\\
14.3545738835997	-0.854674860534752\\
14.3564484859894	-0.8546746489721\\
14.358323088233	-0.854674437483308\\
14.3601976903306	-0.854674226068341\\
14.3620722922823	-0.854674014727166\\
14.363946894088	-0.854673803459748\\
14.3658214957479	-0.854673592266052\\
14.367696097262	-0.854673381146044\\
14.3695706986302	-0.85467317009969\\
14.3714452998528	-0.854672959126955\\
14.3733199009297	-0.854672748227805\\
14.375194501861	-0.854672537402206\\
14.3770691026468	-0.854672326650123\\
14.378943703287	-0.854672115971522\\
14.3808183037818	-0.85467190536637\\
14.3826929041311	-0.854671694834631\\
14.3845675043351	-0.854671484376272\\
14.3864421043938	-0.854671273991258\\
14.3883167043072	-0.854671063679555\\
14.3901913040754	-0.854670853441129\\
14.3920659036985	-0.854670643275947\\
14.3939405031764	-0.854670433183973\\
14.3958151025093	-0.854670223165174\\
14.3976897016972	-0.854670013219516\\
14.3995643007401	-0.854669803346964\\
14.4014388996381	-0.854669593547485\\
14.4033134983913	-0.854669383821045\\
14.4051880969996	-0.85466917416761\\
14.4070626954632	-0.854668964587145\\
14.4089372937821	-0.854668755079617\\
14.4108118919563	-0.854668545644992\\
14.4126864899859	-0.854668336283236\\
14.4145610878709	-0.854668126994315\\
14.4164356856115	-0.854667917778195\\
14.4183102832075	-0.854667708634843\\
14.4201848806592	-0.854667499564224\\
14.4220594779665	-0.854667290566305\\
14.4239340751295	-0.854667081641052\\
14.4258086721483	-0.854666872788432\\
14.4276832690228	-0.85466666400841\\
14.4295578657531	-0.854666455300952\\
14.4314324623394	-0.854666246666026\\
14.4333070587816	-0.854666038103597\\
14.4351816550798	-0.854665829613632\\
14.437056251234	-0.854665621196097\\
14.4389308472443	-0.854665412850959\\
14.4408054431108	-0.854665204578183\\
14.4426800388334	-0.854664996377737\\
14.4445546344123	-0.854664788249586\\
14.4464292298475	-0.854664580193698\\
14.448303825139	-0.854664372210038\\
14.4501784202869	-0.854664164298573\\
14.4520530152912	-0.854663956459271\\
14.453927610152	-0.854663748692096\\
14.4558022048693	-0.854663540997016\\
14.4576767994433	-0.854663333373997\\
14.4595513938738	-0.854663125823007\\
14.4614259881611	-0.85466291834401\\
14.463300582305	-0.854662710936975\\
14.4651751763058	-0.854662503601868\\
14.4670497701634	-0.854662296338655\\
14.4689243638778	-0.854662089147304\\
14.4707989574492	-0.85466188202778\\
14.4726735508776	-0.854661674980051\\
14.474548144163	-0.854661468004083\\
14.4764227373055	-0.854661261099843\\
14.4782973303051	-0.854661054267298\\
14.4801719231618	-0.854660847506415\\
14.4820465158758	-0.854660640817161\\
14.4839211084471	-0.854660434199502\\
14.4857957008757	-0.854660227653405\\
14.4876702931617	-0.854660021178837\\
14.4895448853051	-0.854659814775765\\
14.4914194773059	-0.854659608444156\\
14.4932940691643	-0.854659402183977\\
14.4951686608802	-0.854659195995194\\
14.4970432524538	-0.854658989877776\\
14.498917843885	-0.854658783831688\\
14.500792435174	-0.854658577856898\\
14.5026670263207	-0.854658371953373\\
14.5045416173252	-0.85465816612108\\
14.5064162081876	-0.854657960359986\\
14.5082907989078	-0.854657754670058\\
14.5101653894861	-0.854657549051263\\
14.5120399799223	-0.854657343503568\\
14.5139145702166	-0.854657138026941\\
14.515789160369	-0.854656932621348\\
14.5176637503796	-0.854656727286757\\
14.5195383402484	-0.854656522023135\\
14.5214129299754	-0.85465631683045\\
14.5232875195607	-0.854656111708668\\
14.5251621090044	-0.854655906657756\\
14.5270366983064	-0.854655701677683\\
14.5289112874669	-0.854655496768415\\
14.5307858764859	-0.854655291929919\\
14.5326604653635	-0.854655087162164\\
14.5345350540996	-0.854654882465116\\
14.5364096426944	-0.854654677838742\\
14.5382842311478	-0.854654473283011\\
14.54015881946	-0.854654268797889\\
14.542033407631	-0.854654064383344\\
14.5439079956608	-0.854653860039344\\
14.5457825835495	-0.854653655765855\\
14.5476571712971	-0.854653451562846\\
14.5495317589037	-0.854653247430284\\
14.5514063463693	-0.854653043368136\\
14.553280933694	-0.85465283937637\\
14.5551555208779	-0.854652635454953\\
14.5570301079208	-0.854652431603854\\
14.5589046948231	-0.854652227823039\\
14.5607792815845	-0.854652024112476\\
14.5626538682053	-0.854651820472134\\
14.5645284546855	-0.854651616901979\\
14.5664030410251	-0.854651413401979\\
14.5682776272241	-0.854651209972102\\
14.5701522132826	-0.854651006612316\\
14.5720267992007	-0.854650803322589\\
14.5739013849784	-0.854650600102887\\
14.5757759706158	-0.85465039695318\\
14.5776505561128	-0.854650193873434\\
14.5795251414696	-0.854649990863618\\
14.5813997266862	-0.854649787923699\\
14.5832743117627	-0.854649585053645\\
14.585148896699	-0.854649382253424\\
14.5870234814953	-0.854649179523004\\
14.5888980661515	-0.854648976862353\\
14.5907726506678	-0.854648774271439\\
14.5926472350442	-0.85464857175023\\
14.5945218192807	-0.854648369298693\\
14.5963964033774	-0.854648166916797\\
14.5982709873343	-0.854647964604509\\
14.6001455711515	-0.854647762361798\\
14.602020154829	-0.854647560188632\\
14.6038947383669	-0.854647358084978\\
14.6057693217652	-0.854647156050805\\
14.607643905024	-0.854646954086081\\
14.6095184881432	-0.854646752190774\\
14.6113930711231	-0.854646550364851\\
14.6132676539635	-0.854646348608283\\
14.6151422366646	-0.854646146921035\\
14.6170168192264	-0.854645945303077\\
14.618891401649	-0.854645743754377\\
14.6207659839323	-0.854645542274902\\
14.6226405660765	-0.854645340864622\\
14.6245151480816	-0.854645139523504\\
14.6263897299476	-0.854644938251517\\
14.6282643116746	-0.854644737048629\\
14.6301388932627	-0.854644535914808\\
14.6320134747118	-0.854644334850023\\
14.6338880560221	-0.854644133854241\\
14.6357626371935	-0.854643932927432\\
14.6376372182262	-0.854643732069563\\
14.6395117991201	-0.854643531280604\\
14.6413863798754	-0.854643330560522\\
14.643260960492	-0.854643129909286\\
14.6451355409701	-0.854642929326864\\
14.6470101213096	-0.854642728813225\\
14.6488847015106	-0.854642528368338\\
14.6507592815731	-0.85464232799217\\
14.6526338614973	-0.85464212768469\\
14.6545084412831	-0.854641927445867\\
14.6563830209306	-0.85464172727567\\
14.6582576004399	-0.854641527174067\\
14.6601321798109	-0.854641327141026\\
14.6620067590438	-0.854641127176516\\
14.6638813381386	-0.854640927280507\\
14.6657559170953	-0.854640727452965\\
14.6676304959139	-0.854640527693861\\
14.6695050745946	-0.854640328003163\\
14.6713796531374	-0.854640128380839\\
14.6732542315423	-0.854639928826858\\
14.6751288098094	-0.85463972934119\\
14.6770033879386	-0.854639529923802\\
14.6788779659302	-0.854639330574663\\
14.680752543784	-0.854639131293743\\
14.6826271215002	-0.85463893208101\\
14.6845016990788	-0.854638732936433\\
14.6863762765198	-0.85463853385998\\
14.6882508538233	-0.854638334851622\\
14.6901254309894	-0.854638135911325\\
14.6920000080181	-0.85463793703906\\
14.6938745849094	-0.854637738234796\\
14.6957491616633	-0.8546375394985\\
14.69762373828	-0.854637340830143\\
14.6994983147595	-0.854637142229693\\
14.7013728911018	-0.854636943697119\\
14.703247467307	-0.854636745232391\\
14.705122043375	-0.854636546835476\\
14.7069966193061	-0.854636348506345\\
14.7088711951001	-0.854636150244966\\
14.7107457707572	-0.854635952051308\\
14.7126203462774	-0.854635753925341\\
14.7144949216607	-0.854635555867033\\
14.7163694969072	-0.854635357876354\\
14.718244072017	-0.854635159953272\\
14.7201186469901	-0.854634962097758\\
14.7219932218264	-0.85463476430978\\
14.7238677965262	-0.854634566589307\\
14.7257423710894	-0.854634368936309\\
14.727616945516	-0.854634171350755\\
14.7294915198062	-0.854633973832613\\
14.7313660939599	-0.854633776381854\\
14.7332406679773	-0.854633578998447\\
14.7351152418583	-0.854633381682361\\
14.736989815603	-0.854633184433564\\
14.7388643892114	-0.854632987252028\\
14.7407389626836	-0.85463279013772\\
14.7426135360197	-0.854632593090611\\
14.7444881092197	-0.854632396110669\\
14.7463626822836	-0.854632199197864\\
14.7482372552114	-0.854632002352166\\
14.7501118280033	-0.854631805573544\\
14.7519864006593	-0.854631608861968\\
14.7538609731794	-0.854631412217406\\
14.7557355455636	-0.854631215639828\\
14.7576101178121	-0.854631019129205\\
14.7594846899248	-0.854630822685505\\
14.7613592619018	-0.854630626308698\\
14.7632338337431	-0.854630429998754\\
14.7651084054489	-0.854630233755642\\
14.7669829770191	-0.854630037579331\\
14.7688575484537	-0.854629841469792\\
14.7707321197529	-0.854629645426994\\
14.7726066909167	-0.854629449450906\\
14.774481261945	-0.854629253541499\\
14.7763558328381	-0.854629057698742\\
14.7782304035959	-0.854628861922605\\
14.7801049742184	-0.854628666213057\\
14.7819795447057	-0.854628470570068\\
14.7838541150579	-0.854628274993608\\
14.7857286852749	-0.854628079483647\\
14.7876032553569	-0.854627884040154\\
14.7894778253039	-0.854627688663099\\
14.791352395116	-0.854627493352453\\
14.7932269647931	-0.854627298108185\\
14.7951015343353	-0.854627102930264\\
14.7969761037427	-0.854626907818662\\
14.7988506730153	-0.854626712773346\\
14.8007252421531	-0.854626517794289\\
14.8025998111563	-0.854626322881458\\
14.8044743800248	-0.854626128034825\\
14.8063489487587	-0.85462593325436\\
14.8082235173581	-0.854625738540031\\
14.8100980858229	-0.85462554389181\\
14.8119726541533	-0.854625349309667\\
14.8138472223492	-0.85462515479357\\
14.8157217904108	-0.854624960343491\\
14.817596358338	-0.8546247659594\\
14.819470926131	-0.854624571641266\\
14.8213454937897	-0.85462437738906\\
14.8232200613142	-0.854624183202751\\
14.8250946287045	-0.85462398908231\\
14.8269691959608	-0.854623795027708\\
14.828843763083	-0.854623601038913\\
14.8307183300711	-0.854623407115897\\
14.8325928969253	-0.85462321325863\\
14.8344674636456	-0.854623019467081\\
14.836342030232	-0.854622825741221\\
14.8382165966846	-0.854622632081021\\
14.8400911630033	-0.85462243848645\\
14.8419657291884	-0.854622244957479\\
14.8438402952397	-0.854622051494078\\
14.8457148611573	-0.854621858096217\\
14.8475894269414	-0.854621664763867\\
14.8494639925919	-0.854621471496999\\
14.8513385581088	-0.854621278295582\\
14.8532131234923	-0.854621085159586\\
14.8550876887424	-0.854620892088983\\
14.8569622538591	-0.854620699083743\\
14.8588368188424	-0.854620506143836\\
14.8607113836924	-0.854620313269232\\
14.8625859484092	-0.854620120459903\\
14.8644605129928	-0.854619927715818\\
14.8663350774432	-0.854619735036948\\
14.8682096417605	-0.854619542423263\\
14.8700842059447	-0.854619349874735\\
14.8719587699959	-0.854619157391333\\
14.8738333339141	-0.854618964973029\\
14.8757078976994	-0.854618772619792\\
14.8775824613518	-0.854618580331594\\
14.8794570248713	-0.854618388108404\\
14.881331588258	-0.854618195950195\\
14.883206151512	-0.854618003856935\\
14.8850807146333	-0.854617811828597\\
14.8869552776218	-0.85461761986515\\
14.8888298404778	-0.854617427966566\\
14.8907044032012	-0.854617236132815\\
14.892578965792	-0.854617044363867\\
14.8944535282503	-0.854616852659694\\
14.8963280905762	-0.854616661020267\\
14.8982026527697	-0.854616469445555\\
14.9000772148308	-0.854616277935531\\
14.9019517767596	-0.854616086490164\\
14.9038263385562	-0.854615895109426\\
14.9057009002205	-0.854615703793287\\
14.9075754617526	-0.854615512541718\\
14.9094500231526	-0.854615321354691\\
14.9113245844205	-0.854615130232176\\
14.9131991455564	-0.854614939174143\\
14.9150737065602	-0.854614748180565\\
14.9169482674321	-0.854614557251411\\
14.918822828172	-0.854614366386653\\
14.9206973887801	-0.854614175586263\\
14.9225719492564	-0.854613984850209\\
14.9244465096009	-0.854613794178465\\
14.9263210698136	-0.854613603571001\\
14.9281956298947	-0.854613413027787\\
14.9300701898441	-0.854613222548795\\
14.9319447496618	-0.854613032133997\\
14.9338193093481	-0.854612841783362\\
14.9356938689028	-0.854612651496863\\
14.937568428326	-0.85461246127447\\
14.9394429876178	-0.854612271116155\\
14.9413175467782	-0.854612081021888\\
14.9431921058072	-0.854611890991642\\
14.945066664705	-0.854611701025386\\
14.9469412234715	-0.854611511123092\\
14.9488157821068	-0.854611321284732\\
14.9506903406109	-0.854611131510277\\
14.9525648989839	-0.854610941799698\\
14.9544394572258	-0.854610752152966\\
14.9563140153367	-0.854610562570053\\
14.9581885733166	-0.854610373050929\\
14.9600631311655	-0.854610183595567\\
};
\addplot [color=red,solid,forget plot]
  table[row sep=crcr]{%
14.9600631311655	-0.854610183595567\\
14.9619376888836	-0.854609994203937\\
14.9638122464707	-0.854609804876011\\
14.9656868039271	-0.85460961561176\\
14.9675613612527	-0.854609426411156\\
14.9694359184475	-0.85460923727417\\
14.9713104755117	-0.854609048200774\\
14.9731850324452	-0.854608859190938\\
14.9750595892481	-0.854608670244635\\
14.9769341459205	-0.854608481361836\\
14.9788087024623	-0.854608292542511\\
14.9806832588737	-0.854608103786634\\
14.9825578151547	-0.854607915094175\\
14.9844323713052	-0.854607726465106\\
14.9863069273255	-0.854607537899398\\
14.9881814832154	-0.854607349397023\\
14.9900560389751	-0.854607160957953\\
14.9919305946046	-0.854606972582159\\
14.9938051501039	-0.854606784269613\\
14.9956797054731	-0.854606596020286\\
14.9975542607122	-0.85460640783415\\
14.9994288158213	-0.854606219711177\\
15.0013033708004	-0.854606031651338\\
15.0031779256495	-0.854605843654605\\
15.0050524803688	-0.85460565572095\\
15.0069270349582	-0.854605467850344\\
15.0088015894177	-0.85460528004276\\
15.0106761437475	-0.854605092298169\\
15.0125506979476	-0.854604904616542\\
15.0144252520179	-0.854604716997852\\
15.0162998059587	-0.85460452944207\\
15.0181743597698	-0.854604341949169\\
15.0200489134513	-0.854604154519119\\
15.0219234670034	-0.854603967151893\\
15.023798020426	-0.854603779847464\\
15.0256725737191	-0.854603592605801\\
15.0275471268829	-0.854603405426878\\
15.0294216799173	-0.854603218310667\\
15.0312962328224	-0.854603031257138\\
15.0331707855983	-0.854602844266266\\
15.0350453382449	-0.85460265733802\\
15.0369198907624	-0.854602470472373\\
15.0387944431507	-0.854602283669298\\
15.04066899541	-0.854602096928766\\
15.0425435475402	-0.85460191025075\\
15.0444180995414	-0.85460172363522\\
15.0462926514137	-0.85460153708215\\
15.048167203157	-0.854601350591511\\
15.0500417547715	-0.854601164163276\\
15.0519163062572	-0.854600977797416\\
15.053790857614	-0.854600791493904\\
15.0556654088421	-0.854600605252712\\
15.0575399599416	-0.854600419073812\\
15.0594145109123	-0.854600232957176\\
15.0612890617545	-0.854600046902777\\
15.0631636124681	-0.854599860910585\\
15.0650381630531	-0.854599674980575\\
15.0669127135097	-0.854599489112718\\
15.0687872638378	-0.854599303306986\\
15.0706618140375	-0.854599117563351\\
15.0725363641089	-0.854598931881786\\
15.074410914052	-0.854598746262263\\
15.0762854638667	-0.854598560704755\\
15.0781600135533	-0.854598375209233\\
15.0800345631116	-0.85459818977567\\
15.0819091125419	-0.854598004404038\\
15.083783661844	-0.85459781909431\\
15.085658211018	-0.854597633846458\\
15.0875327600641	-0.854597448660455\\
15.0894073089821	-0.854597263536273\\
15.0912818577723	-0.854597078473884\\
15.0931564064345	-0.85459689347326\\
15.0950309549689	-0.854596708534375\\
15.0969055033755	-0.854596523657201\\
15.0987800516543	-0.854596338841709\\
15.1006545998054	-0.854596154087874\\
15.1025291478288	-0.854595969395667\\
15.1044036957246	-0.85459578476506\\
15.1062782434928	-0.854595600196027\\
15.1081527911334	-0.854595415688539\\
15.1100273386466	-0.85459523124257\\
15.1119018860322	-0.854595046858092\\
15.1137764332905	-0.854594862535078\\
15.1156509804213	-0.8545946782735\\
15.1175255274248	-0.854594494073331\\
15.1194000743011	-0.854594309934544\\
15.12127462105	-0.85459412585711\\
15.1231491676718	-0.854593941841004\\
15.1250237141664	-0.854593757886197\\
15.1268982605338	-0.854593573992663\\
15.1287728067742	-0.854593390160374\\
15.1306473528875	-0.854593206389302\\
15.1325218988738	-0.854593022679422\\
15.1343964447331	-0.854592839030704\\
15.1362709904656	-0.854592655443123\\
15.1381455360711	-0.854592471916651\\
15.1400200815498	-0.854592288451261\\
15.1418946269017	-0.854592105046925\\
15.1437691721269	-0.854591921703617\\
15.1456437172254	-0.854591738421309\\
15.1475182621972	-0.854591555199975\\
15.1493928070423	-0.854591372039586\\
15.1512673517609	-0.854591188940117\\
15.1531418963529	-0.85459100590154\\
15.1550164408185	-0.854590822923827\\
15.1568909851575	-0.854590640006953\\
15.1587655293702	-0.854590457150889\\
15.1606400734565	-0.85459027435561\\
15.1625146174164	-0.854590091621087\\
15.1643891612501	-0.854589908947294\\
15.1662637049575	-0.854589726334204\\
15.1681382485387	-0.85458954378179\\
15.1700127919937	-0.854589361290026\\
15.1718873353226	-0.854589178858883\\
15.1737618785255	-0.854588996488336\\
15.1756364216022	-0.854588814178356\\
15.177510964553	-0.854588631928919\\
15.1793855073778	-0.854588449739996\\
15.1812600500767	-0.854588267611561\\
15.1831345926498	-0.854588085543586\\
15.185009135097	-0.854587903536046\\
15.1868836774184	-0.854587721588914\\
15.188758219614	-0.854587539702161\\
15.1906327616839	-0.854587357875763\\
15.1925073036282	-0.854587176109691\\
15.1943818454469	-0.85458699440392\\
15.1962563871399	-0.854586812758423\\
15.1981309287074	-0.854586631173172\\
15.2000054701494	-0.854586449648141\\
15.201880011466	-0.854586268183304\\
15.2037545526571	-0.854586086778633\\
15.2056290937229	-0.854585905434103\\
15.2075036346633	-0.854585724149686\\
15.2093781754784	-0.854585542925357\\
15.2112527161682	-0.854585361761087\\
15.2131272567329	-0.854585180656851\\
15.2150017971724	-0.854584999612622\\
15.2168763374867	-0.854584818628374\\
15.2187508776759	-0.85458463770408\\
15.2206254177401	-0.854584456839713\\
15.2224999576793	-0.854584276035247\\
15.2243744974936	-0.854584095290656\\
15.2262490371829	-0.854583914605913\\
15.2281235767473	-0.854583733980991\\
15.2299981161869	-0.854583553415864\\
15.2318726555016	-0.854583372910505\\
15.2337471946916	-0.854583192464889\\
15.2356217337569	-0.854583012078989\\
15.2374962726975	-0.854582831752778\\
15.2393708115135	-0.854582651486229\\
15.2412453502049	-0.854582471279318\\
15.2431198887717	-0.854582291132016\\
15.244994427214	-0.854582111044299\\
15.2468689655318	-0.854581931016139\\
15.2487435037252	-0.85458175104751\\
15.2506180417942	-0.854581571138386\\
15.2524925797388	-0.854581391288741\\
15.2543671175592	-0.854581211498548\\
15.2562416552552	-0.854581031767781\\
15.2581161928271	-0.854580852096414\\
15.2599907302747	-0.854580672484421\\
15.2618652675982	-0.854580492931775\\
15.2637398047976	-0.854580313438451\\
15.2656143418729	-0.854580134004421\\
15.2674888788242	-0.854579954629661\\
15.2693634156515	-0.854579775314143\\
15.2712379523549	-0.854579596057841\\
15.2731124889344	-0.854579416860731\\
15.27498702539	-0.854579237722784\\
15.2768615617218	-0.854579058643976\\
15.2787360979298	-0.85457887962428\\
15.280610634014	-0.85457870066367\\
15.2824851699746	-0.85457852176212\\
15.2843597058115	-0.854578342919604\\
15.2862342415247	-0.854578164136096\\
15.2881087771144	-0.85457798541157\\
15.2899833125806	-0.854577806746\\
15.2918578479232	-0.85457762813936\\
15.2937323831425	-0.854577449591624\\
15.2956069182382	-0.854577271102767\\
15.2974814532106	-0.854577092672761\\
15.2993559880597	-0.854576914301582\\
15.3012305227855	-0.854576735989202\\
15.303105057388	-0.854576557735598\\
15.3049795918673	-0.854576379540742\\
15.3068541262234	-0.854576201404608\\
15.3087286604564	-0.854576023327171\\
15.3106031945663	-0.854575845308406\\
15.3124777285532	-0.854575667348285\\
15.314352262417	-0.854575489446784\\
15.3162267961579	-0.854575311603876\\
15.3181013297758	-0.854575133819536\\
15.3199758632708	-0.854574956093739\\
15.321850396643	-0.854574778426457\\
15.3237249298924	-0.854574600817666\\
15.325599463019	-0.854574423267339\\
15.3274739960228	-0.854574245775452\\
15.329348528904	-0.854574068341978\\
15.3312230616625	-0.854573890966892\\
15.3330975942984	-0.854573713650168\\
15.3349721268118	-0.85457353639178\\
15.3368466592026	-0.854573359191703\\
15.3387211914709	-0.854573182049911\\
15.3405957236168	-0.854573004966378\\
15.3424702556402	-0.85457282794108\\
15.3443447875413	-0.854572650973989\\
15.34621931932	-0.854572474065082\\
15.3480938509765	-0.854572297214331\\
15.3499683825107	-0.854572120421713\\
15.3518429139227	-0.8545719436872\\
15.3537174452125	-0.854571767010768\\
15.3555919763802	-0.854571590392391\\
15.3574665074258	-0.854571413832044\\
15.3593410383493	-0.854571237329701\\
15.3612155691509	-0.854571060885336\\
15.3630900998304	-0.854570884498925\\
15.364964630388	-0.854570708170441\\
15.3668391608238	-0.85457053189986\\
15.3687136911376	-0.854570355687156\\
15.3705882213297	-0.854570179532304\\
15.3724627514	-0.854570003435278\\
15.3743372813486	-0.854569827396053\\
15.3762118111754	-0.854569651414603\\
15.3780863408806	-0.854569475490904\\
15.3799608704642	-0.854569299624929\\
15.3818353999262	-0.854569123816654\\
15.3837099292667	-0.854568948066053\\
15.3855844584857	-0.854568772373102\\
15.3874589875832	-0.854568596737774\\
15.3893335165593	-0.854568421160044\\
15.3912080454141	-0.854568245639888\\
15.3930825741475	-0.85456807017728\\
15.3949571027596	-0.854567894772195\\
15.3968316312504	-0.854567719424607\\
15.39870615962	-0.854567544134492\\
15.4005806878685	-0.854567368901824\\
15.4024552159958	-0.854567193726579\\
15.404329744002	-0.85456701860873\\
15.4062042718871	-0.854566843548253\\
15.4080787996512	-0.854566668545123\\
15.4099533272944	-0.854566493599315\\
15.4118278548166	-0.854566318710803\\
15.4137023822178	-0.854566143879563\\
15.4155769094983	-0.85456596910557\\
15.4174514366579	-0.854565794388798\\
15.4193259636967	-0.854565619729222\\
15.4212004906147	-0.854565445126818\\
15.4230750174121	-0.85456527058156\\
15.4249495440888	-0.854565096093423\\
15.4268240706448	-0.854564921662383\\
15.4286985970803	-0.854564747288414\\
15.4305731233952	-0.854564572971492\\
15.4324476495896	-0.854564398711592\\
15.4343221756636	-0.854564224508688\\
15.4361967016171	-0.854564050362756\\
15.4380712274502	-0.854563876273771\\
15.4399457531629	-0.854563702241708\\
15.4418202787554	-0.854563528266542\\
15.4436948042275	-0.854563354348248\\
15.4455693295794	-0.854563180486802\\
15.4474438548112	-0.854563006682179\\
15.4493183799227	-0.854562832934353\\
15.4511929049142	-0.854562659243301\\
15.4530674297855	-0.854562485608996\\
15.4549419545369	-0.854562312031415\\
15.4568164791682	-0.854562138510533\\
15.4586910036795	-0.854561965046325\\
15.460565528071	-0.854561791638766\\
15.4624400523425	-0.854561618287832\\
15.4643145764942	-0.854561444993497\\
15.4661891005261	-0.854561271755738\\
15.4680636244382	-0.854561098574529\\
15.4699381482306	-0.854560925449846\\
15.4718126719033	-0.854560752381664\\
15.4736871954563	-0.854560579369959\\
15.4755617188897	-0.854560406414706\\
15.4774362422036	-0.85456023351588\\
15.4793107653979	-0.854560060673456\\
15.4811852884727	-0.854559887887411\\
15.483059811428	-0.854559715157719\\
15.484934334264	-0.854559542484357\\
15.4868088569805	-0.854559369867299\\
15.4886833795777	-0.85455919730652\\
15.4905579020556	-0.854559024801998\\
15.4924324244143	-0.854558852353706\\
15.4943069466537	-0.854558679961621\\
15.4961814687739	-0.854558507625717\\
15.498055990775	-0.854558335345972\\
15.4999305126569	-0.854558163122359\\
15.5018050344198	-0.854557990954855\\
15.5036795560636	-0.854557818843435\\
15.5055540775885	-0.854557646788075\\
15.5074285989944	-0.854557474788751\\
15.5093031202813	-0.854557302845437\\
15.5111776414494	-0.85455713095811\\
15.5130521624987	-0.854556959126746\\
15.5149266834291	-0.854556787351319\\
15.5168012042408	-0.854556615631806\\
15.5186757249338	-0.854556443968183\\
15.520550245508	-0.854556272360424\\
15.5224247659637	-0.854556100808507\\
15.5242992863007	-0.854555929312406\\
15.5261738065191	-0.854555757872097\\
15.528048326619	-0.854555586487556\\
15.5299228466004	-0.854555415158758\\
15.5317973664633	-0.854555243885681\\
15.5336718862078	-0.854555072668298\\
15.535546405834	-0.854554901506587\\
15.5374209253418	-0.854554730400523\\
15.5392954447313	-0.854554559350081\\
15.5411699640025	-0.854554388355238\\
15.5430444831555	-0.85455421741597\\
15.5449190021903	-0.854554046532252\\
15.5467935211069	-0.85455387570406\\
15.5486680399054	-0.85455370493137\\
15.5505425585859	-0.854553534214158\\
15.5524170771483	-0.854553363552401\\
15.5542915955927	-0.854553192946073\\
15.5561661139192	-0.854553022395151\\
15.5580406321277	-0.854552851899611\\
15.5599151502184	-0.854552681459428\\
15.5617896681912	-0.85455251107458\\
15.5636641860462	-0.854552340745041\\
15.5655387037834	-0.854552170470788\\
15.5674132214029	-0.854552000251797\\
15.5692877389047	-0.854551830088044\\
15.5711622562888	-0.854551659979504\\
15.5730367735553	-0.854551489926155\\
15.5749112907042	-0.854551319927972\\
15.5767858077356	-0.854551149984931\\
15.5786603246495	-0.854550980097008\\
15.5805348414459	-0.854550810264179\\
15.5824093581249	-0.854550640486421\\
15.5842838746864	-0.85455047076371\\
15.5861583911307	-0.854550301096022\\
15.5880329074576	-0.854550131483332\\
15.5899074236672	-0.854549961925618\\
15.5917819397596	-0.854549792422855\\
15.5936564557348	-0.85454962297502\\
15.5955309715928	-0.854549453582088\\
15.5974054873337	-0.854549284244037\\
15.5992800029575	-0.854549114960842\\
15.6011545184642	-0.854548945732479\\
15.603029033854	-0.854548776558925\\
15.6049035491267	-0.854548607440156\\
15.6067780642825	-0.854548438376149\\
15.6086525793215	-0.85454826936688\\
15.6105270942435	-0.854548100412324\\
15.6124016090487	-0.854547931512459\\
15.6142761237372	-0.854547762667261\\
15.6161506383088	-0.854547593876705\\
15.6180251527638	-0.85454742514077\\
15.6198996671021	-0.85454725645943\\
15.6217741813237	-0.854547087832662\\
15.6236486954288	-0.854546919260443\\
15.6255232094173	-0.854546750742749\\
15.6273977232892	-0.854546582279557\\
15.6292722370447	-0.854546413870843\\
15.6311467506837	-0.854546245516583\\
15.6330212642063	-0.854546077216754\\
15.6348957776125	-0.854545908971333\\
15.6367702909024	-0.854545740780295\\
15.638644804076	-0.854545572643619\\
15.6405193171333	-0.854545404561279\\
15.6423938300744	-0.854545236533252\\
15.6442683428993	-0.854545068559516\\
15.646142855608	-0.854544900640046\\
15.6480173682006	-0.85454473277482\\
15.6498918806772	-0.854544564963813\\
15.6517663930377	-0.854544397207003\\
15.6536409052822	-0.854544229504366\\
15.6555154174107	-0.854544061855879\\
15.6573899294233	-0.854543894261518\\
15.65926444132	-0.85454372672126\\
15.6611389531009	-0.854543559235081\\
15.6630134647659	-0.854543391802959\\
15.6648879763152	-0.85454322442487\\
15.6667624877487	-0.854543057100791\\
15.6686369990665	-0.854542889830698\\
15.6705115102687	-0.854542722614569\\
15.6723860213552	-0.854542555452379\\
15.6742605323261	-0.854542388344106\\
15.6761350431814	-0.854542221289726\\
15.6780095539212	-0.854542054289217\\
15.6798840645456	-0.854541887342555\\
15.6817585750545	-0.854541720449716\\
15.683633085448	-0.854541553610679\\
15.6855075957261	-0.854541386825418\\
15.6873821058889	-0.854541220093912\\
15.6892566159364	-0.854541053416138\\
15.6911311258686	-0.854540886792071\\
15.6930056356856	-0.854540720221689\\
15.6948801453874	-0.85454055370497\\
15.696754654974	-0.854540387241888\\
15.6986291644455	-0.854540220832423\\
15.700503673802	-0.85454005447655\\
15.7023781830434	-0.854539888174247\\
15.7042526921698	-0.85453972192549\\
15.7061272011813	-0.854539555730257\\
15.7080017100778	-0.854539389588524\\
15.7098762188594	-0.854539223500269\\
15.7117507275262	-0.854539057465468\\
15.7136252360781	-0.854538891484098\\
15.7154997445153	-0.854538725556137\\
15.7173742528377	-0.854538559681562\\
15.7192487610454	-0.854538393860349\\
15.7211232691384	-0.854538228092476\\
15.7229977771168	-0.854538062377919\\
15.7248722849806	-0.854537896716657\\
15.7267467927298	-0.854537731108665\\
15.7286213003645	-0.854537565553921\\
15.7304958078847	-0.854537400052403\\
15.7323703152904	-0.854537234604087\\
15.7342448225818	-0.85453706920895\\
15.7361193297587	-0.85453690386697\\
15.7379938368213	-0.854536738578124\\
15.7398683437696	-0.854536573342389\\
15.7417428506037	-0.854536408159743\\
15.7436173573235	-0.854536243030161\\
15.745491863929	-0.854536077953623\\
15.7473663704205	-0.854535912930104\\
15.7492408767978	-0.854535747959582\\
15.751115383061	-0.854535583042035\\
15.7529898892102	-0.85453541817744\\
15.7548643952453	-0.854535253365773\\
15.7567389011665	-0.854535088607013\\
15.7586134069737	-0.854534923901136\\
15.760487912667	-0.854534759248121\\
15.7623624182465	-0.854534594647943\\
15.7642369237121	-0.854534430100581\\
15.7661114290639	-0.854534265606013\\
15.7679859343019	-0.854534101164214\\
15.7698604394263	-0.854533936775164\\
15.7717349444369	-0.854533772438838\\
15.7736094493339	-0.854533608155215\\
15.7754839541173	-0.854533443924272\\
15.777358458787	-0.854533279745986\\
15.7792329633433	-0.854533115620336\\
15.781107467786	-0.854532951547297\\
15.7829819721153	-0.854532787526848\\
15.7848564763311	-0.854532623558967\\
15.7867309804335	-0.85453245964363\\
15.7886054844226	-0.854532295780816\\
15.7904799882983	-0.854532131970501\\
15.7923544920608	-0.854531968212664\\
15.7942289957099	-0.854531804507282\\
15.7961034992459	-0.854531640854332\\
15.7979780026687	-0.854531477253792\\
15.7998525059783	-0.85453131370564\\
15.8017270091748	-0.854531150209852\\
15.8036015122583	-0.854530986766408\\
15.8054760152287	-0.854530823375284\\
15.8073505180861	-0.854530660036458\\
15.8092250208305	-0.854530496749907\\
15.811099523462	-0.85453033351561\\
15.8129740259806	-0.854530170333544\\
15.8148485283864	-0.854530007203686\\
15.8167230306793	-0.854529844126015\\
15.8185975328594	-0.854529681100508\\
15.8204720349268	-0.854529518127142\\
15.8223465368815	-0.854529355205896\\
15.8242210387235	-0.854529192336747\\
15.8260955404528	-0.854529029519673\\
15.8279700420696	-0.854528866754652\\
15.8298445435737	-0.854528704041661\\
15.8317190449653	-0.854528541380679\\
15.8335935462445	-0.854528378771683\\
15.8354680474111	-0.85452821621465\\
15.8373425484653	-0.854528053709559\\
15.8392170494072	-0.854527891256388\\
15.8410915502366	-0.854527728855113\\
15.8429660509538	-0.854527566505714\\
15.8448405515587	-0.854527404208169\\
15.8467150520513	-0.854527241962454\\
15.8485895524317	-0.854527079768548\\
15.8504640526999	-0.854526917626428\\
15.8523385528559	-0.854526755536074\\
15.8542130528999	-0.854526593497462\\
15.8560875528318	-0.85452643151057\\
15.8579620526516	-0.854526269575377\\
15.8598365523595	-0.85452610769186\\
15.8617110519553	-0.854525945859998\\
15.8635855514393	-0.854525784079768\\
15.8654600508113	-0.854525622351148\\
15.8673345500715	-0.854525460674117\\
15.8692090492199	-0.854525299048652\\
15.8710835482565	-0.854525137474732\\
15.8729580471813	-0.854524975952334\\
15.8748325459944	-0.854524814481436\\
15.8767070446958	-0.854524653062017\\
15.8785815432856	-0.854524491694055\\
15.8804560417637	-0.854524330377527\\
15.8823305401303	-0.854524169112413\\
15.8842050383853	-0.854524007898689\\
15.8860795365288	-0.854523846736334\\
15.8879540345609	-0.854523685625327\\
15.8898285324815	-0.854523524565645\\
15.8917030302907	-0.854523363557266\\
15.8935775279886	-0.854523202600169\\
15.8954520255751	-0.854523041694332\\
15.8973265230504	-0.854522880839734\\
15.8992010204143	-0.854522720036351\\
15.9010755176671	-0.854522559284163\\
15.9029500148086	-0.854522398583147\\
15.904824511839	-0.854522237933283\\
15.9066990087583	-0.854522077334548\\
15.9085735055665	-0.85452191678692\\
15.9104480022636	-0.854521756290378\\
15.9123224988498	-0.8545215958449\\
15.9141969953249	-0.854521435450465\\
15.9160714916892	-0.85452127510705\\
15.9179459879425	-0.854521114814634\\
15.9198204840849	-0.854520954573195\\
15.9216949801165	-0.854520794382712\\
15.9235694760373	-0.854520634243163\\
15.9254439718473	-0.854520474154526\\
15.9273184675466	-0.854520314116781\\
15.9291929631352	-0.854520154129904\\
15.9310674586131	-0.854519994193875\\
15.9329419539804	-0.854519834308671\\
15.9348164492371	-0.854519674474272\\
15.9366909443832	-0.854519514690656\\
15.9385654394189	-0.854519354957802\\
15.940439934344	-0.854519195275687\\
15.9423144291587	-0.85451903564429\\
15.9441889238629	-0.854518876063589\\
15.9460634184568	-0.854518716533564\\
15.9479379129404	-0.854518557054193\\
15.9498124073136	-0.854518397625454\\
15.9516869015765	-0.854518238247325\\
15.9535613957292	-0.854518078919786\\
15.9554358897717	-0.854517919642815\\
15.957310383704	-0.85451776041639\\
15.9591848775261	-0.85451760124049\\
15.9610593712382	-0.854517442115093\\
15.9629338648402	-0.854517283040179\\
15.9648083583321	-0.854517124015725\\
15.9666828517141	-0.854516965041711\\
15.9685573449861	-0.854516806118114\\
15.9704318381481	-0.854516647244915\\
15.9723063312003	-0.85451648842209\\
15.9741808241426	-0.85451632964962\\
15.976055316975	-0.854516170927482\\
15.9779298096977	-0.854516012255655\\
15.9798043023106	-0.854515853634119\\
15.9816787948138	-0.854515695062851\\
15.9835532872073	-0.85451553654183\\
15.9854277794911	-0.854515378071036\\
15.9873022716654	-0.854515219650447\\
15.98917676373	-0.854515061280041\\
15.9910512556851	-0.854514902959798\\
15.9929257475306	-0.854514744689696\\
15.9948002392667	-0.854514586469714\\
15.9966747308934	-0.85451442829983\\
15.9985492224106	-0.854514270180025\\
16.0004237138184	-0.854514112110275\\
16.0022982051169	-0.854513954090561\\
16.0041726963061	-0.854513796120861\\
16.006047187386	-0.854513638201154\\
16.0079216783567	-0.854513480331419\\
16.0097961692181	-0.854513322511635\\
16.0116706599704	-0.85451316474178\\
16.0135451506135	-0.854513007021834\\
16.0154196411476	-0.854512849351775\\
16.0172941315726	-0.854512691731582\\
16.0191686218885	-0.854512534161235\\
16.0210431120954	-0.854512376640711\\
16.0229176021934	-0.854512219169991\\
16.0247920921824	-0.854512061749054\\
16.0266665820626	-0.854511904377877\\
16.0285410718338	-0.85451174705644\\
16.0304155614963	-0.854511589784723\\
16.0322900510499	-0.854511432562703\\
16.0341645404948	-0.854511275390361\\
16.0360390298309	-0.854511118267674\\
16.0379135190584	-0.854510961194623\\
16.0397880081772	-0.854510804171187\\
16.0416624971874	-0.854510647197343\\
16.043536986089	-0.854510490273072\\
16.045411474882	-0.854510333398353\\
16.0472859635665	-0.854510176573164\\
16.0491604521425	-0.854510019797485\\
16.05103494061	-0.854509863071294\\
16.0529094289692	-0.854509706394572\\
16.0547839172199	-0.854509549767297\\
16.0566584053623	-0.854509393189448\\
16.0585328933964	-0.854509236661004\\
16.0604073813222	-0.854509080181945\\
16.0622818691397	-0.85450892375225\\
16.0641563568491	-0.854508767371898\\
16.0660308444502	-0.854508611040868\\
16.0679053319432	-0.854508454759139\\
16.0697798193281	-0.854508298526691\\
16.0716543066048	-0.854508142343503\\
16.0735287937736	-0.854507986209554\\
16.0754032808343	-0.854507830124824\\
16.0772777677871	-0.854507674089291\\
16.0791522546319	-0.854507518102936\\
16.0810267413688	-0.854507362165736\\
16.0829012279978	-0.854507206277672\\
16.084775714519	-0.854507050438723\\
16.0866502009323	-0.854506894648869\\
16.0885246872379	-0.854506738908087\\
16.0903991734357	-0.854506583216359\\
16.0922736595259	-0.854506427573663\\
16.0941481455083	-0.854506271979979\\
16.0960226313831	-0.854506116435286\\
16.0978971171503	-0.854505960939563\\
16.0997716028099	-0.85450580549279\\
16.101646088362	-0.854505650094947\\
16.1035205738066	-0.854505494746012\\
16.1053950591437	-0.854505339445965\\
16.1072695443734	-0.854505184194786\\
16.1091440294956	-0.854505028992454\\
16.1110185145105	-0.854504873838948\\
16.112892999418	-0.854504718734249\\
16.1147674842183	-0.854504563678335\\
16.1166419689112	-0.854504408671186\\
16.118516453497	-0.854504253712781\\
16.1203909379755	-0.854504098803101\\
16.1222654223468	-0.854503943942124\\
16.1241399066111	-0.85450378912983\\
16.1260143907682	-0.854503634366199\\
16.1278888748182	-0.85450347965121\\
16.1297633587612	-0.854503324984843\\
16.1316378425972	-0.854503170367078\\
16.1335123263262	-0.854503015797893\\
16.1353868099483	-0.854502861277269\\
16.1372612934635	-0.854502706805186\\
16.1391357768718	-0.854502552381622\\
16.1410102601732	-0.854502398006558\\
16.1428847433679	-0.854502243679973\\
16.1447592264558	-0.854502089401847\\
16.1466337094369	-0.854501935172159\\
16.1485081923114	-0.85450178099089\\
16.1503826750792	-0.854501626858018\\
16.1522571577403	-0.854501472773525\\
16.1541316402949	-0.854501318737388\\
16.1560061227428	-0.854501164749589\\
16.1578806050843	-0.854501010810106\\
16.1597550873192	-0.85450085691892\\
16.1616295694476	-0.85450070307601\\
16.1635040514696	-0.854500549281357\\
16.1653785333853	-0.854500395534939\\
16.1672530151945	-0.854500241836737\\
16.1691274968974	-0.85450008818673\\
16.171001978494	-0.854499934584899\\
16.1728764599843	-0.854499781031222\\
16.1747509413684	-0.854499627525681\\
16.1766254226463	-0.854499474068255\\
16.178499903818	-0.854499320658923\\
16.1803743848835	-0.854499167297665\\
16.182248865843	-0.854499013984462\\
16.1841233466964	-0.854498860719294\\
16.1859978274437	-0.854498707502139\\
16.1878723080851	-0.854498554332979\\
16.1897467886204	-0.854498401211792\\
16.1916212690498	-0.85449824813856\\
16.1934957493733	-0.854498095113261\\
16.195370229591	-0.854497942135876\\
16.1972447097028	-0.854497789206385\\
16.1991191897088	-0.854497636324768\\
16.200993669609	-0.854497483491005\\
16.2028681494034	-0.854497330705075\\
16.2047426290922	-0.854497177966959\\
16.2066171086752	-0.854497025276636\\
16.2084915881527	-0.854496872634088\\
16.2103660675245	-0.854496720039293\\
16.2122405467907	-0.854496567492232\\
16.2141150259514	-0.854496414992886\\
16.2159895050066	-0.854496262541233\\
16.2178639839562	-0.854496110137254\\
16.2197384628005	-0.854495957780929\\
16.2216129415393	-0.854495805472239\\
16.2234874201728	-0.854495653211163\\
16.2253618987009	-0.854495500997681\\
16.2272363771236	-0.854495348831774\\
16.2291108554411	-0.854495196713421\\
16.2309853336534	-0.854495044642604\\
16.2328598117604	-0.854494892619301\\
16.2347342897622	-0.854494740643494\\
16.2366087676589	-0.854494588715162\\
16.2384832454505	-0.854494436834286\\
16.2403577231369	-0.854494285000846\\
16.2422322007183	-0.854494133214821\\
16.2441066781947	-0.854493981476193\\
16.2459811555661	-0.854493829784941\\
16.2478556328326	-0.854493678141046\\
16.2497301099941	-0.854493526544488\\
16.2516045870507	-0.854493374995247\\
16.2534790640025	-0.854493223493304\\
16.2553535408494	-0.854493072038638\\
16.2572280175916	-0.854492920631231\\
16.2591024942289	-0.854492769271062\\
16.2609769707616	-0.854492617958111\\
16.2628514471895	-0.85449246669236\\
16.2647259235128	-0.854492315473787\\
16.2666003997315	-0.854492164302375\\
16.2684748758455	-0.854492013178103\\
16.270349351855	-0.854491862100951\\
16.2722238277599	-0.854491711070899\\
16.2740983035604	-0.854491560087929\\
16.2759727792563	-0.854491409152021\\
16.2778472548479	-0.854491258263154\\
16.279721730335	-0.85449110742131\\
16.2815962057178	-0.854490956626469\\
16.2834706809962	-0.854490805878611\\
16.2853451561703	-0.854490655177716\\
16.2872196312401	-0.854490504523766\\
16.2890941062057	-0.854490353916741\\
16.290968581067	-0.85449020335662\\
16.2928430558242	-0.854490052843385\\
16.2947175304773	-0.854489902377016\\
16.2965920050262	-0.854489751957494\\
16.298466479471	-0.854489601584799\\
16.3003409538118	-0.854489451258912\\
16.3022154280486	-0.854489300979813\\
16.3040899021814	-0.854489150747482\\
16.3059643762102	-0.854489000561901\\
16.3078388501351	-0.85448885042305\\
16.3097133239561	-0.854488700330909\\
16.3115877976733	-0.854488550285459\\
16.3134622712866	-0.854488400286681\\
16.3153367447962	-0.854488250334555\\
16.3172112182019	-0.854488100429062\\
16.319085691504	-0.854487950570182\\
16.3209601647023	-0.854487800757897\\
16.322834637797	-0.854487650992186\\
16.3247091107881	-0.854487501273031\\
16.3265835836755	-0.854487351600412\\
16.3284580564594	-0.85448720197431\\
16.3303325291398	-0.854487052394705\\
16.3322070017166	-0.854486902861579\\
16.33408147419	-0.854486753374911\\
16.3359559465599	-0.854486603934683\\
16.3378304188264	-0.854486454540876\\
16.3397048909895	-0.854486305193469\\
16.3415793630493	-0.854486155892444\\
16.3434538350058	-0.854486006637782\\
16.345328306859	-0.854485857429464\\
16.3472027786089	-0.854485708267469\\
16.3490772502557	-0.854485559151779\\
16.3509517217992	-0.854485410082376\\
16.3528261932396	-0.854485261059238\\
16.3547006645769	-0.854485112082348\\
16.356575135811	-0.854484963151687\\
16.3584496069422	-0.854484814267234\\
16.3603240779702	-0.854484665428971\\
16.3621985488953	-0.854484516636879\\
16.3640730197175	-0.854484367890939\\
16.3659474904367	-0.854484219191131\\
16.367821961053	-0.854484070537437\\
16.3696964315664	-0.854483921929836\\
16.371570901977	-0.854483773368311\\
16.3734453722848	-0.854483624852843\\
16.3753198424898	-0.854483476383411\\
16.377194312592	-0.854483327959997\\
16.3790687825916	-0.854483179582582\\
16.3809432524885	-0.854483031251147\\
16.3828177222827	-0.854482882965673\\
16.3846921919743	-0.854482734726141\\
16.3865666615634	-0.854482586532532\\
16.3884411310499	-0.854482438384826\\
16.3903156004338	-0.854482290283006\\
16.3921900697153	-0.854482142227051\\
16.3940645388943	-0.854481994216943\\
16.3959390079709	-0.854481846252663\\
16.3978134769451	-0.854481698334192\\
16.399687945817	-0.854481550461511\\
16.4015624145865	-0.854481402634602\\
16.4034368832537	-0.854481254853444\\
16.4053113518186	-0.85448110711802\\
16.4071858202813	-0.85448095942831\\
16.4090602886418	-0.854480811784295\\
16.4109347569002	-0.854480664185957\\
16.4128092250563	-0.854480516633277\\
16.4146836931104	-0.854480369126236\\
16.4165581610624	-0.854480221664815\\
16.4184326289123	-0.854480074248995\\
16.4203070966603	-0.854479926878757\\
16.4221815643062	-0.854479779554082\\
16.4240560318502	-0.854479632274953\\
16.4259304992923	-0.854479485041349\\
16.4278049666324	-0.854479337853252\\
16.4296794338707	-0.854479190710644\\
16.4315539010072	-0.854479043613505\\
16.4334283680419	-0.854478896561817\\
16.4353028349748	-0.854478749555561\\
16.437177301806	-0.854478602594718\\
16.4390517685355	-0.85447845567927\\
16.4409262351633	-0.854478308809197\\
16.4428007016895	-0.854478161984481\\
16.4446751681141	-0.854478015205104\\
16.446549634437	-0.854477868471047\\
16.4484241006585	-0.85447772178229\\
16.4502985667784	-0.854477575138816\\
16.4521730327968	-0.854477428540606\\
16.4540474987138	-0.85447728198764\\
16.4559219645294	-0.854477135479901\\
16.4577964302435	-0.854476989017369\\
16.4596708958563	-0.854476842600026\\
16.4615453613678	-0.854476696227854\\
16.463419826778	-0.854476549900834\\
16.4652942920869	-0.854476403618946\\
16.4671687572946	-0.854476257382174\\
16.469043222401	-0.854476111190497\\
16.4709176874063	-0.854475965043898\\
16.4727921523105	-0.854475818942358\\
16.4746666171135	-0.854475672885858\\
16.4765410818155	-0.854475526874379\\
16.4784155464164	-0.854475380907905\\
16.4802900109162	-0.854475234986414\\
16.4821644753151	-0.85447508910989\\
16.4840389396131	-0.854474943278314\\
16.4859134038101	-0.854474797491667\\
16.4877878679062	-0.854474651749931\\
16.4896623319015	-0.854474506053087\\
16.4915367957959	-0.854474360401116\\
16.4934112595895	-0.854474214794001\\
16.4952857232824	-0.854474069231723\\
16.4971601868745	-0.854473923714263\\
16.4990346503658	-0.854473778241603\\
16.5009091137566	-0.854473632813725\\
16.5027835770466	-0.85447348743061\\
16.5046580402361	-0.85447334209224\\
16.5065325033249	-0.854473196798596\\
16.5084069663132	-0.85447305154966\\
16.510281429201	-0.854472906345413\\
16.5121558919883	-0.854472761185838\\
16.5140303546751	-0.854472616070916\\
16.5159048172615	-0.854472471000628\\
16.5177792797475	-0.854472325974956\\
16.5196537421331	-0.854472180993883\\
16.5215282044184	-0.854472036057389\\
16.5234026666033	-0.854471891165456\\
16.525277128688	-0.854471746318066\\
16.5271515906725	-0.8544716015152\\
16.5290260525567	-0.854471456756842\\
16.5309005143407	-0.854471312042971\\
16.5327749760246	-0.85447116737357\\
16.5346494376083	-0.854471022748621\\
16.5365238990919	-0.854470878168105\\
16.5383983604755	-0.854470733632004\\
16.5402728217591	-0.8544705891403\\
16.5421472829426	-0.854470444692974\\
16.5440217440262	-0.85447030029001\\
16.5458962050098	-0.854470155931387\\
16.5477706658935	-0.854470011617089\\
16.5496451266773	-0.854469867347096\\
16.5515195873613	-0.854469723121391\\
16.5533940479455	-0.854469578939956\\
16.5552685084298	-0.854469434802773\\
16.5571429688144	-0.854469290709822\\
16.5590174290993	-0.854469146661087\\
16.5608918892845	-0.854469002656549\\
16.56276634937	-0.85446885869619\\
16.5646408093559	-0.854468714779991\\
16.5665152692421	-0.854468570907936\\
16.5683897290288	-0.854468427080005\\
16.5702641887159	-0.854468283296181\\
16.5721386483035	-0.854468139556445\\
16.5740131077917	-0.854467995860779\\
16.5758875671803	-0.854467852209166\\
16.5777620264696	-0.854467708601588\\
16.5796364856594	-0.854467565038026\\
16.5815109447499	-0.854467421518462\\
16.5833854037411	-0.854467278042878\\
16.5852598626329	-0.854467134611257\\
16.5871343214255	-0.854466991223581\\
16.5890087801188	-0.85446684787983\\
16.5908832387129	-0.854466704579988\\
16.5927576972079	-0.854466561324036\\
16.5946321556036	-0.854466418111957\\
16.5965066139003	-0.854466274943733\\
16.5983810720978	-0.854466131819345\\
16.6002555301963	-0.854465988738775\\
16.6021299881958	-0.854465845702006\\
16.6040044460962	-0.854465702709021\\
16.6058789038977	-0.8544655597598\\
16.6077533616003	-0.854465416854326\\
16.6096278192039	-0.854465273992581\\
16.6115022767087	-0.854465131174548\\
16.6133767341145	-0.854464988400208\\
16.6152511914216	-0.854464845669544\\
16.6171256486299	-0.854464702982537\\
16.6190001057394	-0.854464560339171\\
16.6208745627502	-0.854464417739426\\
16.6227490196623	-0.854464275183286\\
16.6246234764757	-0.854464132670732\\
16.6264979331905	-0.854463990201747\\
16.6283723898066	-0.854463847776312\\
16.6302468463242	-0.854463705394411\\
16.6321213027432	-0.854463563056025\\
16.6339957590637	-0.854463420761136\\
16.6358702152857	-0.854463278509728\\
16.6377446714093	-0.854463136301781\\
16.6396191274344	-0.854462994137279\\
16.6414935833611	-0.854462852016203\\
16.6433680391894	-0.854462709938536\\
16.6452424949194	-0.854462567904261\\
16.6471169505511	-0.854462425913359\\
16.6489914060845	-0.854462283965812\\
16.6508658615196	-0.854462142061604\\
16.6527403168566	-0.854462000200716\\
16.6546147720953	-0.854461858383131\\
16.6564892272359	-0.854461716608831\\
16.6583636822783	-0.854461574877798\\
16.6602381372226	-0.854461433190015\\
16.6621125920689	-0.854461291545465\\
16.6639870468171	-0.854461149944129\\
16.6658615014673	-0.85446100838599\\
16.6677359560195	-0.85446086687103\\
16.6696104104737	-0.854460725399232\\
16.67148486483	-0.854460583970579\\
16.6733593190885	-0.854460442585052\\
16.675233773249	-0.854460301242634\\
16.6771082273118	-0.854460159943307\\
16.6789826812767	-0.854460018687055\\
16.6808571351438	-0.854459877473859\\
16.6827315889132	-0.854459736303702\\
16.6846060425849	-0.854459595176566\\
16.6864804961589	-0.854459454092434\\
16.6883549496352	-0.854459313051289\\
16.6902294030139	-0.854459172053112\\
16.692103856295	-0.854459031097887\\
16.6939783094785	-0.854458890185595\\
16.6958527625645	-0.854458749316221\\
16.697727215553	-0.854458608489745\\
16.699601668444	-0.85445846770615\\
16.7014761212375	-0.85445832696542\\
16.7033505739337	-0.854458186267537\\
16.7052250265324	-0.854458045612482\\
16.7070994790338	-0.85445790500024\\
16.7089739314378	-0.854457764430792\\
16.7108483837445	-0.854457623904121\\
16.712722835954	-0.85445748342021\\
16.7145972880662	-0.854457342979041\\
16.7164717400812	-0.854457202580597\\
16.718346191999	-0.85445706222486\\
16.7202206438196	-0.854456921911814\\
16.7220950955431	-0.85445678164144\\
16.7239695471695	-0.854456641413722\\
16.7258439986989	-0.854456501228643\\
16.7277184501312	-0.854456361086184\\
16.7295929014665	-0.854456220986329\\
16.7314673527048	-0.85445608092906\\
16.7333418038462	-0.85445594091436\\
16.7352162548906	-0.854455800942212\\
16.7370907058382	-0.854455661012598\\
16.7389651566889	-0.854455521125501\\
16.7408396074427	-0.854455381280905\\
16.7427140580998	-0.854455241478791\\
16.74458850866	-0.854455101719142\\
16.7464629591235	-0.854454962001942\\
16.7483374094903	-0.854454822327173\\
16.7502118597605	-0.854454682694817\\
16.7520863099339	-0.854454543104858\\
16.7539607600107	-0.854454403557279\\
16.755835209991	-0.854454264052062\\
16.7577096598746	-0.854454124589189\\
16.7595841096617	-0.854453985168645\\
16.7614585593523	-0.854453845790412\\
16.7633330089464	-0.854453706454472\\
16.7652074584441	-0.854453567160808\\
16.7670819078453	-0.854453427909404\\
16.7689563571501	-0.854453288700242\\
16.7708308063586	-0.854453149533305\\
16.7727052554707	-0.854453010408575\\
16.7745797044865	-0.854452871326037\\
16.7764541534061	-0.854452732285673\\
16.7783286022294	-0.854452593287465\\
16.7802030509564	-0.854452454331397\\
16.7820774995873	-0.854452315417451\\
16.783951948122	-0.854452176545611\\
16.7858263965605	-0.85445203771586\\
16.787700844903	-0.85445189892818\\
16.7895752931494	-0.854451760182554\\
16.7914497412997	-0.854451621478966\\
16.793324189354	-0.854451482817398\\
16.7951986373123	-0.854451344197833\\
16.7970730851747	-0.854451205620255\\
16.7989475329411	-0.854451067084646\\
16.8008219806116	-0.854450928590989\\
16.8026964281863	-0.854450790139269\\
16.8045708756651	-0.854450651729466\\
16.8064453230481	-0.854450513361565\\
16.8083197703352	-0.854450375035549\\
16.8101942175267	-0.8544502367514\\
16.8120686646224	-0.854450098509102\\
16.8139431116224	-0.854449960308638\\
16.8158175585267	-0.854449822149991\\
16.8176920053354	-0.854449684033144\\
16.8195664520484	-0.85444954595808\\
16.8214408986659	-0.854449407924782\\
16.8233153451878	-0.854449269933234\\
16.8251897916142	-0.854449131983418\\
16.8270642379451	-0.854448994075318\\
16.8289386841805	-0.854448856208916\\
16.8308131303205	-0.854448718384197\\
16.832687576365	-0.854448580601143\\
16.8345620223142	-0.854448442859737\\
16.836436468168	-0.854448305159962\\
16.8383109139265	-0.854448167501803\\
16.8401853595897	-0.854448029885241\\
16.8420598051576	-0.85444789231026\\
16.8439342506302	-0.854447754776844\\
16.8458086960077	-0.854447617284975\\
16.84768314129	-0.854447479834637\\
16.8495575864771	-0.854447342425813\\
16.8514320315691	-0.854447205058486\\
16.8533064765659	-0.85444706773264\\
16.8551809214677	-0.854446930448258\\
16.8570553662745	-0.854446793205323\\
16.8589298109863	-0.854446656003818\\
16.860804255603	-0.854446518843727\\
16.8626787001248	-0.854446381725033\\
16.8645531445517	-0.85444624464772\\
16.8664275888837	-0.85444610761177\\
16.8683020331208	-0.854445970617167\\
16.8701764772631	-0.854445833663894\\
16.8720509213105	-0.854445696751935\\
16.8739253652632	-0.854445559881273\\
16.8757998091211	-0.854445423051892\\
16.8776742528843	-0.854445286263774\\
16.8795486965527	-0.854445149516903\\
16.8814231401265	-0.854445012811263\\
16.8832975836057	-0.854444876146837\\
16.8851720269902	-0.854444739523608\\
16.8870464702802	-0.854444602941559\\
16.8889209134756	-0.854444466400675\\
16.8907953565764	-0.854444329900939\\
16.8926697995828	-0.854444193442334\\
16.8945442424947	-0.854444057024843\\
16.8964186853121	-0.85444392064845\\
16.8982931280352	-0.854443784313138\\
16.9001675706638	-0.854443648018892\\
16.9020420131981	-0.854443511765693\\
16.903916455638	-0.854443375553527\\
16.9057908979836	-0.854443239382376\\
16.907665340235	-0.854443103252224\\
16.9095397823921	-0.854442967163054\\
16.911414224455	-0.85444283111485\\
16.9132886664237	-0.854442695107595\\
16.9151631082982	-0.854442559141273\\
16.9170375500786	-0.854442423215868\\
16.9189119917649	-0.854442287331363\\
16.9207864333571	-0.854442151487742\\
16.9226608748553	-0.854442015684988\\
16.9245353162594	-0.854441879923085\\
16.9264097575696	-0.854441744202015\\
16.9282841987858	-0.854441608521764\\
16.930158639908	-0.854441472882315\\
16.9320330809364	-0.854441337283651\\
16.9339075218708	-0.854441201725755\\
16.9357819627114	-0.854441066208612\\
16.9376564034582	-0.854440930732205\\
16.9395308441111	-0.854440795296518\\
16.9414052846703	-0.854440659901534\\
16.9432797251358	-0.854440524547237\\
16.9451541655075	-0.854440389233611\\
16.9470286057856	-0.854440253960639\\
16.94890304597	-0.854440118728306\\
16.9507774860608	-0.854439983536594\\
16.952651926058	-0.854439848385487\\
16.9545263659616	-0.85443971327497\\
16.9564008057716	-0.854439578205026\\
16.9582752454881	-0.854439443175638\\
16.9601496851112	-0.854439308186791\\
16.9620241246408	-0.854439173238468\\
16.9638985640769	-0.854439038330652\\
16.9657730034197	-0.854438903463329\\
16.967647442669	-0.854438768636481\\
16.969521881825	-0.854438633850092\\
16.9713963208877	-0.854438499104145\\
16.9732707598571	-0.854438364398626\\
16.9751451987332	-0.854438229733517\\
16.9770196375161	-0.854438095108803\\
16.9788940762058	-0.854437960524467\\
16.9807685148023	-0.854437825980493\\
16.9826429533056	-0.854437691476864\\
16.9845173917158	-0.854437557013566\\
16.9863918300329	-0.854437422590581\\
16.988266268257	-0.854437288207893\\
16.990140706388	-0.854437153865487\\
16.9920151444259	-0.854437019563346\\
16.9938895823709	-0.854436885301453\\
16.995764020223	-0.854436751079794\\
16.997638457982	-0.854436616898352\\
16.9995128956482	-0.85443648275711\\
17.0013873332215	-0.854436348656053\\
17.003261770702	-0.854436214595164\\
17.0051362080896	-0.854436080574428\\
17.0070106453845	-0.854435946593829\\
17.0088850825865	-0.854435812653349\\
17.0107595196959	-0.854435678752975\\
17.0126339567125	-0.854435544892688\\
17.0145083936364	-0.854435411072474\\
17.0163828304677	-0.854435277292316\\
17.0182572672063	-0.854435143552198\\
17.0201317038524	-0.854435009852104\\
17.0220061404058	-0.854434876192019\\
17.0238805768667	-0.854434742571926\\
17.0257550132351	-0.854434608991809\\
17.0276294495111	-0.854434475451653\\
17.0295038856945	-0.854434341951441\\
17.0313783217855	-0.854434208491157\\
17.0332527577841	-0.854434075070786\\
17.0351271936903	-0.854433941690311\\
17.0370016295042	-0.854433808349716\\
17.0388760652257	-0.854433675048987\\
17.040750500855	-0.854433541788106\\
17.0426249363919	-0.854433408567057\\
17.0444993718366	-0.854433275385826\\
17.0463738071891	-0.854433142244396\\
17.0482482424495	-0.854433009142751\\
17.0501226776176	-0.854432876080875\\
17.0519971126936	-0.854432743058753\\
17.0538715476775	-0.854432610076368\\
17.0557459825693	-0.854432477133705\\
17.0576204173691	-0.854432344230747\\
17.0594948520769	-0.85443221136748\\
17.0613692866926	-0.854432078543886\\
17.0632437212164	-0.854431945759951\\
17.0651181556482	-0.854431813015659\\
17.0669925899882	-0.854431680310993\\
17.0688670242362	-0.854431547645938\\
17.0707414583923	-0.854431415020479\\
17.0726158924567	-0.854431282434598\\
17.0744903264292	-0.854431149888282\\
17.07636476031	-0.854431017381513\\
17.0782391940989	-0.854430884914276\\
17.0801136277962	-0.854430752486555\\
17.0819880614018	-0.854430620098335\\
17.0838624949157	-0.8544304877496\\
17.0857369283379	-0.854430355440334\\
17.0876113616686	-0.854430223170521\\
17.0894857949076	-0.854430090940146\\
17.0913602280551	-0.854429958749193\\
17.0932346611111	-0.854429826597646\\
17.0951090940755	-0.85442969448549\\
17.0969835269485	-0.854429562412709\\
17.09885795973	-0.854429430379286\\
17.1007323924201	-0.854429298385208\\
17.1026068250187	-0.854429166430457\\
17.1044812575261	-0.854429034515019\\
17.106355689942	-0.854428902638877\\
17.1082301222667	-0.854428770802016\\
17.1101045545	-0.854428639004421\\
17.1119789866421	-0.854428507246075\\
17.113853418693	-0.854428375526963\\
17.1157278506526	-0.85442824384707\\
17.1176022825211	-0.85442811220638\\
17.1194767142984	-0.854427980604877\\
17.1213511459845	-0.854427849042546\\
17.1232255775796	-0.854427717519371\\
17.1251000090836	-0.854427586035337\\
17.1269744404966	-0.854427454590428\\
17.1288488718185	-0.854427323184628\\
17.1307233030494	-0.854427191817923\\
17.1325977341894	-0.854427060490295\\
17.1344721652384	-0.854426929201731\\
17.1363465961965	-0.854426797952214\\
17.1382210270638	-0.854426666741729\\
17.1400954578401	-0.854426535570261\\
17.1419698885256	-0.854426404437793\\
17.1438443191204	-0.854426273344311\\
17.1457187496243	-0.854426142289799\\
17.1475931800375	-0.854426011274241\\
17.14946761036	-0.854425880297622\\
17.1513420405918	-0.854425749359927\\
17.1532164707329	-0.85442561846114\\
17.1550909007833	-0.854425487601246\\
17.1569653307431	-0.854425356780229\\
17.1588397606124	-0.854425225998073\\
17.1607141903911	-0.854425095254765\\
17.1625886200792	-0.854424964550287\\
17.1644630496768	-0.854424833884624\\
17.166337479184	-0.854424703257762\\
17.1682119086007	-0.854424572669685\\
17.1700863379269	-0.854424442120377\\
17.1719607671628	-0.854424311609823\\
17.1738351963083	-0.854424181138008\\
17.1757096253634	-0.854424050704916\\
17.1775840543282	-0.854423920310532\\
17.1794584832027	-0.854423789954841\\
17.1813329119869	-0.854423659637827\\
17.1832073406809	-0.854423529359475\\
17.1850817692847	-0.85442339911977\\
17.1869561977983	-0.854423268918696\\
17.1888306262217	-0.854423138756238\\
17.1907050545549	-0.854423008632381\\
17.1925794827981	-0.854422878547109\\
17.1944539109512	-0.854422748500408\\
17.1963283390142	-0.854422618492261\\
17.1982027669872	-0.854422488522655\\
17.2000771948702	-0.854422358591572\\
17.2019516226632	-0.854422228698999\\
17.2038260503662	-0.85442209884492\\
17.2057004779794	-0.854421969029319\\
17.2075749055026	-0.854421839252182\\
17.209449332936	-0.854421709513493\\
17.2113237602795	-0.854421579813237\\
17.2131981875332	-0.854421450151399\\
17.2150726146971	-0.854421320527963\\
17.2169470417712	-0.854421190942915\\
17.2188214687556	-0.85442106139624\\
17.2206958956503	-0.854420931887921\\
17.2225703224553	-0.854420802417944\\
17.2244447491706	-0.854420672986294\\
17.2263191757963	-0.854420543592955\\
17.2281936023324	-0.854420414237913\\
17.2300680287789	-0.854420284921153\\
17.2319424551358	-0.854420155642658\\
17.2338168814033	-0.854420026402415\\
17.2356913075812	-0.854419897200407\\
17.2375657336696	-0.85441976803662\\
17.2394401596686	-0.854419638911039\\
17.2413145855782	-0.854419509823649\\
17.2431890113984	-0.854419380774435\\
17.2450634371292	-0.85441925176338\\
17.2469378627706	-0.854419122790472\\
17.2488122883228	-0.854418993855693\\
17.2506867137856	-0.85441886495903\\
17.2525611391592	-0.854418736100467\\
17.2544355644436	-0.85441860727999\\
17.2563099896387	-0.854418478497582\\
17.2581844147446	-0.85441834975323\\
17.2600588397614	-0.854418221046918\\
17.261933264689	-0.854418092378631\\
17.2638076895276	-0.854417963748354\\
17.265682114277	-0.854417835156072\\
17.2675565389374	-0.854417706601771\\
17.2694309635088	-0.854417578085434\\
17.2713053879912	-0.854417449607048\\
17.2731798123846	-0.854417321166597\\
17.275054236689	-0.854417192764066\\
17.2769286609045	-0.854417064399441\\
17.2788030850311	-0.854416936072706\\
17.2806775090689	-0.854416807783847\\
17.2825519330178	-0.854416679532848\\
17.2844263568778	-0.854416551319694\\
17.2863007806491	-0.854416423144371\\
17.2881752043316	-0.854416295006864\\
17.2900496279254	-0.854416166907158\\
17.2919240514304	-0.854416038845238\\
17.2937984748468	-0.854415910821089\\
17.2956728981745	-0.854415782834697\\
17.2975473214135	-0.854415654886045\\
17.299421744564	-0.85441552697512\\
17.3012961676258	-0.854415399101907\\
17.3031705905991	-0.854415271266391\\
17.3050450134839	-0.854415143468556\\
17.3069194362801	-0.854415015708389\\
17.3087938589879	-0.854414887985873\\
17.3106682816072	-0.854414760300996\\
17.3125427041381	-0.854414632653741\\
17.3144171265806	-0.854414505044094\\
17.3162915489346	-0.854414377472039\\
17.3181659712004	-0.854414249937564\\
17.3200403933778	-0.854414122440651\\
17.3219148154669	-0.854413994981288\\
17.3237892374678	-0.854413867559458\\
17.3256636593803	-0.854413740175147\\
17.3275380812047	-0.854413612828341\\
17.3294125029409	-0.854413485519025\\
17.3312869245889	-0.854413358247184\\
17.3331613461487	-0.854413231012802\\
17.3350357676204	-0.854413103815867\\
17.3369101890041	-0.854412976656362\\
17.3387846102996	-0.854412849534273\\
17.3406590315072	-0.854412722449585\\
17.3425334526267	-0.854412595402285\\
17.3444078736582	-0.854412468392356\\
17.3462822946017	-0.854412341419784\\
17.3481567154573	-0.854412214484555\\
17.350031136225	-0.854412087586654\\
17.3519055569048	-0.854411960726067\\
17.3537799774968	-0.854411833902778\\
17.3556543980009	-0.854411707116773\\
17.3575288184172	-0.854411580368037\\
17.3594032387457	-0.854411453656556\\
17.3612776589864	-0.854411326982315\\
17.3631520791394	-0.8544112003453\\
17.3650264992047	-0.854411073745496\\
17.3669009191823	-0.854410947182888\\
17.3687753390723	-0.854410820657461\\
17.3706497588746	-0.854410694169202\\
17.3725241785893	-0.854410567718095\\
17.3743985982165	-0.854410441304127\\
17.3762730177561	-0.854410314927281\\
17.3781474372081	-0.854410188587545\\
17.3800218565727	-0.854410062284902\\
17.3818962758497	-0.85440993601934\\
17.3837706950394	-0.854409809790843\\
17.3856451141416	-0.854409683599396\\
17.3875195331564	-0.854409557444985\\
17.3893939520838	-0.854409431327597\\
17.3912683709238	-0.854409305247215\\
17.3931427896766	-0.854409179203826\\
17.395017208342	-0.854409053197415\\
17.3968916269202	-0.854408927227968\\
17.3987660454111	-0.854408801295469\\
17.4006404638148	-0.854408675399906\\
17.4025148821313	-0.854408549541263\\
17.4043893003606	-0.854408423719525\\
17.4062637185028	-0.854408297934679\\
17.4081381365578	-0.85440817218671\\
17.4100125545258	-0.854408046475603\\
17.4118869724067	-0.854407920801344\\
17.4137613902005	-0.854407795163919\\
17.4156358079073	-0.854407669563312\\
17.4175102255271	-0.854407543999511\\
17.41938464306	-0.8544074184725\\
17.4212590605059	-0.854407292982265\\
17.4231334778649	-0.854407167528792\\
17.425007895137	-0.854407042112065\\
17.4268823123222	-0.854406916732072\\
17.4287567294206	-0.854406791388797\\
17.4306311464322	-0.854406666082226\\
17.432505563357	-0.854406540812344\\
17.434379980195	-0.854406415579139\\
17.4362543969462	-0.854406290382594\\
17.4381288136108	-0.854406165222696\\
17.4400032301887	-0.85440604009943\\
17.4418776466799	-0.854405915012782\\
17.4437520630844	-0.854405789962739\\
17.4456264794024	-0.854405664949284\\
17.4475008956337	-0.854405539972405\\
17.4493753117785	-0.854405415032087\\
17.4512497278367	-0.854405290128315\\
17.4531241438085	-0.854405165261075\\
17.4549985596937	-0.854405040430354\\
17.4568729754925	-0.854404915636136\\
17.4587473912048	-0.854404790878408\\
17.4606218068308	-0.854404666157156\\
17.4624962223703	-0.854404541472364\\
17.4643706378235	-0.854404416824019\\
17.4662450531903	-0.854404292212107\\
17.4681194684708	-0.854404167636613\\
17.4699938836651	-0.854404043097523\\
17.471868298773	-0.854403918594823\\
17.4737427137948	-0.854403794128499\\
17.4756171287303	-0.854403669698536\\
17.4774915435796	-0.854403545304921\\
17.4793659583428	-0.854403420947639\\
17.4812403730198	-0.854403296626676\\
17.4831147876107	-0.854403172342018\\
17.4849892021156	-0.85440304809365\\
17.4868636165343	-0.854402923881559\\
17.488738030867	-0.854402799705731\\
17.4906124451138	-0.854402675566151\\
17.4924868592745	-0.854402551462804\\
17.4943612733493	-0.854402427395678\\
17.4962356873381	-0.854402303364758\\
17.498110101241	-0.85440217937003\\
17.499984515058	-0.854402055411479\\
17.5018589287892	-0.854401931489092\\
17.5037333424345	-0.854401807602854\\
17.505607755994	-0.854401683752752\\
17.5074821694677	-0.854401559938771\\
17.5093565828557	-0.854401436160897\\
17.5112309961579	-0.854401312419117\\
17.5131054093744	-0.854401188713415\\
17.5149798225052	-0.854401065043779\\
17.5168542355504	-0.854400941410194\\
17.5187286485099	-0.854400817812646\\
17.5206030613838	-0.854400694251121\\
17.5224774741721	-0.854400570725605\\
17.5243518868749	-0.854400447236084\\
17.5262262994921	-0.854400323782544\\
17.5281007120238	-0.854400200364971\\
17.52997512447	-0.85440007698335\\
17.5318495368307	-0.854399953637669\\
17.533723949106	-0.854399830327913\\
17.5355983612959	-0.854399707054068\\
17.5374727734003	-0.85439958381612\\
17.5393471854195	-0.854399460614055\\
17.5412215973532	-0.854399337447859\\
17.5430960092017	-0.854399214317518\\
17.5449704209649	-0.854399091223019\\
17.5468448326428	-0.854398968164347\\
17.5487192442354	-0.854398845141489\\
17.5505936557429	-0.85439872215443\\
17.5524680671651	-0.854398599203156\\
17.5543424785022	-0.854398476287655\\
17.5562168897541	-0.854398353407911\\
17.5580913009209	-0.854398230563911\\
17.5599657120027	-0.854398107755641\\
17.5618401229993	-0.854397984983087\\
17.5637145339109	-0.854397862246235\\
17.5655889447375	-0.854397739545072\\
17.5674633554791	-0.854397616879583\\
17.5693377661357	-0.854397494249755\\
17.5712121767074	-0.854397371655574\\
17.5730865871941	-0.854397249097026\\
17.5749609975959	-0.854397126574097\\
17.5768354079129	-0.854397004086773\\
17.578709818145	-0.854396881635041\\
17.5805842282923	-0.854396759218887\\
17.5824586383548	-0.854396636838297\\
17.5843330483325	-0.854396514493257\\
17.5862074582255	-0.854396392183753\\
17.5880818680337	-0.854396269909771\\
17.5899562777573	-0.854396147671299\\
17.5918306873962	-0.854396025468321\\
17.5937050969504	-0.854395903300825\\
17.5955795064199	-0.854395781168796\\
17.5974539158049	-0.854395659072221\\
17.5993283251053	-0.854395537011086\\
17.6012027343212	-0.854395414985377\\
17.6030771434525	-0.854395292995081\\
17.6049515524993	-0.854395171040184\\
17.6068259614616	-0.854395049120671\\
17.6087003703395	-0.85439492723653\\
17.6105747791329	-0.854394805387747\\
17.6124491878419	-0.854394683574308\\
17.6143235964666	-0.854394561796199\\
17.6161980050068	-0.854394440053406\\
17.6180724134628	-0.854394318345917\\
17.6199468218344	-0.854394196673717\\
17.6218212301217	-0.854394075036792\\
17.6236956383248	-0.854393953435129\\
17.6255700464437	-0.854393831868715\\
17.6274444544783	-0.854393710337535\\
17.6293188624287	-0.854393588841577\\
17.631193270295	-0.854393467380825\\
17.6330676780771	-0.854393345955268\\
17.6349420857751	-0.85439322456489\\
17.6368164933891	-0.854393103209679\\
17.6386909009189	-0.854392981889621\\
17.6405653083647	-0.854392860604703\\
17.6424397157265	-0.85439273935491\\
17.6443141230043	-0.854392618140229\\
17.6461885301981	-0.854392496960647\\
17.648062937308	-0.85439237581615\\
17.6499373443339	-0.854392254706724\\
17.651811751276	-0.854392133632356\\
17.6536861581341	-0.854392012593033\\
17.6555605649084	-0.85439189158874\\
17.6574349715989	-0.854391770619464\\
17.6593093782056	-0.854391649685193\\
17.6611837847285	-0.854391528785911\\
17.6630581911677	-0.854391407921606\\
17.6649325975231	-0.854391287092264\\
17.6668070037948	-0.854391166297872\\
17.6686814099829	-0.854391045538416\\
17.6705558160873	-0.854390924813883\\
17.672430222108	-0.854390804124259\\
17.6743046280451	-0.854390683469531\\
17.6761790338987	-0.854390562849684\\
17.6780534396687	-0.854390442264707\\
17.6799278453551	-0.854390321714585\\
17.6818022509581	-0.854390201199305\\
17.6836766564775	-0.854390080718853\\
17.6855510619135	-0.854389960273216\\
17.687425467266	-0.85438983986238\\
17.6892998725351	-0.854389719486333\\
17.6911742777209	-0.85438959914506\\
17.6930486828232	-0.854389478838549\\
17.6949230878422	-0.854389358566785\\
17.6967974927779	-0.854389238329756\\
17.6986718976303	-0.854389118127448\\
17.7005463023994	-0.854388997959848\\
17.7024207070853	-0.854388877826942\\
17.7042951116879	-0.854388757728717\\
17.7061695162074	-0.854388637665159\\
17.7080439206436	-0.854388517636255\\
17.7099183249967	-0.854388397641993\\
17.7117927292667	-0.854388277682358\\
17.7136671334536	-0.854388157757337\\
17.7155415375573	-0.854388037866916\\
17.7174159415781	-0.854387918011083\\
17.7192903455158	-0.854387798189825\\
17.7211647493704	-0.854387678403127\\
17.7230391531421	-0.854387558650976\\
17.7249135568309	-0.85438743893336\\
17.7267879604366	-0.854387319250265\\
17.7286623639595	-0.854387199601677\\
17.7305367673995	-0.854387079987584\\
17.7324111707566	-0.854386960407972\\
17.7342855740309	-0.854386840862827\\
17.7361599772223	-0.854386721352137\\
17.738034380331	-0.854386601875888\\
17.7399087833568	-0.854386482434068\\
17.7417831863	-0.854386363026662\\
17.7436575891604	-0.854386243653657\\
17.745531991938	-0.854386124315041\\
17.7474063946331	-0.854386005010799\\
17.7492807972454	-0.85438588574092\\
17.7511551997752	-0.854385766505389\\
17.7530296022223	-0.854385647304194\\
17.7549040045868	-0.85438552813732\\
17.7567784068688	-0.854385409004756\\
17.7586528090682	-0.854385289906488\\
17.7605272111852	-0.854385170842502\\
17.7624016132196	-0.854385051812786\\
17.7642760151716	-0.854384932817326\\
17.7661504170411	-0.854384813856109\\
17.7680248188282	-0.854384694929122\\
17.7698992205329	-0.854384576036351\\
17.7717736221553	-0.854384457177785\\
17.7736480236953	-0.854384338353409\\
17.775522425153	-0.85438421956321\\
17.7773968265283	-0.854384100807175\\
17.7792712278214	-0.854383982085292\\
17.7811456290323	-0.854383863397547\\
17.7830200301609	-0.854383744743926\\
17.7848944312073	-0.854383626124418\\
17.7867688321715	-0.854383507539008\\
17.7886432330536	-0.854383388987684\\
17.7905176338535	-0.854383270470432\\
17.7923920345713	-0.85438315198724\\
17.794266435207	-0.854383033538094\\
17.7961408357607	-0.854382915122982\\
17.7980152362323	-0.854382796741889\\
17.7998896366219	-0.854382678394804\\
17.8017640369295	-0.854382560081713\\
17.8036384371552	-0.854382441802604\\
17.8055128372989	-0.854382323557462\\
17.8073872373606	-0.854382205346276\\
17.8092616373405	-0.854382087169031\\
17.8111360372385	-0.854381969025716\\
17.8130104370546	-0.854381850916316\\
17.8148848367889	-0.85438173284082\\
17.8167592364414	-0.854381614799214\\
17.8186336360121	-0.854381496791485\\
17.820508035501	-0.854381378817619\\
17.8223824349082	-0.854381260877605\\
17.8242568342337	-0.854381142971429\\
17.8261312334775	-0.854381025099079\\
17.8280056326396	-0.85438090726054\\
17.8298800317201	-0.854380789455801\\
17.831754430719	-0.854380671684848\\
17.8336288296363	-0.854380553947668\\
17.835503228472	-0.854380436244249\\
17.8373776272261	-0.854380318574577\\
17.8392520258987	-0.85438020093864\\
17.8411264244898	-0.854380083336425\\
17.8430008229995	-0.854379965767918\\
17.8448752214276	-0.854379848233107\\
17.8467496197744	-0.854379730731979\\
17.8486240180397	-0.854379613264522\\
17.8504984162236	-0.854379495830721\\
17.8523728143262	-0.854379378430565\\
17.8542472123474	-0.85437926106404\\
17.8561216102873	-0.854379143731134\\
17.8579960081459	-0.854379026431833\\
17.8598704059233	-0.854378909166126\\
17.8617448036193	-0.854378791933998\\
17.8636192012342	-0.854378674735438\\
17.8654935987679	-0.854378557570432\\
17.8673679962203	-0.854378440438967\\
17.8692423935917	-0.854378323341031\\
17.8711167908819	-0.854378206276612\\
17.8729911880909	-0.854378089245695\\
17.8748655852189	-0.854377972248268\\
17.8767399822659	-0.854377855284319\\
17.8786143792317	-0.854377738353835\\
17.8804887761166	-0.854377621456802\\
17.8823631729205	-0.854377504593209\\
17.8842375696434	-0.854377387763042\\
17.8861119662853	-0.854377270966288\\
17.8879863628463	-0.854377154202936\\
17.8898607593265	-0.854377037472971\\
17.8917351557257	-0.854376920776382\\
17.8936095520441	-0.854376804113155\\
17.8954839482816	-0.854376687483279\\
17.8973583444384	-0.854376570886739\\
17.8992327405143	-0.854376454323524\\
17.9011071365095	-0.85437633779362\\
17.9029815324239	-0.854376221297015\\
17.9048559282577	-0.854376104833697\\
17.9067303240107	-0.854375988403652\\
17.908604719683	-0.854375872006868\\
17.9104791152748	-0.854375755643332\\
17.9123535107858	-0.854375639313032\\
17.9142279062163	-0.854375523015955\\
17.9161023015662	-0.854375406752087\\
17.9179766968355	-0.854375290521417\\
17.9198510920243	-0.854375174323932\\
17.9217254871326	-0.854375058159619\\
17.9235998821604	-0.854374942028466\\
17.9254742771077	-0.85437482593046\\
17.9273486719746	-0.854374709865588\\
17.929223066761	-0.854374593833837\\
17.9310974614671	-0.854374477835195\\
17.9329718560928	-0.85437436186965\\
17.9348462506381	-0.854374245937189\\
17.9367206451031	-0.854374130037799\\
17.9385950394878	-0.854374014171467\\
17.9404694337922	-0.854373898338181\\
17.9423438280164	-0.854373782537929\\
17.9442182221603	-0.854373666770698\\
17.946092616224	-0.854373551036475\\
17.9479670102075	-0.854373435335247\\
17.9498414041108	-0.854373319667003\\
17.951715797934	-0.854373204031729\\
17.953590191677	-0.854373088429413\\
17.95546458534	-0.854372972860043\\
17.9573389789229	-0.854372857323606\\
17.9592133724257	-0.854372741820089\\
17.9610877658485	-0.85437262634948\\
17.9629621591912	-0.854372510911766\\
17.964836552454	-0.854372395506936\\
17.9667109456368	-0.854372280134975\\
17.9685853387397	-0.854372164795873\\
17.9704597317626	-0.854372049489615\\
17.9723341247057	-0.854371934216191\\
17.9742085175689	-0.854371818975587\\
17.9760829103522	-0.854371703767791\\
17.9779573030557	-0.85437158859279\\
17.9798316956794	-0.854371473450573\\
17.9817060882232	-0.854371358341126\\
17.9835804806874	-0.854371243264436\\
17.9854548730718	-0.854371128220493\\
17.9873292653765	-0.854371013209283\\
17.9892036576014	-0.854370898230793\\
17.9910780497468	-0.854370783285012\\
17.9929524418124	-0.854370668371926\\
17.9948268337984	-0.854370553491524\\
17.9967012257049	-0.854370438643793\\
17.9985756175317	-0.854370323828721\\
18.000450009279	-0.854370209046295\\
18.0023244009467	-0.854370094296502\\
18.004198792535	-0.854369979579331\\
18.0060731840437	-0.85436986489477\\
18.007947575473	-0.854369750242805\\
18.0098219668228	-0.854369635623424\\
18.0116963580932	-0.854369521036615\\
18.0135707492842	-0.854369406482366\\
18.0154451403958	-0.854369291960664\\
18.0173195314281	-0.854369177471497\\
18.019193922381	-0.854369063014852\\
18.0210683132546	-0.854368948590718\\
18.022942704049	-0.854368834199081\\
18.024817094764	-0.854368719839931\\
18.0266914853998	-0.854368605513253\\
18.0285658759564	-0.854368491219036\\
18.0304402664338	-0.854368376957268\\
18.032314656832	-0.854368262727936\\
18.0341890471511	-0.854368148531029\\
18.036063437391	-0.854368034366533\\
18.0379378275518	-0.854367920234436\\
18.0398122176335	-0.854367806134727\\
18.0416866076362	-0.854367692067392\\
18.0435609975598	-0.85436757803242\\
18.0454353874044	-0.854367464029799\\
18.04730977717	-0.854367350059516\\
18.0491841668566	-0.854367236121558\\
18.0510585564642	-0.854367122215914\\
18.052932945993	-0.854367008342572\\
18.0548073354428	-0.854366894501519\\
18.0566817248137	-0.854366780692743\\
18.0585561141058	-0.854366666916231\\
18.060430503319	-0.854366553171972\\
18.0623048924534	-0.854366439459953\\
18.064179281509	-0.854366325780162\\
18.0660536704858	-0.854366212132588\\
18.0679280593839	-0.854366098517216\\
18.0698024482032	-0.854365984934037\\
18.0716768369438	-0.854365871383037\\
18.0735512256058	-0.854365757864204\\
18.075425614189	-0.854365644377526\\
18.0773000026937	-0.854365530922991\\
18.0791743911197	-0.854365417500586\\
18.0810487794671	-0.8543653041103\\
18.0829231677359	-0.85436519075212\\
18.0847975559262	-0.854365077426035\\
18.0866719440379	-0.854364964132032\\
18.0885463320711	-0.854364850870099\\
18.0904207200259	-0.854364737640223\\
18.0922951079022	-0.854364624442394\\
18.0941694957	-0.854364511276598\\
18.0960438834194	-0.854364398142824\\
18.0979182710604	-0.854364285041059\\
18.099792658623	-0.854364171971291\\
18.1016670461073	-0.854364058933509\\
18.1035414335133	-0.8543639459277\\
18.1054158208409	-0.854363832953852\\
18.1072902080902	-0.854363720011953\\
18.1091645952613	-0.854363607101991\\
18.1110389823541	-0.854363494223953\\
18.1129133693687	-0.854363381377829\\
18.1147877563051	-0.854363268563606\\
18.1166621431633	-0.854363155781271\\
18.1185365299434	-0.854363043030813\\
18.1204109166453	-0.85436293031222\\
18.1222853032691	-0.85436281762548\\
18.1241596898149	-0.85436270497058\\
18.1260340762825	-0.854362592347508\\
18.1279084626721	-0.854362479756254\\
18.1297828489837	-0.854362367196804\\
18.1316572352173	-0.854362254669147\\
18.1335316213729	-0.85436214217327\\
18.1354060074505	-0.854362029709162\\
18.1372803934502	-0.854361917276811\\
18.139154779372	-0.854361804876205\\
18.1410291652159	-0.854361692507331\\
18.1429035509819	-0.854361580170178\\
18.14477793667	-0.854361467864734\\
18.1466523222803	-0.854361355590986\\
18.1485267078129	-0.854361243348924\\
18.1504010932676	-0.854361131138535\\
18.1522754786446	-0.854361018959806\\
18.1541498639438	-0.854360906812727\\
18.1560242491653	-0.854360794697285\\
18.1578986343091	-0.854360682613469\\
18.1597730193753	-0.854360570561265\\
18.1616474043638	-0.854360458540664\\
18.1635217892746	-0.854360346551651\\
18.1653961741079	-0.854360234594217\\
18.1672705588635	-0.854360122668348\\
18.1691449435416	-0.854360010774034\\
18.1710193281422	-0.854359898911261\\
18.1728937126652	-0.854359787080018\\
18.1747680971107	-0.854359675280294\\
18.1766424814787	-0.854359563512076\\
18.1785168657693	-0.854359451775353\\
18.1803912499824	-0.854359340070112\\
18.1822656341182	-0.854359228396343\\
18.1841400181765	-0.854359116754032\\
18.1860144021575	-0.854359005143169\\
18.1878887860611	-0.854358893563741\\
18.1897631698874	-0.854358782015736\\
18.1916375536363	-0.854358670499143\\
18.193511937308	-0.854358559013951\\
18.1953863209025	-0.854358447560146\\
18.1972607044197	-0.854358336137718\\
18.1991350878596	-0.854358224746654\\
18.2010094712224	-0.854358113386943\\
18.202883854508	-0.854358002058573\\
18.2047582377164	-0.854357890761532\\
18.2066326208477	-0.854357779495808\\
18.2085070039019	-0.854357668261391\\
18.210381386879	-0.854357557058267\\
18.2122557697791	-0.854357445886425\\
18.2141301526021	-0.854357334745854\\
18.216004535348	-0.854357223636541\\
18.217878918017	-0.854357112558475\\
18.219753300609	-0.854357001511644\\
18.221627683124	-0.854356890496037\\
18.2235020655621	-0.854356779511641\\
18.2253764479232	-0.854356668558446\\
18.2272508302075	-0.854356557636439\\
18.2291252124149	-0.854356446745608\\
18.2309995945454	-0.854356335885942\\
18.2328739765991	-0.85435622505743\\
18.234748358576	-0.854356114260059\\
18.2366227404761	-0.854356003493817\\
18.2384971222994	-0.854355892758694\\
18.240371504046	-0.854355782054678\\
18.2422458857159	-0.854355671381756\\
18.2441202673091	-0.854355560739917\\
18.2459946488256	-0.85435545012915\\
18.2478690302654	-0.854355339549442\\
18.2497434116286	-0.854355229000783\\
18.2516177929151	-0.85435511848316\\
18.2534921741251	-0.854355007996563\\
18.2553665552585	-0.854354897540978\\
18.2572409363154	-0.854354787116395\\
18.2591153172957	-0.854354676722803\\
18.2609896981995	-0.854354566360188\\
18.2628640790268	-0.854354456028541\\
18.2647384597777	-0.854354345727848\\
18.2666128404521	-0.8543542354581\\
18.26848722105	-0.854354125219283\\
18.2703616015716	-0.854354015011387\\
18.2722359820168	-0.854353904834399\\
18.2741103623856	-0.854353794688309\\
18.2759847426781	-0.854353684573105\\
18.2778591228943	-0.854353574488775\\
18.2797335030342	-0.854353464435307\\
18.2816078830978	-0.854353354412691\\
18.2834822630851	-0.854353244420914\\
18.2853566429963	-0.854353134459964\\
18.2872310228312	-0.854353024529832\\
18.2891054025899	-0.854352914630504\\
18.2909797822724	-0.85435280476197\\
18.2928541618788	-0.854352694924217\\
18.2947285414091	-0.854352585117235\\
18.2966029208633	-0.854352475341011\\
18.2984773002414	-0.854352365595535\\
18.3003516795434	-0.854352255880795\\
18.3022260587693	-0.854352146196779\\
18.3041004379193	-0.854352036543476\\
18.3059748169933	-0.854351926920874\\
18.3078491959912	-0.854351817328962\\
18.3097235749133	-0.854351707767728\\
18.3115979537594	-0.854351598237162\\
18.3134723325295	-0.854351488737251\\
18.3153467112238	-0.854351379267984\\
18.3172210898422	-0.854351269829349\\
18.3190954683848	-0.854351160421336\\
18.3209698468515	-0.854351051043932\\
18.3228442252424	-0.854350941697126\\
18.3247186035575	-0.854350832380908\\
18.3265929817969	-0.854350723095264\\
18.3284673599605	-0.854350613840185\\
18.3303417380484	-0.854350504615658\\
18.3322161160606	-0.854350395421673\\
18.3340904939971	-0.854350286258217\\
18.3359648718579	-0.854350177125279\\
18.3378392496431	-0.854350068022849\\
18.3397136273527	-0.854349958950914\\
18.3415880049867	-0.854349849909463\\
18.3434623825451	-0.854349740898485\\
18.345336760028	-0.854349631917969\\
18.3472111374353	-0.854349522967902\\
18.3490855147671	-0.854349414048275\\
18.3509598920234	-0.854349305159074\\
18.3528342692042	-0.85434919630029\\
18.3547086463096	-0.85434908747191\\
18.3565830233395	-0.854348978673924\\
18.3584574002941	-0.85434886990632\\
18.3603317771733	-0.854348761169086\\
18.362206153977	-0.854348652462212\\
18.3640805307055	-0.854348543785686\\
18.3659549073586	-0.854348435139496\\
18.3678292839364	-0.854348326523632\\
18.3697036604389	-0.854348217938082\\
18.3715780368662	-0.854348109382835\\
18.3734524132182	-0.854348000857879\\
18.375326789495	-0.854347892363204\\
18.3772011656966	-0.854347783898798\\
18.3790755418231	-0.854347675464649\\
18.3809499178743	-0.854347567060747\\
18.3828242938505	-0.854347458687079\\
18.3846986697515	-0.854347350343636\\
18.3865730455774	-0.854347242030405\\
18.3884474213282	-0.854347133747376\\
18.390321797004	-0.854347025494537\\
18.3921961726048	-0.854346917271877\\
18.3940705481305	-0.854346809079384\\
18.3959449235813	-0.854346700917048\\
18.397819298957	-0.854346592784857\\
18.3996936742579	-0.854346484682801\\
18.4015680494838	-0.854346376610867\\
18.4034424246348	-0.854346268569044\\
18.4053167997109	-0.854346160557322\\
18.4071911747121	-0.854346052575689\\
18.4090655496385	-0.854345944624134\\
18.4109399244901	-0.854345836702646\\
18.4128142992668	-0.854345728811213\\
18.4146886739688	-0.854345620949825\\
18.416563048596	-0.85434551311847\\
18.4184374231485	-0.854345405317138\\
18.4203117976262	-0.854345297545816\\
18.4221861720293	-0.854345189804494\\
18.4240605463577	-0.854345082093161\\
18.4259349206114	-0.854344974411805\\
18.4278092947904	-0.854344866760415\\
18.4296836688949	-0.854344759138981\\
18.4315580429247	-0.854344651547491\\
18.43343241688	-0.854344543985934\\
18.4353067907607	-0.854344436454299\\
18.4371811645669	-0.854344328952574\\
18.4390555382986	-0.854344221480749\\
18.4409299119558	-0.854344114038813\\
18.4428042855385	-0.854344006626754\\
18.4446786590467	-0.854343899244561\\
18.4465530324806	-0.854343791892224\\
18.44842740584	-0.854343684569731\\
18.450301779125	-0.854343577277071\\
18.4521761523356	-0.854343470014233\\
18.4540505254719	-0.854343362781206\\
18.4559248985339	-0.854343255577978\\
18.4577992715215	-0.85434314840454\\
18.4596736444349	-0.854343041260879\\
18.461548017274	-0.854342934146985\\
18.4634223900389	-0.854342827062847\\
18.4652967627295	-0.854342720008454\\
18.4671711353459	-0.854342612983794\\
18.4690455078881	-0.854342505988856\\
18.4709198803562	-0.854342399023631\\
18.4727942527501	-0.854342292088105\\
18.4746686250699	-0.85434218518227\\
18.4765429973156	-0.854342078306112\\
18.4784173694872	-0.854341971459623\\
18.4802917415848	-0.85434186464279\\
18.4821661136083	-0.854341757855602\\
18.4840404855578	-0.854341651098049\\
18.4859148574333	-0.85434154437012\\
18.4877892292348	-0.854341437671803\\
18.4896636009623	-0.854341331003088\\
18.4915379726159	-0.854341224363964\\
18.4934123441956	-0.854341117754419\\
18.4952867157014	-0.854341011174443\\
18.4971610871333	-0.854340904624025\\
18.4990354584913	-0.854340798103154\\
18.5009098297755	-0.854340691611818\\
18.5027842009859	-0.854340585150008\\
18.5046585721225	-0.854340478717712\\
18.5065329431853	-0.854340372314918\\
18.5084073141744	-0.854340265941617\\
18.5102816850897	-0.854340159597798\\
18.5121560559313	-0.854340053283449\\
18.5140304266992	-0.854339946998559\\
18.5159047973934	-0.854339840743118\\
18.517779168014	-0.854339734517114\\
18.519653538561	-0.854339628320538\\
18.5215279090343	-0.854339522153377\\
18.523402279434	-0.854339416015621\\
18.5252766497602	-0.85433930990726\\
18.5271510200128	-0.854339203828282\\
18.5290253901919	-0.854339097778676\\
18.5308997602974	-0.854338991758432\\
18.5327741303295	-0.854338885767538\\
18.5346485002881	-0.854338779805985\\
18.5365228701732	-0.85433867387376\\
18.538397239985	-0.854338567970854\\
18.5402716097233	-0.854338462097255\\
18.5421459793882	-0.854338356252953\\
18.5440203489797	-0.854338250437936\\
18.5458947184979	-0.854338144652194\\
18.5477690879427	-0.854338038895716\\
18.5496434573143	-0.854337933168492\\
18.5515178266125	-0.85433782747051\\
18.5533921958375	-0.854337721801759\\
18.5552665649893	-0.85433761616223\\
18.5571409340678	-0.854337510551911\\
18.5590153030731	-0.85433740497079\\
18.5608896720052	-0.854337299418859\\
18.5627640408641	-0.854337193896105\\
18.5646384096499	-0.854337088402518\\
18.5665127783625	-0.854336982938087\\
18.5683871470021	-0.854336877502802\\
18.5702615155686	-0.854336772096652\\
18.5721358840619	-0.854336666719625\\
18.5740102524823	-0.854336561371712\\
18.5758846208296	-0.854336456052901\\
18.5777589891039	-0.854336350763182\\
18.5796333573052	-0.854336245502544\\
18.5815077254336	-0.854336140270976\\
18.583382093489	-0.854336035068468\\
18.5852564614714	-0.854335929895008\\
18.587130829381	-0.854335824750587\\
18.5890051972177	-0.854335719635193\\
18.5908795649815	-0.854335614548816\\
18.5927539326724	-0.854335509491445\\
18.5946283002906	-0.854335404463069\\
18.5965026678359	-0.854335299463678\\
18.5983770353084	-0.854335194493261\\
18.6002514027082	-0.854335089551807\\
18.6021257700352	-0.854334984639306\\
18.6040001372895	-0.854334879755747\\
18.605874504471	-0.854334774901119\\
18.6077488715799	-0.854334670075412\\
18.6096232386162	-0.854334565278614\\
18.6114976055797	-0.854334460510716\\
18.6133719724707	-0.854334355771707\\
18.615246339289	-0.854334251061576\\
18.6171207060348	-0.854334146380313\\
18.6189950727079	-0.854334041727906\\
18.6208694393086	-0.854333937104346\\
18.6227438058367	-0.854333832509621\\
18.6246181722923	-0.854333727943721\\
18.6264925386754	-0.854333623406635\\
18.628366904986	-0.854333518898353\\
18.6302412712242	-0.854333414418865\\
18.6321156373899	-0.854333309968159\\
18.6339900034833	-0.854333205546225\\
18.6358643695042	-0.854333101153052\\
18.6377387354528	-0.85433299678863\\
18.6396131013291	-0.854332892452949\\
18.641487467133	-0.854332788145997\\
18.6433618328645	-0.854332683867765\\
18.6452361985239	-0.854332579618241\\
18.6471105641109	-0.854332475397414\\
18.6489849296257	-0.854332371205276\\
18.6508592950682	-0.854332267041814\\
18.6527336604386	-0.854332162907019\\
18.6546080257367	-0.85433205880088\\
18.6564823909627	-0.854331954723386\\
18.6583567561165	-0.854331850674526\\
18.6602311211982	-0.854331746654292\\
18.6621054862078	-0.85433164266267\\
18.6639798511453	-0.854331538699653\\
18.6658542160107	-0.854331434765228\\
18.667728580804	-0.854331330859385\\
18.6696029455254	-0.854331226982114\\
18.6714773101747	-0.854331123133404\\
18.673351674752	-0.854331019313245\\
18.6752260392574	-0.854330915521627\\
18.6771004036907	-0.854330811758538\\
18.6789747680522	-0.854330708023968\\
18.6808491323417	-0.854330604317908\\
18.6827234965594	-0.854330500640346\\
18.6845978607052	-0.854330396991272\\
18.6864722247791	-0.854330293370676\\
18.6883465887811	-0.854330189778546\\
18.6902209527114	-0.854330086214874\\
18.6920953165699	-0.854329982679647\\
18.6939696803565	-0.854329879172857\\
18.6958440440715	-0.854329775694491\\
18.6977184077147	-0.854329672244541\\
18.6995927712861	-0.854329568822995\\
18.7014671347859	-0.854329465429843\\
18.703341498214	-0.854329362065075\\
18.7052158615704	-0.85432925872868\\
18.7070902248552	-0.854329155420648\\
};
\addplot [color=green,solid,forget plot]
  table[row sep=crcr]{%
0	0.1\\
0.0224632231456145	0.100008430943879\\
0.0449223713638281	0.100036398838157\\
0.0673711924404585	0.100083826706216\\
0.0898034789452708	0.100150579605725\\
0.112213096503054	0.100236466138357\\
0.134594011040437	0.10034124060555\\
0.156940314633586	0.100464605754619\\
0.179246249622742	0.100606216045504\\
0.201506230708838	0.100765681357569\\
0.223714864802914	0.100942571048402\\
0.245866968458122	0.1011364182727\\
0.267957582774245	0.101346724468891\\
0.289981985723546	0.101572963923982\\
0.311935701902218	0.101814588332654\\
0.333814509762048	0.102071031274478\\
0.355614446420859	0.102341712542553\\
0.377331810187033	0.102626042267425\\
0.398963160962422	0.10292342479112\\
0.420505318709457	0.10323326225712\\
0.441955360182294	0.103554957892611\\
0.46331061412925	0.103887918969075\\
0.484568655175138	0.10423155943596\\
0.505727296588457	0.104585302229697\\
0.526784582130505	0.104948581266547\\
0.547738777172428	0.105320843132802\\
0.568588359252751	0.105701548489624\\
0.589332008232935	0.106090173212549\\
0.609968596192604	0.106486209287386\\
0.630497177189917	0.106889165485127\\
0.650916976996582	0.107298567838679\\
0.671227382901604	0.107713959943817\\
0.691427933663268	0.108134903105969\\
0.71151830967537	0.108560976353262\\
0.731498323401301	0.108991776334907\\
0.751367910118463	0.109426917122501\\
0.771127119005553	0.109866029930232\\
0.790776104596539	0.110308762768387\\
0.810315118617581	0.110754780043042\\
0.829744502216619	0.111203762113269\\
0.849064678589903	0.111655404815839\\
0.868276146005066	0.112109418966062\\
0.887379471216601	0.112565529842234\\
0.906375283266488	0.113023476660054\\
0.925264267660277	0.11348301204241\\
0.944047160907045	0.113943901489071\\
0.962724745410175	0.114405922850041\\
0.981297844694921	0.114868865805661\\
0.999767318957992	0.11533253135598\\
1.018134060924	0.115796731321366\\
1.03639899199342	0.116261287855943\\
1.05456305866678	0.116726032975019\\
1.07262722922992	0.11719080809739\\
1.09059249068538	0.117655463603117\\
1.10845984591565	0.118119858407148\\
1.12623031106408	0.118583859548989\\
1.14390491312009	0.11904734179845\\
1.16148468769573	0.119510187277378\\
1.17897067698116	0.119972285097187\\
1.19636392786742	0.120433531011912\\
1.21366549022523	0.120893827086442\\
1.23087641532925	0.121353081379548\\
1.24799775441786	0.121811207641278\\
1.26503055737899	0.12226812502426\\
1.28197587155307	0.122723757808449\\
1.29883474064497	0.123178035138827\\
1.3156082037368	0.123630890775565\\
1.33229729439453	0.124082262856162\\
1.34890303986133	0.124532093669064\\
1.36542646033135	0.12498032943829\\
1.38186856829793	0.125426920118587\\
1.39823036797046	0.125871819200653\\
1.41451285475497	0.126314983525973\\
1.43071701479331	0.12675637311084\\
1.44684382455648	0.127195950979127\\
1.46289425048792	0.12763368300341\\
1.47886924869279	0.128069537754042\\
1.49476976466962	0.128503486355807\\
1.51059673308079	0.128935502351777\\
1.526351077559	0.129365561574039\\
1.54203371054641	0.129793642020947\\
1.55764553316405	0.130219723740586\\
1.5731874351088	0.130643788720132\\
1.58866029457569	0.131065820780833\\
1.60406497820329	0.131485805478317\\
1.61940234104027	0.131903730007969\\
1.63467322653111	0.132319583115129\\
1.64987846651948	0.132733355009848\\
1.66501888126745	0.133145037286011\\
1.68009527948923	0.133554622844562\\
1.69510845839801	0.133962105820664\\
1.71005920376461	0.134367481514575\\
1.7249482899869	0.134770746326048\\
1.7397764801687	0.135171897692097\\
1.75454452620747	0.135570934027941\\
1.76925316888955	0.135967854670966\\
1.78390313799236	0.136362659827567\\
1.79849515239264	0.136755350522703\\
1.81302992018011	0.137145928552045\\
1.82750813877579	0.137534396436577\\
1.84193049505446	0.137920757379522\\
1.85629766547064	0.138305015225485\\
1.87061031618768	0.138687174421696\\
1.88486910320927	0.139067239981237\\
1.89907467251331	0.139445217448171\\
1.91322766018727	0.139821112864462\\
1.92732869256515	0.140194932738595\\
1.94137838636534	0.140566684015825\\
1.95537734882933	0.14093637404995\\
1.96932617786077	0.14130401057655\\
1.98322546216494	0.141669601687606\\
1.99707578138798	0.142033155807439\\
2.01087770625611	0.142394681669888\\
2.02463179871433	0.142754188296678\\
2.03833861206457	0.143111684976916\\
2.05199869110311	0.143467181247646\\
2.06561257225717	0.143820686875425\\
2.07918078372047	0.14417221183886\\
2.09270384558764	0.144521766312058\\
2.10618226998757	0.144869360648943\\
2.1196165612153	0.145215005368406\\
2.13300721586267	0.145558711140222\\
2.14635472294744	0.145900488771727\\
2.15965956404093	0.146240349195193\\
2.17292221339419	0.14657830345587\\
2.18614313806241	0.146914362700676\\
2.19932279802784	0.147248538167481\\
2.21246164632094	0.147580841174972\\
2.22556012913992	0.147911283113061\\
2.23861868596843	0.148239875433815\\
2.25163774969165	0.148566629642872\\
2.26461774671055	0.148891557291333\\
2.2775590970544	0.149214669968091\\
2.29046221449152	0.149535979292585\\
2.30332750663826	0.149855496907958\\
2.31615537506624	0.150173234474591\\
2.32894621540776	0.150489203664003\\
2.3417004174595	0.150803416153088\\
2.35441836528449	0.151115883618691\\
2.36710043731227	0.151426617732479\\
2.37974700643739	0.151735630156118\\
2.39235844011617	0.15204293253673\\
2.40493510046171	0.152348536502603\\
2.41747734433732	0.152652453659172\\
2.42998552344819	0.152954695585226\\
2.44245998443148	0.153255273829349\\
2.4549010689447	0.153554199906578\\
2.46730911375258	0.153851485295264\\
2.47968445081227	0.154147141434127\\
2.49202740735701	0.154441179719505\\
2.50433830597827	0.154733611502769\\
2.51661746470632	0.15502444808791\\
2.52886519708934	0.155313700729287\\
2.54108181227105	0.155601380629521\\
2.55326761506685	0.155887498937539\\
2.56542290603856	0.156172066746743\\
2.57754798156776	0.156455095093323\\
2.58964313392771	0.15673659495468\\
2.60170865135395	0.157016577247972\\
2.61374481811353	0.157295052828766\\
2.62575191457291	0.157572032489802\\
2.63773021726463	0.157847526959848\\
2.64967999895267	0.158121546902655\\
2.66160152869655	0.158394102916001\\
2.67349507191423	0.158665205530814\\
2.68536089044381	0.158934865210385\\
2.69719924260405	0.15920309234965\\
2.70901038325372	0.159469897274549\\
2.72079456384983	0.159735290241448\\
2.73255203250474	0.159999281436631\\
2.74428303404218	0.160261880975854\\
2.75598781005221	0.160523098903951\\
2.7676665989451	0.160782945194503\\
2.77931963600421	0.161041429749552\\
2.79094715343785	0.161298562399372\\
2.80254938043016	0.16155435290228\\
2.81412654319101	0.161808810944494\\
2.82567886500493	0.162061946140036\\
2.83720656627916	0.162313768030665\\
2.84870986459072	0.16256428608586\\
2.86018897473266	0.162813509702828\\
2.87164410875936	0.163061448206553\\
2.88307547603103	0.163308110849864\\
2.89448328325733	0.163553506813553\\
2.90586773454017	0.163797645206499\\
2.91722903141574	0.164040535065832\\
2.92856737289566	0.164282185357118\\
2.9398829555075	0.164522604974567\\
2.95117597333435	0.16476180274126\\
2.96244661805382	0.164999787409406\\
2.97369507897623	0.165236567660602\\
2.98492154308206	0.165472152106129\\
2.99612619505875	0.165706549287253\\
3.00730921733681	0.165939767675546\\
3.01847079012525	0.166171815673217\\
3.02961109144631	0.166402701613469\\
3.04073029716964	0.166632433760852\\
3.05182858104573	0.166861020311644\\
3.06290611473885	0.167088469394228\\
3.07396306785928	0.167314789069494\\
3.08499960799498	0.167539987331238\\
3.09601590074272	0.16776407210658\\
3.10701210973856	0.167987051256382\\
3.11798839668787	0.168208932575676\\
3.12894492139471	0.168429723794104\\
3.13988184179073	0.168649432576355\\
3.15079931396352	0.168868066522616\\
3.1616974921845	0.169085633169024\\
3.1725765289362	0.16930213998812\\
3.18343657493917	0.169517594389315\\
3.19427777917834	0.169732003719355\\
3.20510028892886	0.169945375262783\\
3.21590424978163	0.17015771624242\\
3.22668980566817	0.170369033819831\\
3.23745709888526	0.170579335095804\\
3.24820627011895	0.170788627110826\\
3.25893745846827	0.170996916845566\\
3.2696508014685	0.171204211221349\\
3.28034643511396	0.171410517100642\\
3.2910244938805	0.171615841287534\\
3.30168511074751	0.171820190528215\\
3.3123284172196	0.172023571511462\\
3.32295454334787	0.172225990869118\\
3.33356361775083	0.172427455176574\\
3.34415576763493	0.172627970953247\\
3.35473111881477	0.172827544663063\\
3.36528979573293	0.173026182714934\\
3.37583192147947	0.173223891463234\\
3.3863576178111	0.173420677208277\\
3.39686700516999	0.17361654619679\\
3.40736020270234	0.173811504622386\\
3.4178373282765	0.174005558626035\\
3.42829849850092	0.174198714296531\\
3.4387438287417	0.174390977670963\\
3.44917343313988	0.174582354735173\\
3.45958742462844	0.174772851424227\\
3.469985914949	0.174962473622866\\
3.48036901466825	0.175151227165968\\
3.4907368331941	0.175339117839004\\
3.50108947879155	0.175526151378487\\
3.51142705859833	0.17571233347242\\
3.52174967864023	0.175897669760749\\
3.53205744384621	0.176082165835801\\
3.54235045806327	0.176265827242726\\
3.55262882407103	0.176448659479936\\
3.56289264359609	0.176630667999539\\
3.57314201732621	0.176811858207769\\
3.58337704492416	0.176992235465419\\
3.59359782504141	0.177171805088261\\
3.60380445533157	0.177350572347472\\
3.61399703246364	0.17752854247005\\
3.62417565213502	0.177705720639235\\
3.6343404090843	0.177882111994913\\
3.6444913971039	0.178057721634033\\
3.65462870905242	0.178232554611009\\
3.66475243686687	0.178406615938122\\
3.67486267157466	0.178579910585922\\
3.68495950330545	0.178752443483621\\
3.69504302130271	0.178924219519486\\
3.70511331393522	0.179095243541229\\
3.71517046870831	0.179265520356393\\
3.72521457227492	0.179435054732732\\
3.73524571044657	0.179603851398591\\
3.74526396820402	0.179771915043283\\
3.7552694297079	0.179939250317461\\
3.76526217830909	0.180105861833483\\
3.77524229655897	0.180271754165781\\
3.7852098662195	0.180436931851225\\
3.79516496827314	0.180601399389474\\
3.80510768293265	0.180765161243341\\
3.8150380896507	0.180928221839137\\
3.82495626712933	0.181090585567026\\
3.8348622933293	0.181252256781365\\
3.84475624547931	0.181413239801051\\
3.85463820008498	0.181573538909855\\
3.86450823293781	0.181733158356762\\
3.87436641912396	0.181892102356301\\
3.88421283303287	0.182050375088876\\
3.89404754836579	0.182207980701088\\
3.90387063814416	0.182364923306062\\
3.91368217471786	0.182521206983766\\
3.92348222977339	0.182676835781325\\
3.93327087434183	0.182831813713335\\
3.94304817880677	0.182986144762175\\
3.95281421291209	0.183139832878315\\
3.96256904576962	0.183292881980615\\
3.97231274586666	0.18344529595663\\
3.98204538107348	0.183597078662908\\
3.99176701865062	0.183748233925283\\
4.00147772525609	0.18389876553917\\
4.01117756695254	0.184048677269851\\
4.02086660921424	0.184197972852761\\
4.03054491693399	0.184346655993776\\
4.04021255442997	0.184494730369487\\
4.0498695854524	0.184642199627481\\
4.05951607319022	0.184789067386617\\
4.06915208027754	0.184935337237294\\
4.07877766880015	0.185081012741723\\
4.08839290030179	0.185226097434192\\
4.09799783579044	0.185370594821331\\
4.10759253574446	0.18551450838237\\
4.11717706011866	0.185657841569401\\
4.12675146835028	0.18580059780763\\
4.13631581936492	0.185942780495633\\
4.14587017158233	0.186084393005603\\
4.15541458292214	0.186225438683602\\
4.16494911080954	0.186365920849802\\
4.17447381218085	0.186505842798727\\
4.183988743489	0.1866452077995\\
4.19349396070898	0.186784019096073\\
4.20298951934317	0.186922279907463\\
4.21247547442661	0.187059993427992\\
4.22195188053222	0.187197162827506\\
4.23141879177592	0.187333791251614\\
4.24087626182167	0.187469881821905\\
4.2503243438865	0.187605437636174\\
4.25976309074538	0.187740461768646\\
4.26919255473612	0.18787495727019\\
4.27861278776415	0.188008927168537\\
4.28802384130721	0.188142374468494\\
4.29742576642006	0.188275302152158\\
4.30681861373903	0.18840771317912\\
4.3162024334866	0.188539610486677\\
4.32557727547583	0.188670996990039\\
4.3349431891148	0.188801875582524\\
4.34430022341099	0.188932249135768\\
4.35364842697552	0.189062120499919\\
4.36298784802742	0.189191492503834\\
4.37231853439784	0.189320367955275\\
4.38164053353413	0.189448749641102\\
4.39095389250395	0.189576640327462\\
4.40025865799928	0.189704042759981\\
4.4095548763404	0.189830959663945\\
4.41884259347978	0.189957393744492\\
4.42812185500595	0.19008334768679\\
4.43739270614737	0.190208824156218\\
4.44665519177611	0.19033382579855\\
4.45590935641162	0.190458355240126\\
4.4651552442244	0.190582415088031\\
4.4743928990396	0.190706007930269\\
4.48362236434061	0.190829136335933\\
4.49284368327257	0.190951802855376\\
4.50205689864587	0.191074010020379\\
4.51126205293961	0.191195760344319\\
4.52045918830495	0.19131705632233\\
4.52964834656848	0.191437900431472\\
4.53882956923555	0.191558295130887\\
4.54800289749352	0.191678242861961\\
4.55716837221498	0.191797746048484\\
4.56632603396096	0.191916807096802\\
4.57547592298406	0.192035428395976\\
4.58461807923154	0.192153612317934\\
4.59375254234842	0.192271361217623\\
4.60287935168052	0.192388677433157\\
4.61199854627739	0.192505563285969\\
4.62111016489532	0.192622021080956\\
4.63021424600026	0.192738053106625\\
4.63931082777066	0.192853661635236\\
4.64839994810036	0.192968848922948\\
4.65748164460136	0.193083617209958\\
4.66655595460665	0.193197968720638\\
4.67562291517291	0.19331190566368\\
4.68468256308324	0.193425430232226\\
4.69373493484982	0.193538544604007\\
4.70278006671658	0.193651250941477\\
4.71181799466179	0.193763551391947\\
4.72084875440065	0.193875448087712\\
4.72987238138785	0.193986943146186\\
4.73888891082007	0.194098038670029\\
4.74789837763848	0.194208736747271\\
4.75690081653118	0.194319039451445\\
4.76589626193567	0.194428948841707\\
4.7748847480412	0.194538466962961\\
4.7838663087912	0.19464759584598\\
4.79284097788557	0.194756337507531\\
4.80180878878302	0.19486469395049\\
4.81076977470335	0.194972667163966\\
4.81972396862974	0.195080259123412\\
4.82867140331095	0.195187471790748\\
4.83761211126354	0.195294307114473\\
4.84654612477406	0.195400767029779\\
4.85547347590121	0.195506853458664\\
4.86439419647795	0.195612568310044\\
4.87330831811363	0.195717913479866\\
4.88221587219608	0.195822890851211\\
4.89111688989363	0.195927502294412\\
4.90001140215719	0.196031749667153\\
4.90889943972223	0.196135634814579\\
4.9177810331108	0.196239159569401\\
4.92665621263345	0.196342325752002\\
4.93552500839123	0.196445135170536\\
4.94438745027755	0.196547589621033\\
4.95324356798013	0.196649690887502\\
4.96209339098285	0.196751440742026\\
4.97093694856763	0.196852840944866\\
4.97977426981623	0.196953893244557\\
4.98860538361209	0.197054599378005\\
4.99743031864213	0.197154961070586\\
5.00624910339848	0.197254980036237\\
5.0150617661803	0.197354657977554\\
5.02386833509547	0.197453996585885\\
5.03266883806231	0.19755299754142\\
5.04146330281128	0.197651662513288\\
5.05025175688667	0.19774999315964\\
5.05903422764823	0.197847991127747\\
5.06781074227284	0.197945658054085\\
5.0765813277561	0.198042995564422\\
5.08534601091397	0.198140005273908\\
5.09410481838434	0.198236688787161\\
5.10285777662858	0.198333047698351\\
5.11160491193312	0.198429083591288\\
5.12034625041096	0.198524798039502\\
5.12908181800323	0.198620192606328\\
5.13781164048063	0.198715268844991\\
5.14653574344497	0.198810028298683\\
5.15525415233059	0.198904472500648\\
5.16396689240588	0.198998602974259\\
5.17267398877464	0.199092421233098\\
5.18137546637755	0.199185928781036\\
5.19007134999358	0.19927912711231\\
5.19876166424135	0.199372017711599\\
5.20744643358051	0.199464602054103\\
5.21612568231316	0.199556881605613\\
5.2247994345851	0.199648857822594\\
5.23346771438726	0.199740532152253\\
5.24213054555693	0.199831906032613\\
5.25078795177916	0.19992298089259\\
5.25943995658795	0.20001375815206\\
5.26808658336761	0.200104239221933\\
5.27672785535399	0.200194425504224\\
5.2853637956357	0.200284318392123\\
5.29399442715543	0.200373919270061\\
5.30261977271107	0.200463229513785\\
5.31123985495703	0.200552250490421\\
5.31985469640534	0.200640983558544\\
5.3284643194269	0.200729430068243\\
5.33706874625264	0.200817591361189\\
5.34566799897468	0.2009054687707\\
5.35426209954746	0.200993063621807\\
5.36285106978892	0.201080377231316\\
5.37143493138159	0.201167410907873\\
5.38001370587371	0.201254165952028\\
5.38858741468037	0.201340643656297\\
5.39715607908455	0.201426845305225\\
5.40571972023823	0.201512772175446\\
5.41427835916346	0.201598425535743\\
5.42283201675341	0.201683806647111\\
5.43138071377342	0.201768916762818\\
5.43992447086206	0.201853757128458\\
5.44846330853211	0.201938328982017\\
5.45699724717163	0.202022633553925\\
5.46552630704492	0.202106672067118\\
5.47405050829356	0.202190445737095\\
5.48256987093735	0.20227395577197\\
5.49108441487533	0.202357203372535\\
5.49959415988673	0.202440189732308\\
5.50809912563192	0.202522916037596\\
5.51659933165335	0.202605383467542\\
5.52509479737652	0.202687593194185\\
5.53358554211089	0.20276954638251\\
5.54207158505079	0.202851244190503\\
5.55055294527634	0.202932687769203\\
5.55902964175438	0.203013878262754\\
5.5675016933393	0.203094816808457\\
5.57596911877398	0.20317550453682\\
5.58443193669065	0.203255942571611\\
5.59289016561173	0.203336132029908\\
5.60134382395073	0.203416074022146\\
5.60979293001308	0.20349576965217\\
5.61823750199696	0.203575220017281\\
5.62667755799416	0.203654426208287\\
5.63511311599089	0.203733389309551\\
5.64354419386859	0.203812110399035\\
5.65197080940476	0.203890590548354\\
5.66039298027375	0.203968830822815\\
5.66881072404753	0.20404683228147\\
5.67722405819654	0.204124595977158\\
5.68563300009041	0.204202122956553\\
5.69403756699876	0.204279414260207\\
5.70243777609194	0.204356470922597\\
5.71083364444181	0.204433293972168\\
5.71922518902251	0.204509884431379\\
5.72761242671113	0.204586243316742\\
5.73599537428852	0.204662371638872\\
5.74437404843998	0.204738270402523\\
5.752748465756	0.204813940606637\\
5.76111864273297	0.204889383244381\\
5.76948459577385	0.204964599303191\\
5.77784634118895	0.205039589764814\\
5.78620389519655	0.205114355605348\\
5.79455727392364	0.205188897795282\\
5.80290649340656	0.20526321729954\\
5.81125156959173	0.205337315077514\\
5.81959251833625	0.205411192083113\\
5.82792935540862	0.205484849264794\\
5.83626209648937	0.205558287565606\\
5.84459075717173	0.205631507923225\\
5.85291535296225	0.205704511269996\\
5.86123589928145	0.20577729853297\\
5.86955241146446	0.20584987063394\\
5.87786490476165	0.205922228489478\\
5.88617339433923	0.205994373010975\\
5.8944778952799	0.206066305104676\\
5.90277842258341	0.206138025671716\\
5.91107499116723	0.206209535608157\\
5.91936761586708	0.206280835805023\\
5.92765631143759	0.206351927148336\\
5.93594109255282	0.20642281051915\\
5.94422197380693	0.20649348679359\\
5.95249896971465	0.206563956842878\\
5.96077209471195	0.206634221533379\\
5.96904136315656	0.206704281726625\\
5.97730678932852	0.206774138279354\\
5.98556838743079	0.20684379204354\\
5.99382617158974	0.20691324386643\\
6.00208015585574	0.206982494590576\\
6.01033035420369	0.207051545053864\\
6.01857678053355	0.20712039608955\\
6.02681944867089	0.207189048526292\\
6.03505837236741	0.207257503188181\\
6.04329356530143	0.20732576089477\\
6.05152504107849	0.20739382246111\\
6.05975281323178	0.207461688697778\\
6.06797689522269	0.207529360410908\\
6.07619730044132	0.207596838402223\\
6.08441404220697	0.207664123469063\\
6.09262713376862	0.207731216404418\\
6.10083658830545	0.207798117996954\\
6.10904241892733	0.207864829031045\\
6.11724463867528	0.207931350286804\\
6.12544326052194	0.207997682540106\\
6.1336382973721	0.208063826562623\\
6.14182976206312	0.208129783121852\\
6.1500176673654	0.208195552981136\\
6.15820202598289	0.208261136899704\\
6.16638285055348	0.208326535632688\\
6.17456015364951	0.208391749931157\\
6.18273394777819	0.208456780542142\\
6.19090424538207	0.208521628208664\\
6.19907105883947	0.208586293669761\\
6.20723440046491	0.208650777660513\\
6.21539428250957	0.208715080912073\\
6.22355071716171	0.208779204151687\\
6.23170371654708	0.208843148102725\\
6.23985329272938	0.208906913484707\\
6.24799945771066	0.208970501013323\\
6.25614222343175	0.209033911400466\\
6.26428160177264	0.209097145354254\\
6.27241760455294	0.209160203579052\\
6.28055024353225	0.209223086775503\\
6.2886795304106	0.209285795640547\\
6.29680547682879	0.209348330867449\\
6.30492809436885	0.209410693145824\\
6.31304739455442	0.209472883161656\\
6.32116338885109	0.209534901597328\\
6.32927608866686	0.209596749131642\\
6.33738550535247	0.209658426439843\\
6.34549165020181	0.209719934193645\\
6.35359453445228	0.209781273061249\\
6.36169416928518	0.209842443707372\\
6.36979056582607	0.209903446793264\\
6.37788373514515	0.209964282976737\\
6.38597368825759	0.21002495291218\\
6.39406043612396	0.210085457250589\\
6.40214398965051	0.210145796639582\\
6.41022435968959	0.210205971723426\\
6.41830155703997	0.210265983143056\\
6.42637559244719	0.210325831536099\\
6.43444647660392	0.210385517536893\\
6.44251422015032	0.210445041776507\\
6.45057883367434	0.210504404882769\\
6.45864032771209	0.210563607480278\\
6.46669871274816	0.210622650190431\\
6.47475399921598	0.210681533631441\\
6.48280619749813	0.210740258418357\\
6.49085531792664	0.210798825163087\\
6.4989013707834	0.210857234474416\\
6.50694436630039	0.210915486958024\\
6.51498431466006	0.210973583216511\\
6.52302122599561	0.211031523849413\\
6.53105511039135	0.21108930945322\\
6.53908597788297	0.211146940621402\\
6.54711383845787	0.21120441794442\\
6.55513870205547	0.21126174200975\\
6.5631605785675	0.211318913401902\\
6.57117947783833	0.211375932702436\\
6.57919540966523	0.211432800489983\\
6.58720838379873	0.211489517340262\\
6.59521840994284	0.211546083826101\\
6.6032254977554	0.211602500517452\\
6.61122965684836	0.211658767981408\\
6.61923089678805	0.211714886782229\\
6.62722922709547	0.211770857481348\\
6.63522465724661	0.211826680637398\\
6.64321719667269	0.211882356806226\\
6.65120685476045	0.21193788654091\\
6.65919364085244	0.211993270391777\\
6.66717756424728	0.212048508906421\\
6.67515863419997	0.212103602629719\\
6.68313685992209	0.212158552103846\\
6.69111225058215	0.212213357868296\\
6.69908481530581	0.212268020459894\\
6.70705456317613	0.212322540412819\\
6.71502150323391	0.212376918258611\\
6.72298564447784	0.212431154526197\\
6.73094699586486	0.2124852497419\\
6.73890556631036	0.212539204429458\\
6.74686136468844	0.212593019110041\\
6.7548143998322	0.212646694302263\\
6.76276468053395	0.212700230522203\\
6.77071221554546	0.212753628283416\\
6.77865701357824	0.212806888096949\\
6.78659908330376	0.212860010471358\\
6.7945384333537	0.212912995912725\\
6.80247507232019	0.212965844924667\\
6.81040900875605	0.213018558008357\\
6.81834025117505	0.213071135662535\\
6.82626880805208	0.213123578383526\\
6.83419468782349	0.213175886665252\\
6.84211789888721	0.213228060999247\\
6.85003844960306	0.213280101874673\\
6.85795634829296	0.213332009778333\\
6.86587160324113	0.213383785194686\\
6.87378422269436	0.213435428605859\\
6.8816942148622	0.213486940491664\\
6.88960158791717	0.213538321329611\\
6.89750634999505	0.213589571594921\\
6.90540850919502	0.21364069176054\\
6.91330807357992	0.213691682297153\\
6.92120505117646	0.213742543673198\\
6.92909944997543	0.213793276354878\\
6.9369912779319	0.213843880806175\\
6.94488054296548	0.213894357488864\\
6.95276725296045	0.213944706862525\\
6.96065141576604	0.213994929384558\\
6.96853303919661	0.214045025510192\\
6.97641213103184	0.214094995692504\\
6.98428869901697	0.214144840382424\\
6.99216275086294	0.214194560028755\\
7.00003429424668	0.214244155078182\\
7.00790333681121	0.214293625975284\\
7.01576988616593	0.214342973162548\\
7.02363394988675	0.21439219708038\\
7.0314955355163	0.21444129816712\\
7.03935465056416	0.214490276859049\\
7.04721130250699	0.214539133590407\\
7.05506549878878	0.214587868793401\\
7.062917246821	0.214636482898218\\
7.07076655398279	0.214684976333037\\
7.07861342762119	0.214733349524041\\
7.08645787505126	0.214781602895427\\
7.0942999035563	0.21482973686942\\
7.10213952038805	0.214877751866284\\
7.10997673276683	0.214925648304331\\
7.11781154788174	0.214973426599934\\
7.12564397289084	0.215021087167541\\
7.13347401492134	0.21506863041968\\
7.14130168106973	0.215116056766976\\
7.14912697840201	0.215163366618158\\
7.15694991395384	0.215210560380072\\
7.1647704947307	0.215257638457692\\
7.17258872770808	0.215304601254129\\
7.18040461983164	0.215351449170644\\
7.18821817801737	0.215398182606656\\
7.1960294091518	0.215444801959757\\
7.2038383200921	0.215491307625715\\
7.21164491766631	0.215537699998492\\
7.21944920867344	0.215583979470252\\
7.22725119988371	0.215630146431368\\
7.23505089803862	0.215676201270436\\
7.2428483098512	0.215722144374285\\
7.25064344200612	0.215767976127984\\
7.25843630115983	0.215813696914854\\
7.26622689394078	0.21585930711648\\
7.27401522694952	0.215904807112717\\
7.28180130675887	0.215950197281701\\
7.28958513991411	0.215995477999862\\
7.29736673293306	0.216040649641927\\
7.3051460923063	0.216085712580937\\
7.31292322449729	0.216130667188251\\
7.32069813594253	0.216175513833558\\
7.32847083305169	0.216220252884886\\
7.33624132220779	0.216264884708611\\
7.34400960976729	0.216309409669467\\
7.35177570206032	0.216353828130554\\
7.35953960539075	0.216398140453348\\
7.36730132603636	0.21644234699771\\
7.375060870249	0.216486448121896\\
7.3828182442547	0.216530444182562\\
7.39057345425385	0.216574335534779\\
7.39832650642128	0.216618122532037\\
7.40607740690646	0.216661805526256\\
7.41382616183362	0.216705384867794\\
7.42157277730186	0.216748860905455\\
7.42931725938531	0.2167922339865\\
7.43705961413328	0.216835504456652\\
7.44479984757035	0.216878672660109\\
7.45253796569654	0.216921738939549\\
7.46027397448744	0.216964703636136\\
7.46800787989432	0.217007567089538\\
7.47573968784427	0.217050329637922\\
7.48346940424035	0.217092991617975\\
7.49119703496169	0.217135553364903\\
7.49892258586363	0.217178015212442\\
7.50664606277785	0.217220377492868\\
7.5143674715125	0.217262640537004\\
7.52208681785229	0.217304804674225\\
7.52980410755869	0.217346870232471\\
7.53751934636997	0.217388837538248\\
7.54523254000137	0.217430706916645\\
7.55294369414523	0.217472478691334\\
7.56065281447106	0.217514153184579\\
7.56835990662573	0.217555730717248\\
7.57606497623353	0.217597211608815\\
7.58376802889633	0.217638596177372\\
7.59146907019366	0.217679884739633\\
7.59916810568287	0.217721077610945\\
7.60686514089921	0.217762175105291\\
7.61456018135598	0.217803177535301\\
7.62225323254461	0.217844085212258\\
7.62994429993479	0.217884898446106\\
7.6376333889746	0.217925617545456\\
7.6453205050906	0.217966242817593\\
7.65300565368793	0.218006774568485\\
7.66068884015048	0.218047213102787\\
7.66837006984093	0.218087558723853\\
7.67604934810091	0.218127811733736\\
7.68372668025107	0.218167972433202\\
7.69140207159123	0.218208041121731\\
7.69907552740046	0.21824801809753\\
7.7067470529372	0.218287903657532\\
7.71441665343934	0.218327698097411\\
7.72208433412438	0.218367401711582\\
7.72975010018948	0.218407014793212\\
7.73741395681159	0.218446537634224\\
7.74507590914755	0.218485970525305\\
7.7527359623342	0.218525313755912\\
7.76039412148845	0.21856456761428\\
7.76805039170746	0.218603732387426\\
7.77570477806863	0.218642808361156\\
7.78335728562979	0.218681795820073\\
7.79100791942928	0.218720695047581\\
7.798656684486	0.218759506325894\\
7.80630358579959	0.21879822993604\\
7.81394862835045	0.218836866157869\\
7.8215918170999	0.218875415270056\\
7.82923315699022	0.218913877550111\\
7.83687265294479	0.218952253274384\\
7.84451030986819	0.218990542718069\\
7.85214613264623	0.219028746155213\\
7.85978012614614	0.219066863858718\\
7.86741229521657	0.219104896100352\\
7.87504264468775	0.219142843150751\\
7.88267117937156	0.219180705279427\\
7.89029790406162	0.219218482754773\\
7.89792282353337	0.219256175844068\\
7.90554594254418	0.219293784813485\\
7.91316726583343	0.219331309928094\\
7.92078679812263	0.219368751451869\\
7.92840454411544	0.219406109647696\\
7.93602050849783	0.219443384777372\\
7.94363469593812	0.21948057710162\\
7.95124711108712	0.219517686880086\\
7.95885775857814	0.219554714371349\\
7.96646664302714	0.219591659832925\\
7.97407376903281	0.219628523521273\\
7.98167914117662	0.219665305691802\\
7.98928276402294	0.219702006598873\\
7.99688464211909	0.219738626495805\\
8.00448477999547	0.219775165634885\\
8.0120831821656	0.219811624267365\\
8.01967985312622	0.219848002643476\\
8.02727479735738	0.219884301012426\\
8.03486801932251	0.219920519622412\\
8.04245952346851	0.219956658720617\\
8.05004931422582	0.219992718553223\\
8.05763739600851	0.22002869936541\\
8.06522377321436	0.220064601401368\\
8.07280845022492	0.220100424904292\\
8.08039143140562	0.220136170116397\\
8.08797272110582	0.220171837278918\\
8.09555232365892	0.220207426632115\\
8.1031302433824	0.220242938415278\\
8.11070648457791	0.220278372866734\\
8.11828105153137	0.22031373022385\\
8.12585394851301	0.220349010723036\\
8.13342517977748	0.220384214599754\\
8.1409947495639	0.220419342088522\\
8.14856266209593	0.220454393422915\\
8.15612892158187	0.220489368835571\\
8.16369353221472	0.220524268558202\\
8.17125649817226	0.220559092821587\\
8.17881782361709	0.220593841855588\\
8.18637751269676	0.220628515889147\\
8.19393556954379	0.220663115150296\\
8.20149199827577	0.220697639866154\\
8.20904680299544	0.220732090262941\\
8.2165999877907	0.220766466565976\\
8.22415155673478	0.220800768999683\\
8.2317015138862	0.220834997787596\\
8.23924986328895	0.220869153152362\\
8.24679660897245	0.220903235315748\\
8.2543417549517	0.220937244498643\\
8.26188530522733	0.220971180921064\\
8.26942726378563	0.221005044802157\\
8.27696763459867	0.221038836360207\\
8.28450642162432	0.221072555812637\\
8.29204362880635	0.221106203376015\\
8.2995792600745	0.221139779266055\\
8.3071133193445	0.221173283697627\\
8.31464581051819	0.221206716884756\\
8.32217673748355	0.221240079040627\\
8.32970610411479	0.221273370377592\\
8.33723391427238	0.22130659110717\\
8.34476017180316	0.221339741440055\\
8.35228488054036	0.221372821586116\\
8.35980804430368	0.221405831754404\\
8.36732966689937	0.221438772153156\\
8.37484975212027	0.221471642989797\\
8.38236830374586	0.221504444470945\\
8.38988532554238	0.221537176802414\\
8.39740082126282	0.221569840189222\\
8.40491479464703	0.221602434835588\\
8.41242724942174	0.221634960944941\\
8.41993818930069	0.221667418719923\\
8.4274476179846	0.22169980836239\\
8.43495553916129	0.221732130073421\\
8.44246195650574	0.221764384053316\\
8.4499668736801	0.221796570501603\\
8.45747029433381	0.221828689617042\\
8.46497222210362	0.221860741597627\\
8.47247266061365	0.221892726640591\\
8.47997161347546	0.221924644942407\\
8.48746908428811	0.221956496698798\\
8.49496507663821	0.221988282104732\\
8.50245959409995	0.222020001354433\\
8.50995264023522	0.222051654641379\\
8.5174442185936	0.22208324215831\\
8.52493433271247	0.222114764097229\\
8.53242298611701	0.222146220649405\\
8.5399101823203	0.22217761200538\\
8.54739592482338	0.222208938354967\\
8.55488021711524	0.222240199887259\\
8.56236306267296	0.222271396790628\\
8.56984446496169	0.222302529252732\\
8.57732442743477	0.222333597460515\\
8.58480295353372	0.222364601600213\\
8.59228004668832	0.222395541857358\\
8.5997557103167	0.222426418416775\\
8.60722994782532	0.222457231462596\\
8.61470276260907	0.222487981178253\\
8.62217415805132	0.222518667746487\\
8.62964413752395	0.22254929134935\\
8.63711270438743	0.222579852168208\\
8.64457986199084	0.222610350383745\\
8.65204561367194	0.222640786175963\\
8.65950996275722	0.222671159724192\\
8.66697291256193	0.222701471207084\\
8.67443446639016	0.222731720802625\\
8.68189462753488	0.222761908688132\\
8.68935339927798	0.222792035040259\\
8.6968107848903	0.222822100034998\\
8.70426678763174	0.222852103847686\\
8.71172141075124	0.222882046653003\\
8.71917465748688	0.222911928624979\\
8.72662653106589	0.222941749936995\\
8.73407703470474	0.222971510761787\\
8.74152617160912	0.223001211271447\\
8.74897394497406	0.22303085163743\\
8.75642035798395	0.223060432030552\\
8.76386541381256	0.223089952620997\\
8.77130911562312	0.223119413578318\\
8.77875146656835	0.22314881507144\\
8.78619246979053	0.223178157268664\\
8.79363212842149	0.223207440337667\\
8.80107044558273	0.22323666444551\\
8.80850742438541	0.223265829758634\\
8.8159430679304	0.223294936442869\\
8.82337737930835	0.223323984663434\\
8.83081036159973	0.223352974584941\\
8.83824201787484	0.223381906371394\\
8.8456723511939	0.223410780186198\\
8.85310136460706	0.223439596192157\\
8.86052906115447	0.223468354551478\\
8.86795544386628	0.223497055425775\\
8.87538051576275	0.223525698976069\\
8.88280427985423	0.223554285362793\\
8.89022673914124	0.223582814745795\\
8.89764789661449	0.223611287284339\\
8.90506775525492	0.223639703137107\\
8.9124863180338	0.223668062462206\\
8.91990358791267	0.223696365417164\\
8.92731956784346	0.223724612158939\\
8.93473426076853	0.223752802843917\\
8.94214766962065	0.223780937627918\\
8.9495597973231	0.223809016666194\\
8.95697064678969	0.223837040113439\\
8.96438022092479	0.223865008123782\\
8.97178852262339	0.223892920850799\\
8.97919555477114	0.223920778447507\\
8.98660132024435	0.223948581066373\\
8.99400582191008	0.223976328859314\\
9.00140906262617	0.224004021977698\\
9.00881104524124	0.22403166057235\\
9.01621177259477	0.22405924479355\\
9.02361124751712	0.22408677479104\\
9.0310094728296	0.224114250714022\\
9.03840645134444	0.224141672711164\\
9.04580218586489	0.224169040930602\\
9.05319667918525	0.224196355519938\\
9.06058993409087	0.22422361662625\\
9.06798195335824	0.224250824396087\\
9.07537273975498	0.224277978975475\\
9.08276229603989	0.224305080509919\\
9.09015062496303	0.224332129144406\\
9.09753772926568	0.224359125023405\\
9.10492361168043	0.224386068290872\\
9.11230827493123	0.224412959090249\\
9.11969172173335	0.224439797564469\\
9.12707395479351	0.22446658385596\\
9.13445497680985	0.22449331810664\\
9.14183479047198	0.224520000457927\\
9.14921339846105	0.224546631050738\\
9.15659080344972	0.224573210025491\\
9.16396700810225	0.224599737522107\\
9.17134201507453	0.224626213680013\\
9.17871582701409	0.224652638638143\\
9.18608844656013	0.224679012534944\\
9.19345987634359	0.224705335508372\\
9.20083011898716	0.224731607695899\\
9.20819917710533	0.224757829234512\\
9.21556705330438	0.224784000260718\\
9.22293375018247	0.224810120910545\\
9.23029927032965	0.224836191319542\\
9.23766361632787	0.224862211622784\\
9.24502679075107	0.224888181954874\\
9.25238879616516	0.22491410244994\\
9.25974963512805	0.224939973241646\\
9.26710931018974	0.224965794463185\\
9.27446782389229	0.224991566247289\\
9.28182517876989	0.225017288726223\\
9.28918137734887	0.225042962031794\\
9.29653642214777	0.225068586295348\\
9.3038903156773	0.225094161647777\\
9.31124306044045	0.225119688219515\\
9.31859465893248	0.225145166140545\\
9.32594511364094	0.225170595540397\\
9.33329442704573	0.225195976548155\\
9.34064260161914	0.225221309292453\\
9.34798963982583	0.225246593901481\\
9.3553355441229	0.225271830502986\\
9.36268031695993	0.225297019224272\\
9.37002396077897	0.225322160192205\\
9.3773664780146	0.225347253533214\\
9.38470787109397	0.225372299373291\\
9.39204814243677	0.225397297837993\\
9.39938729445534	0.225422249052448\\
9.40672532955466	0.225447153141352\\
9.41406225013236	0.225472010228972\\
9.42139805857879	0.225496820439149\\
9.42873275727701	0.225521583895301\\
9.43606634860286	0.225546300720421\\
9.44339883492495	0.225570971037082\\
9.45073021860472	0.225595594967436\\
9.45806050199644	0.225620172633221\\
9.46538968744727	0.225644704155756\\
9.47271777729727	0.225669189655947\\
9.4800447738794	0.225693629254288\\
9.48737067951962	0.225718023070863\\
9.49469549653685	0.225742371225346\\
9.50201922724302	0.225766673837004\\
9.50934187394312	0.225790931024701\\
9.5166634389352	0.225815142906894\\
9.52398392451039	0.225839309601642\\
9.53130333295298	0.2258634312266\\
9.53862166654036	0.225887507899026\\
9.54593892754314	0.225911539735783\\
9.55325511822511	0.225935526853335\\
9.5605702408433	0.225959469367757\\
9.567884297648	0.225983367394728\\
9.57519729088276	0.226007221049539\\
9.58250922278447	0.226031030447092\\
9.58982009558334	0.226054795701903\\
9.59712991150294	0.2260785169281\\
9.60443867276023	0.226102194239431\\
9.61174638156559	0.226125827749258\\
9.61905304012283	0.226149417570565\\
9.62635865062924	0.226172963815957\\
9.63366321527557	0.22619646659766\\
9.64096673624611	0.226219926027526\\
9.64826921571868	0.226243342217031\\
9.65557065586468	0.22626671527728\\
9.66287105884908	0.226290045319007\\
9.67017042683048	0.226313332452573\\
9.67746876196111	0.226336576787976\\
9.68476606638686	0.226359778434843\\
9.69206234224733	0.226382937502439\\
9.69935759167581	0.226406054099664\\
9.70665181679934	0.226429128335057\\
9.71394501973872	0.226452160316794\\
9.72123720260853	0.226475150152696\\
9.72852836751717	0.226498097950222\\
9.73581851656686	0.226521003816479\\
9.74310765185369	0.226543867858216\\
9.7503957754676	0.226566690181832\\
9.75768288949248	0.226589470893371\\
9.76496899600611	0.226612210098528\\
9.77225409708023	0.226634907902651\\
9.77953819478056	0.226657564410737\\
9.78682129116681	0.226680179727442\\
9.79410338829272	0.226702753957071\\
9.80138448820604	0.226725287203592\\
9.80866459294862	0.226747779570627\\
9.81594370455638	0.226770231161459\\
9.82322182505934	0.226792642079033\\
9.83049895648168	0.226815012425955\\
9.83777510084171	0.226837342304496\\
9.84505026015192	0.22685963181659\\
9.852324436419	0.226881881063839\\
9.85959763164385	0.226904090147514\\
9.86686984782164	0.226926259168553\\
9.87414108694178	0.226948388227566\\
9.88141135098796	0.226970477424833\\
9.88868064193818	0.22699252686031\\
9.89594896176479	0.227014536633625\\
9.90321631243446	0.227036506844083\\
9.91048269590825	0.227058437590666\\
9.91774811414159	0.227080328972035\\
9.92501256908435	0.227102181086528\\
9.93227606268081	0.227123994032169\\
9.93953859686971	0.227145767906659\\
9.94680017358428	0.227167502807385\\
9.95406079475222	0.227189198831421\\
9.96132046229576	0.227210856075523\\
9.96857917813167	0.227232474636137\\
9.97583694417126	0.227254054609396\\
9.98309376232044	0.227275596091125\\
9.99034963447969	0.227297099176839\\
9.99760456254414	0.227318563961744\\
10.0048585484035	0.227339990540741\\
10.0121115939422	0.227361379008427\\
10.0193637010394	0.227382729459091\\
10.0266148715687	0.227404041986725\\
10.0338651073988	0.227425316685014\\
10.0411144103928	0.227446553647345\\
10.0483627824086	0.227467752966807\\
10.0556102252992	0.227488914736188\\
10.0628567409119	0.227510039047982\\
10.0701023310892	0.227531125994386\\
10.0773469976683	0.227552175667303\\
10.0845907424813	0.227573188158342\\
10.0918335673549	0.227594163558821\\
10.0990754741111	0.227615101959765\\
10.1063164645665	0.227636003451912\\
10.1135565405328	0.227656868125709\\
10.1207957038165	0.227677696071317\\
10.1280339562193	0.227698487378609\\
10.1352712995375	0.227719242137174\\
10.1425077355628	0.227739960436316\\
10.1497432660816	0.227760642365057\\
10.1569778928756	0.227781288012136\\
10.1642116177214	0.227801897466012\\
10.1714444423906	0.227822470814863\\
10.1786763686501	0.22784300814659\\
10.1859073982618	0.227863509548815\\
10.1931375329826	0.227883975108884\\
10.2003667745647	0.227904404913869\\
10.2075951247555	0.227924799050565\\
10.2148225852974	0.227945157605497\\
10.2220491579282	0.227965480664914\\
10.2292748443807	0.227985768314797\\
10.2364996463831	0.228006020640857\\
10.2437235656588	0.228026237728533\\
10.2509466039263	0.228046419662999\\
10.2581687628997	0.22806656652916\\
10.2653900442882	0.228086678411658\\
10.2726104497963	0.228106755394867\\
10.2798299811238	0.228126797562899\\
10.287048639966	0.228146804999602\\
10.2942664280135	0.228166777788563\\
10.3014833469522	0.228186716013109\\
10.3086993984636	0.228206619756306\\
10.3159145842242	0.228226489100961\\
10.3231289059065	0.228246324129625\\
10.330342365178	0.228266124924589\\
10.3375549637018	0.228285891567893\\
10.3447667031365	0.228305624141317\\
10.3519775851363	0.228325322726391\\
10.3591876113507	0.228344987404391\\
10.3663967834249	0.22836461825634\\
10.3736051029995	0.228384215363012\\
10.3808125717107	0.228403778804929\\
10.3880191911905	0.228423308662366\\
10.395224963066	0.22844280501535\\
10.4024298889604	0.228462267943657\\
10.4096339704923	0.228481697526823\\
10.416837209276	0.228501093844134\\
10.4240396069212	0.228520456974634\\
10.4312411650337	0.228539786997123\\
10.4384418852147	0.228559083990159\\
10.4456417690611	0.228578348032057\\
10.4528408181656	0.228597579200894\\
10.4600390341166	0.228616777574505\\
10.4672364184982	0.228635943230487\\
10.4744329728904	0.228655076246201\\
10.4816286988688	0.228674176698768\\
10.4888235980048	0.228693244665075\\
10.4960176718658	0.228712280221774\\
10.5032109220148	0.228731283445281\\
10.5104033500107	0.22875025441178\\
10.5175949574082	0.228769193197223\\
10.524785745758	0.228788099877328\\
10.5319757166066	0.228806974527586\\
10.5391648714963	0.228825817223254\\
10.5463532119655	0.228844628039363\\
10.5535407395482	0.228863407050716\\
10.5607274557747	0.228882154331886\\
10.567913362171	0.228900869957222\\
10.5750984602592	0.228919554000847\\
10.5822827515572	0.228938206536658\\
10.5894662375791	0.22895682763833\\
10.5966489198348	0.228975417379313\\
10.6038307998303	0.228993975832837\\
10.6110118790678	0.22901250307191\\
10.6181921590453	0.229030999169317\\
10.6253716412569	0.229049464197627\\
10.6325503271928	0.229067898229187\\
10.6397282183393	0.229086301336129\\
10.6469053161789	0.229104673590364\\
10.6540816221899	0.229123015063589\\
10.6612571378471	0.229141325827287\\
10.6684318646212	0.229159605952721\\
10.6756058039791	0.229177855510945\\
10.6827789573838	0.229196074572797\\
10.6899513262947	0.229214263208903\\
10.6971229121671	0.229232421489679\\
10.7042937164527	0.229250549485327\\
10.7114637405994	0.229268647265842\\
10.7186329860512	0.229286714901007\\
10.7258014542484	0.229304752460399\\
10.7329691466277	0.229322760013385\\
10.7401360646219	0.229340737629126\\
10.74730220966	0.229358685376577\\
10.7544675831676	0.229376603324487\\
10.7616321865662	0.229394491541401\\
10.768796021274	0.229412350095659\\
10.7759590887054	0.229430179055398\\
10.783121390271	0.229447978488554\\
10.7902829273778	0.229465748462858\\
10.7974437014294	0.229483489045843\\
10.8046037138254	0.229501200304841\\
10.8117629659622	0.229518882306984\\
10.8189214592323	0.229536535119205\\
10.8260791950246	0.22955415880824\\
10.8332361747247	0.229571753440627\\
10.8403923997145	0.229589319082708\\
10.8475478713721	0.229606855800626\\
10.8547025910725	0.229624363660334\\
10.8618565601869	0.229641842727587\\
10.8690097800831	0.229659293067947\\
10.8761622521252	0.229676714746782\\
10.8833139776741	0.22969410782927\\
10.890464958087	0.229711472380396\\
10.8976151947178	0.229728808464953\\
10.9047646889168	0.229746116147546\\
10.9119134420309	0.229763395492588\\
10.9190614554035	0.229780646564305\\
10.9262087303749	0.229797869426734\\
10.9333552682815	0.229815064143724\\
10.9405010704567	0.229832230778938\\
10.9476461382302	0.229849369395853\\
10.9547904729287	0.22986648005776\\
10.9619340758752	0.229883562827765\\
10.9690769483895	0.22990061776879\\
10.976219091788	0.229917644943574\\
10.9833605073839	0.229934644414672\\
10.9905011964868	0.229951616244458\\
10.9976411604032	0.229968560495124\\
11.0047804004364	0.229985477228681\\
11.0119189178862	0.23000236650696\\
11.0190567140491	0.230019228391613\\
11.0261937902186	0.230036062944112\\
11.0333301476847	0.230052870225751\\
11.0404657877342	0.230069650297647\\
11.0476007116508	0.230086403220739\\
11.0547349207148	0.230103129055791\\
11.0618684162033	0.230119827863389\\
11.0690011993904	0.230136499703947\\
11.0761332715468	0.230153144637702\\
11.0832646339401	0.230169762724717\\
11.0903952878347	0.230186354024885\\
11.0975252344919	0.230202918597922\\
11.1046544751698	0.230219456503374\\
11.1117830111232	0.230235967800617\\
11.1189108436042	0.230252452548855\\
11.1260379738613	0.23026891080712\\
11.1331644031401	0.230285342634277\\
11.1402901326833	0.230301748089021\\
11.1474151637302	0.23031812722988\\
11.1545394975171	0.230334480115211\\
11.1616631352773	0.230350806803206\\
11.168786078241	0.230367107351891\\
11.1759083276353	0.230383381819125\\
11.1830298846844	0.230399630262602\\
11.1901507506094	0.230415852739849\\
11.1972709266283	0.230432049308233\\
11.2043904139561	0.230448220024954\\
11.211509213805	0.230464364947049\\
11.2186273273839	0.230480484131395\\
11.225744755899	0.230496577634704\\
11.2328615005534	0.230512645513528\\
11.2399775625471	0.230528687824259\\
11.2470929430775	0.230544704623128\\
11.2542076433387	0.230560695966205\\
11.2613216645221	0.230576661909403\\
11.268435007816	0.230592602508475\\
11.275547674406	0.230608517819016\\
11.2826596654745	0.230624407896466\\
11.2897709822012	0.230640272796104\\
11.296881625763	0.230656112573056\\
11.3039915973338	0.23067192728229\\
11.3111008980845	0.230687716978621\\
11.3182095291833	0.230703481716706\\
11.3253174917957	0.230719221551051\\
11.3324247870839	0.230734936536007\\
11.3395314162078	0.230750626725772\\
11.346637380324	0.230766292174391\\
11.3537426805867	0.230781932935757\\
11.360847318147	0.230797549063612\\
11.3679512941533	0.230813140611547\\
11.3750546097513	0.230828707633002\\
11.3821572660838	0.230844250181267\\
11.3892592642908	0.230859768309484\\
11.3963606055096	0.230875262070643\\
11.4034612908749	0.230890731517589\\
11.4105613215184	0.230906176703018\\
11.4176606985691	0.230921597679477\\
11.4247594231535	0.230936994499368\\
11.4318574963951	0.230952367214945\\
11.4389549194148	0.230967715878319\\
11.4460516933309	0.230983040541452\\
11.4531478192589	0.230998341256165\\
11.4602432983116	0.23101361807413\\
11.4673381315992	0.231028871046879\\
11.4744323202291	0.231044100225799\\
11.4815258653063	0.231059305662135\\
11.4886187679328	0.231074487406988\\
11.4957110292082	0.231089645511319\\
11.5028026502294	0.231104780025946\\
11.5098936320907	0.231119891001548\\
11.5169839758836	0.231134978488661\\
11.5240736826974	0.231150042537683\\
11.5311627536184	0.231165083198872\\
11.5382511897304	0.231180100522346\\
11.5453389921148	0.231195094558087\\
11.5524261618501	0.231210065355935\\
11.5595127000126	0.231225012965596\\
11.5665986076758	0.231239937436638\\
11.5736838859106	0.231254838818491\\
11.5807685357855	0.231269717160449\\
11.5878525583664	0.231284572511673\\
11.5949359547168	0.231299404921184\\
11.6020187258974	0.231314214437874\\
11.6091008729667	0.231329001110494\\
11.6161823969803	0.231343764987668\\
11.6232632989918	0.23135850611788\\
11.6303435800519	0.231373224549487\\
11.637423241209	0.231387920330708\\
11.644502283509	0.231402593509635\\
11.6515807079953	0.231417244134224\\
11.6586585157088	0.231431872252302\\
11.6657357076881	0.231446477911566\\
11.6728122849692	0.23146106115958\\
11.6798882485857	0.23147562204378\\
11.6869635995688	0.231490160611474\\
11.6940383389474	0.231504676909837\\
11.7011124677476	0.231519170985918\\
11.7081859869935	0.231533642886639\\
11.7152588977065	0.231548092658791\\
11.7223312009059	0.231562520349041\\
11.7294028976084	0.231576926003927\\
11.7364739888284	0.231591309669861\\
11.7435444755778	0.231605671393129\\
11.7506143588663	0.231620011219892\\
11.7576836397011	0.231634329196186\\
11.7647523190873	0.231648625367922\\
11.7718203980274	0.231662899780884\\
11.7788878775215	0.231677152480737\\
11.7859547585678	0.231691383513019\\
11.7930210421617	0.231705592923146\\
11.8000867292965	0.23171978075641\\
11.8071518209631	0.231733947057984\\
11.8142163181504	0.231748091872916\\
11.8212802218446	0.231762215246134\\
11.8283435330298	0.231776317222443\\
11.8354062526878	0.231790397846531\\
11.8424683817982	0.231804457162963\\
11.8495299213382	0.231818495216184\\
11.8565908722829	0.23183251205052\\
11.8636512356049	0.23184650771018\\
11.8707110122748	0.231860482239252\\
11.8777702032609	0.231874435681706\\
11.8848288095291	0.231888368081394\\
11.8918868320433	0.231902279482051\\
11.8989442717651	0.231916169927296\\
11.9060011296539	0.231930039460629\\
11.9130574066667	0.231943888125435\\
11.9201131037586	0.231957715964984\\
11.9271682218824	0.231971523022427\\
11.9342227619885	0.231985309340804\\
11.9412767250256	0.231999074963038\\
11.9483301119396	0.232012819931937\\
11.9553829236748	0.232026544290196\\
11.9624351611731	0.232040248080397\\
11.9694868253741	0.232053931345007\\
11.9765379172155	0.23206759412638\\
11.9835884376327	0.232081236466761\\
11.9906383875591	0.232094858408277\\
11.9976877679258	0.232108459992949\\
12.004736579662	0.232122041262682\\
12.0117848236945	0.232135602259272\\
12.0188325009483	0.232149143024405\\
12.025879612346	0.232162663599654\\
12.0329261588083	0.232176164026484\\
12.0399721412538	0.232189644346251\\
12.0470175605989	0.2322031046002\\
12.054062417758	0.232216544829468\\
12.0611067136434	0.232229965075082\\
12.0681504491654	0.232243365377963\\
12.0751936252322	0.232256745778923\\
12.0822362427498	0.232270106318665\\
12.0892783026224	0.232283447037788\\
12.096319805752	0.232296767976782\\
12.1033607530387	0.23231006917603\\
12.1104011453803	0.232323350675811\\
12.1174409836728	0.232336612516296\\
12.1244802688102	0.232349854737552\\
12.1315190016844	0.23236307737954\\
12.1385571831852	0.232376280482117\\
12.1455948142006	0.232389464085036\\
12.1526318956166	0.232402628227944\\
12.1596684283169	0.232415772950385\\
12.1667044131836	0.232428898291801\\
12.1737398510966	0.232442004291529\\
12.180774742934	0.232455090988805\\
12.1878090895716	0.23246815842276\\
12.1948428918836	0.232481206632426\\
12.2018761507422	0.232494235656731\\
12.2089088670174	0.232507245534503\\
12.2159410415774	0.232520236304467\\
12.2229726752886	0.232533208005249\\
12.2300037690152	0.232546160675373\\
12.2370343236197	0.232559094353265\\
12.2440643399627	0.232572009077249\\
12.2510938189025	0.232584904885551\\
12.2581227612961	0.232597781816296\\
12.265151167998	0.232610639907512\\
12.2721790398611	0.232623479197129\\
12.2792063777365	0.232636299722975\\
12.2862331824732	0.232649101522784\\
12.2932594549183	0.23266188463419\\
12.3002851959174	0.232674649094732\\
12.3073104063136	0.232687394941849\\
12.3143350869488	0.232700122212885\\
12.3213592386625	0.232712830945089\\
12.3283828622926	0.232725521175611\\
12.3354059586753	0.232738192941507\\
12.3424285286445	0.232750846279738\\
12.3494505730327	0.232763481227167\\
12.3564720926703	0.232776097820566\\
12.3634930883861	0.23278869609661\\
12.3705135610068	0.23280127609188\\
12.3775335113575	0.232813837842863\\
12.3845529402613	0.232826381385954\\
12.3915718485398	0.232838906757451\\
12.3985902370124	0.232851413993564\\
12.405608106497	0.232863903130405\\
12.4126254578096	0.232876374203997\\
12.4196422917644	0.232888827250269\\
12.4266586091738	0.23290126230506\\
12.4336744108484	0.232913679404116\\
12.4406896975973	0.23292607858309\\
12.4477044702274	0.232938459877548\\
12.4547187295441	0.232950823322961\\
12.4617324763511	0.232963168954714\\
12.4687457114501	0.232975496808097\\
12.4757584356413	0.232987806918314\\
12.482770649723	0.233000099320477\\
12.4897823544917	0.233012374049609\\
12.4967935507425	0.233024631140646\\
12.5038042392684	0.233036870628432\\
12.5108144208608	0.233049092547726\\
12.5178240963095	0.233061296933194\\
12.5248332664025	0.23307348381942\\
12.531841931926	0.233085653240895\\
12.5388500936647	0.233097805232026\\
12.5458577524013	0.233109939827131\\
12.5528649089171	0.233122057060442\\
12.5598715639916	0.233134156966105\\
12.5668777184026	0.233146239578177\\
12.5738833729263	0.233158304930633\\
12.580888528337	0.233170353057359\\
12.5878931854075	0.233182383992157\\
12.5948973449091	0.233194397768742\\
12.6019010076111	0.233206394420747\\
12.6089041742813	0.233218373981717\\
12.6159068456859	0.233230336485115\\
12.6229090225894	0.233242281964319\\
12.6299107057546	0.233254210452622\\
12.6369118959428	0.233266121983236\\
12.6439125939135	0.233278016589286\\
12.6509128004248	0.233289894303816\\
12.6579125162328	0.233301755159788\\
12.6649117420924	0.233313599190078\\
12.6719104787565	0.233325426427484\\
12.6789087269768	0.233337236904718\\
12.6859064875031	0.233349030654412\\
12.6929037610836	0.233360807709116\\
12.699900548465	0.233372568101299\\
12.7068968503924	0.233384311863348\\
12.7138926676094	0.23339603902757\\
12.7208880008577	0.233407749626191\\
12.7278828508777	0.233419443691356\\
12.7348772184082	0.233431121255131\\
12.7418711041864	0.233442782349501\\
12.7488645089478	0.233454427006373\\
12.7558574334265	0.233466055257571\\
12.7628498783551	0.233477667134845\\
12.7698418444643	0.233489262669862\\
12.7768333324837	0.233500841894212\\
12.783824343141	0.233512404839407\\
12.7908148771625	0.233523951536878\\
12.7978049352731	0.233535482017981\\
12.8047945181959	0.233546996313994\\
12.8117836266526	0.233558494456117\\
12.8187722613634	0.233569976475471\\
12.8257604230469	0.233581442403103\\
12.8327481124203	0.233592892269982\\
12.8397353301992	0.233604326106999\\
12.8467220770978	0.23361574394497\\
12.8537083538285	0.233627145814636\\
12.8606941611026	0.233638531746659\\
12.8676794996296	0.23364990177163\\
12.8746643701177	0.23366125592006\\
12.8816487732735	0.233672594222387\\
12.8886327098022	0.233683916708974\\
12.8956161804073	0.233695223410109\\
12.9025991857912	0.233706514356007\\
12.9095817266545	0.233717789576805\\
12.9165638036965	0.23372904910257\\
12.9235454176149	0.233740292963293\\
12.9305265691062	0.233751521188893\\
12.9375072588651	0.233762733809212\\
12.9444874875852	0.233773930854024\\
12.9514672559584	0.233785112353026\\
12.9584465646751	0.233796278335844\\
12.9654254144246	0.233807428832031\\
12.9724038058945	0.233818563871069\\
12.9793817397709	0.233829683482367\\
12.9863592167387	0.233840787695261\\
12.9933362374813	0.233851876539017\\
13.0003128026805	0.233862950042829\\
13.007288913017	0.23387400823582\\
13.0142645691699	0.233885051147042\\
13.0212397718168	0.233896078805476\\
13.0282145216341	0.233907091240032\\
13.0351888192966	0.233918088479552\\
13.0421626654779	0.233929070552804\\
13.0491360608501	0.23394003748849\\
13.0561090060839	0.233950989315239\\
13.0630815018486	0.233961926061613\\
13.0700535488123	0.233972847756104\\
13.0770251476413	0.233983754427133\\
13.0839962990011	0.233994646103055\\
13.0909670035553	0.234005522812154\\
13.0979372619664	0.234016384582647\\
13.1049070748955	0.234027231442683\\
13.1118764430024	0.234038063420341\\
13.1188453669454	0.234048880543635\\
13.1258138473815	0.234059682840507\\
13.1327818849665	0.234070470338837\\
13.1397494803545	0.234081243066433\\
13.1467166341986	0.234092001051039\\
13.1536833471505	0.23410274432033\\
13.1606496198604	0.234113472901916\\
13.1676154529773	0.234124186823339\\
13.1745808471489	0.234134886112077\\
13.1815458030214	0.234145570795539\\
13.1885103212399	0.234156240901071\\
13.195474402448	0.234166896455951\\
13.2024380472881	0.234177537487393\\
13.2094012564013	0.234188164022545\\
13.2163640304272	0.234198776088489\\
13.2233263700043	0.234209373712244\\
13.2302882757698	0.234219956920764\\
13.2372497483595	0.234230525740936\\
13.2442107884079	0.234241080199587\\
13.2511713965483	0.234251620323475\\
13.2581315734127	0.234262146139297\\
13.2650913196316	0.234272657673687\\
13.2720506358347	0.234283154953212\\
13.2790095226498	0.234293638004378\\
13.285967980704	0.234304106853629\\
13.2929260106228	0.234314561527342\\
13.2998836130305	0.234325002051835\\
13.3068407885502	0.234335428453362\\
13.3137975378035	0.234345840758113\\
13.3207538614112	0.234356238992218\\
13.3277097599924	0.234366623181743\\
13.3346652341651	0.234376993352694\\
13.3416202845463	0.234387349531014\\
13.3485749117513	0.234397691742584\\
13.3555291163945	0.234408020013225\\
13.3624828990889	0.234418334368696\\
13.3694362604464	0.234428634834694\\
13.3763892010775	0.234438921436857\\
13.3833417215916	0.234449194200761\\
13.3902938225969	0.234459453151922\\
13.3972455047003	0.234469698315796\\
13.4041967685074	0.234479929717778\\
13.4111476146227	0.234490147383204\\
13.4180980436496	0.234500351337348\\
13.42504805619	0.234510541605428\\
13.4319976528449	0.2345207182126\\
13.4389468342139	0.23453088118396\\
13.4458956008953	0.234541030544548\\
13.4528439534866	0.234551166319341\\
13.4597918925838	0.234561288533261\\
13.4667394187816	0.23457139721117\\
13.4736865326739	0.23458149237787\\
13.480633234853	0.234591574058108\\
13.4875795259104	0.234601642276569\\
13.4945254064362	0.234611697057882\\
13.5014708770193	0.23462173842662\\
13.5084159382476	0.234631766407296\\
13.5153605907075	0.234641781024366\\
13.5223048349848	0.234651782302228\\
13.5292486716635	0.234661770265226\\
13.5361921013269	0.234671744937643\\
13.5431351245569	0.234681706343708\\
13.5500777419343	0.234691654507592\\
13.5570199540389	0.234701589453411\\
13.5639617614491	0.234711511205224\\
13.5709031647424	0.234721419787032\\
13.5778441644949	0.234731315222783\\
13.5847847612818	0.234741197536367\\
13.591724955677	0.23475106675162\\
13.5986647482534	0.234760922892321\\
13.6056041395826	0.234770765982195\\
13.6125431302353	0.234780596044912\\
13.6194817207808	0.234790413104084\\
13.6264199117876	0.234800217183272\\
13.6333577038227	0.234810008305981\\
13.6402950974523	0.23481978649566\\
13.6472320932414	0.234829551775705\\
13.6541686917538	0.234839304169457\\
13.6611048935522	0.234849043700205\\
13.6680406991984	0.234858770391181\\
13.6749761092529	0.234868484265566\\
13.6819111242751	0.234878185346485\\
13.6888457448234	0.23488787365701\\
13.695779971455	0.234897549220162\\
13.7027138047261	0.234907212058906\\
13.7096472451918	0.234916862196155\\
13.716580293406	0.23492649965477\\
13.7235129499218	0.234936124457559\\
13.7304452152908	0.234945736627275\\
13.7373770900639	0.234955336186622\\
13.7443085747908	0.23496492315825\\
13.75123967002	0.234974497564757\\
13.7581703762991	0.23498405942869\\
13.7651006941745	0.234993608772542\\
13.7720306241917	0.235003145618756\\
13.778960166895	0.235012669989724\\
13.7858893228277	0.235022181907785\\
13.7928180925321	0.235031681395227\\
13.7997464765492	0.235041168474289\\
13.8066744754192	0.235050643167156\\
13.8136020896813	0.235060105495964\\
13.8205293198734	0.235069555482798\\
13.8274561665325	0.235078993149693\\
13.8343826301946	0.235088418518632\\
13.8413087113945	0.235097831611549\\
13.8482344106663	0.235107232450328\\
13.8551597285427	0.235116621056803\\
13.8620846655554	0.235125997452757\\
13.8690092222355	0.235135361659925\\
13.8759333991125	0.235144713699992\\
13.8828571967152	0.235154053594592\\
13.8897806155714	0.235163381365311\\
13.8967036562077	0.235172697033687\\
13.9036263191499	0.235182000621207\\
13.9105486049226	0.23519129214931\\
13.9174705140495	0.235200571639387\\
13.9243920470532	0.235209839112779\\
13.9313132044555	0.23521909459078\\
13.9382339867769	0.235228338094633\\
13.9451543945372	0.235237569645537\\
13.9520744282549	0.23524678926464\\
13.9589940884479	0.235255996973041\\
13.9659133756326	0.235265192791796\\
13.9728322903249	0.235274376741908\\
13.9797508330393	0.235283548844336\\
13.9866690042897	0.235292709119989\\
13.9935868045888	0.235301857589732\\
14.0005042344482	0.235310994274381\\
14.0074212943789	0.235320119194703\\
14.0143379848905	0.235329232371423\\
14.0212543064919	0.235338333825214\\
14.028170259691	0.235347423576706\\
14.0350858449946	0.235356501646482\\
14.0420010629087	0.235365568055077\\
14.0489159139382	0.235374622822981\\
14.0558303985872	0.235383665970638\\
14.0627445173586	0.235392697518446\\
14.0696582707545	0.235401717486757\\
14.0765716592762	0.235410725895876\\
14.0834846834237	0.235419722766066\\
14.0903973436962	0.23542870811754\\
14.0973096405922	0.235437681970469\\
14.1042215746089	0.235446644344977\\
14.1111331462427	0.235455595261144\\
14.118044355989	0.235464534739005\\
14.1249552043425	0.235473462798549\\
14.1318656917966	0.235482379459722\\
14.1387758188441	0.235491284742424\\
14.1456855859766	0.23550017866651\\
14.1525949936849	0.235509061251793\\
14.159504042459	0.235517932518041\\
14.1664127327877	0.235526792484975\\
14.1733210651591	0.235535641172277\\
14.1802290400603	0.23554447859958\\
14.1871366579774	0.235553304786478\\
14.1940439193958	0.235562119752518\\
14.2009508247998	0.235570923517204\\
14.2078573746729	0.235579716099999\\
14.2147635694975	0.235588497520319\\
14.2216694097553	0.23559726779754\\
14.2285748959271	0.235606026950994\\
14.2354800284927	0.235614774999969\\
14.242384807931	0.235623511963711\\
14.24928923472	0.235632237861424\\
14.2561933093369	0.235640952712269\\
14.2630970322579	0.235649656535364\\
14.2700004039584	0.235658349349785\\
14.2769034249129	0.235667031174566\\
14.2838060955949	0.235675702028699\\
14.2907084164771	0.235684361931133\\
14.2976103880313	0.235693010900776\\
14.3045120107285	0.235701648956494\\
14.3114132850387	0.235710276117112\\
14.3183142114311	0.235718892401413\\
14.325214790374	0.235727497828137\\
14.3321150223348	0.235736092415986\\
14.3390149077801	0.235744676183618\\
14.3459144471756	0.235753249149651\\
14.3528136409862	0.235761811332662\\
14.3597124896757	0.235770362751186\\
14.3666109937073	0.23577890342372\\
14.3735091535433	0.235787433368718\\
14.380406969645	0.235795952604593\\
14.387304442473	0.23580446114972\\
14.3942015724869	0.235812959022431\\
14.4010983601456	0.23582144624102\\
14.4079948059071	0.23582992282374\\
14.4148909102284	0.235838388788804\\
14.421786673566	0.235846844154384\\
14.4286820963752	0.235855288938615\\
14.4355771791106	0.23586372315959\\
14.4424719222261	0.235872146835363\\
14.4493663261745	0.235880559983948\\
14.4562603914079	0.235888962623321\\
14.4631541183776	0.235897354771417\\
14.4700475075341	0.235905736446135\\
14.4769405593269	0.23591410766533\\
14.4838332742049	0.235922468446823\\
14.4907256526158	0.235930818808393\\
14.497617695007	0.235939158767782\\
14.5045094018247	0.235947488342692\\
14.5114007735144	0.235955807550788\\
14.5182918105207	0.235964116409695\\
14.5251825132876	0.235972414937\\
14.532072882258	0.235980703150253\\
14.5389629178742	0.235988981066965\\
14.5458526205776	0.235997248704609\\
14.5527419908088	0.236005506080619\\
14.5596310290077	0.236013753212394\\
14.5665197356132	0.236021990117294\\
14.5734081110635	0.236030216812639\\
14.5802961557961	0.236038433315714\\
14.5871838702476	0.236046639643767\\
14.5940712548536	0.236054835814008\\
14.6009583100494	0.236063021843608\\
14.607845036269	0.236071197749704\\
14.614731433946	0.236079363549393\\
14.6216175035129	0.236087519259738\\
14.6285032454016	0.236095664897763\\
14.6353886600432	0.236103800480455\\
14.642273747868	0.236111926024767\\
14.6491585093054	0.236120041547613\\
14.6560429447843	0.236128147065871\\
14.6629270547325	0.236136242596384\\
14.6698108395772	0.236144328155957\\
14.6766942997448	0.23615240376136\\
14.683577435661	0.236160469429326\\
14.6904602477505	0.236168525176554\\
14.6973427364376	0.236176571019704\\
14.7042249021454	0.236184606975403\\
14.7111067452966	0.236192633060241\\
14.7179882663129	0.236200649290774\\
14.7248694656154	0.236208655683519\\
14.7317503436244	0.236216652254961\\
14.7386309007593	0.23622463902155\\
14.7455111374389	0.236232615999697\\
14.7523910540812	0.236240583205781\\
14.7592706511034	0.236248540656146\\
14.7661499289221	0.2362564883671\\
14.773028887953	0.236264426354916\\
14.7799075286111	0.236272354635832\\
14.7867858513106	0.236280273226054\\
14.7936638564651	0.23628818214175\\
14.8005415444873	0.236296081399056\\
14.8074189157893	0.236303971014071\\
14.8142959707823	0.236311851002863\\
14.8211727098769	0.236319721381464\\
14.8280491334828	0.236327582165871\\
14.8349252420093	0.236335433372048\\
14.8418010358646	0.236343275015926\\
14.8486765154563	0.2363511071134\\
14.8555516811914	0.236358929680333\\
14.8624265334759	0.236366742732554\\
14.8693010727154	0.236374546285858\\
14.8761752993146	0.236382340356007\\
14.8830492136774	0.236390124958728\\
14.8899228162071	0.236397900109716\\
14.8967961073062	0.236405665824634\\
14.9036690873767	0.236413422119109\\
14.9105417568196	0.236421169008738\\
14.9174141160352	0.236428906509082\\
14.9242861654234	0.236436634635671\\
14.9311579053831	0.236444353404002\\
14.9380293363126	0.236452062829539\\
14.9449004586094	0.236459762927713\\
14.9517712726704	0.236467453713923\\
14.9586417788917	0.236475135203535\\
14.9655119776689	0.236482807411883\\
14.9723818693966	0.236490470354269\\
14.9792514544689	0.236498124045963\\
14.9861207332792	0.2365057685022\\
14.9929897062201	0.236513403738187\\
14.9998583736836	0.236521029769097\\
15.0067267360609	0.236528646610072\\
15.0135947937427	0.236536254276219\\
15.0204625471189	0.236543852782619\\
15.0273299965786	0.236551442144315\\
15.0341971425103	0.236559022376324\\
15.041063985302	0.236566593493628\\
15.0479305253406	0.236574155511179\\
15.0547967630128	0.236581708443897\\
15.0616626987043	0.236589252306671\\
15.0685283328002	0.23659678711436\\
15.0753936656848	0.236604312881789\\
15.0822586977421	0.236611829623756\\
15.089123429355	0.236619337355024\\
15.095987860906	0.236626836090328\\
15.1028519927768	0.236634325844372\\
15.1097158253484	0.236641806631827\\
15.1165793590013	0.236649278467336\\
15.1234425941152	0.23665674136551\\
15.1303055310692	0.23666419534093\\
15.1371681702416	0.236671640408148\\
15.1440305120102	0.236679076581683\\
15.1508925567521	0.236686503876025\\
15.1577543048437	0.236693922305634\\
15.1646157566607	0.236701331884941\\
15.1714769125784	0.236708732628345\\
15.178337772971	0.236716124550216\\
15.1851983382124	0.236723507664895\\
15.1920586086758	0.236730881986691\\
15.1989185847337	0.236738247529886\\
15.2057782667578	0.23674560430873\\
15.2126376551195	0.236752952337446\\
15.2194967501892	0.236760291630224\\
15.2263555523369	0.236767622201229\\
15.2332140619318	0.236774944064592\\
15.2400722793425	0.23678225723442\\
15.2469302049371	0.236789561724785\\
15.2537878390828	0.236796857549735\\
15.2606451821465	0.236804144723286\\
15.2675022344941	0.236811423259427\\
15.2743589964911	0.236818693172115\\
15.2812154685022	0.236825954475282\\
15.2880716508918	0.236833207182829\\
15.2949275440232	0.236840451308629\\
15.3017831482595	0.236847686866526\\
15.3086384639629	0.236854913870336\\
15.3154934914951	0.236862132333847\\
15.322348231217	0.236869342270816\\
15.3292026834892	0.236876543694976\\
15.3360568486714	0.236883736620028\\
15.3429107271227	0.236890921059647\\
15.3497643192017	0.236898097027479\\
15.3566176252664	0.236905264537143\\
15.3634706456741	0.236912423602228\\
15.3703233807814	0.236919574236297\\
15.3771758309445	0.236926716452884\\
15.3840279965188	0.236933850265497\\
15.3908798778592	0.236940975687615\\
15.3977314753199	0.236948092732689\\
15.4045827892547	0.236955201414144\\
15.4114338200164	0.236962301745376\\
15.4182845679576	0.236969393739754\\
15.4251350334301	0.23697647741062\\
15.431985216785	0.23698355277129\\
15.4388351183731	0.23699061983505\\
15.4456847385444	0.236997678615161\\
15.4525340776483	0.237004729124857\\
15.4593831360335	0.237011771377343\\
15.4662319140484	0.2370188053858\\
15.4730804120406	0.237025831163379\\
15.479928630357	0.237032848723208\\
15.4867765693443	0.237039858078384\\
15.4936242293481	0.23704685924198\\
15.5004716107139	0.237053852227043\\
15.5073187137862	0.237060837046592\\
15.5141655389091	0.237067813713619\\
15.5210120864263	0.237074782241092\\
15.5278583566805	0.237081742641951\\
15.5347043500141	0.237088694929109\\
15.541550066769	0.237095639115454\\
15.5483955072861	0.237102575213849\\
15.5552406719063	0.237109503237129\\
15.5620855609694	0.237116423198103\\
15.5689301748149	0.237123335109556\\
15.5757745137817	0.237130238984245\\
15.5826185782081	0.237137134834903\\
15.5894623684317	0.237144022674235\\
15.5963058847898	0.237150902514923\\
15.6031491276189	0.237157774369622\\
15.609992097255	0.237164638250961\\
15.6168347940335	0.237171494171543\\
15.6236772182893	0.237178342143949\\
15.6305193703568	0.237185182180731\\
15.6373612505696	0.237192014294418\\
15.644202859261	0.237198838497511\\
15.6510441967635	0.237205654802489\\
15.6578852634092	0.237212463221805\\
15.6647260595296	0.237219263767885\\
15.6715665854556	0.237226056453132\\
15.6784068415176	0.237232841289924\\
15.6852468280454	0.237239618290613\\
15.6920865453683	0.237246387467527\\
15.6989259938151	0.237253148832971\\
15.7057651737137	0.237259902399221\\
15.712604085392	0.237266648178532\\
15.7194427291769	0.237273386183133\\
15.7262811053949	0.23728011642523\\
15.733119214372	0.237286838917002\\
15.7399570564337	0.237293553670606\\
15.7467946319047	0.237300260698172\\
15.7536319411094	0.23730696001181\\
15.7604689843717	0.237313651623601\\
15.7673057620146	0.237320335545606\\
15.774142274361	0.237327011789859\\
15.780978521733	0.23733368036837\\
15.7878145044522	0.237340341293128\\
15.7946502228398	0.237346994576095\\
15.8014856772161	0.237353640229211\\
15.8083208679014	0.237360278264391\\
15.815155795215	0.237366908693527\\
15.8219904594759	0.237373531528487\\
15.8288248610025	0.237380146781116\\
15.8356590001127	0.237386754463234\\
15.8424928771238	0.237393354586639\\
15.8493264923528	0.237399947163105\\
15.8561598461158	0.237406532204382\\
15.8629929387287	0.237413109722199\\
15.8698257705067	0.237419679728258\\
15.8766583417646	0.237426242234241\\
15.8834906528165	0.237432797251806\\
15.8903227039761	0.237439344792586\\
15.8971544955567	0.237445884868194\\
15.9039860278707	0.237452417490218\\
15.9108173012305	0.237458942670223\\
15.9176483159475	0.237465460419753\\
15.9244790723329	0.237471970750326\\
15.9313095706972	0.237478473673441\\
15.9381398113506	0.237484969200571\\
15.9449697946025	0.237491457343168\\
15.9517995207621	0.237497938112662\\
15.9586289901378	0.237504411520459\\
15.9654582030377	0.237510877577943\\
15.9722871597692	0.237517336296476\\
15.9791158606394	0.237523787687396\\
15.9859443059549	0.237530231762022\\
15.9927724960215	0.237536668531646\\
15.9996004311448	0.237543098007543\\
16.0064281116297	0.237549520200962\\
16.0132555377808	0.237555935123131\\
16.0200827099021	0.237562342785256\\
16.026909628297	0.237568743198522\\
16.0337362932685	0.23757513637409\\
16.0405627051191	0.2375815223231\\
16.0473888641508	0.237587901056671\\
16.0542147706651	0.2375942725859\\
16.061040424963	0.237600636921861\\
16.0678658273451	0.237606994075607\\
16.0746909781113	0.237613344058169\\
16.0815158775611	0.237619686880558\\
16.0883405259937	0.237626022553761\\
16.0951649237075	0.237632351088746\\
16.1019890710006	0.237638672496458\\
16.1088129681706	0.23764498678782\\
16.1156366155146	0.237651293973736\\
16.1224600133292	0.237657594065086\\
16.1292831619104	0.237663887072731\\
16.136106061554	0.23767017300751\\
16.1429287125551	0.23767645188024\\
16.1497511152083	0.237682723701718\\
16.1565732698078	0.23768898848272\\
16.1633951766475	0.237695246234\\
16.1702168360204	0.237701496966293\\
16.1770382482193	0.23770774069031\\
16.1838594135366	0.237713977416744\\
16.190680332264	0.237720207156267\\
16.1975010046929	0.237726429919528\\
16.2043214311141	0.237732645717158\\
16.2111416118181	0.237738854559765\\
16.2179615470946	0.237745056457938\\
16.2247812372332	0.237751251422246\\
16.2316006825229	0.237757439463235\\
16.2384198832522	0.237763620591434\\
16.245238839709	0.237769794817348\\
16.2520575521811	0.237775962151463\\
16.2588760209554	0.237782122604247\\
16.2656942463186	0.237788276186145\\
16.272512228557	0.237794422907582\\
16.2793299679562	0.237800562778963\\
16.2861474648015	0.237806695810675\\
16.2929647193777	0.237812822013083\\
16.2997817319691	0.237818941396531\\
16.3065985028597	0.237825053971345\\
16.3134150323327	0.23783115974783\\
16.3202313206713	0.237837258736273\\
16.3270473681579	0.237843350946938\\
16.3338631750745	0.237849436390071\\
16.3406787417028	0.237855515075899\\
16.3474940683239	0.237861587014629\\
16.3543091552186	0.237867652216446\\
16.3611240026669	0.237873710691519\\
16.3679386109489	0.237879762449994\\
16.3747529803437	0.237885807502001\\
16.3815671111304	0.237891845857648\\
16.3883810035872	0.237897877527024\\
16.3951946579924	0.2379039025202\\
16.4020080746233	0.237909920847226\\
16.4088212537572	0.237915932518133\\
16.4156341956706	0.237921937542933\\
16.4224469006399	0.237927935931621\\
16.4292593689408	0.237933927694168\\
16.4360716008487	0.237939912840531\\
16.4428835966384	0.237945891380645\\
16.4496953565845	0.237951863324426\\
16.4565068809609	0.237957828681773\\
16.4633181700412	0.237963787462564\\
16.4701292240987	0.237969739676658\\
16.476940043406	0.237975685333899\\
16.4837506282354	0.237981624444106\\
16.4905609788587	0.237987557017085\\
16.4973710955474	0.237993483062621\\
16.5041809785725	0.237999402590478\\
16.5109906282045	0.238005315610406\\
16.5178000447135	0.238011222132132\\
16.5246092283692	0.238017122165369\\
16.531418179441	0.238023015719807\\
16.5382268981976	0.23802890280512\\
16.5450353849074	0.238034783430965\\
16.5518436398385	0.238040657606976\\
16.5586516632584	0.238046525342774\\
16.5654594554342	0.238052386647959\\
16.5722670166327	0.238058241532112\\
16.5790743471201	0.238064090004798\\
16.5858814471624	0.238069932075561\\
16.5926883170248	0.238075767753931\\
16.5994949569726	0.238081597049417\\
16.6063013672703	0.23808741997151\\
16.6131075481821	0.238093236529684\\
16.6199134999718	0.238099046733394\\
16.6267192229027	0.238104850592079\\
16.6335247172377	0.238110648115158\\
16.6403299832395	0.238116439312034\\
16.6471350211701	0.23812222419209\\
16.6539398312912	0.238128002764695\\
16.6607444138642	0.238133775039196\\
16.6675487691498	0.238139541024925\\
16.6743528974086	0.238145300731196\\
16.6811567989005	0.238151054167306\\
16.6879604738854	0.238156801342531\\
16.6947639226224	0.238162542266135\\
16.7015671453703	0.23816827694736\\
16.7083701423875	0.238174005395433\\
16.7151729139322	0.238179727619564\\
16.7219754602619	0.238185443628942\\
16.7287777816337	0.238191153432744\\
16.7355798783047	0.238196857040126\\
16.742381750531	0.238202554460229\\
16.7491833985688	0.238208245702174\\
16.7559848226736	0.238213930775067\\
16.7627860231007	0.238219609687998\\
16.7695870001049	0.238225282450038\\
16.7763877539405	0.23823094907024\\
16.7831882848616	0.238236609557644\\
16.7899885931218	0.238242263921269\\
16.7967886789743	0.238247912170119\\
16.8035885426719	0.238253554313181\\
16.8103881844671	0.238259190359425\\
16.8171876046118	0.238264820317805\\
16.8239868033579	0.238270444197257\\
16.8307857809564	0.238276062006701\\
16.8375845376582	0.23828167375504\\
16.8443830737138	0.23828727945116\\
16.8511813893734	0.238292879103932\\
16.8579794848865	0.23829847272221\\
16.8647773605025	0.238304060314829\\
16.8715750164703	0.238309641890612\\
16.8783724530384	0.238315217458361\\
16.885169670455	0.238320787026864\\
16.8919666689678	0.238326350604893\\
16.8987634488242	0.238331908201203\\
16.9055600102712	0.238337459824532\\
16.9123563535553	0.238343005483602\\
16.9191524789229	0.238348545187121\\
16.9259483866198	0.238354078943778\\
16.9327440768913	0.238359606762247\\
16.9395395499827	0.238365128651186\\
16.9463348061386	0.238370644619237\\
16.9531298456033	0.238376154675025\\
16.9599246686207	0.23838165882716\\
16.9667192754346	0.238387157084236\\
16.973513666288	0.238392649454831\\
16.9803078414237	0.238398135947506\\
16.9871018010843	0.238403616570809\\
16.9938955455117	0.238409091333269\\
17.0006890749477	0.2384145602434\\
17.0074823896337	0.238420023309703\\
17.0142754898105	0.238425480540659\\
17.0210683757188	0.238430931944736\\
17.0278610475987	0.238436377530387\\
17.0346535056902	0.238441817306046\\
17.0414457502326	0.238447251280136\\
17.0482377814652	0.238452679461061\\
17.0550295996266	0.23845810185721\\
17.0618212049552	0.238463518476958\\
17.068612597689	0.238468929328664\\
17.0754037780657	0.23847433442067\\
17.0821947463226	0.238479733761306\\
17.0889855026965	0.238485127358883\\
17.0957760474241	0.238490515221699\\
17.1025663807414	0.238495897358036\\
17.1093565028844	0.238501273776162\\
17.1161464140886	0.238506644484327\\
17.1229361145889	0.238512009490769\\
17.1297256046203	0.238517368803708\\
17.136514884417	0.238522722431352\\
17.1433039542131	0.238528070381892\\
17.1500928142423	0.238533412663504\\
17.1568814647379	0.238538749284349\\
17.1636699059328	0.238544080252574\\
17.1704581380598	0.23854940557631\\
17.1772461613509	0.238554725263674\\
17.1840339760382	0.238560039322767\\
17.1908215823532	0.238565347761678\\
17.197608980527	0.238570650588477\\
17.2043961707906	0.238575947811222\\
17.2111831533744	0.238581239437957\\
17.2179699285086	0.238586525476708\\
17.2247564964229	0.23859180593549\\
17.2315428573468	0.238597080822301\\
17.2383290115094	0.238602350145125\\
17.2451149591394	0.238607613911933\\
17.2519007004653	0.238612872130678\\
17.2586862357151	0.238618124809302\\
17.2654715651165	0.238623371955731\\
17.2722566888969	0.238628613577877\\
17.2790416072832	0.238633849683636\\
17.2858263205023	0.238639080280893\\
17.2926108287804	0.238644305377515\\
17.2993951323434	0.238649524981357\\
17.3061792314172	0.238654739100259\\
17.3129631262269	0.238659947742047\\
17.3197468169975	0.238665150914533\\
17.3265303039537	0.238670348625514\\
17.3333135873198	0.238675540882774\\
17.3400966673196	0.238680727694082\\
17.3468795441769	0.238685909067192\\
17.353662218115	0.238691085009847\\
17.3604446893566	0.238696255529773\\
17.3672269581246	0.238701420634683\\
17.374009024641	0.238706580332277\\
17.380790889128	0.23871173463024\\
17.387572551807	0.238716883536243\\
17.3943540128993	0.238722027057944\\
17.401135272626	0.238727165202987\\
17.4079163312075	0.238732297979001\\
17.4146971888642	0.238737425393602\\
17.421477845816	0.238742547454393\\
17.4282583022825	0.238747664168963\\
17.4350385584831	0.238752775544887\\
17.4418186146366	0.238757881589725\\
17.4485984709617	0.238762982311027\\
17.4553781276767	0.238768077716325\\
17.4621575849996	0.23877316781314\\
17.468936843148	0.238778252608981\\
17.4757159023393	0.238783332111339\\
17.4824947627905	0.238788406327695\\
17.4892734247181	0.238793475265517\\
17.4960518883387	0.238798538932256\\
17.5028301538681	0.238803597335354\\
17.5096082215222	0.238808650482236\\
17.5163860915162	0.238813698380316\\
17.5231637640654	0.238818741036993\\
17.5299412393843	0.238823778459655\\
17.5367185176874	0.238828810655674\\
17.5434955991889	0.238833837632412\\
17.5502724841025	0.238838859397214\\
17.5570491726417	0.238843875957415\\
17.5638256650196	0.238848887320336\\
17.5706019614491	0.238853893493283\\
17.5773780621427	0.238858894483553\\
17.5841539673125	0.238863890298426\\
17.5909296771705	0.238868880945171\\
17.5977051919283	0.238873866431044\\
17.604480511797	0.238878846763287\\
17.6112556369877	0.238883821949129\\
17.618030567711	0.238888791995789\\
17.6248053041771	0.238893756910469\\
17.6315798465962	0.238898716700361\\
17.6383541951778	0.238903671372643\\
17.6451283501314	0.238908620934481\\
17.6519023116661	0.238913565393028\\
17.6586760799906	0.238918504755423\\
17.6654496553134	0.238923439028793\\
17.6722230378426	0.238928368220255\\
17.6789962277861	0.238933292336909\\
17.6857692253515	0.238938211385845\\
17.6925420307458	0.23894312537414\\
17.6993146441762	0.238948034308858\\
17.7060870658491	0.238952938197051\\
17.712859295971	0.238957837045759\\
17.7196313347478	0.238962730862007\\
17.7264031823852	0.23896761965281\\
17.7331748390886	0.238972503425171\\
17.7399463050632	0.238977382186078\\
17.7467175805137	0.238982255942509\\
17.7534886656447	0.238987124701429\\
17.7602595606603	0.238991988469789\\
17.7670302657644	0.23899684725453\\
17.7738007811607	0.239001701062581\\
17.7805711070524	0.239006549900856\\
17.7873412436426	0.239011393776259\\
17.7941111911339	0.239016232695682\\
17.8008809497288	0.239021066666003\\
17.8076505196294	0.239025895694089\\
17.8144199010375	0.239030719786795\\
17.8211890941546	0.239035538950965\\
17.8279580991819	0.239040353193428\\
17.8347269163204	0.239045162521003\\
17.8414955457708	0.239049966940498\\
17.8482639877332	0.239054766458706\\
17.8550322424079	0.239059561082411\\
17.8618003099946	0.239064350818383\\
17.8685681906927	0.239069135673381\\
17.8753358847014	0.239073915654153\\
17.8821033922196	0.239078690767432\\
17.8888707134459	0.239083461019944\\
17.8956378485786	0.239088226418399\\
17.9024047978157	0.239092986969497\\
17.909171561355	0.239097742679926\\
17.9159381393939	0.239102493556363\\
17.9227045321296	0.239107239605472\\
17.9294707397588	0.239111980833905\\
17.9362367624783	0.239116717248305\\
17.9430026004843	0.239121448855301\\
17.9497682539727	0.239126175661511\\
17.9565337231395	0.239130897673541\\
17.9632990081799	0.239135614897987\\
17.9700641092891	0.239140327341432\\
17.976829026662	0.239145035010447\\
17.9835937604933	0.239149737911593\\
17.9903583109772	0.23915443605142\\
17.9971226783077	0.239159129436464\\
18.0038868626787	0.239163818073252\\
18.0106508642835	0.239168501968299\\
18.0174146833153	0.239173181128107\\
18.0241783199672	0.23917785555917\\
18.0309417744316	0.239182525267967\\
18.037705046901	0.239187190260969\\
18.0444681375674	0.239191850544633\\
18.0512310466226	0.239196506125406\\
18.0579937742581	0.239201157009725\\
18.0647563206653	0.239205803204014\\
18.071518686035	0.239210444714686\\
18.078280870558	0.239215081548144\\
18.0850428744246	0.239219713710778\\
18.091804697825	0.239224341208969\\
18.0985663409491	0.239228964049086\\
18.1053278039865	0.239233582237487\\
18.1120890871265	0.239238195780519\\
18.1188501905581	0.239242804684517\\
18.1256111144702	0.239247408955806\\
18.1323718590512	0.239252008600701\\
18.1391324244894	0.239256603625505\\
18.1458928109727	0.23926119403651\\
18.1526530186889	0.239265779839996\\
18.1594130478254	0.239270361042236\\
18.1661728985693	0.239274937649487\\
18.1729325711075	0.239279509667999\\
18.1796920656267	0.23928407710401\\
18.1864513823132	0.239288639963746\\
18.193210521353	0.239293198253426\\
18.1999694829322	0.239297751979253\\
18.206728267236	0.239302301147424\\
18.21348687445	0.239306845764122\\
18.220245304759	0.239311385835521\\
18.227003558348	0.239315921367785\\
18.2337616354012	0.239320452367065\\
18.2405195361031	0.239324978839504\\
18.2472772606375	0.239329500791232\\
18.2540348091883	0.239334018228371\\
18.2607921819387	0.239338531157031\\
18.2675493790721	0.239343039583311\\
18.2743064007713	0.239347543513301\\
18.2810632472191	0.23935204295308\\
18.2878199185977	0.239356537908715\\
18.2945764150894	0.239361028386264\\
18.3013327368761	0.239365514391776\\
18.3080888841393	0.239369995931287\\
18.3148448570605	0.239374473010824\\
18.3216006558207	0.239378945636404\\
18.3283562806009	0.239383413814033\\
18.3351117315815	0.239387877549706\\
18.341867008943	0.239392336849409\\
18.3486221128655	0.239396791719118\\
18.3553770435287	0.239401242164798\\
18.3621318011123	0.239405688192403\\
18.3688863857955	0.239410129807879\\
18.3756407977575	0.23941456701716\\
18.382395037177	0.23941899982617\\
18.3891491042327	0.239423428240825\\
18.3959029991027	0.239427852267027\\
18.4026567219653	0.239432271910672\\
18.4094102729981	0.239436687177644\\
18.4161636523787	0.239441098073816\\
18.4229168602845	0.239445504605053\\
18.4296698968925	0.239449906777209\\
18.4364227623794	0.239454304596128\\
18.4431754569219	0.239458698067644\\
18.4499279806963	0.239463087197582\\
18.4566803338785	0.239467471991755\\
18.4634325166445	0.239471852455969\\
18.4701845291697	0.239476228596018\\
18.4769363716295	0.239480600417686\\
18.4836880441989	0.239484967926749\\
18.4904395470528	0.239489331128972\\
18.4971908803658	0.23949369003011\\
18.5039420443121	0.239498044635909\\
18.5106930390659	0.239502394952104\\
18.5174438648009	0.239506740984423\\
18.5241945216909	0.23951108273858\\
18.5309450099091	0.239515420220283\\
18.5376953296286	0.239519753435229\\
18.5444454810223	0.239524082389106\\
18.5511954642629	0.23952840708759\\
18.5579452795226	0.239532727536351\\
18.5646949269737	0.239537043741047\\
18.5714444067881	0.239541355707327\\
18.5781937191373	0.23954566344083\\
18.5849428641929	0.239549966947187\\
18.591691842126	0.239554266232018\\
18.5984406531075	0.239558561300934\\
18.6051892973081	0.239562852159537\\
18.6119377748983	0.239567138813418\\
18.6186860860484	0.23957142126816\\
18.6254342309282	0.239575699529336\\
18.6321822097076	0.239579973602511\\
18.6389300225561	0.239584243493239\\
18.6456776696429	0.239588509207064\\
18.6524251511371	0.239592770749523\\
18.6591724672074	0.239597028126143\\
18.6659196180225	0.239601281342439\\
18.6726666037507	0.239605530403921\\
18.6794134245601	0.239609775316087\\
18.6861600806185	0.239614016084426\\
18.6929065720936	0.239618252714419\\
18.6996528991527	0.239622485211536\\
18.7063990619631	0.23962671358124\\
18.7131450606917	0.239630937828983\\
18.7198908955051	0.239635157960208\\
18.7266365665699	0.23963937398035\\
18.7333820740523	0.239643585894834\\
18.7401274181183	0.239647793709077\\
18.7468725989336	0.239651997428486\\
18.7536176166639	0.239656197058458\\
18.7603624714744	0.239660392604383\\
18.7671071635303	0.23966458407164\\
18.7738516929964	0.239668771465602\\
18.7805960600373	0.239672954791629\\
18.7873402648174	0.239677134055076\\
18.794084307501	0.239681309261285\\
18.8008281882519	0.239685480415593\\
18.807571907234	0.239689647523326\\
18.8143154646106	0.239693810589802\\
18.8210588605451	0.239697969620328\\
18.8278020952006	0.239702124620204\\
18.8345451687398	0.239706275594722\\
18.8412880813253	0.239710422549164\\
18.8480308331196	0.239714565488803\\
18.8547734242848	0.239718704418903\\
18.8615158549827	0.239722839344721\\
18.8682581253752	0.239726970271503\\
18.8750002356237	0.239731097204487\\
18.8817421858894	0.239735220148904\\
18.8884839763334	0.239739339109973\\
18.8952256071166	0.239743454092908\\
18.9019670783994	0.239747565102912\\
18.9087083903424	0.23975167214518\\
18.9154495431055	0.239755775224898\\
18.9221905368489	0.239759874347244\\
18.9289313717322	0.239763969517387\\
18.9356720479148	0.239768060740487\\
18.9424125655561	0.239772148021698\\
18.9491529248152	0.239776231366161\\
18.9558931258508	0.239780310779013\\
18.9626331688217	0.239784386265379\\
18.9693730538861	0.239788457830378\\
18.9761127812024	0.239792525479118\\
18.9828523509284	0.239796589216703\\
18.989591763222	0.239800649048223\\
18.9963310182407	0.239804704978763\\
19.0030701161417	0.239808757013399\\
19.0098090570823	0.239812805157198\\
19.0165478412193	0.239816849415221\\
19.0232864687095	0.239820889792516\\
19.0300249397092	0.239824926294127\\
19.0367632543747	0.239828958925088\\
19.0435014128621	0.239832987690425\\
19.0502394153273	0.239837012595155\\
19.0569772619258	0.239841033644287\\
19.063714952813	0.239845050842823\\
19.0704524881441	0.239849064195756\\
19.0771898680742	0.239853073708069\\
19.083927092758	0.23985707938474\\
19.09066416235	0.239861081230736\\
19.0974010770047	0.239865079251018\\
19.1041378368761	0.239869073450538\\
19.1108744421182	0.239873063834238\\
19.1176108928848	0.239877050407056\\
19.1243471893293	0.239881033173918\\
19.131083331605	0.239885012139745\\
19.1378193198651	0.239888987309447\\
19.1445551542624	0.239892958687927\\
19.1512908349496	0.239896926280082\\
19.1580263620793	0.239900890090798\\
19.1647617358037	0.239904850124955\\
19.1714969562748	0.239908806387424\\
19.1782320236445	0.239912758883069\\
19.1849669380645	0.239916707616744\\
19.1917016996862	0.239920652593298\\
19.1984363086609	0.23992459381757\\
19.2051707651395	0.239928531294391\\
19.2119050692731	0.239932465028585\\
19.2186392212121	0.239936395024969\\
19.225373221107	0.23994032128835\\
19.2321070691081	0.239944243823527\\
19.2388407653653	0.239948162635295\\
19.2455743100286	0.239952077728437\\
19.2523077032474	0.239955989107729\\
19.2590409451713	0.239959896777942\\
19.2657740359494	0.239963800743835\\
19.2725069757308	0.239967701010163\\
19.2792397646643	0.239971597581671\\
19.2859724028985	0.239975490463097\\
19.2927048905819	0.239979379659171\\
19.2994372278625	0.239983265174616\\
19.3061694148886	0.239987147014147\\
19.3129014518078	0.239991025182471\\
19.3196333387678	0.239994899684288\\
19.3263650759161	0.239998770524289\\
19.3330966633998	0.240002637707159\\
19.339828101366	0.240006501237575\\
19.3465593899615	0.240010361120206\\
19.353290529333	0.240014217359713\\
19.3600215196268	0.240018069960751\\
19.3667523609893	0.240021918927966\\
19.3734830535665	0.240025764265996\\
19.3802135975042	0.240029605979475\\
19.3869439929481	0.240033444073024\\
19.3936742400436	0.240037278551262\\
19.400404338936	0.240041109418796\\
19.4071342897705	0.240044936680229\\
19.4138640926918	0.240048760340154\\
19.4205937478447	0.240052580403159\\
19.4273232553736	0.240056396873823\\
19.4340526154229	0.240060209756717\\
19.4407818281366	0.240064019056406\\
19.4475108936587	0.240067824777448\\
19.4542398121329	0.240071626924391\\
19.4609685837028	0.24007542550178\\
19.4676972085116	0.240079220514148\\
19.4744256867025	0.240083011966023\\
19.4811540184185	0.240086799861927\\
19.4878822038024	0.240090584206373\\
19.4946102429967	0.240094365003866\\
19.5013381361438	0.240098142258905\\
19.508065883386	0.240101915975982\\
19.5147934848652	0.240105686159582\\
19.5215209407232	0.240109452814181\\
19.5282482511018	0.240113215944249\\
19.5349754161423	0.240116975554249\\
19.541702435986	0.240120731648636\\
19.548429310774	0.24012448423186\\
19.5551560406472	0.240128233308361\\
19.5618826257462	0.240131978882573\\
19.5686090662116	0.240135720958924\\
19.5753353621837	0.240139459541834\\
19.5820615138027	0.240143194635715\\
19.5887875212085	0.240146926244973\\
19.5955133845409	0.240150654374008\\
19.6022391039394	0.24015437902721\\
19.6089646795435	0.240158100208965\\
19.6156901114924	0.240161817923649\\
19.6224153999251	0.240165532175635\\
19.6291405449805	0.240169242969285\\
19.6358655467973	0.240172950308957\\
19.6425904055138	0.240176654199\\
19.6493151212685	0.240180354643756\\
19.6560396941994	0.240184051647563\\
19.6627641244445	0.240187745214748\\
19.6694884121416	0.240191435349635\\
19.6762125574282	0.240195122056537\\
19.6829365604417	0.240198805339764\\
19.6896604213193	0.240202485203617\\
19.6963841401981	0.24020616165239\\
19.7031077172148	0.240209834690371\\
19.7098311525063	0.240213504321842\\
19.7165544462089	0.240217170551076\\
19.7232775984591	0.240220833382341\\
19.7300006093928	0.240224492819897\\
19.7367234791462	0.240228148867998\\
19.7434462078549	0.240231801530891\\
19.7501687956546	0.240235450812816\\
19.7568912426806	0.240239096718007\\
19.7636135490683	0.240242739250691\\
19.7703357149527	0.240246378415087\\
19.7770577404686	0.24025001421541\\
19.7837796257509	0.240253646655866\\
19.790501370934	0.240257275740654\\
19.7972229761523	0.240260901473969\\
19.80394444154	0.240264523859998\\
19.8106657672311	0.240268142902919\\
19.8173869533595	0.240271758606908\\
19.8241080000587	0.24027537097613\\
19.8308289074623	0.240278980014747\\
19.8375496757037	0.240282585726911\\
19.8442703049158	0.240286188116771\\
19.8509907952318	0.240289787188466\\
19.8577111467843	0.240293382946131\\
19.8644313597061	0.240296975393893\\
19.8711514341296	0.240300564535874\\
19.8778713701869	0.240304150376188\\
19.8845911680103	0.240307732918943\\
19.8913108277317	0.240311312168241\\
19.8980303494828	0.240314888128177\\
19.9047497333952	0.240318460802839\\
19.9114689796003	0.24032203019631\\
19.9181880882294	0.240325596312665\\
19.9249070594135	0.240329159155974\\
19.9316258932836	0.2403327187303\\
19.9383445899704	0.240336275039699\\
19.9450631496044	0.240339828088222\\
19.951781572316	0.240343377879913\\
19.9584998582356	0.240346924418808\\
19.965218007493	0.24035046770894\\
19.9719360202183	0.240354007754332\\
19.9786538965411	0.240357544559004\\
19.9853716365909	0.240361078126968\\
19.9920892404973	0.240364608462229\\
19.9988067083893	0.240368135568787\\
20.0055240403961	0.240371659450636\\
20.0122412366465	0.240375180111762\\
20.0189582972693	0.240378697556147\\
20.0256752223929	0.240382211787764\\
20.0323920121458	0.240385722810583\\
20.0391086666562	0.240389230628565\\
20.0458251860522	0.240392735245667\\
20.0525415704616	0.240396236665838\\
20.0592578200121	0.240399734893022\\
20.0659739348314	0.240403229931156\\
20.0726899150468	0.240406721784172\\
20.0794057607855	0.240410210455995\\
20.0861214721746	0.240413695950543\\
20.0928370493411	0.24041717827173\\
20.0995524924116	0.240420657423462\\
20.1062678015127	0.240424133409641\\
20.1129829767708	0.24042760623416\\
20.1196980183123	0.240431075900909\\
20.1264129262631	0.240434542413769\\
20.1331277007492	0.240438005776618\\
20.1398423418965	0.240441465993325\\
20.1465568498303	0.240444923067755\\
20.1532712246763	0.240448377003767\\
20.1599854665597	0.240451827805212\\
20.1666995756057	0.240455275475938\\
20.1734135519391	0.240458720019784\\
20.1801273956847	0.240462161440585\\
20.1868411069673	0.240465599742169\\
20.1935546859113	0.240469034928359\\
20.2002681326411	0.240472467002972\\
20.2069814472807	0.240475895969818\\
20.2136946299542	0.240479321832702\\
20.2204076807854	0.240482744595422\\
20.227120599898	0.240486164261772\\
20.2338333874156	0.240489580835539\\
20.2405460434615	0.240492994320505\\
20.247258568159	0.240496404720443\\
20.253970961631	0.240499812039124\\
20.2606832240006	0.240503216280312\\
20.2673953553903	0.240506617447764\\
20.2741073559229	0.240510015545233\\
20.2808192257207	0.240513410576464\\
20.287530964906	0.240516802545199\\
20.294242573601	0.240520191455171\\
20.3009540519275	0.24052357731011\\
20.3076654000075	0.240526960113739\\
20.3143766179625	0.240530339869775\\
20.3210877059141	0.24053371658193\\
20.3277986639835	0.240537090253909\\
20.3345094922921	0.240540460889414\\
20.3412201909607	0.240543828492138\\
20.3479307601104	0.24054719306577\\
20.3546411998618	0.240550554613994\\
20.3613515103355	0.240553913140487\\
20.3680616916519	0.24055726864892\\
20.3747717439313	0.240560621142961\\
20.3814816672937	0.240563970626268\\
20.3881914618593	0.240567317102498\\
20.3949011277477	0.240570660575299\\
20.4016106650787	0.240574001048316\\
20.4083200739717	0.240577338525186\\
20.4150293545461	0.240580673009541\\
20.4217385069211	0.240584004505008\\
20.4284475312158	0.24058733301521\\
20.4351564275491	0.240590658543761\\
20.4418651960396	0.240593981094272\\
20.4485738368061	0.240597300670347\\
20.455282349967	0.240600617275586\\
20.4619907356405	0.240603930913582\\
20.4686989939449	0.240607241587924\\
20.4754071249981	0.240610549302193\\
20.482115128918	0.240613854059967\\
20.4888230058224	0.240617155864819\\
20.4955307558287	0.240620454720313\\
20.5022383790543	0.240623750630011\\
20.5089458756166	0.240627043597468\\
20.5156532456327	0.240630333626234\\
20.5223604892195	0.240633620719853\\
20.5290676064939	0.240636904881865\\
20.5357745975725	0.240640186115802\\
20.5424814625718	0.240643464425194\\
20.5491882016083	0.240646739813563\\
20.5558948147981	0.240650012284427\\
20.5626013022574	0.240653281841297\\
20.5693076641021	0.240656548487681\\
20.576013900448	0.240659812227079\\
20.5827200114108	0.240663073062988\\
20.5894259971059	0.2406663309989\\
20.5961318576488	0.240669586038298\\
20.6028375931545	0.240672838184663\\
20.6095432037383	0.240676087441471\\
20.6162486895151	0.24067933381219\\
20.6229540505996	0.240682577300285\\
20.6296592871064	0.240685817909215\\
20.6363643991501	0.240689055642434\\
20.6430693868451	0.24069229050339\\
20.6497742503055	0.240695522495527\\
20.6564789896454	0.240698751622282\\
20.6631836049787	0.240701977887089\\
20.6698880964192	0.240705201293374\\
20.6765924640806	0.240708421844561\\
20.6832967080764	0.240711639544066\\
20.6900008285199	0.240714854395303\\
20.6967048255243	0.240718066401677\\
20.7034086992028	0.24072127556659\\
20.7101124496681	0.240724481893439\\
20.7168160770333	0.240727685385616\\
20.7235195814108	0.240730886046506\\
20.7302229629132	0.240734083879491\\
20.7369262216529	0.240737278887947\\
20.7436293577422	0.240740471075244\\
20.750332371293	0.24074366044475\\
20.7570352624174	0.240746846999824\\
20.7637380312272	0.240750030743823\\
20.7704406778341	0.240753211680096\\
20.7771432023495	0.24075638981199\\
20.7838456048849	0.240759565142845\\
20.7905478855516	0.240762737675996\\
20.7972500444607	0.240765907414774\\
20.8039520817231	0.240769074362505\\
20.8106539974497	0.240772238522509\\
20.8173557917513	0.240775399898102\\
20.8240574647383	0.240778558492593\\
20.8307590165213	0.240781714309288\\
20.8374604472105	0.240784867351488\\
20.844161756916	0.240788017622489\\
20.850862945748	0.240791165125581\\
20.8575640138163	0.240794309864049\\
20.8642649612307	0.240797451841174\\
20.8709657881007	0.240800591060233\\
20.8776664945358	0.240803727524495\\
20.8843670806455	0.240806861237227\\
20.8910675465389	0.24080999220169\\
20.897767892325	0.24081312042114\\
20.9044681181129	0.240816245898828\\
20.9111682240114	0.240819368638001\\
20.917868210129	0.240822488641899\\
20.9245680765744	0.240825605913761\\
20.931267823456	0.240828720456817\\
20.937967450882	0.240831832274295\\
20.9446669589605	0.240834941369417\\
20.9513663477997	0.240838047745399\\
20.9580656175073	0.240841151405455\\
20.964764768191	0.240844252352793\\
20.9714637999586	0.240847350590614\\
20.9781627129174	0.240850446122118\\
20.9848615071749	0.240853538950497\\
20.9915601828381	0.240856629078941\\
20.9982587400143	0.240859716510632\\
21.0049571788103	0.24086280124875\\
21.011655499333	0.24086588329647\\
21.018353701689	0.240868962656961\\
21.025051785985	0.240872039333387\\
21.0317497523272	0.24087511332891\\
21.0384476008221	0.240878184646683\\
21.0451453315758	0.240881253289859\\
21.0518429446943	0.240884319261583\\
21.0585404402835	0.240887382564996\\
21.0652378184492	0.240890443203236\\
21.071935079297	0.240893501179433\\
21.0786322229325	0.240896556496716\\
21.085329249461	0.240899609158207\\
21.0920261589878	0.240902659167024\\
21.098722951618	0.240905706526281\\
21.1054196274566	0.240908751239087\\
21.1121161866086	0.240911793308544\\
21.1188126291785	0.240914832737754\\
21.1255089552711	0.240917869529812\\
21.1322051649909	0.240920903687806\\
21.1389012584421	0.240923935214824\\
21.145597235729	0.240926964113947\\
21.1522930969558	0.24092999038825\\
21.1589888422264	0.240933014040807\\
21.1656844716446	0.240936035074685\\
21.1723799853143	0.240939053492946\\
21.1790753833389	0.24094206929865\\
21.185770665822	0.24094508249485\\
21.1924658328668	0.240948093084596\\
21.1991608845767	0.240951101070932\\
21.2058558210548	0.240954106456899\\
21.2125506424039	0.240957109245533\\
21.2192453487269	0.240960109439865\\
21.2259399401266	0.240963107042923\\
21.2326344167056	0.240966102057728\\
21.2393287785662	0.240969094487299\\
21.246023025811	0.24097208433465\\
21.252717158542	0.240975071602789\\
21.2594111768614	0.240978056294721\\
21.2661050808712	0.240981038413446\\
21.2727988706732	0.240984017961961\\
21.2794925463691	0.240986994943256\\
21.2861861080606	0.240989969360319\\
21.2928795558491	0.240992941216133\\
21.299572889836	0.240995910513675\\
21.3062661101226	0.240998877255919\\
21.3129592168098	0.241001841445835\\
21.3196522099988	0.241004803086388\\
21.3263450897904	0.241007762180538\\
21.3330378562853	0.241010718731243\\
21.3397305095842	0.241013672741454\\
21.3464230497876	0.241016624214119\\
21.3531154769958	0.241019573152181\\
21.3598077913091	0.241022519558579\\
21.3664999928277	0.241025463436248\\
21.3731920816516	0.241028404788119\\
21.3798840578806	0.241031343617117\\
21.3865759216146	0.241034279926165\\
21.3932676729532	0.241037213718181\\
21.399959311996	0.241040144996076\\
21.4066508388423	0.241043073762761\\
21.4133422535915	0.24104600002114\\
21.4200335563427	0.241048923774114\\
21.4267247471951	0.241051845024579\\
21.4334158262475	0.241054763775427\\
21.4401067935988	0.241057680029546\\
21.4467976493476	0.24106059378982\\
21.4534883935926	0.241063505059127\\
21.4601790264323	0.241066413840344\\
21.4668695479649	0.24106932013634\\
21.4735599582887	0.241072223949984\\
21.4802502575018	0.241075125284136\\
21.4869404457023	0.241078024141657\\
21.4936305229879	0.2410809205254\\
21.5003204894565	0.241083814438215\\
21.5070103452057	0.241086705882949\\
21.513700090333	0.241089594862442\\
21.5203897249358	0.241092481379533\\
21.5270792491114	0.241095365437055\\
21.533768662957	0.241098247037838\\
21.5404579665697	0.241101126184707\\
21.5471471600464	0.241104002880483\\
21.5538362434839	0.241106877127982\\
21.5605252169789	0.241109748930019\\
21.5672140806281	0.241112618289402\\
21.5739028345279	0.241115485208936\\
21.5805914787747	0.241118349691421\\
21.5872800134647	0.241121211739654\\
21.5939684386941	0.241124071356428\\
21.6006567545588	0.241126928544531\\
21.6073449611549	0.241129783306748\\
21.614033058578	0.241132635645859\\
21.620721046924	0.241135485564641\\
21.6274089262882	0.241138333065866\\
21.6340966967662	0.241141178152302\\
21.6407843584534	0.241144020826715\\
21.6474719114449	0.241146861091863\\
21.6541593558359	0.241149698950504\\
21.6608466917213	0.24115253440539\\
21.667533919196	0.24115536745927\\
21.6742210383549	0.241158198114887\\
21.6809080492925	0.241161026374983\\
21.6875949521034	0.241163852242293\\
21.6942817468821	0.241166675719551\\
21.7009684337229	0.241169496809485\\
21.7076550127199	0.24117231551482\\
21.7143414839673	0.241175131838275\\
21.7210278475591	0.241177945782569\\
21.7277141035892	0.241180757350414\\
21.7344002521512	0.241183566544519\\
21.7410862933389	0.241186373367589\\
21.7477722272458	0.241189177822325\\
21.7544580539653	0.241191979911425\\
21.7611437735908	0.241194779637582\\
21.7678293862155	0.241197577003485\\
21.7745148919324	0.24120037201182\\
21.7812002908346	0.241203164665269\\
21.7878855830149	0.241205954966509\\
21.7945707685661	0.241208742918215\\
21.8012558475809	0.241211528523058\\
21.8079408201518	0.241214311783702\\
21.8146256863713	0.241217092702812\\
21.8213104463317	0.241219871283045\\
21.8279951001252	0.241222647527057\\
21.8346796478439	0.241225421437498\\
21.8413640895799	0.241228193017016\\
21.848048425425	0.241230962268254\\
21.8547326554711	0.241233729193853\\
21.8614167798098	0.241236493796448\\
21.8681007985327	0.24123925607867\\
21.8747847117312	0.241242016043149\\
21.8814685194967	0.24124477369251\\
21.8881522219206	0.241247529029372\\
21.8948358190938	0.241250282056353\\
21.9015193111075	0.241253032776066\\
21.9082026980525	0.241255781191122\\
21.9148859800198	0.241258527304125\\
21.9215691571	0.241261271117678\\
21.9282522293837	0.241264012634379\\
21.9349351969614	0.241266751856824\\
21.9416180599236	0.241269488787602\\
21.9483008183605	0.241272223429302\\
21.9549834723623	0.241274955784506\\
21.9616660220191	0.241277685855796\\
21.9683484674209	0.241280413645747\\
21.9750308086575	0.241283139156931\\
21.9817130458187	0.241285862391918\\
21.9883951789941	0.241288583353272\\
21.9950772082734	0.241291302043556\\
22.0017591337459	0.241294018465327\\
22.008440955501	0.24129673262114\\
22.015122673628	0.241299444513545\\
22.0218042882159	0.241302154145089\\
22.0284857993538	0.241304861518316\\
22.0351672071307	0.241307566635765\\
22.0418485116354	0.241310269499972\\
22.0485297129565	0.241312970113471\\
22.0552108111828	0.241315668478789\\
22.0618918064026	0.241318364598453\\
22.0685726987045	0.241321058474984\\
22.0752534881766	0.241323750110901\\
22.0819341749074	0.241326439508717\\
22.0886147589847	0.241329126670945\\
22.0952952404967	0.24133181160009\\
22.1019756195312	0.241334494298659\\
22.1086558961761	0.24133717476915\\
22.115336070519	0.24133985301406\\
22.1220161426474	0.241342529035884\\
22.128696112649	0.24134520283711\\
22.135375980611	0.241347874420225\\
22.1420557466208	0.241350543787712\\
22.1487354107655	0.241353210942049\\
22.1554149731323	0.241355875885714\\
22.1620944338081	0.241358538621176\\
22.1687737928797	0.241361199150907\\
22.1754530504341	0.24136385747737\\
22.1821322065578	0.241366513603027\\
22.1888112613374	0.241369167530337\\
22.1954902148594	0.241371819261754\\
22.2021690672102	0.24137446879973\\
22.208847818476	0.241377116146713\\
22.2155264687431	0.241379761305147\\
22.2222050180975	0.241382404277472\\
22.2288834666251	0.241385045066127\\
22.235561814412	0.241387683673545\\
22.2422400615438	0.241390320102158\\
22.2489182081062	0.241392954354392\\
22.2555962541848	0.241395586432671\\
22.2622741998651	0.241398216339415\\
22.2689520452325	0.241400844077042\\
22.2756297903723	0.241403469647965\\
22.2823074353696	0.241406093054595\\
22.2889849803095	0.241408714299337\\
22.295662425277	0.241411333384596\\
22.3023397703571	0.241413950312771\\
22.3090170156345	0.24141656508626\\
22.3156941611939	0.241419177707455\\
22.3223712071198	0.241421788178746\\
22.3290481534969	0.241424396502521\\
22.3357250004095	0.241427002681161\\
22.3424017479419	0.241429606717048\\
22.3490783961783	0.241432208612557\\
22.3557549452029	0.241434808370062\\
22.3624313950996	0.241437405991932\\
22.3691077459524	0.241440001480535\\
22.3757839978451	0.241442594838234\\
22.3824601508614	0.241445186067388\\
22.389136205085	0.241447775170354\\
22.3958121605993	0.241450362149485\\
22.4024880174879	0.241452947007132\\
22.4091637758341	0.241455529745641\\
22.415839435721	0.241458110367356\\
22.422514997232	0.241460688874617\\
22.4291904604499	0.241463265269761\\
22.4358658254578	0.241465839555121\\
22.4425410923386	0.241468411733028\\
22.449216261175	0.241470981805809\\
22.4558913320497	0.241473549775788\\
22.4625663050453	0.241476115645286\\
22.4692411802443	0.24147867941662\\
22.475915957729	0.241481241092104\\
22.4825906375818	0.241483800674049\\
22.4892652198848	0.241486358164763\\
22.4959397047202	0.241488913566551\\
22.5026140921701	0.241491466881713\\
22.5092883823162	0.241494018112549\\
22.5159625752404	0.241496567261352\\
22.5226366710246	0.241499114330415\\
22.5293106697503	0.241501659322026\\
22.535984571499	0.24150420223847\\
22.5426583763523	0.24150674308203\\
22.5493320843914	0.241509281854984\\
22.5560056956978	0.241511818559609\\
22.5626792103524	0.241514353198177\\
22.5693526284365	0.241516885772957\\
22.5760259500311	0.241519416286217\\
22.5826991752169	0.241521944740219\\
22.5893723040749	0.241524471137223\\
22.5960453366857	0.241526995479486\\
22.6027182731301	0.241529517769262\\
22.6093911134884	0.241532038008802\\
22.6160638578412	0.241534556200353\\
22.6227365062688	0.24153707234616\\
22.6294090588514	0.241539586448463\\
22.6360815156693	0.241542098509503\\
22.6427538768025	0.241544608531512\\
22.649426142331	0.241547116516724\\
22.6560983123347	0.241549622467368\\
22.6627703868933	0.241552126385669\\
22.6694423660867	0.24155462827385\\
22.6761142499945	0.24155712813413\\
22.6827860386961	0.241559625968728\\
22.689457732271	0.241562121779855\\
22.6961293307986	0.241564615569723\\
22.7028008343582	0.241567107340539\\
22.7094722430288	0.241569597094508\\
22.7161435568897	0.241572084833831\\
22.7228147760197	0.241574570560706\\
22.7294859004978	0.241577054277329\\
22.7361569304029	0.241579535985891\\
22.7428278658136	0.241582015688583\\
22.7494987068086	0.241584493387591\\
22.7561694534664	0.241586969085097\\
22.7628401058655	0.241589442783282\\
22.7695106640843	0.241591914484324\\
22.776181128201	0.241594384190396\\
22.7828514982939	0.24159685190367\\
22.7895217744411	0.241599317626314\\
22.7961919567205	0.241601781360493\\
22.8028620452101	0.24160424310837\\
22.8095320399878	0.241606702872104\\
22.8162019411313	0.241609160653851\\
22.8228717487183	0.241611616455765\\
22.8295414628264	0.241614070279996\\
22.836211083533	0.241616522128692\\
22.8428806109155	0.241618972003998\\
22.8495500450514	0.241621419908054\\
22.8562193860178	0.241623865843\\
22.8628886338918	0.241626309810971\\
22.8695577887506	0.241628751814101\\
22.8762268506711	0.241631191854518\\
22.8828958197301	0.241633629934351\\
22.8895646960046	0.241636066055723\\
22.8962334795712	0.241638500220755\\
22.9029021705064	0.241640932431566\\
22.909570768887	0.24164336269027\\
22.9162392747893	0.241645790998981\\
22.9229076882897	0.241648217359808\\
22.9295760094645	0.241650641774858\\
22.9362442383898	0.241653064246234\\
22.9429123751419	0.241655484776038\\
22.9495804197967	0.241657903366367\\
22.9562483724302	0.241660320019317\\
22.9629162331182	0.241662734736979\\
22.9695840019365	0.241665147521445\\
22.9762516789608	0.241667558374799\\
22.9829192642667	0.241669967299127\\
22.9895867579298	0.241672374296508\\
22.9962541600254	0.241674779369022\\
23.0029214706289	0.241677182518743\\
23.0095886898156	0.241679583747744\\
23.0162558176607	0.241681983058094\\
23.0229228542393	0.24168438045186\\
23.0295897996263	0.241686775931107\\
23.0362566538968	0.241689169497895\\
23.0429234171255	0.241691561154283\\
23.0495900893872	0.241693950902326\\
23.0562566707567	0.241696338744078\\
23.0629231613085	0.241698724681588\\
23.0695895611171	0.241701108716903\\
23.076255870257	0.241703490852069\\
23.0829220888025	0.241705871089126\\
23.0895882168279	0.241708249430113\\
23.0962542544073	0.241710625877067\\
23.1029202016149	0.241713000432021\\
23.1095860585248	0.241715373097006\\
23.1162518252107	0.241717743874049\\
23.1229175017466	0.241720112765176\\
23.1295830882063	0.241722479772409\\
23.1362485846635	0.241724844897767\\
23.1429139911917	0.241727208143267\\
23.1495793078645	0.241729569510925\\
23.1562445347554	0.24173192900275\\
23.1629096719376	0.241734286620752\\
23.1695747194846	0.241736642366937\\
23.1762396774694	0.241738996243307\\
23.1829045459653	0.241741348251865\\
23.1895693250452	0.241743698394606\\
23.1962340147821	0.241746046673528\\
23.202898615249	0.241748393090621\\
23.2095631265185	0.241750737647877\\
23.2162275486635	0.241753080347281\\
23.2228918817565	0.241755421190818\\
23.2295561258701	0.241757760180471\\
23.2362202810768	0.241760097318218\\
23.242884347449	0.241762432606035\\
23.249548325059	0.241764766045897\\
23.2562122139791	0.241767097639774\\
23.2628760142814	0.241769427389634\\
23.2695397260379	0.241771755297444\\
23.2762033493207	0.241774081365167\\
23.2828668842018	0.241776405594763\\
23.2895303307528	0.24177872798819\\
23.2961936890457	0.241781048547403\\
23.3028569591521	0.241783367274354\\
23.3095201411435	0.241785684170995\\
23.3161832350916	0.241787999239271\\
23.3228462410677	0.241790312481127\\
23.3295091591432	0.241792623898507\\
23.3361719893895	0.241794933493349\\
23.3428347318777	0.241797241267589\\
23.3494973866789	0.241799547223163\\
23.3561599538643	0.241801851362003\\
23.3628224335047	0.241804153686036\\
23.3694848256711	0.24180645419719\\
23.3761471304343	0.241808752897388\\
23.3828093478651	0.241811049788553\\
23.3894714780341	0.241813344872602\\
23.3961335210118	0.241815638151451\\
23.4027954768689	0.241817929627016\\
23.4094573456757	0.241820219301205\\
23.4161191275026	0.241822507175929\\
23.4227808224199	0.241824793253093\\
23.4294424304978	0.241827077534601\\
23.4361039518063	0.241829360022352\\
23.4427653864156	0.241831640718246\\
23.4494267343956	0.241833919624179\\
23.4560879958163	0.241836196742043\\
23.4627491707473	0.241838472073729\\
23.4694102592585	0.241840745621126\\
23.4760712614195	0.241843017386119\\
23.4827321773	0.241845287370591\\
23.4893930069694	0.241847555576424\\
23.4960537504972	0.241849822005494\\
23.5027144079527	0.241852086659678\\
23.5093749794052	0.241854349540849\\
23.5160354649239	0.241856610650878\\
23.522695864578	0.241858869991632\\
23.5293561784366	0.241861127564977\\
23.5360164065685	0.241863383372777\\
23.5426765490427	0.241865637416891\\
23.5493366059281	0.241867889699179\\
23.5559965772934	0.241870140221497\\
23.5626564632073	0.241872388985696\\
23.5693162637384	0.241874635993629\\
23.5759759789552	0.241876881247143\\
23.5826356089262	0.241879124748084\\
23.5892951537197	0.241881366498296\\
23.5959546134042	0.24188360649962\\
23.6026139880478	0.241885844753894\\
23.6092732777186	0.241888081262954\\
23.6159324824848	0.241890316028635\\
23.6225916024144	0.241892549052766\\
23.6292506375754	0.241894780337178\\
23.6359095880355	0.241897009883696\\
23.6425684538626	0.241899237694145\\
23.6492272351244	0.241901463770345\\
23.6558859318886	0.241903688114116\\
23.6625445442227	0.241905910727275\\
23.6692030721942	0.241908131611637\\
23.6758615158706	0.241910350769012\\
23.6825198753191	0.24191256820121\\
23.6891781506072	0.241914783910039\\
23.6958363418019	0.241916997897303\\
23.7024944489704	0.241919210164805\\
23.7091524721799	0.241921420714344\\
23.7158104114972	0.241923629547718\\
23.7224682669893	0.241925836666723\\
23.729126038723	0.241928042073151\\
23.7357837267651	0.241930245768792\\
23.7424413311824	0.241932447755435\\
23.7490988520414	0.241934648034865\\
23.7557562894087	0.241936846608866\\
23.7624136433507	0.241939043479219\\
23.769070913934	0.241941238647702\\
23.7757281012247	0.241943432116093\\
23.7823852052893	0.241945623886163\\
23.7890422261939	0.241947813959687\\
23.7956991640046	0.241950002338432\\
23.8023560187874	0.241952189024165\\
23.8090127906085	0.241954374018652\\
23.8156694795336	0.241956557323655\\
23.8223260856286	0.241958738940934\\
23.8289826089594	0.241960918872246\\
23.8356390495915	0.241963097119346\\
23.8422954075906	0.241965273683989\\
23.8489516830223	0.241967448567924\\
23.8556078759521	0.241969621772901\\
23.8622639864454	0.241971793300665\\
23.8689200145675	0.24197396315296\\
23.8755759603837	0.241976131331528\\
23.8822318239592	0.241978297838108\\
23.8888876053592	0.241980462674438\\
23.8955433046488	0.241982625842251\\
23.9021989218928	0.241984787343281\\
23.9088544571563	0.241986947179257\\
23.9155099105041	0.241989105351908\\
23.922165282001	0.241991261862959\\
23.9288205717117	0.241993416714134\\
23.9354757797009	0.241995569907153\\
23.9421309060332	0.241997721443737\\
23.948785950773	0.2419998713256\\
23.9554409139848	0.242002019554459\\
23.962095795733	0.242004166132024\\
23.9687505960819	0.242006311060007\\
23.9754053150957	0.242008454340114\\
23.9820599528386	0.242010595974051\\
23.9887145093746	0.242012735963521\\
23.9953689847679	0.242014874310225\\
24.0020233790824	0.242017011015863\\
24.008677692382	0.24201914608213\\
24.0153319247304	0.242021279510721\\
24.0219860761915	0.242023411303328\\
24.028640146829	0.242025541461641\\
24.0352941367064	0.242027669987348\\
24.0419480458873	0.242029796882135\\
24.0486018744353	0.242031922147683\\
24.0552556224136	0.242034045785676\\
24.0619092898858	0.242036167797792\\
24.068562876915	0.242038288185706\\
24.0752163835645	0.242040406951095\\
24.0818698098974	0.242042524095631\\
24.0885231559768	0.242044639620983\\
24.0951764218657	0.24204675352882\\
24.1018296076271	0.242048865820807\\
24.1084827133238	0.242050976498609\\
24.1151357390187	0.242053085563887\\
24.1217886847745	0.242055193018299\\
24.1284415506539	0.242057298863505\\
24.1350943367195	0.242059403101158\\
24.1417470430338	0.242061505732911\\
24.1483996696593	0.242063606760416\\
24.1550522166585	0.242065706185321\\
24.1617046840937	0.242067804009273\\
24.1683570720271	0.242069900233915\\
24.1750093805211	0.24207199486089\\
24.1816616096376	0.242074087891838\\
24.1883137594389	0.242076179328398\\
24.1949658299869	0.242078269172205\\
24.2016178213436	0.242080357424892\\
24.2082697335709	0.242082444088092\\
24.2149215667306	0.242084529163434\\
24.2215733208844	0.242086612652546\\
24.2282249960941	0.242088694557052\\
24.2348765924212	0.242090774878576\\
24.2415281099274	0.242092853618739\\
24.2481795486741	0.242094930779159\\
24.2548309087227	0.242097006361455\\
24.2614821901347	0.242099080367241\\
24.2681333929712	0.242101152798129\\
24.2747845172937	0.24210322365573\\
24.2814355631631	0.242105292941652\\
24.2880865306407	0.242107360657503\\
24.2947374197875	0.242109426804887\\
24.3013882306644	0.242111491385406\\
24.3080389633323	0.24211355440066\\
24.3146896178522	0.242115615852247\\
24.3213401942848	0.242117675741764\\
24.3279906926908	0.242119734070805\\
24.3346411131309	0.242121790840962\\
24.3412914556656	0.242123846053825\\
24.3479417203555	0.242125899710982\\
24.3545919072611	0.242127951814018\\
24.3612420164428	0.242130002364518\\
24.3678920479609	0.242132051364063\\
24.3745420018756	0.242134098814234\\
24.3811918782472	0.242136144716608\\
24.3878416771359	0.242138189072761\\
24.3944913986017	0.242140231884266\\
24.4011410427046	0.242142273152696\\
24.4077906095047	0.242144312879619\\
24.4144400990617	0.242146351066604\\
24.4210895114356	0.242148387715216\\
24.4277388466861	0.24215042282702\\
24.4343881048729	0.242152456403575\\
24.4410372860556	0.242154488446443\\
24.4476863902939	0.242156518957181\\
24.4543354176473	0.242158547937344\\
24.4609843681751	0.242160575388487\\
24.4676332419369	0.24216260131216\\
24.474282038992	0.242164625709913\\
24.4809307593995	0.242166648583295\\
24.4875794032188	0.24216866993385\\
24.4942279705089	0.242170689763123\\
24.500876461329	0.242172708072656\\
24.5075248757381	0.242174724863987\\
24.5141732137952	0.242176740138655\\
24.5208214755591	0.242178753898197\\
24.5274696610887	0.242180766144144\\
24.5341177704427	0.24218277687803\\
24.5407658036799	0.242184786101385\\
24.547413760859	0.242186793815736\\
24.5540616420384	0.24218880002261\\
24.5607094472769	0.242190804723531\\
24.5673571766328	0.24219280792002\\
24.5740048301645	0.242194809613599\\
24.5806524079304	0.242196809805785\\
24.5872999099888	0.242198808498094\\
24.5939473363979	0.242200805692042\\
24.6005946872159	0.24220280138914\\
24.6072419625008	0.2422047955909\\
24.6138891623108	0.242206788298829\\
24.6205362867038	0.242208779514435\\
24.6271833357378	0.242210769239223\\
24.6338303094705	0.242212757474694\\
24.6404772079599	0.242214744222351\\
24.6471240312636	0.242216729483692\\
24.6537707794394	0.242218713260215\\
24.6604174525448	0.242220695553414\\
24.6670640506375	0.242222676364783\\
24.6737105737749	0.242224655695814\\
24.6803570220145	0.242226633547996\\
24.6870033954136	0.242228609922816\\
24.6936496940297	0.242230584821761\\
24.7002959179199	0.242232558246315\\
24.7069420671415	0.242234530197959\\
24.7135881417517	0.242236500678173\\
24.7202341418074	0.242238469688437\\
24.7268800673658	0.242240437230225\\
24.7335259184838	0.242242403305013\\
24.7401716952183	0.242244367914274\\
24.7468173976262	0.242246331059478\\
24.7534630257643	0.242248292742094\\
24.7601085796893	0.242250252963589\\
24.7667540594579	0.242252211725428\\
24.7733994651266	0.242254169029075\\
24.7800447967522	0.242256124875991\\
24.786690054391	0.242258079267636\\
24.7933352380995	0.242260032205468\\
24.7999803479341	0.242261983690942\\
24.8066253839512	0.242263933725513\\
24.8132703462069	0.242265882310632\\
24.8199152347576	0.242267829447752\\
24.8265600496593	0.242269775138319\\
24.8332047909681	0.242271719383781\\
24.8398494587401	0.242273662185583\\
24.8464940530313	0.242275603545167\\
24.8531385738976	0.242277543463976\\
24.8597830213948	0.242279481943449\\
24.8664273955788	0.242281418985023\\
24.8730716965053	0.242283354590135\\
24.87971592423	0.242285288760217\\
24.8863600788085	0.242287221496704\\
24.8930041602965	0.242289152801024\\
24.8996481687493	0.242291082674607\\
24.9062921042226	0.24229301111888\\
24.9129359667716	0.242294938135268\\
24.9195797564518	0.242296863725193\\
24.9262234733184	0.242298787890077\\
24.9328671174267	0.242300710631341\\
24.9395106888318	0.242302631950402\\
24.946154187589	0.242304551848676\\
24.9527976137531	0.242306470327577\\
24.9594409673793	0.242308387388518\\
24.9660842485225	0.24231030303291\\
24.9727274572376	0.242312217262162\\
24.9793705935794	0.242314130077681\\
24.9860136576027	0.242316041480872\\
24.9926566493623	0.24231795147314\\
24.9992995689128	0.242319860055886\\
25.0059424163088	0.24232176723051\\
25.0125851916049	0.242323672998412\\
25.0192278948556	0.242325577360987\\
25.0258705261154	0.24232748031963\\
25.0325130854385	0.242329381875736\\
25.0391555728795	0.242331282030694\\
25.0457979884925	0.242333180785896\\
25.0524403323317	0.242335078142728\\
25.0590826044514	0.242336974102578\\
25.0657248049057	0.242338868666829\\
25.0723669337486	0.242340761836865\\
25.0790089910341	0.242342653614066\\
25.0856509768162	0.242344543999812\\
25.0922928911487	0.242346432995481\\
25.0989347340856	0.242348320602448\\
25.1055765056805	0.242350206822087\\
25.1122182059872	0.242352091655771\\
25.1188598350595	0.242353975104871\\
25.1255013929509	0.242355857170756\\
25.1321428797149	0.242357737854793\\
25.1387842954051	0.242359617158347\\
25.145425640075	0.242361495082783\\
25.1520669137779	0.242363371629462\\
25.1587081165672	0.242365246799746\\
25.1653492484962	0.242367120594992\\
25.1719903096182	0.242368993016559\\
25.1786312999862	0.242370864065802\\
25.1852722196535	0.242372733744074\\
25.1919130686731	0.242374602052727\\
25.198553847098	0.242376468993112\\
25.2051945549813	0.242378334566578\\
25.2118351923758	0.242380198774471\\
25.2184757593343	0.242382061618137\\
25.2251162559098	0.242383923098919\\
25.2317566821549	0.24238578321816\\
25.2383970381224	0.2423876419772\\
25.2450373238649	0.242389499377378\\
25.2516775394349	0.24239135542003\\
25.2583176848851	0.242393210106492\\
25.2649577602679	0.242395063438097\\
25.2715977656358	0.242396915416179\\
25.2782377010411	0.242398766042066\\
25.2848775665362	0.242400615317088\\
25.2915173621733	0.242402463242572\\
25.2981570880046	0.242404309819844\\
25.3047967440824	0.242406155050226\\
25.3114363304588	0.242407998935042\\
25.3180758471858	0.242409841475612\\
25.3247152943154	0.242411682673254\\
25.3313546718995	0.242413522529287\\
25.3379939799902	0.242415361045025\\
25.3446332186392	0.242417198221783\\
25.3512723878984	0.242419034060872\\
25.3579114878194	0.242420868563605\\
25.3645505184541	0.24242270173129\\
25.371189479854	0.242424533565234\\
25.3778283720707	0.242426364066743\\
25.3844671951558	0.242428193237122\\
25.3911059491607	0.242430021077673\\
25.3977446341369	0.242431847589698\\
25.4043832501359	0.242433672774495\\
25.4110217972088	0.242435496633364\\
25.4176602754071	0.242437319167599\\
25.4242986847818	0.242439140378496\\
25.4309370253843	0.242440960267348\\
25.4375752972656	0.242442778835446\\
25.4442135004769	0.24244459608408\\
25.450851635069	0.242446412014538\\
25.4574897010931	0.242448226628107\\
25.4641276986	0.242450039926073\\
25.4707656276406	0.242451851909719\\
25.4774034882658	0.242453662580326\\
25.4840412805262	0.242455471939175\\
25.4906790044726	0.242457279987546\\
25.4973166601557	0.242459086726714\\
25.5039542476261	0.242460892157957\\
25.5105917669343	0.242462696282547\\
25.517229218131	0.242464499101758\\
25.5238666012664	0.24246630061686\\
25.5305039163911	0.242468100829123\\
25.5371411635554	0.242469899739815\\
25.5437783428097	0.242471697350202\\
25.5504154542041	0.242473493661548\\
25.5570524977889	0.242475288675118\\
25.5636894736143	0.242477082392172\\
25.5703263817303	0.242478874813971\\
25.5769632221871	0.242480665941773\\
25.5835999950346	0.242482455776836\\
25.5902367003228	0.242484244320414\\
25.5968733381016	0.242486031573762\\
25.6035099084209	0.242487817538132\\
25.6101464113305	0.242489602214775\\
25.6167828468802	0.24249138560494\\
25.6234192151196	0.242493167709875\\
25.6300555160984	0.242494948530827\\
25.6366917498662	0.242496728069039\\
25.6433279164727	0.242498506325756\\
25.6499640159672	0.242500283302219\\
25.6566000483994	0.242502058999668\\
25.6632360138185	0.242503833419343\\
25.669871912274	0.242505606562479\\
25.6765077438151	0.242507378430313\\
25.6831435084912	0.242509149024079\\
25.6897792063515	0.242510918345009\\
25.696414837445	0.242512686394335\\
25.7030504018211	0.242514453173286\\
25.7096858995286	0.242516218683091\\
25.7163213306167	0.242517982924976\\
25.7229566951344	0.242519745900166\\
25.7295919931304	0.242521507609885\\
25.7362272246538	0.242523268055355\\
25.7428623897533	0.242525027237797\\
25.7494974884778	0.242526785158429\\
25.7561325208759	0.24252854181847\\
25.7627674869964	0.242530297219136\\
25.7694023868878	0.242532051361642\\
25.7760372205988	0.2425338042472\\
25.7826719881779	0.242535555877023\\
25.7893066896736	0.242537306252321\\
25.7959413251343	0.242539055374302\\
25.8025758946085	0.242540803244175\\
25.8092103981444	0.242542549863144\\
25.8158448357904	0.242544295232416\\
25.8224792075947	0.242546039353191\\
25.8291135136056	0.242547782226673\\
25.8357477538711	0.24254952385406\\
25.8423819284394	0.242551264236553\\
25.8490160373586	0.242553003375347\\
25.8556500806766	0.242554741271638\\
25.8622840584414	0.242556477926621\\
25.868917970701	0.242558213341489\\
25.8755518175031	0.242559947517433\\
25.8821855988957	0.242561680455643\\
25.8888193149266	0.242563412157306\\
25.8954529656433	0.242565142623611\\
25.9020865510937	0.242566871855744\\
25.9087200713252	0.242568599854887\\
25.9153535263857	0.242570326622224\\
25.9219869163225	0.242572052158936\\
25.9286202411832	0.242573776466203\\
25.9352535010151	0.242575499545203\\
25.9418866958658	0.242577221397114\\
25.9485198257826	0.242578942023111\\
25.9551528908127	0.242580661424368\\
25.9617858910035	0.242582379602058\\
25.9684188264021	0.242584096557353\\
25.9750516970557	0.242585812291422\\
25.9816845030114	0.242587526805433\\
25.9883172443163	0.242589240100555\\
25.9949499210175	0.242590952177952\\
26.0015825331619	0.242592663038788\\
26.0082150807964	0.242594372684228\\
26.0148475639679	0.242596081115432\\
26.0214799827233	0.24259778833356\\
26.0281123371094	0.242599494339771\\
26.0347446271729	0.242601199135223\\
26.0413768529606	0.242602902721071\\
26.048009014519	0.24260460509847\\
26.0546411118948	0.242606306268572\\
26.0612731451346	0.24260800623253\\
26.0679051142848	0.242609704991494\\
26.074537019392	0.242611402546613\\
26.0811688605027	0.242613098899035\\
26.087800637663	0.242614794049905\\
26.0944323509195	0.242616488000369\\
26.1010640003185	0.242618180751569\\
26.107695585906	0.242619872304649\\
26.1143271077285	0.242621562660749\\
26.1209585658319	0.242623251821008\\
26.1275899602625	0.242624939786564\\
26.1342212910663	0.242626626558553\\
26.1408525582893	0.242628312138112\\
26.1474837619775	0.242629996526373\\
26.1541149021768	0.24263167972447\\
26.1607459789332	0.242633361733533\\
26.1673769922924	0.242635042554693\\
26.1740079423003	0.242636722189077\\
26.1806388290026	0.242638400637814\\
26.187269652445	0.242640077902028\\
26.1939004126732	0.242641753982844\\
26.2005311097328	0.242643428881385\\
26.2071617436694	0.242645102598773\\
26.2137923145285	0.242646775136128\\
26.2204228223556	0.24264844649457\\
26.2270532671962	0.242650116675215\\
26.2336836490956	0.242651785679181\\
26.2403139680992	0.242653453507583\\
26.2469442242523	0.242655120161533\\
26.2535744176001	0.242656785642144\\
26.260204548188	0.242658449950529\\
26.266834616061	0.242660113087795\\
26.2734646212643	0.242661775055052\\
26.280094563843	0.242663435853406\\
26.2867244438421	0.242665095483964\\
26.2933542613067	0.24266675394783\\
26.2999840162816	0.242668411246107\\
26.3066137088119	0.242670067379896\\
26.3132433389423	0.242671722350298\\
26.3198729067178	0.242673376158413\\
26.326502412183	0.242675028805338\\
26.3331318553828	0.24267668029217\\
26.3397612363617	0.242678330620004\\
26.3463905551646	0.242679979789934\\
26.353019811836	0.242681627803052\\
26.3596490064204	0.242683274660451\\
26.3662781389623	0.242684920363219\\
26.3729072095064	0.242686564912447\\
26.3795362180969	0.24268820830922\\
26.3861651647783	0.242689850554626\\
26.392794049595	0.242691491649749\\
26.3994228725912	0.242693131595673\\
26.4060516338112	0.24269477039348\\
26.4126803332992	0.242696408044251\\
26.4193089710995	0.242698044549066\\
26.4259375472561	0.242699679909003\\
26.4325660618131	0.24270131412514\\
26.4391945148147	0.242702947198552\\
26.4458229063047	0.242704579130315\\
26.4524512363273	0.2427062099215\\
26.4590795049262	0.242707839573181\\
26.4657077121454	0.242709468086428\\
26.4723358580287	0.242711095462311\\
26.4789639426199	0.242712721701898\\
26.4855919659628	0.242714346806256\\
26.4922199281011	0.242715970776451\\
26.4988478290784	0.242717593613547\\
26.5054756689384	0.242719215318608\\
26.5121034477247	0.242720835892694\\
26.5187311654807	0.242722455336868\\
26.5253588222501	0.242724073652189\\
26.5319864180763	0.242725690839714\\
26.5386139530026	0.242727306900501\\
26.5452414270725	0.242728921835605\\
26.5518688403293	0.242730535646082\\
26.5584961928163	0.242732148332983\\
26.5651234845768	0.242733759897361\\
26.5717507156539	0.242735370340267\\
26.5783778860908	0.242736979662751\\
26.5850049959306	0.242738587865859\\
26.5916320452165	0.24274019495064\\
26.5982590339915	0.24274180091814\\
26.6048859622986	0.242743405769402\\
26.6115128301807	0.242745009505471\\
26.6181396376807	0.242746612127387\\
26.6247663848416	0.242748213636193\\
26.6313930717062	0.242749814032927\\
26.6380196983172	0.242751413318628\\
26.6446462647175	0.242753011494334\\
26.6512727709497	0.24275460856108\\
26.6578992170566	0.2427562045199\\
26.6645256030807	0.24275779937183\\
26.6711519290646	0.2427593931179\\
26.6777781950509	0.242760985759142\\
26.6844044010822	0.242762577296585\\
26.6910305472008	0.242764167731259\\
26.6976566334492	0.242765757064192\\
26.7042826598698	0.242767345296408\\
26.7109086265049	0.242768932428933\\
26.7175345333969	0.242770518462792\\
26.724160380588	0.242772103399006\\
26.7307861681204	0.242773687238598\\
26.7374118960363	0.242775269982587\\
26.744037564378	0.242776851631993\\
26.7506631731874	0.242778432187832\\
26.7572887225066	0.242780011651123\\
26.7639142123777	0.242781590022881\\
26.7705396428426	0.242783167304119\\
26.7771650139434	0.242784743495851\\
26.7837903257218	0.242786318599089\\
26.7904155782199	0.242787892614844\\
26.7970407714793	0.242789465544125\\
26.8036659055418	0.24279103738794\\
26.8102909804493	0.242792608147297\\
26.8169159962435	0.242794177823201\\
26.8235409529659	0.242795746416658\\
26.8301658506582	0.242797313928671\\
26.8367906893621	0.242798880360243\\
26.8434154691189	0.242800445712375\\
26.8500401899704	0.242802009986067\\
26.8566648519579	0.242803573182318\\
26.8632894551228	0.242805135302125\\
26.8699139995066	0.242806696346487\\
26.8765384851506	0.242808256316397\\
26.883162912096	0.242809815212851\\
26.8897872803843	0.242811373036841\\
26.8964115900566	0.24281292978936\\
26.903035841154	0.242814485471398\\
26.9096600337178	0.242816040083945\\
26.9162841677891	0.24281759362799\\
26.9229082434089	0.24281914610452\\
26.9295322606184	0.242820697514521\\
26.9361562194583	0.242822247858978\\
26.9427801199699	0.242823797138876\\
26.9494039621938	0.242825345355197\\
26.9560277461712	0.242826892508922\\
26.9626514719427	0.242828438601033\\
26.9692751395492	0.242829983632508\\
26.9758987490314	0.242831527604326\\
26.9825223004301	0.242833070517464\\
26.989145793786	0.242834612372898\\
26.9957692291397	0.242836153171603\\
27.0023926065318	0.242837692914552\\
27.0090159260029	0.242839231602718\\
27.0156391875936	0.242840769237072\\
27.0222623913442	0.242842305818585\\
27.0288855372954	0.242843841348226\\
27.0355086254875	0.242845375826964\\
27.0421316559608	0.242846909255764\\
27.0487546287558	0.242848441635592\\
27.0553775439127	0.242849972967415\\
27.0620004014719	0.242851503252194\\
27.0686232014734	0.242853032490892\\
27.0752459439576	0.242854560684472\\
27.0818686289645	0.242856087833892\\
27.0884912565344	0.242857613940113\\
27.0951138267072	0.242859139004092\\
27.101736339523	0.242860663026786\\
27.1083587950218	0.242862186009151\\
27.1149811932436	0.242863707952141\\
27.1216035342283	0.242865228856711\\
27.1282258180158	0.242866748723812\\
27.1348480446459	0.242868267554397\\
27.1414702141585	0.242869785349414\\
27.1480923265933	0.242871302109814\\
27.1547143819901	0.242872817836545\\
27.1613363803886	0.242874332530553\\
27.1679583218284	0.242875846192785\\
27.1745802063491	0.242877358824185\\
27.1812020339904	0.242878870425697\\
27.1878238047918	0.242880380998264\\
27.1944455187929	0.242881890542827\\
27.201067176033	0.242883399060327\\
27.2076887765517	0.242884906551702\\
27.2143103203883	0.242886413017892\\
27.2209318075823	0.242887918459834\\
27.2275532381728	0.242889422878463\\
27.2341746121994	0.242890926274715\\
27.2407959297011	0.242892428649523\\
27.2474171907172	0.242893930003821\\
27.2540383952869	0.242895430338541\\
27.2606595434494	0.242896929654613\\
27.2672806352437	0.242898427952966\\
27.2739016707089	0.24289992523453\\
27.2805226498842	0.242901421500232\\
27.2871435728083	0.242902916750999\\
27.2937644395204	0.242904410987756\\
27.3003852500594	0.242905904211426\\
27.3070060044641	0.242907396422935\\
27.3136267027734	0.242908887623203\\
27.3202473450261	0.242910377813152\\
27.326867931261	0.242911866993702\\
27.3334884615169	0.242913355165773\\
27.3401089358324	0.242914842330282\\
27.3467293542462	0.242916328488145\\
27.353349716797	0.24291781364028\\
27.3599700235234	0.242919297787601\\
27.3665902744639	0.242920780931021\\
27.373210469657	0.242922263071454\\
27.3798306091413	0.242923744209811\\
27.3864506929553	0.242925224347003\\
27.3930707211372	0.242926703483939\\
27.3996906937256	0.242928181621528\\
27.4063106107587	0.242929658760678\\
27.4129304722749	0.242931134902295\\
27.4195502783125	0.242932610047284\\
27.4261700289096	0.242934084196551\\
27.4327897241046	0.242935557350997\\
27.4394093639355	0.242937029511527\\
27.4460289484406	0.24293850067904\\
27.4526484776579	0.242939970854438\\
27.4592679516254	0.24294144003862\\
27.4658873703813	0.242942908232483\\
27.4725067339635	0.242944375436926\\
27.4791260424099	0.242945841652844\\
27.4857452957585	0.242947306881133\\
27.4923644940473	0.242948771122686\\
27.4989836373139	0.242950234378397\\
27.5056027255963	0.242951696649158\\
27.5122217589322	0.242953157935859\\
27.5188407373594	0.242954618239392\\
27.5254596609156	0.242956077560646\\
27.5320785296384	0.242957535900507\\
27.5386973435656	0.242958993259864\\
27.5453161027347	0.242960449639602\\
27.5519348071832	0.242961905040606\\
27.5585534569489	0.242963359463761\\
27.565172052069	0.242964812909949\\
27.5717905925812	0.242966265380052\\
27.5784090785228	0.242967716874952\\
27.5850275099312	0.242969167395528\\
27.5916458868439	0.24297061694266\\
27.5982642092981	0.242972065517225\\
27.6048824773311	0.2429735131201\\
27.6115006909802	0.242974959752162\\
27.6181188502826	0.242976405414286\\
27.6247369552756	0.242977850107345\\
27.6313550059962	0.242979293832212\\
27.6379730024816	0.24298073658976\\
27.6445909447689	0.242982178380859\\
27.6512088328952	0.24298361920638\\
27.6578266668974	0.242985059067192\\
27.6644444468125	0.242986497964163\\
27.6710621726776	0.242987935898159\\
27.6776798445295	0.242989372870048\\
27.6842974624051	0.242990808880694\\
27.6909150263412	0.242992243930961\\
27.6975325363747	0.242993678021712\\
27.7041499925424	0.242995111153811\\
27.710767394881	0.242996543328117\\
27.7173847434272	0.242997974545492\\
27.7240020382177	0.242999404806794\\
27.7306192792892	0.243000834112882\\
27.7372364666782	0.243002262464613\\
27.7438536004214	0.243003689862843\\
27.7504706805552	0.243005116308429\\
27.7570877071163	0.243006541802224\\
27.763704680141	0.243007966345082\\
27.7703215996659	0.243009389937855\\
27.7769384657273	0.243010812581395\\
27.7835552783617	0.243012234276552\\
27.7901720376053	0.243013655024177\\
27.7967887434945	0.243015074825117\\
27.8034053960655	0.243016493680221\\
27.8100219953546	0.243017911590335\\
27.8166385413981	0.243019328556305\\
27.8232550342321	0.243020744578976\\
27.8298714738926	0.243022159659191\\
27.836487860416	0.243023573797794\\
27.8431041938382	0.243024986995627\\
27.8497204741952	0.243026399253531\\
27.8563367015232	0.243027810572345\\
27.862952875858	0.24302922095291\\
27.8695689972356	0.243030630396062\\
27.876185065692	0.24303203890264\\
27.882801081263	0.24303344647348\\
27.8894170439844	0.243034853109417\\
27.8960329538921	0.243036258811285\\
27.9026488110219	0.243037663579918\\
27.9092646154095	0.243039067416149\\
27.9158803670907	0.243040470320809\\
27.922496066101	0.243041872294729\\
27.9291117124762	0.243043273338739\\
27.935727306252	0.243044673453667\\
27.9423428474638	0.243046072640342\\
27.9489583361472	0.24304747089959\\
27.9555737723378	0.243048868232237\\
27.962189156071	0.243050264639109\\
27.9688044873823	0.24305166012103\\
27.9754197663071	0.243053054678822\\
27.9820349928809	0.243054448313309\\
27.988650167139	0.243055841025312\\
27.9952652891166	0.243057232815651\\
28.0018803588492	0.243058623685146\\
28.0084953763719	0.243060013634615\\
28.0151103417201	0.243061402664876\\
28.0217252549288	0.243062790776747\\
28.0283401160333	0.243064177971042\\
28.0349549250687	0.243065564248578\\
28.0415696820702	0.243066949610167\\
28.0481843870727	0.243068334056624\\
28.0547990401113	0.24306971758876\\
28.061413641221	0.243071100207388\\
28.0680281904369	0.243072481913316\\
28.0746426877938	0.243073862707355\\
28.0812571333267	0.243075242590314\\
28.0878715270704	0.243076621563\\
28.0944858690598	0.243077999626221\\
28.1011001593298	0.243079376780781\\
28.107714397915	0.243080753027486\\
28.1143285848502	0.243082128367141\\
28.1209427201703	0.243083502800548\\
28.1275568039098	0.24308487632851\\
28.1341708361034	0.243086248951828\\
28.1407848167858	0.243087620671303\\
28.1473987459916	0.243088991487735\\
28.1540126237552	0.243090361401922\\
28.1606264501114	0.243091730414662\\
28.1672402250944	0.243093098526752\\
28.1738539487389	0.243094465738989\\
28.1804676210793	0.243095832052167\\
28.1870812421499	0.243097197467081\\
28.1936948119852	0.243098561984524\\
28.2003083306195	0.243099925605289\\
28.2069217980871	0.243101288330168\\
28.2135352144223	0.243102650159951\\
28.2201485796594	0.243104011095428\\
28.2267618938326	0.243105371137389\\
28.233375156976	0.243106730286621\\
28.2399883691239	0.243108088543912\\
28.2466015303103	0.243109445910049\\
28.2532146405694	0.243110802385816\\
28.2598276999352	0.243112157971999\\
28.2664407084418	0.243113512669381\\
28.2730536661232	0.243114866478746\\
28.2796665730134	0.243116219400875\\
28.2862794291462	0.243117571436549\\
28.2928922345557	0.243118922586549\\
28.2995049892756	0.243120272851655\\
28.3061176933399	0.243121622232645\\
28.3127303467824	0.243122970730296\\
28.3193429496369	0.243124318345386\\
28.3259555019371	0.243125665078691\\
28.3325680037168	0.243127010930986\\
28.3391804550096	0.243128355903044\\
28.3457928558493	0.24312969999564\\
28.3524052062695	0.243131043209547\\
28.3590175063037	0.243132385545535\\
28.3656297559857	0.243133727004376\\
28.3722419553488	0.24313506758684\\
28.3788541044267	0.243136407293696\\
28.3854662032529	0.243137746125713\\
28.3920782518607	0.243139084083657\\
28.3986902502837	0.243140421168295\\
28.4053021985552	0.243141757380394\\
28.4119140967086	0.243143092720718\\
28.4185259447773	0.243144427190032\\
28.4251377427945	0.243145760789098\\
28.4317494907935	0.243147093518679\\
28.4383611888076	0.243148425379537\\
28.44497283687	0.243149756372431\\
28.4515844350139	0.243151086498124\\
28.4581959832724	0.243152415757372\\
28.4648074816788	0.243153744150935\\
28.471418930266	0.243155071679569\\
28.4780303290672	0.243156398344032\\
28.4846416781153	0.243157724145079\\
28.4912529774435	0.243159049083466\\
28.4978642270847	0.243160373159945\\
28.5044754270719	0.24316169637527\\
28.5110865774379	0.243163018730194\\
28.5176976782157	0.243164340225469\\
28.5243087294381	0.243165660861844\\
28.530919731138	0.24316698064007\\
28.5375306833482	0.243168299560896\\
28.5441415861014	0.243169617625071\\
28.5507524394304	0.243170934833341\\
28.557363243368	0.243172251186454\\
28.5639739979467	0.243173566685154\\
28.5705847031994	0.243174881330188\\
28.5771953591585	0.243176195122299\\
28.5838059658568	0.243177508062231\\
28.5904165233267	0.243178820150725\\
28.5970270316009	0.243180131388525\\
28.6036374907118	0.24318144177637\\
28.6102479006919	0.243182751315002\\
28.6168582615737	0.243184060005158\\
28.6234685733896	0.243185367847578\\
28.630078836172	0.243186674843\\
28.6366890499533	0.243187980992159\\
28.6432992147658	0.243189286295793\\
28.6499093306419	0.243190590754636\\
28.6565193976137	0.243191894369423\\
28.6631294157137	0.243193197140887\\
28.6697393849739	0.243194499069762\\
28.6763493054267	0.243195800156779\\
28.6829591771041	0.24319710040267\\
28.6895690000383	0.243198399808165\\
28.6961787742614	0.243199698373993\\
28.7027884998055	0.243200996100884\\
28.7093981767027	0.243202292989565\\
28.716007804985	0.243203589040764\\
28.7226173846843	0.243204884255208\\
28.7292269158327	0.243206178633621\\
28.7358363984621	0.243207472176728\\
28.7424458326044	0.243208764885255\\
28.7490552182915	0.243210056759923\\
28.7556645555553	0.243211347801456\\
28.7622738444275	0.243212638010575\\
28.76888308494	0.243213927388002\\
28.7754922771245	0.243215215934455\\
28.7821014210128	0.243216503650656\\
28.7887105166366	0.243217790537321\\
28.7953195640277	0.243219076595169\\
28.8019285632175	0.243220361824917\\
28.8085375142379	0.243221646227281\\
28.8151464171203	0.243222929802977\\
28.8217552718964	0.243224212552719\\
28.8283640785977	0.243225494477221\\
28.8349728372557	0.243226775577197\\
28.841581547902	0.243228055853357\\
28.8481902105679	0.243229335306415\\
28.854798825285	0.243230613937081\\
28.8614073920846	0.243231891746064\\
28.868015910998	0.243233168734075\\
28.8746243820568	0.243234444901821\\
28.8812328052921	0.24323572025001\\
28.8878411807353	0.243236994779349\\
28.8944495084177	0.243238268490545\\
28.9010577883704	0.243239541384302\\
28.9076660206248	0.243240813461325\\
28.9142742052119	0.243242084722319\\
28.920882342163	0.243243355167986\\
28.9274904315092	0.243244624799028\\
28.9340984732816	0.243245893616147\\
28.9407064675112	0.243247161620045\\
28.9473144142292	0.24324842881142\\
28.9539223134665	0.243249695190972\\
28.9605301652541	0.243250960759401\\
28.967137969623	0.243252225517402\\
28.9737457266042	0.243253489465675\\
28.9803534362286	0.243254752604914\\
28.9869610985269	0.243256014935816\\
28.9935687135302	0.243257276459074\\
29.0001762812692	0.243258537175384\\
29.0067838017747	0.243259797085438\\
29.0133912750776	0.243261056189929\\
29.0199987012085	0.243262314489548\\
29.0266060801982	0.243263571984987\\
29.0332134120774	0.243264828676937\\
29.0398206968767	0.243266084566085\\
29.0464279346268	0.243267339653121\\
29.0530351253584	0.243268593938734\\
29.0596422691019	0.24326984742361\\
29.066249365888	0.243271100108437\\
29.0728564157472	0.243272351993899\\
29.07946341871	0.243273603080681\\
29.0860703748069	0.24327485336947\\
29.0926772840684	0.243276102860946\\
29.0992841465248	0.243277351555795\\
29.1058909622066	0.243278599454697\\
29.1124977311441	0.243279846558334\\
29.1191044533677	0.243281092867387\\
29.1257111289077	0.243282338382536\\
29.1323177577945	0.243283583104459\\
29.1389243400582	0.243284827033836\\
29.1455308757292	0.243286070171344\\
29.1521373648376	0.243287312517659\\
29.1587438074136	0.243288554073459\\
29.1653502034874	0.243289794839419\\
29.1719565530891	0.243291034816213\\
29.1785628562489	0.243292274004516\\
29.1851691129968	0.243293512405\\
29.1917753233629	0.24329475001834\\
29.1983814873771	0.243295986845205\\
29.2049876050696	0.243297222886269\\
29.2115936764703	0.2432984581422\\
29.2181997016092	0.24329969261367\\
29.2248056805161	0.243300926301346\\
29.231411613221	0.243302159205897\\
29.2380174997537	0.243303391327991\\
29.2446233401441	0.243304622668295\\
29.2512291344221	0.243305853227475\\
29.2578348826173	0.243307083006196\\
29.2644405847597	0.243308312005122\\
29.2710462408789	0.243309540224919\\
29.2776518510046	0.243310767666249\\
29.2842574151666	0.243311994329775\\
29.2908629333944	0.243313220216158\\
29.2974684057179	0.243314445326061\\
29.3040738321665	0.243315669660143\\
29.3106792127698	0.243316893219064\\
29.3172845475575	0.243318116003483\\
29.323889836559	0.243319338014058\\
29.3304950798039	0.243320559251448\\
29.3371002773216	0.243321779716309\\
29.3437054291417	0.243322999409297\\
29.3503105352935	0.243324218331068\\
29.3569155958065	0.243325436482276\\
29.36352061071	0.243326653863577\\
29.3701255800334	0.243327870475622\\
29.376730503806	0.243329086319066\\
29.3833353820572	0.24333030139456\\
29.3899402148163	0.243331515702755\\
29.3965450021124	0.243332729244302\\
29.4031497439748	0.243333942019851\\
29.4097544404328	0.243335154030051\\
29.4163590915154	0.243336365275551\\
29.422963697252	0.243337575756998\\
29.4295682576715	0.24333878547504\\
29.4361727728031	0.243339994430324\\
29.4427772426759	0.243341202623495\\
29.4493816673189	0.243342410055198\\
29.4559860467611	0.243343616726077\\
29.4625903810317	0.243344822636777\\
29.4691946701595	0.243346027787941\\
29.4757989141734	0.24334723218021\\
29.4824031131026	0.243348435814227\\
29.4890072669757	0.243349638690632\\
29.4956113758218	0.243350840810066\\
29.5022154396696	0.243352042173169\\
29.508819458548	0.243353242780579\\
29.5154234324858	0.243354442632934\\
29.5220273615118	0.243355641730873\\
29.5286312456547	0.243356840075033\\
29.5352350849433	0.243358037666048\\
29.5418388794063	0.243359234504556\\
29.5484426290723	0.243360430591191\\
29.5550463339701	0.243361625926587\\
29.5616499941281	0.243362820511377\\
29.5682536095751	0.243364014346195\\
29.5748571803397	0.243365207431673\\
29.5814607064503	0.243366399768442\\
29.5880641879355	0.243367591357134\\
29.5946676248239	0.243368782198378\\
29.6012710171438	0.243369972292804\\
29.6078743649238	0.24337116164104\\
29.6144776681924	0.243372350243716\\
29.6210809269778	0.243373538101458\\
29.6276841413085	0.243374725214893\\
29.6342873112129	0.243375911584648\\
29.6408904367193	0.243377097211348\\
29.647493517856	0.243378282095618\\
29.6540965546512	0.243379466238082\\
29.6606995471334	0.243380649639364\\
29.6673024953306	0.243381832300086\\
29.6739053992711	0.243383014220872\\
29.6805082589832	0.243384195402341\\
29.6871110744949	0.243385375845117\\
29.6937138458344	0.243386555549817\\
29.7003165730298	0.243387734517063\\
29.7069192561093	0.243388912747473\\
29.7135218951008	0.243390090241666\\
29.7201244900325	0.243391267000259\\
29.7267270409324	0.243392443023868\\
29.7333295478284	0.243393618313111\\
29.7399320107485	0.243394792868604\\
29.7465344297207	0.24339596669096\\
29.7531368047729	0.243397139780795\\
29.7597391359331	0.243398312138723\\
29.766341423229	0.243399483765355\\
29.7729436666886	0.243400654661306\\
29.7795458663396	0.243401824827186\\
29.7861480222099	0.243402994263607\\
29.7927501343273	0.24340416297118\\
29.7993522027194	0.243405330950513\\
29.8059542274142	0.243406498202218\\
29.8125562084392	0.243407664726901\\
29.8191581458222	0.243408830525171\\
29.8257600395907	0.243409995597635\\
29.8323618897726	0.2434111599449\\
29.8389636963953	0.243412323567573\\
29.8455654594865	0.243413486466257\\
29.8521671790737	0.243414648641559\\
29.8587688551845	0.243415810094082\\
29.8653704878464	0.24341697082443\\
29.871972077087	0.243418130833206\\
29.8785736229336	0.243419290121011\\
29.8851751254137	0.243420448688448\\
29.8917765845548	0.243421606536117\\
29.8983780003843	0.243422763664618\\
29.9049793729295	0.243423920074553\\
29.9115807022178	0.243425075766518\\
29.9181819882766	0.243426230741114\\
29.9247832311331	0.243427384998937\\
29.9313844308147	0.243428538540585\\
29.9379855873486	0.243429691366654\\
29.9445867007621	0.243430843477741\\
29.9511877710823	0.243431994874441\\
29.9577887983365	0.243433145557348\\
29.9643897825518	0.243434295527056\\
29.9709907237555	0.243435444784159\\
29.9775916219745	0.24343659332925\\
29.9841924772361	0.243437741162921\\
29.9907932895673	0.243438888285763\\
29.9973940589951	0.243440034698368\\
30.0039947855467	0.243441180401325\\
30.0105954692489	0.243442325395225\\
30.0171961101289	0.243443469680656\\
30.0237967082135	0.243444613258208\\
30.0303972635297	0.243445756128467\\
30.0369977761045	0.243446898292022\\
30.0435982459647	0.243448039749458\\
30.0501986731371	0.243449180501362\\
30.0567990576488	0.24345032054832\\
};
\addplot [color=green,solid,forget plot]
  table[row sep=crcr]{%
30.0567990576488	0.24345032054832\\
30.0633993995264	0.243451459890915\\
30.0699996987968	0.243452598529733\\
30.0765999554867	0.243453736465357\\
30.083200169623	0.24345487369837\\
30.0898003412324	0.243456010229354\\
30.0964004703415	0.243457146058891\\
30.1030005569772	0.243458281187562\\
30.1096006011659	0.243459415615947\\
30.1162006029345	0.243460549344628\\
30.1228005623095	0.243461682374182\\
30.1294004793175	0.243462814705189\\
30.1360003539851	0.243463946338227\\
30.1426001863389	0.243465077273874\\
30.1491999764054	0.243466207512705\\
30.155799724211	0.243467337055299\\
30.1623994297824	0.24346846590223\\
30.168999093146	0.243469594054073\\
30.1755987143281	0.243470721511403\\
30.1821982933553	0.243471848274795\\
30.188797830254	0.243472974344821\\
30.1953973250504	0.243474099722054\\
30.201996777771	0.243475224407066\\
30.2085961884422	0.243476348400429\\
30.2151955570901	0.243477471702713\\
30.2217948837412	0.24347859431449\\
30.2283941684216	0.243479716236329\\
30.2349934111577	0.243480837468798\\
30.2415926119756	0.243481958012468\\
30.2481917709015	0.243483077867905\\
30.2547908879617	0.243484197035677\\
30.2613899631823	0.243485315516351\\
30.2679889965893	0.243486433310494\\
30.274587988209	0.24348755041867\\
30.2811869380675	0.243488666841445\\
30.2877858461907	0.243489782579383\\
30.2943847126048	0.243490897633048\\
30.3009835373357	0.243492012003004\\
30.3075823204095	0.243493125689814\\
30.3141810618522	0.243494238694039\\
30.3207797616896	0.24349535101624\\
30.3273784199478	0.24349646265698\\
30.3339770366527	0.243497573616818\\
30.3405756118301	0.243498683896314\\
30.347174145506	0.243499793496028\\
30.3537726377061	0.243500902416517\\
30.3603710884564	0.243502010658341\\
30.3669694977826	0.243503118222057\\
30.3735678657105	0.243504225108221\\
30.3801661922658	0.24350533131739\\
30.3867644774744	0.243506436850119\\
30.3933627213619	0.243507541706965\\
30.399960923954	0.243508645888481\\
30.4065590852764	0.243509749395222\\
30.4131572053549	0.243510852227741\\
30.4197552842149	0.243511954386591\\
30.4263533218821	0.243513055872324\\
30.4329513183821	0.243514156685492\\
30.4395492737405	0.243515256826647\\
30.4461471879829	0.243516356296339\\
30.4527450611347	0.243517455095118\\
30.4593428932215	0.243518553223533\\
30.4659406842687	0.243519650682134\\
30.4725384343018	0.243520747471469\\
30.4791361433463	0.243521843592085\\
30.4857338114275	0.243522939044531\\
30.492331438571	0.243524033829351\\
30.4989290248019	0.243525127947094\\
30.5055265701458	0.243526221398304\\
30.512124074628	0.243527314183526\\
30.5187215382737	0.243528406303305\\
30.5253189611083	0.243529497758185\\
30.531916343157	0.243530588548708\\
30.5385136844451	0.243531678675418\\
30.5451109849978	0.243532768138857\\
30.5517082448403	0.243533856939567\\
30.5583054639978	0.243534945078088\\
30.5649026424955	0.243536032554961\\
30.5714997803586	0.243537119370726\\
30.5780968776121	0.243538205525922\\
30.5846939342812	0.243539291021089\\
30.5912909503909	0.243540375856765\\
30.5978879259663	0.243541460033487\\
30.6044848610324	0.243542543551792\\
30.6110817556144	0.243543626412218\\
30.6176786097371	0.243544708615301\\
30.6242754234255	0.243545790161575\\
30.6308721967047	0.243546871051576\\
30.6374689295995	0.243547951285839\\
30.6440656221349	0.243549030864897\\
30.6506622743357	0.243550109789284\\
30.6572588862269	0.243551188059532\\
30.6638554578333	0.243552265676174\\
30.6704519891797	0.243553342639741\\
30.6770484802909	0.243554418950765\\
30.6836449311918	0.243555494609777\\
30.6902413419071	0.243556569617306\\
30.6968377124616	0.243557643973882\\
30.7034340428799	0.243558717680034\\
30.7100303331869	0.24355979073629\\
30.7166265834072	0.243560863143179\\
30.7232227935654	0.243561934901227\\
30.7298189636862	0.243563006010963\\
30.7364150937943	0.243564076472911\\
30.7430111839143	0.243565146287598\\
30.7496072340707	0.243566215455549\\
30.7562032442881	0.243567283977289\\
30.762799214591	0.243568351853341\\
30.7693951450041	0.24356941908423\\
30.7759910355517	0.243570485670479\\
30.7825868862585	0.24357155161261\\
30.7891826971488	0.243572616911145\\
30.7957784682471	0.243573681566606\\
30.8023741995778	0.243574745579513\\
30.8089698911653	0.243575808950388\\
30.8155655430341	0.243576871679749\\
30.8221611552084	0.243577933768117\\
30.8287567277127	0.24357899521601\\
30.8353522605712	0.243580056023947\\
30.8419477538083	0.243581116192445\\
30.8485432074482	0.243582175722022\\
30.8551386215152	0.243583234613194\\
30.8617339960336	0.243584292866478\\
30.8683293310276	0.243585350482389\\
30.8749246265213	0.243586407461443\\
30.8815198825391	0.243587463804154\\
30.8881150991049	0.243588519511036\\
30.8947102762431	0.243589574582602\\
30.9013054139776	0.243590629019367\\
30.9079005123327	0.243591682821842\\
30.9144955713323	0.24359273599054\\
30.9210905910007	0.243593788525971\\
30.9276855713617	0.243594840428648\\
30.9342805124395	0.24359589169908\\
30.940875414258	0.243596942337777\\
30.9474702768413	0.243597992345249\\
30.9540651002132	0.243599041722005\\
30.9606598843978	0.243600090468554\\
30.967254629419	0.243601138585402\\
30.9738493353006	0.243602186073058\\
30.9804440020666	0.243603232932028\\
30.9870386297409	0.243604279162818\\
30.9936332183472	0.243605324765936\\
31.0002277679095	0.243606369741885\\
31.0068222784515	0.24360741409117\\
31.013416749997	0.243608457814297\\
31.0200111825698	0.243609500911768\\
31.0266055761936	0.243610543384088\\
31.0331999308922	0.243611585231758\\
31.0397942466893	0.243612626455281\\
31.0463885236086	0.243613667055159\\
31.0529827616737	0.243614707031893\\
31.0595769609083	0.243615746385983\\
31.066171121336	0.243616785117931\\
31.0727652429806	0.243617823228236\\
31.0793593258654	0.243618860717396\\
31.0859533700142	0.243619897585911\\
31.0925473754504	0.243620933834279\\
31.0991413421977	0.243621969462998\\
31.1057352702795	0.243623004472564\\
31.1123291597194	0.243624038863475\\
31.1189230105408	0.243625072636227\\
31.1255168227671	0.243626105791315\\
31.1321105964219	0.243627138329234\\
31.1387043315285	0.243628170250481\\
31.1452980281104	0.243629201555547\\
31.1518916861909	0.243630232244928\\
31.1584853057934	0.243631262319116\\
31.1650788869413	0.243632291778605\\
31.1716724296578	0.243633320623886\\
31.1782659339663	0.243634348855451\\
31.18485939989	0.243635376473791\\
31.1914528274523	0.243636403479398\\
31.1980462166764	0.243637429872761\\
31.2046395675855	0.243638455654369\\
31.2112328802028	0.243639480824714\\
31.2178261545515	0.243640505384282\\
31.2244193906549	0.243641529333563\\
31.231012588536	0.243642552673043\\
31.237605748218	0.243643575403212\\
31.244198869724	0.243644597524554\\
31.2507919530771	0.243645619037557\\
31.2573849983005	0.243646639942706\\
31.263978005417	0.243647660240487\\
31.2705709744499	0.243648679931385\\
31.2771639054221	0.243649699015883\\
31.2837567983566	0.243650717494467\\
31.2903496532764	0.243651735367618\\
31.2969424702046	0.24365275263582\\
31.3035352491639	0.243653769299556\\
31.3101279901774	0.243654785359307\\
31.316720693268	0.243655800815555\\
31.3233133584586	0.243656815668781\\
31.329905985772	0.243657829919466\\
31.336498575231	0.243658843568088\\
31.3430911268587	0.243659856615128\\
31.3496836406777	0.243660869061065\\
31.3562761167108	0.243661880906378\\
31.3628685549809	0.243662892151544\\
31.3694609555107	0.243663902797041\\
31.3760533183229	0.243664912843347\\
31.3826456434403	0.243665922290937\\
31.3892379308856	0.243666931140289\\
31.3958301806815	0.243667939391877\\
31.4024223928506	0.243668947046177\\
31.4090145674156	0.243669954103664\\
31.4156067043991	0.243670960564812\\
31.4221988038238	0.243671966430094\\
31.4287908657123	0.243672971699985\\
31.435382890087	0.243673976374957\\
31.4419748769707	0.243674980455482\\
31.4485668263857	0.243675983942032\\
31.4551587383548	0.243676986835079\\
31.4617506129002	0.243677989135094\\
31.4683424500447	0.243678990842546\\
31.4749342498105	0.243679991957907\\
31.4815260122203	0.243680992481646\\
31.4881177372963	0.243681992414231\\
31.4947094250611	0.243682991756132\\
31.501301075537	0.243683990507816\\
31.5078926887463	0.243684988669752\\
31.5144842647116	0.243685986242406\\
31.521075803455	0.243686983226246\\
31.527667304999	0.243687979621736\\
31.5342587693658	0.243688975429345\\
31.5408501965777	0.243689970649536\\
31.547441586657	0.243690965282774\\
31.554032939626	0.243691959329525\\
31.5606242555069	0.243692952790251\\
31.5672155343218	0.243693945665418\\
31.5738067760931	0.243694937955486\\
31.5803979808429	0.24369592966092\\
31.5869891485933	0.243696920782181\\
31.5935802793664	0.243697911319731\\
31.6001713731846	0.243698901274032\\
31.6067624300697	0.243699890645543\\
31.613353450044	0.243700879434725\\
31.6199444331294	0.243701867642039\\
31.6265353793481	0.243702855267943\\
31.6331262887221	0.243703842312896\\
31.6397171612735	0.243704828777357\\
31.6463079970241	0.243705814661784\\
31.652898795996	0.243706799966635\\
31.6594895582113	0.243707784692366\\
31.6660802836917	0.243708768839434\\
31.6726709724593	0.243709752408295\\
31.679261624536	0.243710735399406\\
31.6858522399436	0.243711717813221\\
31.6924428187041	0.243712699650195\\
31.6990333608392	0.243713680910783\\
31.705623866371	0.243714661595438\\
31.7122143353211	0.243715641704615\\
31.7188047677114	0.243716621238766\\
31.7253951635637	0.243717600198343\\
31.7319855228998	0.243718578583799\\
31.7385758457414	0.243719556395586\\
31.7451661321102	0.243720533634155\\
31.751756382028	0.243721510299956\\
31.7583465955165	0.243722486393441\\
31.7649367725974	0.243723461915058\\
31.7715269132923	0.243724436865258\\
31.7781170176229	0.243725411244489\\
31.7847070856108	0.2437263850532\\
31.7912971172777	0.243727358291839\\
31.7978871126451	0.243728330960854\\
31.8044770717346	0.243729303060692\\
31.8110669945679	0.2437302745918\\
31.8176568811664	0.243731245554625\\
31.8242467315517	0.243732215949611\\
31.8308365457452	0.243733185777206\\
31.8374263237686	0.243734155037853\\
31.8440160656432	0.243735123731997\\
31.8506057713906	0.243736091860083\\
31.8571954410321	0.243737059422554\\
31.8637850745893	0.243738026419854\\
31.8703746720834	0.243738992852426\\
31.876964233536	0.243739958720711\\
31.8835537589684	0.243740924025152\\
31.8901432484019	0.24374188876619\\
31.8967327018579	0.243742852944267\\
31.9033221193578	0.243743816559824\\
31.9099115009227	0.243744779613299\\
31.9165008465741	0.243745742105135\\
31.9230901563332	0.243746704035768\\
31.9296794302212	0.24374766540564\\
31.9362686682595	0.243748626215187\\
31.9428578704691	0.243749586464849\\
31.9494470368714	0.243750546155063\\
31.9560361674875	0.243751505286266\\
31.9626252623386	0.243752463858895\\
31.9692143214458	0.243753421873386\\
31.9758033448303	0.243754379330175\\
31.9823923325133	0.243755336229698\\
31.9889812845157	0.243756292572391\\
31.9955702008588	0.243757248358686\\
32.0021590815635	0.24375820358902\\
32.008747926651	0.243759158263824\\
32.0153367361422	0.243760112383534\\
32.0219255100583	0.243761065948582\\
32.0285142484201	0.2437620189594\\
32.0351029512488	0.243762971416421\\
32.0416916185652	0.243763923320076\\
32.0482802503903	0.243764874670797\\
32.0548688467451	0.243765825469014\\
32.0614574076504	0.243766775715157\\
32.0680459331273	0.243767725409657\\
32.0746344231965	0.243768674552944\\
32.081222877879	0.243769623145446\\
32.0878112971955	0.243770571187592\\
32.094399681167	0.243771518679811\\
32.1009880298143	0.24377246562253\\
32.1075763431582	0.243773412016178\\
32.1141646212194	0.24377435786118\\
32.1207528640187	0.243775303157963\\
32.127341071577	0.243776247906955\\
32.1339292439149	0.24377719210858\\
32.1405173810531	0.243778135763264\\
32.1471054830124	0.243779078871431\\
32.1536935498135	0.243780021433508\\
32.160281581477	0.243780963449917\\
32.1668695780235	0.243781904921083\\
32.1734575394739	0.243782845847428\\
32.1800454658485	0.243783786229376\\
32.1866333571682	0.243784726067349\\
32.1932212134534	0.243785665361769\\
32.1998090347247	0.243786604113058\\
32.2063968210027	0.243787542321637\\
32.212984572308	0.243788479987927\\
32.2195722886611	0.243789417112349\\
32.2261599700824	0.243790353695321\\
32.2327476165925	0.243791289737265\\
32.2393352282119	0.243792225238599\\
32.2459228049609	0.243793160199742\\
32.2525103468602	0.243794094621113\\
32.2590978539301	0.243795028503129\\
32.2656853261909	0.243795961846208\\
32.2722727636632	0.243796894650767\\
32.2788601663673	0.243797826917222\\
32.2854475343235	0.243798758645992\\
32.2920348675523	0.24379968983749\\
32.2986221660739	0.243800620492133\\
32.3052094299087	0.243801550610336\\
32.3117966590769	0.243802480192514\\
32.3183838535989	0.24380340923908\\
32.3249710134949	0.24380433775045\\
32.3315581387852	0.243805265727035\\
32.33814522949	0.243806193169251\\
32.3447322856296	0.243807120077509\\
32.3513193072241	0.243808046452221\\
32.3579062942938	0.2438089722938\\
32.3644932468587	0.243809897602657\\
32.3710801649392	0.243810822379203\\
32.3776670485552	0.243811746623849\\
32.384253897727	0.243812670337005\\
32.3908407124747	0.243813593519082\\
32.3974274928183	0.243814516170488\\
32.404014238778	0.243815438291633\\
32.4106009503737	0.243816359882926\\
32.4171876276256	0.243817280944774\\
32.4237742705537	0.243818201477587\\
32.4303608791779	0.243819121481771\\
32.4369474535184	0.243820040957733\\
32.443533993595	0.243820959905881\\
32.4501204994278	0.243821878326621\\
32.4567069710367	0.243822796220358\\
32.4632934084417	0.243823713587499\\
32.4698798116626	0.243824630428448\\
32.4764661807194	0.243825546743611\\
32.4830525156319	0.243826462533391\\
32.4896388164201	0.243827377798193\\
32.4962250831039	0.24382829253842\\
32.5028113157029	0.243829206754476\\
32.5093975142371	0.243830120446763\\
32.5159836787264	0.243831033615684\\
32.5225698091904	0.243831946261642\\
32.529155905649	0.243832858385036\\
32.5357419681219	0.24383376998627\\
32.5423279966289	0.243834681065744\\
32.5489139911898	0.243835591623857\\
32.5554999518242	0.243836501661012\\
32.5620858785519	0.243837411177606\\
32.5686717713925	0.243838320174041\\
32.5752576303657	0.243839228650713\\
32.5818434554912	0.243840136608023\\
32.5884292467887	0.243841044046369\\
32.5950150042777	0.243841950966147\\
32.6016007279778	0.243842857367756\\
32.6081864179088	0.243843763251593\\
32.6147720740901	0.243844668618054\\
32.6213576965413	0.243845573467536\\
32.6279432852819	0.243846477800435\\
32.6345288403316	0.243847381617145\\
32.6411143617098	0.243848284918063\\
32.6476998494361	0.243849187703583\\
32.6542853035298	0.243850089974099\\
32.6608707240106	0.243850991730005\\
32.6674561108978	0.243851892971696\\
32.674041464211	0.243852793699564\\
32.6806267839695	0.243853693914002\\
32.6872120701927	0.243854593615402\\
32.6937973229001	0.243855492804158\\
32.7003825421111	0.24385639148066\\
32.7069677278449	0.243857289645299\\
32.713552880121	0.243858187298468\\
32.7201379989588	0.243859084440556\\
32.7267230843774	0.243859981071954\\
32.7333081363964	0.243860877193052\\
32.7398931550348	0.243861772804238\\
32.7464781403121	0.243862667905903\\
32.7530630922474	0.243863562498434\\
32.7596480108601	0.243864456582221\\
32.7662328961693	0.243865350157651\\
32.7728177481944	0.243866243225112\\
32.7794025669544	0.243867135784992\\
32.7859873524687	0.243868027837676\\
32.7925721047562	0.243868919383552\\
32.7991568238363	0.243869810423005\\
32.8057415097281	0.243870700956422\\
32.8123261624507	0.243871590984188\\
32.8189107820232	0.243872480506688\\
32.8254953684647	0.243873369524307\\
32.8320799217943	0.243874258037428\\
32.838664442031	0.243875146046436\\
32.845248929194	0.243876033551715\\
32.8518333833023	0.243876920553647\\
32.8584178043749	0.243877807052615\\
32.8650021924308	0.243878693049002\\
32.871586547489	0.24387957854319\\
32.8781708695684	0.243880463535561\\
32.8847551586882	0.243881348026496\\
32.8913394148672	0.243882232016376\\
32.8979236381243	0.243883115505582\\
32.9045078284785	0.243883998494494\\
32.9110919859487	0.243884880983491\\
32.9176761105537	0.243885762972954\\
32.9242602023126	0.243886644463262\\
32.930844261244	0.243887525454793\\
32.937428287367	0.243888405947925\\
32.9440122807002	0.243889285943038\\
32.9505962412626	0.243890165440509\\
32.957180169073	0.243891044440715\\
32.9637640641501	0.243891922944032\\
32.9703479265127	0.243892800950839\\
32.9769317561796	0.243893678461511\\
32.9835155531695	0.243894555476424\\
32.9900993175012	0.243895431995953\\
32.9966830491934	0.243896308020475\\
33.0032667482648	0.243897183550363\\
33.0098504147341	0.243898058585992\\
33.0164340486199	0.243898933127737\\
33.023017649941	0.243899807175971\\
33.029601218716	0.243900680731068\\
33.0361847549634	0.2439015537934\\
33.042768258702	0.243902426363341\\
33.0493517299504	0.243903298441263\\
33.055935168727	0.243904170027538\\
33.0625185750506	0.243905041122537\\
33.0691019489396	0.243905911726632\\
33.0756852904127	0.243906781840195\\
33.0822685994883	0.243907651463594\\
33.088851876185	0.243908520597202\\
33.0954351205212	0.243909389241387\\
33.1020183325155	0.243910257396519\\
33.1086015121864	0.243911125062968\\
33.1151846595523	0.243911992241102\\
33.1217677746316	0.24391285893129\\
33.1283508574429	0.2439137251339\\
33.1349339080044	0.243914590849299\\
33.1415169263347	0.243915456077856\\
33.1480999124521	0.243916320819938\\
33.1546828663749	0.24391718507591\\
33.1612657881216	0.24391804884614\\
33.1678486777106	0.243918912130994\\
33.17443153516	0.243919774930837\\
33.1810143604884	0.243920637246034\\
33.1875971537138	0.243921499076952\\
33.1941799148548	0.243922360423954\\
33.2007626439295	0.243923221287405\\
33.2073453409562	0.243924081667668\\
33.2139280059532	0.243924941565108\\
33.2205106389386	0.243925800980088\\
33.2270932399308	0.24392665991297\\
33.2336758089479	0.243927518364118\\
33.2402583460082	0.243928376333894\\
33.2468408511297	0.243929233822659\\
33.2534233243307	0.243930090830775\\
33.2600057656294	0.243930947358603\\
33.2665881750438	0.243931803406505\\
33.2731705525921	0.243932658974841\\
33.2797528982925	0.243933514063971\\
33.2863352121629	0.243934368674255\\
33.2929174942215	0.243935222806053\\
33.2994997444864	0.243936076459723\\
33.3060819629756	0.243936929635625\\
33.3126641497072	0.243937782334117\\
33.3192463046991	0.243938634555558\\
33.3258284279695	0.243939486300304\\
33.3324105195363	0.243940337568715\\
33.3389925794175	0.243941188361146\\
33.345574607631	0.243942038677956\\
33.3521566041949	0.243942888519499\\
33.3587385691271	0.243943737886133\\
33.3653205024455	0.243944586778214\\
33.371902404168	0.243945435196096\\
33.3784842743125	0.243946283140135\\
33.385066112897	0.243947130610686\\
33.3916479199393	0.243947977608103\\
33.3982296954573	0.243948824132741\\
33.4048114394688	0.243949670184953\\
33.4113931519916	0.243950515765093\\
33.4179748330437	0.243951360873514\\
33.4245564826428	0.243952205510569\\
33.4311381008066	0.24395304967661\\
33.437719687553	0.243953893371989\\
33.4443012428998	0.243954736597059\\
33.4508827668647	0.243955579352171\\
33.4574642594654	0.243956421637676\\
33.4640457207198	0.243957263453925\\
33.4706271506454	0.243958104801268\\
33.47720854926	0.243958945680056\\
33.4837899165813	0.243959786090638\\
33.490371252627	0.243960626033365\\
33.4969525574147	0.243961465508584\\
33.503533830962	0.243962304516646\\
33.5101150732867	0.243963143057898\\
33.5166962844063	0.243963981132689\\
33.5232774643385	0.243964818741367\\
33.5298586131007	0.24396565588428\\
33.5364397307107	0.243966492561774\\
33.543020817186	0.243967328774196\\
33.5496018725442	0.243968164521894\\
33.5561828968027	0.243968999805213\\
33.5627638899791	0.243969834624499\\
33.569344852091	0.243970668980099\\
33.5759257831558	0.243971502872356\\
33.582506683191	0.243972336301617\\
33.5890875522141	0.243973169268226\\
33.5956683902425	0.243974001772527\\
33.6022491972938	0.243974833814864\\
33.6088299733853	0.243975665395582\\
33.6154107185345	0.243976496515022\\
33.6219914327587	0.243977327173529\\
33.6285721160755	0.243978157371445\\
33.635152768502	0.243978987109113\\
33.6417333900558	0.243979816386874\\
33.6483139807542	0.243980645205071\\
33.6548945406145	0.243981473564044\\
33.6614750696541	0.243982301464135\\
33.6680555678902	0.243983128905685\\
33.6746360353402	0.243983955889034\\
33.6812164720214	0.243984782414523\\
33.687796877951	0.243985608482491\\
33.6943772531463	0.243986434093277\\
33.7009575976246	0.243987259247222\\
33.7075379114031	0.243988083944664\\
33.714118194499	0.243988908185941\\
33.7206984469296	0.243989731971392\\
33.727278668712	0.243990555301354\\
33.7338588598635	0.243991378176167\\
33.7404390204011	0.243992200596165\\
33.7470191503421	0.243993022561688\\
33.7535992497036	0.243993844073071\\
33.7601793185028	0.243994665130651\\
33.7667593567568	0.243995485734764\\
33.7733393644826	0.243996305885745\\
33.7799193416974	0.243997125583931\\
33.7864992884183	0.243997944829655\\
33.7930792046623	0.243998763623254\\
33.7996590904465	0.243999581965062\\
33.8062389457879	0.244000399855412\\
33.8128187707036	0.244001217294639\\
33.8193985652106	0.244002034283077\\
33.8259783293259	0.244002850821058\\
33.8325580630665	0.244003666908915\\
33.8391377664493	0.244004482546982\\
33.8457174394914	0.24400529773559\\
33.8522970822097	0.244006112475072\\
33.8588766946211	0.244006926765759\\
33.8654562767426	0.244007740607983\\
33.8720358285911	0.244008554002074\\
33.8786153501835	0.244009366948365\\
33.8851948415366	0.244010179447184\\
33.8917743026674	0.244010991498863\\
33.8983537335927	0.24401180310373\\
33.9049331343295	0.244012614262117\\
33.9115125048944	0.244013424974352\\
33.9180918453044	0.244014235240764\\
33.9246711555763	0.244015045061681\\
33.9312504357268	0.244015854437433\\
33.9378296857728	0.244016663368346\\
33.9444089057311	0.24401747185475\\
33.9509880956183	0.244018279896971\\
33.9575672554513	0.244019087495336\\
33.9641463852469	0.244019894650173\\
33.9707254850216	0.244020701361807\\
33.9773045547923	0.244021507630566\\
33.9838835945757	0.244022313456774\\
33.9904626043883	0.244023118840759\\
33.997041584247	0.244023923782844\\
34.0036205341684	0.244024728283355\\
34.0101994541692	0.244025532342617\\
34.0167783442659	0.244026335960954\\
34.0233572044752	0.244027139138691\\
34.0299360348137	0.24402794187615\\
34.0365148352981	0.244028744173657\\
34.0430936059448	0.244029546031533\\
34.0496723467706	0.244030347450102\\
34.0562510577919	0.244031148429687\\
34.0628297390254	0.244031948970609\\
34.0694083904875	0.244032749073192\\
34.0759870121947	0.244033548737756\\
34.0825656041637	0.244034347964623\\
34.0891441664109	0.244035146754114\\
34.0957226989528	0.244035945106551\\
34.1023012018059	0.244036743022253\\
34.1088796749866	0.244037540501541\\
34.1154581185114	0.244038337544735\\
34.1220365323967	0.244039134152154\\
34.1286149166591	0.244039930324119\\
34.1351932713148	0.244040726060947\\
34.1417715963803	0.244041521362959\\
34.148349891872	0.244042316230471\\
34.1549281578063	0.244043110663804\\
34.1615063941996	0.244043904663273\\
34.1680846010681	0.244044698229198\\
34.1746627784283	0.244045491361895\\
34.1812409262964	0.244046284061682\\
34.1878190446889	0.244047076328875\\
34.194397133622	0.24404786816379\\
34.200975193112	0.244048659566745\\
34.2075532231751	0.244049450538053\\
34.2141312238278	0.244050241078033\\
34.2207091950861	0.244051031186997\\
34.2272871369665	0.244051820865263\\
34.233865049485	0.244052610113143\\
34.240442932658	0.244053398930954\\
34.2470207865016	0.244054187319008\\
34.253598611032	0.24405497527762\\
34.2601764062655	0.244055762807103\\
34.2667541722182	0.244056549907771\\
34.2733319089063	0.244057336579936\\
34.2799096163459	0.244058122823911\\
34.2864872945531	0.244058908640009\\
34.2930649435442	0.244059694028542\\
34.2996425633351	0.244060478989821\\
34.3062201539421	0.244061263524159\\
34.3127977153811	0.244062047631866\\
34.3193752476684	0.244062831313253\\
34.3259527508198	0.244063614568631\\
34.3325302248516	0.244064397398311\\
34.3391076697798	0.244065179802603\\
34.3456850856203	0.244065961781816\\
34.3522624723893	0.24406674333626\\
34.3588398301026	0.244067524466244\\
34.3654171587764	0.244068305172078\\
34.3719944584265	0.244069085454069\\
34.3785717290691	0.244069865312527\\
34.3851489707199	0.24407064474776\\
34.391726183395	0.244071423760075\\
34.3983033671104	0.24407220234978\\
34.4048805218819	0.244072980517183\\
34.4114576477254	0.24407375826259\\
34.4180347446569	0.244074535586308\\
34.4246118126923	0.244075312488644\\
34.4311888518474	0.244076088969903\\
34.4377658621381	0.244076865030392\\
34.4443428435802	0.244077640670417\\
34.4509197961897	0.244078415890282\\
34.4574967199822	0.244079190690292\\
34.4640736149737	0.244079965070754\\
34.47065048118	0.24408073903197\\
34.4772273186168	0.244081512574245\\
34.4838041273	0.244082285697884\\
34.4903809072452	0.24408305840319\\
34.4969576584684	0.244083830690466\\
34.5035343809851	0.244084602560015\\
34.5101110748113	0.24408537401214\\
34.5166877399625	0.244086145047145\\
34.5232643764545	0.24408691566533\\
34.5298409843031	0.244087685866999\\
34.5364175635238	0.244088455652452\\
34.5429941141325	0.244089225021992\\
34.5495706361447	0.244089993975919\\
34.5561471295761	0.244090762514535\\
34.5627235944423	0.24409153063814\\
34.5693000307591	0.244092298347035\\
34.575876438542	0.244093065641519\\
34.5824528178066	0.244093832521893\\
34.5890291685685	0.244094598988455\\
34.5956054908433	0.244095365041506\\
34.6021817846467	0.244096130681344\\
34.6087580499941	0.244096895908268\\
34.6153342869011	0.244097660722577\\
34.6219104953833	0.244098425124569\\
34.6284866754561	0.244099189114541\\
34.6350628271352	0.244099952692791\\
34.641638950436	0.244100715859617\\
34.648215045374	0.244101478615316\\
34.6547911119647	0.244102240960184\\
34.6613671502236	0.244103002894519\\
34.6679431601661	0.244103764418617\\
34.6745191418078	0.244104525532772\\
34.6810950951639	0.244105286237283\\
34.6876710202501	0.244106046532443\\
34.6942469170816	0.244106806418548\\
34.7008227856739	0.244107565895893\\
34.7073986260423	0.244108324964773\\
34.7139744382024	0.244109083625483\\
34.7205502221694	0.244109841878315\\
34.7271259779586	0.244110599723565\\
34.7337017055856	0.244111357161526\\
34.7402774050655	0.244112114192491\\
34.7468530764138	0.244112870816754\\
34.7534287196456	0.244113627034607\\
34.7600043347765	0.244114382846342\\
34.7665799218215	0.244115138252253\\
34.7731554807961	0.244115893252631\\
34.7797310117154	0.244116647847768\\
34.7863065145949	0.244117402037955\\
34.7928819894496	0.244118155823484\\
34.7994574362948	0.244118909204646\\
34.8060328551459	0.244119662181731\\
34.8126082460179	0.24412041475503\\
34.819183608926	0.244121166924833\\
34.8257589438856	0.244121918691431\\
34.8323342509117	0.244122670055112\\
34.8389095300196	0.244123421016166\\
34.8454847812244	0.244124171574882\\
34.8520600045412	0.24412492173155\\
34.8586351999852	0.244125671486457\\
34.8652103675715	0.244126420839893\\
34.8717855073152	0.244127169792145\\
34.8783606192315	0.244127918343501\\
34.8849357033354	0.244128666494249\\
34.891510759642	0.244129414244677\\
34.8980857881664	0.24413016159507\\
34.9046607889236	0.244130908545716\\
34.9112357619288	0.244131655096902\\
34.9178107071968	0.244132401248914\\
34.9243856247428	0.244133147002038\\
34.9309605145818	0.244133892356559\\
34.9375353767288	0.244134637312764\\
34.9441102111988	0.244135381870937\\
34.9506850180068	0.244136126031364\\
34.9572597971677	0.244136869794329\\
34.9638345486964	0.244137613160117\\
34.9704092726081	0.244138356129012\\
34.9769839689175	0.244139098701298\\
34.9835586376397	0.244139840877259\\
34.9901332787895	0.244140582657178\\
34.9967078923819	0.244141324041338\\
35.0032824784317	0.244142065030023\\
35.0098570369539	0.244142805623515\\
35.0164315679633	0.244143545822097\\
35.0230060714748	0.244144285626051\\
35.0295805475032	0.244145025035658\\
35.0361549960633	0.244145764051201\\
35.0427294171701	0.244146502672961\\
35.0493038108384	0.24414724090122\\
35.0558781770829	0.244147978736257\\
35.0624525159184	0.244148716178354\\
35.0690268273598	0.244149453227791\\
35.0756011114217	0.244150189884849\\
35.0821753681191	0.244150926149807\\
35.0887495974666	0.244151662022946\\
35.095323799479	0.244152397504543\\
35.101897974171	0.24415313259488\\
35.1084721215574	0.244153867294234\\
35.1150462416529	0.244154601602885\\
35.1216203344721	0.24415533552111\\
35.1281944000298	0.244156069049188\\
35.1347684383406	0.244156802187397\\
35.1413424494193	0.244157534936015\\
35.1479164332804	0.244158267295319\\
35.1544903899387	0.244158999265586\\
35.1610643194087	0.244159730847093\\
35.1676382217052	0.244160462040117\\
35.1742120968426	0.244161192844935\\
35.1807859448358	0.244161923261822\\
35.1873597656991	0.244162653291055\\
35.1939335594473	0.244163382932908\\
35.2005073260948	0.244164112187659\\
35.2070810656563	0.244164841055581\\
35.2136547781464	0.244165569536951\\
35.2202284635795	0.244166297632041\\
35.2268021219702	0.244167025341128\\
35.233375753333	0.244167752664486\\
35.2399493576825	0.244168479602387\\
35.2465229350331	0.244169206155107\\
35.2530964853993	0.244169932322918\\
35.2596700087957	0.244170658106095\\
35.2662435052366	0.244171383504909\\
35.2728169747366	0.244172108519633\\
35.2793904173101	0.244172833150541\\
35.2859638329716	0.244173557397904\\
35.2925372217354	0.244174281261995\\
35.299110583616	0.244175004743085\\
35.3056839186279	0.244175727841446\\
35.3122572267853	0.244176450557349\\
35.3188305081027	0.244177172891065\\
35.3254037625945	0.244177894842865\\
35.3319769902751	0.244178616413021\\
35.3385501911587	0.244179337601801\\
35.3451233652598	0.244180058409476\\
35.3516965125927	0.244180778836316\\
35.3582696331717	0.244181498882591\\
35.3648427270111	0.24418221854857\\
35.3714157941253	0.244182937834522\\
35.3779888345285	0.244183656740716\\
35.384561848235	0.244184375267422\\
35.391134835259	0.244185093414906\\
35.397707795615	0.244185811183437\\
35.4042807293171	0.244186528573285\\
35.4108536363795	0.244187245584715\\
35.4174265168165	0.244187962217996\\
35.4239993706423	0.244188678473394\\
35.4305721978712	0.244189394351178\\
35.4371449985173	0.244190109851613\\
35.4437177725948	0.244190824974966\\
35.4502905201179	0.244191539721503\\
35.4568632411008	0.244192254091491\\
35.4634359355577	0.244192968085196\\
35.4700086035027	0.244193681702882\\
35.4765812449499	0.244194394944816\\
35.4831538599135	0.244195107811262\\
35.4897264484076	0.244195820302485\\
35.4962990104463	0.244196532418751\\
35.5028715460437	0.244197244160322\\
35.5094440552139	0.244197955527464\\
35.516016537971	0.244198666520441\\
35.5225889943291	0.244199377139516\\
35.5291614243022	0.244200087384952\\
35.5357338279044	0.244200797257014\\
35.5423062051497	0.244201506755963\\
35.5488785560522	0.244202215882063\\
35.5554508806258	0.244202924635576\\
35.5620231788846	0.244203633016765\\
35.5685954508426	0.244204341025892\\
35.5751676965137	0.244205048663218\\
35.581739915912	0.244205755929005\\
35.5883121090515	0.244206462823515\\
35.594884275946	0.244207169347008\\
35.6014564166096	0.244207875499746\\
35.6080285310562	0.244208581281989\\
35.6146006192997	0.244209286693999\\
35.621172681354	0.244209991736034\\
35.6277447172331	0.244210696408355\\
35.6343167269509	0.244211400711222\\
35.6408887105211	0.244212104644894\\
35.6474606679578	0.244212808209631\\
35.6540325992749	0.244213511405692\\
35.660604504486	0.244214214233335\\
35.6671763836052	0.24421491669282\\
35.6737482366463	0.244215618784404\\
35.680320063623	0.244216320508346\\
35.6868918645493	0.244217021864904\\
35.6934636394388	0.244217722854335\\
35.7000353883056	0.244218423476898\\
35.7066071111632	0.244219123732849\\
35.7131788080255	0.244219823622445\\
35.7197504789064	0.244220523145943\\
35.7263221238194	0.2442212223036\\
35.7328937427785	0.244221921095672\\
35.7394653357973	0.244222619522415\\
35.7460369028896	0.244223317584086\\
35.7526084440691	0.244224015280939\\
35.7591799593495	0.24422471261323\\
35.7657514487445	0.244225409581216\\
35.7723229122679	0.244226106185149\\
35.7788943499332	0.244226802425286\\
35.7854657617543	0.244227498301881\\
35.7920371477447	0.244228193815188\\
35.798608507918	0.244228888965462\\
35.8051798422881	0.244229583752956\\
35.8117511508684	0.244230278177923\\
35.8183224336727	0.244230972240619\\
35.8248936907144	0.244231665941294\\
35.8314649220073	0.244232359280204\\
35.838036127565	0.2442330522576\\
35.844607307401	0.244233744873734\\
35.8511784615289	0.24423443712886\\
35.8577495899623	0.24423512902323\\
35.8643206927147	0.244235820557095\\
35.8708917697997	0.244236511730706\\
35.8774628212308	0.244237202544317\\
35.8840338470216	0.244237892998177\\
35.8906048471855	0.244238583092537\\
35.8971758217362	0.244239272827649\\
35.9037467706871	0.244239962203763\\
35.9103176940516	0.24424065122113\\
35.9168885918433	0.244241339879999\\
35.9234594640757	0.244242028180621\\
35.9300303107622	0.244242716123245\\
35.9366011319162	0.24424340370812\\
35.9431719275513	0.244244090935497\\
35.9497426976808	0.244244777805623\\
35.9563134423181	0.244245464318748\\
35.9628841614768	0.244246150475121\\
35.9694548551702	0.244246836274989\\
35.9760255234116	0.244247521718601\\
35.9825961662145	0.244248206806205\\
35.9891667835923	0.244248891538049\\
35.9957373755583	0.24424957591438\\
36.002307942126	0.244250259935445\\
36.0088784833085	0.244250943601492\\
36.0154489991193	0.244251626912768\\
36.0220194895718	0.244252309869518\\
36.0285899546792	0.244252992471991\\
36.0351603944548	0.24425367472043\\
36.041730808912	0.244254356615084\\
36.048301198064	0.244255038156197\\
36.0548715619241	0.244255719344016\\
36.0614419005057	0.244256400178784\\
36.0680122138219	0.244257080660749\\
36.0745825018861	0.244257760790154\\
36.0811527647114	0.244258440567245\\
36.0877230023111	0.244259119992265\\
36.0942932146985	0.24425979906546\\
36.1008634018868	0.244260477787073\\
36.1074335638891	0.244261156157348\\
36.1140037007186	0.244261834176529\\
36.1205738123887	0.244262511844859\\
36.1271438989124	0.244263189162581\\
36.1337139603029	0.244263866129939\\
36.1402839965733	0.244264542747176\\
36.1468540077369	0.244265219014533\\
36.1534239938068	0.244265894932254\\
36.1599939547961	0.24426657050058\\
36.166563890718	0.244267245719754\\
36.1731338015855	0.244267920590017\\
36.1797036874117	0.244268595111611\\
36.1862735482099	0.244269269284776\\
36.192843383993	0.244269943109756\\
36.1994131947741	0.244270616586789\\
36.2059829805663	0.244271289716117\\
36.2125527413827	0.244271962497981\\
36.2191224772363	0.24427263493262\\
36.2256921881402	0.244273307020276\\
36.2322618741073	0.244273978761186\\
36.2388315351508	0.244274650155593\\
36.2454011712837	0.244275321203734\\
36.2519707825188	0.244275991905849\\
36.2585403688694	0.244276662262177\\
36.2651099303482	0.244277332272957\\
36.2716794669684	0.244278001938428\\
36.2782489787428	0.244278671258827\\
36.2848184656845	0.244279340234394\\
36.2913879278064	0.244280008865366\\
36.2979573651214	0.244280677151981\\
36.3045267776426	0.244281345094477\\
36.3110961653827	0.24428201269309\\
36.3176655283547	0.244282679948059\\
36.3242348665715	0.24428334685962\\
36.3308041800461	0.24428401342801\\
36.3373734687913	0.244284679653466\\
36.3439427328199	0.244285345536223\\
36.350511972145	0.244286011076519\\
36.3570811867792	0.244286676274588\\
36.3636503767355	0.244287341130668\\
36.3702195420268	0.244288005644992\\
36.3767886826658	0.244288669817797\\
36.3833577986654	0.244289333649318\\
36.3899268900383	0.24428999713979\\
36.3964959567975	0.244290660289447\\
36.4030649989557	0.244291323098525\\
36.4096340165257	0.244291985567256\\
36.4162030095203	0.244292647695876\\
36.4227719779522	0.244293309484619\\
36.4293409218342	0.244293970933717\\
36.4359098411791	0.244294632043405\\
36.4424787359996	0.244295292813916\\
36.4490476063084	0.244295953245483\\
36.4556164521184	0.244296613338339\\
36.4621852734421	0.244297273092717\\
36.4687540702923	0.244297932508849\\
36.4753228426818	0.244298591586968\\
36.4818915906231	0.244299250327305\\
36.488460314129	0.244299908730093\\
36.4950290132122	0.244300566795563\\
36.5015976878853	0.244301224523947\\
36.508166338161	0.244301881915475\\
36.5147349640519	0.24430253897038\\
36.5213035655707	0.244303195688893\\
36.52787214273	0.244303852071243\\
36.5344406955424	0.244304508117661\\
36.5410092240206	0.244305163828378\\
36.5475777281771	0.244305819203624\\
36.5541462080245	0.244306474243629\\
36.5607146635754	0.244307128948622\\
36.5672830948424	0.244307783318833\\
36.5738515018381	0.244308437354491\\
36.5804198845751	0.244309091055826\\
36.5869882430658	0.244309744423066\\
36.5935565773228	0.24431039745644\\
36.6001248873587	0.244311050156177\\
36.606693173186	0.244311702522505\\
36.6132614348171	0.244312354555652\\
36.6198296722647	0.244313006255846\\
36.6263978855412	0.244313657623316\\
36.6329660746591	0.244314308658287\\
36.6395342396309	0.244314959360988\\
36.646102380469	0.244315609731647\\
36.6526704971859	0.244316259770489\\
36.6592385897942	0.244316909477742\\
36.6658066583061	0.244317558853632\\
36.6723747027342	0.244318207898386\\
36.6789427230909	0.244318856612229\\
36.6855107193886	0.244319504995389\\
36.6920786916397	0.244320153048091\\
36.6986466398567	0.244320800770559\\
36.7052145640519	0.244321448163021\\
36.7117824642377	0.2443220952257\\
36.7183503404265	0.244322741958823\\
36.7249181926306	0.244323388362613\\
36.7314860208625	0.244324034437296\\
36.7380538251345	0.244324680183096\\
36.7446216054588	0.244325325600237\\
36.751189361848	0.244325970688944\\
36.7577570943142	0.244326615449439\\
36.7643248028698	0.244327259881948\\
36.7708924875272	0.244327903986693\\
36.7774601482985	0.244328547763897\\
36.7840277851962	0.244329191213785\\
36.7905953982325	0.244329834336578\\
36.7971629874196	0.244330477132499\\
36.8037305527699	0.244331119601772\\
36.8102980942955	0.244331761744618\\
36.8168656120088	0.24433240356126\\
36.823433105922	0.244333045051919\\
36.8300005760474	0.244333686216817\\
36.8365680223971	0.244334327056176\\
36.8431354449833	0.244334967570218\\
36.8497028438183	0.244335607759163\\
36.8562702189143	0.244336247623233\\
36.8628375702835	0.244336887162648\\
36.869404897938	0.244337526377629\\
36.8759722018901	0.244338165268397\\
36.8825394821518	0.244338803835171\\
36.8891067387354	0.244339442078172\\
36.895673971653	0.24434007999762\\
36.9022411809167	0.244340717593735\\
36.9088083665387	0.244341354866736\\
36.9153755285311	0.244341991816841\\
36.921942666906	0.244342628444272\\
36.9285097816756	0.244343264749246\\
36.9350768728518	0.244343900731982\\
36.9416439404469	0.244344536392699\\
36.9482109844729	0.244345171731615\\
36.9547780049418	0.244345806748948\\
36.9613450018658	0.244346441444917\\
36.9679119752569	0.24434707581974\\
36.9744789251272	0.244347709873633\\
36.9810458514886	0.244348343606815\\
36.9876127543532	0.244348977019503\\
36.9941796337331	0.244349610111913\\
37.0007464896403	0.244350242884264\\
37.0073133220867	0.24435087533677\\
37.0138801310844	0.24435150746965\\
37.0204469166454	0.24435213928312\\
37.0270136787816	0.244352770777395\\
37.033580417505	0.244353401952691\\
37.0401471328276	0.244354032809225\\
37.0467138247614	0.244354663347212\\
37.0532804933183	0.244355293566868\\
37.0598471385102	0.244355923468407\\
37.0664137603491	0.244356553052046\\
37.0729803588469	0.244357182317998\\
37.0795469340155	0.244357811266479\\
37.0861134858668	0.244358439897702\\
37.0926800144128	0.244359068211883\\
37.0992465196653	0.244359696209236\\
37.1058130016362	0.244360323889974\\
37.1123794603374	0.244360951254312\\
37.1189458957808	0.244361578302463\\
37.1255123079781	0.24436220503464\\
37.1320786969414	0.244362831451057\\
37.1386450626824	0.244363457551926\\
37.1452114052129	0.244364083337462\\
37.1517777245448	0.244364708807876\\
37.1583440206899	0.244365333963381\\
37.16491029366	0.244365958804189\\
37.1714765434669	0.244366583330513\\
37.1780427701225	0.244367207542564\\
37.1846089736384	0.244367831440555\\
37.1911751540266	0.244368455024696\\
37.1977413112987	0.244369078295201\\
37.2043074454665	0.244369701252279\\
37.2108735565418	0.244370323896142\\
37.2174396445363	0.244370946227\\
37.2240057094617	0.244371568245066\\
37.2305717513299	0.244372189950548\\
37.2371377701525	0.244372811343658\\
37.2437037659412	0.244373432424606\\
37.2502697387078	0.244374053193602\\
37.2568356884638	0.244374673650855\\
37.2634016152212	0.244375293796576\\
37.2699675189914	0.244375913630973\\
37.2765333997863	0.244376533154257\\
37.2830992576174	0.244377152366637\\
37.2896650924964	0.24437777126832\\
37.296230904435	0.244378389859517\\
37.3027966934448	0.244379008140435\\
37.3093624595374	0.244379626111284\\
37.3159282027246	0.244380243772271\\
37.3224939230178	0.244380861123605\\
37.3290596204288	0.244381478165494\\
37.335625294969	0.244382094898144\\
37.3421909466502	0.244382711321765\\
37.3487565754839	0.244383327436563\\
37.3553221814817	0.244383943242745\\
37.3618877646551	0.244384558740519\\
37.3684533250157	0.244385173930091\\
37.3750188625752	0.244385788811669\\
37.3815843773449	0.244386403385458\\
37.3881498693365	0.244387017651665\\
37.3947153385616	0.244387631610496\\
37.4012807850315	0.244388245262157\\
37.4078462087579	0.244388858606855\\
37.4144116097523	0.244389471644794\\
37.4209769880261	0.24439008437618\\
37.4275423435909	0.244390696801219\\
37.4341076764581	0.244391308920115\\
37.4406729866392	0.244391920733074\\
37.4472382741457	0.2443925322403\\
37.4538035389891	0.244393143441998\\
37.4603687811807	0.244393754338373\\
37.4669340007321	0.244394364929629\\
37.4734991976548	0.244394975215969\\
37.48006437196	0.244395585197598\\
37.4866295236593	0.24439619487472\\
37.4931946527641	0.244396804247538\\
37.4997597592857	0.244397413316255\\
37.5063248432356	0.244398022081075\\
37.5128899046252	0.244398630542201\\
37.5194549434659	0.244399238699836\\
37.5260199597689	0.244399846554183\\
37.5325849535458	0.244400454105444\\
37.5391499248079	0.244401061353821\\
37.5457148735664	0.244401668299517\\
37.5522797998329	0.244402274942734\\
37.5588447036185	0.244402881283673\\
37.5654095849347	0.244403487322537\\
37.5719744437928	0.244404093059527\\
37.578539280204	0.244404698494844\\
37.5851040941798	0.24440530362869\\
37.5916688857313	0.244405908461266\\
37.5982336548699	0.244406512992772\\
37.6047984016068	0.24440711722341\\
37.6113631259535	0.244407721153379\\
37.617927827921	0.24440832478288\\
37.6244925075207	0.244408928112114\\
37.6310571647639	0.24440953114128\\
37.6376217996617	0.244410133870579\\
37.6441864122255	0.244410736300209\\
37.6507510024664	0.244411338430371\\
37.6573155703957	0.244411940261265\\
37.6638801160246	0.244412541793088\\
37.6704446393643	0.24441314302604\\
37.677009140426	0.24441374396032\\
37.683573619221	0.244414344596128\\
37.6901380757603	0.24441494493366\\
37.6967025100552	0.244415544973116\\
37.7032669221168	0.244416144714694\\
37.7098313119564	0.244416744158592\\
37.716395679585	0.244417343305008\\
37.7229600250139	0.244417942154139\\
37.7295243482541	0.244418540706183\\
37.7360886493167	0.244419138961338\\
37.742652928213	0.244419736919801\\
37.7492171849541	0.244420334581768\\
37.755781419551	0.244420931947437\\
37.7623456320148	0.244421529017004\\
37.7689098223567	0.244422125790666\\
37.7754739905878	0.24442272226862\\
37.7820381367191	0.244423318451061\\
37.7886022607616	0.244423914338185\\
37.7951663627266	0.24442450993019\\
37.8017304426249	0.244425105227269\\
37.8082945004678	0.244425700229619\\
37.8148585362661	0.244426294937436\\
37.821422550031	0.244426889350914\\
37.8279865417735	0.244427483470249\\
37.8345505115046	0.244428077295635\\
37.8411144592354	0.244428670827268\\
37.8476783849768	0.244429264065342\\
37.8542422887398	0.244429857010051\\
37.8608061705354	0.24443044966159\\
37.8673700303747	0.244431042020152\\
37.8739338682685	0.244431634085932\\
37.8804976842279	0.244432225859124\\
37.8870614782639	0.244432817339921\\
37.8936252503873	0.244433408528517\\
37.9001890006092	0.244433999425105\\
37.9067527289404	0.244434590029878\\
37.9133164353919	0.244435180343029\\
37.9198801199747	0.244435770364751\\
37.9264437826996	0.244436360095237\\
37.9330074235776	0.244436949534679\\
37.9395710426195	0.244437538683269\\
37.9461346398363	0.244438127541201\\
37.9526982152389	0.244438716108665\\
37.9592617688381	0.244439304385853\\
37.9658253006448	0.244439892372959\\
37.9723888106699	0.244440480070172\\
37.9789522989243	0.244441067477684\\
37.9855157654188	0.244441654595687\\
37.9920792101642	0.244442241424372\\
37.9986426331713	0.244442827963929\\
38.0052060344511	0.244443414214549\\
38.0117694140144	0.244444000176424\\
38.0183327718719	0.244444585849743\\
38.0248961080344	0.244445171234697\\
38.0314594225129	0.244445756331475\\
38.038022715318	0.244446341140269\\
38.0445859864605	0.244446925661267\\
38.0511492359513	0.244447509894659\\
38.0577124638011	0.244448093840635\\
38.0642756700207	0.244448677499385\\
38.0708388546208	0.244449260871097\\
38.0774020176122	0.24444984395596\\
38.0839651590057	0.244450426754163\\
38.0905282788119	0.244451009265896\\
38.0970913770416	0.244451591491346\\
38.1036544537056	0.244452173430702\\
38.1102175088145	0.244452755084153\\
38.116780542379	0.244453336451885\\
38.1233435544099	0.244453917534089\\
38.1299065449179	0.24445449833095\\
38.1364695139136	0.244455078842657\\
38.1430324614077	0.244455659069397\\
38.1495953874109	0.244456239011358\\
38.1561582919339	0.244456818668726\\
38.1627211749873	0.24445739804169\\
38.1692840365818	0.244457977130435\\
38.175846876728	0.244458555935148\\
38.1824096954366	0.244459134456016\\
38.1889724927181	0.244459712693225\\
38.1955352685833	0.244460290646962\\
38.2020980230427	0.244460868317413\\
38.2086607561069	0.244461445704763\\
38.2152234677866	0.244462022809199\\
38.2217861580923	0.244462599630905\\
38.2283488270347	0.244463176170068\\
38.2349114746242	0.244463752426873\\
38.2414741008716	0.244464328401505\\
38.2480367057873	0.244464904094149\\
38.2545992893819	0.244465479504989\\
38.261161851666	0.244466054634212\\
38.2677243926501	0.244466629482\\
38.2742869123447	0.244467204048539\\
38.2808494107605	0.244467778334013\\
38.2874118879078	0.244468352338606\\
38.2939743437973	0.244468926062502\\
38.3005367784394	0.244469499505885\\
38.3070991918447	0.244470072668938\\
38.3136615840236	0.244470645551845\\
38.3202239549867	0.244471218154789\\
38.3267863047444	0.244471790477954\\
38.3333486333071	0.244472362521522\\
38.3399109406855	0.244472934285677\\
38.3464732268899	0.244473505770601\\
38.3530354919307	0.244474076976476\\
38.3595977358185	0.244474647903486\\
38.3661599585637	0.244475218551813\\
38.3727221601767	0.244475788921638\\
38.3792843406679	0.244476359013144\\
38.3858465000478	0.244476928826513\\
38.3924086383268	0.244477498361926\\
38.3989707555153	0.244478067619564\\
38.4055328516236	0.24447863659961\\
38.4120949266622	0.244479205302245\\
38.4186569806415	0.244479773727649\\
38.4252190135719	0.244480341876004\\
38.4317810254637	0.24448090974749\\
38.4383430163272	0.244481477342289\\
38.4449049861729	0.24448204466058\\
38.4514669350111	0.244482611702544\\
38.4580288628521	0.244483178468361\\
38.4645907697064	0.244483744958211\\
38.4711526555841	0.244484311172275\\
38.4777145204957	0.244484877110732\\
38.4842763644515	0.244485442773761\\
38.4908381874617	0.244486008161543\\
38.4973999895367	0.244486573274256\\
38.5039617706868	0.244487138112079\\
38.5105235309222	0.244487702675193\\
38.5170852702533	0.244488266963775\\
38.5236469886903	0.244488830978004\\
38.5302086862435	0.24448939471806\\
38.5367703629232	0.244489958184119\\
38.5433320187396	0.244490521376362\\
38.549893653703	0.244491084294966\\
38.5564552678236	0.244491646940108\\
38.5630168611116	0.244492209311968\\
38.5695784335773	0.244492771410722\\
38.576139985231	0.244493333236548\\
38.5827015160827	0.244493894789624\\
38.5892630261428	0.244494456070127\\
38.5958245154215	0.244495017078234\\
38.6023859839288	0.244495577814121\\
38.6089474316751	0.244496138277967\\
38.6155088586705	0.244496698469948\\
38.6220702649252	0.24449725839024\\
38.6286316504493	0.24449781803902\\
38.6351930152531	0.244498377416463\\
38.6417543593466	0.244498936522747\\
38.64831568274	0.244499495358047\\
38.6548769854436	0.244500053922539\\
38.6614382674673	0.244500612216399\\
38.6679995288213	0.244501170239802\\
38.6745607695158	0.244501727992924\\
38.6811219895609	0.244502285475939\\
38.6876831889666	0.244502842689024\\
38.6942443677432	0.244503399632354\\
38.7008055259006	0.244503956306102\\
38.707366663449	0.244504512710444\\
38.7139277803984	0.244505068845554\\
38.720488876759	0.244505624711607\\
38.7270499525408	0.244506180308778\\
38.7336110077538	0.244506735637239\\
38.7401720424082	0.244507290697166\\
38.7467330565139	0.244507845488731\\
38.7532940500811	0.24450840001211\\
38.7598550231197	0.244508954267475\\
38.7664159756397	0.244509508255\\
38.7729769076513	0.244510061974858\\
38.7795378191644	0.244510615427222\\
38.7860987101891	0.244511168612266\\
38.7926595807353	0.244511721530162\\
38.799220430813	0.244512274181083\\
38.8057812604322	0.244512826565201\\
38.812342069603	0.24451337868269\\
38.8189028583353	0.244513930533721\\
38.825463626639	0.244514482118466\\
38.8320243745241	0.244515033437098\\
38.8385851020006	0.244515584489789\\
38.8451458090785	0.244516135276709\\
38.8517064957676	0.244516685798032\\
38.858267162078	0.244517236053928\\
38.8648278080195	0.244517786044569\\
38.8713884336021	0.244518335770125\\
38.8779490388356	0.244518885230769\\
38.8845096237301	0.24451943442667\\
38.8910701882955	0.244519983358\\
38.8976307325415	0.24452053202493\\
38.9041912564781	0.244521080427629\\
38.9107517601152	0.244521628566269\\
38.9173122434628	0.244522176441019\\
38.9238727065305	0.24452272405205\\
38.9304331493284	0.244523271399531\\
38.9369935718663	0.244523818483633\\
38.943553974154	0.244524365304525\\
38.9501143562013	0.244524911862376\\
38.9566747180182	0.244525458157357\\
38.9632350596145	0.244526004189635\\
38.969795381	0.244526549959382\\
38.9763556821844	0.244527095466764\\
38.9829159631777	0.244527640711952\\
38.9894762239896	0.244528185695114\\
38.99603646463	0.244528730416419\\
39.0025966851086	0.244529274876035\\
39.0091568854352	0.24452981907413\\
39.0157170656197	0.244530363010873\\
39.0222772256717	0.244530906686431\\
39.0288373656011	0.244531450100973\\
39.0353974854177	0.244531993254667\\
39.0419575851311	0.244532536147679\\
39.0485176647511	0.244533078780178\\
39.0550777242876	0.244533621152331\\
39.0616377637501	0.244534163264305\\
39.0681977831485	0.244534705116267\\
39.0747577824925	0.244535246708385\\
39.0813177617918	0.244535788040824\\
39.0878777210561	0.244536329113752\\
39.0944376602952	0.244536869927336\\
39.1009975795187	0.244537410481741\\
39.1075574787363	0.244537950777135\\
39.1141173579577	0.244538490813682\\
39.1206772171926	0.24453903059155\\
39.1272370564507	0.244539570110904\\
39.1337968757416	0.244540109371909\\
39.140356675075	0.244540648374733\\
39.1469164544606	0.244541187119539\\
39.153476213908	0.244541725606494\\
39.1600359534269	0.244542263835762\\
39.1665956730268	0.244542801807509\\
39.1731553727175	0.2445433395219\\
39.1797150525086	0.244543876979099\\
39.1862747124096	0.244544414179272\\
39.1928343524303	0.244544951122582\\
39.1993939725801	0.244545487809194\\
39.2059535728687	0.244546024239273\\
39.2125131533058	0.244546560412983\\
39.2190727139008	0.244547096330487\\
39.2256322546633	0.24454763199195\\
39.2321917756031	0.244548167397535\\
39.2387512767295	0.244548702547407\\
39.2453107580522	0.244549237441728\\
39.2518702195808	0.244549772080662\\
39.2584296613247	0.244550306464372\\
39.2649890832936	0.244550840593021\\
39.2715484854969	0.244551374466773\\
39.2781078679443	0.24455190808579\\
39.2846672306452	0.244552441450235\\
39.2912265736091	0.24455297456027\\
39.2977858968456	0.244553507416058\\
39.3043452003642	0.244554040017762\\
39.3109044841744	0.244554572365543\\
39.3174637482856	0.244555104459563\\
39.3240229927074	0.244555636299985\\
39.3305822174493	0.244556167886971\\
39.3371414225207	0.244556699220681\\
39.343700607931	0.244557230301278\\
39.3502597736899	0.244557761128923\\
39.3568189198067	0.244558291703777\\
39.3633780462908	0.244558822026002\\
39.3699371531518	0.244559352095758\\
39.376496240399	0.244559881913207\\
39.383055308042	0.244560411478509\\
39.38961435609	0.244560940791825\\
39.3961733845527	0.244561469853315\\
39.4027323934393	0.24456199866314\\
39.4092913827592	0.24456252722146\\
39.415850352522	0.244563055528435\\
39.4224093027369	0.244563583584226\\
39.4289682334135	0.244564111388991\\
39.4355271445609	0.244564638942892\\
39.4420860361888	0.244565166246087\\
39.4486449083063	0.244565693298736\\
39.4552037609229	0.244566220100998\\
39.461762594048	0.244566746653033\\
39.4683214076909	0.244567272954999\\
39.4748802018609	0.244567799007057\\
39.4814389765674	0.244568324809364\\
39.4879977318198	0.244568850362079\\
39.4945564676273	0.244569375665361\\
39.5011151839993	0.244569900719369\\
39.5076738809452	0.24457042552426\\
39.5142325584741	0.244570950080194\\
39.5207912165955	0.244571474387328\\
39.5273498553187	0.244571998445819\\
39.5339084746529	0.244572522255827\\
39.5404670746074	0.244573045817509\\
39.5470256551915	0.244573569131022\\
39.5535842164145	0.244574092196523\\
39.5601427582856	0.244574615014171\\
39.5667012808142	0.244575137584122\\
39.5732597840095	0.244575659906534\\
39.5798182678807	0.244576181981563\\
39.5863767324371	0.244576703809367\\
39.5929351776879	0.244577225390101\\
39.5994936036424	0.244577746723923\\
39.6060520103098	0.24457826781099\\
39.6126103976994	0.244578788651457\\
39.6191687658202	0.24457930924548\\
39.6257271146816	0.244579829593217\\
39.6322854442928	0.244580349694822\\
39.638843754663	0.244580869550452\\
39.6454020458013	0.244581389160262\\
39.6519603177169	0.244581908524408\\
39.6585185704191	0.244582427643046\\
39.665076803917	0.244582946516331\\
39.6716350182198	0.244583465144418\\
39.6781932133366	0.244583983527462\\
39.6847513892766	0.244584501665618\\
39.691309546049	0.244585019559041\\
39.6978676836629	0.244585537207886\\
39.7044258021274	0.244586054612308\\
39.7109839014518	0.244586571772461\\
39.717541981645	0.244587088688499\\
39.7241000427163	0.244587605360576\\
39.7306580846747	0.244588121788848\\
39.7372161075295	0.244588637973467\\
39.7437741112896	0.244589153914588\\
39.7503320959642	0.244589669612364\\
39.7568900615623	0.244590185066949\\
39.7634480080932	0.244590700278497\\
39.7700059355657	0.244591215247161\\
39.7765638439892	0.244591729973094\\
39.7831217333725	0.244592244456449\\
39.7896796037248	0.244592758697381\\
39.7962374550551	0.24459327269604\\
39.8027952873725	0.244593786452581\\
39.809353100686	0.244594299967155\\
39.8159108950047	0.244594813239916\\
39.8224686703376	0.244595326271016\\
39.8290264266938	0.244595839060607\\
39.8355841640822	0.244596351608841\\
39.842141882512	0.244596863915871\\
39.848699581992	0.244597375981848\\
39.8552572625314	0.244597887806925\\
39.861814924139	0.244598399391252\\
39.868372566824	0.244598910734982\\
39.8749301905953	0.244599421838266\\
39.8814877954619	0.244599932701255\\
39.8880453814328	0.244600443324101\\
39.8946029485169	0.244600953706955\\
39.9011604967232	0.244601463849967\\
39.9077180260607	0.244601973753289\\
39.9142755365383	0.244602483417072\\
39.9208330281651	0.244602992841465\\
39.9273905009498	0.24460350202662\\
39.9339479549016	0.244604010972686\\
39.9405053900292	0.244604519679815\\
39.9470628063417	0.244605028148156\\
39.9536202038479	0.24460553637786\\
39.9601775825569	0.244606044369076\\
39.9667349424774	0.244606552121954\\
39.9732922836184	0.244607059636644\\
39.9798496059888	0.244607566913295\\
39.9864069095975	0.244608073952057\\
39.9929641944534	0.24460858075308\\
39.9995214605653	0.244609087316511\\
40.0060787079422	0.244609593642502\\
40.0126359365929	0.244610099731199\\
40.0191931465262	0.244610605582753\\
40.0257503377511	0.244611111197313\\
40.0323075102765	0.244611616575025\\
40.038864664111	0.244612121716041\\
40.0454217992636	0.244612626620506\\
40.0519789157432	0.244613131288571\\
40.0585360135585	0.244613635720383\\
40.0650930927184	0.24461413991609\\
40.0716501532318	0.24461464387584\\
40.0782071951073	0.244615147599781\\
40.0847642183539	0.244615651088061\\
40.0913212229803	0.244616154340826\\
40.0978782089954	0.244616657358226\\
40.1044351764079	0.244617160140406\\
40.1109921252266	0.244617662687515\\
40.1175490554603	0.244618164999699\\
40.1241059671178	0.244618667077105\\
40.1306628602079	0.244619168919881\\
40.1372197347392	0.244619670528172\\
40.1437765907207	0.244620171902127\\
40.1503334281609	0.24462067304189\\
40.1568902470688	0.24462117394761\\
40.163447047453	0.244621674619431\\
40.1700038293222	0.2446221750575\\
40.1765605926852	0.244622675261964\\
40.1831173375508	0.244623175232967\\
40.1896740639276	0.244623674970657\\
40.1962307718244	0.244624174475179\\
40.2027874612498	0.244624673746678\\
40.2093441322126	0.2446251727853\\
40.2159007847215	0.244625671591191\\
40.2224574187851	0.244626170164495\\
40.2290140344122	0.244626668505358\\
40.2355706316115	0.244627166613925\\
40.2421272103915	0.244627664490341\\
40.2486837707611	0.24462816213475\\
40.2552403127288	0.244628659547297\\
40.2617968363033	0.244629156728128\\
40.2683533414933	0.244629653677386\\
40.2749098283074	0.244630150395216\\
40.2814662967543	0.244630646881762\\
40.2880227468426	0.244631143137168\\
40.2945791785809	0.244631639161579\\
40.3011355919779	0.244632134955138\\
40.3076919870422	0.244632630517989\\
40.3142483637824	0.244633125850275\\
40.3208047222071	0.244633620952141\\
40.3273610623249	0.24463411582373\\
40.3339173841445	0.244634610465184\\
40.3404736876744	0.244635104876649\\
40.3470299729232	0.244635599058265\\
40.3535862398995	0.244636093010178\\
40.3601424886119	0.244636586732529\\
40.3666987190689	0.244637080225461\\
40.3732549312791	0.244637573489117\\
40.3798111252512	0.24463806652364\\
40.3863673009936	0.244638559329172\\
40.3929234585148	0.244639051905855\\
40.3994795978235	0.244639544253832\\
40.4060357189282	0.244640036373244\\
40.4125918218374	0.244640528264235\\
40.4191479065597	0.244641019926945\\
40.4257039731035	0.244641511361517\\
40.4322600214774	0.244642002568092\\
40.43881605169	0.244642493546812\\
40.4453720637496	0.244642984297818\\
40.4519280576649	0.244643474821252\\
40.4584840334443	0.244643965117255\\
40.4650399910963	0.244644455185968\\
40.4715959306294	0.244644945027532\\
40.4781518520522	0.244645434642088\\
40.4847077553729	0.244645924029777\\
40.4912636406002	0.244646413190739\\
40.4978195077425	0.244646902125116\\
40.5043753568083	0.244647390833047\\
40.510931187806	0.244647879314673\\
40.5174870007441	0.244648367570134\\
40.524042795631	0.244648855599571\\
40.5305985724751	0.244649343403124\\
40.537154331285	0.244649830980932\\
40.5437100720689	0.244650318333135\\
40.5502657948354	0.244650805459873\\
40.5568214995929	0.244651292361285\\
40.5633771863497	0.244651779037512\\
40.5699328551144	0.244652265488693\\
40.5764885058952	0.244652751714966\\
40.5830441387006	0.244653237716471\\
40.589599753539	0.244653723493347\\
40.5961553504188	0.244654209045733\\
40.6027109293483	0.244654694373768\\
40.6092664903359	0.244655179477591\\
40.6158220333901	0.24465566435734\\
40.6223775585191	0.244656149013153\\
40.6289330657314	0.24465663344517\\
40.6354885550352	0.244657117653529\\
40.642044026439	0.244657601638367\\
40.648599479951	0.244658085399823\\
40.6551549155797	0.244658568938035\\
40.6617103333333	0.244659052253141\\
40.6682657332202	0.244659535345278\\
40.6748211152487	0.244660018214584\\
40.6813764794272	0.244660500861198\\
40.6879318257639	0.244660983285255\\
40.6944871542671	0.244661465486894\\
40.7010424649452	0.244661947466252\\
40.7075977578065	0.244662429223466\\
40.7141530328592	0.244662910758673\\
40.7207082901117	0.24466339207201\\
40.7272635295721	0.244663873163614\\
40.7338187512489	0.244664354033622\\
40.7403739551503	0.244664834682169\\
40.7469291412845	0.244665315109393\\
40.7534843096598	0.24466579531543\\
40.7600394602844	0.244666275300417\\
40.7665945931667	0.244666755064489\\
40.7731497083148	0.244667234607783\\
40.779704805737	0.244667713930434\\
40.7862598854416	0.244668193032579\\
40.7928149474367	0.244668671914353\\
40.7993699917307	0.244669150575892\\
40.8059250183316	0.244669629017332\\
40.8124800272478	0.244670107238808\\
40.8190350184874	0.244670585240454\\
40.8255899920587	0.244671063022408\\
40.8321449479699	0.244671540584803\\
40.8386998862291	0.244672017927775\\
40.8452548068445	0.244672495051459\\
40.8518097098243	0.244672971955989\\
40.8583645951768	0.2446734486415\\
40.86491946291	0.244673925108127\\
40.8714743130322	0.244674401356005\\
40.8780291455515	0.244674877385267\\
40.8845839604761	0.244675353196049\\
40.8911387578141	0.244675828788484\\
40.8976935375737	0.244676304162706\\
40.904248299763	0.24467677931885\\
40.9108030443902	0.244677254257049\\
40.9173577714634	0.244677728977437\\
40.9239124809908	0.244678203480148\\
40.9304671729804	0.244678677765316\\
40.9370218474403	0.244679151833073\\
40.9435765043788	0.244679625683554\\
40.9501311438038	0.244680099316891\\
40.9566857657236	0.244680572733218\\
40.9632403701461	0.244681045932668\\
40.9697949570796	0.244681518915374\\
40.976349526532	0.244681991681468\\
40.9829040785116	0.244682464231084\\
40.9894586130263	0.244682936564354\\
40.9960131300842	0.244683408681411\\
41.0025676296934	0.244683880582387\\
41.0091221118619	0.244684352267414\\
41.0156765765979	0.244684823736625\\
41.0222310239094	0.244685294990152\\
41.0287854538044	0.244685766028127\\
41.0353398662909	0.244686236850682\\
41.0418942613771	0.244686707457948\\
41.0484486390709	0.244687177850059\\
41.0550029993804	0.244687648027144\\
41.0615573423135	0.244688117989336\\
41.0681116678784	0.244688587736767\\
41.0746659760831	0.244689057269567\\
41.0812202669354	0.244689526587868\\
41.0877745404436	0.2446899956918\\
41.0943287966154	0.244690464581496\\
41.100883035459	0.244690933257086\\
41.1074372569824	0.2446914017187\\
41.1139914611934	0.24469186996647\\
41.1205456481001	0.244692338000526\\
41.1270998177106	0.244692805820999\\
41.1336539700326	0.244693273428019\\
41.1402081050743	0.244693740821717\\
41.1467622228435	0.244694208002222\\
41.1533163233483	0.244694674969665\\
41.1598704065965	0.244695141724177\\
41.1664244725961	0.244695608265886\\
41.1729785213551	0.244696074594923\\
41.1795325528814	0.244696540711417\\
41.1860865671829	0.244697006615498\\
41.1926405642676	0.244697472307297\\
41.1991945441434	0.244697937786941\\
41.2057485068181	0.244698403054561\\
41.2123024522997	0.244698868110285\\
41.2188563805962	0.244699332954244\\
41.2254102917153	0.244699797586565\\
41.231964185665	0.244700262007378\\
41.2385180624533	0.244700726216813\\
41.2450719220879	0.244701190214996\\
41.2516257645768	0.244701654002058\\
41.2581795899278	0.244702117578127\\
41.2647333981489	0.24470258094333\\
41.2712871892478	0.244703044097798\\
41.2778409632325	0.244703507041657\\
41.2843947201108	0.244703969775036\\
41.2909484598905	0.244704432298063\\
41.2975021825795	0.244704894610867\\
41.3040558881857	0.244705356713574\\
41.3106095767169	0.244705818606313\\
41.3171632481809	0.244706280289211\\
41.3237169025855	0.244706741762397\\
41.3302705399386	0.244707203025997\\
41.336824160248	0.244707664080138\\
41.3433777635215	0.244708124924949\\
41.3499313497669	0.244708585560556\\
41.356484918992	0.244709045987087\\
41.3630384712046	0.244709506204668\\
41.3695920064126	0.244709966213426\\
41.3761455246237	0.244710426013489\\
41.3826990258456	0.244710885604982\\
41.3892525100863	0.244711344988033\\
41.3958059773533	0.244711804162768\\
41.4023594276546	0.244712263129313\\
41.4089128609979	0.244712721887796\\
41.4154662773909	0.244713180438341\\
41.4220196768415	0.244713638781075\\
41.4285730593572	0.244714096916124\\
41.435126424946	0.244714554843614\\
41.4416797736155	0.244715012563672\\
41.4482331053736	0.244715470076422\\
41.4547864202278	0.24471592738199\\
41.461339718186	0.244716384480502\\
41.4678929992558	0.244716841372084\\
41.474446263445	0.24471729805686\\
41.4809995107614	0.244717754534956\\
41.4875527412125	0.244718210806498\\
41.4941059548062	0.244718666871609\\
41.5006591515501	0.244719122730416\\
41.5072123314518	0.244719578383043\\
41.5137654945192	0.244720033829615\\
41.5203186407599	0.244720489070257\\
41.5268717701816	0.244720944105093\\
41.5334248827918	0.244721398934248\\
41.5399779785985	0.244721853557845\\
41.546531057609	0.244722307976011\\
41.5530841198313	0.244722762188868\\
41.5596371652728	0.244723216196541\\
41.5661901939413	0.244723669999154\\
41.5727432058444	0.24472412359683\\
41.5792962009898	0.244724576989695\\
41.585849179385	0.244725030177871\\
41.5924021410378	0.244725483161481\\
41.5989550859557	0.244725935940651\\
41.6055080141464	0.244726388515503\\
41.6120609256175	0.244726840886161\\
41.6186138203766	0.244727293052747\\
41.6251666984314	0.244727745015385\\
41.6317195597893	0.244728196774199\\
41.6382724044581	0.24472864832931\\
41.6448252324454	0.244729099680843\\
41.6513780437586	0.24472955082892\\
41.6579308384055	0.244730001773663\\
41.6644836163936	0.244730452515196\\
41.6710363777304	0.244730903053641\\
41.6775891224236	0.244731353389119\\
41.6841418504806	0.244731803521755\\
41.6906945619092	0.244732253451669\\
41.6972472567168	0.244732703178985\\
41.703799934911	0.244733152703823\\
41.7103525964994	0.244733602026307\\
41.7169052414894	0.244734051146558\\
41.7234578698887	0.244734500064698\\
41.7300104817047	0.244734948780848\\
41.7365630769451	0.244735397295131\\
41.7431156556173	0.244735845607667\\
41.7496682177288	0.244736293718578\\
41.7562207632872	0.244736741627987\\
41.7627732923	0.244737189336012\\
41.7693258047747	0.244737636842777\\
41.7758783007188	0.244738084148402\\
41.7824307801398	0.244738531253008\\
41.7889832430452	0.244738978156716\\
41.7955356894425	0.244739424859646\\
41.8020881193392	0.24473987136192\\
41.8086405327427	0.244740317663658\\
41.8151929296606	0.24474076376498\\
41.8217453101002	0.244741209666008\\
41.8282976740692	0.24474165536686\\
41.8348500215749	0.244742100867659\\
41.8414023526247	0.244742546168523\\
41.8479546672263	0.244742991269573\\
41.8545069653869	0.244743436170929\\
41.8610592471141	0.24474388087271\\
41.8676115124153	0.244744325375037\\
41.874163761298	0.244744769678029\\
41.8807159937695	0.244745213781806\\
41.8872682098372	0.244745657686488\\
41.8938204095087	0.244746101392193\\
41.9003725927914	0.244746544899041\\
41.9069247596925	0.244746988207152\\
41.9134769102197	0.244747431316644\\
41.9200290443802	0.244747874227637\\
41.9265811621814	0.24474831694025\\
41.9331332636309	0.244748759454601\\
41.9396853487358	0.24474920177081\\
41.9462374175037	0.244749643888995\\
41.952789469942	0.244750085809274\\
41.9593415060579	0.244750527531767\\
41.9658935258589	0.244750969056591\\
41.9724455293523	0.244751410383866\\
41.9789975165456	0.244751851513709\\
41.985549487446	0.244752292446239\\
41.992101442061	0.244752733181573\\
41.9986533803978	0.24475317371983\\
42.0052053024639	0.244753614061127\\
42.0117572082665	0.244754054205583\\
42.018309097813	0.244754494153316\\
42.0248609711108	0.244754933904442\\
42.0314128281671	0.244755373459079\\
42.0379646689893	0.244755812817346\\
42.0445164935847	0.244756251979359\\
42.0510683019607	0.244756690945235\\
42.0576200941245	0.244757129715092\\
42.0641718700835	0.244757568289047\\
42.0707236298449	0.244758006667218\\
42.077275373416	0.24475844484972\\
42.0838271008042	0.244758882836671\\
42.0903788120167	0.244759320628187\\
42.0969305070609	0.244759758224386\\
42.1034821859439	0.244760195625384\\
42.1100338486732	0.244760632831297\\
42.1165854952558	0.244761069842242\\
42.1231371256992	0.244761506658336\\
42.1296887400106	0.244761943279693\\
42.1362403381972	0.244762379706432\\
42.1427919202663	0.244762815938667\\
42.1493434862251	0.244763251976514\\
42.155895036081	0.244763687820091\\
42.162446569841	0.244764123469511\\
42.1689980875125	0.244764558924892\\
42.1755495891028	0.244764994186349\\
42.1821010746189	0.244765429253997\\
42.1886525440682	0.244765864127952\\
42.1952039974579	0.244766298808329\\
42.2017554347951	0.244766733295244\\
42.2083068560872	0.244767167588812\\
42.2148582613412	0.244767601689147\\
42.2214096505645	0.244768035596365\\
42.2279610237642	0.244768469310581\\
42.2345123809474	0.24476890283191\\
42.2410637221215	0.244769336160467\\
42.2476150472935	0.244769769296366\\
42.2541663564707	0.244770202239721\\
42.2607176496602	0.244770634990648\\
42.2672689268692	0.244771067549261\\
42.2738201881048	0.244771499915674\\
42.2803714333743	0.244771932090002\\
42.2869226626848	0.244772364072358\\
42.2934738760434	0.244772795862857\\
42.3000250734573	0.244773227461613\\
42.3065762549337	0.24477365886874\\
42.3131274204796	0.244774090084351\\
42.3196785701023	0.24477452110856\\
42.3262297038088	0.244774951941482\\
42.3327808216063	0.244775382583229\\
42.3393319235019	0.244775813033915\\
42.3458830095027	0.244776243293654\\
42.3524340796159	0.244776673362558\\
42.3589851338485	0.244777103240742\\
42.3655361722077	0.244777532928318\\
42.3720871947006	0.2447779624254\\
42.3786382013342	0.2447783917321\\
42.3851891921156	0.244778820848531\\
42.3917401670521	0.244779249774806\\
42.3982911261505	0.244779678511038\\
42.4048420694181	0.24478010705734\\
42.4113929968619	0.244780535413824\\
42.4179439084889	0.244780963580603\\
42.4244948043063	0.244781391557789\\
42.431045684321	0.244781819345494\\
42.4375965485403	0.244782246943831\\
42.444147396971	0.244782674352912\\
42.4506982296204	0.244783101572848\\
42.4572490464953	0.244783528603753\\
42.4637998476029	0.244783955445737\\
42.4703506329503	0.244784382098914\\
42.4769014025443	0.244784808563393\\
42.4834521563921	0.244785234839288\\
42.4900028945008	0.244785660926709\\
42.4965536168772	0.244786086825768\\
42.5031043235285	0.244786512536577\\
42.5096550144616	0.244786938059247\\
42.5162056896836	0.244787363393889\\
42.5227563492015	0.244787788540615\\
42.5293069930223	0.244788213499535\\
42.5358576211529	0.24478863827076\\
42.5424082336003	0.244789062854402\\
42.5489588303717	0.244789487250571\\
42.5555094114738	0.244789911459378\\
42.5620599769138	0.244790335480933\\
42.5686105266986	0.244790759315348\\
42.5751610608351	0.244791182962733\\
42.5817115793304	0.244791606423197\\
42.5882620821914	0.244792029696853\\
42.594812569425	0.244792452783809\\
42.6013630410382	0.244792875684176\\
42.6079134970381	0.244793298398064\\
42.6144639374314	0.244793720925583\\
42.6210143622252	0.244794143266844\\
42.6275647714264	0.244794565421955\\
42.6341151650419	0.244794987391027\\
42.6406655430787	0.24479540917417\\
42.6472159055436	0.244795830771492\\
42.6537662524437	0.244796252183104\\
42.6603165837859	0.244796673409115\\
42.6668668995769	0.244797094449635\\
42.6734171998239	0.244797515304772\\
42.6799674845335	0.244797935974636\\
42.6865177537129	0.244798356459336\\
42.6930680073688	0.244798776758981\\
42.6996182455082	0.24479919687368\\
42.7061684681379	0.244799616803542\\
42.7127186752649	0.244800036548676\\
42.719268866896	0.24480045610919\\
42.725819043038	0.244800875485194\\
42.732369203698	0.244801294676795\\
42.7389193488827	0.244801713684102\\
42.7454694785989	0.244802132507224\\
42.7520195928537	0.244802551146269\\
42.7585696916537	0.244802969601345\\
42.765119775006	0.24480338787256\\
42.7716698429172	0.244803805960023\\
42.7782198953943	0.244804223863841\\
42.7847699324441	0.244804641584123\\
42.7913199540735	0.244805059120976\\
42.7978699602893	0.244805476474507\\
42.8044199510982	0.244805893644826\\
42.8109699265072	0.244806310632039\\
42.817519886523	0.244806727436253\\
42.8240698311525	0.244807144057577\\
42.8306197604025	0.244807560496118\\
42.8371696742798	0.244807976751982\\
42.8437195727911	0.244808392825278\\
42.8502694559434	0.244808808716112\\
42.8568193237433	0.244809224424591\\
42.8633691761978	0.244809639950823\\
42.8699190133135	0.244810055294913\\
42.8764688350972	0.24481047045697\\
42.8830186415558	0.2448108854371\\
42.889568432696	0.244811300235409\\
42.8961182085246	0.244811714852005\\
42.9026679690483	0.244812129286993\\
42.909217714274	0.244812543540479\\
42.9157674442083	0.244812957612572\\
42.922317158858	0.244813371503375\\
42.92886685823	0.244813785212997\\
42.9354165423308	0.244814198741543\\
42.9419662111674	0.244814612089118\\
42.9485158647463	0.24481502525583\\
42.9550655030744	0.244815438241783\\
42.9616151261584	0.244815851047084\\
42.968164734005	0.244816263671838\\
42.974714326621	0.244816676116152\\
42.981263904013	0.244817088380129\\
42.9878134661877	0.244817500463877\\
42.994363013152	0.244817912367501\\
43.0009125449124	0.244818324091105\\
43.0074620614758	0.244818735634796\\
43.0140115628487	0.244819146998677\\
43.020561049038	0.244819558182856\\
43.0271105200502	0.244819969187435\\
43.0336599758921	0.244820380012521\\
43.0402094165703	0.244820790658218\\
43.0467588420916	0.244821201124631\\
43.0533082524626	0.244821611411865\\
43.0598576476901	0.244822021520025\\
43.0664070277805	0.244822431449214\\
43.0729563927407	0.244822841199538\\
43.0795057425773	0.244823250771101\\
43.0860550772969	0.244823660164008\\
43.0926043969062	0.244824069378361\\
43.0991537014118	0.244824478414267\\
43.1057029908205	0.244824887271828\\
43.1122522651387	0.244825295951149\\
43.1188015243733	0.244825704452334\\
43.1253507685307	0.244826112775487\\
43.1318999976177	0.244826520920711\\
43.1384492116408	0.244826928888111\\
43.1449984106067	0.244827336677789\\
43.151547594522	0.24482774428985\\
43.1580967633933	0.244828151724397\\
43.1646459172272	0.244828558981533\\
43.1711950560303	0.244828966061362\\
43.1777441798093	0.244829372963987\\
43.1842932885707	0.244829779689512\\
43.1908423823211	0.244830186238039\\
43.1973914610671	0.244830592609671\\
43.2039405248153	0.244830998804512\\
43.2104895735723	0.244831404822664\\
43.2170386073447	0.24483181066423\\
43.223587626139	0.244832216329313\\
43.2301366299618	0.244832621818016\\
43.2366856188197	0.244833027130441\\
43.2432345927192	0.24483343226669\\
43.2497835516669	0.244833837226867\\
43.2563324956693	0.244834242011073\\
43.2628814247331	0.244834646619411\\
43.2694303388647	0.244835051051984\\
43.2759792380707	0.244835455308892\\
43.2825281223576	0.244835859390239\\
43.289076991732	0.244836263296126\\
43.2956258462005	0.244836667026655\\
43.3021746857694	0.244837070581928\\
43.3087235104454	0.244837473962048\\
43.315272320235	0.244837877167114\\
43.3218211151447	0.24483828019723\\
43.328369895181	0.244838683052497\\
43.3349186603504	0.244839085733016\\
43.3414674106594	0.244839488238889\\
43.3480161461146	0.244839890570217\\
43.3545648667224	0.244840292727101\\
43.3611135724893	0.244840694709643\\
43.3676622634218	0.244841096517943\\
43.3742109395264	0.244841498152102\\
43.3807596008096	0.244841899612223\\
43.3873082472779	0.244842300898405\\
43.3938568789377	0.244842702010749\\
43.4004054957955	0.244843102949356\\
43.4069540978578	0.244843503714327\\
43.413502685131	0.244843904305763\\
43.4200512576216	0.244844304723763\\
43.4265998153361	0.244844704968428\\
43.4331483582808	0.24484510503986\\
43.4396968864624	0.244845504938157\\
43.4462453998871	0.244845904663421\\
43.4527938985615	0.244846304215751\\
43.459342382492	0.244846703595247\\
43.465890851685	0.24484710280201\\
43.4724393061469	0.24484750183614\\
43.4789877458842	0.244847900697736\\
43.4855361709033	0.244848299386898\\
43.4920845812106	0.244848697903726\\
43.4986329768125	0.24484909624832\\
43.5051813577155	0.244849494420779\\
43.5117297239259	0.244849892421203\\
43.5182780754502	0.24485029024969\\
43.5248264122947	0.244850687906341\\
43.5313747344658	0.244851085391255\\
43.53792304197	0.244851482704531\\
43.5444713348137	0.244851879846268\\
43.5510196130031	0.244852276816565\\
43.5575678765448	0.244852673615522\\
43.564116125445	0.244853070243237\\
43.5706643597101	0.244853466699809\\
43.5772125793466	0.244853862985336\\
43.5837607843608	0.244854259099918\\
43.590308974759	0.244854655043654\\
43.5968571505476	0.244855050816641\\
43.603405311733	0.244855446418979\\
43.6099534583214	0.244855841850765\\
43.6165015903194	0.244856237112098\\
43.6230497077331	0.244856632203077\\
43.629597810569	0.2448570271238\\
43.6361458988334	0.244857421874364\\
43.6426939725326	0.244857816454868\\
43.6492420316729	0.24485821086541\\
43.6557900762607	0.244858605106088\\
43.6623381063023	0.244858999177\\
43.668886121804	0.244859393078243\\
43.6754341227721	0.244859786809915\\
43.681982109213	0.244860180372114\\
43.6885300811329	0.244860573764938\\
43.6950780385381	0.244860966988484\\
43.701625981435	0.244861360042849\\
43.7081739098298	0.24486175292813\\
43.7147218237289	0.244862145644426\\
43.7212697231385	0.244862538191834\\
43.7278176080649	0.24486293057045\\
43.7343654785144	0.244863322780371\\
43.7409133344933	0.244863714821695\\
43.7474611760078	0.244864106694519\\
43.7540090030643	0.244864498398939\\
43.7605568156689	0.244864889935053\\
43.7671046138279	0.244865281302956\\
43.7736523975477	0.244865672502747\\
43.7802001668344	0.24486606353452\\
43.7867479216944	0.244866454398374\\
43.7932956621338	0.244866845094403\\
43.7998433881589	0.244867235622706\\
43.806391099776	0.244867625983377\\
43.8129387969912	0.244868016176514\\
43.8194864798109	0.244868406202213\\
43.8260341482413	0.244868796060569\\
43.8325818022885	0.244869185751678\\
43.8391294419588	0.244869575275638\\
43.8456770672585	0.244869964632542\\
43.8522246781937	0.244870353822489\\
43.8587722747706	0.244870742845572\\
43.8653198569955	0.244871131701889\\
43.8718674248746	0.244871520391534\\
43.878414978414	0.244871908914603\\
43.8849625176201	0.244872297271191\\
43.8915100424988	0.244872685461395\\
43.8980575530566	0.244873073485309\\
43.9046050492995	0.244873461343028\\
43.9111525312337	0.244873849034649\\
43.9176999988654	0.244874236560266\\
43.9242474522008	0.244874623919974\\
43.9307948912461	0.244875011113868\\
43.9373423160073	0.244875398142043\\
43.9438897264908	0.244875785004594\\
43.9504371227027	0.244876171701616\\
43.956984504649	0.244876558233204\\
43.963531872336	0.244876944599451\\
43.9700792257699	0.244877330800454\\
43.9766265649567	0.244877716836306\\
43.9831738899026	0.244878102707102\\
43.9897212006138	0.244878488412936\\
43.9962684970964	0.244878873953903\\
44.0028157793565	0.244879259330097\\
44.0093630474003	0.244879644541612\\
44.0159103012339	0.244880029588542\\
44.0224575408633	0.244880414470982\\
44.0290047662948	0.244880799189025\\
44.0355519775345	0.244881183742765\\
44.0420991745884	0.244881568132296\\
44.0486463574627	0.244881952357712\\
44.0551935261634	0.244882336419107\\
44.0617406806967	0.244882720316574\\
44.0682878210688	0.244883104050207\\
44.0748349472855	0.2448834876201\\
44.0813820593532	0.244883871026345\\
44.0879291572778	0.244884254269037\\
44.0944762410654	0.244884637348268\\
44.1010233107221	0.244885020264132\\
44.1075703662541	0.244885403016723\\
44.1141174076673	0.244885785606133\\
44.1206644349678	0.244886168032455\\
44.1272114481618	0.244886550295782\\
44.1337584472552	0.244886932396208\\
44.1403054322542	0.244887314333824\\
44.1468524031647	0.244887696108725\\
44.1533993599929	0.244888077721002\\
44.1599463027448	0.244888459170749\\
44.1664932314264	0.244888840458057\\
44.1730401460438	0.24488922158302\\
44.179587046603	0.24488960254573\\
44.1861339331101	0.244889983346279\\
44.1926808055711	0.244890363984759\\
44.1992276639919	0.244890744461264\\
44.2057745083787	0.244891124775885\\
44.2123213387375	0.244891504928714\\
44.2188681550743	0.244891884919843\\
44.2254149573951	0.244892264749365\\
44.2319617457058	0.244892644417371\\
44.2385085200126	0.244893023923953\\
44.2450552803214	0.244893403269203\\
44.2516020266383	0.244893782453213\\
44.2581487589691	0.244894161476074\\
44.26469547732	0.244894540337878\\
44.2712421816969	0.244894919038717\\
44.2777888721057	0.244895297578682\\
44.2843355485526	0.244895675957864\\
44.2908822110434	0.244896054176355\\
44.2974288595841	0.244896432234246\\
44.3039754941808	0.244896810131628\\
44.3105221148393	0.244897187868593\\
44.3170687215657	0.244897565445231\\
44.3236153143659	0.244897942861634\\
44.3301618932458	0.244898320117893\\
44.3367084582115	0.244898697214098\\
44.3432550092689	0.24489907415034\\
44.3498015464239	0.24489945092671\\
44.3563480696824	0.2448998275433\\
44.3628945790505	0.244900204000198\\
44.3694410745341	0.244900580297496\\
44.375987556139	0.244900956435285\\
44.3825340238713	0.244901332413655\\
44.3890804777368	0.244901708232696\\
44.3956269177415	0.244902083892499\\
44.4021733438913	0.244902459393154\\
44.4087197561921	0.244902834734751\\
44.4152661546499	0.24490320991738\\
44.4218125392705	0.244903584941131\\
44.4283589100599	0.244903959806094\\
44.434905267024	0.24490433451236\\
44.4414516101686	0.244904709060018\\
44.4479979394997	0.244905083449157\\
44.4545442550231	0.244905457679868\\
44.4610905567448	0.244905831752241\\
44.4676368446707	0.244906205666364\\
44.4741831188066	0.244906579422328\\
44.4807293791584	0.244906953020222\\
44.4872756257319	0.244907326460135\\
44.4938218585332	0.244907699742156\\
44.500368077568	0.244908072866376\\
44.5069142828422	0.244908445832883\\
44.5134604743616	0.244908818641767\\
44.5200066521322	0.244909191293116\\
44.5265528161598	0.24490956378702\\
44.5330989664502	0.244909936123567\\
44.5396451030094	0.244910308302846\\
44.5461912258431	0.244910680324947\\
44.5527373349571	0.244911052189958\\
44.5592834303574	0.244911423897968\\
44.5658295120498	0.244911795449066\\
44.5723755800401	0.244912166843339\\
44.5789216343341	0.244912538080878\\
44.5854676749377	0.244912909161769\\
44.5920137018567	0.244913280086102\\
44.5985597150969	0.244913650853965\\
44.6051057146641	0.244914021465446\\
44.6116517005642	0.244914391920633\\
44.6181976728029	0.244914762219615\\
44.6247436313861	0.24491513236248\\
44.6312895763196	0.244915502349315\\
44.6378355076092	0.244915872180209\\
44.6443814252606	0.24491624185525\\
44.6509273292797	0.244916611374526\\
44.6574732196723	0.244916980738123\\
44.6640190964441	0.244917349946131\\
44.6705649596009	0.244917718998636\\
44.6771108091486	0.244918087895726\\
44.6836566450929	0.24491845663749\\
44.6902024674395	0.244918825224013\\
44.6967482761943	0.244919193655385\\
44.703294071363	0.244919561931691\\
44.7098398529514	0.24491993005302\\
44.7163856209652	0.244920298019458\\
44.7229313754102	0.244920665831094\\
44.7294771162922	0.244921033488013\\
44.7360228436169	0.244921400990303\\
44.7425685573901	0.244921768338051\\
44.7491142576174	0.244922135531344\\
44.7556599443047	0.244922502570268\\
44.7622056174577	0.244922869454912\\
44.7687512770821	0.24492323618536\\
44.7752969231837	0.2449236027617\\
44.7818425557681	0.244923969184019\\
44.7883881748412	0.244924335452404\\
44.7949337804086	0.244924701566939\\
44.8014793724761	0.244925067527713\\
44.8080249510493	0.244925433334812\\
44.814570516134	0.244925798988321\\
44.821116067736	0.244926164488327\\
44.8276616058608	0.244926529834917\\
44.8342071305142	0.244926895028176\\
44.840752641702	0.24492726006819\\
44.8472981394297	0.244927624955046\\
44.8538436237032	0.244927989688829\\
44.860389094528	0.244928354269625\\
44.8669345519099	0.244928718697521\\
44.8734799958546	0.244929082972601\\
44.8800254263677	0.244929447094953\\
44.886570843455	0.24492981106466\\
44.893116247122	0.244930174881809\\
44.8996616373745	0.244930538546486\\
44.9062070142182	0.244930902058776\\
44.9127523776586	0.244931265418764\\
44.9192977277016	0.244931628626535\\
44.9258430643526	0.244931991682176\\
44.9323883876174	0.24493235458577\\
44.9389336975017	0.244932717337404\\
44.945478994011	0.244933079937163\\
44.952024277151	0.24493344238513\\
44.9585695469275	0.244933804681393\\
44.9651148033459	0.244934166826034\\
44.971660046412	0.24493452881914\\
44.9782052761313	0.244934890660796\\
44.9847504925096	0.244935252351084\\
44.9912956955524	0.244935613890092\\
44.9978408852653	0.244935975277903\\
45.004386061654	0.244936336514601\\
45.0109312247242	0.244936697600272\\
45.0174763744813	0.244937058534999\\
45.024021510931	0.244937419318868\\
45.030566634079	0.244937779951962\\
45.0371117439308	0.244938140434366\\
45.0436568404921	0.244938500766163\\
45.0502019237684	0.244938860947439\\
45.0567469937653	0.244939220978277\\
45.0632920504884	0.244939580858761\\
45.0698370939433	0.244939940588976\\
45.0763821241357	0.244940300169005\\
45.082927141071	0.244940659598931\\
45.0894721447548	0.24494101887884\\
45.0960171351928	0.244941378008814\\
45.1025621123905	0.244941736988938\\
45.1091070763535	0.244942095819294\\
45.1156520270873	0.244942454499967\\
45.1221969645975	0.24494281303104\\
45.1287418888897	0.244943171412596\\
45.1352867999694	0.24494352964472\\
45.1418316978422	0.244943887727493\\
45.1483765825136	0.244944245661\\
45.1549214539892	0.244944603445324\\
45.1614663122745	0.244944961080547\\
45.1680111573751	0.244945318566754\\
45.1745559892965	0.244945675904026\\
45.1811008080442	0.244946033092448\\
45.1876456136238	0.244946390132101\\
45.1941904060408	0.244946747023069\\
45.2007351853007	0.244947103765435\\
45.207279951409	0.244947460359282\\
45.2138247043713	0.244947816804691\\
45.2203694441931	0.244948173101746\\
45.2269141708799	0.24494852925053\\
45.2334588844373	0.244948885251124\\
45.2400035848706	0.244949241103612\\
45.2465482721854	0.244949596808075\\
45.2530929463873	0.244949952364597\\
45.2596376074817	0.244950307773259\\
45.2661822554741	0.244950663034143\\
45.27272689037	0.244951018147333\\
45.2792715121749	0.244951373112909\\
45.2858161208943	0.244951727930954\\
45.2923607165337	0.244952082601551\\
45.2989052990985	0.24495243712478\\
45.3054498685942	0.244952791500725\\
45.3119944250263	0.244953145729466\\
45.3185389684003	0.244953499811085\\
45.3250834987216	0.244953853745665\\
45.3316280159957	0.244954207533286\\
45.3381725202281	0.244954561174031\\
45.3447170114241	0.244954914667982\\
45.3512614895894	0.244955268015218\\
45.3578059547293	0.244955621215823\\
45.3643504068492	0.244955974269877\\
45.3708948459547	0.244956327177462\\
45.3774392720511	0.244956679938659\\
45.383983685144	0.244957032553549\\
45.3905280852387	0.244957385022213\\
45.3970724723406	0.244957737344733\\
45.4036168464552	0.24495808952119\\
45.410161207588	0.244958441551664\\
45.4167055557444	0.244958793436236\\
45.4232498909297	0.244959145174988\\
45.4297942131494	0.244959496767999\\
45.4363385224089	0.244959848215352\\
45.4428828187136	0.244960199517127\\
45.449427102069	0.244960550673403\\
45.4559713724804	0.244960901684263\\
45.4625156299533	0.244961252549785\\
45.469059874493	0.244961603270052\\
45.475604106105	0.244961953845143\\
45.4821483247945	0.244962304275139\\
45.4886925305671	0.24496265456012\\
45.4952367234282	0.244963004700166\\
45.501780903383	0.244963354695357\\
45.508325070437	0.244963704545775\\
45.5148692245956	0.244964054251497\\
45.5214133658641	0.244964403812606\\
45.5279574942479	0.24496475322918\\
45.5345016097524	0.2449651025013\\
45.5410457123829	0.244965451629046\\
45.5475898021449	0.244965800612497\\
45.5541338790437	0.244966149451733\\
45.5606779430845	0.244966498146834\\
45.5672219942729	0.244966846697879\\
45.5737660326141	0.244967195104949\\
45.5803100581135	0.244967543368122\\
45.5868540707764	0.244967891487478\\
45.5933980706082	0.244968239463097\\
45.5999420576143	0.244968587295057\\
45.6064860317999	0.244968934983439\\
45.6130299931704	0.244969282528321\\
45.6195739417311	0.244969629929783\\
45.6261178774873	0.244969977187904\\
45.6326618004445	0.244970324302763\\
45.6392057106078	0.244970671274439\\
45.6457496079827	0.244971018103011\\
45.6522934925744	0.244971364788558\\
45.6588373643882	0.244971711331159\\
45.6653812234295	0.244972057730892\\
45.6719250697036	0.244972403987837\\
45.6784689032158	0.244972750102072\\
45.6850127239713	0.244973096073675\\
45.6915565319755	0.244973441902726\\
45.6981003272337	0.244973787589304\\
45.7046441097512	0.244974133133485\\
45.7111878795331	0.24497447853535\\
45.717731636585	0.244974823794976\\
45.7242753809119	0.244975168912441\\
45.7308191125193	0.244975513887825\\
45.7373628314124	0.244975858721204\\
45.7439065375964	0.244976203412658\\
45.7504502310766	0.244976547962265\\
45.7569939118584	0.244976892370102\\
45.7635375799469	0.244977236636248\\
45.7700812353475	0.24497758076078\\
45.7766248780653	0.244977924743776\\
45.7831685081057	0.244978268585315\\
45.7897121254739	0.244978612285474\\
45.7962557301752	0.244978955844331\\
45.8027993222147	0.244979299261963\\
45.8093429015979	0.244979642538449\\
45.8158864683298	0.244979985673865\\
45.8224300224157	0.24498032866829\\
45.828973563861	0.2449806715218\\
45.8355170926707	0.244981014234473\\
45.8420606088502	0.244981356806388\\
45.8486041124046	0.24498169923762\\
45.8551476033393	0.244982041528247\\
45.8616910816593	0.244982383678346\\
45.86823454737	0.244982725687995\\
45.8747780004766	0.244983067557271\\
45.8813214409842	0.24498340928625\\
45.8878648688981	0.24498375087501\\
45.8944082842235	0.244984092323628\\
45.9009516869655	0.24498443363218\\
45.9074950771295	0.244984774800744\\
45.9140384547206	0.244985115829395\\
45.920581819744	0.244985456718212\\
45.9271251722048	0.24498579746727\\
45.9336685121083	0.244986138076647\\
45.9402118394597	0.244986478546418\\
45.9467551542642	0.24498681887666\\
45.9532984565269	0.244987159067451\\
45.959841746253	0.244987499118865\\
45.9663850234477	0.24498783903098\\
45.9729282881161	0.244988178803872\\
45.9794715402635	0.244988518437617\\
45.986014779895	0.244988857932292\\
45.9925580070158	0.244989197287972\\
45.999101221631	0.244989536504733\\
46.0056444237458	0.244989875582653\\
46.0121876133653	0.244990214521806\\
46.0187307904948	0.244990553322268\\
46.0252739551393	0.244990891984116\\
46.031817107304	0.244991230507426\\
46.038360246994	0.244991568892272\\
46.0449033742146	0.244991907138732\\
46.0514464889707	0.24499224524688\\
46.0579895912677	0.244992583216792\\
46.0645326811105	0.244992921048545\\
46.0710757585044	0.244993258742212\\
46.0776188234544	0.244993596297871\\
46.0841618759657	0.244993933715595\\
46.0907049160434	0.244994270995461\\
46.0972479436927	0.244994608137545\\
46.1037909589186	0.24499494514192\\
46.1103339617262	0.244995282008663\\
46.1168769521208	0.244995618737849\\
46.1234199301073	0.244995955329552\\
46.1299628956909	0.244996291783848\\
46.1365058488766	0.244996628100812\\
46.1430487896697	0.244996964280519\\
46.1495917180752	0.244997300323043\\
46.1561346340981	0.24499763622846\\
46.1626775377436	0.244997971996845\\
46.1692204290168	0.244998307628271\\
46.1757633079227	0.244998643122815\\
46.1823061744664	0.24499897848055\\
46.1888490286531	0.244999313701551\\
46.1953918704877	0.244999648785893\\
46.2019346999754	0.24499998373365\\
46.2084775171213	0.245000318544897\\
46.2150203219303	0.245000653219708\\
46.2215631144076	0.245000987758157\\
46.2281058945583	0.245001322160319\\
46.2346486623873	0.245001656426268\\
46.2411914178998	0.245001990556078\\
46.2477341611008	0.245002324549823\\
46.2542768919954	0.245002658407578\\
46.2608196105886	0.245002992129416\\
46.2673623168854	0.245003325715412\\
46.273905010891	0.245003659165639\\
46.2804476926103	0.245003992480171\\
46.2869903620484	0.245004325659083\\
46.2935330192103	0.245004658702447\\
46.300075664101	0.245004991610338\\
46.3066182967257	0.24500532438283\\
46.3131609170892	0.245005657019996\\
46.3197035251967	0.245005989521909\\
46.3262461210531	0.245006321888643\\
46.3327887046636	0.245006654120273\\
46.339331276033	0.24500698621687\\
46.3458738351664	0.245007318178509\\
46.3524163820688	0.245007650005262\\
46.3589589167453	0.245007981697204\\
46.3655014392008	0.245008313254407\\
46.3720439494403	0.245008644676945\\
46.3785864474689	0.245008975964891\\
46.3851289332915	0.245009307118317\\
46.3916714069132	0.245009638137297\\
46.3982138683388	0.245009969021905\\
46.4047563175735	0.245010299772212\\
46.4112987546222	0.245010630388291\\
46.4178411794898	0.245010960870217\\
46.4243835921814	0.24501129121806\\
46.430925992702	0.245011621431895\\
46.4374683810564	0.245011951511794\\
46.4440107572498	0.245012281457829\\
46.450553121287	0.245012611270073\\
46.457095473173	0.245012940948599\\
46.4636378129129	0.245013270493478\\
46.4701801405114	0.245013599904785\\
46.4767224559737	0.24501392918259\\
46.4832647593047	0.245014258326967\\
46.4898070505092	0.245014587337987\\
46.4963493295924	0.245014916215724\\
46.5028915965591	0.245015244960248\\
46.5094338514142	0.245015573571633\\
46.5159760941627	0.24501590204995\\
46.5225183248096	0.245016230395271\\
46.5290605433598	0.245016558607669\\
46.5356027498182	0.245016886687215\\
46.5421449441898	0.245017214633981\\
46.5486871264794	0.24501754244804\\
46.555229296692	0.245017870129463\\
46.5617714548326	0.245018197678321\\
46.5683136009061	0.245018525094687\\
46.5748557349173	0.245018852378632\\
46.5813978568712	0.245019179530228\\
46.5879399667727	0.245019506549546\\
46.5944820646267	0.245019833436657\\
46.6010241504382	0.245020160191635\\
46.607566224212	0.245020486814549\\
46.614108285953	0.245020813305471\\
46.6206503356661	0.245021139664472\\
46.6271923733563	0.245021465891624\\
46.6337343990284	0.245021791986999\\
46.6402764126873	0.245022117950666\\
46.6468184143379	0.245022443782698\\
46.6533604039851	0.245022769483165\\
46.6599023816337	0.245023095052138\\
46.6664443472888	0.245023420489689\\
46.672986300955	0.245023745795888\\
46.6795282426374	0.245024070970807\\
46.6860701723407	0.245024396014515\\
46.69261209007	0.245024720927085\\
46.6991539958299	0.245025045708586\\
46.7056958896254	0.245025370359089\\
46.7122377714614	0.245025694878666\\
46.7187796413427	0.245026019267386\\
46.7253214992741	0.24502634352532\\
46.7318633452606	0.245026667652539\\
46.7384051793069	0.245026991649113\\
46.744947001418	0.245027315515113\\
46.7514888115986	0.245027639250608\\
46.7580306098537	0.24502796285567\\
46.764572396188	0.245028286330369\\
46.7711141706064	0.245028609674774\\
46.7776559331137	0.245028932888956\\
46.7841976837148	0.245029255972985\\
46.7907394224145	0.245029578926932\\
46.7972811492176	0.245029901750865\\
46.803822864129	0.245030224444856\\
46.8103645671535	0.245030547008974\\
46.8169062582958	0.245030869443289\\
46.8234479375609	0.245031191747871\\
46.8299896049534	0.24503151392279\\
46.8365312604784	0.245031835968115\\
46.8430729041404	0.245032157883916\\
46.8496145359445	0.245032479670263\\
46.8561561558953	0.245032801327225\\
46.8626977639976	0.245033122854873\\
46.8692393602564	0.245033444253275\\
46.8757809446762	0.245033765522501\\
46.8823225172621	0.24503408666262\\
46.8888640780187	0.245034407673702\\
46.8954056269508	0.245034728555816\\
46.9019471640632	0.245035049309032\\
46.9084886893608	0.245035369933419\\
46.9150302028482	0.245035690429045\\
46.9215717045304	0.24503601079598\\
46.9281131944119	0.245036331034293\\
46.9346546724977	0.245036651144054\\
46.9411961387925	0.245036971125331\\
46.947737593301	0.245037290978192\\
46.954279036028	0.245037610702708\\
46.9608204669784	0.245037930298947\\
46.9673618861567	0.245038249766977\\
46.9739032935679	0.245038569106868\\
46.9804446892167	0.245038888318688\\
46.9869860731077	0.245039207402507\\
46.9935274452459	0.245039526358391\\
47.0000688056358	0.245039845186411\\
47.0066101542823	0.245040163886634\\
47.0131514911901	0.24504048245913\\
47.019692816364	0.245040800903966\\
47.0262341298086	0.245041119221212\\
47.0327754315287	0.245041437410934\\
47.0393167215291	0.245041755473203\\
47.0458579998144	0.245042073408085\\
47.0523992663895	0.24504239121565\\
47.058940521259	0.245042708895965\\
47.0654817644276	0.245043026449098\\
47.0720229959001	0.245043343875119\\
47.0785642156812	0.245043661174093\\
47.0851054237756	0.245043978346091\\
47.091646620188	0.245044295391179\\
47.0981878049232	0.245044612309425\\
47.1047289779857	0.245044929100898\\
47.1112701393804	0.245045245765665\\
47.117811289112	0.245045562303794\\
47.124352427185	0.245045878715352\\
47.1308935536043	0.245046195000408\\
47.1374346683746	0.245046511159029\\
47.1439757715004	0.245046827191282\\
47.1505168629866	0.245047143097236\\
47.1570579428378	0.245047458876957\\
47.1635990110586	0.245047774530513\\
47.1701400676538	0.245048090057971\\
47.176681112628	0.2450484054594\\
47.183222145986	0.245048720734865\\
47.1897631677323	0.245049035884435\\
47.1963041778717	0.245049350908176\\
47.2028451764089	0.245049665806156\\
47.2093861633484	0.245049980578442\\
47.215927138695	0.245050295225101\\
47.2224681024533	0.2450506097462\\
47.229009054628	0.245050924141806\\
47.2355499952238	0.245051238411986\\
47.2420909242452	0.245051552556806\\
47.248631841697	0.245051866576335\\
47.2551727475837	0.245052180470638\\
47.2617136419102	0.245052494239783\\
47.2682545246808	0.245052807883835\\
47.2747953959005	0.245053121402863\\
47.2813362555737	0.245053434796932\\
47.287877103705	0.245053748066108\\
47.2944179402993	0.24505406121046\\
47.3009587653609	0.245054374230052\\
47.3074995788947	0.245054687124952\\
47.3140403809052	0.245054999895226\\
47.3205811713971	0.24505531254094\\
47.3271219503749	0.245055625062161\\
47.3336627178433	0.245055937458955\\
47.3402034738069	0.245056249731388\\
47.3467442182703	0.245056561879527\\
47.3532849512382	0.245056873903437\\
47.359825672715	0.245057185803184\\
47.3663663827056	0.245057497578836\\
47.3729070812143	0.245057809230457\\
47.379447768246	0.245058120758113\\
47.385988443805	0.245058432161872\\
47.3925291078961	0.245058743441798\\
47.3990697605239	0.245059054597957\\
47.4056104016929	0.245059365630416\\
47.4121510314077	0.245059676539239\\
47.4186916496729	0.245059987324493\\
47.4252322564931	0.245060297986244\\
47.4317728518729	0.245060608524556\\
47.4383134358169	0.245060918939496\\
47.4448540083296	0.245061229231129\\
47.4513945694155	0.245061539399521\\
47.4579351190794	0.245061849444736\\
47.4644756573257	0.245062159366841\\
47.471016184159	0.245062469165901\\
47.4775566995839	0.245062778841981\\
47.484097203605	0.245063088395146\\
47.4906376962267	0.245063397825462\\
47.4971781774537	0.245063707132994\\
47.5037186472905	0.245064016317806\\
47.5102591057417	0.245064325379965\\
47.5167995528118	0.245064634319535\\
47.5233399885054	0.245064943136581\\
47.5298804128269	0.245065251831168\\
47.5364208257811	0.245065560403361\\
47.5429612273723	0.245065868853225\\
47.5495016176052	0.245066177180825\\
47.5560419964842	0.245066485386226\\
47.5625823640139	0.245066793469491\\
47.5691227201989	0.245067101430687\\
47.5756630650436	0.245067409269878\\
47.5822033985526	0.245067716987128\\
47.5887437207304	0.245068024582502\\
47.5952840315815	0.245068332056064\\
47.6018243311105	0.24506863940788\\
47.6083646193218	0.245068946638012\\
47.6149048962201	0.245069253746527\\
47.6214451618097	0.245069560733488\\
47.6279854160952	0.24506986759896\\
47.6345256590811	0.245070174343006\\
47.641065890772	0.245070480965691\\
47.6476061111722	0.24507078746708\\
47.6541463202863	0.245071093847236\\
47.6606865181189	0.245071400106224\\
47.6672267046744	0.245071706244108\\
47.6737668799572	0.245072012260951\\
47.680307043972	0.245072318156818\\
47.6868471967231	0.245072623931772\\
47.693387338215	0.245072929585878\\
47.6999274684523	0.245073235119199\\
47.7064675874395	0.2450735405318\\
47.7130076951809	0.245073845823743\\
47.7195477916811	0.245074150995094\\
47.7260878769445	0.245074456045914\\
47.7326279509757	0.245074760976269\\
47.739168013779	0.245075065786222\\
47.745708065359	0.245075370475835\\
47.7522481057201	0.245075675045174\\
47.7587881348668	0.2450759794943\\
47.7653281528035	0.245076283823279\\
47.7718681595347	0.245076588032172\\
47.7784081550649	0.245076892121044\\
47.7849481393984	0.245077196089958\\
47.7914881125398	0.245077499938977\\
47.7980280744935	0.245077803668164\\
47.8045680252639	0.245078107277582\\
47.8111079648555	0.245078410767295\\
47.8176478932727	0.245078714137366\\
47.8241878105199	0.245079017387858\\
47.8307277166017	0.245079320518833\\
47.8372676115224	0.245079623530355\\
47.8438074952864	0.245079926422487\\
47.8503473678982	0.245080229195291\\
47.8568872293622	0.245080531848831\\
47.8634270796829	0.24508083438317\\
47.8699669188646	0.245081136798369\\
47.8765067469118	0.245081439094492\\
47.8830465638288	0.245081741271602\\
47.8895863696202	0.24508204332976\\
47.8961261642903	0.245082345269031\\
47.9026659478435	0.245082647089476\\
47.9092057202842	0.245082948791157\\
47.9157454816168	0.245083250374138\\
47.9222852318458	0.245083551838481\\
47.9288249709755	0.245083853184248\\
47.9353646990104	0.245084154411501\\
47.9419044159548	0.245084455520303\\
47.9484441218131	0.245084756510716\\
47.9549838165897	0.245085057382803\\
47.961523500289	0.245085358136625\\
47.9680631729154	0.245085658772245\\
47.9746028344732	0.245085959289725\\
47.981142484967	0.245086259689127\\
47.9876821244009	0.245086559970512\\
47.9942217527794	0.245086860133944\\
48.000761370107	0.245087160179483\\
48.0073009763878	0.245087460107193\\
48.0138405716264	0.245087759917133\\
48.0203801558271	0.245088059609368\\
48.0269197289943	0.245088359183957\\
48.0334592911322	0.245088658640964\\
48.0399988422453	0.245088957980449\\
48.046538382338	0.245089257202475\\
48.0530779114145	0.245089556307103\\
48.0596174294793	0.245089855294394\\
48.0661569365367	0.24509015416441\\
48.0726964325911	0.245090452917214\\
48.0792359176467	0.245090751552865\\
48.0857753917079	0.245091050071426\\
48.0923148547792	0.245091348472957\\
48.0988543068647	0.245091646757521\\
48.1053937479689	0.245091944925179\\
48.1119331780961	0.245092242975991\\
48.1184725972506	0.245092540910019\\
48.1250120054368	0.245092838727325\\
48.1315514026589	0.245093136427969\\
48.1380907889213	0.245093434012013\\
48.1446301642284	0.245093731479517\\
48.1511695285844	0.245094028830542\\
48.1577088819937	0.24509432606515\\
48.1642482244606	0.245094623183402\\
48.1707875559894	0.245094920185358\\
48.1773268765844	0.245095217071079\\
48.1838661862499	0.245095513840626\\
48.1904054849902	0.24509581049406\\
48.1969447728097	0.245096107031442\\
48.2034840497126	0.245096403452832\\
48.2100233157033	0.245096699758291\\
48.2165625707859	0.245096995947879\\
48.223101814965	0.245097292021658\\
48.2296410482446	0.245097587979687\\
48.2361802706292	0.245097883822027\\
48.242719482123	0.245098179548739\\
48.2492586827303	0.245098475159884\\
48.2557978724553	0.24509877065552\\
48.2623370513025	0.24509906603571\\
48.2688762192759	0.245099361300512\\
48.27541537638	0.245099656449988\\
48.281954522619	0.245099951484198\\
48.2884936579972	0.245100246403201\\
48.2950327825188	0.245100541207059\\
48.3015718961881	0.24510083589583\\
48.3081109990095	0.245101130469575\\
48.314650090987	0.245101424928355\\
48.3211891721251	0.245101719272229\\
48.3277282424279	0.245102013501256\\
48.3342673018997	0.245102307615498\\
48.3408063505449	0.245102601615014\\
48.3473453883676	0.245102895499863\\
48.353884415372	0.245103189270106\\
48.3604234315625	0.245103482925802\\
48.3669624369433	0.245103776467011\\
48.3735014315186	0.245104069893793\\
48.3800404152926	0.245104363206207\\
48.3865793882697	0.245104656404313\\
48.393118350454	0.24510494948817\\
48.3996573018498	0.245105242457839\\
48.4061962424614	0.245105535313378\\
48.4127351722928	0.245105828054846\\
48.4192740913485	0.245106120682304\\
48.4258129996325	0.245106413195811\\
48.4323518971492	0.245106705595425\\
48.4388907839027	0.245106997881207\\
48.4454296598973	0.245107290053215\\
48.4519685251372	0.24510758211151\\
48.4585073796266	0.245107874056149\\
48.4650462233697	0.245108165887192\\
48.4715850563707	0.245108457604698\\
48.4781238786339	0.245108749208726\\
48.4846626901634	0.245109040699336\\
48.4912014909635	0.245109332076586\\
48.4977402810383	0.245109623340534\\
48.504279060392	0.245109914491241\\
48.5108178290289	0.245110205528765\\
48.5173565869532	0.245110496453165\\
48.523895334169	0.245110787264499\\
48.5304340706805	0.245111077962826\\
48.5369727964919	0.245111368548206\\
48.5435115116074	0.245111659020696\\
48.5500502160313	0.245111949380355\\
48.5565889097676	0.245112239627242\\
48.5631275928205	0.245112529761415\\
48.5696662651943	0.245112819782934\\
48.5762049268931	0.245113109691856\\
48.5827435779211	0.24511339948824\\
48.5892822182824	0.245113689172144\\
48.5958208479813	0.245113978743627\\
48.6023594670218	0.245114268202747\\
48.6088980754082	0.245114557549562\\
48.6154366731446	0.24511484678413\\
48.6219752602352	0.245115135906511\\
48.6285138366842	0.245115424916761\\
48.6350524024956	0.245115713814939\\
48.6415909576737	0.245116002601104\\
48.6481295022225	0.245116291275312\\
48.6546680361464	0.245116579837624\\
48.6612065594493	0.245116868288095\\
48.6677450721354	0.245117156626784\\
48.674283574209	0.24511744485375\\
48.6808220656741	0.24511773296905\\
48.6873605465348	0.245118020972741\\
48.6938990167953	0.245118308864882\\
48.7004374764598	0.245118596645531\\
48.7069759255324	0.245118884314744\\
48.7135143640171	0.245119171872581\\
48.7200527919182	0.245119459319098\\
48.7265912092398	0.245119746654353\\
48.7331296159859	0.245120033878404\\
48.7396680121607	0.245120320991308\\
48.7462063977683	0.245120607993122\\
48.7527447728129	0.245120894883905\\
48.7592831372985	0.245121181663713\\
48.7658214912293	0.245121468332605\\
48.7723598346094	0.245121754890637\\
48.7788981674429	0.245122041337866\\
48.7854364897338	0.245122327674351\\
48.7919748014864	0.245122613900147\\
48.7985131027046	0.245122900015313\\
48.8050513933927	0.245123186019906\\
48.8115896735547	0.245123471913982\\
48.8181279431946	0.245123757697599\\
48.8246662023166	0.245124043370814\\
48.8312044509249	0.245124328933683\\
48.8377426890233	0.245124614386265\\
48.8442809166162	0.245124899728615\\
48.8508191337075	0.245125184960791\\
48.8573573403013	0.24512547008285\\
48.8638955364017	0.245125755094848\\
48.8704337220128	0.245126039996842\\
48.8769718971387	0.245126324788889\\
48.8835100617834	0.245126609471046\\
48.890048215951	0.24512689404337\\
48.8965863596456	0.245127178505916\\
48.9031244928713	0.245127462858742\\
48.9096626156321	0.245127747101904\\
48.916200727932	0.245128031235459\\
48.9227388297752	0.245128315259463\\
48.9292769211657	0.245128599173973\\
48.9358150021076	0.245128882979046\\
48.9423530726048	0.245129166674737\\
48.9488911326616	0.245129450261102\\
48.9554291822818	0.245129733738199\\
48.9619672214697	0.245130017106084\\
48.9685052502291	0.245130300364812\\
48.9750432685642	0.245130583514441\\
48.9815812764791	0.245130866555025\\
48.9881192739776	0.245131149486622\\
48.994657261064	0.245131432309287\\
49.0011952377422	0.245131715023077\\
49.0077332040162	0.245131997628048\\
49.0142711598901	0.245132280124255\\
49.020809105368	0.245132562511754\\
49.0273470404538	0.245132844790602\\
49.0338849651515	0.245133126960854\\
49.0404228794653	0.245133409022567\\
49.0469607833991	0.245133690975795\\
49.0534986769569	0.245133972820595\\
49.0600365601428	0.245134254557023\\
49.0665744329608	0.245134536185134\\
49.0731122954149	0.245134817704984\\
49.079650147509	0.245135099116628\\
49.0861879892473	0.245135380420122\\
49.0927258206337	0.245135661615523\\
49.0992636416722	0.245135942702884\\
49.1058014523668	0.245136223682262\\
49.1123392527216	0.245136504553713\\
49.1188770427405	0.24513678531729\\
49.1254148224276	0.245137065973051\\
49.1319525917867	0.245137346521051\\
49.138490350822	0.245137626961344\\
49.1450280995373	0.245137907293986\\
49.1515658379368	0.245138187519032\\
49.1581035660243	0.245138467636537\\
49.1646412838039	0.245138747646557\\
49.1711789912796	0.245139027549147\\
49.1777166884553	0.245139307344362\\
49.184254375335	0.245139587032256\\
49.1907920519227	0.245139866612886\\
49.1973297182223	0.245140146086305\\
49.2038673742379	0.245140425452569\\
49.2104050199733	0.245140704711733\\
49.2169426554327	0.245140983863852\\
49.2234802806199	0.24514126290898\\
49.2300178955388	0.245141541847173\\
49.2365555001936	0.245141820678484\\
49.243093094588	0.24514209940297\\
49.2496306787262	0.245142378020684\\
49.256168252612	0.245142656531682\\
49.2627058162493	0.245142934936017\\
49.2692433696422	0.245143213233745\\
49.2757809127947	0.24514349142492\\
49.2823184457105	0.245143769509597\\
49.2888559683937	0.24514404748783\\
49.2953934808483	0.245144325359674\\
49.3019309830781	0.245144603125183\\
49.3084684750872	0.245144880784411\\
49.3150059568794	0.245145158337413\\
49.3215434284587	0.245145435784243\\
49.328080889829	0.245145713124956\\
49.3346183409942	0.245145990359605\\
49.3411557819584	0.245146267488245\\
49.3476932127253	0.245146544510931\\
49.354230633299	0.245146821427715\\
49.3607680436834	0.245147098238654\\
49.3673054438823	0.245147374943799\\
49.3738428338997	0.245147651543206\\
49.3803802137395	0.245147928036929\\
49.3869175834057	0.245148204425022\\
49.3934549429021	0.245148480707537\\
49.3999922922326	0.245148756884531\\
49.4065296314013	0.245149032956055\\
49.4130669604119	0.245149308922165\\
49.4196042792684	0.245149584782913\\
49.4261415879746	0.245149860538355\\
49.4326788865346	0.245150136188543\\
49.4392161749521	0.245150411733531\\
49.4457534532311	0.245150687173372\\
49.4522907213755	0.245150962508122\\
49.4588279793891	0.245151237737832\\
49.465365227276	0.245151512862557\\
49.4719024650398	0.245151787882351\\
49.4784396926846	0.245152062797266\\
49.4849769102142	0.245152337607356\\
49.4915141176326	0.245152612312674\\
49.4980513149435	0.245152886913275\\
49.5045885021508	0.245153161409211\\
49.5111256792586	0.245153435800536\\
49.5176628462705	0.245153710087303\\
49.5242000031905	0.245153984269565\\
49.5307371500225	0.245154258347375\\
49.5372742867703	0.245154532320787\\
49.5438114134379	0.245154806189854\\
49.550348530029	0.245155079954629\\
49.5568856365475	0.245155353615165\\
49.5634227329973	0.245155627171516\\
49.5699598193823	0.245155900623733\\
49.5764968957062	0.24515617397187\\
49.5830339619731	0.245156447215981\\
49.5895710181866	0.245156720356117\\
49.5961080643507	0.245156993392333\\
49.6026451004692	0.24515726632468\\
49.609182126546	0.245157539153211\\
49.615719142585	0.245157811877981\\
49.6222561485898	0.24515808449904\\
49.6287931445645	0.245158357016442\\
49.6353301305128	0.245158629430239\\
49.6418671064385	0.245158901740485\\
49.6484040723456	0.245159173947232\\
49.6549410282378	0.245159446050532\\
49.661477974119	0.245159718050438\\
49.668014909993	0.245159989947002\\
49.6745518358637	0.245160261740278\\
49.6810887517348	0.245160533430316\\
49.6876256576101	0.245160805017171\\
49.6941625534936	0.245161076500894\\
49.700699439389	0.245161347881537\\
49.7072363153002	0.245161619159154\\
49.7137731812309	0.245161890333795\\
49.720310037185	0.245162161405514\\
49.7268468831662	0.245162432374363\\
49.7333837191785	0.245162703240393\\
49.7399205452256	0.245162974003658\\
49.7464573613112	0.245163244664208\\
49.7529941674393	0.245163515222097\\
49.7595309636136	0.245163785677376\\
49.766067749838	0.245164056030097\\
49.7726045261161	0.245164326280313\\
49.7791412924519	0.245164596428075\\
49.7856780488491	0.245164866473435\\
49.7922147953114	0.245165136416445\\
49.7987515318428	0.245165406257157\\
49.8052882584469	0.245165675995622\\
49.8118249751277	0.245165945631893\\
49.8183616818887	0.245166215166022\\
49.8248983787339	0.245166484598059\\
49.8314350656671	0.245166753928057\\
49.8379717426919	0.245167023156068\\
49.8445084098122	0.245167292282142\\
49.8510450670317	0.245167561306332\\
49.8575817143543	0.245167830228689\\
49.8641183517836	0.245168099049265\\
49.8706549793236	0.245168367768111\\
49.8771915969778	0.245168636385279\\
49.8837282047502	0.245168904900819\\
49.8902648026444	0.245169173314784\\
49.8968013906642	0.245169441627225\\
49.9033379688134	0.245169709838193\\
49.9098745370958	0.245169977947739\\
49.9164110955151	0.245170245955915\\
49.922947644075	0.245170513862773\\
49.9294841827793	0.245170781668362\\
49.9360207116318	0.245171049372734\\
49.9425572306362	0.245171316975941\\
49.9490937397962	0.245171584478034\\
49.9556302391156	0.245171851879063\\
49.9621667285982	0.24517211917908\\
49.9687032082477	0.245172386378136\\
49.9752396780677	0.245172653476281\\
49.9817761380621	0.245172920473567\\
49.9883125882347	0.245173187370044\\
49.994849028589	0.245173454165764\\
50.0013854591289	0.245173720860777\\
50.0079218798581	0.245173987455134\\
50.0144582907802	0.245174253948886\\
50.0209946918992	0.245174520342083\\
50.0275310832185	0.245174786634777\\
50.0340674647421	0.245175052827018\\
50.0406038364736	0.245175318918856\\
50.0471401984167	0.245175584910343\\
50.0536765505751	0.245175850801529\\
50.0602128929526	0.245176116592464\\
50.0667492255529	0.245176382283199\\
50.0732855483796	0.245176647873785\\
50.0798218614365	0.245176913364271\\
50.0863581647273	0.245177178754709\\
50.0928944582557	0.245177444045148\\
50.0994307420254	0.24517770923564\\
50.1059670160401	0.245177974326234\\
50.1125032803035	0.245178239316981\\
50.1190395348193	0.245178504207931\\
50.1255757795912	0.245178768999134\\
50.1321120146229	0.24517903369064\\
50.138648239918	0.245179298282501\\
50.1451844554804	0.245179562774765\\
50.1517206613136	0.245179827167483\\
50.1582568574213	0.245180091460705\\
50.1647930438072	0.245180355654481\\
50.1713292204751	0.245180619748861\\
50.1778653874285	0.245180883743895\\
50.1844015446713	0.245181147639633\\
50.1909376922069	0.245181411436125\\
50.1974738300392	0.245181675133421\\
50.2040099581718	0.245181938731571\\
50.2105460766083	0.245182202230624\\
50.2170821853525	0.24518246563063\\
50.223618284408	0.24518272893164\\
50.2301543737784	0.245182992133702\\
50.2366904534675	0.245183255236867\\
50.2432265234789	0.245183518241184\\
50.2497625838162	0.245183781146703\\
50.2562986344831	0.245184043953474\\
50.2628346754833	0.245184306661545\\
50.2693707068204	0.245184569270967\\
50.2759067284981	0.24518483178179\\
50.2824427405201	0.245185094194061\\
50.2889787428899	0.245185356507832\\
50.2955147356112	0.245185618723151\\
50.3020507186877	0.245185880840069\\
50.308586692123	0.245186142858633\\
50.3151226559207	0.245186404778894\\
50.3216586100846	0.245186666600901\\
50.3281945546181	0.245186928324703\\
50.3347304895251	0.245187189950349\\
50.341266414809	0.245187451477889\\
50.3478023304736	0.245187712907372\\
50.3543382365224	0.245187974238846\\
50.3608741329592	0.245188235472362\\
50.3674100197874	0.245188496607967\\
50.3739458970108	0.245188757645712\\
50.380481764633	0.245189018585645\\
50.3870176226575	0.245189279427815\\
50.3935534710881	0.245189540172271\\
50.4000893099283	0.245189800819063\\
50.4066251391817	0.245190061368238\\
50.413160958852	0.245190321819846\\
50.4196967689427	0.245190582173936\\
50.4262325694576	0.245190842430556\\
50.4327683604001	0.245191102589755\\
50.4393041417739	0.245191362651583\\
50.4458399135826	0.245191622616087\\
50.4523756758298	0.245191882483317\\
50.4589114285192	0.24519214225332\\
50.4654471716542	0.245192401926146\\
50.4719829052385	0.245192661501844\\
50.4785186292758	0.245192920980461\\
50.4850543437695	0.245193180362047\\
50.4915900487233	0.245193439646649\\
50.4981257441408	0.245193698834317\\
50.5046614300255	0.245193957925099\\
50.5111971063811	0.245194216919043\\
50.5177327732111	0.245194475816197\\
50.5242684305192	0.245194734616611\\
50.5308040783088	0.245194993320331\\
50.5373397165836	0.245195251927407\\
50.5438753453472	0.245195510437887\\
50.5504109646031	0.245195768851819\\
50.556946574355	0.245196027169251\\
50.5634821746063	0.245196285390231\\
50.5700177653606	0.245196543514808\\
50.5765533466216	0.24519680154303\\
50.5830889183928	0.245197059474944\\
50.5896244806777	0.2451973173106\\
50.5961600334799	0.245197575050043\\
50.602695576803	0.245197832693324\\
50.6092311106506	0.245198090240489\\
50.6157666350261	0.245198347691587\\
50.6223021499332	0.245198605046666\\
50.6288376553754	0.245198862305773\\
50.6353731513563	0.245199119468956\\
50.6419086378794	0.245199376536264\\
50.6484441149482	0.245199633507743\\
50.6549795825663	0.245199890383442\\
50.6615150407373	0.245200147163408\\
50.6680504894647	0.245200403847689\\
50.674585928752	0.245200660436333\\
50.6811213586028	0.245200916929387\\
50.6876567790206	0.245201173326899\\
50.694192190009	0.245201429628916\\
50.7007275915715	0.245201685835487\\
50.7072629837115	0.245201941946658\\
50.7137983664328	0.245202197962477\\
50.7203337397387	0.245202453882991\\
50.7268691036328	0.245202709708249\\
50.7334044581187	0.245202965438296\\
50.7399398031998	0.245203221073182\\
50.7464751388797	0.245203476612952\\
50.7530104651619	0.245203732057654\\
50.7595457820499	0.245203987407336\\
50.7660810895473	0.245204242662045\\
50.7726163876576	0.245204497821828\\
50.7791516763842	0.245204752886732\\
50.7856869557306	0.245205007856805\\
50.7922222257005	0.245205262732093\\
50.7987574862973	0.245205517512643\\
50.8052927375245	0.245205772198503\\
50.8118279793856	0.24520602678972\\
50.8183632118841	0.245206281286341\\
50.8248984350235	0.245206535688412\\
50.8314336488073	0.245206789995981\\
50.8379688532391	0.245207044209094\\
50.8445040483223	0.245207298327799\\
50.8510392340603	0.245207552352143\\
50.8575744104568	0.245207806282171\\
50.8641095775152	0.245208060117931\\
50.8706447352389	0.245208313859471\\
50.8771798836315	0.245208567506835\\
50.8837150226964	0.245208821060072\\
50.8902501524372	0.245209074519228\\
50.8967852728573	0.245209327884349\\
50.9033203839602	0.245209581155482\\
50.9098554857494	0.245209834332675\\
50.9163905782283	0.245210087415972\\
50.9229256614005	0.245210340405422\\
50.9294607352693	0.24521059330107\\
50.9359957998384	0.245210846102963\\
50.9425308551111	0.245211098811147\\
50.9490659010908	0.245211351425669\\
50.9556009377812	0.245211603946575\\
50.9621359651856	0.245211856373912\\
50.9686709833075	0.245212108707726\\
50.9752059921504	0.245212360948063\\
50.9817409917177	0.245212613094969\\
50.9882759820129	0.245212865148492\\
50.9948109630395	0.245213117108676\\
51.0013459348008	0.245213368975569\\
51.0078808973004	0.245213620749216\\
51.0144158505416	0.245213872429663\\
51.0209507945281	0.245214124016957\\
51.0274857292631	0.245214375511144\\
51.0340206547502	0.245214626912269\\
51.0405555709927	0.24521487822038\\
51.0470904779942	0.245215129435521\\
51.0536253757581	0.245215380557739\\
51.0601602642877	0.245215631587079\\
51.0666951435866	0.245215882523589\\
51.0732300136582	0.245216133367313\\
51.0797648745059	0.245216384118297\\
51.0862997261332	0.245216634776587\\
51.0928345685434	0.24521688534223\\
51.0993694017401	0.24521713581527\\
51.1059042257266	0.245217386195753\\
51.1124390405063	0.245217636483726\\
51.1189738460827	0.245217886679234\\
51.1255086424592	0.245218136782323\\
51.1320434296393	0.245218386793037\\
51.1385782076263	0.245218636711424\\
51.1451129764236	0.245218886537528\\
51.1516477360347	0.245219136271395\\
51.1581824864631	0.24521938591307\\
51.164717227712	0.245219635462599\\
51.1712519597849	0.245219884920028\\
51.1777866826852	0.245220134285401\\
51.1843213964164	0.245220383558765\\
51.1908561009818	0.245220632740164\\
51.1973907963849	0.245220881829644\\
51.2039254826289	0.245221130827251\\
51.2104601597174	0.245221379733029\\
51.2169948276538	0.245221628547024\\
51.2235294864413	0.245221877269281\\
51.2300641360835	0.245222125899845\\
51.2365987765837	0.245222374438762\\
51.2431334079453	0.245222622886077\\
51.2496680301718	0.245222871241834\\
51.2562026432664	0.245223119506079\\
51.2627372472325	0.245223367678857\\
51.2692718420737	0.245223615760213\\
51.2758064277931	0.245223863750192\\
51.2823410043943	0.245224111648839\\
51.2888755718806	0.245224359456199\\
51.2954101302554	0.245224607172317\\
51.3019446795221	0.245224854797237\\
51.3084792196839	0.245225102331005\\
51.3150137507444	0.245225349773666\\
51.3215482727068	0.245225597125264\\
51.3280827855746	0.245225844385844\\
51.3346172893511	0.245226091555451\\
51.3411517840397	0.245226338634129\\
51.3476862696437	0.245226585621924\\
51.3542207461665	0.24522683251888\\
51.3607552136115	0.245227079325041\\
51.3672896719821	0.245227326040453\\
51.3738241212815	0.24522757266516\\
51.3803585615131	0.245227819199206\\
51.3868929926804	0.245228065642636\\
51.3934274147866	0.245228311995494\\
51.3999618278351	0.245228558257826\\
51.4064962318292	0.245228804429675\\
51.4130306267723	0.245229050511086\\
51.4195650126678	0.245229296502103\\
51.426099389519	0.245229542402771\\
51.4326337573292	0.245229788213134\\
51.4391681161018	0.245230033933236\\
51.4457024658401	0.245230279563122\\
51.4522368065474	0.245230525102836\\
51.4587711382271	0.245230770552422\\
51.4653054608826	0.245231015911925\\
51.4718397745171	0.245231261181388\\
51.478374079134	0.245231506360856\\
51.4849083747366	0.245231751450372\\
51.4914426613282	0.245231996449982\\
51.4979769389123	0.245232241359729\\
51.504511207492	0.245232486179657\\
51.5110454670707	0.24523273090981\\
51.5175797176518	0.245232975550232\\
51.5241139592385	0.245233220100968\\
51.5306481918343	0.24523346456206\\
51.5371824154423	0.245233708933554\\
51.5437166300659	0.245233953215492\\
51.5502508357084	0.245234197407919\\
51.5567850323732	0.245234441510879\\
51.5633192200635	0.245234685524415\\
51.5698533987827	0.245234929448572\\
51.5763875685341	0.245235173283392\\
51.5829217293209	0.24523541702892\\
51.5894558811464	0.2452356606852\\
51.5959900240141	0.245235904252274\\
51.6025241579271	0.245236147730187\\
51.6090582828888	0.245236391118983\\
51.6155923989025	0.245236634418704\\
51.6221265059714	0.245236877629395\\
51.6286606040989	0.245237120751099\\
51.6351946932882	0.24523736378386\\
51.6417287735427	0.24523760672772\\
51.6482628448656	0.245237849582725\\
51.6547969072602	0.245238092348916\\
51.6613309607298	0.245238335026337\\
51.6678650052777	0.245238577615032\\
51.6743990409072	0.245238820115045\\
51.6809330676215	0.245239062526418\\
51.6874670854239	0.245239304849194\\
51.6940010943178	0.245239547083418\\
51.7005350943063	0.245239789229132\\
51.7070690853928	0.245240031286379\\
51.7136030675805	0.245240273255203\\
51.7201370408727	0.245240515135647\\
51.7266710052727	0.245240756927754\\
51.7332049607838	0.245240998631567\\
51.7397389074091	0.245241240247129\\
51.746272845152	0.245241481774484\\
51.7528067740157	0.245241723213674\\
51.7593406940035	0.245241964564742\\
51.7658746051187	0.245242205827732\\
51.7724085073645	0.245242447002685\\
51.7789424007442	0.245242688089647\\
51.785476285261	0.245242929088658\\
51.7920101609181	0.245243169999762\\
51.7985440277189	0.245243410823002\\
51.8050778856666	0.245243651558421\\
51.8116117347644	0.245243892206062\\
51.8181455750156	0.245244132765966\\
51.8246794064234	0.245244373238178\\
51.831213228991	0.24524461362274\\
51.8377470427218	0.245244853919694\\
51.8442808476189	0.245245094129083\\
51.8508146436856	0.24524533425095\\
51.8573484309251	0.245245574285337\\
51.8638822093407	0.245245814232287\\
51.8704159789356	0.245246054091843\\
51.876949739713	0.245246293864047\\
51.8834834916762	0.245246533548942\\
51.8900172348283	0.24524677314657\\
51.8965509691727	0.245247012656973\\
51.9030846947125	0.245247252080194\\
51.909618411451	0.245247491416276\\
51.9161521193914	0.245247730665261\\
51.9226858185369	0.245247969827191\\
51.9292195088907	0.245248208902108\\
51.9357531904561	0.245248447890055\\
51.9422868632362	0.245248686791075\\
51.9488205272344	0.245248925605208\\
51.9553541824537	0.245249164332499\\
51.9618878288974	0.245249402972988\\
51.9684214665688	0.245249641526718\\
51.974955095471	0.245249879993732\\
51.9814887156072	0.24525011837407\\
51.9880223269807	0.245250356667776\\
51.9945559295946	0.245250594874892\\
52.0010895234522	0.245250832995459\\
52.0076231085566	0.24525107102952\\
52.0141566849111	0.245251308977116\\
52.0206902525188	0.24525154683829\\
52.027223811383	0.245251784613083\\
52.0337573615068	0.245252022301538\\
52.0402909028935	0.245252259903696\\
52.0468244355462	0.245252497419599\\
52.0533579594681	0.245252734849289\\
52.0598914746624	0.245252972192808\\
52.0664249811323	0.245253209450198\\
52.0729584788809	0.2452534466215\\
52.0794919679116	0.245253683706756\\
52.0860254482274	0.245253920706009\\
52.0925589198315	0.245254157619299\\
52.0990923827271	0.245254394446668\\
52.1056258369174	0.245254631188158\\
52.1121592824056	0.24525486784381\\
52.1186927191948	0.245255104413667\\
52.1252261472882	0.24525534089777\\
52.131759566689	0.245255577296159\\
52.1382929774003	0.245255813608878\\
52.1448263794254	0.245256049835967\\
52.1513597727673	0.245256285977467\\
52.1578931574293	0.245256522033421\\
52.1644265334145	0.245256758003869\\
52.1709599007261	0.245256993888854\\
52.1774932593673	0.245257229688415\\
52.1840266093411	0.245257465402596\\
52.1905599506508	0.245257701031436\\
52.1970932832995	0.245257936574978\\
52.2036266072903	0.245258172033262\\
52.2101599226265	0.24525840740633\\
52.2166932293112	0.245258642694224\\
52.2232265273475	0.245258877896983\\
52.2297598167385	0.24525911301465\\
52.2362930974875	0.245259348047265\\
52.2428263695975	0.245259582994871\\
52.2493596330717	0.245259817857507\\
52.2558928879133	0.245260052635214\\
52.2624261341254	0.245260287328035\\
52.2689593717111	0.24526052193601\\
52.2754926006736	0.245260756459179\\
52.282025821016	0.245260990897584\\
52.2885590327414	0.245261225251267\\
52.295092235853	0.245261459520267\\
52.3016254303539	0.245261693704625\\
52.3081586162472	0.245261927804383\\
52.3146917935361	0.245262161819582\\
52.3212249622236	0.245262395750261\\
52.327758122313	0.245262629596463\\
52.3342912738074	0.245262863358227\\
52.3408244167098	0.245263097035595\\
52.3473575510233	0.245263330628607\\
52.3538906767512	0.245263564137304\\
52.3604237938965	0.245263797561726\\
52.3669569024623	0.245264030901915\\
52.3734900024518	0.245264264157911\\
52.3800230938681	0.245264497329754\\
52.3865561767142	0.245264730417485\\
52.3930892509933	0.245264963421145\\
52.3996223167085	0.245265196340773\\
52.4061553738629	0.245265429176412\\
52.4126884224597	0.2452656619281\\
52.4192214625018	0.245265894595879\\
52.4257544939925	0.245266127179789\\
52.4322875169348	0.245266359679871\\
52.4388205313318	0.245266592096164\\
52.4453535371866	0.245266824428709\\
52.4518865345023	0.245267056677547\\
52.458419523282	0.245267288842717\\
52.4649525035289	0.245267520924261\\
52.4714854752459	0.245267752922217\\
52.4780184384363	0.245267984836628\\
52.484551393103	0.245268216667532\\
52.4910843392491	0.24526844841497\\
52.4976172768778	0.245268680078982\\
52.5041502059922	0.245268911659608\\
52.5106831265953	0.245269143156888\\
52.5172160386902	0.245269374570863\\
52.5237489422799	0.245269605901573\\
52.5302818373676	0.245269837149057\\
52.5368147239564	0.245270068313355\\
52.5433476020493	0.245270299394508\\
52.5498804716493	0.245270530392556\\
52.5564133327596	0.245270761307538\\
52.5629461853833	0.245270992139494\\
52.5694790295233	0.245271222888465\\
52.5760118651829	0.245271453554489\\
52.5825446923649	0.245271684137608\\
52.5890775110726	0.245271914637861\\
52.5956103213089	0.245272145055287\\
52.602143123077	0.245272375389926\\
52.6086759163799	0.245272605641819\\
52.6152087012206	0.245272835811004\\
52.6217414776022	0.245273065897522\\
52.6282742455278	0.245273295901412\\
52.6348070050004	0.245273525822714\\
52.6413397560231	0.245273755661468\\
52.6478724985989	0.245273985417713\\
52.6544052327309	0.245274215091488\\
52.6609379584221	0.245274444682834\\
52.6674706756756	0.24527467419179\\
52.6740033844944	0.245274903618394\\
52.6805360848815	0.245275132962688\\
52.6870687768401	0.24527536222471\\
52.6936014603732	0.2452755914045\\
52.7001341354837	0.245275820502097\\
52.7066668021748	0.24527604951754\\
52.7131994604494	0.24527627845087\\
52.7197321103106	0.245276507302124\\
52.7262647517615	0.245276736071343\\
52.732797384805	0.245276964758566\\
52.7393300094443	0.245277193363833\\
52.7458626256823	0.245277421887181\\
52.752395233522	0.245277650328652\\
52.7589278329665	0.245277878688283\\
52.7654604240189	0.245278106966114\\
52.7719930066821	0.245278335162184\\
52.7785255809591	0.245278563276533\\
52.7850581468531	0.245278791309199\\
52.7915907043669	0.245279019260222\\
52.7981232535037	0.24527924712964\\
52.8046557942664	0.245279474917493\\
52.811188326658	0.245279702623819\\
52.8177208506817	0.245279930248659\\
52.8242533663403	0.24528015779205\\
52.8307858736369	0.245280385254031\\
52.8373183725745	0.245280612634642\\
52.8438508631561	0.245280839933921\\
52.8503833453847	0.245281067151908\\
52.8569158192634	0.24528129428864\\
52.8634482847951	0.245281521344158\\
52.8699807419828	0.245281748318499\\
52.8765131908296	0.245281975211703\\
52.8830456313384	0.245282202023808\\
52.8895780635122	0.245282428754854\\
52.896110487354	0.245282655404878\\
52.9026429028669	0.245282881973919\\
52.9091753100537	0.245283108462017\\
52.9157077089176	0.24528333486921\\
52.9222400994615	0.245283561195536\\
52.9287724816884	0.245283787441033\\
52.9353048556012	0.245284013605742\\
52.941837221203	0.245284239689699\\
52.9483695784968	0.245284465692945\\
52.9549019274855	0.245284691615516\\
52.9614342681721	0.245284917457452\\
52.9679666005596	0.245285143218791\\
52.974498924651	0.245285368899572\\
52.9810312404492	0.245285594499833\\
52.9875635479573	0.245285820019612\\
52.9940958471782	0.245286045458948\\
53.0006281381148	0.245286270817879\\
53.0071604207702	0.245286496096443\\
53.0136926951473	0.245286721294679\\
53.0202249612491	0.245286946412625\\
53.0267572190786	0.245287171450319\\
53.0332894686387	0.2452873964078\\
53.0398217099324	0.245287621285106\\
53.0463539429627	0.245287846082274\\
53.0528861677324	0.245288070799343\\
53.0594183842447	0.245288295436352\\
53.0659505925024	0.245288519993337\\
53.0724827925085	0.245288744470339\\
53.0790149842659	0.245288968867393\\
53.0855471677776	0.245289193184539\\
53.0920793430466	0.245289417421815\\
53.0986115100759	0.245289641579258\\
53.1051436688683	0.245289865656907\\
53.1116758194268	0.2452900896548\\
53.1182079617543	0.245290313572973\\
53.1247400958539	0.245290537411466\\
53.1312722217285	0.245290761170317\\
53.1378043393809	0.245290984849562\\
53.1443364488142	0.24529120844924\\
53.1508685500313	0.24529143196939\\
53.1574006430351	0.245291655410047\\
53.1639327278285	0.245291878771251\\
53.1704648044146	0.24529210205304\\
53.1769968727962	0.24529232525545\\
53.1835289329762	0.245292548378519\\
53.1900609849577	0.245292771422286\\
53.1965930287435	0.245292994386788\\
53.2031250643365	0.245293217272062\\
53.2096570917398	0.245293440078147\\
53.2161891109561	0.245293662805079\\
53.2227211219885	0.245293885452897\\
53.2292531248399	0.245294108021637\\
53.2357851195132	0.245294330511338\\
53.2423171060113	0.245294552922037\\
53.2488490843371	0.245294775253772\\
53.2553810544936	0.245294997506579\\
53.2619130164836	0.245295219680497\\
53.2684449703101	0.245295441775562\\
53.2749769159761	0.245295663791813\\
53.2815088534843	0.245295885729286\\
53.2880407828378	0.245296107588019\\
53.2945727040394	0.24529632936805\\
53.301104617092	0.245296551069414\\
53.3076365219986	0.245296772692151\\
53.3141684187621	0.245296994236297\\
53.3207003073853	0.245297215701889\\
53.3272321878711	0.245297437088965\\
53.3337640602226	0.245297658397562\\
53.3402959244425	0.245297879627716\\
53.3468277805338	0.245298100779466\\
53.3533596284994	0.245298321852848\\
53.3598914683421	0.245298542847899\\
53.3664233000649	0.245298763764657\\
53.3729551236706	0.245298984603158\\
53.3794869391621	0.24529920536344\\
53.3860187465425	0.245299426045539\\
53.3925505458144	0.245299646649493\\
53.3990823369808	0.245299867175338\\
53.4056141200447	0.245300087623112\\
53.4121458950088	0.245300307992852\\
53.4186776618761	0.245300528284593\\
53.4252094206495	0.245300748498374\\
53.4317411713317	0.245300968634232\\
53.4382729139258	0.245301188692202\\
53.4448046484346	0.245301408672322\\
53.451336374861	0.245301628574628\\
53.4578680932078	0.245301848399158\\
53.4643998034779	0.245302068145948\\
53.4709315056742	0.245302287815035\\
53.4774631997996	0.245302507406456\\
53.4839948858569	0.245302726920247\\
53.4905265638491	0.245302946356445\\
53.4970582337788	0.245303165715086\\
53.5035898956492	0.245303384996208\\
53.5101215494629	0.245303604199846\\
53.5166531952229	0.245303823326038\\
53.523184832932	0.245304042374821\\
53.5297164625932	0.245304261346229\\
53.5362480842091	0.245304480240301\\
53.5427796977828	0.245304699057072\\
53.549311303317	0.24530491779658\\
53.5558429008147	0.24530513645886\\
53.5623744902786	0.245305355043949\\
53.5689060717117	0.245305573551883\\
53.5754376451167	0.245305791982699\\
53.5819692104966	0.245306010336433\\
53.5885007678541	0.245306228613122\\
53.5950323171922	0.245306446812801\\
53.6015638585136	0.245306664935508\\
53.6080953918213	0.245306882981278\\
53.614626917118	0.245307100950148\\
53.6211584344066	0.245307318842154\\
53.6276899436899	0.245307536657331\\
53.6342214449709	0.245307754395718\\
53.6407529382522	0.245307972057348\\
53.6472844235368	0.24530818964226\\
53.6538159008275	0.245308407150488\\
53.6603473701271	0.245308624582069\\
53.6668788314384	0.245308841937039\\
53.6734102847644	0.245309059215434\\
53.6799417301077	0.245309276417291\\
53.6864731674713	0.245309493542644\\
53.693004596858	0.245309710591531\\
53.6995360182706	0.245309927563987\\
53.7060674317119	0.245310144460048\\
53.7125988371847	0.24531036127975\\
53.7191302346919	0.245310578023129\\
53.7256616242363	0.245310794690221\\
53.7321930058207	0.245311011281062\\
53.7387243794479	0.245311227795688\\
53.7452557451208	0.245311444234134\\
53.7517871028421	0.245311660596436\\
53.7583184526147	0.24531187688263\\
53.7648497944413	0.245312093092753\\
53.7713811283249	0.245312309226838\\
53.7779124542682	0.245312525284924\\
53.7844437722739	0.245312741267044\\
53.790975082345	0.245312957173235\\
53.7975063844842	0.245313173003533\\
53.8040376786943	0.245313388757972\\
53.8105689649782	0.24531360443659\\
53.8171002433386	0.24531382003942\\
53.8236315137783	0.2453140355665\\
53.8301627763002	0.245314251017864\\
53.836694030907	0.245314466393548\\
53.8432252776015	0.245314681693588\\
53.8497565163865	0.245314896918019\\
53.8562877472649	0.245315112066876\\
53.8628189702393	0.245315327140195\\
53.8693501853127	0.245315542138012\\
53.8758813924877	0.245315757060361\\
53.8824125917673	0.245315971907279\\
53.8889437831541	0.245316186678801\\
53.8954749666509	0.245316401374961\\
53.9020061422606	0.245316615995796\\
53.9085373099858	0.24531683054134\\
53.9150684698295	0.245317045011629\\
53.9215996217944	0.245317259406699\\
53.9281307658832	0.245317473726584\\
53.9346619020988	0.245317687971319\\
53.9411930304439	0.245317902140941\\
53.9477241509212	0.245318116235484\\
53.9542552635337	0.245318330254983\\
53.9607863682839	0.245318544199473\\
53.9673174651748	0.24531875806899\\
53.9738485542091	0.245318971863568\\
53.9803796353895	0.245319185583244\\
53.9869107087188	0.245319399228051\\
53.9934417741998	0.245319612798025\\
53.9999728318353	0.2453198262932\\
54.0065038816279	0.245320039713613\\
54.0130349235805	0.245320253059298\\
54.0195659576959	0.245320466330289\\
54.0260969839768	0.245320679526623\\
54.0326280024259	0.245320892648332\\
54.039159013046	0.245321105695454\\
54.0456900158399	0.245321318668022\\
54.0522210108103	0.245321531566072\\
54.0587519979599	0.245321744389637\\
54.0652829772916	0.245321957138754\\
54.0718139488081	0.245322169813456\\
54.0783449125121	0.245322382413779\\
54.0848758684063	0.245322594939758\\
54.0914068164936	0.245322807391426\\
54.0979377567767	0.24532301976882\\
54.1044686892582	0.245323232071973\\
54.110999613941	0.24532344430092\\
54.1175305308278	0.245323656455696\\
54.1240614399214	0.245323868536335\\
54.1305923412244	0.245324080542873\\
54.1371232347396	0.245324292475343\\
54.1436541204698	0.245324504333781\\
54.1501849984176	0.24532471611822\\
54.1567158685859	0.245324927828696\\
54.1632467309773	0.245325139465243\\
54.1697775855946	0.245325351027895\\
54.1763084324405	0.245325562516687\\
54.1828392715177	0.245325773931653\\
54.189370102829	0.245325985272829\\
54.1959009263771	0.245326196540247\\
54.2024317421647	0.245326407733943\\
54.2089625501945	0.245326618853951\\
54.2154933504694	0.245326829900305\\
54.2220241429919	0.24532704087304\\
54.2285549277648	0.24532725177219\\
54.2350857047908	0.245327462597789\\
54.2416164740726	0.245327673349872\\
54.248147235613	0.245327884028472\\
54.2546779894147	0.245328094633625\\
54.2612087354804	0.245328305165364\\
54.2677394738128	0.245328515623723\\
54.2742702044145	0.245328726008737\\
54.2808009272884	0.245328936320439\\
54.2873316424371	0.245329146558865\\
54.2938623498634	0.245329356724048\\
54.3003930495698	0.245329566816022\\
54.3069237415592	0.245329776834821\\
54.3134544258343	0.245329986780479\\
54.3199851023977	0.24533019665303\\
54.3265157712522	0.245330406452509\\
54.3330464324004	0.245330616178949\\
54.339577085845	0.245330825832385\\
54.3461077315888	0.24533103541285\\
54.3526383696344	0.245331244920378\\
54.3591689999846	0.245331454355003\\
54.365699622642	0.245331663716759\\
54.3722302376093	0.24533187300568\\
54.3787608448892	0.245332082221799\\
54.3852914444844	0.245332291365152\\
54.3918220363976	0.24533250043577\\
54.3983526206315	0.245332709433689\\
54.4048831971887	0.245332918358942\\
54.4114137660719	0.245333127211563\\
54.4179443272839	0.245333335991585\\
54.4244748808273	0.245333544699042\\
54.4310054267048	0.245333753333968\\
54.4375359649191	0.245333961896397\\
54.4440664954727	0.245334170386362\\
54.4505970183685	0.245334378803897\\
54.4571275336092	0.245334587149036\\
54.4636580411972	0.245334795421811\\
54.4701885411355	0.245335003622258\\
54.4767190334265	0.245335211750408\\
54.483249518073	0.245335419806297\\
54.4897799950777	0.245335627789956\\
54.4963104644432	0.245335835701421\\
54.5028409261722	0.245336043540724\\
54.5093713802673	0.245336251307899\\
54.5159018267313	0.245336459002978\\
54.5224322655668	0.245336666625997\\
54.5289626967764	0.245336874176988\\
54.5354931203628	0.245337081655984\\
54.5420235363286	0.245337289063019\\
54.5485539446766	0.245337496398126\\
54.5550843454094	0.245337703661339\\
54.5616147385296	0.245337910852691\\
54.5681451240399	0.245338117972214\\
54.5746755019429	0.245338325019944\\
54.5812058722414	0.245338531995911\\
54.5877362349379	0.245338738900151\\
54.594266590035	0.245338945732696\\
54.6007969375356	0.245339152493579\\
54.6073272774421	0.245339359182834\\
54.6138576097572	0.245339565800493\\
54.6203879344836	0.24533977234659\\
54.626918251624	0.245339978821158\\
54.6334485611809	0.24534018522423\\
54.639978863157	0.245340391555839\\
54.646509157555	0.245340597816018\\
54.6530394443775	0.2453408040048\\
54.659569723627	0.245341010122219\\
54.6660999953064	0.245341216168306\\
54.6726302594181	0.245341422143096\\
54.6791605159648	0.245341628046621\\
54.6856907649492	0.245341833878914\\
54.6922210063739	0.245342039640009\\
54.6987512402415	0.245342245329937\\
54.7052814665547	0.245342450948732\\
54.711811685316	0.245342656496426\\
54.7183418965282	0.245342861973054\\
54.7248721001937	0.245343067378646\\
54.7314022963153	0.245343272713237\\
54.7379324848956	0.245343477976859\\
54.7444626659372	0.245343683169545\\
54.7509928394427	0.245343888291327\\
54.7575230054147	0.245344093342238\\
54.7640531638558	0.245344298322311\\
54.7705833147687	0.24534450323158\\
54.777113458156	0.245344708070075\\
54.7836435940203	0.24534491283783\\
54.7901737223642	0.245345117534879\\
54.7967038431904	0.245345322161252\\
54.8032339565013	0.245345526716983\\
54.8097640622997	0.245345731202105\\
54.8162941605881	0.24534593561665\\
54.8228242513692	0.24534613996065\\
54.8293543346456	0.245346344234139\\
54.8358844104198	0.245346548437148\\
54.8424144786944	0.245346752569711\\
54.8489445394722	0.245346956631859\\
54.8554745927555	0.245347160623625\\
54.8620046385472	0.245347364545041\\
54.8685346768497	0.245347568396141\\
54.8750647076657	0.245347772176956\\
54.8815947309978	0.245347975887518\\
54.8881247468485	0.245348179527861\\
54.8946547552204	0.245348383098016\\
54.9011847561162	0.245348586598015\\
54.9077147495384	0.245348790027892\\
54.9142447354896	0.245348993387678\\
54.9207747139724	0.245349196677406\\
54.9273046849894	0.245349399897107\\
54.9338346485432	0.245349603046815\\
54.9403646046364	0.245349806126561\\
54.9468945532715	0.245350009136377\\
54.9534244944511	0.245350212076296\\
54.9599544281779	0.24535041494635\\
54.9664843544543	0.24535061774657\\
54.9730142732831	0.24535082047699\\
54.9795441846666	0.245351023137641\\
54.9860740886076	0.245351225728555\\
54.9926039851086	0.245351428249765\\
54.9991338741722	0.245351630701302\\
55.0056637558009	0.245351833083198\\
55.0121936299973	0.245352035395485\\
55.018723496764	0.245352237638196\\
55.0252533561036	0.245352439811363\\
55.0317832080186	0.245352641915017\\
55.0383130525116	0.245352843949189\\
55.0448428895851	0.245353045913914\\
55.0513727192418	0.245353247809221\\
55.0579025414842	0.245353449635143\\
55.0644323563148	0.245353651391712\\
55.0709621637363	0.245353853078959\\
55.0774919637511	0.245354054696917\\
55.0840217563618	0.245354256245618\\
55.090551541571	0.245354457725092\\
55.0970813193813	0.245354659135372\\
55.1036110897952	0.24535486047649\\
55.1101408528152	0.245355061748477\\
55.1166706084439	0.245355262951366\\
55.1232003566839	0.245355464085187\\
55.1297300975377	0.245355665149972\\
55.1362598310078	0.245355866145753\\
55.1427895570969	0.245356067072562\\
55.1493192758074	0.245356267930431\\
55.1558489871419	0.24535646871939\\
55.162378691103	0.245356669439472\\
55.1689083876931	0.245356870090708\\
55.1754380769149	0.24535707067313\\
55.1819677587708	0.245357271186769\\
55.1884974332634	0.245357471631657\\
55.1950271003953	0.245357672007825\\
55.201556760169	0.245357872315304\\
55.208086412587	0.245358072554127\\
55.2146160576519	0.245358272724324\\
55.2211456953661	0.245358472825928\\
55.2276753257323	0.245358672858968\\
55.2342049487529	0.245358872823478\\
55.2407345644305	0.245359072719488\\
55.2472641727677	0.245359272547029\\
55.2537937737669	0.245359472306133\\
55.2603233674306	0.245359671996831\\
55.2668529537614	0.245359871619155\\
55.2733825327619	0.245360071173136\\
55.2799121044345	0.245360270658805\\
55.2864416687818	0.245360470076193\\
55.2929712258063	0.245360669425331\\
55.2995007755105	0.245360868706251\\
55.3060303178969	0.245361067918984\\
55.312559852968	0.245361267063561\\
55.3190893807264	0.245361466140014\\
55.3256189011746	0.245361665148373\\
55.3321484143151	0.245361864088669\\
55.3386779201504	0.245362062960934\\
55.345207418683	0.245362261765199\\
55.3517369099154	0.245362460501494\\
55.3582663938502	0.245362659169851\\
55.3647958704898	0.245362857770301\\
55.3713253398368	0.245363056302875\\
55.3778548018936	0.245363254767604\\
55.3843842566628	0.245363453164518\\
55.3909137041469	0.24536365149365\\
55.3974431443483	0.245363849755029\\
55.4039725772696	0.245364047948686\\
55.4105020029133	0.245364246074654\\
55.4170314212818	0.245364444132961\\
55.4235608323777	0.245364642123641\\
55.4300902362035	0.245364840046722\\
55.4366196327617	0.245365037902236\\
55.4431490220547	0.245365235690214\\
55.449678404085	0.245365433410687\\
55.4562077788552	0.245365631063685\\
55.4627371463678	0.245365828649239\\
55.4692665066251	0.245366026167381\\
55.4757958596298	0.24536622361814\\
55.4823252053843	0.245366421001547\\
55.4888545438911	0.245366618317634\\
55.4953838751526	0.24536681556643\\
55.5019131991714	0.245367012747966\\
55.5084425159499	0.245367209862274\\
55.5149718254907	0.245367406909384\\
55.5215011277962	0.245367603889325\\
55.5280304228688	0.24536780080213\\
55.5345597107111	0.245367997647828\\
55.5410889913256	0.24536819442645\\
55.5476182647147	0.245368391138027\\
55.5541475308808	0.245368587782588\\
55.5606767898266	0.245368784360166\\
55.5672060415543	0.245368980870789\\
55.5737352860666	0.245369177314489\\
55.5802645233659	0.245369373691296\\
55.5867937534546	0.24536957000124\\
55.5933229763352	0.245369766244353\\
55.5998521920103	0.245369962420663\\
55.6063814004822	0.245370158530202\\
55.6129106017534	0.245370354573\\
55.6194397958263	0.245370550549088\\
55.6259689827036	0.245370746458495\\
55.6324981623875	0.245370942301252\\
55.6390273348806	0.245371138077389\\
55.6455565001854	0.245371333786937\\
55.6520856583042	0.245371529429926\\
55.6586148092395	0.245371725006386\\
55.6651439529939	0.245371920516347\\
55.6716730895697	0.24537211595984\\
55.6782022189694	0.245372311336894\\
55.6847313411955	0.245372506647541\\
55.6912604562504	0.245372701891809\\
55.6977895641366	0.24537289706973\\
55.7043186648565	0.245373092181332\\
55.7108477584125	0.245373287226648\\
55.7173768448072	0.245373482205705\\
55.7239059240429	0.245373677118536\\
55.7304349961221	0.245373871965169\\
55.7369640610472	0.245374066745634\\
55.7434931188207	0.245374261459962\\
55.7500221694451	0.245374456108183\\
55.7565512129227	0.245374650690327\\
55.763080249256	0.245374845206424\\
55.7696092784475	0.245375039656502\\
55.7761383004995	0.245375234040594\\
55.7826673154146	0.245375428358728\\
55.7891963231951	0.245375622610934\\
55.7957253238435	0.245375816797243\\
55.8022543173623	0.245376010917684\\
55.8087833037537	0.245376204972286\\
55.8153122830204	0.245376398961081\\
55.8218412551647	0.245376592884097\\
55.828370220189	0.245376786741364\\
55.8348991780958	0.245376980532913\\
55.8414281288874	0.245377174258773\\
55.8479570725664	0.245377367918973\\
55.8544860091351	0.245377561513544\\
55.861014938596	0.245377755042515\\
55.8675438609515	0.245377948505916\\
55.874072776204	0.245378141903777\\
55.8806016843559	0.245378335236127\\
55.8871305854096	0.245378528502995\\
55.8936594793676	0.245378721704413\\
55.9001883662323	0.245378914840408\\
55.9067172460061	0.245379107911011\\
55.9132461186913	0.245379300916251\\
55.9197749842905	0.245379493856158\\
55.926303842806	0.245379686730762\\
55.9328326942403	0.245379879540091\\
55.9393615385957	0.245380072284176\\
55.9458903758747	0.245380264963046\\
55.9524192060796	0.245380457576731\\
55.9589480292129	0.245380650125259\\
55.965476845277	0.245380842608661\\
55.9720056542742	0.245381035026965\\
55.9785344562071	0.245381227380202\\
55.9850632510779	0.2453814196684\\
55.9915920388891	0.24538161189159\\
55.9981208196431	0.245381804049799\\
56.0046495933423	0.245381996143059\\
56.011178359989	0.245382188171397\\
56.0177071195858	0.245382380134844\\
56.0242358721349	0.245382572033428\\
56.0307646176387	0.245382763867179\\
56.0372933560998	0.245382955636127\\
56.0438220875204	0.2453831473403\\
56.0503508119029	0.245383338979727\\
56.0568795292497	0.245383530554439\\
56.0634082395633	0.245383722064463\\
56.069936942846	0.24538391350983\\
56.0764656391002	0.245384104890568\\
56.0829943283283	0.245384296206707\\
56.0895230105326	0.245384487458276\\
56.0960516857156	0.245384678645303\\
56.1025803538796	0.245384869767819\\
56.109109015027	0.245385060825851\\
56.1156376691602	0.245385251819429\\
56.1221663162816	0.245385442748583\\
56.1286949563935	0.245385633613341\\
56.1352235894984	0.245385824413732\\
56.1417522155985	0.245386015149785\\
56.1482808346964	0.245386205821529\\
56.1548094467943	0.245386396428994\\
56.1613380518946	0.245386586972207\\
56.1678666499997	0.245386777451199\\
56.1743952411119	0.245386967865997\\
56.1809238252337	0.245387158216632\\
56.1874524023674	0.245387348503131\\
56.1939809725154	0.245387538725524\\
56.20050953568	0.245387728883839\\
56.2070380918636	0.245387918978106\\
56.2135666410686	0.245388109008352\\
56.2200951832973	0.245388298974608\\
56.226623718552	0.245388488876902\\
56.2331522468353	0.245388678715262\\
56.2396807681493	0.245388868489717\\
56.2462092824965	0.245389058200296\\
56.2527377898792	0.245389247847028\\
56.2592662902998	0.245389437429941\\
56.2657947837607	0.245389626949065\\
56.2723232702641	0.245389816404427\\
};
\addplot [color=green,solid,forget plot]
  table[row sep=crcr]{%
56.2723232702641	0.245389816404427\\
56.2788517498125	0.245390005796057\\
56.2853802224082	0.245390195123983\\
56.2919086880535	0.245390384388234\\
56.2984371467508	0.245390573588838\\
56.3049655985025	0.245390762725824\\
56.3114940433108	0.24539095179922\\
56.3180224811782	0.245391140809056\\
56.324550912107	0.245391329755359\\
56.3310793360995	0.245391518638159\\
56.3376077531581	0.245391707457483\\
56.3441361632852	0.24539189621336\\
56.3506645664829	0.245392084905819\\
56.3571929627538	0.245392273534888\\
56.3637213521001	0.245392462100596\\
56.3702497345242	0.245392650602971\\
56.3767781100284	0.245392839042041\\
56.3833064786151	0.245393027417835\\
56.3898348402865	0.245393215730381\\
56.3963631950451	0.245393403979708\\
56.4028915428931	0.245393592165844\\
56.4094198838329	0.245393780288818\\
56.4159482178668	0.245393968348656\\
56.4224765449972	0.245394156345389\\
56.4290048652263	0.245394344279044\\
56.4355331785566	0.24539453214965\\
56.4420614849902	0.245394719957234\\
56.4485897845297	0.245394907701826\\
56.4551180771772	0.245395095383452\\
56.4616463629351	0.245395283002142\\
56.4681746418058	0.245395470557924\\
56.4747029137915	0.245395658050826\\
56.4812311788945	0.245395845480875\\
56.4877594371173	0.245396032848101\\
56.494287688462	0.245396220152531\\
56.5008159329311	0.245396407394193\\
56.5073441705269	0.245396594573116\\
56.5138724012515	0.245396781689328\\
56.5204006251075	0.245396968742856\\
56.526928842097	0.245397155733729\\
56.5334570522225	0.245397342661974\\
56.5399852554861	0.245397529527621\\
56.5465134518903	0.245397716330696\\
56.5530416414373	0.245397903071228\\
56.5595698241294	0.245398089749245\\
56.5660979999689	0.245398276364774\\
56.5726261689582	0.245398462917844\\
56.5791543310996	0.245398649408483\\
56.5856824863953	0.245398835836718\\
56.5922106348477	0.245399022202578\\
56.598738776459	0.24539920850609\\
56.6052669112316	0.245399394747282\\
56.6117950391678	0.245399580926182\\
56.6183231602698	0.245399767042818\\
56.6248512745399	0.245399953097218\\
56.6313793819806	0.245400139089409\\
56.6379074825939	0.245400325019419\\
56.6444355763824	0.245400510887276\\
56.6509636633481	0.245400696693009\\
56.6574917434935	0.245400882436643\\
56.6640198168208	0.245401068118208\\
56.6705478833323	0.245401253737731\\
56.6770759430303	0.24540143929524\\
56.6836039959171	0.245401624790762\\
56.690132041995	0.245401810224325\\
56.6966600812663	0.245401995595956\\
56.7031881137332	0.245402180905684\\
56.709716139398	0.245402366153536\\
56.716244158263	0.245402551339539\\
56.7227721703306	0.245402736463722\\
56.7293001756029	0.245402921526111\\
56.7358281740823	0.245403106526734\\
56.742356165771	0.245403291465619\\
56.7488841506713	0.245403476342794\\
56.7554121287856	0.245403661158285\\
56.761940100116	0.24540384591212\\
56.7684680646648	0.245404030604328\\
56.7749960224344	0.245404215234934\\
56.7815239734269	0.245404399803968\\
56.7880519176447	0.245404584311455\\
56.7945798550901	0.245404768757424\\
56.8011077857652	0.245404953141902\\
56.8076357096725	0.245405137464916\\
56.814163626814	0.245405321726494\\
56.8206915371922	0.245405505926663\\
56.8272194408092	0.245405690065451\\
56.8337473376674	0.245405874142884\\
56.840275227769	0.24540605815899\\
56.8468031111163	0.245406242113797\\
56.8533309877115	0.245406426007331\\
56.8598588575568	0.24540660983962\\
56.8663867206547	0.245406793610692\\
56.8729145770072	0.245406977320573\\
56.8794424266167	0.24540716096929\\
56.8859702694854	0.245407344556871\\
56.8924981056156	0.245407528083343\\
56.8990259350095	0.245407711548734\\
56.9055537576694	0.245407894953069\\
56.9120815735976	0.245408078296378\\
56.9186093827962	0.245408261578685\\
56.9251371852676	0.24540844480002\\
56.9316649810139	0.245408627960408\\
56.9381927700375	0.245408811059878\\
56.9447205523406	0.245408994098455\\
56.9512483279254	0.245409177076167\\
56.9577760967942	0.245409359993042\\
56.9643038589493	0.245409542849105\\
56.9708316143928	0.245409725644385\\
56.977359363127	0.245409908378907\\
56.9838871051541	0.2454100910527\\
56.9904148404765	0.24541027366579\\
56.9969425690962	0.245410456218204\\
57.0034702910157	0.245410638709969\\
57.009998006237	0.245410821141111\\
57.0165257147626	0.245411003511659\\
57.0230534165944	0.245411185821638\\
57.0295811117349	0.245411368071076\\
57.0361088001863	0.245411550259999\\
57.0426364819507	0.245411732388434\\
57.0491641570304	0.245411914456408\\
57.0556918254277	0.245412096463948\\
57.0622194871448	0.245412278411081\\
57.0687471421838	0.245412460297833\\
57.0752747905471	0.245412642124232\\
57.0818024322369	0.245412823890303\\
57.0883300672553	0.245413005596074\\
57.0948576956046	0.245413187241572\\
57.1013853172871	0.245413368826823\\
57.1079129323049	0.245413550351853\\
57.1144405406603	0.245413731816691\\
57.1209681423555	0.245413913221361\\
57.1274957373927	0.245414094565891\\
57.1340233257741	0.245414275850308\\
57.140550907502	0.245414457074638\\
57.1470784825786	0.245414638238907\\
57.153606051006	0.245414819343143\\
57.1601336127866	0.245415000387372\\
57.1666611679224	0.24541518137162\\
57.1731887164158	0.245415362295915\\
57.1797162582689	0.245415543160281\\
57.186243793484	0.245415723964747\\
57.1927713220632	0.245415904709338\\
57.1992988440088	0.245416085394082\\
57.205826359323	0.245416266019004\\
57.2123538680079	0.245416446584131\\
57.2188813700658	0.245416627089489\\
57.225408865499	0.245416807535105\\
57.2319363543095	0.245416987921006\\
57.2384638364996	0.245417168247217\\
57.2449913120715	0.245417348513765\\
57.2515187810275	0.245417528720677\\
57.2580462433696	0.245417708867978\\
57.2645736991001	0.245417888955696\\
57.2711011482213	0.245418068983856\\
57.2776285907352	0.245418248952485\\
57.2841560266441	0.245418428861608\\
57.2906834559503	0.245418608711254\\
57.2972108786558	0.245418788501447\\
57.3037382947628	0.245418968232213\\
57.3102657042737	0.24541914790358\\
57.3167931071905	0.245419327515573\\
57.3233205035154	0.245419507068219\\
57.3298478932507	0.245419686561544\\
57.3363752763985	0.245419865995573\\
57.3429026529611	0.245420045370334\\
57.3494300229405	0.245420224685852\\
57.355957386339	0.245420403942153\\
57.3624847431588	0.245420583139264\\
57.3690120934021	0.24542076227721\\
57.375539437071	0.245420941356019\\
57.3820667741677	0.245421120375715\\
57.3885941046944	0.245421299336325\\
57.3951214286533	0.245421478237875\\
57.4016487460465	0.245421657080391\\
57.4081760568763	0.245421835863898\\
57.4147033611448	0.245422014588424\\
57.4212306588541	0.245422193253994\\
57.4277579500066	0.245422371860634\\
57.4342852346043	0.24542255040837\\
57.4408125126493	0.245422728897228\\
57.447339784144	0.245422907327233\\
57.4538670490904	0.245423085698413\\
57.4603943074908	0.245423264010792\\
57.4669215593472	0.245423442264397\\
57.4734488046619	0.245423620459253\\
57.479976043437	0.245423798595386\\
57.4865032756747	0.245423976672823\\
57.4930305013772	0.245424154691589\\
57.4995577205465	0.245424332651709\\
57.506084933185	0.245424510553211\\
57.5126121392947	0.245424688396118\\
57.5191393388779	0.245424866180458\\
57.5256665319366	0.245425043906256\\
57.532193718473	0.245425221573538\\
57.5387208984893	0.245425399182329\\
57.5452480719876	0.245425576732655\\
57.5517752389702	0.245425754224543\\
57.5583023994391	0.245425931658017\\
57.5648295533965	0.245426109033103\\
57.5713567008446	0.245426286349827\\
57.5778838417855	0.245426463608215\\
57.5844109762214	0.245426640808292\\
57.5909381041544	0.245426817950084\\
57.5974652255867	0.245426995033617\\
57.6039923405204	0.245427172058915\\
57.6105194489576	0.245427349026005\\
57.6170465509006	0.245427525934913\\
57.6235736463514	0.245427702785663\\
57.6301007353122	0.245427879578282\\
57.6366278177852	0.245428056312795\\
57.6431548937725	0.245428232989227\\
57.6496819632762	0.245428409607604\\
57.6562090262984	0.245428586167951\\
57.6627360828414	0.245428762670294\\
57.6692631329073	0.245428939114659\\
57.6757901764982	0.24542911550107\\
57.6823172136161	0.245429291829553\\
57.6888442442634	0.245429468100134\\
57.695371268442	0.245429644312838\\
57.7018982861542	0.24542982046769\\
57.7084252974021	0.245429996564717\\
57.7149523021878	0.245430172603942\\
57.7214793005134	0.245430348585391\\
57.7280062923811	0.245430524509091\\
57.7345332777929	0.245430700375066\\
57.7410602567511	0.245430876183341\\
57.7475872292578	0.245431051933941\\
57.7541141953151	0.245431227626893\\
57.760641154925	0.245431403262221\\
57.7671681080898	0.24543157883995\\
57.7736950548115	0.245431754360106\\
57.7802219950924	0.245431929822714\\
57.7867489289344	0.2454321052278\\
57.7932758563398	0.245432280575387\\
57.7998027773106	0.245432455865502\\
57.8063296918489	0.24543263109817\\
57.812856599957	0.245432806273416\\
57.8193835016369	0.245432981391264\\
57.8259103968907	0.245433156451741\\
57.8324372857205	0.245433331454871\\
57.8389641681285	0.245433506400679\\
57.8454910441167	0.245433681289191\\
57.8520179136873	0.245433856120431\\
57.8585447768424	0.245434030894425\\
57.8650716335841	0.245434205611197\\
57.8715984839146	0.245434380270773\\
57.8781253278358	0.245434554873177\\
57.88465216535	0.245434729418435\\
57.8911789964592	0.245434903906571\\
57.8977058211656	0.245435078337611\\
57.9042326394712	0.24543525271158\\
57.9107594513782	0.245435427028502\\
57.9172862568886	0.245435601288403\\
57.9238130560046	0.245435775491306\\
57.9303398487282	0.245435949637238\\
57.9368666350617	0.245436123726223\\
57.9433934150069	0.245436297758286\\
57.9499201885662	0.245436471733452\\
57.9564469557415	0.245436645651746\\
57.962973716535	0.245436819513192\\
57.9695004709487	0.245436993317815\\
57.9760272189848	0.245437167065641\\
57.9825539606453	0.245437340756694\\
57.9890806959324	0.245437514390998\\
57.9956074248481	0.245437687968579\\
58.0021341473946	0.245437861489461\\
58.0086608635738	0.245438034953669\\
58.015187573388	0.245438208361228\\
58.0217142768392	0.245438381712162\\
58.0282409739294	0.245438555006497\\
58.0347676646608	0.245438728244256\\
58.0412943490355	0.245438901425465\\
58.0478210270555	0.245439074550148\\
58.054347698723	0.24543924761833\\
58.06087436404	0.245439420630035\\
58.0674010230085	0.245439593585289\\
58.0739276756307	0.245439766484115\\
58.0804543219087	0.245439939326538\\
58.0869809618446	0.245440112112583\\
58.0935075954403	0.245440284842275\\
58.1000342226981	0.245440457515637\\
58.1065608436199	0.245440630132696\\
58.1130874582079	0.245440802693474\\
58.1196140664641	0.245440975197996\\
58.1261406683905	0.245441147646288\\
58.1326672639894	0.245441320038373\\
58.1391938532627	0.245441492374277\\
58.1457204362125	0.245441664654023\\
58.152247012841	0.245441836877635\\
58.15877358315	0.245442009045139\\
58.1653001471418	0.245442181156559\\
58.1718267048184	0.245442353211919\\
58.1783532561819	0.245442525211244\\
58.1848798012343	0.245442697154557\\
58.1914063399776	0.245442869041884\\
58.1979328724141	0.245443040873249\\
58.2044593985457	0.245443212648675\\
58.2109859183744	0.245443384368188\\
58.2175124319024	0.245443556031812\\
58.2240389391317	0.24544372763957\\
58.2305654400644	0.245443899191488\\
58.2370919347025	0.245444070687589\\
58.2436184230481	0.245444242127898\\
58.2501449051032	0.245444413512439\\
58.2566713808699	0.245444584841236\\
58.2631978503503	0.245444756114314\\
58.2697243135464	0.245444927331696\\
58.2762507704603	0.245445098493407\\
58.2827772210939	0.245445269599472\\
58.2893036654495	0.245445440649913\\
58.295830103529	0.245445611644756\\
58.3023565353344	0.245445782584025\\
58.3088829608678	0.245445953467743\\
58.3154093801314	0.245446124295935\\
58.321935793127	0.245446295068624\\
58.3284621998568	0.245446465785836\\
58.3349886003228	0.245446636447594\\
58.3415149945271	0.245446807053922\\
58.3480413824716	0.245446977604843\\
58.3545677641585	0.245447148100384\\
58.3610941395898	0.245447318540566\\
58.3676205087675	0.245447488925414\\
58.3741468716937	0.245447659254953\\
58.3806732283703	0.245447829529206\\
58.3871995787995	0.245447999748197\\
58.3937259229833	0.245448169911951\\
58.4002522609237	0.24544834002049\\
58.4067785926228	0.245448510073839\\
58.4133049180825	0.245448680072023\\
58.419831237305	0.245448850015064\\
58.4263575502922	0.245449019902986\\
58.4328838570462	0.245449189735815\\
58.439410157569	0.245449359513573\\
58.4459364518627	0.245449529236284\\
58.4524627399292	0.245449698903973\\
58.4589890217706	0.245449868516662\\
58.465515297389	0.245450038074376\\
58.4720415667864	0.245450207577139\\
58.4785678299647	0.245450377024974\\
58.485094086926	0.245450546417906\\
58.4916203376724	0.245450715755957\\
58.4981465822059	0.245450885039153\\
58.5046728205284	0.245451054267515\\
58.511199052642	0.245451223441069\\
58.5177252785488	0.245451392559838\\
58.5242514982507	0.245451561623845\\
58.5307777117497	0.245451730633114\\
58.537303919048	0.24545189958767\\
58.5438301201474	0.245452068487535\\
58.5503563150501	0.245452237332733\\
58.556882503758	0.245452406123289\\
58.5634086862732	0.245452574859224\\
58.5699348625976	0.245452743540564\\
58.5764610327333	0.245452912167332\\
58.5829871966822	0.245453080739551\\
58.5895133544465	0.245453249257245\\
58.5960395060281	0.245453417720437\\
58.602565651429	0.245453586129151\\
58.6090917906512	0.245453754483411\\
58.6156179236968	0.24545392278324\\
58.6221440505677	0.245454091028661\\
58.6286701712659	0.245454259219698\\
58.6351962857935	0.245454427356375\\
58.6417223941525	0.245454595438715\\
58.6482484963448	0.245454763466742\\
58.6547745923725	0.245454931440478\\
58.6613006822375	0.245455099359948\\
58.6678267659419	0.245455267225174\\
58.6743528434877	0.245455435036181\\
58.6808789148768	0.245455602792991\\
58.6874049801113	0.245455770495628\\
58.6939310391932	0.245455938144115\\
58.7004570921244	0.245456105738476\\
58.7069831389069	0.245456273278734\\
58.7135091795428	0.245456440764913\\
58.7200352140341	0.245456608197035\\
58.7265612423827	0.245456775575124\\
58.7330872645906	0.245456942899203\\
58.7396132806598	0.245457110169296\\
58.7461392905923	0.245457277385425\\
58.7526652943901	0.245457444547615\\
58.7591912920552	0.245457611655888\\
58.7657172835896	0.245457778710268\\
58.7722432689952	0.245457945710777\\
58.7787692482741	0.245458112657439\\
58.7852952214282	0.245458279550277\\
58.7918211884595	0.245458446389315\\
58.7983471493701	0.245458613174575\\
58.8048731041618	0.245458779906081\\
58.8113990528366	0.245458946583856\\
58.8179249953966	0.245459113207923\\
58.8244509318438	0.245459279778304\\
58.83097686218	0.245459446295024\\
58.8375027864073	0.245459612758105\\
58.8440287045277	0.245459779167571\\
58.8505546165431	0.245459945523444\\
58.8570805224555	0.245460111825747\\
58.8636064222669	0.245460278074504\\
58.8701323159793	0.245460444269737\\
58.8766582035946	0.24546061041147\\
58.8831840851148	0.245460776499725\\
58.8897099605419	0.245460942534526\\
58.8962358298778	0.245461108515896\\
58.9027616931246	0.245461274443857\\
58.9092875502841	0.245461440318432\\
58.9158134013584	0.245461606139645\\
58.9223392463494	0.245461771907518\\
58.9288650852591	0.245461937622075\\
58.9353909180894	0.245462103283337\\
58.9419167448424	0.245462268891329\\
58.9484425655199	0.245462434446072\\
58.954968380124	0.245462599947591\\
58.9614941886566	0.245462765395907\\
58.9680199911197	0.245462930791043\\
58.9745457875152	0.245463096133023\\
58.9810715778451	0.245463261421869\\
58.9875973621113	0.245463426657604\\
58.9941231403158	0.245463591840251\\
59.0006489124606	0.245463756969833\\
59.0071746785476	0.245463922046371\\
59.0137004385788	0.24546408706989\\
59.0202261925561	0.245464252040412\\
59.0267519404815	0.245464416957959\\
59.033277682357	0.245464581822554\\
59.0398034181844	0.245464746634221\\
59.0463291479657	0.245464911392981\\
59.052854871703	0.245465076098858\\
59.0593805893981	0.245465240751873\\
59.0659063010529	0.245465405352051\\
59.0724320066695	0.245465569899413\\
59.0789577062498	0.245465734393981\\
59.0854833997957	0.24546589883578\\
59.0920090873092	0.245466063224831\\
59.0985347687922	0.245466227561156\\
59.1050604442467	0.245466391844779\\
59.1115861136745	0.245466556075722\\
59.1181117770777	0.245466720254008\\
59.1246374344582	0.245466884379659\\
59.1311630858179	0.245467048452697\\
59.1376887311588	0.245467212473146\\
59.1442143704828	0.245467376441027\\
59.1507400037918	0.245467540356364\\
59.1572656310879	0.245467704219178\\
59.1637912523728	0.245467868029493\\
59.1703168676485	0.24546803178733\\
59.1768424769171	0.245468195492712\\
59.1833680801804	0.245468359145662\\
59.1898936774403	0.245468522746202\\
59.1964192686988	0.245468686294355\\
59.2029448539578	0.245468849790142\\
59.2094704332193	0.245469013233586\\
59.2159960064851	0.24546917662471\\
59.2225215737573	0.245469339963536\\
59.2290471350376	0.245469503250087\\
59.2355726903282	0.245469666484384\\
59.2420982396308	0.24546982966645\\
59.2486237829474	0.245469992796308\\
59.25514932028	0.245470155873979\\
59.2616748516304	0.245470318899486\\
59.2682003770005	0.245470481872852\\
59.2747258963924	0.245470644794098\\
59.2812514098079	0.245470807663247\\
59.287776917249	0.245470970480321\\
59.2943024187175	0.245471133245342\\
59.3008279142154	0.245471295958333\\
59.3073534037446	0.245471458619315\\
59.313878887307	0.245471621228311\\
59.3204043649045	0.245471783785344\\
59.3269298365391	0.245471946290435\\
59.3334553022126	0.245472108743606\\
59.3399807619271	0.24547227114488\\
59.3465062156843	0.245472433494279\\
59.3530316634862	0.245472595791825\\
59.3595571053347	0.24547275803754\\
59.3660825412317	0.245472920231446\\
59.3726079711792	0.245473082373565\\
59.379133395179	0.24547324446392\\
59.3856588132331	0.245473406502532\\
59.3921842253433	0.245473568489424\\
59.3987096315116	0.245473730424617\\
59.4052350317398	0.245473892308134\\
59.4117604260299	0.245474054139996\\
59.4182858143838	0.245474215920226\\
59.4248111968034	0.245474377648846\\
59.4313365732905	0.245474539325877\\
59.4378619438471	0.245474700951342\\
59.4443873084752	0.245474862525262\\
59.4509126671765	0.24547502404766\\
59.4574380199529	0.245475185518558\\
59.4639633668065	0.245475346937977\\
59.470488707739	0.245475508305939\\
59.4770140427524	0.245475669622467\\
59.4835393718486	0.245475830887581\\
59.4900646950294	0.245475992101305\\
59.4965900122968	0.24547615326366\\
59.5031153236527	0.245476314374667\\
59.5096406290989	0.245476475434349\\
59.5161659286373	0.245476636442728\\
59.5226912222699	0.245476797399825\\
59.5292165099984	0.245476958305661\\
59.5357417918249	0.24547711916026\\
59.5422670677512	0.245477279963643\\
59.5487923377792	0.245477440715831\\
59.5553176019107	0.245477601416846\\
59.5618428601477	0.245477762066711\\
59.568368112492	0.245477922665446\\
59.5748933589455	0.245478083213074\\
59.5814185995102	0.245478243709616\\
59.5879438341878	0.245478404155094\\
59.5944690629804	0.245478564549529\\
59.6009942858896	0.245478724892944\\
59.6075195029175	0.24547888518536\\
59.614044714066	0.245479045426799\\
59.6205699193368	0.245479205617282\\
59.6270951187319	0.245479365756831\\
59.6336203122531	0.245479525845467\\
59.6401454999024	0.245479685883213\\
59.6466706816815	0.24547984587009\\
59.6531958575925	0.245480005806119\\
59.659721027637	0.245480165691323\\
59.6662461918171	0.245480325525721\\
59.6727713501346	0.245480485309337\\
59.6792965025914	0.245480645042192\\
59.6858216491893	0.245480804724307\\
59.6923467899302	0.245480964355704\\
59.698871924816	0.245481123936404\\
59.7053970538485	0.245481283466429\\
59.7119221770297	0.245481442945801\\
59.7184472943613	0.24548160237454\\
59.7249724058452	0.245481761752668\\
59.7314975114834	0.245481921080208\\
59.7380226112777	0.245482080357179\\
59.7445477052298	0.245482239583604\\
59.7510727933418	0.245482398759504\\
59.7575978756155	0.245482557884901\\
59.7641229520527	0.245482716959815\\
59.7706480226552	0.245482875984269\\
59.777173087425	0.245483034958283\\
59.7836981463639	0.245483193881879\\
59.7902231994738	0.245483352755079\\
59.7967482467564	0.245483511577903\\
59.8032732882138	0.245483670350374\\
59.8097983238477	0.245483829072512\\
59.8163233536599	0.245483987744338\\
59.8228483776525	0.245484146365875\\
59.8293733958271	0.245484304937143\\
59.8358984081856	0.245484463458163\\
59.8424234147299	0.245484621928958\\
59.8489484154619	0.245484780349547\\
59.8554734103834	0.245484938719953\\
59.8619983994963	0.245485097040196\\
59.8685233828023	0.245485255310298\\
59.8750483603034	0.24548541353028\\
59.8815733320013	0.245485571700163\\
59.888098297898	0.245485729819969\\
59.8946232579953	0.245485887889718\\
59.901148212295	0.245486045909432\\
59.907673160799	0.245486203879132\\
59.9141981035091	0.245486361798839\\
59.9207230404271	0.245486519668574\\
59.927247971555	0.245486677488359\\
59.9337728968945	0.245486835258214\\
59.9402978164474	0.24548699297816\\
59.9468227302157	0.245487150648219\\
59.9533476382011	0.245487308268412\\
59.9598725404055	0.245487465838759\\
59.9663974368308	0.245487623359282\\
59.9729223274787	0.245487780830002\\
59.9794472123511	0.24548793825094\\
59.9859720914499	0.245488095622117\\
59.9924969647768	0.245488252943554\\
59.9990218323337	0.245488410215271\\
60.0055466941225	0.245488567437291\\
60.0120715501449	0.245488724609633\\
60.0185964004029	0.245488881732319\\
60.0251212448981	0.24548903880537\\
60.0316460836325	0.245489195828807\\
60.0381709166079	0.24548935280265\\
60.0446957438261	0.245489509726921\\
60.051220565289	0.245489666601641\\
60.0577453809983	0.24548982342683\\
60.0642701909558	0.245489980202509\\
60.0707949951635	0.245490136928699\\
60.0773197936232	0.245490293605422\\
60.0838445863366	0.245490450232697\\
60.0903693733056	0.245490606810546\\
60.0968941545319	0.24549076333899\\
60.1034189300176	0.245490919818049\\
60.1099436997642	0.245491076247744\\
60.1164684637737	0.245491232628096\\
60.1229932220479	0.245491388959126\\
60.1295179745886	0.245491545240855\\
60.1360427213976	0.245491701473302\\
60.1425674624767	0.24549185765649\\
60.1490921978278	0.245492013790439\\
60.1556169274526	0.24549216987517\\
60.1621416513531	0.245492325910702\\
60.1686663695309	0.245492481897058\\
60.1751910819879	0.245492637834258\\
60.1817157887259	0.245492793722321\\
60.1882404897467	0.24549294956127\\
60.1947651850521	0.245493105351125\\
60.201289874644	0.245493261091906\\
60.2078145585242	0.245493416783634\\
60.2143392366944	0.24549357242633\\
60.2208639091564	0.245493728020014\\
60.2273885759121	0.245493883564708\\
60.2339132369633	0.24549403906043\\
60.2404378923118	0.245494194507203\\
60.2469625419593	0.245494349905047\\
60.2534871859078	0.245494505253982\\
60.2600118241589	0.245494660554028\\
60.2665364567145	0.245494815805208\\
60.2730610835764	0.24549497100754\\
60.2795857047464	0.245495126161045\\
60.2861103202262	0.245495281265745\\
60.2926349300178	0.245495436321659\\
60.2991595341228	0.245495591328808\\
60.3056841325431	0.245495746287212\\
60.3122087252805	0.245495901196893\\
60.3187333123367	0.24549605605787\\
60.3252578937136	0.245496210870163\\
60.331782469413	0.245496365633794\\
60.3383070394366	0.245496520348783\\
60.3448316037863	0.24549667501515\\
60.3513561624638	0.245496829632916\\
60.3578807154709	0.2454969842021\\
60.3644052628095	0.245497138722724\\
60.3709298044812	0.245497293194808\\
60.3774543404879	0.245497447618371\\
60.3839788708314	0.245497601993436\\
60.3905033955135	0.245497756320021\\
60.397027914536	0.245497910598147\\
60.4035524279005	0.245498064827834\\
60.410076935609	0.245498219009104\\
60.4166014376632	0.245498373141975\\
60.4231259340649	0.245498527226469\\
60.4296504248159	0.245498681262606\\
60.4361749099179	0.245498835250405\\
60.4426993893728	0.245498989189888\\
60.4492238631823	0.245499143081074\\
60.4557483313481	0.245499296923984\\
60.4622727938722	0.245499450718637\\
60.4687972507562	0.245499604465055\\
60.4753217020019	0.245499758163257\\
60.4818461476111	0.245499911813264\\
60.4883705875857	0.245500065415095\\
60.4948950219272	0.245500218968771\\
60.5014194506376	0.245500372474313\\
60.5079438737186	0.245500525931739\\
60.514468291172	0.245500679341071\\
60.5209927029995	0.245500832702328\\
60.527517109203	0.245500986015531\\
60.5340415097841	0.2455011392807\\
60.5405659047447	0.245501292497855\\
60.5470902940865	0.245501445667015\\
60.5536146778114	0.245501598788201\\
60.5601390559209	0.245501751861434\\
60.5666634284171	0.245501904886733\\
60.5731877953015	0.245502057864117\\
60.5797121565759	0.245502210793608\\
60.5862365122422	0.245502363675226\\
60.5927608623021	0.245502516508989\\
60.5992852067574	0.245502669294919\\
60.6058095456097	0.245502822033035\\
60.612333878861	0.245502974723357\\
60.6188582065128	0.245503127365906\\
60.6253825285671	0.2455032799607\\
60.6319068450256	0.245503432507761\\
60.6384311558899	0.245503585007108\\
60.6449554611619	0.245503737458761\\
60.6514797608434	0.24550388986274\\
60.658004054936	0.245504042219064\\
60.6645283434416	0.245504194527755\\
60.6710526263619	0.245504346788831\\
60.6775769036986	0.245504499002313\\
60.6841011754535	0.24550465116822\\
60.6906254416284	0.245504803286572\\
60.697149702225	0.245504955357389\\
60.7036739572451	0.245505107380692\\
60.7101982066903	0.245505259356499\\
60.7167224505626	0.245505411284831\\
60.7232466888635	0.245505563165707\\
60.7297709215949	0.245505714999147\\
60.7362951487585	0.245505866785172\\
60.7428193703561	0.2455060185238\\
60.7493435863893	0.245506170215052\\
60.75586779686	0.245506321858947\\
60.7623920017699	0.245506473455505\\
60.7689162011208	0.245506625004746\\
60.7754403949143	0.245506776506689\\
60.7819645831522	0.245506927961355\\
60.7884887658362	0.245507079368762\\
60.7950129429682	0.245507230728932\\
60.8015371145498	0.245507382041882\\
60.8080612805828	0.245507533307634\\
60.814585441069	0.245507684526206\\
60.8211095960099	0.245507835697618\\
60.8276337454075	0.245507986821891\\
60.8341578892634	0.245508137899043\\
60.8406820275794	0.245508288929094\\
60.8472061603571	0.245508439912064\\
60.8537302875984	0.245508590847973\\
60.860254409305	0.245508741736839\\
60.8667785254786	0.245508892578683\\
60.8733026361209	0.245509043373525\\
60.8798267412337	0.245509194121383\\
60.8863508408186	0.245509344822277\\
60.8928749348775	0.245509495476227\\
60.8993990234121	0.245509646083253\\
60.905923106424	0.245509796643373\\
60.9124471839151	0.245509947156608\\
60.918971255887	0.245510097622977\\
60.9254953223414	0.245510248042499\\
60.9320193832802	0.245510398415194\\
60.9385434387049	0.245510548741081\\
60.9450674886175	0.245510699020181\\
60.9515915330194	0.245510849252511\\
60.9581155719126	0.245510999438092\\
60.9646396052987	0.245511149576943\\
60.9711636331794	0.245511299669084\\
60.9776876555564	0.245511449714534\\
60.9842116724315	0.245511599713312\\
60.9907356838065	0.245511749665438\\
60.9972596896829	0.245511899570931\\
61.0037836900625	0.245512049429811\\
61.0103076849471	0.245512199242096\\
61.0168316743384	0.245512349007807\\
61.023355658238	0.245512498726962\\
61.0298796366478	0.245512648399581\\
61.0364036095693	0.245512798025683\\
61.0429275770043	0.245512947605288\\
61.0494515389546	0.245513097138414\\
61.0559754954218	0.245513246625082\\
61.0624994464077	0.24551339606531\\
61.069023391914	0.245513545459117\\
61.0755473319423	0.245513694806524\\
61.0820712664944	0.245513844107549\\
61.088595195572	0.245513993362211\\
61.0951191191768	0.24551414257053\\
61.1016430373105	0.245514291732524\\
61.1081669499748	0.245514440848214\\
61.1146908571714	0.245514589917618\\
61.1212147589021	0.245514738940755\\
61.1277386551685	0.245514887917645\\
61.1342625459723	0.245515036848307\\
61.1407864313153	0.245515185732759\\
61.1473103111991	0.245515334571022\\
61.1538341856254	0.245515483363114\\
61.160358054596	0.245515632109054\\
61.1668819181125	0.245515780808862\\
61.1734057761766	0.245515929462557\\
61.1799296287901	0.245516078070157\\
61.1864534759547	0.245516226631682\\
61.1929773176719	0.245516375147151\\
61.1995011539436	0.245516523616583\\
61.2060249847714	0.245516672039996\\
61.212548810157	0.245516820417411\\
61.2190726301022	0.245516968748846\\
61.2255964446085	0.24551711703432\\
61.2321202536777	0.245517265273852\\
61.2386440573116	0.245517413467462\\
61.2451678555117	0.245517561615167\\
61.2516916482798	0.245517709716988\\
61.2582154356176	0.245517857772943\\
61.2647392175267	0.24551800578305\\
61.2712629940088	0.24551815374733\\
61.2777867650657	0.245518301665801\\
61.284310530699	0.245518449538482\\
61.2908342909104	0.245518597365392\\
61.2973580457016	0.24551874514655\\
61.3038817950743	0.245518892881974\\
61.3104055390301	0.245519040571684\\
61.3169292775708	0.245519188215698\\
61.323453010698	0.245519335814036\\
61.3299767384134	0.245519483366716\\
61.3365004607187	0.245519630873758\\
61.3430241776155	0.245519778335179\\
61.3495478891056	0.245519925750999\\
61.3560715951907	0.245520073121237\\
61.3625952958723	0.245520220445911\\
61.3691189911523	0.245520367725041\\
61.3756426810322	0.245520514958645\\
61.3821663655138	0.245520662146741\\
61.3886900445986	0.24552080928935\\
61.3952137182885	0.245520956386489\\
61.401737386585	0.245521103438177\\
61.4082610494899	0.245521250444433\\
61.4147847070048	0.245521397405276\\
61.4213083591314	0.245521544320724\\
61.4278320058713	0.245521691190797\\
61.4343556472263	0.245521838015512\\
61.4408792831979	0.245521984794889\\
61.447402913788	0.245522131528947\\
61.4539265389981	0.245522278217703\\
61.4604501588299	0.245522424861177\\
61.466973773285	0.245522571459388\\
61.4734973823653	0.245522718012354\\
61.4800209860722	0.245522864520093\\
61.4865445844075	0.245523010982625\\
61.4930681773728	0.245523157399967\\
61.4995917649699	0.24552330377214\\
61.5061153472003	0.24552345009916\\
61.5126389240657	0.245523596381047\\
61.5191624955679	0.24552374261782\\
61.5256860617084	0.245523888809497\\
61.5322096224889	0.245524034956096\\
61.538733177911	0.245524181057636\\
61.5452567279766	0.245524327114136\\
61.5517802726871	0.245524473125614\\
61.5583038120443	0.245524619092089\\
61.5648273460497	0.245524765013579\\
61.5713508747052	0.245524910890103\\
61.5778743980123	0.24552505672168\\
61.5843979159726	0.245525202508327\\
61.5909214285879	0.245525348250063\\
61.5974449358598	0.245525493946907\\
61.6039684377899	0.245525639598878\\
61.6104919343799	0.245525785205993\\
61.6170154256314	0.245525930768271\\
61.6235389115462	0.245526076285731\\
61.6300623921258	0.24552622175839\\
61.6365858673718	0.245526367186269\\
61.6431093372861	0.245526512569384\\
61.6496328018701	0.245526657907754\\
61.6561562611256	0.245526803201398\\
61.6626797150541	0.245526948450334\\
61.6692031636574	0.24552709365458\\
61.6757266069371	0.245527238814156\\
61.6822500448948	0.245527383929078\\
61.6887734775322	0.245527528999366\\
61.6952969048508	0.245527674025037\\
61.7018203268525	0.245527819006111\\
61.7083437435387	0.245527963942605\\
61.7148671549112	0.245528108834538\\
61.7213905609716	0.245528253681928\\
61.7279139617215	0.245528398484793\\
61.7344373571625	0.245528543243152\\
61.7409607472964	0.245528687957023\\
61.7474841321247	0.245528832626424\\
61.7540075116491	0.245528977251373\\
61.7605308858711	0.245529121831889\\
61.7670542547926	0.24552926636799\\
61.773577618415	0.245529410859694\\
61.7801009767401	0.245529555307019\\
61.7866243297694	0.245529699709984\\
61.7931476775046	0.245529844068607\\
61.7996710199473	0.245529988382905\\
61.8061943570992	0.245530132652898\\
61.8127176889618	0.245530276878602\\
61.8192410155369	0.245530421060037\\
61.825764336826	0.245530565197221\\
61.8322876528308	0.245530709290172\\
61.8388109635529	0.245530853338907\\
61.845334268994	0.245530997343445\\
61.8518575691556	0.245531141303804\\
61.8583808640394	0.245531285220002\\
61.864904153647	0.245531429092058\\
61.87142743798	0.245531572919989\\
61.8779507170401	0.245531716703813\\
61.8844739908289	0.245531860443549\\
61.8909972593481	0.245532004139214\\
61.8975205225991	0.245532147790827\\
61.9040437805837	0.245532291398406\\
61.9105670333035	0.245532434961968\\
61.9170902807601	0.245532578481532\\
61.9236135229551	0.245532721957115\\
61.9301367598902	0.245532865388736\\
61.9366599915669	0.245533008776413\\
61.9431832179869	0.245533152120164\\
61.9497064391518	0.245533295420006\\
61.9562296550631	0.245533438675957\\
61.9627528657227	0.245533581888037\\
61.9692760711319	0.245533725056261\\
61.9757992712926	0.245533868180649\\
61.9823224662062	0.245534011261219\\
61.9888456558744	0.245534154297987\\
61.9953688402987	0.245534297290973\\
62.0018920194809	0.245534440240194\\
62.0084151934225	0.245534583145668\\
62.0149383621252	0.245534726007412\\
62.0214615255905	0.245534868825446\\
62.02798468382	0.245535011599786\\
62.0345078368155	0.24553515433045\\
62.0410309845783	0.245535297017456\\
62.0475541271103	0.245535439660823\\
62.0540772644129	0.245535582260568\\
62.0606003964879	0.245535724816708\\
62.0671235233367	0.245535867329262\\
62.073646644961	0.245536009798247\\
62.0801697613625	0.245536152223682\\
62.0866928725426	0.245536294605583\\
62.0932159785031	0.245536436943969\\
62.0997390792455	0.245536579238857\\
62.1062621747714	0.245536721490266\\
62.1127852650824	0.245536863698213\\
62.1193083501801	0.245537005862715\\
62.1258314300662	0.245537147983791\\
62.1323545047421	0.245537290061458\\
62.1388775742096	0.245537432095733\\
62.1454006384702	0.245537574086635\\
62.1519236975255	0.245537716034182\\
62.1584467513771	0.24553785793839\\
62.1649698000266	0.245537999799278\\
62.1714928434757	0.245538141616863\\
62.1780158817258	0.245538283391163\\
62.1845389147785	0.245538425122196\\
62.1910619426356	0.245538566809979\\
62.1975849652985	0.245538708454529\\
62.2041079827689	0.245538850055865\\
62.2106309950484	0.245538991614004\\
62.2171540021385	0.245539133128964\\
62.2236770040408	0.245539274600762\\
62.230200000757	0.245539416029415\\
62.2367229922885	0.245539557414942\\
62.2432459786371	0.24553969875736\\
62.2497689598043	0.245539840056687\\
62.2562919357916	0.245539981312939\\
62.2628149066007	0.245540122526135\\
62.2693378722331	0.245540263696292\\
62.2758608326905	0.245540404823428\\
62.2823837879744	0.24554054590756\\
62.2889067380864	0.245540686948705\\
62.295429683028	0.245540827946882\\
62.3019526228009	0.245540968902107\\
62.3084755574067	0.245541109814398\\
62.3149984868469	0.245541250683773\\
62.3215214111231	0.245541391510249\\
62.3280443302369	0.245541532293844\\
62.3345672441898	0.245541673034574\\
62.3410901529835	0.245541813732457\\
62.3476130566195	0.245541954387512\\
62.3541359550994	0.245542094999754\\
62.3606588484247	0.245542235569202\\
62.3671817365972	0.245542376095873\\
62.3737046196182	0.245542516579784\\
62.3802274974894	0.245542657020953\\
62.3867503702124	0.245542797419397\\
62.3932732377888	0.245542937775134\\
62.39979610022	0.24554307808818\\
62.4063189575078	0.245543218358554\\
62.4128418096536	0.245543358586271\\
62.419364656659	0.245543498771351\\
62.4258874985257	0.24554363891381\\
62.4324103352551	0.245543779013665\\
62.4389331668488	0.245543919070934\\
62.4454559933085	0.245544059085634\\
62.4519788146356	0.245544199057782\\
62.4585016308318	0.245544338987396\\
62.4650244418985	0.245544478874493\\
62.4715472478375	0.245544618719089\\
62.4780700486501	0.245544758521203\\
62.4845928443381	0.245544898280852\\
62.4911156349029	0.245545037998052\\
62.4976384203462	0.245545177672821\\
62.5041612006694	0.245545317305177\\
62.5106839758742	0.245545456895135\\
62.5172067459621	0.245545596442715\\
62.5237295109347	0.245545735947932\\
62.5302522707935	0.245545875410804\\
62.5367750255401	0.245546014831349\\
62.543297775176	0.245546154209582\\
62.5498205197029	0.245546293545522\\
62.5563432591222	0.245546432839186\\
62.5628659934355	0.24554657209059\\
62.5693887226444	0.245546711299752\\
62.5759114467504	0.245546850466689\\
62.5824341657551	0.245546989591418\\
62.5889568796601	0.245547128673956\\
62.5954795884668	0.245547267714321\\
62.6020022921769	0.245547406712529\\
62.6085249907919	0.245547545668597\\
62.6150476843133	0.245547684582542\\
62.6215703727427	0.245547823454382\\
62.6280930560817	0.245547962284134\\
62.6346157343318	0.245548101071814\\
62.6411384074945	0.24554823981744\\
62.6476610755714	0.245548378521028\\
62.654183738564	0.245548517182596\\
62.660706396474	0.245548655802161\\
62.6672290493027	0.245548794379739\\
62.6737516970519	0.245548932915348\\
62.680274339723	0.245549071409004\\
62.6867969773175	0.245549209860725\\
62.6933196098371	0.245549348270527\\
62.6998422372832	0.245549486638428\\
62.7063648596574	0.245549624964445\\
62.7128874769613	0.245549763248593\\
62.7194100891964	0.245549901490891\\
62.7259326963641	0.245550039691355\\
62.7324552984662	0.245550177850002\\
62.738977895504	0.24555031596685\\
62.7455004874792	0.245550454041914\\
62.7520230743933	0.245550592075211\\
62.7585456562477	0.24555073006676\\
62.7650682330442	0.245550868016575\\
62.7715908047841	0.245551005924676\\
62.7781133714691	0.245551143791077\\
62.7846359331006	0.245551281615796\\
62.7911584896802	0.245551419398851\\
62.7976810412095	0.245551557140257\\
62.8042035876899	0.245551694840031\\
62.810726129123	0.245551832498191\\
62.8172486655103	0.245551970114753\\
62.8237711968534	0.245552107689733\\
62.8302937231538	0.24555224522315\\
62.836816244413	0.245552382715018\\
62.8433387606325	0.245552520165356\\
62.849861271814	0.24555265757418\\
62.8563837779588	0.245552794941507\\
62.8629062790686	0.245552932267353\\
62.8694287751448	0.245553069551735\\
62.875951266189	0.24555320679467\\
62.8824737522028	0.245553343996175\\
62.8889962331876	0.245553481156266\\
62.8955187091449	0.24555361827496\\
62.9020411800763	0.245553755352273\\
62.9085636459834	0.245553892388223\\
62.9150861068675	0.245554029382826\\
62.9216085627304	0.245554166336099\\
62.9281310135734	0.245554303248058\\
62.9346534593981	0.24555444011872\\
62.941175900206	0.245554576948102\\
62.9476983359986	0.24555471373622\\
62.9542207667775	0.24555485048309\\
62.9607431925442	0.24555498718873\\
62.9672656133002	0.245555123853157\\
62.973788029047	0.245555260476386\\
62.9803104397861	0.245555397058434\\
62.986832845519	0.245555533599318\\
62.9933552462473	0.245555670099054\\
62.9998776419725	0.24555580655766\\
63.0064000326961	0.24555594297515\\
63.0129224184196	0.245556079351543\\
63.0194447991444	0.245556215686855\\
63.0259671748723	0.245556351981102\\
63.0324895456045	0.2455564882343\\
63.0390119113427	0.245556624446466\\
63.0455342720883	0.245556760617617\\
63.0520566278429	0.24555689674777\\
63.058578978608	0.24555703283694\\
63.0651013243851	0.245557168885144\\
63.0716236651757	0.245557304892398\\
63.0781460009812	0.24555744085872\\
63.0846683318033	0.245557576784125\\
63.0911906576434	0.24555771266863\\
63.097712978503	0.245557848512251\\
63.1042352943836	0.245557984315005\\
63.1107576052867	0.245558120076909\\
63.1172799112139	0.245558255797978\\
63.1238022121666	0.245558391478229\\
63.1303245081464	0.245558527117678\\
63.1368467991547	0.245558662716343\\
63.143369085193	0.245558798274238\\
63.1498913662629	0.245558933791381\\
63.1564136423658	0.245559069267788\\
63.1629359135033	0.245559204703476\\
63.1694581796768	0.245559340098459\\
63.1759804408878	0.245559475452756\\
63.1825026971379	0.245559610766382\\
63.1890249484285	0.245559746039354\\
63.1955471947611	0.245559881271688\\
63.2020694361372	0.2455600164634\\
63.2085916725584	0.245560151614506\\
63.2151139040261	0.245560286725023\\
63.2216361305418	0.245560421794968\\
63.2281583521069	0.245560556824355\\
63.2346805687231	0.245560691813202\\
63.2412027803918	0.245560826761526\\
63.2477249871144	0.245560961669341\\
63.2542471888926	0.245561096536665\\
63.2607693857276	0.245561231363513\\
63.2672915776212	0.245561366149902\\
63.2738137645746	0.245561500895848\\
63.2803359465895	0.245561635601367\\
63.2868581236673	0.245561770266476\\
63.2933802958095	0.24556190489119\\
63.2999024630176	0.245562039475526\\
63.3064246252931	0.245562174019501\\
63.3129467826374	0.245562308523129\\
63.3194689350521	0.245562442986427\\
63.3259910825386	0.245562577409413\\
63.3325132250984	0.2455627117921\\
63.339035362733	0.245562846134507\\
63.3455574954439	0.245562980436648\\
63.3520796232326	0.24556311469854\\
63.3586017461005	0.245563248920199\\
63.3651238640491	0.245563383101642\\
63.3716459770799	0.245563517242883\\
63.3781680851945	0.245563651343941\\
63.3846901883941	0.245563785404829\\
63.3912122866804	0.245563919425565\\
63.3977343800549	0.245564053406165\\
63.4042564685189	0.245564187346644\\
63.410778552074	0.245564321247019\\
63.4173006307217	0.245564455107305\\
63.4238227044634	0.24556458892752\\
63.4303447733005	0.245564722707678\\
63.4368668372347	0.245564856447795\\
63.4433888962673	0.245564990147889\\
63.4499109503998	0.245565123807974\\
63.4564329996337	0.245565257428067\\
63.4629550439705	0.245565391008184\\
63.4694770834116	0.245565524548341\\
63.4759991179585	0.245565658048553\\
63.4825211476127	0.245565791508837\\
63.4890431723757	0.245565924929208\\
63.4955651922488	0.245566058309683\\
63.5020872072337	0.245566191650277\\
63.5086092173317	0.245566324951007\\
63.5151312225443	0.245566458211888\\
63.521653222873	0.245566591432936\\
63.5281752183192	0.245566724614167\\
63.5346972088845	0.245566857755597\\
63.5412191945703	0.245566990857242\\
63.547741175378	0.245567123919118\\
63.5542631513091	0.245567256941241\\
63.560785122365	0.245567389923626\\
63.5673070885474	0.245567522866289\\
63.5738290498575	0.245567655769246\\
63.5803510062968	0.245567788632514\\
63.5868729578669	0.245567921456107\\
63.5933949045692	0.245568054240042\\
63.5999168464051	0.245568186984334\\
63.6064387833761	0.245568319689\\
63.6129607154837	0.245568452354055\\
63.6194826427293	0.245568584979515\\
63.6260045651143	0.245568717565395\\
63.6325264826403	0.245568850111712\\
63.6390483953086	0.245568982618481\\
63.6455703031208	0.245569115085718\\
63.6520922060783	0.245569247513438\\
63.6586141041826	0.245569379901658\\
63.665135997435	0.245569512250393\\
63.6716578858371	0.245569644559659\\
63.6781797693903	0.245569776829472\\
63.6847016480961	0.245569909059847\\
63.6912235219559	0.2455700412508\\
63.6977453909711	0.245570173402347\\
63.7042672551433	0.245570305514504\\
63.7107891144738	0.245570437587285\\
63.7173109689641	0.245570569620707\\
63.7238328186157	0.245570701614786\\
63.73035466343	0.245570833569536\\
63.7368765034085	0.245570965484975\\
63.7433983385525	0.245571097361117\\
63.7499201688636	0.245571229197978\\
63.7564419943432	0.245571360995573\\
63.7629638149928	0.245571492753919\\
63.7694856308137	0.24557162447303\\
63.7760074418074	0.245571756152923\\
63.7825292479754	0.245571887793614\\
63.7890510493192	0.245572019395116\\
63.7955728458401	0.245572150957447\\
63.8020946375395	0.245572282480622\\
63.8086164244191	0.245572413964657\\
63.8151382064801	0.245572545409566\\
63.821659983724	0.245572676815365\\
63.8281817561522	0.245572808182071\\
63.8347035237663	0.245572939509698\\
63.8412252865676	0.245573070798262\\
63.8477470445576	0.245573202047779\\
63.8542687977377	0.245573333258264\\
63.8607905461093	0.245573464429733\\
63.8673122896739	0.2455735955622\\
63.8738340284329	0.245573726655682\\
63.8803557623878	0.245573857710195\\
63.8868774915399	0.245573988725752\\
63.8933992158908	0.245574119702371\\
63.8999209354418	0.245574250640066\\
63.9064426501944	0.245574381538853\\
63.9129643601501	0.245574512398748\\
63.9194860653101	0.245574643219765\\
63.9260077656761	0.24557477400192\\
63.9325294612493	0.245574904745229\\
63.9390511520313	0.245575035449707\\
63.9455728380235	0.24557516611537\\
63.9520945192272	0.245575296742232\\
63.958616195644	0.245575427330309\\
63.9651378672752	0.245575557879617\\
63.9716595341223	0.245575688390171\\
63.9781811961867	0.245575818861986\\
63.9847028534698	0.245575949295078\\
63.9912245059731	0.245576079689462\\
63.997746153698	0.245576210045153\\
64.0042677966458	0.245576340362167\\
64.0107894348181	0.245576470640519\\
64.0173110682163	0.245576600880224\\
64.0238326968417	0.245576731081298\\
64.0303543206957	0.245576861243756\\
64.0368759397799	0.245576991367613\\
64.0433975540957	0.245577121452885\\
64.0499191636443	0.245577251499586\\
64.0564407684274	0.245577381507732\\
64.0629623684463	0.245577511477339\\
64.0694839637023	0.245577641408422\\
64.076005554197	0.245577771300995\\
64.0825271399317	0.245577901155074\\
64.0890487209079	0.245578030970675\\
64.0955702971269	0.245578160747812\\
64.1020918685903	0.245578290486501\\
64.1086134352993	0.245578420186757\\
64.1151349972555	0.245578549848595\\
64.1216565544602	0.24557867947203\\
64.1281781069148	0.245578809057079\\
64.1346996546208	0.245578938603754\\
64.1412211975796	0.245579068112073\\
64.1477427357926	0.24557919758205\\
64.1542642692611	0.2455793270137\\
64.1607857979867	0.245579456407039\\
64.1673073219707	0.245579585762081\\
64.1738288412144	0.245579715078841\\
64.1803503557195	0.245579844357336\\
64.1868718654871	0.245579973597579\\
64.1933933705188	0.245580102799587\\
64.199914870816	0.245580231963373\\
64.20643636638	0.245580361088954\\
64.2129578572123	0.245580490176345\\
64.2194793433143	0.245580619225559\\
64.2260008246873	0.245580748236614\\
64.2325223013328	0.245580877209522\\
64.2390437732522	0.245581006144301\\
64.2455652404469	0.245581135040964\\
64.2520867029183	0.245581263899527\\
64.2586081606677	0.245581392720004\\
64.2651296136967	0.245581521502412\\
64.2716510620065	0.245581650246764\\
64.2781725055986	0.245581778953076\\
64.2846939444745	0.245581907621363\\
64.2912153786354	0.24558203625164\\
64.2977368080828	0.245582164843922\\
64.3042582328181	0.245582293398223\\
64.3107796528427	0.24558242191456\\
64.317301068158	0.245582550392946\\
64.3238224787653	0.245582678833397\\
64.3303438846662	0.245582807235927\\
64.3368652858619	0.245582935600553\\
64.3433866823538	0.245583063927288\\
64.3499080741434	0.245583192216147\\
64.3564294612321	0.245583320467146\\
64.3629508436212	0.2455834486803\\
64.3694722213122	0.245583576855622\\
64.3759935943064	0.245583704993129\\
64.3825149626052	0.245583833092835\\
64.38903632621	0.245583961154755\\
64.3955576851223	0.245584089178904\\
64.4020790393433	0.245584217165297\\
64.4086003888745	0.245584345113949\\
64.4151217337173	0.245584473024873\\
64.421643073873	0.245584600898087\\
64.4281644093431	0.245584728733603\\
64.4346857401289	0.245584856531438\\
64.4412070662318	0.245584984291605\\
64.4477283876533	0.24558511201412\\
64.4542497043946	0.245585239698998\\
64.4607710164572	0.245585367346253\\
64.4672923238425	0.2455854949559\\
64.4738136265518	0.245585622527954\\
64.4803349245865	0.24558575006243\\
64.4868562179481	0.245585877559342\\
64.4933775066378	0.245586005018706\\
64.4998987906571	0.245586132440536\\
64.5064200700074	0.245586259824846\\
64.51294134469	0.245586387171652\\
64.5194626147063	0.245586514480969\\
64.5259838800577	0.24558664175281\\
64.5325051407456	0.245586768987191\\
64.5390263967713	0.245586896184127\\
64.5455476481363	0.245587023343632\\
64.5520688948419	0.245587150465721\\
64.5585901368894	0.245587277550408\\
64.5651113742803	0.245587404597709\\
64.571632607016	0.245587531607638\\
64.5781538350977	0.245587658580209\\
64.584675058527	0.245587785515438\\
64.591196277305	0.245587912413339\\
64.5977174914334	0.245588039273927\\
64.6042387009133	0.245588166097216\\
64.6107599057462	0.245588292883221\\
64.6172811059335	0.245588419631956\\
64.6238023014764	0.245588546343437\\
64.6303234923765	0.245588673017678\\
64.636844678635	0.245588799654693\\
64.6433658602534	0.245588926254498\\
64.6498870372329	0.245589052817106\\
64.656408209575	0.245589179342532\\
64.6629293772811	0.245589305830791\\
64.6694505403524	0.245589432281898\\
64.6759716987905	0.245589558695867\\
64.6824928525965	0.245589685072712\\
64.689014001772	0.245589811412449\\
64.6955351463182	0.245589937715091\\
64.7020562862365	0.245590063980654\\
64.7085774215284	0.245590190209151\\
64.7150985521951	0.245590316400598\\
64.721619678238	0.245590442555008\\
64.7281407996585	0.245590568672397\\
64.734661916458	0.245590694752779\\
64.7411830286378	0.245590820796168\\
64.7477041361992	0.245590946802579\\
64.7542252391437	0.245591072772026\\
64.7607463374725	0.245591198704524\\
64.7672674311872	0.245591324600088\\
64.7737885202889	0.245591450458731\\
64.780309604779	0.245591576280469\\
64.786830684659	0.245591702065315\\
64.7933517599302	0.245591827813284\\
64.7998728305939	0.245591953524391\\
64.8063938966516	0.24559207919865\\
64.8129149581044	0.245592204836075\\
64.8194360149539	0.245592330436681\\
64.8259570672013	0.245592456000483\\
64.832478114848	0.245592581527494\\
64.8389991578955	0.245592707017729\\
64.8455201963449	0.245592832471202\\
64.8520412301977	0.245592957887928\\
64.8585622594552	0.245593083267922\\
64.8650832841188	0.245593208611197\\
64.8716043041898	0.245593333917768\\
64.8781253196696	0.245593459187649\\
64.8846463305595	0.245593584420855\\
64.8911673368609	0.2455937096174\\
64.8976883385752	0.245593834777298\\
64.9042093357036	0.245593959900564\\
64.9107303282475	0.245594084987211\\
64.9172513162083	0.245594210037255\\
64.9237722995872	0.245594335050709\\
64.9302932783858	0.245594460027589\\
64.9368142526053	0.245594584967907\\
64.943335222247	0.245594709871679\\
64.9498561873123	0.245594834738918\\
64.9563771478025	0.24559495956964\\
64.9628981037191	0.245595084363857\\
64.9694190550632	0.245595209121585\\
64.9759400018363	0.245595333842838\\
64.9824609440398	0.24559545852763\\
64.9889818816748	0.245595583175975\\
64.9955028147429	0.245595707787887\\
65.0020237432453	0.245595832363381\\
65.0085446671834	0.245595956902471\\
65.0150655865585	0.245596081405171\\
65.0215865013719	0.245596205871495\\
65.028107411625	0.245596330301457\\
65.0346283173191	0.245596454695073\\
65.0411492184556	0.245596579052355\\
65.0476701150358	0.245596703373318\\
65.0541910070611	0.245596827657976\\
65.0607118945327	0.245596951906343\\
65.067232777452	0.245597076118434\\
65.0737536558204	0.245597200294263\\
65.0802745296391	0.245597324433843\\
65.0867953989095	0.245597448537189\\
65.093316263633	0.245597572604315\\
65.0998371238108	0.245597696635236\\
65.1063579794444	0.245597820629964\\
65.112878830535	0.245597944588515\\
65.1193996770839	0.245598068510902\\
65.1259205190926	0.24559819239714\\
65.1324413565622	0.245598316247242\\
65.1389621894943	0.245598440061223\\
65.14548301789	0.245598563839097\\
65.1520038417507	0.245598687580877\\
65.1585246610777	0.245598811286578\\
65.1650454758725	0.245598934956214\\
65.1715662861362	0.2455990585898\\
65.1780870918702	0.245599182187348\\
65.1846078930759	0.245599305748873\\
65.1911286897546	0.245599429274389\\
65.1976494819075	0.24559955276391\\
65.2041702695361	0.24559967621745\\
65.2106910526417	0.245599799635023\\
65.2172118312254	0.245599923016643\\
65.2237326052888	0.245600046362324\\
65.2302533748331	0.245600169672081\\
65.2367741398596	0.245600292945926\\
65.2432949003697	0.245600416183874\\
65.2498156563646	0.245600539385939\\
65.2563364078458	0.245600662552135\\
65.2628571548145	0.245600785682475\\
65.2693778972719	0.245600908776974\\
65.2758986352196	0.245601031835646\\
65.2824193686587	0.245601154858505\\
65.2889400975906	0.245601277845564\\
65.2954608220166	0.245601400796837\\
65.301981541938	0.245601523712339\\
65.3085022573561	0.245601646592083\\
65.3150229682723	0.245601769436083\\
65.3215436746879	0.245601892244353\\
65.3280643766041	0.245602015016907\\
65.3345850740224	0.245602137753758\\
65.3411057669439	0.245602260454922\\
65.3476264553701	0.24560238312041\\
65.3541471393022	0.245602505750238\\
65.3606678187415	0.245602628344419\\
65.3671884936894	0.245602750902968\\
65.3737091641471	0.245602873425896\\
65.3802298301161	0.24560299591322\\
65.3867504915975	0.245603118364952\\
65.3932711485927	0.245603240781106\\
65.399791801103	0.245603363161697\\
65.4063124491297	0.245603485506737\\
65.4128330926741	0.245603607816241\\
65.4193537317376	0.245603730090222\\
65.4258743663214	0.245603852328695\\
65.4323949964268	0.245603974531672\\
65.4389156220552	0.245604096699168\\
65.4454362432078	0.245604218831197\\
65.451956859886	0.245604340927772\\
65.458477472091	0.245604462988907\\
65.4649980798242	0.245604585014616\\
65.4715186830869	0.245604707004912\\
65.4780392818803	0.24560482895981\\
65.4845598762058	0.245604950879322\\
65.4910804660647	0.245605072763463\\
65.4976010514582	0.245605194612246\\
65.5041216323877	0.245605316425686\\
65.5106422088545	0.245605438203795\\
65.5171627808599	0.245605559946587\\
65.5236833484051	0.245605681654076\\
65.5302039114915	0.245605803326277\\
65.5367244701203	0.245605924963201\\
65.543245024293	0.245606046564864\\
65.5497655740106	0.245606168131278\\
65.5562861192747	0.245606289662458\\
65.5628066600864	0.245606411158417\\
65.569327196447	0.245606532619168\\
65.5758477283579	0.245606654044725\\
65.5823682558203	0.245606775435103\\
65.5888887788356	0.245606896790314\\
65.595409297405	0.245607018110372\\
65.6019298115298	0.24560713939529\\
65.6084503212113	0.245607260645083\\
65.6149708264508	0.245607381859764\\
65.6214913272496	0.245607503039346\\
65.628011823609	0.245607624183843\\
65.6345323155303	0.245607745293269\\
65.6410528030147	0.245607866367637\\
65.6475732860636	0.24560798740696\\
65.6540937646782	0.245608108411253\\
65.6606142388599	0.245608229380529\\
65.6671347086099	0.2456083503148\\
65.6736551739295	0.245608471214082\\
65.68017563482	0.245608592078387\\
65.6866960912827	0.245608712907729\\
65.6932165433188	0.245608833702121\\
65.6997369909297	0.245608954461577\\
65.7062574341166	0.24560907518611\\
65.7127778728809	0.245609195875734\\
65.7192983072238	0.245609316530462\\
65.7258187371465	0.245609437150309\\
65.7323391626504	0.245609557735286\\
65.7388595837368	0.245609678285408\\
65.7453800004069	0.245609798800688\\
65.7519004126621	0.24560991928114\\
65.7584208205035	0.245610039726777\\
65.7649412239325	0.245610160137613\\
65.7714616229504	0.24561028051366\\
65.7779820175584	0.245610400854933\\
65.7845024077578	0.245610521161444\\
65.79102279355	0.245610641433208\\
65.7975431749361	0.245610761670237\\
65.8040635519174	0.245610881872545\\
65.8105839244953	0.245611002040145\\
65.817104292671	0.245611122173051\\
65.8236246564458	0.245611242271276\\
65.8301450158209	0.245611362334834\\
65.8366653707977	0.245611482363738\\
65.8431857213773	0.245611602358\\
65.8497060675612	0.245611722317635\\
65.8562264093505	0.245611842242656\\
65.8627467467465	0.245611962133077\\
65.8692670797505	0.24561208198891\\
65.8757874083638	0.245612201810168\\
65.8823077325877	0.245612321596866\\
65.8888280524233	0.245612441349016\\
65.8953483678721	0.245612561066633\\
65.9018686789352	0.245612680749728\\
65.9083889856139	0.245612800398316\\
65.9149092879095	0.245612920012409\\
65.9214295858233	0.245613039592021\\
65.9279498793565	0.245613159137166\\
65.9344701685104	0.245613278647856\\
65.9409904532863	0.245613398124104\\
65.9475107336854	0.245613517565925\\
65.954031009709	0.245613636973331\\
65.9605512813584	0.245613756346336\\
65.9670715486348	0.245613875684952\\
65.9735918115395	0.245613994989193\\
65.9801120700738	0.245614114259073\\
65.9866323242388	0.245614233494604\\
65.993152574036	0.245614352695799\\
65.9996728194665	0.245614471862672\\
66.0061930605316	0.245614590995236\\
66.0127132972326	0.245614710093505\\
66.0192335295707	0.245614829157491\\
66.0257537575472	0.245614948187207\\
66.0322739811634	0.245615067182667\\
66.0387942004205	0.245615186143884\\
66.0453144153197	0.245615305070871\\
66.0518346258624	0.245615423963641\\
66.0583548320498	0.245615542822208\\
66.0648750338831	0.245615661646584\\
66.0713952313637	0.245615780436783\\
66.0779154244926	0.245615899192817\\
66.0844356132714	0.245616017914701\\
66.090955797701	0.245616136602446\\
66.097475977783	0.245616255256066\\
66.1039961535184	0.245616373875575\\
66.1105163249085	0.245616492460985\\
66.1170364919546	0.245616611012309\\
66.123556654658	0.245616729529561\\
66.1300768130199	0.245616848012753\\
66.1365969670415	0.245616966461899\\
66.1431171167242	0.245617084877011\\
66.1496372620691	0.245617203258104\\
66.1561574030775	0.245617321605189\\
66.1626775397506	0.245617439918279\\
66.1691976720898	0.245617558197389\\
66.1757178000962	0.245617676442531\\
66.1822379237712	0.245617794653717\\
66.1887580431158	0.245617912830962\\
66.1952781581315	0.245618030974277\\
66.2017982688195	0.245618149083676\\
66.2083183751809	0.245618267159173\\
66.2148384772171	0.245618385200779\\
66.2213585749293	0.245618503208508\\
66.2278786683187	0.245618621182373\\
66.2343987573866	0.245618739122388\\
66.2409188421342	0.245618857028563\\
66.2474389225629	0.245618974900914\\
66.2539589986737	0.245619092739453\\
66.2604790704679	0.245619210544192\\
66.2669991379469	0.245619328315145\\
66.2735192011118	0.245619446052325\\
66.2800392599639	0.245619563755744\\
66.2865593145045	0.245619681425416\\
66.2930793647347	0.245619799061353\\
66.2995994106558	0.245619916663568\\
66.3061194522691	0.245620034232075\\
66.3126394895757	0.245620151766886\\
66.319159522577	0.245620269268013\\
66.3256795512742	0.245620386735471\\
66.3321995756684	0.245620504169272\\
66.338719595761	0.245620621569428\\
66.3452396115532	0.245620738935953\\
66.3517596230462	0.245620856268859\\
66.3582796302413	0.245620973568159\\
66.3647996331396	0.245621090833867\\
66.3713196317425	0.245621208065995\\
66.3778396260511	0.245621325264555\\
66.3843596160667	0.245621442429561\\
66.3908796017905	0.245621559561025\\
66.3973995832238	0.245621676658961\\
66.4039195603678	0.245621793723381\\
66.4104395332238	0.245621910754298\\
66.4169595017928	0.245622027751725\\
66.4234794660763	0.245622144715674\\
66.4299994260754	0.245622261646158\\
66.4365193817914	0.24562237854319\\
66.4430393332254	0.245622495406784\\
66.4495592803787	0.245622612236951\\
66.4560792232526	0.245622729033704\\
66.4625991618482	0.245622845797057\\
66.4691190961669	0.245622962527021\\
66.4756390262097	0.245623079223611\\
66.482158951978	0.245623195886837\\
66.488678873473	0.245623312516714\\
66.4951987906959	0.245623429113254\\
66.501718703648	0.245623545676469\\
66.5082386123303	0.245623662206373\\
66.5147585167443	0.245623778702977\\
66.5212784168911	0.245623895166296\\
66.5277983127719	0.245624011596341\\
66.5343182043879	0.245624127993125\\
66.5408380917405	0.24562424435666\\
66.5473579748307	0.245624360686961\\
66.5538778536598	0.245624476984038\\
66.5603977282291	0.245624593247906\\
66.5669175985398	0.245624709478575\\
66.573437464593	0.245624825676061\\
66.5799573263901	0.245624941840373\\
66.5864771839321	0.245625057971527\\
66.5929970372204	0.245625174069533\\
66.5995168862562	0.245625290134405\\
66.6060367310406	0.245625406166156\\
66.612556571575	0.245625522164798\\
66.6190764078605	0.245625638130343\\
66.6255962398983	0.245625754062804\\
66.6321160676896	0.245625869962194\\
66.6386358912357	0.245625985828526\\
66.6451557105378	0.245626101661812\\
66.6516755255971	0.245626217462064\\
66.6581953364148	0.245626333229295\\
66.6647151429921	0.245626448963518\\
66.6712349453303	0.245626564664746\\
66.6777547434305	0.24562668033299\\
66.684274537294	0.245626795968264\\
66.6907943269219	0.24562691157058\\
66.6973141123156	0.24562702713995\\
66.7038338934761	0.245627142676387\\
66.7103536704048	0.245627258179904\\
66.7168734431028	0.245627373650513\\
66.7233932115713	0.245627489088227\\
66.7299129758115	0.245627604493057\\
66.7364327358247	0.245627719865018\\
66.7429524916121	0.24562783520412\\
66.7494722431748	0.245627950510378\\
66.7559919905141	0.245628065783802\\
66.7625117336312	0.245628181024406\\
66.7690314725272	0.245628296232202\\
66.7755512072035	0.245628411407203\\
66.7820709376611	0.24562852654942\\
66.7885906639014	0.245628641658868\\
66.7951103859255	0.245628756735557\\
66.8016301037346	0.2456288717795\\
66.8081498173299	0.245628986790711\\
66.8146695267126	0.2456291017692\\
66.821189231884	0.245629216714982\\
66.8277089328452	0.245629331628067\\
66.8342286295974	0.245629446508469\\
66.8407483221418	0.2456295613562\\
66.8472680104797	0.245629676171272\\
66.8537876946122	0.245629790953699\\
66.8603073745406	0.245629905703491\\
66.866827050266	0.245630020420662\\
66.8733467217896	0.245630135105223\\
66.8798663891126	0.245630249757189\\
66.8863860522363	0.245630364376569\\
66.8929057111618	0.245630478963378\\
66.8994253658904	0.245630593517627\\
66.9059450164231	0.245630708039329\\
66.9124646627613	0.245630822528496\\
66.9189843049061	0.245630936985141\\
66.9255039428587	0.245631051409275\\
66.9320235766203	0.245631165800912\\
66.9385432061922	0.245631280160063\\
66.9450628315754	0.24563139448674\\
66.9515824527712	0.245631508780957\\
66.9581020697808	0.245631623042725\\
66.9646216826053	0.245631737272057\\
66.9711412912461	0.245631851468965\\
66.9776608957042	0.245631965633461\\
66.9841804959809	0.245632079765557\\
66.9907000920772	0.245632193865267\\
66.9972196839946	0.245632307932601\\
67.003739271734	0.245632421967573\\
67.0102588552968	0.245632535970194\\
67.0167784346841	0.245632649940478\\
67.0232980098971	0.245632763878435\\
67.029817580937	0.245632877784079\\
67.0363371478049	0.245632991657421\\
67.0428567105021	0.245633105498475\\
67.0493762690297	0.245633219307251\\
67.055895823389	0.245633333083762\\
67.0624153735811	0.245633446828022\\
67.0689349196072	0.24563356054004\\
67.0754544614684	0.245633674219831\\
67.0819739991661	0.245633787867406\\
67.0884935327013	0.245633901482777\\
67.0950130620753	0.245634015065956\\
67.1015325872891	0.245634128616957\\
67.1080521083441	0.24563424213579\\
67.1145716252414	0.245634355622468\\
67.1210911379822	0.245634469077003\\
67.1276106465676	0.245634582499407\\
67.1341301509988	0.245634695889693\\
67.1406496512771	0.245634809247873\\
67.1471691474036	0.245634922573958\\
67.1536886393794	0.245635035867961\\
67.1602081272058	0.245635149129895\\
67.166727610884	0.24563526235977\\
67.173247090415	0.2456353755576\\
67.1797665658002	0.245635488723395\\
67.1862860370406	0.24563560185717\\
67.1928055041375	0.245635714958935\\
67.1993249670919	0.245635828028702\\
67.2058444259052	0.245635941066485\\
67.2123638805785	0.245636054072294\\
67.2188833311129	0.245636167046142\\
67.2254027775097	0.245636279988041\\
67.2319222197699	0.245636392898003\\
67.2384416578949	0.24563650577604\\
67.2449610918856	0.245636618622164\\
67.2514805217434	0.245636731436388\\
67.2579999474694	0.245636844218722\\
67.2645193690648	0.24563695696918\\
67.2710387865307	0.245637069687773\\
67.2775581998684	0.245637182374514\\
67.2840776090789	0.245637295029414\\
67.2905970141634	0.245637407652485\\
67.2971164151232	0.245637520243739\\
67.3036358119594	0.245637632803189\\
67.3101552046731	0.245637745330847\\
67.3166745932656	0.245637857826723\\
67.323193977738	0.245637970290831\\
67.3297133580914	0.245638082723182\\
67.3362327343271	0.245638195123789\\
67.3427521064462	0.245638307492663\\
67.3492714744498	0.245638419829816\\
67.3557908383392	0.24563853213526\\
67.3623101981155	0.245638644409007\\
67.3688295537798	0.245638756651069\\
67.3753489053334	0.245638868861459\\
67.3818682527773	0.245638981040187\\
67.3883875961129	0.245639093187266\\
67.3949069353411	0.245639205302708\\
67.4014262704632	0.245639317386524\\
67.4079456014804	0.245639429438728\\
67.4144649283938	0.245639541459329\\
67.4209842512046	0.245639653448342\\
67.4275035699139	0.245639765405776\\
67.4340228845229	0.245639877331645\\
67.4405421950327	0.24563998922596\\
67.4470615014446	0.245640101088732\\
67.4535808037596	0.245640212919975\\
67.460100101979	0.245640324719699\\
67.4666193961039	0.245640436487917\\
67.4731386861354	0.24564054822464\\
67.4796579720747	0.245640659929881\\
67.486177253923	0.24564077160365\\
67.4926965316814	0.245640883245961\\
67.499215805351	0.245640994856825\\
67.5057350749332	0.245641106436253\\
67.5122543404289	0.245641217984257\\
67.5187736018393	0.24564132950085\\
67.5252928591657	0.245641440986042\\
67.5318121124091	0.245641552439847\\
67.5383313615707	0.245641663862275\\
67.5448506066517	0.245641775253339\\
67.5513698476532	0.24564188661305\\
67.5578890845763	0.24564199794142\\
67.5644083174223	0.245642109238461\\
67.5709275461923	0.245642220504184\\
67.5774467708873	0.245642331738602\\
67.5839659915087	0.245642442941726\\
67.5904852080574	0.245642554113568\\
67.5970044205348	0.245642665254139\\
67.6035236289419	0.245642776363452\\
67.6100428332798	0.245642887441518\\
67.6165620335497	0.245642998488349\\
67.6230812297529	0.245643109503956\\
67.6296004218903	0.245643220488351\\
67.6361196099632	0.245643331441547\\
67.6426387939727	0.245643442363554\\
67.64915797392	0.245643553254385\\
67.6556771498062	0.245643664114051\\
67.6621963216324	0.245643774942563\\
67.6687154893998	0.245643885739934\\
67.6752346531096	0.245643996506175\\
67.6817538127628	0.245644107241299\\
67.6882729683607	0.245644217945315\\
67.6947921199043	0.245644328618237\\
67.7013112673949	0.245644439260076\\
67.7078304108335	0.245644549870843\\
67.7143495502213	0.24564466045055\\
67.7208686855594	0.24564477099921\\
67.727387816849	0.245644881516832\\
67.7339069440913	0.24564499200343\\
67.7404260672873	0.245645102459015\\
67.7469451864382	0.245645212883597\\
67.7534643015452	0.24564532327719\\
67.7599834126093	0.245645433639805\\
67.7665025196317	0.245645543971452\\
67.7730216226137	0.245645654272144\\
67.7795407215562	0.245645764541893\\
67.7860598164604	0.24564587478071\\
67.7925789073275	0.245645984988606\\
67.7990979941586	0.245646095165593\\
67.8056170769549	0.245646205311684\\
67.8121361557174	0.245646315426888\\
67.8186552304474	0.245646425511218\\
67.8251743011459	0.245646535564686\\
67.8316933678141	0.245646645587302\\
67.8382124304531	0.245646755579079\\
67.844731489064	0.245646865540028\\
67.8512505436481	0.245646975470161\\
67.8577695942063	0.245647085369489\\
67.8642886407399	0.245647195238023\\
67.87080768325	0.245647305075776\\
67.8773267217377	0.245647414882758\\
67.8838457562041	0.245647524658982\\
67.8903647866504	0.245647634404458\\
67.8968838130777	0.245647744119198\\
67.9034028354872	0.245647853803214\\
67.9099218538799	0.245647963456517\\
67.916440868257	0.245648073079119\\
67.9229598786196	0.245648182671031\\
67.9294788849689	0.245648292232265\\
67.935997887306	0.245648401762831\\
67.9425168856319	0.245648511262742\\
67.9490358799479	0.245648620732009\\
67.955554870255	0.245648730170644\\
67.9620738565545	0.245648839578657\\
67.9685928388473	0.245648948956061\\
67.9751118171347	0.245649058302867\\
67.9816307914177	0.245649167619085\\
67.9881497616975	0.245649276904729\\
67.9946687279752	0.245649386159808\\
68.001187690252	0.245649495384335\\
68.0077066485289	0.24564960457832\\
68.014225602807	0.245649713741776\\
68.0207445530876	0.245649822874714\\
68.0272634993717	0.245649931977144\\
68.0337824416604	0.245650041049079\\
68.0403013799549	0.24565015009053\\
68.0468203142563	0.245650259101508\\
68.0533392445657	0.245650368082024\\
68.0598581708842	0.245650477032091\\
68.0663770932129	0.245650585951718\\
68.072896011553	0.245650694840919\\
68.0794149259056	0.245650803699703\\
68.0859338362717	0.245650912528083\\
68.0924527426526	0.245651021326069\\
68.0989716450493	0.245651130093673\\
68.105490543463	0.245651238830907\\
68.1120094378947	0.245651347537781\\
68.1185283283456	0.245651456214307\\
68.1250472148168	0.245651564860497\\
68.1315660973094	0.245651673476361\\
68.1380849758246	0.245651782061911\\
68.1446038503634	0.245651890617158\\
68.1511227209269	0.245651999142114\\
68.1576415875163	0.245652107636789\\
68.1641604501326	0.245652216101195\\
68.1706793087771	0.245652324535344\\
68.1771981634508	0.245652432939246\\
68.1837170141547	0.245652541312914\\
68.1902358608901	0.245652649656357\\
68.1967547036581	0.245652757969588\\
68.2032735424597	0.245652866252617\\
68.209792377296	0.245652974505456\\
68.2163112081682	0.245653082728116\\
68.2228300350774	0.245653190920609\\
68.2293488580247	0.245653299082945\\
68.2358676770111	0.245653407215136\\
68.2423864920379	0.245653515317193\\
68.2489053031061	0.245653623389127\\
68.2554241102168	0.245653731430949\\
68.2619429133711	0.245653839442671\\
68.2684617125701	0.245653947424304\\
68.274980507815	0.245654055375859\\
68.2814992991068	0.245654163297347\\
68.2880180864467	0.24565427118878\\
68.2945368698357	0.245654379050168\\
68.301055649275	0.245654486881523\\
68.3075744247656	0.245654594682855\\
68.3140931963087	0.245654702454177\\
68.3206119639054	0.245654810195499\\
68.3271307275567	0.245654917906832\\
68.3336494872638	0.245655025588188\\
68.3401682430277	0.245655133239578\\
68.3466869948497	0.245655240861012\\
68.3532057427307	0.245655348452502\\
68.3597244866719	0.245655456014059\\
68.3662432266743	0.245655563545694\\
68.3727619627391	0.245655671047419\\
68.3792806948674	0.245655778519244\\
68.3857994230603	0.24565588596118\\
68.3923181473189	0.245655993373239\\
68.3988368676442	0.245656100755432\\
68.4053555840374	0.245656208107769\\
68.4118742964995	0.245656315430262\\
68.4183930050317	0.245656422722922\\
68.424911709635	0.24565652998576\\
68.4314304103107	0.245656637218788\\
68.4379491070596	0.245656744422015\\
68.444467799883	0.245656851595453\\
68.450986488782	0.245656958739114\\
68.4575051737575	0.245657065853008\\
68.4640238548108	0.245657172937146\\
68.4705425319429	0.24565727999154\\
68.477061205155	0.2456573870162\\
68.483579874448	0.245657494011137\\
68.4900985398232	0.245657600976363\\
68.4966172012815	0.245657707911888\\
68.5031358588241	0.245657814817724\\
68.5096545124521	0.245657921693881\\
68.5161731621665	0.24565802854037\\
68.5226918079685	0.245658135357204\\
68.5292104498592	0.245658242144391\\
68.5357290878395	0.245658348901944\\
68.5422477219107	0.245658455629873\\
68.5487663520739	0.24565856232819\\
68.55528497833	0.245658668996905\\
68.5618036006802	0.24565877563603\\
68.5683222191255	0.245658882245574\\
68.5748408336672	0.24565898882555\\
68.5813594443061	0.245659095375968\\
68.5878780510436	0.24565920189684\\
68.5943966538805	0.245659308388175\\
68.600915252818	0.245659414849985\\
68.6074338478573	0.245659521282282\\
68.6139524389993	0.245659627685075\\
68.6204710262452	0.245659734058376\\
68.626989609596	0.245659840402195\\
68.6335081890528	0.245659946716544\\
68.6400267646168	0.245660053001434\\
68.6465453362889	0.245660159256875\\
68.6530639040703	0.245660265482879\\
68.6595824679621	0.245660371679455\\
68.6661010279653	0.245660477846616\\
68.6726195840811	0.245660583984372\\
68.6791381363104	0.245660690092733\\
68.6856566846544	0.245660796171711\\
68.6921752291142	0.245660902221317\\
68.6986937696908	0.245661008241561\\
68.7052123063854	0.245661114232455\\
68.7117308391989	0.245661220194008\\
68.7182493681325	0.245661326126233\\
68.7247678931873	0.245661432029139\\
68.7312864143642	0.245661537902738\\
68.7378049316646	0.24566164374704\\
68.7443234450892	0.245661749562057\\
68.7508419546394	0.245661855347799\\
68.7573604603161	0.245661961104277\\
68.7638789621204	0.245662066831501\\
68.7703974600533	0.245662172529483\\
68.7769159541161	0.245662278198234\\
68.7834344443097	0.245662383837763\\
68.7899529306352	0.245662489448083\\
68.7964714130936	0.245662595029203\\
68.8029898916862	0.245662700581134\\
68.8095083664138	0.245662806103888\\
68.8160268372777	0.245662911597475\\
68.8225453042788	0.245663017061906\\
68.8290637674183	0.245663122497191\\
68.8355822266972	0.245663227903341\\
68.8421006821165	0.245663333280368\\
68.8486191336775	0.245663438628281\\
68.855137581381	0.245663543947092\\
68.8616560252283	0.245663649236811\\
68.8681744652203	0.245663754497449\\
68.8746929013582	0.245663859729017\\
68.8812113336429	0.245663964931525\\
68.8877297620756	0.245664070104985\\
68.8942481866574	0.245664175249406\\
68.9007666073892	0.2456642803648\\
68.9072850242723	0.245664385451177\\
68.9138034373075	0.245664490508548\\
68.9203218464961	0.245664595536924\\
68.9268402518391	0.245664700536315\\
68.9333586533374	0.245664805506732\\
68.9398770509923	0.245664910448186\\
68.9463954448047	0.245665015360687\\
68.9529138347758	0.245665120244246\\
68.9594322209065	0.245665225098874\\
68.965950603198	0.245665329924582\\
68.9724689816514	0.245665434721379\\
68.9789873562676	0.245665539489277\\
68.9855057270477	0.245665644228286\\
68.9920240939928	0.245665748938417\\
68.998542457104	0.24566585361968\\
69.0050608163823	0.245665958272087\\
69.0115791718288	0.245666062895648\\
69.0180975234446	0.245666167490373\\
69.0246158712306	0.245666272056272\\
69.031134215188	0.245666376593358\\
69.0376525553179	0.24566648110164\\
69.0441708916212	0.245666585581128\\
69.0506892240991	0.245666690031835\\
69.0572075527525	0.245666794453769\\
69.0637258775826	0.245666898846941\\
69.0702441985904	0.245667003211363\\
69.076762515777	0.245667107547044\\
69.0832808291434	0.245667211853996\\
69.0897991386907	0.245667316132229\\
69.0963174444199	0.245667420381753\\
69.1028357463321	0.245667524602579\\
69.1093540444284	0.245667628794717\\
69.1158723387097	0.245667732958179\\
69.1223906291772	0.245667837092974\\
69.1289089158319	0.245667941199114\\
69.1354271986748	0.245668045276608\\
69.1419454777071	0.245668149325467\\
69.1484637529297	0.245668253345702\\
69.1549820243438	0.245668357337323\\
69.1615002919503	0.245668461300341\\
69.1680185557503	0.245668565234766\\
69.1745368157449	0.245668669140608\\
69.1810550719351	0.245668773017879\\
69.1875733243219	0.245668876866589\\
69.1940915729065	0.245668980686748\\
69.2006098176899	0.245669084478366\\
69.2071280586731	0.245669188241455\\
69.2136462958571	0.245669291976025\\
69.220164529243	0.245669395682085\\
69.2266827588319	0.245669499359647\\
69.2332009846248	0.245669603008721\\
69.2397192066228	0.245669706629317\\
69.2462374248268	0.245669810221447\\
69.252755639238	0.245669913785119\\
69.2592738498574	0.245670017320346\\
69.265792056686	0.245670120827137\\
69.2723102597249	0.245670224305502\\
69.2788284589751	0.245670327755452\\
69.2853466544377	0.245670431176998\\
69.2918648461137	0.24567053457015\\
69.2983830340042	0.245670637934918\\
69.3049012181101	0.245670741271313\\
69.3114193984326	0.245670844579345\\
};
\addplot [color=mycolor1,solid,forget plot]
  table[row sep=crcr]{%
0	-0.1\\
0.0224632231456145	-0.100008430943879\\
0.0449223713638281	-0.100036398838157\\
0.0673711924404585	-0.100083826706216\\
0.0898034789452708	-0.100150579605725\\
0.112213096503054	-0.100236466138357\\
0.134594011040437	-0.10034124060555\\
0.156940314633586	-0.100464605754619\\
0.179246249622742	-0.100606216045504\\
0.201506230708838	-0.100765681357569\\
0.223714864802914	-0.100942571048402\\
0.245866968458122	-0.1011364182727\\
0.267957582774245	-0.101346724468891\\
0.289981985723546	-0.101572963923982\\
0.311935701902218	-0.101814588332654\\
0.333814509762048	-0.102071031274478\\
0.355614446420859	-0.102341712542553\\
0.377331810187033	-0.102626042267425\\
0.398963160962422	-0.10292342479112\\
0.420505318709457	-0.10323326225712\\
0.441955360182294	-0.103554957892611\\
0.46331061412925	-0.103887918969075\\
0.484568655175138	-0.10423155943596\\
0.505727296588457	-0.104585302229697\\
0.526784582130505	-0.104948581266547\\
0.547738777172428	-0.105320843132802\\
0.568588359252751	-0.105701548489624\\
0.589332008232935	-0.106090173212549\\
0.609968596192604	-0.106486209287386\\
0.630497177189917	-0.106889165485127\\
0.650916976996582	-0.107298567838679\\
0.671227382901604	-0.107713959943817\\
0.691427933663268	-0.108134903105969\\
0.71151830967537	-0.108560976353262\\
0.731498323401301	-0.108991776334907\\
0.751367910118463	-0.109426917122501\\
0.771127119005553	-0.109866029930232\\
0.790776104596539	-0.110308762768387\\
0.810315118617581	-0.110754780043042\\
0.829744502216619	-0.111203762113269\\
0.849064678589903	-0.111655404815839\\
0.868276146005066	-0.112109418966062\\
0.887379471216601	-0.112565529842234\\
0.906375283266488	-0.113023476660054\\
0.925264267660277	-0.11348301204241\\
0.944047160907045	-0.113943901489071\\
0.962724745410175	-0.114405922850041\\
0.981297844694921	-0.114868865805661\\
0.999767318957992	-0.11533253135598\\
1.018134060924	-0.115796731321366\\
1.03639899199342	-0.116261287855943\\
1.05456305866678	-0.116726032975019\\
1.07262722922992	-0.11719080809739\\
1.09059249068538	-0.117655463603117\\
1.10845984591565	-0.118119858407148\\
1.12623031106408	-0.118583859548989\\
1.14390491312009	-0.11904734179845\\
1.16148468769573	-0.119510187277378\\
1.17897067698116	-0.119972285097187\\
1.19636392786742	-0.120433531011912\\
1.21366549022523	-0.120893827086442\\
1.23087641532925	-0.121353081379548\\
1.24799775441786	-0.121811207641278\\
1.26503055737899	-0.12226812502426\\
1.28197587155307	-0.122723757808449\\
1.29883474064497	-0.123178035138827\\
1.3156082037368	-0.123630890775565\\
1.33229729439453	-0.124082262856162\\
1.34890303986133	-0.124532093669064\\
1.36542646033135	-0.12498032943829\\
1.38186856829793	-0.125426920118587\\
1.39823036797046	-0.125871819200653\\
1.41451285475497	-0.126314983525973\\
1.43071701479331	-0.12675637311084\\
1.44684382455648	-0.127195950979127\\
1.46289425048792	-0.12763368300341\\
1.47886924869279	-0.128069537754042\\
1.49476976466962	-0.128503486355807\\
1.51059673308079	-0.128935502351777\\
1.526351077559	-0.129365561574039\\
1.54203371054641	-0.129793642020947\\
1.55764553316405	-0.130219723740586\\
1.5731874351088	-0.130643788720132\\
1.58866029457569	-0.131065820780833\\
1.60406497820329	-0.131485805478317\\
1.61940234104027	-0.131903730007969\\
1.63467322653111	-0.132319583115129\\
1.64987846651948	-0.132733355009848\\
1.66501888126745	-0.133145037286011\\
1.68009527948923	-0.133554622844562\\
1.69510845839801	-0.133962105820664\\
1.71005920376461	-0.134367481514575\\
1.7249482899869	-0.134770746326048\\
1.7397764801687	-0.135171897692097\\
1.75454452620747	-0.135570934027941\\
1.76925316888955	-0.135967854670966\\
1.78390313799236	-0.136362659827567\\
1.79849515239264	-0.136755350522703\\
1.81302992018011	-0.137145928552045\\
1.82750813877579	-0.137534396436577\\
1.84193049505446	-0.137920757379522\\
1.85629766547064	-0.138305015225485\\
1.87061031618768	-0.138687174421696\\
1.88486910320927	-0.139067239981237\\
1.89907467251331	-0.139445217448171\\
1.91322766018727	-0.139821112864462\\
1.92732869256515	-0.140194932738595\\
1.94137838636534	-0.140566684015825\\
1.95537734882933	-0.14093637404995\\
1.96932617786077	-0.14130401057655\\
1.98322546216494	-0.141669601687606\\
1.99707578138798	-0.142033155807439\\
2.01087770625611	-0.142394681669888\\
2.02463179871433	-0.142754188296678\\
2.03833861206457	-0.143111684976916\\
2.05199869110311	-0.143467181247646\\
2.06561257225717	-0.143820686875425\\
2.07918078372047	-0.14417221183886\\
2.09270384558764	-0.144521766312058\\
2.10618226998757	-0.144869360648943\\
2.1196165612153	-0.145215005368406\\
2.13300721586267	-0.145558711140222\\
2.14635472294744	-0.145900488771727\\
2.15965956404093	-0.146240349195193\\
2.17292221339419	-0.14657830345587\\
2.18614313806241	-0.146914362700676\\
2.19932279802784	-0.147248538167481\\
2.21246164632094	-0.147580841174972\\
2.22556012913992	-0.147911283113061\\
2.23861868596843	-0.148239875433815\\
2.25163774969165	-0.148566629642872\\
2.26461774671055	-0.148891557291333\\
2.2775590970544	-0.149214669968091\\
2.29046221449152	-0.149535979292585\\
2.30332750663826	-0.149855496907958\\
2.31615537506624	-0.150173234474591\\
2.32894621540776	-0.150489203664003\\
2.3417004174595	-0.150803416153088\\
2.35441836528449	-0.151115883618691\\
2.36710043731227	-0.151426617732479\\
2.37974700643739	-0.151735630156118\\
2.39235844011617	-0.15204293253673\\
2.40493510046171	-0.152348536502603\\
2.41747734433732	-0.152652453659172\\
2.42998552344819	-0.152954695585226\\
2.44245998443148	-0.153255273829349\\
2.4549010689447	-0.153554199906578\\
2.46730911375258	-0.153851485295264\\
2.47968445081227	-0.154147141434127\\
2.49202740735701	-0.154441179719505\\
2.50433830597827	-0.154733611502769\\
2.51661746470632	-0.15502444808791\\
2.52886519708934	-0.155313700729287\\
2.54108181227105	-0.155601380629521\\
2.55326761506685	-0.155887498937539\\
2.56542290603856	-0.156172066746743\\
2.57754798156776	-0.156455095093323\\
2.58964313392771	-0.15673659495468\\
2.60170865135395	-0.157016577247972\\
2.61374481811353	-0.157295052828766\\
2.62575191457291	-0.157572032489802\\
2.63773021726463	-0.157847526959848\\
2.64967999895267	-0.158121546902655\\
2.66160152869655	-0.158394102916001\\
2.67349507191423	-0.158665205530814\\
2.68536089044381	-0.158934865210385\\
2.69719924260405	-0.15920309234965\\
2.70901038325372	-0.159469897274549\\
2.72079456384983	-0.159735290241448\\
2.73255203250474	-0.159999281436631\\
2.74428303404218	-0.160261880975854\\
2.75598781005221	-0.160523098903951\\
2.7676665989451	-0.160782945194503\\
2.77931963600421	-0.161041429749552\\
2.79094715343785	-0.161298562399372\\
2.80254938043016	-0.16155435290228\\
2.81412654319101	-0.161808810944494\\
2.82567886500493	-0.162061946140036\\
2.83720656627916	-0.162313768030665\\
2.84870986459072	-0.16256428608586\\
2.86018897473266	-0.162813509702828\\
2.87164410875936	-0.163061448206553\\
2.88307547603103	-0.163308110849864\\
2.89448328325733	-0.163553506813553\\
2.90586773454017	-0.163797645206499\\
2.91722903141574	-0.164040535065832\\
2.92856737289566	-0.164282185357118\\
2.9398829555075	-0.164522604974567\\
2.95117597333435	-0.16476180274126\\
2.96244661805382	-0.164999787409406\\
2.97369507897623	-0.165236567660602\\
2.98492154308206	-0.165472152106129\\
2.99612619505875	-0.165706549287253\\
3.00730921733681	-0.165939767675546\\
3.01847079012525	-0.166171815673217\\
3.02961109144631	-0.166402701613469\\
3.04073029716964	-0.166632433760852\\
3.05182858104573	-0.166861020311644\\
3.06290611473885	-0.167088469394228\\
3.07396306785928	-0.167314789069494\\
3.08499960799498	-0.167539987331238\\
3.09601590074272	-0.16776407210658\\
3.10701210973856	-0.167987051256382\\
3.11798839668787	-0.168208932575676\\
3.12894492139471	-0.168429723794104\\
3.13988184179073	-0.168649432576355\\
3.15079931396352	-0.168868066522616\\
3.1616974921845	-0.169085633169024\\
3.1725765289362	-0.16930213998812\\
3.18343657493917	-0.169517594389315\\
3.19427777917834	-0.169732003719355\\
3.20510028892886	-0.169945375262783\\
3.21590424978163	-0.17015771624242\\
3.22668980566817	-0.170369033819831\\
3.23745709888526	-0.170579335095804\\
3.24820627011895	-0.170788627110826\\
3.25893745846827	-0.170996916845566\\
3.2696508014685	-0.171204211221349\\
3.28034643511396	-0.171410517100642\\
3.2910244938805	-0.171615841287534\\
3.30168511074751	-0.171820190528215\\
3.3123284172196	-0.172023571511462\\
3.32295454334787	-0.172225990869118\\
3.33356361775083	-0.172427455176574\\
3.34415576763493	-0.172627970953247\\
3.35473111881477	-0.172827544663063\\
3.36528979573293	-0.173026182714934\\
3.37583192147947	-0.173223891463234\\
3.3863576178111	-0.173420677208277\\
3.39686700516999	-0.17361654619679\\
3.40736020270234	-0.173811504622386\\
3.4178373282765	-0.174005558626035\\
3.42829849850092	-0.174198714296531\\
3.4387438287417	-0.174390977670963\\
3.44917343313988	-0.174582354735173\\
3.45958742462844	-0.174772851424227\\
3.469985914949	-0.174962473622866\\
3.48036901466825	-0.175151227165968\\
3.4907368331941	-0.175339117839004\\
3.50108947879155	-0.175526151378487\\
3.51142705859833	-0.17571233347242\\
3.52174967864023	-0.175897669760749\\
3.53205744384621	-0.176082165835801\\
3.54235045806327	-0.176265827242726\\
3.55262882407103	-0.176448659479936\\
3.56289264359609	-0.176630667999539\\
3.57314201732621	-0.176811858207769\\
3.58337704492416	-0.176992235465419\\
3.59359782504141	-0.177171805088261\\
3.60380445533157	-0.177350572347472\\
3.61399703246364	-0.17752854247005\\
3.62417565213502	-0.177705720639235\\
3.6343404090843	-0.177882111994913\\
3.6444913971039	-0.178057721634033\\
3.65462870905242	-0.178232554611009\\
3.66475243686687	-0.178406615938122\\
3.67486267157466	-0.178579910585922\\
3.68495950330545	-0.178752443483621\\
3.69504302130271	-0.178924219519486\\
3.70511331393522	-0.179095243541229\\
3.71517046870831	-0.179265520356393\\
3.72521457227492	-0.179435054732732\\
3.73524571044657	-0.179603851398591\\
3.74526396820402	-0.179771915043283\\
3.7552694297079	-0.179939250317461\\
3.76526217830909	-0.180105861833483\\
3.77524229655897	-0.180271754165781\\
3.7852098662195	-0.180436931851225\\
3.79516496827314	-0.180601399389474\\
3.80510768293265	-0.180765161243341\\
3.8150380896507	-0.180928221839137\\
3.82495626712933	-0.181090585567026\\
3.8348622933293	-0.181252256781365\\
3.84475624547931	-0.181413239801051\\
3.85463820008498	-0.181573538909855\\
3.86450823293781	-0.181733158356762\\
3.87436641912396	-0.181892102356301\\
3.88421283303287	-0.182050375088876\\
3.89404754836579	-0.182207980701088\\
3.90387063814416	-0.182364923306062\\
3.91368217471786	-0.182521206983766\\
3.92348222977339	-0.182676835781325\\
3.93327087434183	-0.182831813713335\\
3.94304817880677	-0.182986144762175\\
3.95281421291209	-0.183139832878315\\
3.96256904576962	-0.183292881980615\\
3.97231274586666	-0.18344529595663\\
3.98204538107348	-0.183597078662908\\
3.99176701865062	-0.183748233925283\\
4.00147772525609	-0.18389876553917\\
4.01117756695254	-0.184048677269851\\
4.02086660921424	-0.184197972852761\\
4.03054491693399	-0.184346655993776\\
4.04021255442997	-0.184494730369487\\
4.0498695854524	-0.184642199627481\\
4.05951607319022	-0.184789067386617\\
4.06915208027754	-0.184935337237294\\
4.07877766880015	-0.185081012741723\\
4.08839290030179	-0.185226097434192\\
4.09799783579044	-0.185370594821331\\
4.10759253574446	-0.18551450838237\\
4.11717706011866	-0.185657841569401\\
4.12675146835028	-0.18580059780763\\
4.13631581936492	-0.185942780495633\\
4.14587017158233	-0.186084393005603\\
4.15541458292214	-0.186225438683602\\
4.16494911080954	-0.186365920849802\\
4.17447381218085	-0.186505842798727\\
4.183988743489	-0.1866452077995\\
4.19349396070898	-0.186784019096073\\
4.20298951934317	-0.186922279907463\\
4.21247547442661	-0.187059993427992\\
4.22195188053222	-0.187197162827506\\
4.23141879177592	-0.187333791251614\\
4.24087626182167	-0.187469881821905\\
4.2503243438865	-0.187605437636174\\
4.25976309074538	-0.187740461768646\\
4.26919255473612	-0.18787495727019\\
4.27861278776415	-0.188008927168537\\
4.28802384130721	-0.188142374468494\\
4.29742576642006	-0.188275302152158\\
4.30681861373903	-0.18840771317912\\
4.3162024334866	-0.188539610486677\\
4.32557727547583	-0.188670996990039\\
4.3349431891148	-0.188801875582524\\
4.34430022341099	-0.188932249135768\\
4.35364842697552	-0.189062120499919\\
4.36298784802742	-0.189191492503834\\
4.37231853439784	-0.189320367955275\\
4.38164053353413	-0.189448749641102\\
4.39095389250395	-0.189576640327462\\
4.40025865799928	-0.189704042759981\\
4.4095548763404	-0.189830959663945\\
4.41884259347978	-0.189957393744492\\
4.42812185500595	-0.19008334768679\\
4.43739270614737	-0.190208824156218\\
4.44665519177611	-0.19033382579855\\
4.45590935641162	-0.190458355240126\\
4.4651552442244	-0.190582415088031\\
4.4743928990396	-0.190706007930269\\
4.48362236434061	-0.190829136335933\\
4.49284368327257	-0.190951802855376\\
4.50205689864587	-0.191074010020379\\
4.51126205293961	-0.191195760344319\\
4.52045918830495	-0.19131705632233\\
4.52964834656848	-0.191437900431472\\
4.53882956923555	-0.191558295130887\\
4.54800289749352	-0.191678242861961\\
4.55716837221498	-0.191797746048484\\
4.56632603396096	-0.191916807096802\\
4.57547592298406	-0.192035428395976\\
4.58461807923154	-0.192153612317934\\
4.59375254234842	-0.192271361217623\\
4.60287935168052	-0.192388677433157\\
4.61199854627739	-0.192505563285969\\
4.62111016489532	-0.192622021080956\\
4.63021424600026	-0.192738053106625\\
4.63931082777066	-0.192853661635236\\
4.64839994810036	-0.192968848922948\\
4.65748164460136	-0.193083617209958\\
4.66655595460665	-0.193197968720638\\
4.67562291517291	-0.19331190566368\\
4.68468256308324	-0.193425430232226\\
4.69373493484982	-0.193538544604007\\
4.70278006671658	-0.193651250941477\\
4.71181799466179	-0.193763551391947\\
4.72084875440065	-0.193875448087712\\
4.72987238138785	-0.193986943146186\\
4.73888891082007	-0.194098038670029\\
4.74789837763848	-0.194208736747271\\
4.75690081653118	-0.194319039451445\\
4.76589626193567	-0.194428948841707\\
4.7748847480412	-0.194538466962961\\
4.7838663087912	-0.19464759584598\\
4.79284097788557	-0.194756337507531\\
4.80180878878302	-0.19486469395049\\
4.81076977470335	-0.194972667163966\\
4.81972396862974	-0.195080259123412\\
4.82867140331095	-0.195187471790748\\
4.83761211126354	-0.195294307114473\\
4.84654612477406	-0.195400767029779\\
4.85547347590121	-0.195506853458664\\
4.86439419647795	-0.195612568310044\\
4.87330831811363	-0.195717913479866\\
4.88221587219608	-0.195822890851211\\
4.89111688989363	-0.195927502294412\\
4.90001140215719	-0.196031749667153\\
4.90889943972223	-0.196135634814579\\
4.9177810331108	-0.196239159569401\\
4.92665621263345	-0.196342325752002\\
4.93552500839123	-0.196445135170536\\
4.94438745027755	-0.196547589621033\\
4.95324356798013	-0.196649690887502\\
4.96209339098285	-0.196751440742026\\
4.97093694856763	-0.196852840944866\\
4.97977426981623	-0.196953893244557\\
4.98860538361209	-0.197054599378005\\
4.99743031864213	-0.197154961070586\\
5.00624910339848	-0.197254980036237\\
5.0150617661803	-0.197354657977554\\
5.02386833509547	-0.197453996585885\\
5.03266883806231	-0.19755299754142\\
5.04146330281128	-0.197651662513288\\
5.05025175688667	-0.19774999315964\\
5.05903422764823	-0.197847991127747\\
5.06781074227284	-0.197945658054085\\
5.0765813277561	-0.198042995564422\\
5.08534601091397	-0.198140005273908\\
5.09410481838434	-0.198236688787161\\
5.10285777662858	-0.198333047698351\\
5.11160491193312	-0.198429083591288\\
5.12034625041096	-0.198524798039502\\
5.12908181800323	-0.198620192606328\\
5.13781164048063	-0.198715268844991\\
5.14653574344497	-0.198810028298683\\
5.15525415233059	-0.198904472500648\\
5.16396689240588	-0.198998602974259\\
5.17267398877464	-0.199092421233098\\
5.18137546637755	-0.199185928781036\\
5.19007134999358	-0.19927912711231\\
5.19876166424135	-0.199372017711599\\
5.20744643358051	-0.199464602054103\\
5.21612568231316	-0.199556881605613\\
5.2247994345851	-0.199648857822594\\
5.23346771438726	-0.199740532152253\\
5.24213054555693	-0.199831906032613\\
5.25078795177916	-0.19992298089259\\
5.25943995658795	-0.20001375815206\\
5.26808658336761	-0.200104239221933\\
5.27672785535399	-0.200194425504224\\
5.2853637956357	-0.200284318392123\\
5.29399442715543	-0.200373919270061\\
5.30261977271107	-0.200463229513785\\
5.31123985495703	-0.200552250490421\\
5.31985469640534	-0.200640983558544\\
5.3284643194269	-0.200729430068243\\
5.33706874625264	-0.200817591361189\\
5.34566799897468	-0.2009054687707\\
5.35426209954746	-0.200993063621807\\
5.36285106978892	-0.201080377231316\\
5.37143493138159	-0.201167410907873\\
5.38001370587371	-0.201254165952028\\
5.38858741468037	-0.201340643656297\\
5.39715607908455	-0.201426845305225\\
5.40571972023823	-0.201512772175446\\
5.41427835916346	-0.201598425535743\\
5.42283201675341	-0.201683806647111\\
5.43138071377342	-0.201768916762818\\
5.43992447086206	-0.201853757128458\\
5.44846330853211	-0.201938328982017\\
5.45699724717163	-0.202022633553925\\
5.46552630704492	-0.202106672067118\\
5.47405050829356	-0.202190445737095\\
5.48256987093735	-0.20227395577197\\
5.49108441487533	-0.202357203372535\\
5.49959415988673	-0.202440189732308\\
5.50809912563192	-0.202522916037596\\
5.51659933165335	-0.202605383467542\\
5.52509479737652	-0.202687593194185\\
5.53358554211089	-0.20276954638251\\
5.54207158505079	-0.202851244190503\\
5.55055294527634	-0.202932687769203\\
5.55902964175438	-0.203013878262754\\
5.5675016933393	-0.203094816808457\\
5.57596911877398	-0.20317550453682\\
5.58443193669065	-0.203255942571611\\
5.59289016561173	-0.203336132029908\\
5.60134382395073	-0.203416074022146\\
5.60979293001308	-0.20349576965217\\
5.61823750199696	-0.203575220017281\\
5.62667755799416	-0.203654426208287\\
5.63511311599089	-0.203733389309551\\
5.64354419386859	-0.203812110399035\\
5.65197080940476	-0.203890590548354\\
5.66039298027375	-0.203968830822815\\
5.66881072404753	-0.20404683228147\\
5.67722405819654	-0.204124595977158\\
5.68563300009041	-0.204202122956553\\
5.69403756699876	-0.204279414260207\\
5.70243777609194	-0.204356470922597\\
5.71083364444181	-0.204433293972168\\
5.71922518902251	-0.204509884431379\\
5.72761242671113	-0.204586243316742\\
5.73599537428852	-0.204662371638872\\
5.74437404843998	-0.204738270402523\\
5.752748465756	-0.204813940606637\\
5.76111864273297	-0.204889383244381\\
5.76948459577385	-0.204964599303191\\
5.77784634118895	-0.205039589764814\\
5.78620389519655	-0.205114355605348\\
5.79455727392364	-0.205188897795282\\
5.80290649340656	-0.20526321729954\\
5.81125156959173	-0.205337315077514\\
5.81959251833625	-0.205411192083113\\
5.82792935540862	-0.205484849264794\\
5.83626209648937	-0.205558287565606\\
5.84459075717173	-0.205631507923225\\
5.85291535296225	-0.205704511269996\\
5.86123589928145	-0.20577729853297\\
5.86955241146446	-0.20584987063394\\
5.87786490476165	-0.205922228489478\\
5.88617339433923	-0.205994373010975\\
5.8944778952799	-0.206066305104676\\
5.90277842258341	-0.206138025671716\\
5.91107499116723	-0.206209535608157\\
5.91936761586708	-0.206280835805023\\
5.92765631143759	-0.206351927148336\\
5.93594109255282	-0.20642281051915\\
5.94422197380693	-0.20649348679359\\
5.95249896971465	-0.206563956842878\\
5.96077209471195	-0.206634221533379\\
5.96904136315656	-0.206704281726625\\
5.97730678932852	-0.206774138279354\\
5.98556838743079	-0.20684379204354\\
5.99382617158974	-0.20691324386643\\
6.00208015585574	-0.206982494590576\\
6.01033035420369	-0.207051545053864\\
6.01857678053355	-0.20712039608955\\
6.02681944867089	-0.207189048526292\\
6.03505837236741	-0.207257503188181\\
6.04329356530143	-0.20732576089477\\
6.05152504107849	-0.20739382246111\\
6.05975281323178	-0.207461688697778\\
6.06797689522269	-0.207529360410908\\
6.07619730044132	-0.207596838402223\\
6.08441404220697	-0.207664123469063\\
6.09262713376862	-0.207731216404418\\
6.10083658830545	-0.207798117996954\\
6.10904241892733	-0.207864829031045\\
6.11724463867528	-0.207931350286804\\
6.12544326052194	-0.207997682540106\\
6.1336382973721	-0.208063826562623\\
6.14182976206312	-0.208129783121852\\
6.1500176673654	-0.208195552981136\\
6.15820202598289	-0.208261136899704\\
6.16638285055348	-0.208326535632688\\
6.17456015364951	-0.208391749931157\\
6.18273394777819	-0.208456780542142\\
6.19090424538207	-0.208521628208664\\
6.19907105883947	-0.208586293669761\\
6.20723440046491	-0.208650777660513\\
6.21539428250957	-0.208715080912073\\
6.22355071716171	-0.208779204151687\\
6.23170371654708	-0.208843148102725\\
6.23985329272938	-0.208906913484707\\
6.24799945771066	-0.208970501013323\\
6.25614222343175	-0.209033911400466\\
6.26428160177264	-0.209097145354254\\
6.27241760455294	-0.209160203579052\\
6.28055024353225	-0.209223086775503\\
6.2886795304106	-0.209285795640547\\
6.29680547682879	-0.209348330867449\\
6.30492809436885	-0.209410693145824\\
6.31304739455442	-0.209472883161656\\
6.32116338885109	-0.209534901597328\\
6.32927608866686	-0.209596749131642\\
6.33738550535247	-0.209658426439843\\
6.34549165020181	-0.209719934193645\\
6.35359453445228	-0.209781273061249\\
6.36169416928518	-0.209842443707372\\
6.36979056582607	-0.209903446793264\\
6.37788373514515	-0.209964282976737\\
6.38597368825759	-0.21002495291218\\
6.39406043612396	-0.210085457250589\\
6.40214398965051	-0.210145796639582\\
6.41022435968959	-0.210205971723426\\
6.41830155703997	-0.210265983143056\\
6.42637559244719	-0.210325831536099\\
6.43444647660392	-0.210385517536893\\
6.44251422015032	-0.210445041776507\\
6.45057883367434	-0.210504404882769\\
6.45864032771209	-0.210563607480278\\
6.46669871274816	-0.210622650190431\\
6.47475399921598	-0.210681533631441\\
6.48280619749813	-0.210740258418357\\
6.49085531792664	-0.210798825163087\\
6.4989013707834	-0.210857234474416\\
6.50694436630039	-0.210915486958024\\
6.51498431466006	-0.210973583216511\\
6.52302122599561	-0.211031523849413\\
6.53105511039135	-0.21108930945322\\
6.53908597788297	-0.211146940621402\\
6.54711383845787	-0.21120441794442\\
6.55513870205547	-0.21126174200975\\
6.5631605785675	-0.211318913401902\\
6.57117947783833	-0.211375932702436\\
6.57919540966523	-0.211432800489983\\
6.58720838379873	-0.211489517340262\\
6.59521840994284	-0.211546083826101\\
6.6032254977554	-0.211602500517452\\
6.61122965684836	-0.211658767981408\\
6.61923089678805	-0.211714886782229\\
6.62722922709547	-0.211770857481348\\
6.63522465724661	-0.211826680637398\\
6.64321719667269	-0.211882356806226\\
6.65120685476045	-0.21193788654091\\
6.65919364085244	-0.211993270391777\\
6.66717756424728	-0.212048508906421\\
6.67515863419997	-0.212103602629719\\
6.68313685992209	-0.212158552103846\\
6.69111225058215	-0.212213357868296\\
6.69908481530581	-0.212268020459894\\
6.70705456317613	-0.212322540412819\\
6.71502150323391	-0.212376918258611\\
6.72298564447784	-0.212431154526197\\
6.73094699586486	-0.2124852497419\\
6.73890556631036	-0.212539204429458\\
6.74686136468844	-0.212593019110041\\
6.7548143998322	-0.212646694302263\\
6.76276468053395	-0.212700230522203\\
6.77071221554546	-0.212753628283416\\
6.77865701357824	-0.212806888096949\\
6.78659908330376	-0.212860010471358\\
6.7945384333537	-0.212912995912725\\
6.80247507232019	-0.212965844924667\\
6.81040900875605	-0.213018558008357\\
6.81834025117505	-0.213071135662535\\
6.82626880805208	-0.213123578383526\\
6.83419468782349	-0.213175886665252\\
6.84211789888721	-0.213228060999247\\
6.85003844960306	-0.213280101874673\\
6.85795634829296	-0.213332009778333\\
6.86587160324113	-0.213383785194686\\
6.87378422269436	-0.213435428605859\\
6.8816942148622	-0.213486940491664\\
6.88960158791717	-0.213538321329611\\
6.89750634999505	-0.213589571594921\\
6.90540850919502	-0.21364069176054\\
6.91330807357992	-0.213691682297153\\
6.92120505117646	-0.213742543673198\\
6.92909944997543	-0.213793276354878\\
6.9369912779319	-0.213843880806175\\
6.94488054296548	-0.213894357488864\\
6.95276725296045	-0.213944706862525\\
6.96065141576604	-0.213994929384558\\
6.96853303919661	-0.214045025510192\\
6.97641213103184	-0.214094995692504\\
6.98428869901697	-0.214144840382424\\
6.99216275086294	-0.214194560028755\\
7.00003429424668	-0.214244155078182\\
7.00790333681121	-0.214293625975284\\
7.01576988616593	-0.214342973162548\\
7.02363394988675	-0.21439219708038\\
7.0314955355163	-0.21444129816712\\
7.03935465056416	-0.214490276859049\\
7.04721130250699	-0.214539133590407\\
7.05506549878878	-0.214587868793401\\
7.062917246821	-0.214636482898218\\
7.07076655398279	-0.214684976333037\\
7.07861342762119	-0.214733349524041\\
7.08645787505126	-0.214781602895427\\
7.0942999035563	-0.21482973686942\\
7.10213952038805	-0.214877751866284\\
7.10997673276683	-0.214925648304331\\
7.11781154788174	-0.214973426599934\\
7.12564397289084	-0.215021087167541\\
7.13347401492134	-0.21506863041968\\
7.14130168106973	-0.215116056766976\\
7.14912697840201	-0.215163366618158\\
7.15694991395384	-0.215210560380072\\
7.1647704947307	-0.215257638457692\\
7.17258872770808	-0.215304601254129\\
7.18040461983164	-0.215351449170644\\
7.18821817801737	-0.215398182606656\\
7.1960294091518	-0.215444801959757\\
7.2038383200921	-0.215491307625715\\
7.21164491766631	-0.215537699998492\\
7.21944920867344	-0.215583979470252\\
7.22725119988371	-0.215630146431368\\
7.23505089803862	-0.215676201270436\\
7.2428483098512	-0.215722144374285\\
7.25064344200612	-0.215767976127984\\
7.25843630115983	-0.215813696914854\\
7.26622689394078	-0.21585930711648\\
7.27401522694952	-0.215904807112717\\
7.28180130675887	-0.215950197281701\\
7.28958513991411	-0.215995477999862\\
7.29736673293306	-0.216040649641927\\
7.3051460923063	-0.216085712580937\\
7.31292322449729	-0.216130667188251\\
7.32069813594253	-0.216175513833558\\
7.32847083305169	-0.216220252884886\\
7.33624132220779	-0.216264884708611\\
7.34400960976729	-0.216309409669467\\
7.35177570206032	-0.216353828130554\\
7.35953960539075	-0.216398140453348\\
7.36730132603636	-0.21644234699771\\
7.375060870249	-0.216486448121896\\
7.3828182442547	-0.216530444182562\\
7.39057345425385	-0.216574335534779\\
7.39832650642128	-0.216618122532037\\
7.40607740690646	-0.216661805526256\\
7.41382616183362	-0.216705384867794\\
7.42157277730186	-0.216748860905455\\
7.42931725938531	-0.2167922339865\\
7.43705961413328	-0.216835504456652\\
7.44479984757035	-0.216878672660109\\
7.45253796569654	-0.216921738939549\\
7.46027397448744	-0.216964703636136\\
7.46800787989432	-0.217007567089538\\
7.47573968784427	-0.217050329637922\\
7.48346940424035	-0.217092991617975\\
7.49119703496169	-0.217135553364903\\
7.49892258586363	-0.217178015212442\\
7.50664606277785	-0.217220377492868\\
7.5143674715125	-0.217262640537004\\
7.52208681785229	-0.217304804674225\\
7.52980410755869	-0.217346870232471\\
7.53751934636997	-0.217388837538248\\
7.54523254000137	-0.217430706916645\\
7.55294369414523	-0.217472478691334\\
7.56065281447106	-0.217514153184579\\
7.56835990662573	-0.217555730717248\\
7.57606497623353	-0.217597211608815\\
7.58376802889633	-0.217638596177372\\
7.59146907019366	-0.217679884739633\\
7.59916810568287	-0.217721077610945\\
7.60686514089921	-0.217762175105291\\
7.61456018135598	-0.217803177535301\\
7.62225323254461	-0.217844085212258\\
7.62994429993479	-0.217884898446106\\
7.6376333889746	-0.217925617545456\\
7.6453205050906	-0.217966242817593\\
7.65300565368793	-0.218006774568485\\
7.66068884015048	-0.218047213102787\\
7.66837006984093	-0.218087558723853\\
7.67604934810091	-0.218127811733736\\
7.68372668025107	-0.218167972433202\\
7.69140207159123	-0.218208041121731\\
7.69907552740046	-0.21824801809753\\
7.7067470529372	-0.218287903657532\\
7.71441665343934	-0.218327698097411\\
7.72208433412438	-0.218367401711582\\
7.72975010018948	-0.218407014793212\\
7.73741395681159	-0.218446537634224\\
7.74507590914755	-0.218485970525305\\
7.7527359623342	-0.218525313755912\\
7.76039412148845	-0.21856456761428\\
7.76805039170746	-0.218603732387426\\
7.77570477806863	-0.218642808361156\\
7.78335728562979	-0.218681795820073\\
7.79100791942928	-0.218720695047581\\
7.798656684486	-0.218759506325894\\
7.80630358579959	-0.21879822993604\\
7.81394862835045	-0.218836866157869\\
7.8215918170999	-0.218875415270056\\
7.82923315699022	-0.218913877550111\\
7.83687265294479	-0.218952253274384\\
7.84451030986819	-0.218990542718069\\
7.85214613264623	-0.219028746155213\\
7.85978012614614	-0.219066863858718\\
7.86741229521657	-0.219104896100352\\
7.87504264468775	-0.219142843150751\\
7.88267117937156	-0.219180705279427\\
7.89029790406162	-0.219218482754773\\
7.89792282353337	-0.219256175844068\\
7.90554594254418	-0.219293784813485\\
7.91316726583343	-0.219331309928094\\
7.92078679812263	-0.219368751451869\\
7.92840454411544	-0.219406109647696\\
7.93602050849783	-0.219443384777372\\
7.94363469593812	-0.21948057710162\\
7.95124711108712	-0.219517686880086\\
7.95885775857814	-0.219554714371349\\
7.96646664302714	-0.219591659832925\\
7.97407376903281	-0.219628523521273\\
7.98167914117662	-0.219665305691802\\
7.98928276402294	-0.219702006598873\\
7.99688464211909	-0.219738626495805\\
8.00448477999547	-0.219775165634885\\
8.0120831821656	-0.219811624267365\\
8.01967985312622	-0.219848002643476\\
8.02727479735738	-0.219884301012426\\
8.03486801932251	-0.219920519622412\\
8.04245952346851	-0.219956658720617\\
8.05004931422582	-0.219992718553223\\
8.05763739600851	-0.22002869936541\\
8.06522377321436	-0.220064601401368\\
8.07280845022492	-0.220100424904292\\
8.08039143140562	-0.220136170116397\\
8.08797272110582	-0.220171837278918\\
8.09555232365892	-0.220207426632115\\
8.1031302433824	-0.220242938415278\\
8.11070648457791	-0.220278372866734\\
8.11828105153137	-0.22031373022385\\
8.12585394851301	-0.220349010723036\\
8.13342517977748	-0.220384214599754\\
8.1409947495639	-0.220419342088522\\
8.14856266209593	-0.220454393422915\\
8.15612892158187	-0.220489368835571\\
8.16369353221472	-0.220524268558202\\
8.17125649817226	-0.220559092821587\\
8.17881782361709	-0.220593841855588\\
8.18637751269676	-0.220628515889147\\
8.19393556954379	-0.220663115150296\\
8.20149199827577	-0.220697639866154\\
8.20904680299544	-0.220732090262941\\
8.2165999877907	-0.220766466565976\\
8.22415155673478	-0.220800768999683\\
8.2317015138862	-0.220834997787596\\
8.23924986328895	-0.220869153152362\\
8.24679660897245	-0.220903235315748\\
8.2543417549517	-0.220937244498643\\
8.26188530522733	-0.220971180921064\\
8.26942726378563	-0.221005044802157\\
8.27696763459867	-0.221038836360207\\
8.28450642162432	-0.221072555812637\\
8.29204362880635	-0.221106203376015\\
8.2995792600745	-0.221139779266055\\
8.3071133193445	-0.221173283697627\\
8.31464581051819	-0.221206716884756\\
8.32217673748355	-0.221240079040627\\
8.32970610411479	-0.221273370377592\\
8.33723391427238	-0.22130659110717\\
8.34476017180316	-0.221339741440055\\
8.35228488054036	-0.221372821586116\\
8.35980804430368	-0.221405831754404\\
8.36732966689937	-0.221438772153156\\
8.37484975212027	-0.221471642989797\\
8.38236830374586	-0.221504444470945\\
8.38988532554238	-0.221537176802414\\
8.39740082126282	-0.221569840189222\\
8.40491479464703	-0.221602434835588\\
8.41242724942174	-0.221634960944941\\
8.41993818930069	-0.221667418719923\\
8.4274476179846	-0.22169980836239\\
8.43495553916129	-0.221732130073421\\
8.44246195650574	-0.221764384053316\\
8.4499668736801	-0.221796570501603\\
8.45747029433381	-0.221828689617042\\
8.46497222210362	-0.221860741597627\\
8.47247266061365	-0.221892726640591\\
8.47997161347546	-0.221924644942407\\
8.48746908428811	-0.221956496698798\\
8.49496507663821	-0.221988282104732\\
8.50245959409995	-0.222020001354433\\
8.50995264023522	-0.222051654641379\\
8.5174442185936	-0.22208324215831\\
8.52493433271247	-0.222114764097229\\
8.53242298611701	-0.222146220649405\\
8.5399101823203	-0.22217761200538\\
8.54739592482338	-0.222208938354967\\
8.55488021711524	-0.222240199887259\\
8.56236306267296	-0.222271396790628\\
8.56984446496169	-0.222302529252732\\
8.57732442743477	-0.222333597460515\\
8.58480295353372	-0.222364601600213\\
8.59228004668832	-0.222395541857358\\
8.5997557103167	-0.222426418416775\\
8.60722994782532	-0.222457231462596\\
8.61470276260907	-0.222487981178253\\
8.62217415805132	-0.222518667746487\\
8.62964413752395	-0.22254929134935\\
8.63711270438743	-0.222579852168208\\
8.64457986199084	-0.222610350383745\\
8.65204561367194	-0.222640786175963\\
8.65950996275722	-0.222671159724192\\
8.66697291256193	-0.222701471207084\\
8.67443446639016	-0.222731720802625\\
8.68189462753488	-0.222761908688132\\
8.68935339927798	-0.222792035040259\\
8.6968107848903	-0.222822100034998\\
8.70426678763174	-0.222852103847686\\
8.71172141075124	-0.222882046653003\\
8.71917465748688	-0.222911928624979\\
8.72662653106589	-0.222941749936995\\
8.73407703470474	-0.222971510761787\\
8.74152617160912	-0.223001211271447\\
8.74897394497406	-0.22303085163743\\
8.75642035798395	-0.223060432030552\\
8.76386541381256	-0.223089952620997\\
8.77130911562312	-0.223119413578318\\
8.77875146656835	-0.22314881507144\\
8.78619246979053	-0.223178157268664\\
8.79363212842149	-0.223207440337667\\
8.80107044558273	-0.22323666444551\\
8.80850742438541	-0.223265829758634\\
8.8159430679304	-0.223294936442869\\
8.82337737930835	-0.223323984663434\\
8.83081036159973	-0.223352974584941\\
8.83824201787484	-0.223381906371394\\
8.8456723511939	-0.223410780186198\\
8.85310136460706	-0.223439596192157\\
8.86052906115447	-0.223468354551478\\
8.86795544386628	-0.223497055425775\\
8.87538051576275	-0.223525698976069\\
8.88280427985423	-0.223554285362793\\
8.89022673914124	-0.223582814745795\\
8.89764789661449	-0.223611287284339\\
8.90506775525492	-0.223639703137107\\
8.9124863180338	-0.223668062462206\\
8.91990358791267	-0.223696365417164\\
8.92731956784346	-0.223724612158939\\
8.93473426076853	-0.223752802843917\\
8.94214766962065	-0.223780937627918\\
8.9495597973231	-0.223809016666194\\
8.95697064678969	-0.223837040113439\\
8.96438022092479	-0.223865008123782\\
8.97178852262339	-0.223892920850799\\
8.97919555477114	-0.223920778447507\\
8.98660132024435	-0.223948581066373\\
8.99400582191008	-0.223976328859314\\
9.00140906262617	-0.224004021977698\\
9.00881104524124	-0.22403166057235\\
9.01621177259477	-0.22405924479355\\
9.02361124751712	-0.22408677479104\\
9.0310094728296	-0.224114250714022\\
9.03840645134444	-0.224141672711164\\
9.04580218586489	-0.224169040930602\\
9.05319667918525	-0.224196355519938\\
9.06058993409087	-0.22422361662625\\
9.06798195335824	-0.224250824396087\\
9.07537273975498	-0.224277978975475\\
9.08276229603989	-0.224305080509919\\
9.09015062496303	-0.224332129144406\\
9.09753772926568	-0.224359125023405\\
9.10492361168043	-0.224386068290872\\
9.11230827493123	-0.224412959090249\\
9.11969172173335	-0.224439797564469\\
9.12707395479351	-0.22446658385596\\
9.13445497680985	-0.22449331810664\\
9.14183479047198	-0.224520000457927\\
9.14921339846105	-0.224546631050738\\
9.15659080344972	-0.224573210025491\\
9.16396700810225	-0.224599737522107\\
9.17134201507453	-0.224626213680013\\
9.17871582701409	-0.224652638638143\\
9.18608844656013	-0.224679012534944\\
9.19345987634359	-0.224705335508372\\
9.20083011898716	-0.224731607695899\\
9.20819917710533	-0.224757829234512\\
9.21556705330438	-0.224784000260718\\
9.22293375018247	-0.224810120910545\\
9.23029927032965	-0.224836191319542\\
9.23766361632787	-0.224862211622784\\
9.24502679075107	-0.224888181954874\\
9.25238879616516	-0.22491410244994\\
9.25974963512805	-0.224939973241646\\
9.26710931018974	-0.224965794463185\\
9.27446782389229	-0.224991566247289\\
9.28182517876989	-0.225017288726223\\
9.28918137734887	-0.225042962031794\\
9.29653642214777	-0.225068586295348\\
9.3038903156773	-0.225094161647777\\
9.31124306044045	-0.225119688219515\\
9.31859465893248	-0.225145166140545\\
9.32594511364094	-0.225170595540397\\
9.33329442704573	-0.225195976548155\\
9.34064260161914	-0.225221309292453\\
9.34798963982583	-0.225246593901481\\
9.3553355441229	-0.225271830502986\\
9.36268031695993	-0.225297019224272\\
9.37002396077897	-0.225322160192205\\
9.3773664780146	-0.225347253533214\\
9.38470787109397	-0.225372299373291\\
9.39204814243677	-0.225397297837993\\
9.39938729445534	-0.225422249052448\\
9.40672532955466	-0.225447153141352\\
9.41406225013236	-0.225472010228972\\
9.42139805857879	-0.225496820439149\\
9.42873275727701	-0.225521583895301\\
9.43606634860286	-0.225546300720421\\
9.44339883492495	-0.225570971037082\\
9.45073021860472	-0.225595594967436\\
9.45806050199644	-0.225620172633221\\
9.46538968744727	-0.225644704155756\\
9.47271777729727	-0.225669189655947\\
9.4800447738794	-0.225693629254288\\
9.48737067951962	-0.225718023070863\\
9.49469549653685	-0.225742371225346\\
9.50201922724302	-0.225766673837004\\
9.50934187394312	-0.225790931024701\\
9.5166634389352	-0.225815142906894\\
9.52398392451039	-0.225839309601642\\
9.53130333295298	-0.2258634312266\\
9.53862166654036	-0.225887507899026\\
9.54593892754314	-0.225911539735783\\
9.55325511822511	-0.225935526853335\\
9.5605702408433	-0.225959469367757\\
9.567884297648	-0.225983367394728\\
9.57519729088276	-0.226007221049539\\
9.58250922278447	-0.226031030447092\\
9.58982009558334	-0.226054795701903\\
9.59712991150294	-0.2260785169281\\
9.60443867276023	-0.226102194239431\\
9.61174638156559	-0.226125827749258\\
9.61905304012283	-0.226149417570565\\
9.62635865062924	-0.226172963815957\\
9.63366321527557	-0.22619646659766\\
9.64096673624611	-0.226219926027526\\
9.64826921571868	-0.226243342217031\\
9.65557065586468	-0.22626671527728\\
9.66287105884908	-0.226290045319007\\
9.67017042683048	-0.226313332452573\\
9.67746876196111	-0.226336576787976\\
9.68476606638686	-0.226359778434843\\
9.69206234224733	-0.226382937502439\\
9.69935759167581	-0.226406054099664\\
9.70665181679934	-0.226429128335057\\
9.71394501973872	-0.226452160316794\\
9.72123720260853	-0.226475150152696\\
9.72852836751717	-0.226498097950222\\
9.73581851656686	-0.226521003816479\\
9.74310765185369	-0.226543867858216\\
9.7503957754676	-0.226566690181832\\
9.75768288949248	-0.226589470893371\\
9.76496899600611	-0.226612210098528\\
9.77225409708023	-0.226634907902651\\
9.77953819478056	-0.226657564410737\\
9.78682129116681	-0.226680179727442\\
9.79410338829272	-0.226702753957071\\
9.80138448820604	-0.226725287203592\\
9.80866459294862	-0.226747779570627\\
9.81594370455638	-0.226770231161459\\
9.82322182505934	-0.226792642079033\\
9.83049895648168	-0.226815012425955\\
9.83777510084171	-0.226837342304496\\
9.84505026015192	-0.22685963181659\\
9.852324436419	-0.226881881063839\\
9.85959763164385	-0.226904090147514\\
9.86686984782164	-0.226926259168553\\
9.87414108694178	-0.226948388227566\\
9.88141135098796	-0.226970477424833\\
9.88868064193818	-0.22699252686031\\
9.89594896176479	-0.227014536633625\\
9.90321631243446	-0.227036506844083\\
9.91048269590825	-0.227058437590666\\
9.91774811414159	-0.227080328972035\\
9.92501256908435	-0.227102181086528\\
9.93227606268081	-0.227123994032169\\
9.93953859686971	-0.227145767906659\\
9.94680017358428	-0.227167502807385\\
9.95406079475222	-0.227189198831421\\
9.96132046229576	-0.227210856075523\\
9.96857917813167	-0.227232474636137\\
9.97583694417126	-0.227254054609396\\
9.98309376232044	-0.227275596091125\\
9.99034963447969	-0.227297099176839\\
9.99760456254414	-0.227318563961744\\
10.0048585484035	-0.227339990540741\\
10.0121115939422	-0.227361379008427\\
10.0193637010394	-0.227382729459091\\
10.0266148715687	-0.227404041986725\\
10.0338651073988	-0.227425316685014\\
10.0411144103928	-0.227446553647345\\
10.0483627824086	-0.227467752966807\\
10.0556102252992	-0.227488914736188\\
10.0628567409119	-0.227510039047982\\
10.0701023310892	-0.227531125994386\\
10.0773469976683	-0.227552175667303\\
10.0845907424813	-0.227573188158342\\
10.0918335673549	-0.227594163558821\\
10.0990754741111	-0.227615101959765\\
10.1063164645665	-0.227636003451912\\
10.1135565405328	-0.227656868125709\\
10.1207957038165	-0.227677696071317\\
10.1280339562193	-0.227698487378609\\
10.1352712995375	-0.227719242137174\\
10.1425077355628	-0.227739960436316\\
10.1497432660816	-0.227760642365057\\
10.1569778928756	-0.227781288012136\\
10.1642116177214	-0.227801897466012\\
10.1714444423906	-0.227822470814863\\
10.1786763686501	-0.22784300814659\\
10.1859073982618	-0.227863509548815\\
10.1931375329826	-0.227883975108884\\
10.2003667745647	-0.227904404913869\\
10.2075951247555	-0.227924799050565\\
10.2148225852974	-0.227945157605497\\
10.2220491579282	-0.227965480664914\\
10.2292748443807	-0.227985768314797\\
10.2364996463831	-0.228006020640857\\
10.2437235656588	-0.228026237728533\\
10.2509466039263	-0.228046419662999\\
10.2581687628997	-0.22806656652916\\
10.2653900442882	-0.228086678411658\\
10.2726104497963	-0.228106755394867\\
10.2798299811238	-0.228126797562899\\
10.287048639966	-0.228146804999602\\
10.2942664280135	-0.228166777788563\\
10.3014833469522	-0.228186716013109\\
10.3086993984636	-0.228206619756306\\
10.3159145842242	-0.228226489100961\\
10.3231289059065	-0.228246324129625\\
10.330342365178	-0.228266124924589\\
10.3375549637018	-0.228285891567893\\
10.3447667031365	-0.228305624141317\\
10.3519775851363	-0.228325322726391\\
10.3591876113507	-0.228344987404391\\
10.3663967834249	-0.22836461825634\\
10.3736051029995	-0.228384215363012\\
10.3808125717107	-0.228403778804929\\
10.3880191911905	-0.228423308662366\\
10.395224963066	-0.22844280501535\\
10.4024298889604	-0.228462267943657\\
10.4096339704923	-0.228481697526823\\
10.416837209276	-0.228501093844134\\
10.4240396069212	-0.228520456974634\\
10.4312411650337	-0.228539786997123\\
10.4384418852147	-0.228559083990159\\
10.4456417690611	-0.228578348032057\\
10.4528408181656	-0.228597579200894\\
10.4600390341166	-0.228616777574505\\
10.4672364184982	-0.228635943230487\\
10.4744329728904	-0.228655076246201\\
10.4816286988688	-0.228674176698768\\
10.4888235980048	-0.228693244665075\\
10.4960176718658	-0.228712280221774\\
10.5032109220148	-0.228731283445281\\
10.5104033500107	-0.22875025441178\\
10.5175949574082	-0.228769193197223\\
10.524785745758	-0.228788099877328\\
10.5319757166066	-0.228806974527586\\
10.5391648714963	-0.228825817223254\\
10.5463532119655	-0.228844628039363\\
10.5535407395482	-0.228863407050716\\
10.5607274557747	-0.228882154331886\\
10.567913362171	-0.228900869957222\\
10.5750984602592	-0.228919554000847\\
10.5822827515572	-0.228938206536658\\
10.5894662375791	-0.22895682763833\\
10.5966489198348	-0.228975417379313\\
10.6038307998303	-0.228993975832837\\
10.6110118790678	-0.22901250307191\\
10.6181921590453	-0.229030999169317\\
10.6253716412569	-0.229049464197627\\
10.6325503271928	-0.229067898229187\\
10.6397282183393	-0.229086301336129\\
10.6469053161789	-0.229104673590364\\
10.6540816221899	-0.229123015063589\\
10.6612571378471	-0.229141325827287\\
10.6684318646212	-0.229159605952721\\
10.6756058039791	-0.229177855510945\\
10.6827789573838	-0.229196074572797\\
10.6899513262947	-0.229214263208903\\
10.6971229121671	-0.229232421489679\\
10.7042937164527	-0.229250549485327\\
10.7114637405994	-0.229268647265842\\
10.7186329860512	-0.229286714901007\\
10.7258014542484	-0.229304752460399\\
10.7329691466277	-0.229322760013385\\
10.7401360646219	-0.229340737629126\\
10.74730220966	-0.229358685376577\\
10.7544675831676	-0.229376603324487\\
10.7616321865662	-0.229394491541401\\
10.768796021274	-0.229412350095659\\
10.7759590887054	-0.229430179055398\\
10.783121390271	-0.229447978488554\\
10.7902829273778	-0.229465748462858\\
10.7974437014294	-0.229483489045843\\
10.8046037138254	-0.229501200304841\\
10.8117629659622	-0.229518882306984\\
10.8189214592323	-0.229536535119205\\
10.8260791950246	-0.22955415880824\\
10.8332361747247	-0.229571753440627\\
10.8403923997145	-0.229589319082708\\
10.8475478713721	-0.229606855800626\\
10.8547025910725	-0.229624363660334\\
10.8618565601869	-0.229641842727587\\
10.8690097800831	-0.229659293067947\\
10.8761622521252	-0.229676714746782\\
10.8833139776741	-0.22969410782927\\
10.890464958087	-0.229711472380396\\
10.8976151947178	-0.229728808464953\\
10.9047646889168	-0.229746116147546\\
10.9119134420309	-0.229763395492588\\
10.9190614554035	-0.229780646564305\\
10.9262087303749	-0.229797869426734\\
10.9333552682815	-0.229815064143724\\
10.9405010704567	-0.229832230778938\\
10.9476461382302	-0.229849369395853\\
10.9547904729287	-0.22986648005776\\
10.9619340758752	-0.229883562827765\\
10.9690769483895	-0.22990061776879\\
10.976219091788	-0.229917644943574\\
10.9833605073839	-0.229934644414672\\
10.9905011964868	-0.229951616244458\\
10.9976411604032	-0.229968560495124\\
11.0047804004364	-0.229985477228681\\
11.0119189178862	-0.23000236650696\\
11.0190567140491	-0.230019228391613\\
11.0261937902186	-0.230036062944112\\
11.0333301476847	-0.230052870225751\\
11.0404657877342	-0.230069650297647\\
11.0476007116508	-0.230086403220739\\
11.0547349207148	-0.230103129055791\\
11.0618684162033	-0.230119827863389\\
11.0690011993904	-0.230136499703947\\
11.0761332715468	-0.230153144637702\\
11.0832646339401	-0.230169762724717\\
11.0903952878347	-0.230186354024885\\
11.0975252344919	-0.230202918597922\\
11.1046544751698	-0.230219456503374\\
11.1117830111232	-0.230235967800617\\
11.1189108436042	-0.230252452548855\\
11.1260379738613	-0.23026891080712\\
11.1331644031401	-0.230285342634277\\
11.1402901326833	-0.230301748089021\\
11.1474151637302	-0.23031812722988\\
11.1545394975171	-0.230334480115211\\
11.1616631352773	-0.230350806803206\\
11.168786078241	-0.230367107351891\\
11.1759083276353	-0.230383381819125\\
11.1830298846844	-0.230399630262602\\
11.1901507506094	-0.230415852739849\\
11.1972709266283	-0.230432049308233\\
11.2043904139561	-0.230448220024954\\
11.211509213805	-0.230464364947049\\
11.2186273273839	-0.230480484131395\\
11.225744755899	-0.230496577634704\\
11.2328615005534	-0.230512645513528\\
11.2399775625471	-0.230528687824259\\
11.2470929430775	-0.230544704623128\\
11.2542076433387	-0.230560695966205\\
11.2613216645221	-0.230576661909403\\
11.268435007816	-0.230592602508475\\
11.275547674406	-0.230608517819016\\
11.2826596654745	-0.230624407896466\\
11.2897709822012	-0.230640272796104\\
11.296881625763	-0.230656112573056\\
11.3039915973338	-0.23067192728229\\
11.3111008980845	-0.230687716978621\\
11.3182095291833	-0.230703481716706\\
11.3253174917957	-0.230719221551051\\
11.3324247870839	-0.230734936536007\\
11.3395314162078	-0.230750626725772\\
11.346637380324	-0.230766292174391\\
11.3537426805867	-0.230781932935757\\
11.360847318147	-0.230797549063612\\
11.3679512941533	-0.230813140611547\\
11.3750546097513	-0.230828707633002\\
11.3821572660838	-0.230844250181267\\
11.3892592642908	-0.230859768309484\\
11.3963606055096	-0.230875262070643\\
11.4034612908749	-0.230890731517589\\
11.4105613215184	-0.230906176703018\\
11.4176606985691	-0.230921597679477\\
11.4247594231535	-0.230936994499368\\
11.4318574963951	-0.230952367214945\\
11.4389549194148	-0.230967715878319\\
11.4460516933309	-0.230983040541452\\
11.4531478192589	-0.230998341256165\\
11.4602432983116	-0.23101361807413\\
11.4673381315992	-0.231028871046879\\
11.4744323202291	-0.231044100225799\\
11.4815258653063	-0.231059305662135\\
11.4886187679328	-0.231074487406988\\
11.4957110292082	-0.231089645511319\\
11.5028026502294	-0.231104780025946\\
11.5098936320907	-0.231119891001548\\
11.5169839758836	-0.231134978488661\\
11.5240736826974	-0.231150042537683\\
11.5311627536184	-0.231165083198872\\
11.5382511897304	-0.231180100522346\\
11.5453389921148	-0.231195094558087\\
11.5524261618501	-0.231210065355935\\
11.5595127000126	-0.231225012965596\\
11.5665986076758	-0.231239937436638\\
11.5736838859106	-0.231254838818491\\
11.5807685357855	-0.231269717160449\\
11.5878525583664	-0.231284572511673\\
11.5949359547168	-0.231299404921184\\
11.6020187258974	-0.231314214437874\\
11.6091008729667	-0.231329001110494\\
11.6161823969803	-0.231343764987668\\
11.6232632989918	-0.23135850611788\\
11.6303435800519	-0.231373224549487\\
11.637423241209	-0.231387920330708\\
11.644502283509	-0.231402593509635\\
11.6515807079953	-0.231417244134224\\
11.6586585157088	-0.231431872252302\\
11.6657357076881	-0.231446477911566\\
11.6728122849692	-0.23146106115958\\
11.6798882485857	-0.23147562204378\\
11.6869635995688	-0.231490160611474\\
11.6940383389474	-0.231504676909837\\
11.7011124677476	-0.231519170985918\\
11.7081859869935	-0.231533642886639\\
11.7152588977065	-0.231548092658791\\
11.7223312009059	-0.231562520349041\\
11.7294028976084	-0.231576926003927\\
11.7364739888284	-0.231591309669861\\
11.7435444755778	-0.231605671393129\\
11.7506143588663	-0.231620011219892\\
11.7576836397011	-0.231634329196186\\
11.7647523190873	-0.231648625367922\\
11.7718203980274	-0.231662899780884\\
11.7788878775215	-0.231677152480737\\
11.7859547585678	-0.231691383513019\\
11.7930210421617	-0.231705592923146\\
11.8000867292965	-0.23171978075641\\
11.8071518209631	-0.231733947057984\\
11.8142163181504	-0.231748091872916\\
11.8212802218446	-0.231762215246134\\
11.8283435330298	-0.231776317222443\\
11.8354062526878	-0.231790397846531\\
11.8424683817982	-0.231804457162963\\
11.8495299213382	-0.231818495216184\\
11.8565908722829	-0.23183251205052\\
11.8636512356049	-0.23184650771018\\
11.8707110122748	-0.231860482239252\\
11.8777702032609	-0.231874435681706\\
11.8848288095291	-0.231888368081394\\
11.8918868320433	-0.231902279482051\\
11.8989442717651	-0.231916169927296\\
11.9060011296539	-0.231930039460629\\
11.9130574066667	-0.231943888125435\\
11.9201131037586	-0.231957715964984\\
11.9271682218824	-0.231971523022427\\
11.9342227619885	-0.231985309340804\\
11.9412767250256	-0.231999074963038\\
11.9483301119396	-0.232012819931937\\
11.9553829236748	-0.232026544290196\\
11.9624351611731	-0.232040248080397\\
11.9694868253741	-0.232053931345007\\
11.9765379172155	-0.23206759412638\\
11.9835884376327	-0.232081236466761\\
11.9906383875591	-0.232094858408277\\
11.9976877679258	-0.232108459992949\\
12.004736579662	-0.232122041262682\\
12.0117848236945	-0.232135602259272\\
12.0188325009483	-0.232149143024405\\
12.025879612346	-0.232162663599654\\
12.0329261588083	-0.232176164026484\\
12.0399721412538	-0.232189644346251\\
12.0470175605989	-0.2322031046002\\
12.054062417758	-0.232216544829468\\
12.0611067136434	-0.232229965075082\\
12.0681504491654	-0.232243365377963\\
12.0751936252322	-0.232256745778923\\
12.0822362427498	-0.232270106318665\\
12.0892783026224	-0.232283447037788\\
12.096319805752	-0.232296767976782\\
12.1033607530387	-0.23231006917603\\
12.1104011453803	-0.232323350675811\\
12.1174409836728	-0.232336612516296\\
12.1244802688102	-0.232349854737552\\
12.1315190016844	-0.23236307737954\\
12.1385571831852	-0.232376280482117\\
12.1455948142006	-0.232389464085036\\
12.1526318956166	-0.232402628227944\\
12.1596684283169	-0.232415772950385\\
12.1667044131836	-0.232428898291801\\
12.1737398510966	-0.232442004291529\\
12.180774742934	-0.232455090988805\\
12.1878090895716	-0.23246815842276\\
12.1948428918836	-0.232481206632426\\
12.2018761507422	-0.232494235656731\\
12.2089088670174	-0.232507245534503\\
12.2159410415774	-0.232520236304467\\
12.2229726752886	-0.232533208005249\\
12.2300037690152	-0.232546160675373\\
12.2370343236197	-0.232559094353265\\
12.2440643399627	-0.232572009077249\\
12.2510938189025	-0.232584904885551\\
12.2581227612961	-0.232597781816296\\
12.265151167998	-0.232610639907512\\
12.2721790398611	-0.232623479197129\\
12.2792063777365	-0.232636299722975\\
12.2862331824732	-0.232649101522784\\
12.2932594549183	-0.23266188463419\\
12.3002851959174	-0.232674649094732\\
12.3073104063136	-0.232687394941849\\
12.3143350869488	-0.232700122212885\\
12.3213592386625	-0.232712830945089\\
12.3283828622926	-0.232725521175611\\
12.3354059586753	-0.232738192941507\\
12.3424285286445	-0.232750846279738\\
12.3494505730327	-0.232763481227167\\
12.3564720926703	-0.232776097820566\\
12.3634930883861	-0.23278869609661\\
12.3705135610068	-0.23280127609188\\
12.3775335113575	-0.232813837842863\\
12.3845529402613	-0.232826381385954\\
12.3915718485398	-0.232838906757451\\
12.3985902370124	-0.232851413993564\\
12.405608106497	-0.232863903130405\\
12.4126254578096	-0.232876374203997\\
12.4196422917644	-0.232888827250269\\
12.4266586091738	-0.23290126230506\\
12.4336744108484	-0.232913679404116\\
12.4406896975973	-0.23292607858309\\
12.4477044702274	-0.232938459877548\\
12.4547187295441	-0.232950823322961\\
12.4617324763511	-0.232963168954714\\
12.4687457114501	-0.232975496808097\\
12.4757584356413	-0.232987806918314\\
12.482770649723	-0.233000099320477\\
12.4897823544917	-0.233012374049609\\
12.4967935507425	-0.233024631140646\\
12.5038042392684	-0.233036870628432\\
12.5108144208608	-0.233049092547726\\
12.5178240963095	-0.233061296933194\\
12.5248332664025	-0.23307348381942\\
12.531841931926	-0.233085653240895\\
12.5388500936647	-0.233097805232026\\
12.5458577524013	-0.233109939827131\\
12.5528649089171	-0.233122057060442\\
12.5598715639916	-0.233134156966105\\
12.5668777184026	-0.233146239578177\\
12.5738833729263	-0.233158304930633\\
12.580888528337	-0.233170353057359\\
12.5878931854075	-0.233182383992157\\
12.5948973449091	-0.233194397768742\\
12.6019010076111	-0.233206394420747\\
12.6089041742813	-0.233218373981717\\
12.6159068456859	-0.233230336485115\\
12.6229090225894	-0.233242281964319\\
12.6299107057546	-0.233254210452622\\
12.6369118959428	-0.233266121983236\\
12.6439125939135	-0.233278016589286\\
12.6509128004248	-0.233289894303816\\
12.6579125162328	-0.233301755159788\\
12.6649117420924	-0.233313599190078\\
12.6719104787565	-0.233325426427484\\
12.6789087269768	-0.233337236904718\\
12.6859064875031	-0.233349030654412\\
12.6929037610836	-0.233360807709116\\
12.699900548465	-0.233372568101299\\
12.7068968503924	-0.233384311863348\\
12.7138926676094	-0.23339603902757\\
12.7208880008577	-0.233407749626191\\
12.7278828508777	-0.233419443691356\\
12.7348772184082	-0.233431121255131\\
12.7418711041864	-0.233442782349501\\
12.7488645089478	-0.233454427006373\\
12.7558574334265	-0.233466055257571\\
12.7628498783551	-0.233477667134845\\
12.7698418444643	-0.233489262669862\\
12.7768333324837	-0.233500841894212\\
12.783824343141	-0.233512404839407\\
12.7908148771625	-0.233523951536878\\
12.7978049352731	-0.233535482017981\\
12.8047945181959	-0.233546996313994\\
12.8117836266526	-0.233558494456117\\
12.8187722613634	-0.233569976475471\\
12.8257604230469	-0.233581442403103\\
12.8327481124203	-0.233592892269982\\
12.8397353301992	-0.233604326106999\\
12.8467220770978	-0.23361574394497\\
12.8537083538285	-0.233627145814636\\
12.8606941611026	-0.233638531746659\\
12.8676794996296	-0.23364990177163\\
12.8746643701177	-0.23366125592006\\
12.8816487732735	-0.233672594222387\\
12.8886327098022	-0.233683916708974\\
12.8956161804073	-0.233695223410109\\
12.9025991857912	-0.233706514356007\\
12.9095817266545	-0.233717789576805\\
12.9165638036965	-0.23372904910257\\
12.9235454176149	-0.233740292963293\\
12.9305265691062	-0.233751521188893\\
12.9375072588651	-0.233762733809212\\
12.9444874875852	-0.233773930854024\\
12.9514672559584	-0.233785112353026\\
12.9584465646751	-0.233796278335844\\
12.9654254144246	-0.233807428832031\\
12.9724038058945	-0.233818563871069\\
12.9793817397709	-0.233829683482367\\
12.9863592167387	-0.233840787695261\\
12.9933362374813	-0.233851876539017\\
13.0003128026805	-0.233862950042829\\
13.007288913017	-0.23387400823582\\
13.0142645691699	-0.233885051147042\\
13.0212397718168	-0.233896078805476\\
13.0282145216341	-0.233907091240032\\
13.0351888192966	-0.233918088479552\\
13.0421626654779	-0.233929070552804\\
13.0491360608501	-0.23394003748849\\
13.0561090060839	-0.233950989315239\\
13.0630815018486	-0.233961926061613\\
13.0700535488123	-0.233972847756104\\
13.0770251476413	-0.233983754427133\\
13.0839962990011	-0.233994646103055\\
13.0909670035553	-0.234005522812154\\
13.0979372619664	-0.234016384582647\\
13.1049070748955	-0.234027231442683\\
13.1118764430024	-0.234038063420341\\
13.1188453669454	-0.234048880543635\\
13.1258138473815	-0.234059682840507\\
13.1327818849665	-0.234070470338837\\
13.1397494803545	-0.234081243066433\\
13.1467166341986	-0.234092001051039\\
13.1536833471505	-0.23410274432033\\
13.1606496198604	-0.234113472901916\\
13.1676154529773	-0.234124186823339\\
13.1745808471489	-0.234134886112077\\
13.1815458030214	-0.234145570795539\\
13.1885103212399	-0.234156240901071\\
13.195474402448	-0.234166896455951\\
13.2024380472881	-0.234177537487393\\
13.2094012564013	-0.234188164022545\\
13.2163640304272	-0.234198776088489\\
13.2233263700043	-0.234209373712244\\
13.2302882757698	-0.234219956920764\\
13.2372497483595	-0.234230525740936\\
13.2442107884079	-0.234241080199587\\
13.2511713965483	-0.234251620323475\\
13.2581315734127	-0.234262146139297\\
13.2650913196316	-0.234272657673687\\
13.2720506358347	-0.234283154953212\\
13.2790095226498	-0.234293638004378\\
13.285967980704	-0.234304106853629\\
13.2929260106228	-0.234314561527342\\
13.2998836130305	-0.234325002051835\\
13.3068407885502	-0.234335428453362\\
13.3137975378035	-0.234345840758113\\
13.3207538614112	-0.234356238992218\\
13.3277097599924	-0.234366623181743\\
13.3346652341651	-0.234376993352694\\
13.3416202845463	-0.234387349531014\\
13.3485749117513	-0.234397691742584\\
13.3555291163945	-0.234408020013225\\
13.3624828990889	-0.234418334368696\\
13.3694362604464	-0.234428634834694\\
13.3763892010775	-0.234438921436857\\
13.3833417215916	-0.234449194200761\\
13.3902938225969	-0.234459453151922\\
13.3972455047003	-0.234469698315796\\
13.4041967685074	-0.234479929717778\\
13.4111476146227	-0.234490147383204\\
13.4180980436496	-0.234500351337348\\
13.42504805619	-0.234510541605428\\
13.4319976528449	-0.2345207182126\\
13.4389468342139	-0.23453088118396\\
13.4458956008953	-0.234541030544548\\
13.4528439534866	-0.234551166319341\\
13.4597918925838	-0.234561288533261\\
13.4667394187816	-0.23457139721117\\
13.4736865326739	-0.23458149237787\\
13.480633234853	-0.234591574058108\\
13.4875795259104	-0.234601642276569\\
13.4945254064362	-0.234611697057882\\
13.5014708770193	-0.23462173842662\\
13.5084159382476	-0.234631766407296\\
13.5153605907075	-0.234641781024366\\
13.5223048349848	-0.234651782302228\\
13.5292486716635	-0.234661770265226\\
13.5361921013269	-0.234671744937643\\
13.5431351245569	-0.234681706343708\\
13.5500777419343	-0.234691654507592\\
13.5570199540389	-0.234701589453411\\
13.5639617614491	-0.234711511205224\\
13.5709031647424	-0.234721419787032\\
13.5778441644949	-0.234731315222783\\
13.5847847612818	-0.234741197536367\\
13.591724955677	-0.23475106675162\\
13.5986647482534	-0.234760922892321\\
13.6056041395826	-0.234770765982195\\
13.6125431302353	-0.234780596044912\\
13.6194817207808	-0.234790413104084\\
13.6264199117876	-0.234800217183272\\
13.6333577038227	-0.234810008305981\\
13.6402950974523	-0.23481978649566\\
13.6472320932414	-0.234829551775705\\
13.6541686917538	-0.234839304169457\\
13.6611048935522	-0.234849043700205\\
13.6680406991984	-0.234858770391181\\
13.6749761092529	-0.234868484265566\\
13.6819111242751	-0.234878185346485\\
13.6888457448234	-0.23488787365701\\
13.695779971455	-0.234897549220162\\
13.7027138047261	-0.234907212058906\\
13.7096472451918	-0.234916862196155\\
13.716580293406	-0.23492649965477\\
13.7235129499218	-0.234936124457559\\
13.7304452152908	-0.234945736627275\\
13.7373770900639	-0.234955336186622\\
13.7443085747908	-0.23496492315825\\
13.75123967002	-0.234974497564757\\
13.7581703762991	-0.23498405942869\\
13.7651006941745	-0.234993608772542\\
13.7720306241917	-0.235003145618756\\
13.778960166895	-0.235012669989724\\
13.7858893228277	-0.235022181907785\\
13.7928180925321	-0.235031681395227\\
13.7997464765492	-0.235041168474289\\
13.8066744754192	-0.235050643167156\\
13.8136020896813	-0.235060105495964\\
13.8205293198734	-0.235069555482798\\
13.8274561665325	-0.235078993149693\\
13.8343826301946	-0.235088418518632\\
13.8413087113945	-0.235097831611549\\
13.8482344106663	-0.235107232450328\\
13.8551597285427	-0.235116621056803\\
13.8620846655554	-0.235125997452757\\
13.8690092222355	-0.235135361659925\\
13.8759333991125	-0.235144713699992\\
13.8828571967152	-0.235154053594592\\
13.8897806155714	-0.235163381365311\\
13.8967036562077	-0.235172697033687\\
13.9036263191499	-0.235182000621207\\
13.9105486049226	-0.23519129214931\\
13.9174705140495	-0.235200571639387\\
13.9243920470532	-0.235209839112779\\
13.9313132044555	-0.23521909459078\\
13.9382339867769	-0.235228338094633\\
13.9451543945372	-0.235237569645537\\
13.9520744282549	-0.23524678926464\\
13.9589940884479	-0.235255996973041\\
13.9659133756326	-0.235265192791796\\
13.9728322903249	-0.235274376741908\\
13.9797508330393	-0.235283548844336\\
13.9866690042897	-0.235292709119989\\
13.9935868045888	-0.235301857589732\\
14.0005042344482	-0.235310994274381\\
14.0074212943789	-0.235320119194703\\
14.0143379848905	-0.235329232371423\\
14.0212543064919	-0.235338333825214\\
14.028170259691	-0.235347423576706\\
14.0350858449946	-0.235356501646482\\
14.0420010629087	-0.235365568055077\\
14.0489159139382	-0.235374622822981\\
14.0558303985872	-0.235383665970638\\
14.0627445173586	-0.235392697518446\\
14.0696582707545	-0.235401717486757\\
14.0765716592762	-0.235410725895876\\
14.0834846834237	-0.235419722766066\\
14.0903973436962	-0.23542870811754\\
14.0973096405922	-0.235437681970469\\
14.1042215746089	-0.235446644344977\\
14.1111331462427	-0.235455595261144\\
14.118044355989	-0.235464534739005\\
14.1249552043425	-0.235473462798549\\
14.1318656917966	-0.235482379459722\\
14.1387758188441	-0.235491284742424\\
14.1456855859766	-0.23550017866651\\
14.1525949936849	-0.235509061251793\\
14.159504042459	-0.235517932518041\\
14.1664127327877	-0.235526792484975\\
14.1733210651591	-0.235535641172277\\
14.1802290400603	-0.23554447859958\\
14.1871366579774	-0.235553304786478\\
14.1940439193958	-0.235562119752518\\
14.2009508247998	-0.235570923517204\\
14.2078573746729	-0.235579716099999\\
14.2147635694975	-0.235588497520319\\
14.2216694097553	-0.23559726779754\\
14.2285748959271	-0.235606026950994\\
14.2354800284927	-0.235614774999969\\
14.242384807931	-0.235623511963711\\
14.24928923472	-0.235632237861424\\
14.2561933093369	-0.235640952712269\\
14.2630970322579	-0.235649656535364\\
14.2700004039584	-0.235658349349785\\
14.2769034249129	-0.235667031174566\\
14.2838060955949	-0.235675702028699\\
14.2907084164771	-0.235684361931133\\
14.2976103880313	-0.235693010900776\\
14.3045120107285	-0.235701648956494\\
14.3114132850387	-0.235710276117112\\
14.3183142114311	-0.235718892401413\\
14.325214790374	-0.235727497828137\\
14.3321150223348	-0.235736092415986\\
14.3390149077801	-0.235744676183618\\
14.3459144471756	-0.235753249149651\\
14.3528136409862	-0.235761811332662\\
14.3597124896757	-0.235770362751186\\
14.3666109937073	-0.23577890342372\\
14.3735091535433	-0.235787433368718\\
14.380406969645	-0.235795952604593\\
14.387304442473	-0.23580446114972\\
14.3942015724869	-0.235812959022431\\
14.4010983601456	-0.23582144624102\\
14.4079948059071	-0.23582992282374\\
14.4148909102284	-0.235838388788804\\
14.421786673566	-0.235846844154384\\
14.4286820963752	-0.235855288938615\\
14.4355771791106	-0.23586372315959\\
14.4424719222261	-0.235872146835363\\
14.4493663261745	-0.235880559983948\\
14.4562603914079	-0.235888962623321\\
14.4631541183776	-0.235897354771417\\
14.4700475075341	-0.235905736446135\\
14.4769405593269	-0.23591410766533\\
14.4838332742049	-0.235922468446823\\
14.4907256526158	-0.235930818808393\\
14.497617695007	-0.235939158767782\\
14.5045094018247	-0.235947488342692\\
14.5114007735144	-0.235955807550788\\
14.5182918105207	-0.235964116409695\\
14.5251825132876	-0.235972414937\\
14.532072882258	-0.235980703150253\\
14.5389629178742	-0.235988981066965\\
14.5458526205776	-0.235997248704609\\
14.5527419908088	-0.236005506080619\\
14.5596310290077	-0.236013753212394\\
14.5665197356132	-0.236021990117294\\
14.5734081110635	-0.236030216812639\\
14.5802961557961	-0.236038433315714\\
14.5871838702476	-0.236046639643767\\
14.5940712548536	-0.236054835814008\\
14.6009583100494	-0.236063021843608\\
14.607845036269	-0.236071197749704\\
14.614731433946	-0.236079363549393\\
14.6216175035129	-0.236087519259738\\
14.6285032454016	-0.236095664897763\\
14.6353886600432	-0.236103800480455\\
14.642273747868	-0.236111926024767\\
14.6491585093054	-0.236120041547613\\
14.6560429447843	-0.236128147065871\\
14.6629270547325	-0.236136242596384\\
14.6698108395772	-0.236144328155957\\
14.6766942997448	-0.23615240376136\\
14.683577435661	-0.236160469429326\\
14.6904602477505	-0.236168525176554\\
14.6973427364376	-0.236176571019704\\
14.7042249021454	-0.236184606975403\\
14.7111067452966	-0.236192633060241\\
14.7179882663129	-0.236200649290774\\
14.7248694656154	-0.236208655683519\\
14.7317503436244	-0.236216652254961\\
14.7386309007593	-0.23622463902155\\
14.7455111374389	-0.236232615999697\\
14.7523910540812	-0.236240583205781\\
14.7592706511034	-0.236248540656146\\
14.7661499289221	-0.2362564883671\\
14.773028887953	-0.236264426354916\\
14.7799075286111	-0.236272354635832\\
14.7867858513106	-0.236280273226054\\
14.7936638564651	-0.23628818214175\\
14.8005415444873	-0.236296081399056\\
14.8074189157893	-0.236303971014071\\
14.8142959707823	-0.236311851002863\\
14.8211727098769	-0.236319721381464\\
14.8280491334828	-0.236327582165871\\
14.8349252420093	-0.236335433372048\\
14.8418010358646	-0.236343275015926\\
14.8486765154563	-0.2363511071134\\
14.8555516811914	-0.236358929680333\\
14.8624265334759	-0.236366742732554\\
14.8693010727154	-0.236374546285858\\
14.8761752993146	-0.236382340356007\\
14.8830492136774	-0.236390124958728\\
14.8899228162071	-0.236397900109716\\
14.8967961073062	-0.236405665824634\\
14.9036690873767	-0.236413422119109\\
14.9105417568196	-0.236421169008738\\
14.9174141160352	-0.236428906509082\\
14.9242861654234	-0.236436634635671\\
14.9311579053831	-0.236444353404002\\
14.9380293363126	-0.236452062829539\\
14.9449004586094	-0.236459762927713\\
14.9517712726704	-0.236467453713923\\
14.9586417788917	-0.236475135203535\\
14.9655119776689	-0.236482807411883\\
14.9723818693966	-0.236490470354269\\
14.9792514544689	-0.236498124045963\\
14.9861207332792	-0.2365057685022\\
14.9929897062201	-0.236513403738187\\
14.9998583736836	-0.236521029769097\\
15.0067267360609	-0.236528646610072\\
15.0135947937427	-0.236536254276219\\
15.0204625471189	-0.236543852782619\\
15.0273299965786	-0.236551442144315\\
15.0341971425103	-0.236559022376324\\
15.041063985302	-0.236566593493628\\
15.0479305253406	-0.236574155511179\\
15.0547967630128	-0.236581708443897\\
15.0616626987043	-0.236589252306671\\
15.0685283328002	-0.23659678711436\\
15.0753936656848	-0.236604312881789\\
15.0822586977421	-0.236611829623756\\
15.089123429355	-0.236619337355024\\
15.095987860906	-0.236626836090328\\
15.1028519927768	-0.236634325844372\\
15.1097158253484	-0.236641806631827\\
15.1165793590013	-0.236649278467336\\
15.1234425941152	-0.23665674136551\\
15.1303055310692	-0.23666419534093\\
15.1371681702416	-0.236671640408148\\
15.1440305120102	-0.236679076581683\\
15.1508925567521	-0.236686503876025\\
15.1577543048437	-0.236693922305634\\
15.1646157566607	-0.236701331884941\\
15.1714769125784	-0.236708732628345\\
15.178337772971	-0.236716124550216\\
15.1851983382124	-0.236723507664895\\
15.1920586086758	-0.236730881986691\\
15.1989185847337	-0.236738247529886\\
15.2057782667578	-0.23674560430873\\
15.2126376551195	-0.236752952337446\\
15.2194967501892	-0.236760291630224\\
15.2263555523369	-0.236767622201229\\
15.2332140619318	-0.236774944064592\\
15.2400722793425	-0.23678225723442\\
15.2469302049371	-0.236789561724785\\
15.2537878390828	-0.236796857549735\\
15.2606451821465	-0.236804144723286\\
15.2675022344941	-0.236811423259427\\
15.2743589964911	-0.236818693172115\\
15.2812154685022	-0.236825954475282\\
15.2880716508918	-0.236833207182829\\
15.2949275440232	-0.236840451308629\\
15.3017831482595	-0.236847686866526\\
15.3086384639629	-0.236854913870336\\
15.3154934914951	-0.236862132333847\\
15.322348231217	-0.236869342270816\\
15.3292026834892	-0.236876543694976\\
15.3360568486714	-0.236883736620028\\
15.3429107271227	-0.236890921059647\\
15.3497643192017	-0.236898097027479\\
15.3566176252664	-0.236905264537143\\
15.3634706456741	-0.236912423602228\\
15.3703233807814	-0.236919574236297\\
15.3771758309445	-0.236926716452884\\
15.3840279965188	-0.236933850265497\\
15.3908798778592	-0.236940975687615\\
15.3977314753199	-0.236948092732689\\
15.4045827892547	-0.236955201414144\\
15.4114338200164	-0.236962301745376\\
15.4182845679576	-0.236969393739754\\
15.4251350334301	-0.23697647741062\\
15.431985216785	-0.23698355277129\\
15.4388351183731	-0.23699061983505\\
15.4456847385444	-0.236997678615161\\
15.4525340776483	-0.237004729124857\\
15.4593831360335	-0.237011771377343\\
15.4662319140484	-0.2370188053858\\
15.4730804120406	-0.237025831163379\\
15.479928630357	-0.237032848723208\\
15.4867765693443	-0.237039858078384\\
15.4936242293481	-0.23704685924198\\
15.5004716107139	-0.237053852227043\\
15.5073187137862	-0.237060837046592\\
15.5141655389091	-0.237067813713619\\
15.5210120864263	-0.237074782241092\\
15.5278583566805	-0.237081742641951\\
15.5347043500141	-0.237088694929109\\
15.541550066769	-0.237095639115454\\
15.5483955072861	-0.237102575213849\\
15.5552406719063	-0.237109503237129\\
15.5620855609694	-0.237116423198103\\
15.5689301748149	-0.237123335109556\\
15.5757745137817	-0.237130238984245\\
15.5826185782081	-0.237137134834903\\
15.5894623684317	-0.237144022674235\\
15.5963058847898	-0.237150902514923\\
15.6031491276189	-0.237157774369622\\
15.609992097255	-0.237164638250961\\
15.6168347940335	-0.237171494171543\\
15.6236772182893	-0.237178342143949\\
15.6305193703568	-0.237185182180731\\
15.6373612505696	-0.237192014294418\\
15.644202859261	-0.237198838497511\\
15.6510441967635	-0.237205654802489\\
15.6578852634092	-0.237212463221805\\
15.6647260595296	-0.237219263767885\\
15.6715665854556	-0.237226056453132\\
15.6784068415176	-0.237232841289924\\
15.6852468280454	-0.237239618290613\\
15.6920865453683	-0.237246387467527\\
15.6989259938151	-0.237253148832971\\
15.7057651737137	-0.237259902399221\\
15.712604085392	-0.237266648178532\\
15.7194427291769	-0.237273386183133\\
15.7262811053949	-0.23728011642523\\
15.733119214372	-0.237286838917002\\
15.7399570564337	-0.237293553670606\\
15.7467946319047	-0.237300260698172\\
15.7536319411094	-0.23730696001181\\
15.7604689843717	-0.237313651623601\\
15.7673057620146	-0.237320335545606\\
15.774142274361	-0.237327011789859\\
15.780978521733	-0.23733368036837\\
15.7878145044522	-0.237340341293128\\
15.7946502228398	-0.237346994576095\\
15.8014856772161	-0.237353640229211\\
15.8083208679014	-0.237360278264391\\
15.815155795215	-0.237366908693527\\
15.8219904594759	-0.237373531528487\\
15.8288248610025	-0.237380146781116\\
15.8356590001127	-0.237386754463234\\
15.8424928771238	-0.237393354586639\\
15.8493264923528	-0.237399947163105\\
15.8561598461158	-0.237406532204382\\
15.8629929387287	-0.237413109722199\\
15.8698257705067	-0.237419679728258\\
15.8766583417646	-0.237426242234241\\
15.8834906528165	-0.237432797251806\\
15.8903227039761	-0.237439344792586\\
15.8971544955567	-0.237445884868194\\
15.9039860278707	-0.237452417490218\\
15.9108173012305	-0.237458942670223\\
15.9176483159475	-0.237465460419753\\
15.9244790723329	-0.237471970750326\\
15.9313095706972	-0.237478473673441\\
15.9381398113506	-0.237484969200571\\
15.9449697946025	-0.237491457343168\\
15.9517995207621	-0.237497938112662\\
15.9586289901378	-0.237504411520459\\
15.9654582030377	-0.237510877577943\\
15.9722871597692	-0.237517336296476\\
15.9791158606394	-0.237523787687396\\
15.9859443059549	-0.237530231762022\\
15.9927724960215	-0.237536668531646\\
15.9996004311448	-0.237543098007543\\
16.0064281116297	-0.237549520200962\\
16.0132555377808	-0.237555935123131\\
16.0200827099021	-0.237562342785256\\
16.026909628297	-0.237568743198522\\
16.0337362932685	-0.23757513637409\\
16.0405627051191	-0.2375815223231\\
16.0473888641508	-0.237587901056671\\
16.0542147706651	-0.2375942725859\\
16.061040424963	-0.237600636921861\\
16.0678658273451	-0.237606994075607\\
16.0746909781113	-0.237613344058169\\
16.0815158775611	-0.237619686880558\\
16.0883405259937	-0.237626022553761\\
16.0951649237075	-0.237632351088746\\
16.1019890710006	-0.237638672496458\\
16.1088129681706	-0.23764498678782\\
16.1156366155146	-0.237651293973736\\
16.1224600133292	-0.237657594065086\\
16.1292831619104	-0.237663887072731\\
16.136106061554	-0.23767017300751\\
16.1429287125551	-0.23767645188024\\
16.1497511152083	-0.237682723701718\\
16.1565732698078	-0.23768898848272\\
16.1633951766475	-0.237695246234\\
16.1702168360204	-0.237701496966293\\
16.1770382482193	-0.23770774069031\\
16.1838594135366	-0.237713977416744\\
16.190680332264	-0.237720207156267\\
16.1975010046929	-0.237726429919528\\
16.2043214311141	-0.237732645717158\\
16.2111416118181	-0.237738854559765\\
16.2179615470946	-0.237745056457938\\
16.2247812372332	-0.237751251422246\\
16.2316006825229	-0.237757439463235\\
16.2384198832522	-0.237763620591434\\
16.245238839709	-0.237769794817348\\
16.2520575521811	-0.237775962151463\\
16.2588760209554	-0.237782122604247\\
16.2656942463186	-0.237788276186145\\
16.272512228557	-0.237794422907582\\
16.2793299679562	-0.237800562778963\\
16.2861474648015	-0.237806695810675\\
16.2929647193777	-0.237812822013083\\
16.2997817319691	-0.237818941396531\\
16.3065985028597	-0.237825053971345\\
16.3134150323327	-0.23783115974783\\
16.3202313206713	-0.237837258736273\\
16.3270473681579	-0.237843350946938\\
16.3338631750745	-0.237849436390071\\
16.3406787417028	-0.237855515075899\\
16.3474940683239	-0.237861587014629\\
16.3543091552186	-0.237867652216446\\
16.3611240026669	-0.237873710691519\\
16.3679386109489	-0.237879762449994\\
16.3747529803437	-0.237885807502001\\
16.3815671111304	-0.237891845857648\\
16.3883810035872	-0.237897877527024\\
16.3951946579924	-0.2379039025202\\
16.4020080746233	-0.237909920847226\\
16.4088212537572	-0.237915932518133\\
16.4156341956706	-0.237921937542933\\
16.4224469006399	-0.237927935931621\\
16.4292593689408	-0.237933927694168\\
16.4360716008487	-0.237939912840531\\
16.4428835966384	-0.237945891380645\\
16.4496953565845	-0.237951863324426\\
16.4565068809609	-0.237957828681773\\
16.4633181700412	-0.237963787462564\\
16.4701292240987	-0.237969739676658\\
16.476940043406	-0.237975685333899\\
16.4837506282354	-0.237981624444106\\
16.4905609788587	-0.237987557017085\\
16.4973710955474	-0.237993483062621\\
16.5041809785725	-0.237999402590478\\
16.5109906282045	-0.238005315610406\\
16.5178000447135	-0.238011222132132\\
16.5246092283692	-0.238017122165369\\
16.531418179441	-0.238023015719807\\
16.5382268981976	-0.23802890280512\\
16.5450353849074	-0.238034783430965\\
16.5518436398385	-0.238040657606976\\
16.5586516632584	-0.238046525342774\\
16.5654594554342	-0.238052386647959\\
16.5722670166327	-0.238058241532112\\
16.5790743471201	-0.238064090004798\\
16.5858814471624	-0.238069932075561\\
16.5926883170248	-0.238075767753931\\
16.5994949569726	-0.238081597049417\\
16.6063013672703	-0.23808741997151\\
16.6131075481821	-0.238093236529684\\
16.6199134999718	-0.238099046733394\\
16.6267192229027	-0.238104850592079\\
16.6335247172377	-0.238110648115158\\
16.6403299832395	-0.238116439312034\\
16.6471350211701	-0.23812222419209\\
16.6539398312912	-0.238128002764695\\
16.6607444138642	-0.238133775039196\\
16.6675487691498	-0.238139541024925\\
16.6743528974086	-0.238145300731196\\
16.6811567989005	-0.238151054167306\\
16.6879604738854	-0.238156801342531\\
16.6947639226224	-0.238162542266135\\
16.7015671453703	-0.23816827694736\\
16.7083701423875	-0.238174005395433\\
16.7151729139322	-0.238179727619564\\
16.7219754602619	-0.238185443628942\\
16.7287777816337	-0.238191153432744\\
16.7355798783047	-0.238196857040126\\
16.742381750531	-0.238202554460229\\
16.7491833985688	-0.238208245702174\\
16.7559848226736	-0.238213930775067\\
16.7627860231007	-0.238219609687998\\
16.7695870001049	-0.238225282450038\\
16.7763877539405	-0.23823094907024\\
16.7831882848616	-0.238236609557644\\
16.7899885931218	-0.238242263921269\\
16.7967886789743	-0.238247912170119\\
16.8035885426719	-0.238253554313181\\
16.8103881844671	-0.238259190359425\\
16.8171876046118	-0.238264820317805\\
16.8239868033579	-0.238270444197257\\
16.8307857809564	-0.238276062006701\\
16.8375845376582	-0.23828167375504\\
16.8443830737138	-0.23828727945116\\
16.8511813893734	-0.238292879103932\\
16.8579794848865	-0.23829847272221\\
16.8647773605025	-0.238304060314829\\
16.8715750164703	-0.238309641890612\\
16.8783724530384	-0.238315217458361\\
16.885169670455	-0.238320787026864\\
16.8919666689678	-0.238326350604893\\
16.8987634488242	-0.238331908201203\\
16.9055600102712	-0.238337459824532\\
16.9123563535553	-0.238343005483602\\
16.9191524789229	-0.238348545187121\\
16.9259483866198	-0.238354078943778\\
16.9327440768913	-0.238359606762247\\
16.9395395499827	-0.238365128651186\\
16.9463348061386	-0.238370644619237\\
16.9531298456033	-0.238376154675025\\
16.9599246686207	-0.23838165882716\\
16.9667192754346	-0.238387157084236\\
16.973513666288	-0.238392649454831\\
16.9803078414237	-0.238398135947506\\
16.9871018010843	-0.238403616570809\\
16.9938955455117	-0.238409091333269\\
17.0006890749477	-0.2384145602434\\
17.0074823896337	-0.238420023309703\\
17.0142754898105	-0.238425480540659\\
17.0210683757188	-0.238430931944736\\
17.0278610475987	-0.238436377530387\\
17.0346535056902	-0.238441817306046\\
17.0414457502326	-0.238447251280136\\
17.0482377814652	-0.238452679461061\\
17.0550295996266	-0.23845810185721\\
17.0618212049552	-0.238463518476958\\
17.068612597689	-0.238468929328664\\
17.0754037780657	-0.23847433442067\\
17.0821947463226	-0.238479733761306\\
17.0889855026965	-0.238485127358883\\
17.0957760474241	-0.238490515221699\\
17.1025663807414	-0.238495897358036\\
17.1093565028844	-0.238501273776162\\
17.1161464140886	-0.238506644484327\\
17.1229361145889	-0.238512009490769\\
17.1297256046203	-0.238517368803708\\
17.136514884417	-0.238522722431352\\
17.1433039542131	-0.238528070381892\\
17.1500928142423	-0.238533412663504\\
17.1568814647379	-0.238538749284349\\
17.1636699059328	-0.238544080252574\\
17.1704581380598	-0.23854940557631\\
17.1772461613509	-0.238554725263674\\
17.1840339760382	-0.238560039322767\\
17.1908215823532	-0.238565347761678\\
17.197608980527	-0.238570650588477\\
17.2043961707906	-0.238575947811222\\
17.2111831533744	-0.238581239437957\\
17.2179699285086	-0.238586525476708\\
17.2247564964229	-0.23859180593549\\
17.2315428573468	-0.238597080822301\\
17.2383290115094	-0.238602350145125\\
17.2451149591394	-0.238607613911933\\
17.2519007004653	-0.238612872130678\\
17.2586862357151	-0.238618124809302\\
17.2654715651165	-0.238623371955731\\
17.2722566888969	-0.238628613577877\\
17.2790416072832	-0.238633849683636\\
17.2858263205023	-0.238639080280893\\
17.2926108287804	-0.238644305377515\\
17.2993951323434	-0.238649524981357\\
17.3061792314172	-0.238654739100259\\
17.3129631262269	-0.238659947742047\\
17.3197468169975	-0.238665150914533\\
17.3265303039537	-0.238670348625514\\
17.3333135873198	-0.238675540882774\\
17.3400966673196	-0.238680727694082\\
17.3468795441769	-0.238685909067192\\
17.353662218115	-0.238691085009847\\
17.3604446893566	-0.238696255529773\\
17.3672269581246	-0.238701420634683\\
17.374009024641	-0.238706580332277\\
17.380790889128	-0.23871173463024\\
17.387572551807	-0.238716883536243\\
17.3943540128993	-0.238722027057944\\
17.401135272626	-0.238727165202987\\
17.4079163312075	-0.238732297979001\\
17.4146971888642	-0.238737425393602\\
17.421477845816	-0.238742547454393\\
17.4282583022825	-0.238747664168963\\
17.4350385584831	-0.238752775544887\\
17.4418186146366	-0.238757881589725\\
17.4485984709617	-0.238762982311027\\
17.4553781276767	-0.238768077716325\\
17.4621575849996	-0.23877316781314\\
17.468936843148	-0.238778252608981\\
17.4757159023393	-0.238783332111339\\
17.4824947627905	-0.238788406327695\\
17.4892734247181	-0.238793475265517\\
17.4960518883387	-0.238798538932256\\
17.5028301538681	-0.238803597335354\\
17.5096082215222	-0.238808650482236\\
17.5163860915162	-0.238813698380316\\
17.5231637640654	-0.238818741036993\\
17.5299412393843	-0.238823778459655\\
17.5367185176874	-0.238828810655674\\
17.5434955991889	-0.238833837632412\\
17.5502724841025	-0.238838859397214\\
17.5570491726417	-0.238843875957415\\
17.5638256650196	-0.238848887320336\\
17.5706019614491	-0.238853893493283\\
17.5773780621427	-0.238858894483553\\
17.5841539673125	-0.238863890298426\\
17.5909296771705	-0.238868880945171\\
17.5977051919283	-0.238873866431044\\
17.604480511797	-0.238878846763287\\
17.6112556369877	-0.238883821949129\\
17.618030567711	-0.238888791995789\\
17.6248053041771	-0.238893756910469\\
17.6315798465962	-0.238898716700361\\
17.6383541951778	-0.238903671372643\\
17.6451283501314	-0.238908620934481\\
17.6519023116661	-0.238913565393028\\
17.6586760799906	-0.238918504755423\\
17.6654496553134	-0.238923439028793\\
17.6722230378426	-0.238928368220255\\
17.6789962277861	-0.238933292336909\\
17.6857692253515	-0.238938211385845\\
17.6925420307458	-0.23894312537414\\
17.6993146441762	-0.238948034308858\\
17.7060870658491	-0.238952938197051\\
17.712859295971	-0.238957837045759\\
17.7196313347478	-0.238962730862007\\
17.7264031823852	-0.23896761965281\\
17.7331748390886	-0.238972503425171\\
17.7399463050632	-0.238977382186078\\
17.7467175805137	-0.238982255942509\\
17.7534886656447	-0.238987124701429\\
17.7602595606603	-0.238991988469789\\
17.7670302657644	-0.23899684725453\\
17.7738007811607	-0.239001701062581\\
17.7805711070524	-0.239006549900856\\
17.7873412436426	-0.239011393776259\\
17.7941111911339	-0.239016232695682\\
17.8008809497288	-0.239021066666003\\
17.8076505196294	-0.239025895694089\\
17.8144199010375	-0.239030719786795\\
17.8211890941546	-0.239035538950965\\
17.8279580991819	-0.239040353193428\\
17.8347269163204	-0.239045162521003\\
17.8414955457708	-0.239049966940498\\
17.8482639877332	-0.239054766458706\\
17.8550322424079	-0.239059561082411\\
17.8618003099946	-0.239064350818383\\
17.8685681906927	-0.239069135673381\\
17.8753358847014	-0.239073915654153\\
17.8821033922196	-0.239078690767432\\
17.8888707134459	-0.239083461019944\\
17.8956378485786	-0.239088226418399\\
17.9024047978157	-0.239092986969497\\
17.909171561355	-0.239097742679926\\
17.9159381393939	-0.239102493556363\\
17.9227045321296	-0.239107239605472\\
17.9294707397588	-0.239111980833905\\
17.9362367624783	-0.239116717248305\\
17.9430026004843	-0.239121448855301\\
17.9497682539727	-0.239126175661511\\
17.9565337231395	-0.239130897673541\\
17.9632990081799	-0.239135614897987\\
17.9700641092891	-0.239140327341432\\
17.976829026662	-0.239145035010447\\
17.9835937604933	-0.239149737911593\\
17.9903583109772	-0.23915443605142\\
17.9971226783077	-0.239159129436464\\
18.0038868626787	-0.239163818073252\\
18.0106508642835	-0.239168501968299\\
18.0174146833153	-0.239173181128107\\
18.0241783199672	-0.23917785555917\\
18.0309417744316	-0.239182525267967\\
18.037705046901	-0.239187190260969\\
18.0444681375674	-0.239191850544633\\
18.0512310466226	-0.239196506125406\\
18.0579937742581	-0.239201157009725\\
18.0647563206653	-0.239205803204014\\
18.071518686035	-0.239210444714686\\
18.078280870558	-0.239215081548144\\
18.0850428744246	-0.239219713710778\\
18.091804697825	-0.239224341208969\\
18.0985663409491	-0.239228964049086\\
18.1053278039865	-0.239233582237487\\
18.1120890871265	-0.239238195780519\\
18.1188501905581	-0.239242804684517\\
18.1256111144702	-0.239247408955806\\
18.1323718590512	-0.239252008600701\\
18.1391324244894	-0.239256603625505\\
18.1458928109727	-0.23926119403651\\
18.1526530186889	-0.239265779839996\\
18.1594130478254	-0.239270361042236\\
18.1661728985693	-0.239274937649487\\
18.1729325711075	-0.239279509667999\\
18.1796920656267	-0.23928407710401\\
18.1864513823132	-0.239288639963746\\
18.193210521353	-0.239293198253426\\
18.1999694829322	-0.239297751979253\\
18.206728267236	-0.239302301147424\\
18.21348687445	-0.239306845764122\\
18.220245304759	-0.239311385835521\\
18.227003558348	-0.239315921367785\\
18.2337616354012	-0.239320452367065\\
18.2405195361031	-0.239324978839504\\
18.2472772606375	-0.239329500791232\\
18.2540348091883	-0.239334018228371\\
18.2607921819387	-0.239338531157031\\
18.2675493790721	-0.239343039583311\\
18.2743064007713	-0.239347543513301\\
18.2810632472191	-0.23935204295308\\
18.2878199185977	-0.239356537908715\\
18.2945764150894	-0.239361028386264\\
18.3013327368761	-0.239365514391776\\
18.3080888841393	-0.239369995931287\\
18.3148448570605	-0.239374473010824\\
18.3216006558207	-0.239378945636404\\
18.3283562806009	-0.239383413814033\\
18.3351117315815	-0.239387877549706\\
18.341867008943	-0.239392336849409\\
18.3486221128655	-0.239396791719118\\
18.3553770435287	-0.239401242164798\\
18.3621318011123	-0.239405688192403\\
18.3688863857955	-0.239410129807879\\
18.3756407977575	-0.23941456701716\\
18.382395037177	-0.23941899982617\\
18.3891491042327	-0.239423428240825\\
18.3959029991027	-0.239427852267027\\
18.4026567219653	-0.239432271910672\\
18.4094102729981	-0.239436687177644\\
18.4161636523787	-0.239441098073816\\
18.4229168602845	-0.239445504605053\\
18.4296698968925	-0.239449906777209\\
18.4364227623794	-0.239454304596128\\
18.4431754569219	-0.239458698067644\\
18.4499279806963	-0.239463087197582\\
18.4566803338785	-0.239467471991755\\
18.4634325166445	-0.239471852455969\\
18.4701845291697	-0.239476228596018\\
18.4769363716295	-0.239480600417686\\
18.4836880441989	-0.239484967926749\\
18.4904395470528	-0.239489331128972\\
18.4971908803658	-0.23949369003011\\
18.5039420443121	-0.239498044635909\\
18.5106930390659	-0.239502394952104\\
18.5174438648009	-0.239506740984423\\
18.5241945216909	-0.23951108273858\\
18.5309450099091	-0.239515420220283\\
18.5376953296286	-0.239519753435229\\
18.5444454810223	-0.239524082389106\\
18.5511954642629	-0.23952840708759\\
18.5579452795226	-0.239532727536351\\
18.5646949269737	-0.239537043741047\\
18.5714444067881	-0.239541355707327\\
18.5781937191373	-0.23954566344083\\
18.5849428641929	-0.239549966947187\\
18.591691842126	-0.239554266232018\\
18.5984406531075	-0.239558561300934\\
18.6051892973081	-0.239562852159537\\
18.6119377748983	-0.239567138813418\\
18.6186860860484	-0.23957142126816\\
18.6254342309282	-0.239575699529336\\
18.6321822097076	-0.239579973602511\\
18.6389300225561	-0.239584243493239\\
18.6456776696429	-0.239588509207064\\
18.6524251511371	-0.239592770749523\\
18.6591724672074	-0.239597028126143\\
18.6659196180225	-0.239601281342439\\
18.6726666037507	-0.239605530403921\\
18.6794134245601	-0.239609775316087\\
18.6861600806185	-0.239614016084426\\
18.6929065720936	-0.239618252714419\\
18.6996528991527	-0.239622485211536\\
18.7063990619631	-0.23962671358124\\
18.7131450606917	-0.239630937828983\\
18.7198908955051	-0.239635157960208\\
18.7266365665699	-0.23963937398035\\
18.7333820740523	-0.239643585894834\\
18.7401274181183	-0.239647793709077\\
18.7468725989336	-0.239651997428486\\
18.7536176166639	-0.239656197058458\\
18.7603624714744	-0.239660392604383\\
18.7671071635303	-0.23966458407164\\
18.7738516929964	-0.239668771465602\\
18.7805960600373	-0.239672954791629\\
18.7873402648174	-0.239677134055076\\
18.794084307501	-0.239681309261285\\
18.8008281882519	-0.239685480415593\\
18.807571907234	-0.239689647523326\\
18.8143154646106	-0.239693810589802\\
18.8210588605451	-0.239697969620328\\
18.8278020952006	-0.239702124620204\\
18.8345451687398	-0.239706275594722\\
18.8412880813253	-0.239710422549164\\
18.8480308331196	-0.239714565488803\\
18.8547734242848	-0.239718704418903\\
18.8615158549827	-0.239722839344721\\
18.8682581253752	-0.239726970271503\\
18.8750002356237	-0.239731097204487\\
18.8817421858894	-0.239735220148904\\
18.8884839763334	-0.239739339109973\\
18.8952256071166	-0.239743454092908\\
18.9019670783994	-0.239747565102912\\
18.9087083903424	-0.23975167214518\\
18.9154495431055	-0.239755775224898\\
18.9221905368489	-0.239759874347244\\
18.9289313717322	-0.239763969517387\\
18.9356720479148	-0.239768060740487\\
18.9424125655561	-0.239772148021698\\
18.9491529248152	-0.239776231366161\\
18.9558931258508	-0.239780310779013\\
18.9626331688217	-0.239784386265379\\
18.9693730538861	-0.239788457830378\\
18.9761127812024	-0.239792525479118\\
18.9828523509284	-0.239796589216703\\
18.989591763222	-0.239800649048223\\
18.9963310182407	-0.239804704978763\\
19.0030701161417	-0.239808757013399\\
19.0098090570823	-0.239812805157198\\
19.0165478412193	-0.239816849415221\\
19.0232864687095	-0.239820889792516\\
19.0300249397092	-0.239824926294127\\
19.0367632543747	-0.239828958925088\\
19.0435014128621	-0.239832987690425\\
19.0502394153273	-0.239837012595155\\
19.0569772619258	-0.239841033644287\\
19.063714952813	-0.239845050842823\\
19.0704524881441	-0.239849064195756\\
19.0771898680742	-0.239853073708069\\
19.083927092758	-0.23985707938474\\
19.09066416235	-0.239861081230736\\
19.0974010770047	-0.239865079251018\\
19.1041378368761	-0.239869073450538\\
19.1108744421182	-0.239873063834238\\
19.1176108928848	-0.239877050407056\\
19.1243471893293	-0.239881033173918\\
19.131083331605	-0.239885012139745\\
19.1378193198651	-0.239888987309447\\
19.1445551542624	-0.239892958687927\\
19.1512908349496	-0.239896926280082\\
19.1580263620793	-0.239900890090798\\
19.1647617358037	-0.239904850124955\\
19.1714969562748	-0.239908806387424\\
19.1782320236445	-0.239912758883069\\
19.1849669380645	-0.239916707616744\\
19.1917016996862	-0.239920652593298\\
19.1984363086609	-0.23992459381757\\
19.2051707651395	-0.239928531294391\\
19.2119050692731	-0.239932465028585\\
19.2186392212121	-0.239936395024969\\
19.225373221107	-0.23994032128835\\
19.2321070691081	-0.239944243823527\\
19.2388407653653	-0.239948162635295\\
19.2455743100286	-0.239952077728437\\
19.2523077032474	-0.239955989107729\\
19.2590409451713	-0.239959896777942\\
19.2657740359494	-0.239963800743835\\
19.2725069757308	-0.239967701010163\\
19.2792397646643	-0.239971597581671\\
19.2859724028985	-0.239975490463097\\
19.2927048905819	-0.239979379659171\\
19.2994372278625	-0.239983265174616\\
19.3061694148886	-0.239987147014147\\
19.3129014518078	-0.239991025182471\\
19.3196333387678	-0.239994899684288\\
19.3263650759161	-0.239998770524289\\
19.3330966633998	-0.240002637707159\\
19.339828101366	-0.240006501237575\\
19.3465593899615	-0.240010361120206\\
19.353290529333	-0.240014217359713\\
19.3600215196268	-0.240018069960751\\
19.3667523609893	-0.240021918927966\\
19.3734830535665	-0.240025764265996\\
19.3802135975042	-0.240029605979475\\
19.3869439929481	-0.240033444073024\\
19.3936742400436	-0.240037278551262\\
19.400404338936	-0.240041109418796\\
19.4071342897705	-0.240044936680229\\
19.4138640926918	-0.240048760340154\\
19.4205937478447	-0.240052580403159\\
19.4273232553736	-0.240056396873823\\
19.4340526154229	-0.240060209756717\\
19.4407818281366	-0.240064019056406\\
19.4475108936587	-0.240067824777448\\
19.4542398121329	-0.240071626924391\\
19.4609685837028	-0.24007542550178\\
19.4676972085116	-0.240079220514148\\
19.4744256867025	-0.240083011966023\\
19.4811540184185	-0.240086799861927\\
19.4878822038024	-0.240090584206373\\
19.4946102429967	-0.240094365003866\\
19.5013381361438	-0.240098142258905\\
19.508065883386	-0.240101915975982\\
19.5147934848652	-0.240105686159582\\
19.5215209407232	-0.240109452814181\\
19.5282482511018	-0.240113215944249\\
19.5349754161423	-0.240116975554249\\
19.541702435986	-0.240120731648636\\
19.548429310774	-0.24012448423186\\
19.5551560406472	-0.240128233308361\\
19.5618826257462	-0.240131978882573\\
19.5686090662116	-0.240135720958924\\
19.5753353621837	-0.240139459541834\\
19.5820615138027	-0.240143194635715\\
19.5887875212085	-0.240146926244973\\
19.5955133845409	-0.240150654374008\\
19.6022391039394	-0.24015437902721\\
19.6089646795435	-0.240158100208965\\
19.6156901114924	-0.240161817923649\\
19.6224153999251	-0.240165532175635\\
19.6291405449805	-0.240169242969285\\
19.6358655467973	-0.240172950308957\\
19.6425904055138	-0.240176654199\\
19.6493151212685	-0.240180354643756\\
19.6560396941994	-0.240184051647563\\
19.6627641244445	-0.240187745214748\\
19.6694884121416	-0.240191435349635\\
19.6762125574282	-0.240195122056537\\
19.6829365604417	-0.240198805339764\\
19.6896604213193	-0.240202485203617\\
19.6963841401981	-0.24020616165239\\
19.7031077172148	-0.240209834690371\\
19.7098311525063	-0.240213504321842\\
19.7165544462089	-0.240217170551076\\
19.7232775984591	-0.240220833382341\\
19.7300006093928	-0.240224492819897\\
19.7367234791462	-0.240228148867998\\
19.7434462078549	-0.240231801530891\\
19.7501687956546	-0.240235450812816\\
19.7568912426806	-0.240239096718007\\
19.7636135490683	-0.240242739250691\\
19.7703357149527	-0.240246378415087\\
19.7770577404686	-0.24025001421541\\
19.7837796257509	-0.240253646655866\\
19.790501370934	-0.240257275740654\\
19.7972229761523	-0.240260901473969\\
19.80394444154	-0.240264523859998\\
19.8106657672311	-0.240268142902919\\
19.8173869533595	-0.240271758606908\\
19.8241080000587	-0.24027537097613\\
19.8308289074623	-0.240278980014747\\
19.8375496757037	-0.240282585726911\\
19.8442703049158	-0.240286188116771\\
19.8509907952318	-0.240289787188466\\
19.8577111467843	-0.240293382946131\\
19.8644313597061	-0.240296975393893\\
19.8711514341296	-0.240300564535874\\
19.8778713701869	-0.240304150376188\\
19.8845911680103	-0.240307732918943\\
19.8913108277317	-0.240311312168241\\
19.8980303494828	-0.240314888128177\\
19.9047497333952	-0.240318460802839\\
19.9114689796003	-0.24032203019631\\
19.9181880882294	-0.240325596312665\\
19.9249070594135	-0.240329159155974\\
19.9316258932836	-0.2403327187303\\
19.9383445899704	-0.240336275039699\\
19.9450631496044	-0.240339828088222\\
19.951781572316	-0.240343377879913\\
19.9584998582356	-0.240346924418808\\
19.965218007493	-0.24035046770894\\
19.9719360202183	-0.240354007754332\\
19.9786538965411	-0.240357544559004\\
19.9853716365909	-0.240361078126968\\
19.9920892404973	-0.240364608462229\\
19.9988067083893	-0.240368135568787\\
20.0055240403961	-0.240371659450636\\
20.0122412366465	-0.240375180111762\\
20.0189582972693	-0.240378697556147\\
20.0256752223929	-0.240382211787764\\
20.0323920121458	-0.240385722810583\\
20.0391086666562	-0.240389230628565\\
20.0458251860522	-0.240392735245667\\
20.0525415704616	-0.240396236665838\\
20.0592578200121	-0.240399734893022\\
20.0659739348314	-0.240403229931156\\
20.0726899150468	-0.240406721784172\\
20.0794057607855	-0.240410210455995\\
20.0861214721746	-0.240413695950543\\
20.0928370493411	-0.24041717827173\\
20.0995524924116	-0.240420657423462\\
20.1062678015127	-0.240424133409641\\
20.1129829767708	-0.24042760623416\\
20.1196980183123	-0.240431075900909\\
20.1264129262631	-0.240434542413769\\
20.1331277007492	-0.240438005776618\\
20.1398423418965	-0.240441465993325\\
20.1465568498303	-0.240444923067755\\
20.1532712246763	-0.240448377003767\\
20.1599854665597	-0.240451827805212\\
20.1666995756057	-0.240455275475938\\
20.1734135519391	-0.240458720019784\\
20.1801273956847	-0.240462161440585\\
20.1868411069673	-0.240465599742169\\
20.1935546859113	-0.240469034928359\\
20.2002681326411	-0.240472467002972\\
20.2069814472807	-0.240475895969818\\
20.2136946299542	-0.240479321832702\\
20.2204076807854	-0.240482744595422\\
20.227120599898	-0.240486164261772\\
20.2338333874156	-0.240489580835539\\
20.2405460434615	-0.240492994320505\\
20.247258568159	-0.240496404720443\\
20.253970961631	-0.240499812039124\\
20.2606832240006	-0.240503216280312\\
20.2673953553903	-0.240506617447764\\
20.2741073559229	-0.240510015545233\\
20.2808192257207	-0.240513410576464\\
20.287530964906	-0.240516802545199\\
20.294242573601	-0.240520191455171\\
20.3009540519275	-0.24052357731011\\
20.3076654000075	-0.240526960113739\\
20.3143766179625	-0.240530339869775\\
20.3210877059141	-0.24053371658193\\
20.3277986639835	-0.240537090253909\\
20.3345094922921	-0.240540460889414\\
20.3412201909607	-0.240543828492138\\
20.3479307601104	-0.24054719306577\\
20.3546411998618	-0.240550554613994\\
20.3613515103355	-0.240553913140487\\
20.3680616916519	-0.24055726864892\\
20.3747717439313	-0.240560621142961\\
20.3814816672937	-0.240563970626268\\
20.3881914618593	-0.240567317102498\\
20.3949011277477	-0.240570660575299\\
20.4016106650787	-0.240574001048316\\
20.4083200739717	-0.240577338525186\\
20.4150293545461	-0.240580673009541\\
20.4217385069211	-0.240584004505008\\
20.4284475312158	-0.24058733301521\\
20.4351564275491	-0.240590658543761\\
20.4418651960396	-0.240593981094272\\
20.4485738368061	-0.240597300670347\\
20.455282349967	-0.240600617275586\\
20.4619907356405	-0.240603930913582\\
20.4686989939449	-0.240607241587924\\
20.4754071249981	-0.240610549302193\\
20.482115128918	-0.240613854059967\\
20.4888230058224	-0.240617155864819\\
20.4955307558287	-0.240620454720313\\
20.5022383790543	-0.240623750630011\\
20.5089458756166	-0.240627043597468\\
20.5156532456327	-0.240630333626234\\
20.5223604892195	-0.240633620719853\\
20.5290676064939	-0.240636904881865\\
20.5357745975725	-0.240640186115802\\
20.5424814625718	-0.240643464425194\\
20.5491882016083	-0.240646739813563\\
20.5558948147981	-0.240650012284427\\
20.5626013022574	-0.240653281841297\\
20.5693076641021	-0.240656548487681\\
20.576013900448	-0.240659812227079\\
20.5827200114108	-0.240663073062988\\
20.5894259971059	-0.2406663309989\\
20.5961318576488	-0.240669586038298\\
20.6028375931545	-0.240672838184663\\
20.6095432037383	-0.240676087441471\\
20.6162486895151	-0.24067933381219\\
20.6229540505996	-0.240682577300285\\
20.6296592871064	-0.240685817909215\\
20.6363643991501	-0.240689055642434\\
20.6430693868451	-0.24069229050339\\
20.6497742503055	-0.240695522495527\\
20.6564789896454	-0.240698751622282\\
20.6631836049787	-0.240701977887089\\
20.6698880964192	-0.240705201293374\\
20.6765924640806	-0.240708421844561\\
20.6832967080764	-0.240711639544066\\
20.6900008285199	-0.240714854395303\\
20.6967048255243	-0.240718066401677\\
20.7034086992028	-0.24072127556659\\
20.7101124496681	-0.240724481893439\\
20.7168160770333	-0.240727685385616\\
20.7235195814108	-0.240730886046506\\
20.7302229629132	-0.240734083879491\\
20.7369262216529	-0.240737278887947\\
20.7436293577422	-0.240740471075244\\
20.750332371293	-0.24074366044475\\
20.7570352624174	-0.240746846999824\\
20.7637380312272	-0.240750030743823\\
20.7704406778341	-0.240753211680096\\
20.7771432023495	-0.24075638981199\\
20.7838456048849	-0.240759565142845\\
20.7905478855516	-0.240762737675996\\
20.7972500444607	-0.240765907414774\\
20.8039520817231	-0.240769074362505\\
20.8106539974497	-0.240772238522509\\
20.8173557917513	-0.240775399898102\\
20.8240574647383	-0.240778558492593\\
20.8307590165213	-0.240781714309288\\
20.8374604472105	-0.240784867351488\\
20.844161756916	-0.240788017622489\\
20.850862945748	-0.240791165125581\\
20.8575640138163	-0.240794309864049\\
20.8642649612307	-0.240797451841174\\
20.8709657881007	-0.240800591060233\\
20.8776664945358	-0.240803727524495\\
20.8843670806455	-0.240806861237227\\
20.8910675465389	-0.24080999220169\\
20.897767892325	-0.24081312042114\\
20.9044681181129	-0.240816245898828\\
20.9111682240114	-0.240819368638001\\
20.917868210129	-0.240822488641899\\
20.9245680765744	-0.240825605913761\\
20.931267823456	-0.240828720456817\\
20.937967450882	-0.240831832274295\\
20.9446669589605	-0.240834941369417\\
20.9513663477997	-0.240838047745399\\
20.9580656175073	-0.240841151405455\\
20.964764768191	-0.240844252352793\\
20.9714637999586	-0.240847350590614\\
20.9781627129174	-0.240850446122118\\
20.9848615071749	-0.240853538950497\\
20.9915601828381	-0.240856629078941\\
20.9982587400143	-0.240859716510632\\
21.0049571788103	-0.24086280124875\\
21.011655499333	-0.24086588329647\\
21.018353701689	-0.240868962656961\\
21.025051785985	-0.240872039333387\\
21.0317497523272	-0.24087511332891\\
21.0384476008221	-0.240878184646683\\
21.0451453315758	-0.240881253289859\\
21.0518429446943	-0.240884319261583\\
21.0585404402835	-0.240887382564996\\
21.0652378184492	-0.240890443203236\\
21.071935079297	-0.240893501179433\\
21.0786322229325	-0.240896556496716\\
21.085329249461	-0.240899609158207\\
21.0920261589878	-0.240902659167024\\
21.098722951618	-0.240905706526281\\
21.1054196274566	-0.240908751239087\\
21.1121161866086	-0.240911793308544\\
21.1188126291785	-0.240914832737754\\
21.1255089552711	-0.240917869529812\\
21.1322051649909	-0.240920903687806\\
21.1389012584421	-0.240923935214824\\
21.145597235729	-0.240926964113947\\
21.1522930969558	-0.24092999038825\\
21.1589888422264	-0.240933014040807\\
21.1656844716446	-0.240936035074685\\
21.1723799853143	-0.240939053492946\\
21.1790753833389	-0.24094206929865\\
21.185770665822	-0.24094508249485\\
21.1924658328668	-0.240948093084596\\
21.1991608845767	-0.240951101070932\\
21.2058558210548	-0.240954106456899\\
21.2125506424039	-0.240957109245533\\
21.2192453487269	-0.240960109439865\\
21.2259399401266	-0.240963107042923\\
21.2326344167056	-0.240966102057728\\
21.2393287785662	-0.240969094487299\\
21.246023025811	-0.24097208433465\\
21.252717158542	-0.240975071602789\\
21.2594111768614	-0.240978056294721\\
21.2661050808712	-0.240981038413446\\
21.2727988706732	-0.240984017961961\\
21.2794925463691	-0.240986994943256\\
21.2861861080606	-0.240989969360319\\
21.2928795558491	-0.240992941216133\\
21.299572889836	-0.240995910513675\\
21.3062661101226	-0.240998877255919\\
21.3129592168098	-0.241001841445835\\
21.3196522099988	-0.241004803086388\\
21.3263450897904	-0.241007762180538\\
21.3330378562853	-0.241010718731243\\
21.3397305095842	-0.241013672741454\\
21.3464230497876	-0.241016624214119\\
21.3531154769958	-0.241019573152181\\
21.3598077913091	-0.241022519558579\\
21.3664999928277	-0.241025463436248\\
21.3731920816516	-0.241028404788119\\
21.3798840578806	-0.241031343617117\\
21.3865759216146	-0.241034279926165\\
21.3932676729532	-0.241037213718181\\
21.399959311996	-0.241040144996076\\
21.4066508388423	-0.241043073762761\\
21.4133422535915	-0.24104600002114\\
21.4200335563427	-0.241048923774114\\
21.4267247471951	-0.241051845024579\\
21.4334158262475	-0.241054763775427\\
21.4401067935988	-0.241057680029546\\
21.4467976493476	-0.24106059378982\\
21.4534883935926	-0.241063505059127\\
21.4601790264323	-0.241066413840344\\
21.4668695479649	-0.24106932013634\\
21.4735599582887	-0.241072223949984\\
21.4802502575018	-0.241075125284136\\
21.4869404457023	-0.241078024141657\\
21.4936305229879	-0.2410809205254\\
21.5003204894565	-0.241083814438215\\
21.5070103452057	-0.241086705882949\\
21.513700090333	-0.241089594862442\\
21.5203897249358	-0.241092481379533\\
21.5270792491114	-0.241095365437055\\
21.533768662957	-0.241098247037838\\
21.5404579665697	-0.241101126184707\\
21.5471471600464	-0.241104002880483\\
21.5538362434839	-0.241106877127982\\
21.5605252169789	-0.241109748930019\\
21.5672140806281	-0.241112618289402\\
21.5739028345279	-0.241115485208936\\
21.5805914787747	-0.241118349691421\\
21.5872800134647	-0.241121211739654\\
21.5939684386941	-0.241124071356428\\
21.6006567545588	-0.241126928544531\\
21.6073449611549	-0.241129783306748\\
21.614033058578	-0.241132635645859\\
21.620721046924	-0.241135485564641\\
21.6274089262882	-0.241138333065866\\
21.6340966967662	-0.241141178152302\\
21.6407843584534	-0.241144020826715\\
21.6474719114449	-0.241146861091863\\
21.6541593558359	-0.241149698950504\\
21.6608466917213	-0.24115253440539\\
21.667533919196	-0.24115536745927\\
21.6742210383549	-0.241158198114887\\
21.6809080492925	-0.241161026374983\\
21.6875949521034	-0.241163852242293\\
21.6942817468821	-0.241166675719551\\
21.7009684337229	-0.241169496809485\\
21.7076550127199	-0.24117231551482\\
21.7143414839673	-0.241175131838275\\
21.7210278475591	-0.241177945782569\\
21.7277141035892	-0.241180757350414\\
21.7344002521512	-0.241183566544519\\
21.7410862933389	-0.241186373367589\\
21.7477722272458	-0.241189177822325\\
21.7544580539653	-0.241191979911425\\
21.7611437735908	-0.241194779637582\\
21.7678293862155	-0.241197577003485\\
21.7745148919324	-0.24120037201182\\
21.7812002908346	-0.241203164665269\\
21.7878855830149	-0.241205954966509\\
21.7945707685661	-0.241208742918215\\
21.8012558475809	-0.241211528523058\\
21.8079408201518	-0.241214311783702\\
21.8146256863713	-0.241217092702812\\
21.8213104463317	-0.241219871283045\\
21.8279951001252	-0.241222647527057\\
21.8346796478439	-0.241225421437498\\
21.8413640895799	-0.241228193017016\\
21.848048425425	-0.241230962268254\\
21.8547326554711	-0.241233729193853\\
21.8614167798098	-0.241236493796448\\
21.8681007985327	-0.24123925607867\\
21.8747847117312	-0.241242016043149\\
21.8814685194967	-0.24124477369251\\
21.8881522219206	-0.241247529029372\\
21.8948358190938	-0.241250282056353\\
21.9015193111075	-0.241253032776066\\
21.9082026980525	-0.241255781191122\\
21.9148859800198	-0.241258527304125\\
21.9215691571	-0.241261271117678\\
21.9282522293837	-0.241264012634379\\
21.9349351969614	-0.241266751856824\\
21.9416180599236	-0.241269488787602\\
21.9483008183605	-0.241272223429302\\
21.9549834723623	-0.241274955784506\\
21.9616660220191	-0.241277685855796\\
21.9683484674209	-0.241280413645747\\
21.9750308086575	-0.241283139156931\\
21.9817130458187	-0.241285862391918\\
21.9883951789941	-0.241288583353272\\
21.9950772082734	-0.241291302043556\\
22.0017591337459	-0.241294018465327\\
22.008440955501	-0.24129673262114\\
22.015122673628	-0.241299444513545\\
22.0218042882159	-0.241302154145089\\
22.0284857993538	-0.241304861518316\\
22.0351672071307	-0.241307566635765\\
22.0418485116354	-0.241310269499972\\
22.0485297129565	-0.241312970113471\\
22.0552108111828	-0.241315668478789\\
22.0618918064026	-0.241318364598453\\
22.0685726987045	-0.241321058474984\\
22.0752534881766	-0.241323750110901\\
22.0819341749074	-0.241326439508717\\
22.0886147589847	-0.241329126670945\\
22.0952952404967	-0.24133181160009\\
22.1019756195312	-0.241334494298659\\
22.1086558961761	-0.24133717476915\\
22.115336070519	-0.24133985301406\\
22.1220161426474	-0.241342529035884\\
22.128696112649	-0.24134520283711\\
22.135375980611	-0.241347874420225\\
22.1420557466208	-0.241350543787712\\
22.1487354107655	-0.241353210942049\\
22.1554149731323	-0.241355875885714\\
22.1620944338081	-0.241358538621176\\
22.1687737928797	-0.241361199150907\\
22.1754530504341	-0.24136385747737\\
22.1821322065578	-0.241366513603027\\
22.1888112613374	-0.241369167530337\\
22.1954902148594	-0.241371819261754\\
22.2021690672102	-0.24137446879973\\
22.208847818476	-0.241377116146713\\
22.2155264687431	-0.241379761305147\\
22.2222050180975	-0.241382404277472\\
22.2288834666251	-0.241385045066127\\
22.235561814412	-0.241387683673545\\
22.2422400615438	-0.241390320102158\\
22.2489182081062	-0.241392954354392\\
22.2555962541848	-0.241395586432671\\
22.2622741998651	-0.241398216339415\\
22.2689520452325	-0.241400844077042\\
22.2756297903723	-0.241403469647965\\
22.2823074353696	-0.241406093054595\\
22.2889849803095	-0.241408714299337\\
22.295662425277	-0.241411333384596\\
22.3023397703571	-0.241413950312771\\
22.3090170156345	-0.24141656508626\\
22.3156941611939	-0.241419177707455\\
22.3223712071198	-0.241421788178746\\
22.3290481534969	-0.241424396502521\\
22.3357250004095	-0.241427002681161\\
22.3424017479419	-0.241429606717048\\
22.3490783961783	-0.241432208612557\\
22.3557549452029	-0.241434808370062\\
22.3624313950996	-0.241437405991932\\
22.3691077459524	-0.241440001480535\\
22.3757839978451	-0.241442594838234\\
22.3824601508614	-0.241445186067388\\
22.389136205085	-0.241447775170354\\
22.3958121605993	-0.241450362149485\\
22.4024880174879	-0.241452947007132\\
22.4091637758341	-0.241455529745641\\
22.415839435721	-0.241458110367356\\
22.422514997232	-0.241460688874617\\
22.4291904604499	-0.241463265269761\\
22.4358658254578	-0.241465839555121\\
22.4425410923386	-0.241468411733028\\
22.449216261175	-0.241470981805809\\
22.4558913320497	-0.241473549775788\\
22.4625663050453	-0.241476115645286\\
22.4692411802443	-0.24147867941662\\
22.475915957729	-0.241481241092104\\
22.4825906375818	-0.241483800674049\\
22.4892652198848	-0.241486358164763\\
22.4959397047202	-0.241488913566551\\
22.5026140921701	-0.241491466881713\\
22.5092883823162	-0.241494018112549\\
22.5159625752404	-0.241496567261352\\
22.5226366710246	-0.241499114330415\\
22.5293106697503	-0.241501659322026\\
22.535984571499	-0.24150420223847\\
22.5426583763523	-0.24150674308203\\
22.5493320843914	-0.241509281854984\\
22.5560056956978	-0.241511818559609\\
22.5626792103524	-0.241514353198177\\
22.5693526284365	-0.241516885772957\\
22.5760259500311	-0.241519416286217\\
22.5826991752169	-0.241521944740219\\
22.5893723040749	-0.241524471137223\\
22.5960453366857	-0.241526995479486\\
22.6027182731301	-0.241529517769262\\
22.6093911134884	-0.241532038008802\\
22.6160638578412	-0.241534556200353\\
22.6227365062688	-0.24153707234616\\
22.6294090588514	-0.241539586448463\\
22.6360815156693	-0.241542098509503\\
22.6427538768025	-0.241544608531512\\
22.649426142331	-0.241547116516724\\
22.6560983123347	-0.241549622467368\\
22.6627703868933	-0.241552126385669\\
22.6694423660867	-0.24155462827385\\
22.6761142499945	-0.24155712813413\\
22.6827860386961	-0.241559625968728\\
22.689457732271	-0.241562121779855\\
22.6961293307986	-0.241564615569723\\
22.7028008343582	-0.241567107340539\\
22.7094722430288	-0.241569597094508\\
22.7161435568897	-0.241572084833831\\
22.7228147760197	-0.241574570560706\\
22.7294859004978	-0.241577054277329\\
22.7361569304029	-0.241579535985891\\
22.7428278658136	-0.241582015688583\\
22.7494987068086	-0.241584493387591\\
22.7561694534664	-0.241586969085097\\
22.7628401058655	-0.241589442783282\\
22.7695106640843	-0.241591914484324\\
22.776181128201	-0.241594384190396\\
22.7828514982939	-0.24159685190367\\
22.7895217744411	-0.241599317626314\\
22.7961919567205	-0.241601781360493\\
22.8028620452101	-0.24160424310837\\
22.8095320399878	-0.241606702872104\\
22.8162019411313	-0.241609160653851\\
22.8228717487183	-0.241611616455765\\
22.8295414628264	-0.241614070279996\\
22.836211083533	-0.241616522128692\\
22.8428806109155	-0.241618972003998\\
22.8495500450514	-0.241621419908054\\
22.8562193860178	-0.241623865843\\
22.8628886338918	-0.241626309810971\\
22.8695577887506	-0.241628751814101\\
22.8762268506711	-0.241631191854518\\
22.8828958197301	-0.241633629934351\\
22.8895646960046	-0.241636066055723\\
22.8962334795712	-0.241638500220755\\
22.9029021705064	-0.241640932431566\\
22.909570768887	-0.24164336269027\\
22.9162392747893	-0.241645790998981\\
22.9229076882897	-0.241648217359808\\
22.9295760094645	-0.241650641774858\\
22.9362442383898	-0.241653064246234\\
22.9429123751419	-0.241655484776038\\
22.9495804197967	-0.241657903366367\\
22.9562483724302	-0.241660320019317\\
22.9629162331182	-0.241662734736979\\
22.9695840019365	-0.241665147521445\\
22.9762516789608	-0.241667558374799\\
22.9829192642667	-0.241669967299127\\
22.9895867579298	-0.241672374296508\\
22.9962541600254	-0.241674779369022\\
23.0029214706289	-0.241677182518743\\
23.0095886898156	-0.241679583747744\\
23.0162558176607	-0.241681983058094\\
23.0229228542393	-0.24168438045186\\
23.0295897996263	-0.241686775931107\\
23.0362566538968	-0.241689169497895\\
23.0429234171255	-0.241691561154283\\
23.0495900893872	-0.241693950902326\\
23.0562566707567	-0.241696338744078\\
23.0629231613085	-0.241698724681588\\
23.0695895611171	-0.241701108716903\\
23.076255870257	-0.241703490852069\\
23.0829220888025	-0.241705871089126\\
23.0895882168279	-0.241708249430113\\
23.0962542544073	-0.241710625877067\\
23.1029202016149	-0.241713000432021\\
23.1095860585248	-0.241715373097006\\
23.1162518252107	-0.241717743874049\\
23.1229175017466	-0.241720112765176\\
23.1295830882063	-0.241722479772409\\
23.1362485846635	-0.241724844897767\\
23.1429139911917	-0.241727208143267\\
23.1495793078645	-0.241729569510925\\
23.1562445347554	-0.24173192900275\\
23.1629096719376	-0.241734286620752\\
23.1695747194846	-0.241736642366937\\
23.1762396774694	-0.241738996243307\\
23.1829045459653	-0.241741348251865\\
23.1895693250452	-0.241743698394606\\
23.1962340147821	-0.241746046673528\\
23.202898615249	-0.241748393090621\\
23.2095631265185	-0.241750737647877\\
23.2162275486635	-0.241753080347281\\
23.2228918817565	-0.241755421190818\\
23.2295561258701	-0.241757760180471\\
23.2362202810768	-0.241760097318218\\
23.242884347449	-0.241762432606035\\
23.249548325059	-0.241764766045897\\
23.2562122139791	-0.241767097639774\\
23.2628760142814	-0.241769427389634\\
23.2695397260379	-0.241771755297444\\
23.2762033493207	-0.241774081365167\\
23.2828668842018	-0.241776405594763\\
23.2895303307528	-0.24177872798819\\
23.2961936890457	-0.241781048547403\\
23.3028569591521	-0.241783367274354\\
23.3095201411435	-0.241785684170995\\
23.3161832350916	-0.241787999239271\\
23.3228462410677	-0.241790312481127\\
23.3295091591432	-0.241792623898507\\
23.3361719893895	-0.241794933493349\\
23.3428347318777	-0.241797241267589\\
23.3494973866789	-0.241799547223163\\
23.3561599538643	-0.241801851362003\\
23.3628224335047	-0.241804153686036\\
23.3694848256711	-0.24180645419719\\
23.3761471304343	-0.241808752897388\\
23.3828093478651	-0.241811049788553\\
23.3894714780341	-0.241813344872602\\
23.3961335210118	-0.241815638151451\\
23.4027954768689	-0.241817929627016\\
23.4094573456757	-0.241820219301205\\
23.4161191275026	-0.241822507175929\\
23.4227808224199	-0.241824793253093\\
23.4294424304978	-0.241827077534601\\
23.4361039518063	-0.241829360022352\\
23.4427653864156	-0.241831640718246\\
23.4494267343956	-0.241833919624179\\
23.4560879958163	-0.241836196742043\\
23.4627491707473	-0.241838472073729\\
23.4694102592585	-0.241840745621126\\
23.4760712614195	-0.241843017386119\\
23.4827321773	-0.241845287370591\\
23.4893930069694	-0.241847555576424\\
23.4960537504972	-0.241849822005494\\
23.5027144079527	-0.241852086659678\\
23.5093749794052	-0.241854349540849\\
23.5160354649239	-0.241856610650878\\
23.522695864578	-0.241858869991632\\
23.5293561784366	-0.241861127564977\\
23.5360164065685	-0.241863383372777\\
23.5426765490427	-0.241865637416891\\
23.5493366059281	-0.241867889699179\\
23.5559965772934	-0.241870140221497\\
23.5626564632073	-0.241872388985696\\
23.5693162637384	-0.241874635993629\\
23.5759759789552	-0.241876881247143\\
23.5826356089262	-0.241879124748084\\
23.5892951537197	-0.241881366498296\\
23.5959546134042	-0.24188360649962\\
23.6026139880478	-0.241885844753894\\
23.6092732777186	-0.241888081262954\\
23.6159324824848	-0.241890316028635\\
23.6225916024144	-0.241892549052766\\
23.6292506375754	-0.241894780337178\\
23.6359095880355	-0.241897009883696\\
23.6425684538626	-0.241899237694145\\
23.6492272351244	-0.241901463770345\\
23.6558859318886	-0.241903688114116\\
23.6625445442227	-0.241905910727275\\
23.6692030721942	-0.241908131611637\\
23.6758615158706	-0.241910350769012\\
23.6825198753191	-0.24191256820121\\
23.6891781506072	-0.241914783910039\\
23.6958363418019	-0.241916997897303\\
23.7024944489704	-0.241919210164805\\
23.7091524721799	-0.241921420714344\\
23.7158104114972	-0.241923629547718\\
23.7224682669893	-0.241925836666723\\
23.729126038723	-0.241928042073151\\
23.7357837267651	-0.241930245768792\\
23.7424413311824	-0.241932447755435\\
23.7490988520414	-0.241934648034865\\
23.7557562894087	-0.241936846608866\\
23.7624136433507	-0.241939043479219\\
23.769070913934	-0.241941238647702\\
23.7757281012247	-0.241943432116093\\
23.7823852052893	-0.241945623886163\\
23.7890422261939	-0.241947813959687\\
23.7956991640046	-0.241950002338432\\
23.8023560187874	-0.241952189024165\\
23.8090127906085	-0.241954374018652\\
23.8156694795336	-0.241956557323655\\
23.8223260856286	-0.241958738940934\\
23.8289826089594	-0.241960918872246\\
23.8356390495915	-0.241963097119346\\
23.8422954075906	-0.241965273683989\\
23.8489516830223	-0.241967448567924\\
23.8556078759521	-0.241969621772901\\
23.8622639864454	-0.241971793300665\\
23.8689200145675	-0.24197396315296\\
23.8755759603837	-0.241976131331528\\
23.8822318239592	-0.241978297838108\\
23.8888876053592	-0.241980462674438\\
23.8955433046488	-0.241982625842251\\
23.9021989218928	-0.241984787343281\\
23.9088544571563	-0.241986947179257\\
23.9155099105041	-0.241989105351908\\
23.922165282001	-0.241991261862959\\
23.9288205717117	-0.241993416714134\\
23.9354757797009	-0.241995569907153\\
23.9421309060332	-0.241997721443737\\
23.948785950773	-0.2419998713256\\
23.9554409139848	-0.242002019554459\\
23.962095795733	-0.242004166132024\\
23.9687505960819	-0.242006311060007\\
23.9754053150957	-0.242008454340114\\
23.9820599528386	-0.242010595974051\\
23.9887145093746	-0.242012735963521\\
23.9953689847679	-0.242014874310225\\
24.0020233790824	-0.242017011015863\\
24.008677692382	-0.24201914608213\\
24.0153319247304	-0.242021279510721\\
24.0219860761915	-0.242023411303328\\
24.028640146829	-0.242025541461641\\
24.0352941367064	-0.242027669987348\\
24.0419480458873	-0.242029796882135\\
24.0486018744353	-0.242031922147683\\
24.0552556224136	-0.242034045785676\\
24.0619092898858	-0.242036167797792\\
24.068562876915	-0.242038288185706\\
24.0752163835645	-0.242040406951095\\
24.0818698098974	-0.242042524095631\\
24.0885231559768	-0.242044639620983\\
24.0951764218657	-0.24204675352882\\
24.1018296076271	-0.242048865820807\\
24.1084827133238	-0.242050976498609\\
24.1151357390187	-0.242053085563887\\
24.1217886847745	-0.242055193018299\\
24.1284415506539	-0.242057298863505\\
24.1350943367195	-0.242059403101158\\
24.1417470430338	-0.242061505732911\\
24.1483996696593	-0.242063606760416\\
24.1550522166585	-0.242065706185321\\
24.1617046840937	-0.242067804009273\\
24.1683570720271	-0.242069900233915\\
24.1750093805211	-0.24207199486089\\
24.1816616096376	-0.242074087891838\\
24.1883137594389	-0.242076179328398\\
24.1949658299869	-0.242078269172205\\
24.2016178213436	-0.242080357424892\\
24.2082697335709	-0.242082444088092\\
24.2149215667306	-0.242084529163434\\
24.2215733208844	-0.242086612652546\\
24.2282249960941	-0.242088694557052\\
24.2348765924212	-0.242090774878576\\
24.2415281099274	-0.242092853618739\\
24.2481795486741	-0.242094930779159\\
24.2548309087227	-0.242097006361455\\
24.2614821901347	-0.242099080367241\\
24.2681333929712	-0.242101152798129\\
24.2747845172937	-0.24210322365573\\
24.2814355631631	-0.242105292941652\\
24.2880865306407	-0.242107360657503\\
24.2947374197875	-0.242109426804887\\
24.3013882306644	-0.242111491385406\\
24.3080389633323	-0.24211355440066\\
24.3146896178522	-0.242115615852247\\
24.3213401942848	-0.242117675741764\\
24.3279906926908	-0.242119734070805\\
24.3346411131309	-0.242121790840962\\
24.3412914556656	-0.242123846053825\\
24.3479417203555	-0.242125899710982\\
24.3545919072611	-0.242127951814018\\
24.3612420164428	-0.242130002364518\\
24.3678920479609	-0.242132051364063\\
24.3745420018756	-0.242134098814234\\
24.3811918782472	-0.242136144716608\\
24.3878416771359	-0.242138189072761\\
24.3944913986017	-0.242140231884266\\
24.4011410427046	-0.242142273152696\\
24.4077906095047	-0.242144312879619\\
24.4144400990617	-0.242146351066604\\
24.4210895114356	-0.242148387715216\\
24.4277388466861	-0.24215042282702\\
24.4343881048729	-0.242152456403575\\
24.4410372860556	-0.242154488446443\\
24.4476863902939	-0.242156518957181\\
24.4543354176473	-0.242158547937344\\
24.4609843681751	-0.242160575388487\\
24.4676332419369	-0.24216260131216\\
24.474282038992	-0.242164625709913\\
24.4809307593995	-0.242166648583295\\
24.4875794032188	-0.24216866993385\\
24.4942279705089	-0.242170689763123\\
24.500876461329	-0.242172708072656\\
24.5075248757381	-0.242174724863987\\
24.5141732137952	-0.242176740138655\\
24.5208214755591	-0.242178753898197\\
24.5274696610887	-0.242180766144144\\
24.5341177704427	-0.24218277687803\\
24.5407658036799	-0.242184786101385\\
24.547413760859	-0.242186793815736\\
24.5540616420384	-0.24218880002261\\
24.5607094472769	-0.242190804723531\\
24.5673571766328	-0.24219280792002\\
24.5740048301645	-0.242194809613599\\
24.5806524079304	-0.242196809805785\\
24.5872999099888	-0.242198808498094\\
24.5939473363979	-0.242200805692042\\
24.6005946872159	-0.24220280138914\\
24.6072419625008	-0.2422047955909\\
24.6138891623108	-0.242206788298829\\
24.6205362867038	-0.242208779514435\\
24.6271833357378	-0.242210769239223\\
24.6338303094705	-0.242212757474694\\
24.6404772079599	-0.242214744222351\\
24.6471240312636	-0.242216729483692\\
24.6537707794394	-0.242218713260215\\
24.6604174525448	-0.242220695553414\\
24.6670640506375	-0.242222676364783\\
24.6737105737749	-0.242224655695814\\
24.6803570220145	-0.242226633547996\\
24.6870033954136	-0.242228609922816\\
24.6936496940297	-0.242230584821761\\
24.7002959179199	-0.242232558246315\\
24.7069420671415	-0.242234530197959\\
24.7135881417517	-0.242236500678173\\
24.7202341418074	-0.242238469688437\\
24.7268800673658	-0.242240437230225\\
24.7335259184838	-0.242242403305013\\
24.7401716952183	-0.242244367914274\\
24.7468173976262	-0.242246331059478\\
24.7534630257643	-0.242248292742094\\
24.7601085796893	-0.242250252963589\\
24.7667540594579	-0.242252211725428\\
24.7733994651266	-0.242254169029075\\
24.7800447967522	-0.242256124875991\\
24.786690054391	-0.242258079267636\\
24.7933352380995	-0.242260032205468\\
24.7999803479341	-0.242261983690942\\
24.8066253839512	-0.242263933725513\\
24.8132703462069	-0.242265882310632\\
24.8199152347576	-0.242267829447752\\
24.8265600496593	-0.242269775138319\\
24.8332047909681	-0.242271719383781\\
24.8398494587401	-0.242273662185583\\
24.8464940530313	-0.242275603545167\\
24.8531385738976	-0.242277543463976\\
24.8597830213948	-0.242279481943449\\
24.8664273955788	-0.242281418985023\\
24.8730716965053	-0.242283354590135\\
24.87971592423	-0.242285288760217\\
24.8863600788085	-0.242287221496704\\
24.8930041602965	-0.242289152801024\\
24.8996481687493	-0.242291082674607\\
24.9062921042226	-0.24229301111888\\
24.9129359667716	-0.242294938135268\\
24.9195797564518	-0.242296863725193\\
24.9262234733184	-0.242298787890077\\
24.9328671174267	-0.242300710631341\\
24.9395106888318	-0.242302631950402\\
24.946154187589	-0.242304551848676\\
24.9527976137531	-0.242306470327577\\
24.9594409673793	-0.242308387388518\\
24.9660842485225	-0.24231030303291\\
24.9727274572376	-0.242312217262162\\
24.9793705935794	-0.242314130077681\\
24.9860136576027	-0.242316041480872\\
24.9926566493623	-0.24231795147314\\
24.9992995689128	-0.242319860055886\\
25.0059424163088	-0.24232176723051\\
25.0125851916049	-0.242323672998412\\
25.0192278948556	-0.242325577360987\\
25.0258705261154	-0.24232748031963\\
25.0325130854385	-0.242329381875736\\
25.0391555728795	-0.242331282030694\\
25.0457979884925	-0.242333180785896\\
25.0524403323317	-0.242335078142728\\
25.0590826044514	-0.242336974102578\\
25.0657248049057	-0.242338868666829\\
25.0723669337486	-0.242340761836865\\
25.0790089910341	-0.242342653614066\\
25.0856509768162	-0.242344543999812\\
25.0922928911487	-0.242346432995481\\
25.0989347340856	-0.242348320602448\\
25.1055765056805	-0.242350206822087\\
25.1122182059872	-0.242352091655771\\
25.1188598350595	-0.242353975104871\\
25.1255013929509	-0.242355857170756\\
25.1321428797149	-0.242357737854793\\
25.1387842954051	-0.242359617158347\\
25.145425640075	-0.242361495082783\\
25.1520669137779	-0.242363371629462\\
25.1587081165672	-0.242365246799746\\
25.1653492484962	-0.242367120594992\\
25.1719903096182	-0.242368993016559\\
25.1786312999862	-0.242370864065802\\
25.1852722196535	-0.242372733744074\\
25.1919130686731	-0.242374602052727\\
25.198553847098	-0.242376468993112\\
25.2051945549813	-0.242378334566578\\
25.2118351923758	-0.242380198774471\\
25.2184757593343	-0.242382061618137\\
25.2251162559098	-0.242383923098919\\
25.2317566821549	-0.24238578321816\\
25.2383970381224	-0.2423876419772\\
25.2450373238649	-0.242389499377378\\
25.2516775394349	-0.24239135542003\\
25.2583176848851	-0.242393210106492\\
25.2649577602679	-0.242395063438097\\
25.2715977656358	-0.242396915416179\\
25.2782377010411	-0.242398766042066\\
25.2848775665362	-0.242400615317088\\
25.2915173621733	-0.242402463242572\\
25.2981570880046	-0.242404309819844\\
25.3047967440824	-0.242406155050226\\
25.3114363304588	-0.242407998935042\\
25.3180758471858	-0.242409841475612\\
25.3247152943154	-0.242411682673254\\
25.3313546718995	-0.242413522529287\\
25.3379939799902	-0.242415361045025\\
25.3446332186392	-0.242417198221783\\
25.3512723878984	-0.242419034060872\\
25.3579114878194	-0.242420868563605\\
25.3645505184541	-0.24242270173129\\
25.371189479854	-0.242424533565234\\
25.3778283720707	-0.242426364066743\\
25.3844671951558	-0.242428193237122\\
25.3911059491607	-0.242430021077673\\
25.3977446341369	-0.242431847589698\\
25.4043832501359	-0.242433672774495\\
25.4110217972088	-0.242435496633364\\
25.4176602754071	-0.242437319167599\\
25.4242986847818	-0.242439140378496\\
25.4309370253843	-0.242440960267348\\
25.4375752972656	-0.242442778835446\\
25.4442135004769	-0.24244459608408\\
25.450851635069	-0.242446412014538\\
25.4574897010931	-0.242448226628107\\
25.4641276986	-0.242450039926073\\
25.4707656276406	-0.242451851909719\\
25.4774034882658	-0.242453662580326\\
25.4840412805262	-0.242455471939175\\
25.4906790044726	-0.242457279987546\\
25.4973166601557	-0.242459086726714\\
25.5039542476261	-0.242460892157957\\
25.5105917669343	-0.242462696282547\\
25.517229218131	-0.242464499101758\\
25.5238666012664	-0.24246630061686\\
25.5305039163911	-0.242468100829123\\
25.5371411635554	-0.242469899739815\\
25.5437783428097	-0.242471697350202\\
25.5504154542041	-0.242473493661548\\
25.5570524977889	-0.242475288675118\\
25.5636894736143	-0.242477082392172\\
25.5703263817303	-0.242478874813971\\
25.5769632221871	-0.242480665941773\\
25.5835999950346	-0.242482455776836\\
25.5902367003228	-0.242484244320414\\
25.5968733381016	-0.242486031573762\\
25.6035099084209	-0.242487817538132\\
25.6101464113305	-0.242489602214775\\
25.6167828468802	-0.24249138560494\\
25.6234192151196	-0.242493167709875\\
25.6300555160984	-0.242494948530827\\
25.6366917498662	-0.242496728069039\\
25.6433279164727	-0.242498506325756\\
25.6499640159672	-0.242500283302219\\
25.6566000483994	-0.242502058999668\\
25.6632360138185	-0.242503833419343\\
25.669871912274	-0.242505606562479\\
25.6765077438151	-0.242507378430313\\
25.6831435084912	-0.242509149024079\\
25.6897792063515	-0.242510918345009\\
25.696414837445	-0.242512686394335\\
25.7030504018211	-0.242514453173286\\
25.7096858995286	-0.242516218683091\\
25.7163213306167	-0.242517982924976\\
25.7229566951344	-0.242519745900166\\
25.7295919931304	-0.242521507609885\\
25.7362272246538	-0.242523268055355\\
25.7428623897533	-0.242525027237797\\
25.7494974884778	-0.242526785158429\\
25.7561325208759	-0.24252854181847\\
25.7627674869964	-0.242530297219136\\
25.7694023868878	-0.242532051361642\\
25.7760372205988	-0.2425338042472\\
25.7826719881779	-0.242535555877023\\
25.7893066896736	-0.242537306252321\\
25.7959413251343	-0.242539055374302\\
25.8025758946085	-0.242540803244175\\
25.8092103981444	-0.242542549863144\\
25.8158448357904	-0.242544295232416\\
25.8224792075947	-0.242546039353191\\
25.8291135136056	-0.242547782226673\\
25.8357477538711	-0.24254952385406\\
25.8423819284394	-0.242551264236553\\
25.8490160373586	-0.242553003375347\\
25.8556500806766	-0.242554741271638\\
25.8622840584414	-0.242556477926621\\
25.868917970701	-0.242558213341489\\
25.8755518175031	-0.242559947517433\\
25.8821855988957	-0.242561680455643\\
25.8888193149266	-0.242563412157306\\
25.8954529656433	-0.242565142623611\\
25.9020865510937	-0.242566871855744\\
25.9087200713252	-0.242568599854887\\
25.9153535263857	-0.242570326622224\\
25.9219869163225	-0.242572052158936\\
25.9286202411832	-0.242573776466203\\
25.9352535010151	-0.242575499545203\\
25.9418866958658	-0.242577221397114\\
25.9485198257826	-0.242578942023111\\
25.9551528908127	-0.242580661424368\\
25.9617858910035	-0.242582379602058\\
25.9684188264021	-0.242584096557353\\
25.9750516970557	-0.242585812291422\\
25.9816845030114	-0.242587526805433\\
25.9883172443163	-0.242589240100555\\
25.9949499210175	-0.242590952177952\\
26.0015825331619	-0.242592663038788\\
26.0082150807964	-0.242594372684228\\
26.0148475639679	-0.242596081115432\\
26.0214799827233	-0.24259778833356\\
26.0281123371094	-0.242599494339771\\
26.0347446271729	-0.242601199135223\\
26.0413768529606	-0.242602902721071\\
26.048009014519	-0.24260460509847\\
26.0546411118948	-0.242606306268572\\
26.0612731451346	-0.24260800623253\\
26.0679051142848	-0.242609704991494\\
26.074537019392	-0.242611402546613\\
26.0811688605027	-0.242613098899035\\
26.087800637663	-0.242614794049905\\
26.0944323509195	-0.242616488000369\\
26.1010640003185	-0.242618180751569\\
26.107695585906	-0.242619872304649\\
26.1143271077285	-0.242621562660749\\
26.1209585658319	-0.242623251821008\\
26.1275899602625	-0.242624939786564\\
26.1342212910663	-0.242626626558553\\
26.1408525582893	-0.242628312138112\\
26.1474837619775	-0.242629996526373\\
26.1541149021768	-0.24263167972447\\
26.1607459789332	-0.242633361733533\\
26.1673769922924	-0.242635042554693\\
26.1740079423003	-0.242636722189077\\
26.1806388290026	-0.242638400637814\\
26.187269652445	-0.242640077902028\\
26.1939004126732	-0.242641753982844\\
26.2005311097328	-0.242643428881385\\
26.2071617436694	-0.242645102598773\\
26.2137923145285	-0.242646775136128\\
26.2204228223556	-0.24264844649457\\
26.2270532671962	-0.242650116675215\\
26.2336836490956	-0.242651785679181\\
26.2403139680992	-0.242653453507583\\
26.2469442242523	-0.242655120161533\\
26.2535744176001	-0.242656785642144\\
26.260204548188	-0.242658449950529\\
26.266834616061	-0.242660113087795\\
26.2734646212643	-0.242661775055052\\
26.280094563843	-0.242663435853406\\
26.2867244438421	-0.242665095483964\\
26.2933542613067	-0.24266675394783\\
26.2999840162816	-0.242668411246107\\
26.3066137088119	-0.242670067379896\\
26.3132433389423	-0.242671722350298\\
26.3198729067178	-0.242673376158413\\
26.326502412183	-0.242675028805338\\
26.3331318553828	-0.24267668029217\\
26.3397612363617	-0.242678330620004\\
26.3463905551646	-0.242679979789934\\
26.353019811836	-0.242681627803052\\
26.3596490064204	-0.242683274660451\\
26.3662781389623	-0.242684920363219\\
26.3729072095064	-0.242686564912447\\
26.3795362180969	-0.24268820830922\\
26.3861651647783	-0.242689850554626\\
26.392794049595	-0.242691491649749\\
26.3994228725912	-0.242693131595673\\
26.4060516338112	-0.24269477039348\\
26.4126803332992	-0.242696408044251\\
26.4193089710995	-0.242698044549066\\
26.4259375472561	-0.242699679909003\\
26.4325660618131	-0.24270131412514\\
26.4391945148147	-0.242702947198552\\
26.4458229063047	-0.242704579130315\\
26.4524512363273	-0.2427062099215\\
26.4590795049262	-0.242707839573181\\
26.4657077121454	-0.242709468086428\\
26.4723358580287	-0.242711095462311\\
26.4789639426199	-0.242712721701898\\
26.4855919659628	-0.242714346806256\\
26.4922199281011	-0.242715970776451\\
26.4988478290784	-0.242717593613547\\
26.5054756689384	-0.242719215318608\\
26.5121034477247	-0.242720835892694\\
26.5187311654807	-0.242722455336868\\
26.5253588222501	-0.242724073652189\\
26.5319864180763	-0.242725690839714\\
26.5386139530026	-0.242727306900501\\
26.5452414270725	-0.242728921835605\\
26.5518688403293	-0.242730535646082\\
26.5584961928163	-0.242732148332983\\
26.5651234845768	-0.242733759897361\\
26.5717507156539	-0.242735370340267\\
26.5783778860908	-0.242736979662751\\
26.5850049959306	-0.242738587865859\\
26.5916320452165	-0.24274019495064\\
26.5982590339915	-0.24274180091814\\
26.6048859622986	-0.242743405769402\\
26.6115128301807	-0.242745009505471\\
26.6181396376807	-0.242746612127387\\
26.6247663848416	-0.242748213636193\\
26.6313930717062	-0.242749814032927\\
26.6380196983172	-0.242751413318628\\
26.6446462647175	-0.242753011494334\\
26.6512727709497	-0.24275460856108\\
26.6578992170566	-0.2427562045199\\
26.6645256030807	-0.24275779937183\\
26.6711519290646	-0.2427593931179\\
26.6777781950509	-0.242760985759142\\
26.6844044010822	-0.242762577296585\\
26.6910305472008	-0.242764167731259\\
26.6976566334492	-0.242765757064192\\
26.7042826598698	-0.242767345296408\\
26.7109086265049	-0.242768932428933\\
26.7175345333969	-0.242770518462792\\
26.724160380588	-0.242772103399006\\
26.7307861681204	-0.242773687238598\\
26.7374118960363	-0.242775269982587\\
26.744037564378	-0.242776851631993\\
26.7506631731874	-0.242778432187832\\
26.7572887225066	-0.242780011651123\\
26.7639142123777	-0.242781590022881\\
26.7705396428426	-0.242783167304119\\
26.7771650139434	-0.242784743495851\\
26.7837903257218	-0.242786318599089\\
26.7904155782199	-0.242787892614844\\
26.7970407714793	-0.242789465544125\\
26.8036659055418	-0.24279103738794\\
26.8102909804493	-0.242792608147297\\
26.8169159962435	-0.242794177823201\\
26.8235409529659	-0.242795746416658\\
26.8301658506582	-0.242797313928671\\
26.8367906893621	-0.242798880360243\\
26.8434154691189	-0.242800445712375\\
26.8500401899704	-0.242802009986067\\
26.8566648519579	-0.242803573182318\\
26.8632894551228	-0.242805135302125\\
26.8699139995066	-0.242806696346487\\
26.8765384851506	-0.242808256316397\\
26.883162912096	-0.242809815212851\\
26.8897872803843	-0.242811373036841\\
26.8964115900566	-0.24281292978936\\
26.903035841154	-0.242814485471398\\
26.9096600337178	-0.242816040083945\\
26.9162841677891	-0.24281759362799\\
26.9229082434089	-0.24281914610452\\
26.9295322606184	-0.242820697514521\\
26.9361562194583	-0.242822247858978\\
26.9427801199699	-0.242823797138876\\
26.9494039621938	-0.242825345355197\\
26.9560277461712	-0.242826892508922\\
26.9626514719427	-0.242828438601033\\
26.9692751395492	-0.242829983632508\\
26.9758987490314	-0.242831527604326\\
26.9825223004301	-0.242833070517464\\
26.989145793786	-0.242834612372898\\
26.9957692291397	-0.242836153171603\\
27.0023926065318	-0.242837692914552\\
27.0090159260029	-0.242839231602718\\
27.0156391875936	-0.242840769237072\\
27.0222623913442	-0.242842305818585\\
27.0288855372954	-0.242843841348226\\
27.0355086254875	-0.242845375826964\\
27.0421316559608	-0.242846909255764\\
27.0487546287558	-0.242848441635592\\
27.0553775439127	-0.242849972967415\\
27.0620004014719	-0.242851503252194\\
27.0686232014734	-0.242853032490892\\
27.0752459439576	-0.242854560684472\\
27.0818686289645	-0.242856087833892\\
27.0884912565344	-0.242857613940113\\
27.0951138267072	-0.242859139004092\\
27.101736339523	-0.242860663026786\\
27.1083587950218	-0.242862186009151\\
27.1149811932436	-0.242863707952141\\
27.1216035342283	-0.242865228856711\\
27.1282258180158	-0.242866748723812\\
27.1348480446459	-0.242868267554397\\
27.1414702141585	-0.242869785349414\\
27.1480923265933	-0.242871302109814\\
27.1547143819901	-0.242872817836545\\
27.1613363803886	-0.242874332530553\\
27.1679583218284	-0.242875846192785\\
27.1745802063491	-0.242877358824185\\
27.1812020339904	-0.242878870425697\\
27.1878238047918	-0.242880380998264\\
27.1944455187929	-0.242881890542827\\
27.201067176033	-0.242883399060327\\
27.2076887765517	-0.242884906551702\\
27.2143103203883	-0.242886413017892\\
27.2209318075823	-0.242887918459834\\
27.2275532381728	-0.242889422878463\\
27.2341746121994	-0.242890926274715\\
27.2407959297011	-0.242892428649523\\
27.2474171907172	-0.242893930003821\\
27.2540383952869	-0.242895430338541\\
27.2606595434494	-0.242896929654613\\
27.2672806352437	-0.242898427952966\\
27.2739016707089	-0.24289992523453\\
27.2805226498842	-0.242901421500232\\
27.2871435728083	-0.242902916750999\\
27.2937644395204	-0.242904410987756\\
27.3003852500594	-0.242905904211426\\
27.3070060044641	-0.242907396422935\\
27.3136267027734	-0.242908887623203\\
27.3202473450261	-0.242910377813152\\
27.326867931261	-0.242911866993702\\
27.3334884615169	-0.242913355165773\\
27.3401089358324	-0.242914842330282\\
27.3467293542462	-0.242916328488145\\
27.353349716797	-0.24291781364028\\
27.3599700235234	-0.242919297787601\\
27.3665902744639	-0.242920780931021\\
27.373210469657	-0.242922263071454\\
27.3798306091413	-0.242923744209811\\
27.3864506929553	-0.242925224347003\\
27.3930707211372	-0.242926703483939\\
27.3996906937256	-0.242928181621528\\
27.4063106107587	-0.242929658760678\\
27.4129304722749	-0.242931134902295\\
27.4195502783125	-0.242932610047284\\
27.4261700289096	-0.242934084196551\\
27.4327897241046	-0.242935557350997\\
27.4394093639355	-0.242937029511527\\
27.4460289484406	-0.24293850067904\\
27.4526484776579	-0.242939970854438\\
27.4592679516254	-0.24294144003862\\
27.4658873703813	-0.242942908232483\\
27.4725067339635	-0.242944375436926\\
27.4791260424099	-0.242945841652844\\
27.4857452957585	-0.242947306881133\\
27.4923644940473	-0.242948771122686\\
27.4989836373139	-0.242950234378397\\
27.5056027255963	-0.242951696649158\\
27.5122217589322	-0.242953157935859\\
27.5188407373594	-0.242954618239392\\
27.5254596609156	-0.242956077560646\\
27.5320785296384	-0.242957535900507\\
27.5386973435656	-0.242958993259864\\
27.5453161027347	-0.242960449639602\\
27.5519348071832	-0.242961905040606\\
27.5585534569489	-0.242963359463761\\
27.565172052069	-0.242964812909949\\
27.5717905925812	-0.242966265380052\\
27.5784090785228	-0.242967716874952\\
27.5850275099312	-0.242969167395528\\
27.5916458868439	-0.24297061694266\\
27.5982642092981	-0.242972065517225\\
27.6048824773311	-0.2429735131201\\
27.6115006909802	-0.242974959752162\\
27.6181188502826	-0.242976405414286\\
27.6247369552756	-0.242977850107345\\
27.6313550059962	-0.242979293832212\\
27.6379730024816	-0.24298073658976\\
27.6445909447689	-0.242982178380859\\
27.6512088328952	-0.24298361920638\\
27.6578266668974	-0.242985059067192\\
27.6644444468125	-0.242986497964163\\
27.6710621726776	-0.242987935898159\\
27.6776798445295	-0.242989372870048\\
27.6842974624051	-0.242990808880694\\
27.6909150263412	-0.242992243930961\\
27.6975325363747	-0.242993678021712\\
27.7041499925424	-0.242995111153811\\
27.710767394881	-0.242996543328117\\
27.7173847434272	-0.242997974545492\\
27.7240020382177	-0.242999404806794\\
27.7306192792892	-0.243000834112882\\
27.7372364666782	-0.243002262464613\\
27.7438536004214	-0.243003689862843\\
27.7504706805552	-0.243005116308429\\
27.7570877071163	-0.243006541802224\\
27.763704680141	-0.243007966345082\\
27.7703215996659	-0.243009389937855\\
27.7769384657273	-0.243010812581395\\
27.7835552783617	-0.243012234276552\\
27.7901720376053	-0.243013655024177\\
27.7967887434945	-0.243015074825117\\
27.8034053960655	-0.243016493680221\\
27.8100219953546	-0.243017911590335\\
27.8166385413981	-0.243019328556305\\
27.8232550342321	-0.243020744578976\\
27.8298714738926	-0.243022159659191\\
27.836487860416	-0.243023573797794\\
27.8431041938382	-0.243024986995627\\
27.8497204741952	-0.243026399253531\\
27.8563367015232	-0.243027810572345\\
27.862952875858	-0.24302922095291\\
27.8695689972356	-0.243030630396062\\
27.876185065692	-0.24303203890264\\
27.882801081263	-0.24303344647348\\
27.8894170439844	-0.243034853109417\\
27.8960329538921	-0.243036258811285\\
27.9026488110219	-0.243037663579918\\
27.9092646154095	-0.243039067416149\\
27.9158803670907	-0.243040470320809\\
27.922496066101	-0.243041872294729\\
27.9291117124762	-0.243043273338739\\
27.935727306252	-0.243044673453667\\
27.9423428474638	-0.243046072640342\\
27.9489583361472	-0.24304747089959\\
27.9555737723378	-0.243048868232237\\
27.962189156071	-0.243050264639109\\
27.9688044873823	-0.24305166012103\\
27.9754197663071	-0.243053054678822\\
27.9820349928809	-0.243054448313309\\
27.988650167139	-0.243055841025312\\
27.9952652891166	-0.243057232815651\\
28.0018803588492	-0.243058623685146\\
28.0084953763719	-0.243060013634615\\
28.0151103417201	-0.243061402664876\\
28.0217252549288	-0.243062790776747\\
28.0283401160333	-0.243064177971042\\
28.0349549250687	-0.243065564248578\\
28.0415696820702	-0.243066949610167\\
28.0481843870727	-0.243068334056624\\
28.0547990401113	-0.24306971758876\\
28.061413641221	-0.243071100207388\\
28.0680281904369	-0.243072481913316\\
28.0746426877938	-0.243073862707355\\
28.0812571333267	-0.243075242590314\\
28.0878715270704	-0.243076621563\\
28.0944858690598	-0.243077999626221\\
28.1011001593298	-0.243079376780781\\
28.107714397915	-0.243080753027486\\
28.1143285848502	-0.243082128367141\\
28.1209427201703	-0.243083502800548\\
28.1275568039098	-0.24308487632851\\
28.1341708361034	-0.243086248951828\\
28.1407848167858	-0.243087620671303\\
28.1473987459916	-0.243088991487735\\
28.1540126237552	-0.243090361401922\\
28.1606264501114	-0.243091730414662\\
28.1672402250944	-0.243093098526752\\
28.1738539487389	-0.243094465738989\\
28.1804676210793	-0.243095832052167\\
28.1870812421499	-0.243097197467081\\
28.1936948119852	-0.243098561984524\\
28.2003083306195	-0.243099925605289\\
28.2069217980871	-0.243101288330168\\
28.2135352144223	-0.243102650159951\\
28.2201485796594	-0.243104011095428\\
28.2267618938326	-0.243105371137389\\
28.233375156976	-0.243106730286621\\
28.2399883691239	-0.243108088543912\\
28.2466015303103	-0.243109445910049\\
28.2532146405694	-0.243110802385816\\
28.2598276999352	-0.243112157971999\\
28.2664407084418	-0.243113512669381\\
28.2730536661232	-0.243114866478746\\
28.2796665730134	-0.243116219400875\\
28.2862794291462	-0.243117571436549\\
28.2928922345557	-0.243118922586549\\
28.2995049892756	-0.243120272851655\\
28.3061176933399	-0.243121622232645\\
28.3127303467824	-0.243122970730296\\
28.3193429496369	-0.243124318345386\\
28.3259555019371	-0.243125665078691\\
28.3325680037168	-0.243127010930986\\
28.3391804550096	-0.243128355903044\\
28.3457928558493	-0.24312969999564\\
28.3524052062695	-0.243131043209547\\
28.3590175063037	-0.243132385545535\\
28.3656297559857	-0.243133727004376\\
28.3722419553488	-0.24313506758684\\
28.3788541044267	-0.243136407293696\\
28.3854662032529	-0.243137746125713\\
28.3920782518607	-0.243139084083657\\
28.3986902502837	-0.243140421168295\\
28.4053021985552	-0.243141757380394\\
28.4119140967086	-0.243143092720718\\
28.4185259447773	-0.243144427190032\\
28.4251377427945	-0.243145760789098\\
28.4317494907935	-0.243147093518679\\
28.4383611888076	-0.243148425379537\\
28.44497283687	-0.243149756372431\\
28.4515844350139	-0.243151086498124\\
28.4581959832724	-0.243152415757372\\
28.4648074816788	-0.243153744150935\\
28.471418930266	-0.243155071679569\\
28.4780303290672	-0.243156398344032\\
28.4846416781153	-0.243157724145079\\
28.4912529774435	-0.243159049083466\\
28.4978642270847	-0.243160373159945\\
28.5044754270719	-0.24316169637527\\
28.5110865774379	-0.243163018730194\\
28.5176976782157	-0.243164340225469\\
28.5243087294381	-0.243165660861844\\
28.530919731138	-0.24316698064007\\
28.5375306833482	-0.243168299560896\\
28.5441415861014	-0.243169617625071\\
28.5507524394304	-0.243170934833341\\
28.557363243368	-0.243172251186454\\
28.5639739979467	-0.243173566685154\\
28.5705847031994	-0.243174881330188\\
28.5771953591585	-0.243176195122299\\
28.5838059658568	-0.243177508062231\\
28.5904165233267	-0.243178820150725\\
28.5970270316009	-0.243180131388525\\
28.6036374907118	-0.24318144177637\\
28.6102479006919	-0.243182751315002\\
28.6168582615737	-0.243184060005158\\
28.6234685733896	-0.243185367847578\\
28.630078836172	-0.243186674843\\
28.6366890499533	-0.243187980992159\\
28.6432992147658	-0.243189286295793\\
28.6499093306419	-0.243190590754636\\
28.6565193976137	-0.243191894369423\\
28.6631294157137	-0.243193197140887\\
28.6697393849739	-0.243194499069762\\
28.6763493054267	-0.243195800156779\\
28.6829591771041	-0.24319710040267\\
28.6895690000383	-0.243198399808165\\
28.6961787742614	-0.243199698373993\\
28.7027884998055	-0.243200996100884\\
28.7093981767027	-0.243202292989565\\
28.716007804985	-0.243203589040764\\
28.7226173846843	-0.243204884255208\\
28.7292269158327	-0.243206178633621\\
28.7358363984621	-0.243207472176728\\
28.7424458326044	-0.243208764885255\\
28.7490552182915	-0.243210056759923\\
28.7556645555553	-0.243211347801456\\
28.7622738444275	-0.243212638010575\\
28.76888308494	-0.243213927388002\\
28.7754922771245	-0.243215215934455\\
28.7821014210128	-0.243216503650656\\
28.7887105166366	-0.243217790537321\\
28.7953195640277	-0.243219076595169\\
28.8019285632175	-0.243220361824917\\
28.8085375142379	-0.243221646227281\\
28.8151464171203	-0.243222929802977\\
28.8217552718964	-0.243224212552719\\
28.8283640785977	-0.243225494477221\\
28.8349728372557	-0.243226775577197\\
28.841581547902	-0.243228055853357\\
28.8481902105679	-0.243229335306415\\
28.854798825285	-0.243230613937081\\
28.8614073920846	-0.243231891746064\\
28.868015910998	-0.243233168734075\\
28.8746243820568	-0.243234444901821\\
28.8812328052921	-0.24323572025001\\
28.8878411807353	-0.243236994779349\\
28.8944495084177	-0.243238268490545\\
28.9010577883704	-0.243239541384302\\
28.9076660206248	-0.243240813461325\\
28.9142742052119	-0.243242084722319\\
28.920882342163	-0.243243355167986\\
28.9274904315092	-0.243244624799028\\
28.9340984732816	-0.243245893616147\\
28.9407064675112	-0.243247161620045\\
28.9473144142292	-0.24324842881142\\
28.9539223134665	-0.243249695190972\\
28.9605301652541	-0.243250960759401\\
28.967137969623	-0.243252225517402\\
28.9737457266042	-0.243253489465675\\
28.9803534362286	-0.243254752604914\\
28.9869610985269	-0.243256014935816\\
28.9935687135302	-0.243257276459074\\
29.0001762812692	-0.243258537175384\\
29.0067838017747	-0.243259797085438\\
29.0133912750776	-0.243261056189929\\
29.0199987012085	-0.243262314489548\\
29.0266060801982	-0.243263571984987\\
29.0332134120774	-0.243264828676937\\
29.0398206968767	-0.243266084566085\\
29.0464279346268	-0.243267339653121\\
29.0530351253584	-0.243268593938734\\
29.0596422691019	-0.24326984742361\\
29.066249365888	-0.243271100108437\\
29.0728564157472	-0.243272351993899\\
29.07946341871	-0.243273603080681\\
29.0860703748069	-0.24327485336947\\
29.0926772840684	-0.243276102860946\\
29.0992841465248	-0.243277351555795\\
29.1058909622066	-0.243278599454697\\
29.1124977311441	-0.243279846558334\\
29.1191044533677	-0.243281092867387\\
29.1257111289077	-0.243282338382536\\
29.1323177577945	-0.243283583104459\\
29.1389243400582	-0.243284827033836\\
29.1455308757292	-0.243286070171344\\
29.1521373648376	-0.243287312517659\\
29.1587438074136	-0.243288554073459\\
29.1653502034874	-0.243289794839419\\
29.1719565530891	-0.243291034816213\\
29.1785628562489	-0.243292274004516\\
29.1851691129968	-0.243293512405\\
29.1917753233629	-0.24329475001834\\
29.1983814873771	-0.243295986845205\\
29.2049876050696	-0.243297222886269\\
29.2115936764703	-0.2432984581422\\
29.2181997016092	-0.24329969261367\\
29.2248056805161	-0.243300926301346\\
29.231411613221	-0.243302159205897\\
29.2380174997537	-0.243303391327991\\
29.2446233401441	-0.243304622668295\\
29.2512291344221	-0.243305853227475\\
29.2578348826173	-0.243307083006196\\
29.2644405847597	-0.243308312005122\\
29.2710462408789	-0.243309540224919\\
29.2776518510046	-0.243310767666249\\
29.2842574151666	-0.243311994329775\\
29.2908629333944	-0.243313220216158\\
29.2974684057179	-0.243314445326061\\
29.3040738321665	-0.243315669660143\\
29.3106792127698	-0.243316893219064\\
29.3172845475575	-0.243318116003483\\
29.323889836559	-0.243319338014058\\
29.3304950798039	-0.243320559251448\\
29.3371002773216	-0.243321779716309\\
29.3437054291417	-0.243322999409297\\
29.3503105352935	-0.243324218331068\\
29.3569155958065	-0.243325436482276\\
29.36352061071	-0.243326653863577\\
29.3701255800334	-0.243327870475622\\
29.376730503806	-0.243329086319066\\
29.3833353820572	-0.24333030139456\\
29.3899402148163	-0.243331515702755\\
29.3965450021124	-0.243332729244302\\
29.4031497439748	-0.243333942019851\\
29.4097544404328	-0.243335154030051\\
29.4163590915154	-0.243336365275551\\
29.422963697252	-0.243337575756998\\
29.4295682576715	-0.24333878547504\\
29.4361727728031	-0.243339994430324\\
29.4427772426759	-0.243341202623495\\
29.4493816673189	-0.243342410055198\\
29.4559860467611	-0.243343616726077\\
29.4625903810317	-0.243344822636777\\
29.4691946701595	-0.243346027787941\\
29.4757989141734	-0.24334723218021\\
29.4824031131026	-0.243348435814227\\
29.4890072669757	-0.243349638690632\\
29.4956113758218	-0.243350840810066\\
29.5022154396696	-0.243352042173169\\
29.508819458548	-0.243353242780579\\
29.5154234324858	-0.243354442632934\\
29.5220273615118	-0.243355641730873\\
29.5286312456547	-0.243356840075033\\
29.5352350849433	-0.243358037666048\\
29.5418388794063	-0.243359234504556\\
29.5484426290723	-0.243360430591191\\
29.5550463339701	-0.243361625926587\\
29.5616499941281	-0.243362820511377\\
29.5682536095751	-0.243364014346195\\
29.5748571803397	-0.243365207431673\\
29.5814607064503	-0.243366399768442\\
29.5880641879355	-0.243367591357134\\
29.5946676248239	-0.243368782198378\\
29.6012710171438	-0.243369972292804\\
29.6078743649238	-0.24337116164104\\
29.6144776681924	-0.243372350243716\\
29.6210809269778	-0.243373538101458\\
29.6276841413085	-0.243374725214893\\
29.6342873112129	-0.243375911584648\\
29.6408904367193	-0.243377097211348\\
29.647493517856	-0.243378282095618\\
29.6540965546512	-0.243379466238082\\
29.6606995471334	-0.243380649639364\\
29.6673024953306	-0.243381832300086\\
29.6739053992711	-0.243383014220872\\
29.6805082589832	-0.243384195402341\\
29.6871110744949	-0.243385375845117\\
29.6937138458344	-0.243386555549817\\
29.7003165730298	-0.243387734517063\\
29.7069192561093	-0.243388912747473\\
29.7135218951008	-0.243390090241666\\
29.7201244900325	-0.243391267000259\\
29.7267270409324	-0.243392443023868\\
29.7333295478284	-0.243393618313111\\
29.7399320107485	-0.243394792868604\\
29.7465344297207	-0.24339596669096\\
29.7531368047729	-0.243397139780795\\
29.7597391359331	-0.243398312138723\\
29.766341423229	-0.243399483765355\\
29.7729436666886	-0.243400654661306\\
29.7795458663396	-0.243401824827186\\
29.7861480222099	-0.243402994263607\\
29.7927501343273	-0.24340416297118\\
29.7993522027194	-0.243405330950513\\
29.8059542274142	-0.243406498202218\\
29.8125562084392	-0.243407664726901\\
29.8191581458222	-0.243408830525171\\
29.8257600395907	-0.243409995597635\\
29.8323618897726	-0.2434111599449\\
29.8389636963953	-0.243412323567573\\
29.8455654594865	-0.243413486466257\\
29.8521671790737	-0.243414648641559\\
29.8587688551845	-0.243415810094082\\
29.8653704878464	-0.24341697082443\\
29.871972077087	-0.243418130833206\\
29.8785736229336	-0.243419290121011\\
29.8851751254137	-0.243420448688448\\
29.8917765845548	-0.243421606536117\\
29.8983780003843	-0.243422763664618\\
29.9049793729295	-0.243423920074553\\
29.9115807022178	-0.243425075766518\\
29.9181819882766	-0.243426230741114\\
29.9247832311331	-0.243427384998937\\
29.9313844308147	-0.243428538540585\\
29.9379855873486	-0.243429691366654\\
29.9445867007621	-0.243430843477741\\
29.9511877710823	-0.243431994874441\\
29.9577887983365	-0.243433145557348\\
29.9643897825518	-0.243434295527056\\
29.9709907237555	-0.243435444784159\\
29.9775916219745	-0.24343659332925\\
29.9841924772361	-0.243437741162921\\
29.9907932895673	-0.243438888285763\\
29.9973940589951	-0.243440034698368\\
30.0039947855467	-0.243441180401325\\
30.0105954692489	-0.243442325395225\\
30.0171961101289	-0.243443469680656\\
30.0237967082135	-0.243444613258208\\
30.0303972635297	-0.243445756128467\\
30.0369977761045	-0.243446898292022\\
30.0435982459647	-0.243448039749458\\
30.0501986731371	-0.243449180501362\\
30.0567990576488	-0.24345032054832\\
};
\addplot [color=mycolor1,solid,forget plot]
  table[row sep=crcr]{%
30.0567990576488	-0.24345032054832\\
30.0633993995264	-0.243451459890915\\
30.0699996987968	-0.243452598529733\\
30.0765999554867	-0.243453736465357\\
30.083200169623	-0.24345487369837\\
30.0898003412324	-0.243456010229354\\
30.0964004703415	-0.243457146058891\\
30.1030005569772	-0.243458281187562\\
30.1096006011659	-0.243459415615947\\
30.1162006029345	-0.243460549344628\\
30.1228005623095	-0.243461682374182\\
30.1294004793175	-0.243462814705189\\
30.1360003539851	-0.243463946338227\\
30.1426001863389	-0.243465077273874\\
30.1491999764054	-0.243466207512705\\
30.155799724211	-0.243467337055299\\
30.1623994297824	-0.24346846590223\\
30.168999093146	-0.243469594054073\\
30.1755987143281	-0.243470721511403\\
30.1821982933553	-0.243471848274795\\
30.188797830254	-0.243472974344821\\
30.1953973250504	-0.243474099722054\\
30.201996777771	-0.243475224407066\\
30.2085961884422	-0.243476348400429\\
30.2151955570901	-0.243477471702713\\
30.2217948837412	-0.24347859431449\\
30.2283941684216	-0.243479716236329\\
30.2349934111577	-0.243480837468798\\
30.2415926119756	-0.243481958012468\\
30.2481917709015	-0.243483077867905\\
30.2547908879617	-0.243484197035677\\
30.2613899631823	-0.243485315516351\\
30.2679889965893	-0.243486433310494\\
30.274587988209	-0.24348755041867\\
30.2811869380675	-0.243488666841445\\
30.2877858461907	-0.243489782579383\\
30.2943847126048	-0.243490897633048\\
30.3009835373357	-0.243492012003004\\
30.3075823204095	-0.243493125689814\\
30.3141810618522	-0.243494238694039\\
30.3207797616896	-0.24349535101624\\
30.3273784199478	-0.24349646265698\\
30.3339770366527	-0.243497573616818\\
30.3405756118301	-0.243498683896314\\
30.347174145506	-0.243499793496028\\
30.3537726377061	-0.243500902416517\\
30.3603710884564	-0.243502010658341\\
30.3669694977826	-0.243503118222057\\
30.3735678657105	-0.243504225108221\\
30.3801661922658	-0.24350533131739\\
30.3867644774744	-0.243506436850119\\
30.3933627213619	-0.243507541706965\\
30.399960923954	-0.243508645888481\\
30.4065590852764	-0.243509749395222\\
30.4131572053549	-0.243510852227741\\
30.4197552842149	-0.243511954386591\\
30.4263533218821	-0.243513055872324\\
30.4329513183821	-0.243514156685492\\
30.4395492737405	-0.243515256826647\\
30.4461471879829	-0.243516356296339\\
30.4527450611347	-0.243517455095118\\
30.4593428932215	-0.243518553223533\\
30.4659406842687	-0.243519650682134\\
30.4725384343018	-0.243520747471469\\
30.4791361433463	-0.243521843592085\\
30.4857338114275	-0.243522939044531\\
30.492331438571	-0.243524033829351\\
30.4989290248019	-0.243525127947094\\
30.5055265701458	-0.243526221398304\\
30.512124074628	-0.243527314183526\\
30.5187215382737	-0.243528406303305\\
30.5253189611083	-0.243529497758185\\
30.531916343157	-0.243530588548708\\
30.5385136844451	-0.243531678675418\\
30.5451109849978	-0.243532768138857\\
30.5517082448403	-0.243533856939567\\
30.5583054639978	-0.243534945078088\\
30.5649026424955	-0.243536032554961\\
30.5714997803586	-0.243537119370726\\
30.5780968776121	-0.243538205525922\\
30.5846939342812	-0.243539291021089\\
30.5912909503909	-0.243540375856765\\
30.5978879259663	-0.243541460033487\\
30.6044848610324	-0.243542543551792\\
30.6110817556144	-0.243543626412218\\
30.6176786097371	-0.243544708615301\\
30.6242754234255	-0.243545790161575\\
30.6308721967047	-0.243546871051576\\
30.6374689295995	-0.243547951285839\\
30.6440656221349	-0.243549030864897\\
30.6506622743357	-0.243550109789284\\
30.6572588862269	-0.243551188059532\\
30.6638554578333	-0.243552265676174\\
30.6704519891797	-0.243553342639741\\
30.6770484802909	-0.243554418950765\\
30.6836449311918	-0.243555494609777\\
30.6902413419071	-0.243556569617306\\
30.6968377124616	-0.243557643973882\\
30.7034340428799	-0.243558717680034\\
30.7100303331869	-0.24355979073629\\
30.7166265834072	-0.243560863143179\\
30.7232227935654	-0.243561934901227\\
30.7298189636862	-0.243563006010963\\
30.7364150937943	-0.243564076472911\\
30.7430111839143	-0.243565146287598\\
30.7496072340707	-0.243566215455549\\
30.7562032442881	-0.243567283977289\\
30.762799214591	-0.243568351853341\\
30.7693951450041	-0.24356941908423\\
30.7759910355517	-0.243570485670479\\
30.7825868862585	-0.24357155161261\\
30.7891826971488	-0.243572616911145\\
30.7957784682471	-0.243573681566606\\
30.8023741995778	-0.243574745579513\\
30.8089698911653	-0.243575808950388\\
30.8155655430341	-0.243576871679749\\
30.8221611552084	-0.243577933768117\\
30.8287567277127	-0.24357899521601\\
30.8353522605712	-0.243580056023947\\
30.8419477538083	-0.243581116192445\\
30.8485432074482	-0.243582175722022\\
30.8551386215152	-0.243583234613194\\
30.8617339960336	-0.243584292866478\\
30.8683293310276	-0.243585350482389\\
30.8749246265213	-0.243586407461443\\
30.8815198825391	-0.243587463804154\\
30.8881150991049	-0.243588519511036\\
30.8947102762431	-0.243589574582602\\
30.9013054139776	-0.243590629019367\\
30.9079005123327	-0.243591682821842\\
30.9144955713323	-0.24359273599054\\
30.9210905910007	-0.243593788525971\\
30.9276855713617	-0.243594840428648\\
30.9342805124395	-0.24359589169908\\
30.940875414258	-0.243596942337777\\
30.9474702768413	-0.243597992345249\\
30.9540651002132	-0.243599041722005\\
30.9606598843978	-0.243600090468554\\
30.967254629419	-0.243601138585402\\
30.9738493353006	-0.243602186073058\\
30.9804440020666	-0.243603232932028\\
30.9870386297409	-0.243604279162818\\
30.9936332183472	-0.243605324765936\\
31.0002277679095	-0.243606369741885\\
31.0068222784515	-0.24360741409117\\
31.013416749997	-0.243608457814297\\
31.0200111825698	-0.243609500911768\\
31.0266055761936	-0.243610543384088\\
31.0331999308922	-0.243611585231758\\
31.0397942466893	-0.243612626455281\\
31.0463885236086	-0.243613667055159\\
31.0529827616737	-0.243614707031893\\
31.0595769609083	-0.243615746385983\\
31.066171121336	-0.243616785117931\\
31.0727652429806	-0.243617823228236\\
31.0793593258654	-0.243618860717396\\
31.0859533700142	-0.243619897585911\\
31.0925473754504	-0.243620933834279\\
31.0991413421977	-0.243621969462998\\
31.1057352702795	-0.243623004472564\\
31.1123291597194	-0.243624038863475\\
31.1189230105408	-0.243625072636227\\
31.1255168227671	-0.243626105791315\\
31.1321105964219	-0.243627138329234\\
31.1387043315285	-0.243628170250481\\
31.1452980281104	-0.243629201555547\\
31.1518916861909	-0.243630232244928\\
31.1584853057934	-0.243631262319116\\
31.1650788869413	-0.243632291778605\\
31.1716724296578	-0.243633320623886\\
31.1782659339663	-0.243634348855451\\
31.18485939989	-0.243635376473791\\
31.1914528274523	-0.243636403479398\\
31.1980462166764	-0.243637429872761\\
31.2046395675855	-0.243638455654369\\
31.2112328802028	-0.243639480824714\\
31.2178261545515	-0.243640505384282\\
31.2244193906549	-0.243641529333563\\
31.231012588536	-0.243642552673043\\
31.237605748218	-0.243643575403212\\
31.244198869724	-0.243644597524554\\
31.2507919530771	-0.243645619037557\\
31.2573849983005	-0.243646639942706\\
31.263978005417	-0.243647660240487\\
31.2705709744499	-0.243648679931385\\
31.2771639054221	-0.243649699015883\\
31.2837567983566	-0.243650717494467\\
31.2903496532764	-0.243651735367618\\
31.2969424702046	-0.24365275263582\\
31.3035352491639	-0.243653769299556\\
31.3101279901774	-0.243654785359307\\
31.316720693268	-0.243655800815555\\
31.3233133584586	-0.243656815668781\\
31.329905985772	-0.243657829919466\\
31.336498575231	-0.243658843568088\\
31.3430911268587	-0.243659856615128\\
31.3496836406777	-0.243660869061065\\
31.3562761167108	-0.243661880906378\\
31.3628685549809	-0.243662892151544\\
31.3694609555107	-0.243663902797041\\
31.3760533183229	-0.243664912843347\\
31.3826456434403	-0.243665922290937\\
31.3892379308856	-0.243666931140289\\
31.3958301806815	-0.243667939391877\\
31.4024223928506	-0.243668947046177\\
31.4090145674156	-0.243669954103664\\
31.4156067043991	-0.243670960564812\\
31.4221988038238	-0.243671966430094\\
31.4287908657123	-0.243672971699985\\
31.435382890087	-0.243673976374957\\
31.4419748769707	-0.243674980455482\\
31.4485668263857	-0.243675983942032\\
31.4551587383548	-0.243676986835079\\
31.4617506129002	-0.243677989135094\\
31.4683424500447	-0.243678990842546\\
31.4749342498105	-0.243679991957907\\
31.4815260122203	-0.243680992481646\\
31.4881177372963	-0.243681992414231\\
31.4947094250611	-0.243682991756132\\
31.501301075537	-0.243683990507816\\
31.5078926887463	-0.243684988669752\\
31.5144842647116	-0.243685986242406\\
31.521075803455	-0.243686983226246\\
31.527667304999	-0.243687979621736\\
31.5342587693658	-0.243688975429345\\
31.5408501965777	-0.243689970649536\\
31.547441586657	-0.243690965282774\\
31.554032939626	-0.243691959329525\\
31.5606242555069	-0.243692952790251\\
31.5672155343218	-0.243693945665418\\
31.5738067760931	-0.243694937955486\\
31.5803979808429	-0.24369592966092\\
31.5869891485933	-0.243696920782181\\
31.5935802793664	-0.243697911319731\\
31.6001713731846	-0.243698901274032\\
31.6067624300697	-0.243699890645543\\
31.613353450044	-0.243700879434725\\
31.6199444331294	-0.243701867642039\\
31.6265353793481	-0.243702855267943\\
31.6331262887221	-0.243703842312896\\
31.6397171612735	-0.243704828777357\\
31.6463079970241	-0.243705814661784\\
31.652898795996	-0.243706799966635\\
31.6594895582113	-0.243707784692366\\
31.6660802836917	-0.243708768839434\\
31.6726709724593	-0.243709752408295\\
31.679261624536	-0.243710735399406\\
31.6858522399436	-0.243711717813221\\
31.6924428187041	-0.243712699650195\\
31.6990333608392	-0.243713680910783\\
31.705623866371	-0.243714661595438\\
31.7122143353211	-0.243715641704615\\
31.7188047677114	-0.243716621238766\\
31.7253951635637	-0.243717600198343\\
31.7319855228998	-0.243718578583799\\
31.7385758457414	-0.243719556395586\\
31.7451661321102	-0.243720533634155\\
31.751756382028	-0.243721510299956\\
31.7583465955165	-0.243722486393441\\
31.7649367725974	-0.243723461915058\\
31.7715269132923	-0.243724436865258\\
31.7781170176229	-0.243725411244489\\
31.7847070856108	-0.2437263850532\\
31.7912971172777	-0.243727358291839\\
31.7978871126451	-0.243728330960854\\
31.8044770717346	-0.243729303060692\\
31.8110669945679	-0.2437302745918\\
31.8176568811664	-0.243731245554625\\
31.8242467315517	-0.243732215949611\\
31.8308365457452	-0.243733185777206\\
31.8374263237686	-0.243734155037853\\
31.8440160656432	-0.243735123731997\\
31.8506057713906	-0.243736091860083\\
31.8571954410321	-0.243737059422554\\
31.8637850745893	-0.243738026419854\\
31.8703746720834	-0.243738992852426\\
31.876964233536	-0.243739958720711\\
31.8835537589684	-0.243740924025152\\
31.8901432484019	-0.24374188876619\\
31.8967327018579	-0.243742852944267\\
31.9033221193578	-0.243743816559824\\
31.9099115009227	-0.243744779613299\\
31.9165008465741	-0.243745742105135\\
31.9230901563332	-0.243746704035768\\
31.9296794302212	-0.24374766540564\\
31.9362686682595	-0.243748626215187\\
31.9428578704691	-0.243749586464849\\
31.9494470368714	-0.243750546155063\\
31.9560361674875	-0.243751505286266\\
31.9626252623386	-0.243752463858895\\
31.9692143214458	-0.243753421873386\\
31.9758033448303	-0.243754379330175\\
31.9823923325133	-0.243755336229698\\
31.9889812845157	-0.243756292572391\\
31.9955702008588	-0.243757248358686\\
32.0021590815635	-0.24375820358902\\
32.008747926651	-0.243759158263824\\
32.0153367361422	-0.243760112383534\\
32.0219255100583	-0.243761065948582\\
32.0285142484201	-0.2437620189594\\
32.0351029512488	-0.243762971416421\\
32.0416916185652	-0.243763923320076\\
32.0482802503903	-0.243764874670797\\
32.0548688467451	-0.243765825469014\\
32.0614574076504	-0.243766775715157\\
32.0680459331273	-0.243767725409657\\
32.0746344231965	-0.243768674552944\\
32.081222877879	-0.243769623145446\\
32.0878112971955	-0.243770571187592\\
32.094399681167	-0.243771518679811\\
32.1009880298143	-0.24377246562253\\
32.1075763431582	-0.243773412016178\\
32.1141646212194	-0.24377435786118\\
32.1207528640187	-0.243775303157963\\
32.127341071577	-0.243776247906955\\
32.1339292439149	-0.24377719210858\\
32.1405173810531	-0.243778135763264\\
32.1471054830124	-0.243779078871431\\
32.1536935498135	-0.243780021433508\\
32.160281581477	-0.243780963449917\\
32.1668695780235	-0.243781904921083\\
32.1734575394739	-0.243782845847428\\
32.1800454658485	-0.243783786229376\\
32.1866333571682	-0.243784726067349\\
32.1932212134534	-0.243785665361769\\
32.1998090347247	-0.243786604113058\\
32.2063968210027	-0.243787542321637\\
32.212984572308	-0.243788479987927\\
32.2195722886611	-0.243789417112349\\
32.2261599700824	-0.243790353695321\\
32.2327476165925	-0.243791289737265\\
32.2393352282119	-0.243792225238599\\
32.2459228049609	-0.243793160199742\\
32.2525103468602	-0.243794094621113\\
32.2590978539301	-0.243795028503129\\
32.2656853261909	-0.243795961846208\\
32.2722727636632	-0.243796894650767\\
32.2788601663673	-0.243797826917222\\
32.2854475343235	-0.243798758645992\\
32.2920348675523	-0.24379968983749\\
32.2986221660739	-0.243800620492133\\
32.3052094299087	-0.243801550610336\\
32.3117966590769	-0.243802480192514\\
32.3183838535989	-0.24380340923908\\
32.3249710134949	-0.24380433775045\\
32.3315581387852	-0.243805265727035\\
32.33814522949	-0.243806193169251\\
32.3447322856296	-0.243807120077509\\
32.3513193072241	-0.243808046452221\\
32.3579062942938	-0.2438089722938\\
32.3644932468587	-0.243809897602657\\
32.3710801649392	-0.243810822379203\\
32.3776670485552	-0.243811746623849\\
32.384253897727	-0.243812670337005\\
32.3908407124747	-0.243813593519082\\
32.3974274928183	-0.243814516170488\\
32.404014238778	-0.243815438291633\\
32.4106009503737	-0.243816359882926\\
32.4171876276256	-0.243817280944774\\
32.4237742705537	-0.243818201477587\\
32.4303608791779	-0.243819121481771\\
32.4369474535184	-0.243820040957733\\
32.443533993595	-0.243820959905881\\
32.4501204994278	-0.243821878326621\\
32.4567069710367	-0.243822796220358\\
32.4632934084417	-0.243823713587499\\
32.4698798116626	-0.243824630428448\\
32.4764661807194	-0.243825546743611\\
32.4830525156319	-0.243826462533391\\
32.4896388164201	-0.243827377798193\\
32.4962250831039	-0.24382829253842\\
32.5028113157029	-0.243829206754476\\
32.5093975142371	-0.243830120446763\\
32.5159836787264	-0.243831033615684\\
32.5225698091904	-0.243831946261642\\
32.529155905649	-0.243832858385036\\
32.5357419681219	-0.24383376998627\\
32.5423279966289	-0.243834681065744\\
32.5489139911898	-0.243835591623857\\
32.5554999518242	-0.243836501661012\\
32.5620858785519	-0.243837411177606\\
32.5686717713925	-0.243838320174041\\
32.5752576303657	-0.243839228650713\\
32.5818434554912	-0.243840136608023\\
32.5884292467887	-0.243841044046369\\
32.5950150042777	-0.243841950966147\\
32.6016007279778	-0.243842857367756\\
32.6081864179088	-0.243843763251593\\
32.6147720740901	-0.243844668618054\\
32.6213576965413	-0.243845573467536\\
32.6279432852819	-0.243846477800435\\
32.6345288403316	-0.243847381617145\\
32.6411143617098	-0.243848284918063\\
32.6476998494361	-0.243849187703583\\
32.6542853035298	-0.243850089974099\\
32.6608707240106	-0.243850991730005\\
32.6674561108978	-0.243851892971696\\
32.674041464211	-0.243852793699564\\
32.6806267839695	-0.243853693914002\\
32.6872120701927	-0.243854593615402\\
32.6937973229001	-0.243855492804158\\
32.7003825421111	-0.24385639148066\\
32.7069677278449	-0.243857289645299\\
32.713552880121	-0.243858187298468\\
32.7201379989588	-0.243859084440556\\
32.7267230843774	-0.243859981071954\\
32.7333081363964	-0.243860877193052\\
32.7398931550348	-0.243861772804238\\
32.7464781403121	-0.243862667905903\\
32.7530630922474	-0.243863562498434\\
32.7596480108601	-0.243864456582221\\
32.7662328961693	-0.243865350157651\\
32.7728177481944	-0.243866243225112\\
32.7794025669544	-0.243867135784992\\
32.7859873524687	-0.243868027837676\\
32.7925721047562	-0.243868919383552\\
32.7991568238363	-0.243869810423005\\
32.8057415097281	-0.243870700956422\\
32.8123261624507	-0.243871590984188\\
32.8189107820232	-0.243872480506688\\
32.8254953684647	-0.243873369524307\\
32.8320799217943	-0.243874258037428\\
32.838664442031	-0.243875146046436\\
32.845248929194	-0.243876033551715\\
32.8518333833023	-0.243876920553647\\
32.8584178043749	-0.243877807052615\\
32.8650021924308	-0.243878693049002\\
32.871586547489	-0.24387957854319\\
32.8781708695684	-0.243880463535561\\
32.8847551586882	-0.243881348026496\\
32.8913394148672	-0.243882232016376\\
32.8979236381243	-0.243883115505582\\
32.9045078284785	-0.243883998494494\\
32.9110919859487	-0.243884880983491\\
32.9176761105537	-0.243885762972954\\
32.9242602023126	-0.243886644463262\\
32.930844261244	-0.243887525454793\\
32.937428287367	-0.243888405947925\\
32.9440122807002	-0.243889285943038\\
32.9505962412626	-0.243890165440509\\
32.957180169073	-0.243891044440715\\
32.9637640641501	-0.243891922944032\\
32.9703479265127	-0.243892800950839\\
32.9769317561796	-0.243893678461511\\
32.9835155531695	-0.243894555476424\\
32.9900993175012	-0.243895431995953\\
32.9966830491934	-0.243896308020475\\
33.0032667482648	-0.243897183550363\\
33.0098504147341	-0.243898058585992\\
33.0164340486199	-0.243898933127737\\
33.023017649941	-0.243899807175971\\
33.029601218716	-0.243900680731068\\
33.0361847549634	-0.2439015537934\\
33.042768258702	-0.243902426363341\\
33.0493517299504	-0.243903298441263\\
33.055935168727	-0.243904170027538\\
33.0625185750506	-0.243905041122537\\
33.0691019489396	-0.243905911726632\\
33.0756852904127	-0.243906781840195\\
33.0822685994883	-0.243907651463594\\
33.088851876185	-0.243908520597202\\
33.0954351205212	-0.243909389241387\\
33.1020183325155	-0.243910257396519\\
33.1086015121864	-0.243911125062968\\
33.1151846595523	-0.243911992241102\\
33.1217677746316	-0.24391285893129\\
33.1283508574429	-0.2439137251339\\
33.1349339080044	-0.243914590849299\\
33.1415169263347	-0.243915456077856\\
33.1480999124521	-0.243916320819938\\
33.1546828663749	-0.24391718507591\\
33.1612657881216	-0.24391804884614\\
33.1678486777106	-0.243918912130994\\
33.17443153516	-0.243919774930837\\
33.1810143604884	-0.243920637246034\\
33.1875971537138	-0.243921499076952\\
33.1941799148548	-0.243922360423954\\
33.2007626439295	-0.243923221287405\\
33.2073453409562	-0.243924081667668\\
33.2139280059532	-0.243924941565108\\
33.2205106389386	-0.243925800980088\\
33.2270932399308	-0.24392665991297\\
33.2336758089479	-0.243927518364118\\
33.2402583460082	-0.243928376333894\\
33.2468408511297	-0.243929233822659\\
33.2534233243307	-0.243930090830775\\
33.2600057656294	-0.243930947358603\\
33.2665881750438	-0.243931803406505\\
33.2731705525921	-0.243932658974841\\
33.2797528982925	-0.243933514063971\\
33.2863352121629	-0.243934368674255\\
33.2929174942215	-0.243935222806053\\
33.2994997444864	-0.243936076459723\\
33.3060819629756	-0.243936929635625\\
33.3126641497072	-0.243937782334117\\
33.3192463046991	-0.243938634555558\\
33.3258284279695	-0.243939486300304\\
33.3324105195363	-0.243940337568715\\
33.3389925794175	-0.243941188361146\\
33.345574607631	-0.243942038677956\\
33.3521566041949	-0.243942888519499\\
33.3587385691271	-0.243943737886133\\
33.3653205024455	-0.243944586778214\\
33.371902404168	-0.243945435196096\\
33.3784842743125	-0.243946283140135\\
33.385066112897	-0.243947130610686\\
33.3916479199393	-0.243947977608103\\
33.3982296954573	-0.243948824132741\\
33.4048114394688	-0.243949670184953\\
33.4113931519916	-0.243950515765093\\
33.4179748330437	-0.243951360873514\\
33.4245564826428	-0.243952205510569\\
33.4311381008066	-0.24395304967661\\
33.437719687553	-0.243953893371989\\
33.4443012428998	-0.243954736597059\\
33.4508827668647	-0.243955579352171\\
33.4574642594654	-0.243956421637676\\
33.4640457207198	-0.243957263453925\\
33.4706271506454	-0.243958104801268\\
33.47720854926	-0.243958945680056\\
33.4837899165813	-0.243959786090638\\
33.490371252627	-0.243960626033365\\
33.4969525574147	-0.243961465508584\\
33.503533830962	-0.243962304516646\\
33.5101150732867	-0.243963143057898\\
33.5166962844063	-0.243963981132689\\
33.5232774643385	-0.243964818741367\\
33.5298586131007	-0.24396565588428\\
33.5364397307107	-0.243966492561774\\
33.543020817186	-0.243967328774196\\
33.5496018725442	-0.243968164521894\\
33.5561828968027	-0.243968999805213\\
33.5627638899791	-0.243969834624499\\
33.569344852091	-0.243970668980099\\
33.5759257831558	-0.243971502872356\\
33.582506683191	-0.243972336301617\\
33.5890875522141	-0.243973169268226\\
33.5956683902425	-0.243974001772527\\
33.6022491972938	-0.243974833814864\\
33.6088299733853	-0.243975665395582\\
33.6154107185345	-0.243976496515022\\
33.6219914327587	-0.243977327173529\\
33.6285721160755	-0.243978157371445\\
33.635152768502	-0.243978987109113\\
33.6417333900558	-0.243979816386874\\
33.6483139807542	-0.243980645205071\\
33.6548945406145	-0.243981473564044\\
33.6614750696541	-0.243982301464135\\
33.6680555678902	-0.243983128905685\\
33.6746360353402	-0.243983955889034\\
33.6812164720214	-0.243984782414523\\
33.687796877951	-0.243985608482491\\
33.6943772531463	-0.243986434093277\\
33.7009575976246	-0.243987259247222\\
33.7075379114031	-0.243988083944664\\
33.714118194499	-0.243988908185941\\
33.7206984469296	-0.243989731971392\\
33.727278668712	-0.243990555301354\\
33.7338588598635	-0.243991378176167\\
33.7404390204011	-0.243992200596165\\
33.7470191503421	-0.243993022561688\\
33.7535992497036	-0.243993844073071\\
33.7601793185028	-0.243994665130651\\
33.7667593567568	-0.243995485734764\\
33.7733393644826	-0.243996305885745\\
33.7799193416974	-0.243997125583931\\
33.7864992884183	-0.243997944829655\\
33.7930792046623	-0.243998763623254\\
33.7996590904465	-0.243999581965062\\
33.8062389457879	-0.244000399855412\\
33.8128187707036	-0.244001217294639\\
33.8193985652106	-0.244002034283077\\
33.8259783293259	-0.244002850821058\\
33.8325580630665	-0.244003666908915\\
33.8391377664493	-0.244004482546982\\
33.8457174394914	-0.24400529773559\\
33.8522970822097	-0.244006112475072\\
33.8588766946211	-0.244006926765759\\
33.8654562767426	-0.244007740607983\\
33.8720358285911	-0.244008554002074\\
33.8786153501835	-0.244009366948365\\
33.8851948415366	-0.244010179447184\\
33.8917743026674	-0.244010991498863\\
33.8983537335927	-0.24401180310373\\
33.9049331343295	-0.244012614262117\\
33.9115125048944	-0.244013424974352\\
33.9180918453044	-0.244014235240764\\
33.9246711555763	-0.244015045061681\\
33.9312504357268	-0.244015854437433\\
33.9378296857728	-0.244016663368346\\
33.9444089057311	-0.24401747185475\\
33.9509880956183	-0.244018279896971\\
33.9575672554513	-0.244019087495336\\
33.9641463852469	-0.244019894650173\\
33.9707254850216	-0.244020701361807\\
33.9773045547923	-0.244021507630566\\
33.9838835945757	-0.244022313456774\\
33.9904626043883	-0.244023118840759\\
33.997041584247	-0.244023923782844\\
34.0036205341684	-0.244024728283355\\
34.0101994541692	-0.244025532342617\\
34.0167783442659	-0.244026335960954\\
34.0233572044752	-0.244027139138691\\
34.0299360348137	-0.24402794187615\\
34.0365148352981	-0.244028744173657\\
34.0430936059448	-0.244029546031533\\
34.0496723467706	-0.244030347450102\\
34.0562510577919	-0.244031148429687\\
34.0628297390254	-0.244031948970609\\
34.0694083904875	-0.244032749073192\\
34.0759870121947	-0.244033548737756\\
34.0825656041637	-0.244034347964623\\
34.0891441664109	-0.244035146754114\\
34.0957226989528	-0.244035945106551\\
34.1023012018059	-0.244036743022253\\
34.1088796749866	-0.244037540501541\\
34.1154581185114	-0.244038337544735\\
34.1220365323967	-0.244039134152154\\
34.1286149166591	-0.244039930324119\\
34.1351932713148	-0.244040726060947\\
34.1417715963803	-0.244041521362959\\
34.148349891872	-0.244042316230471\\
34.1549281578063	-0.244043110663804\\
34.1615063941996	-0.244043904663273\\
34.1680846010681	-0.244044698229198\\
34.1746627784283	-0.244045491361895\\
34.1812409262964	-0.244046284061682\\
34.1878190446889	-0.244047076328875\\
34.194397133622	-0.24404786816379\\
34.200975193112	-0.244048659566745\\
34.2075532231751	-0.244049450538053\\
34.2141312238278	-0.244050241078033\\
34.2207091950861	-0.244051031186997\\
34.2272871369665	-0.244051820865263\\
34.233865049485	-0.244052610113143\\
34.240442932658	-0.244053398930954\\
34.2470207865016	-0.244054187319008\\
34.253598611032	-0.24405497527762\\
34.2601764062655	-0.244055762807103\\
34.2667541722182	-0.244056549907771\\
34.2733319089063	-0.244057336579936\\
34.2799096163459	-0.244058122823911\\
34.2864872945531	-0.244058908640009\\
34.2930649435442	-0.244059694028542\\
34.2996425633351	-0.244060478989821\\
34.3062201539421	-0.244061263524159\\
34.3127977153811	-0.244062047631866\\
34.3193752476684	-0.244062831313253\\
34.3259527508198	-0.244063614568631\\
34.3325302248516	-0.244064397398311\\
34.3391076697798	-0.244065179802603\\
34.3456850856203	-0.244065961781816\\
34.3522624723893	-0.24406674333626\\
34.3588398301026	-0.244067524466244\\
34.3654171587764	-0.244068305172078\\
34.3719944584265	-0.244069085454069\\
34.3785717290691	-0.244069865312527\\
34.3851489707199	-0.24407064474776\\
34.391726183395	-0.244071423760075\\
34.3983033671104	-0.24407220234978\\
34.4048805218819	-0.244072980517183\\
34.4114576477254	-0.24407375826259\\
34.4180347446569	-0.244074535586308\\
34.4246118126923	-0.244075312488644\\
34.4311888518474	-0.244076088969903\\
34.4377658621381	-0.244076865030392\\
34.4443428435802	-0.244077640670417\\
34.4509197961897	-0.244078415890282\\
34.4574967199822	-0.244079190690292\\
34.4640736149737	-0.244079965070754\\
34.47065048118	-0.24408073903197\\
34.4772273186168	-0.244081512574245\\
34.4838041273	-0.244082285697884\\
34.4903809072452	-0.24408305840319\\
34.4969576584684	-0.244083830690466\\
34.5035343809851	-0.244084602560015\\
34.5101110748113	-0.24408537401214\\
34.5166877399625	-0.244086145047145\\
34.5232643764545	-0.24408691566533\\
34.5298409843031	-0.244087685866999\\
34.5364175635238	-0.244088455652452\\
34.5429941141325	-0.244089225021992\\
34.5495706361447	-0.244089993975919\\
34.5561471295761	-0.244090762514535\\
34.5627235944423	-0.24409153063814\\
34.5693000307591	-0.244092298347035\\
34.575876438542	-0.244093065641519\\
34.5824528178066	-0.244093832521893\\
34.5890291685685	-0.244094598988455\\
34.5956054908433	-0.244095365041506\\
34.6021817846467	-0.244096130681344\\
34.6087580499941	-0.244096895908268\\
34.6153342869011	-0.244097660722577\\
34.6219104953833	-0.244098425124569\\
34.6284866754561	-0.244099189114541\\
34.6350628271352	-0.244099952692791\\
34.641638950436	-0.244100715859617\\
34.648215045374	-0.244101478615316\\
34.6547911119647	-0.244102240960184\\
34.6613671502236	-0.244103002894519\\
34.6679431601661	-0.244103764418617\\
34.6745191418078	-0.244104525532772\\
34.6810950951639	-0.244105286237283\\
34.6876710202501	-0.244106046532443\\
34.6942469170816	-0.244106806418548\\
34.7008227856739	-0.244107565895893\\
34.7073986260423	-0.244108324964773\\
34.7139744382024	-0.244109083625483\\
34.7205502221694	-0.244109841878315\\
34.7271259779586	-0.244110599723565\\
34.7337017055856	-0.244111357161526\\
34.7402774050655	-0.244112114192491\\
34.7468530764138	-0.244112870816754\\
34.7534287196456	-0.244113627034607\\
34.7600043347765	-0.244114382846342\\
34.7665799218215	-0.244115138252253\\
34.7731554807961	-0.244115893252631\\
34.7797310117154	-0.244116647847768\\
34.7863065145949	-0.244117402037955\\
34.7928819894496	-0.244118155823484\\
34.7994574362948	-0.244118909204646\\
34.8060328551459	-0.244119662181731\\
34.8126082460179	-0.24412041475503\\
34.819183608926	-0.244121166924833\\
34.8257589438856	-0.244121918691431\\
34.8323342509117	-0.244122670055112\\
34.8389095300196	-0.244123421016166\\
34.8454847812244	-0.244124171574882\\
34.8520600045412	-0.24412492173155\\
34.8586351999852	-0.244125671486457\\
34.8652103675715	-0.244126420839893\\
34.8717855073152	-0.244127169792145\\
34.8783606192315	-0.244127918343501\\
34.8849357033354	-0.244128666494249\\
34.891510759642	-0.244129414244677\\
34.8980857881664	-0.24413016159507\\
34.9046607889236	-0.244130908545716\\
34.9112357619288	-0.244131655096902\\
34.9178107071968	-0.244132401248914\\
34.9243856247428	-0.244133147002038\\
34.9309605145818	-0.244133892356559\\
34.9375353767288	-0.244134637312764\\
34.9441102111988	-0.244135381870937\\
34.9506850180068	-0.244136126031364\\
34.9572597971677	-0.244136869794329\\
34.9638345486964	-0.244137613160117\\
34.9704092726081	-0.244138356129012\\
34.9769839689175	-0.244139098701298\\
34.9835586376397	-0.244139840877259\\
34.9901332787895	-0.244140582657178\\
34.9967078923819	-0.244141324041338\\
35.0032824784317	-0.244142065030023\\
35.0098570369539	-0.244142805623515\\
35.0164315679633	-0.244143545822097\\
35.0230060714748	-0.244144285626051\\
35.0295805475032	-0.244145025035658\\
35.0361549960633	-0.244145764051201\\
35.0427294171701	-0.244146502672961\\
35.0493038108384	-0.24414724090122\\
35.0558781770829	-0.244147978736257\\
35.0624525159184	-0.244148716178354\\
35.0690268273598	-0.244149453227791\\
35.0756011114217	-0.244150189884849\\
35.0821753681191	-0.244150926149807\\
35.0887495974666	-0.244151662022946\\
35.095323799479	-0.244152397504543\\
35.101897974171	-0.24415313259488\\
35.1084721215574	-0.244153867294234\\
35.1150462416529	-0.244154601602885\\
35.1216203344721	-0.24415533552111\\
35.1281944000298	-0.244156069049188\\
35.1347684383406	-0.244156802187397\\
35.1413424494193	-0.244157534936015\\
35.1479164332804	-0.244158267295319\\
35.1544903899387	-0.244158999265586\\
35.1610643194087	-0.244159730847093\\
35.1676382217052	-0.244160462040117\\
35.1742120968426	-0.244161192844935\\
35.1807859448358	-0.244161923261822\\
35.1873597656991	-0.244162653291055\\
35.1939335594473	-0.244163382932908\\
35.2005073260948	-0.244164112187659\\
35.2070810656563	-0.244164841055581\\
35.2136547781464	-0.244165569536951\\
35.2202284635795	-0.244166297632041\\
35.2268021219702	-0.244167025341128\\
35.233375753333	-0.244167752664486\\
35.2399493576825	-0.244168479602387\\
35.2465229350331	-0.244169206155107\\
35.2530964853993	-0.244169932322918\\
35.2596700087957	-0.244170658106095\\
35.2662435052366	-0.244171383504909\\
35.2728169747366	-0.244172108519633\\
35.2793904173101	-0.244172833150541\\
35.2859638329716	-0.244173557397904\\
35.2925372217354	-0.244174281261995\\
35.299110583616	-0.244175004743085\\
35.3056839186279	-0.244175727841446\\
35.3122572267853	-0.244176450557349\\
35.3188305081027	-0.244177172891065\\
35.3254037625945	-0.244177894842865\\
35.3319769902751	-0.244178616413021\\
35.3385501911587	-0.244179337601801\\
35.3451233652598	-0.244180058409476\\
35.3516965125927	-0.244180778836316\\
35.3582696331717	-0.244181498882591\\
35.3648427270111	-0.24418221854857\\
35.3714157941253	-0.244182937834522\\
35.3779888345285	-0.244183656740716\\
35.384561848235	-0.244184375267422\\
35.391134835259	-0.244185093414906\\
35.397707795615	-0.244185811183437\\
35.4042807293171	-0.244186528573285\\
35.4108536363795	-0.244187245584715\\
35.4174265168165	-0.244187962217996\\
35.4239993706423	-0.244188678473394\\
35.4305721978712	-0.244189394351178\\
35.4371449985173	-0.244190109851613\\
35.4437177725948	-0.244190824974966\\
35.4502905201179	-0.244191539721503\\
35.4568632411008	-0.244192254091491\\
35.4634359355577	-0.244192968085196\\
35.4700086035027	-0.244193681702882\\
35.4765812449499	-0.244194394944816\\
35.4831538599135	-0.244195107811262\\
35.4897264484076	-0.244195820302485\\
35.4962990104463	-0.244196532418751\\
35.5028715460437	-0.244197244160322\\
35.5094440552139	-0.244197955527464\\
35.516016537971	-0.244198666520441\\
35.5225889943291	-0.244199377139516\\
35.5291614243022	-0.244200087384952\\
35.5357338279044	-0.244200797257014\\
35.5423062051497	-0.244201506755963\\
35.5488785560522	-0.244202215882063\\
35.5554508806258	-0.244202924635576\\
35.5620231788846	-0.244203633016765\\
35.5685954508426	-0.244204341025892\\
35.5751676965137	-0.244205048663218\\
35.581739915912	-0.244205755929005\\
35.5883121090515	-0.244206462823515\\
35.594884275946	-0.244207169347008\\
35.6014564166096	-0.244207875499746\\
35.6080285310562	-0.244208581281989\\
35.6146006192997	-0.244209286693999\\
35.621172681354	-0.244209991736034\\
35.6277447172331	-0.244210696408355\\
35.6343167269509	-0.244211400711222\\
35.6408887105211	-0.244212104644894\\
35.6474606679578	-0.244212808209631\\
35.6540325992749	-0.244213511405692\\
35.660604504486	-0.244214214233335\\
35.6671763836052	-0.24421491669282\\
35.6737482366463	-0.244215618784404\\
35.680320063623	-0.244216320508346\\
35.6868918645493	-0.244217021864904\\
35.6934636394388	-0.244217722854335\\
35.7000353883056	-0.244218423476898\\
35.7066071111632	-0.244219123732849\\
35.7131788080255	-0.244219823622445\\
35.7197504789064	-0.244220523145943\\
35.7263221238194	-0.2442212223036\\
35.7328937427785	-0.244221921095672\\
35.7394653357973	-0.244222619522415\\
35.7460369028896	-0.244223317584086\\
35.7526084440691	-0.244224015280939\\
35.7591799593495	-0.24422471261323\\
35.7657514487445	-0.244225409581216\\
35.7723229122679	-0.244226106185149\\
35.7788943499332	-0.244226802425286\\
35.7854657617543	-0.244227498301881\\
35.7920371477447	-0.244228193815188\\
35.798608507918	-0.244228888965462\\
35.8051798422881	-0.244229583752956\\
35.8117511508684	-0.244230278177923\\
35.8183224336727	-0.244230972240619\\
35.8248936907144	-0.244231665941294\\
35.8314649220073	-0.244232359280204\\
35.838036127565	-0.2442330522576\\
35.844607307401	-0.244233744873734\\
35.8511784615289	-0.24423443712886\\
35.8577495899623	-0.24423512902323\\
35.8643206927147	-0.244235820557095\\
35.8708917697997	-0.244236511730706\\
35.8774628212308	-0.244237202544317\\
35.8840338470216	-0.244237892998177\\
35.8906048471855	-0.244238583092537\\
35.8971758217362	-0.244239272827649\\
35.9037467706871	-0.244239962203763\\
35.9103176940516	-0.24424065122113\\
35.9168885918433	-0.244241339879999\\
35.9234594640757	-0.244242028180621\\
35.9300303107622	-0.244242716123245\\
35.9366011319162	-0.24424340370812\\
35.9431719275513	-0.244244090935497\\
35.9497426976808	-0.244244777805623\\
35.9563134423181	-0.244245464318748\\
35.9628841614768	-0.244246150475121\\
35.9694548551702	-0.244246836274989\\
35.9760255234116	-0.244247521718601\\
35.9825961662145	-0.244248206806205\\
35.9891667835923	-0.244248891538049\\
35.9957373755583	-0.24424957591438\\
36.002307942126	-0.244250259935445\\
36.0088784833085	-0.244250943601492\\
36.0154489991193	-0.244251626912768\\
36.0220194895718	-0.244252309869518\\
36.0285899546792	-0.244252992471991\\
36.0351603944548	-0.24425367472043\\
36.041730808912	-0.244254356615084\\
36.048301198064	-0.244255038156197\\
36.0548715619241	-0.244255719344016\\
36.0614419005057	-0.244256400178784\\
36.0680122138219	-0.244257080660749\\
36.0745825018861	-0.244257760790154\\
36.0811527647114	-0.244258440567245\\
36.0877230023111	-0.244259119992265\\
36.0942932146985	-0.24425979906546\\
36.1008634018868	-0.244260477787073\\
36.1074335638891	-0.244261156157348\\
36.1140037007186	-0.244261834176529\\
36.1205738123887	-0.244262511844859\\
36.1271438989124	-0.244263189162581\\
36.1337139603029	-0.244263866129939\\
36.1402839965733	-0.244264542747176\\
36.1468540077369	-0.244265219014533\\
36.1534239938068	-0.244265894932254\\
36.1599939547961	-0.24426657050058\\
36.166563890718	-0.244267245719754\\
36.1731338015855	-0.244267920590017\\
36.1797036874117	-0.244268595111611\\
36.1862735482099	-0.244269269284776\\
36.192843383993	-0.244269943109756\\
36.1994131947741	-0.244270616586789\\
36.2059829805663	-0.244271289716117\\
36.2125527413827	-0.244271962497981\\
36.2191224772363	-0.24427263493262\\
36.2256921881402	-0.244273307020276\\
36.2322618741073	-0.244273978761186\\
36.2388315351508	-0.244274650155593\\
36.2454011712837	-0.244275321203734\\
36.2519707825188	-0.244275991905849\\
36.2585403688694	-0.244276662262177\\
36.2651099303482	-0.244277332272957\\
36.2716794669684	-0.244278001938428\\
36.2782489787428	-0.244278671258827\\
36.2848184656845	-0.244279340234394\\
36.2913879278064	-0.244280008865366\\
36.2979573651214	-0.244280677151981\\
36.3045267776426	-0.244281345094477\\
36.3110961653827	-0.24428201269309\\
36.3176655283547	-0.244282679948059\\
36.3242348665715	-0.24428334685962\\
36.3308041800461	-0.24428401342801\\
36.3373734687913	-0.244284679653466\\
36.3439427328199	-0.244285345536223\\
36.350511972145	-0.244286011076519\\
36.3570811867792	-0.244286676274588\\
36.3636503767355	-0.244287341130668\\
36.3702195420268	-0.244288005644992\\
36.3767886826658	-0.244288669817797\\
36.3833577986654	-0.244289333649318\\
36.3899268900383	-0.24428999713979\\
36.3964959567975	-0.244290660289447\\
36.4030649989557	-0.244291323098525\\
36.4096340165257	-0.244291985567256\\
36.4162030095203	-0.244292647695876\\
36.4227719779522	-0.244293309484619\\
36.4293409218342	-0.244293970933717\\
36.4359098411791	-0.244294632043405\\
36.4424787359996	-0.244295292813916\\
36.4490476063084	-0.244295953245483\\
36.4556164521184	-0.244296613338339\\
36.4621852734421	-0.244297273092717\\
36.4687540702923	-0.244297932508849\\
36.4753228426818	-0.244298591586968\\
36.4818915906231	-0.244299250327305\\
36.488460314129	-0.244299908730093\\
36.4950290132122	-0.244300566795563\\
36.5015976878853	-0.244301224523947\\
36.508166338161	-0.244301881915475\\
36.5147349640519	-0.24430253897038\\
36.5213035655707	-0.244303195688893\\
36.52787214273	-0.244303852071243\\
36.5344406955424	-0.244304508117661\\
36.5410092240206	-0.244305163828378\\
36.5475777281771	-0.244305819203624\\
36.5541462080245	-0.244306474243629\\
36.5607146635754	-0.244307128948622\\
36.5672830948424	-0.244307783318833\\
36.5738515018381	-0.244308437354491\\
36.5804198845751	-0.244309091055826\\
36.5869882430658	-0.244309744423066\\
36.5935565773228	-0.24431039745644\\
36.6001248873587	-0.244311050156177\\
36.606693173186	-0.244311702522505\\
36.6132614348171	-0.244312354555652\\
36.6198296722647	-0.244313006255846\\
36.6263978855412	-0.244313657623316\\
36.6329660746591	-0.244314308658287\\
36.6395342396309	-0.244314959360988\\
36.646102380469	-0.244315609731647\\
36.6526704971859	-0.244316259770489\\
36.6592385897942	-0.244316909477742\\
36.6658066583061	-0.244317558853632\\
36.6723747027342	-0.244318207898386\\
36.6789427230909	-0.244318856612229\\
36.6855107193886	-0.244319504995389\\
36.6920786916397	-0.244320153048091\\
36.6986466398567	-0.244320800770559\\
36.7052145640519	-0.244321448163021\\
36.7117824642377	-0.2443220952257\\
36.7183503404265	-0.244322741958823\\
36.7249181926306	-0.244323388362613\\
36.7314860208625	-0.244324034437296\\
36.7380538251345	-0.244324680183096\\
36.7446216054588	-0.244325325600237\\
36.751189361848	-0.244325970688944\\
36.7577570943142	-0.244326615449439\\
36.7643248028698	-0.244327259881948\\
36.7708924875272	-0.244327903986693\\
36.7774601482985	-0.244328547763897\\
36.7840277851962	-0.244329191213785\\
36.7905953982325	-0.244329834336578\\
36.7971629874196	-0.244330477132499\\
36.8037305527699	-0.244331119601772\\
36.8102980942955	-0.244331761744618\\
36.8168656120088	-0.24433240356126\\
36.823433105922	-0.244333045051919\\
36.8300005760474	-0.244333686216817\\
36.8365680223971	-0.244334327056176\\
36.8431354449833	-0.244334967570218\\
36.8497028438183	-0.244335607759163\\
36.8562702189143	-0.244336247623233\\
36.8628375702835	-0.244336887162648\\
36.869404897938	-0.244337526377629\\
36.8759722018901	-0.244338165268397\\
36.8825394821518	-0.244338803835171\\
36.8891067387354	-0.244339442078172\\
36.895673971653	-0.24434007999762\\
36.9022411809167	-0.244340717593735\\
36.9088083665387	-0.244341354866736\\
36.9153755285311	-0.244341991816841\\
36.921942666906	-0.244342628444272\\
36.9285097816756	-0.244343264749246\\
36.9350768728518	-0.244343900731982\\
36.9416439404469	-0.244344536392699\\
36.9482109844729	-0.244345171731615\\
36.9547780049418	-0.244345806748948\\
36.9613450018658	-0.244346441444917\\
36.9679119752569	-0.24434707581974\\
36.9744789251272	-0.244347709873633\\
36.9810458514886	-0.244348343606815\\
36.9876127543532	-0.244348977019503\\
36.9941796337331	-0.244349610111913\\
37.0007464896403	-0.244350242884264\\
37.0073133220867	-0.24435087533677\\
37.0138801310844	-0.24435150746965\\
37.0204469166454	-0.24435213928312\\
37.0270136787816	-0.244352770777395\\
37.033580417505	-0.244353401952691\\
37.0401471328276	-0.244354032809225\\
37.0467138247614	-0.244354663347212\\
37.0532804933183	-0.244355293566868\\
37.0598471385102	-0.244355923468407\\
37.0664137603491	-0.244356553052046\\
37.0729803588469	-0.244357182317998\\
37.0795469340155	-0.244357811266479\\
37.0861134858668	-0.244358439897702\\
37.0926800144128	-0.244359068211883\\
37.0992465196653	-0.244359696209236\\
37.1058130016362	-0.244360323889974\\
37.1123794603374	-0.244360951254312\\
37.1189458957808	-0.244361578302463\\
37.1255123079781	-0.24436220503464\\
37.1320786969414	-0.244362831451057\\
37.1386450626824	-0.244363457551926\\
37.1452114052129	-0.244364083337462\\
37.1517777245448	-0.244364708807876\\
37.1583440206899	-0.244365333963381\\
37.16491029366	-0.244365958804189\\
37.1714765434669	-0.244366583330513\\
37.1780427701225	-0.244367207542564\\
37.1846089736384	-0.244367831440555\\
37.1911751540266	-0.244368455024696\\
37.1977413112987	-0.244369078295201\\
37.2043074454665	-0.244369701252279\\
37.2108735565418	-0.244370323896142\\
37.2174396445363	-0.244370946227\\
37.2240057094617	-0.244371568245066\\
37.2305717513299	-0.244372189950548\\
37.2371377701525	-0.244372811343658\\
37.2437037659412	-0.244373432424606\\
37.2502697387078	-0.244374053193602\\
37.2568356884638	-0.244374673650855\\
37.2634016152212	-0.244375293796576\\
37.2699675189914	-0.244375913630973\\
37.2765333997863	-0.244376533154257\\
37.2830992576174	-0.244377152366637\\
37.2896650924964	-0.24437777126832\\
37.296230904435	-0.244378389859517\\
37.3027966934448	-0.244379008140435\\
37.3093624595374	-0.244379626111284\\
37.3159282027246	-0.244380243772271\\
37.3224939230178	-0.244380861123605\\
37.3290596204288	-0.244381478165494\\
37.335625294969	-0.244382094898144\\
37.3421909466502	-0.244382711321765\\
37.3487565754839	-0.244383327436563\\
37.3553221814817	-0.244383943242745\\
37.3618877646551	-0.244384558740519\\
37.3684533250157	-0.244385173930091\\
37.3750188625752	-0.244385788811669\\
37.3815843773449	-0.244386403385458\\
37.3881498693365	-0.244387017651665\\
37.3947153385616	-0.244387631610496\\
37.4012807850315	-0.244388245262157\\
37.4078462087579	-0.244388858606855\\
37.4144116097523	-0.244389471644794\\
37.4209769880261	-0.24439008437618\\
37.4275423435909	-0.244390696801219\\
37.4341076764581	-0.244391308920115\\
37.4406729866392	-0.244391920733074\\
37.4472382741457	-0.2443925322403\\
37.4538035389891	-0.244393143441998\\
37.4603687811807	-0.244393754338373\\
37.4669340007321	-0.244394364929629\\
37.4734991976548	-0.244394975215969\\
37.48006437196	-0.244395585197598\\
37.4866295236593	-0.24439619487472\\
37.4931946527641	-0.244396804247538\\
37.4997597592857	-0.244397413316255\\
37.5063248432356	-0.244398022081075\\
37.5128899046252	-0.244398630542201\\
37.5194549434659	-0.244399238699836\\
37.5260199597689	-0.244399846554183\\
37.5325849535458	-0.244400454105444\\
37.5391499248079	-0.244401061353821\\
37.5457148735664	-0.244401668299517\\
37.5522797998329	-0.244402274942734\\
37.5588447036185	-0.244402881283673\\
37.5654095849347	-0.244403487322537\\
37.5719744437928	-0.244404093059527\\
37.578539280204	-0.244404698494844\\
37.5851040941798	-0.24440530362869\\
37.5916688857313	-0.244405908461266\\
37.5982336548699	-0.244406512992772\\
37.6047984016068	-0.24440711722341\\
37.6113631259535	-0.244407721153379\\
37.617927827921	-0.24440832478288\\
37.6244925075207	-0.244408928112114\\
37.6310571647639	-0.24440953114128\\
37.6376217996617	-0.244410133870579\\
37.6441864122255	-0.244410736300209\\
37.6507510024664	-0.244411338430371\\
37.6573155703957	-0.244411940261265\\
37.6638801160246	-0.244412541793088\\
37.6704446393643	-0.24441314302604\\
37.677009140426	-0.24441374396032\\
37.683573619221	-0.244414344596128\\
37.6901380757603	-0.24441494493366\\
37.6967025100552	-0.244415544973116\\
37.7032669221168	-0.244416144714694\\
37.7098313119564	-0.244416744158592\\
37.716395679585	-0.244417343305008\\
37.7229600250139	-0.244417942154139\\
37.7295243482541	-0.244418540706183\\
37.7360886493167	-0.244419138961338\\
37.742652928213	-0.244419736919801\\
37.7492171849541	-0.244420334581768\\
37.755781419551	-0.244420931947437\\
37.7623456320148	-0.244421529017004\\
37.7689098223567	-0.244422125790666\\
37.7754739905878	-0.24442272226862\\
37.7820381367191	-0.244423318451061\\
37.7886022607616	-0.244423914338185\\
37.7951663627266	-0.24442450993019\\
37.8017304426249	-0.244425105227269\\
37.8082945004678	-0.244425700229619\\
37.8148585362661	-0.244426294937436\\
37.821422550031	-0.244426889350914\\
37.8279865417735	-0.244427483470249\\
37.8345505115046	-0.244428077295635\\
37.8411144592354	-0.244428670827268\\
37.8476783849768	-0.244429264065342\\
37.8542422887398	-0.244429857010051\\
37.8608061705354	-0.24443044966159\\
37.8673700303747	-0.244431042020152\\
37.8739338682685	-0.244431634085932\\
37.8804976842279	-0.244432225859124\\
37.8870614782639	-0.244432817339921\\
37.8936252503873	-0.244433408528517\\
37.9001890006092	-0.244433999425105\\
37.9067527289404	-0.244434590029878\\
37.9133164353919	-0.244435180343029\\
37.9198801199747	-0.244435770364751\\
37.9264437826996	-0.244436360095237\\
37.9330074235776	-0.244436949534679\\
37.9395710426195	-0.244437538683269\\
37.9461346398363	-0.244438127541201\\
37.9526982152389	-0.244438716108665\\
37.9592617688381	-0.244439304385853\\
37.9658253006448	-0.244439892372959\\
37.9723888106699	-0.244440480070172\\
37.9789522989243	-0.244441067477684\\
37.9855157654188	-0.244441654595687\\
37.9920792101642	-0.244442241424372\\
37.9986426331713	-0.244442827963929\\
38.0052060344511	-0.244443414214549\\
38.0117694140144	-0.244444000176424\\
38.0183327718719	-0.244444585849743\\
38.0248961080344	-0.244445171234697\\
38.0314594225129	-0.244445756331475\\
38.038022715318	-0.244446341140269\\
38.0445859864605	-0.244446925661267\\
38.0511492359513	-0.244447509894659\\
38.0577124638011	-0.244448093840635\\
38.0642756700207	-0.244448677499385\\
38.0708388546208	-0.244449260871097\\
38.0774020176122	-0.24444984395596\\
38.0839651590057	-0.244450426754163\\
38.0905282788119	-0.244451009265896\\
38.0970913770416	-0.244451591491346\\
38.1036544537056	-0.244452173430702\\
38.1102175088145	-0.244452755084153\\
38.116780542379	-0.244453336451885\\
38.1233435544099	-0.244453917534089\\
38.1299065449179	-0.24445449833095\\
38.1364695139136	-0.244455078842657\\
38.1430324614077	-0.244455659069397\\
38.1495953874109	-0.244456239011358\\
38.1561582919339	-0.244456818668726\\
38.1627211749873	-0.24445739804169\\
38.1692840365818	-0.244457977130435\\
38.175846876728	-0.244458555935148\\
38.1824096954366	-0.244459134456016\\
38.1889724927181	-0.244459712693225\\
38.1955352685833	-0.244460290646962\\
38.2020980230427	-0.244460868317413\\
38.2086607561069	-0.244461445704763\\
38.2152234677866	-0.244462022809199\\
38.2217861580923	-0.244462599630905\\
38.2283488270347	-0.244463176170068\\
38.2349114746242	-0.244463752426873\\
38.2414741008716	-0.244464328401505\\
38.2480367057873	-0.244464904094149\\
38.2545992893819	-0.244465479504989\\
38.261161851666	-0.244466054634212\\
38.2677243926501	-0.244466629482\\
38.2742869123447	-0.244467204048539\\
38.2808494107605	-0.244467778334013\\
38.2874118879078	-0.244468352338606\\
38.2939743437973	-0.244468926062502\\
38.3005367784394	-0.244469499505885\\
38.3070991918447	-0.244470072668938\\
38.3136615840236	-0.244470645551845\\
38.3202239549867	-0.244471218154789\\
38.3267863047444	-0.244471790477954\\
38.3333486333071	-0.244472362521522\\
38.3399109406855	-0.244472934285677\\
38.3464732268899	-0.244473505770601\\
38.3530354919307	-0.244474076976476\\
38.3595977358185	-0.244474647903486\\
38.3661599585637	-0.244475218551813\\
38.3727221601767	-0.244475788921638\\
38.3792843406679	-0.244476359013144\\
38.3858465000478	-0.244476928826513\\
38.3924086383268	-0.244477498361926\\
38.3989707555153	-0.244478067619564\\
38.4055328516236	-0.24447863659961\\
38.4120949266622	-0.244479205302245\\
38.4186569806415	-0.244479773727649\\
38.4252190135719	-0.244480341876004\\
38.4317810254637	-0.24448090974749\\
38.4383430163272	-0.244481477342289\\
38.4449049861729	-0.24448204466058\\
38.4514669350111	-0.244482611702544\\
38.4580288628521	-0.244483178468361\\
38.4645907697064	-0.244483744958211\\
38.4711526555841	-0.244484311172275\\
38.4777145204957	-0.244484877110732\\
38.4842763644515	-0.244485442773761\\
38.4908381874617	-0.244486008161543\\
38.4973999895367	-0.244486573274256\\
38.5039617706868	-0.244487138112079\\
38.5105235309222	-0.244487702675193\\
38.5170852702533	-0.244488266963775\\
38.5236469886903	-0.244488830978004\\
38.5302086862435	-0.24448939471806\\
38.5367703629232	-0.244489958184119\\
38.5433320187396	-0.244490521376362\\
38.549893653703	-0.244491084294966\\
38.5564552678236	-0.244491646940108\\
38.5630168611116	-0.244492209311968\\
38.5695784335773	-0.244492771410722\\
38.576139985231	-0.244493333236548\\
38.5827015160827	-0.244493894789624\\
38.5892630261428	-0.244494456070127\\
38.5958245154215	-0.244495017078234\\
38.6023859839288	-0.244495577814121\\
38.6089474316751	-0.244496138277967\\
38.6155088586705	-0.244496698469948\\
38.6220702649252	-0.24449725839024\\
38.6286316504493	-0.24449781803902\\
38.6351930152531	-0.244498377416463\\
38.6417543593466	-0.244498936522747\\
38.64831568274	-0.244499495358047\\
38.6548769854436	-0.244500053922539\\
38.6614382674673	-0.244500612216399\\
38.6679995288213	-0.244501170239802\\
38.6745607695158	-0.244501727992924\\
38.6811219895609	-0.244502285475939\\
38.6876831889666	-0.244502842689024\\
38.6942443677432	-0.244503399632354\\
38.7008055259006	-0.244503956306102\\
38.707366663449	-0.244504512710444\\
38.7139277803984	-0.244505068845554\\
38.720488876759	-0.244505624711607\\
38.7270499525408	-0.244506180308778\\
38.7336110077538	-0.244506735637239\\
38.7401720424082	-0.244507290697166\\
38.7467330565139	-0.244507845488731\\
38.7532940500811	-0.24450840001211\\
38.7598550231197	-0.244508954267475\\
38.7664159756397	-0.244509508255\\
38.7729769076513	-0.244510061974858\\
38.7795378191644	-0.244510615427222\\
38.7860987101891	-0.244511168612266\\
38.7926595807353	-0.244511721530162\\
38.799220430813	-0.244512274181083\\
38.8057812604322	-0.244512826565201\\
38.812342069603	-0.24451337868269\\
38.8189028583353	-0.244513930533721\\
38.825463626639	-0.244514482118466\\
38.8320243745241	-0.244515033437098\\
38.8385851020006	-0.244515584489789\\
38.8451458090785	-0.244516135276709\\
38.8517064957676	-0.244516685798032\\
38.858267162078	-0.244517236053928\\
38.8648278080195	-0.244517786044569\\
38.8713884336021	-0.244518335770125\\
38.8779490388356	-0.244518885230769\\
38.8845096237301	-0.24451943442667\\
38.8910701882955	-0.244519983358\\
38.8976307325415	-0.24452053202493\\
38.9041912564781	-0.244521080427629\\
38.9107517601152	-0.244521628566269\\
38.9173122434628	-0.244522176441019\\
38.9238727065305	-0.24452272405205\\
38.9304331493284	-0.244523271399531\\
38.9369935718663	-0.244523818483633\\
38.943553974154	-0.244524365304525\\
38.9501143562013	-0.244524911862376\\
38.9566747180182	-0.244525458157357\\
38.9632350596145	-0.244526004189635\\
38.969795381	-0.244526549959382\\
38.9763556821844	-0.244527095466764\\
38.9829159631777	-0.244527640711952\\
38.9894762239896	-0.244528185695114\\
38.99603646463	-0.244528730416419\\
39.0025966851086	-0.244529274876035\\
39.0091568854352	-0.24452981907413\\
39.0157170656197	-0.244530363010873\\
39.0222772256717	-0.244530906686431\\
39.0288373656011	-0.244531450100973\\
39.0353974854177	-0.244531993254667\\
39.0419575851311	-0.244532536147679\\
39.0485176647511	-0.244533078780178\\
39.0550777242876	-0.244533621152331\\
39.0616377637501	-0.244534163264305\\
39.0681977831485	-0.244534705116267\\
39.0747577824925	-0.244535246708385\\
39.0813177617918	-0.244535788040824\\
39.0878777210561	-0.244536329113752\\
39.0944376602952	-0.244536869927336\\
39.1009975795187	-0.244537410481741\\
39.1075574787363	-0.244537950777135\\
39.1141173579577	-0.244538490813682\\
39.1206772171926	-0.24453903059155\\
39.1272370564507	-0.244539570110904\\
39.1337968757416	-0.244540109371909\\
39.140356675075	-0.244540648374733\\
39.1469164544606	-0.244541187119539\\
39.153476213908	-0.244541725606494\\
39.1600359534269	-0.244542263835762\\
39.1665956730268	-0.244542801807509\\
39.1731553727175	-0.2445433395219\\
39.1797150525086	-0.244543876979099\\
39.1862747124096	-0.244544414179272\\
39.1928343524303	-0.244544951122582\\
39.1993939725801	-0.244545487809194\\
39.2059535728687	-0.244546024239273\\
39.2125131533058	-0.244546560412983\\
39.2190727139008	-0.244547096330487\\
39.2256322546633	-0.24454763199195\\
39.2321917756031	-0.244548167397535\\
39.2387512767295	-0.244548702547407\\
39.2453107580522	-0.244549237441728\\
39.2518702195808	-0.244549772080662\\
39.2584296613247	-0.244550306464372\\
39.2649890832936	-0.244550840593021\\
39.2715484854969	-0.244551374466773\\
39.2781078679443	-0.24455190808579\\
39.2846672306452	-0.244552441450235\\
39.2912265736091	-0.24455297456027\\
39.2977858968456	-0.244553507416058\\
39.3043452003642	-0.244554040017762\\
39.3109044841744	-0.244554572365543\\
39.3174637482856	-0.244555104459563\\
39.3240229927074	-0.244555636299985\\
39.3305822174493	-0.244556167886971\\
39.3371414225207	-0.244556699220681\\
39.343700607931	-0.244557230301278\\
39.3502597736899	-0.244557761128923\\
39.3568189198067	-0.244558291703777\\
39.3633780462908	-0.244558822026002\\
39.3699371531518	-0.244559352095758\\
39.376496240399	-0.244559881913207\\
39.383055308042	-0.244560411478509\\
39.38961435609	-0.244560940791825\\
39.3961733845527	-0.244561469853315\\
39.4027323934393	-0.24456199866314\\
39.4092913827592	-0.24456252722146\\
39.415850352522	-0.244563055528435\\
39.4224093027369	-0.244563583584226\\
39.4289682334135	-0.244564111388991\\
39.4355271445609	-0.244564638942892\\
39.4420860361888	-0.244565166246087\\
39.4486449083063	-0.244565693298736\\
39.4552037609229	-0.244566220100998\\
39.461762594048	-0.244566746653033\\
39.4683214076909	-0.244567272954999\\
39.4748802018609	-0.244567799007057\\
39.4814389765674	-0.244568324809364\\
39.4879977318198	-0.244568850362079\\
39.4945564676273	-0.244569375665361\\
39.5011151839993	-0.244569900719369\\
39.5076738809452	-0.24457042552426\\
39.5142325584741	-0.244570950080194\\
39.5207912165955	-0.244571474387328\\
39.5273498553187	-0.244571998445819\\
39.5339084746529	-0.244572522255827\\
39.5404670746074	-0.244573045817509\\
39.5470256551915	-0.244573569131022\\
39.5535842164145	-0.244574092196523\\
39.5601427582856	-0.244574615014171\\
39.5667012808142	-0.244575137584122\\
39.5732597840095	-0.244575659906534\\
39.5798182678807	-0.244576181981563\\
39.5863767324371	-0.244576703809367\\
39.5929351776879	-0.244577225390101\\
39.5994936036424	-0.244577746723923\\
39.6060520103098	-0.24457826781099\\
39.6126103976994	-0.244578788651457\\
39.6191687658202	-0.24457930924548\\
39.6257271146816	-0.244579829593217\\
39.6322854442928	-0.244580349694822\\
39.638843754663	-0.244580869550452\\
39.6454020458013	-0.244581389160262\\
39.6519603177169	-0.244581908524408\\
39.6585185704191	-0.244582427643046\\
39.665076803917	-0.244582946516331\\
39.6716350182198	-0.244583465144418\\
39.6781932133366	-0.244583983527462\\
39.6847513892766	-0.244584501665618\\
39.691309546049	-0.244585019559041\\
39.6978676836629	-0.244585537207886\\
39.7044258021274	-0.244586054612308\\
39.7109839014518	-0.244586571772461\\
39.717541981645	-0.244587088688499\\
39.7241000427163	-0.244587605360576\\
39.7306580846747	-0.244588121788848\\
39.7372161075295	-0.244588637973467\\
39.7437741112896	-0.244589153914588\\
39.7503320959642	-0.244589669612364\\
39.7568900615623	-0.244590185066949\\
39.7634480080932	-0.244590700278497\\
39.7700059355657	-0.244591215247161\\
39.7765638439892	-0.244591729973094\\
39.7831217333725	-0.244592244456449\\
39.7896796037248	-0.244592758697381\\
39.7962374550551	-0.24459327269604\\
39.8027952873725	-0.244593786452581\\
39.809353100686	-0.244594299967155\\
39.8159108950047	-0.244594813239916\\
39.8224686703376	-0.244595326271016\\
39.8290264266938	-0.244595839060607\\
39.8355841640822	-0.244596351608841\\
39.842141882512	-0.244596863915871\\
39.848699581992	-0.244597375981848\\
39.8552572625314	-0.244597887806925\\
39.861814924139	-0.244598399391252\\
39.868372566824	-0.244598910734982\\
39.8749301905953	-0.244599421838266\\
39.8814877954619	-0.244599932701255\\
39.8880453814328	-0.244600443324101\\
39.8946029485169	-0.244600953706955\\
39.9011604967232	-0.244601463849967\\
39.9077180260607	-0.244601973753289\\
39.9142755365383	-0.244602483417072\\
39.9208330281651	-0.244602992841465\\
39.9273905009498	-0.24460350202662\\
39.9339479549016	-0.244604010972686\\
39.9405053900292	-0.244604519679815\\
39.9470628063417	-0.244605028148156\\
39.9536202038479	-0.24460553637786\\
39.9601775825569	-0.244606044369076\\
39.9667349424774	-0.244606552121954\\
39.9732922836184	-0.244607059636644\\
39.9798496059888	-0.244607566913295\\
39.9864069095975	-0.244608073952057\\
39.9929641944534	-0.24460858075308\\
39.9995214605653	-0.244609087316511\\
40.0060787079422	-0.244609593642502\\
40.0126359365929	-0.244610099731199\\
40.0191931465262	-0.244610605582753\\
40.0257503377511	-0.244611111197313\\
40.0323075102765	-0.244611616575025\\
40.038864664111	-0.244612121716041\\
40.0454217992636	-0.244612626620506\\
40.0519789157432	-0.244613131288571\\
40.0585360135585	-0.244613635720383\\
40.0650930927184	-0.24461413991609\\
40.0716501532318	-0.24461464387584\\
40.0782071951073	-0.244615147599781\\
40.0847642183539	-0.244615651088061\\
40.0913212229803	-0.244616154340826\\
40.0978782089954	-0.244616657358226\\
40.1044351764079	-0.244617160140406\\
40.1109921252266	-0.244617662687515\\
40.1175490554603	-0.244618164999699\\
40.1241059671178	-0.244618667077105\\
40.1306628602079	-0.244619168919881\\
40.1372197347392	-0.244619670528172\\
40.1437765907207	-0.244620171902127\\
40.1503334281609	-0.24462067304189\\
40.1568902470688	-0.24462117394761\\
40.163447047453	-0.244621674619431\\
40.1700038293222	-0.2446221750575\\
40.1765605926852	-0.244622675261964\\
40.1831173375508	-0.244623175232967\\
40.1896740639276	-0.244623674970657\\
40.1962307718244	-0.244624174475179\\
40.2027874612498	-0.244624673746678\\
40.2093441322126	-0.2446251727853\\
40.2159007847215	-0.244625671591191\\
40.2224574187851	-0.244626170164495\\
40.2290140344122	-0.244626668505358\\
40.2355706316115	-0.244627166613925\\
40.2421272103915	-0.244627664490341\\
40.2486837707611	-0.24462816213475\\
40.2552403127288	-0.244628659547297\\
40.2617968363033	-0.244629156728128\\
40.2683533414933	-0.244629653677386\\
40.2749098283074	-0.244630150395216\\
40.2814662967543	-0.244630646881762\\
40.2880227468426	-0.244631143137168\\
40.2945791785809	-0.244631639161579\\
40.3011355919779	-0.244632134955138\\
40.3076919870422	-0.244632630517989\\
40.3142483637824	-0.244633125850275\\
40.3208047222071	-0.244633620952141\\
40.3273610623249	-0.24463411582373\\
40.3339173841445	-0.244634610465184\\
40.3404736876744	-0.244635104876649\\
40.3470299729232	-0.244635599058265\\
40.3535862398995	-0.244636093010178\\
40.3601424886119	-0.244636586732529\\
40.3666987190689	-0.244637080225461\\
40.3732549312791	-0.244637573489117\\
40.3798111252512	-0.24463806652364\\
40.3863673009936	-0.244638559329172\\
40.3929234585148	-0.244639051905855\\
40.3994795978235	-0.244639544253832\\
40.4060357189282	-0.244640036373244\\
40.4125918218374	-0.244640528264235\\
40.4191479065597	-0.244641019926945\\
40.4257039731035	-0.244641511361517\\
40.4322600214774	-0.244642002568092\\
40.43881605169	-0.244642493546812\\
40.4453720637496	-0.244642984297818\\
40.4519280576649	-0.244643474821252\\
40.4584840334443	-0.244643965117255\\
40.4650399910963	-0.244644455185968\\
40.4715959306294	-0.244644945027532\\
40.4781518520522	-0.244645434642088\\
40.4847077553729	-0.244645924029777\\
40.4912636406002	-0.244646413190739\\
40.4978195077425	-0.244646902125116\\
40.5043753568083	-0.244647390833047\\
40.510931187806	-0.244647879314673\\
40.5174870007441	-0.244648367570134\\
40.524042795631	-0.244648855599571\\
40.5305985724751	-0.244649343403124\\
40.537154331285	-0.244649830980932\\
40.5437100720689	-0.244650318333135\\
40.5502657948354	-0.244650805459873\\
40.5568214995929	-0.244651292361285\\
40.5633771863497	-0.244651779037512\\
40.5699328551144	-0.244652265488693\\
40.5764885058952	-0.244652751714966\\
40.5830441387006	-0.244653237716471\\
40.589599753539	-0.244653723493347\\
40.5961553504188	-0.244654209045733\\
40.6027109293483	-0.244654694373768\\
40.6092664903359	-0.244655179477591\\
40.6158220333901	-0.24465566435734\\
40.6223775585191	-0.244656149013153\\
40.6289330657314	-0.24465663344517\\
40.6354885550352	-0.244657117653529\\
40.642044026439	-0.244657601638367\\
40.648599479951	-0.244658085399823\\
40.6551549155797	-0.244658568938035\\
40.6617103333333	-0.244659052253141\\
40.6682657332202	-0.244659535345278\\
40.6748211152487	-0.244660018214584\\
40.6813764794272	-0.244660500861198\\
40.6879318257639	-0.244660983285255\\
40.6944871542671	-0.244661465486894\\
40.7010424649452	-0.244661947466252\\
40.7075977578065	-0.244662429223466\\
40.7141530328592	-0.244662910758673\\
40.7207082901117	-0.24466339207201\\
40.7272635295721	-0.244663873163614\\
40.7338187512489	-0.244664354033622\\
40.7403739551503	-0.244664834682169\\
40.7469291412845	-0.244665315109393\\
40.7534843096598	-0.24466579531543\\
40.7600394602844	-0.244666275300417\\
40.7665945931667	-0.244666755064489\\
40.7731497083148	-0.244667234607783\\
40.779704805737	-0.244667713930434\\
40.7862598854416	-0.244668193032579\\
40.7928149474367	-0.244668671914353\\
40.7993699917307	-0.244669150575892\\
40.8059250183316	-0.244669629017332\\
40.8124800272478	-0.244670107238808\\
40.8190350184874	-0.244670585240454\\
40.8255899920587	-0.244671063022408\\
40.8321449479699	-0.244671540584803\\
40.8386998862291	-0.244672017927775\\
40.8452548068445	-0.244672495051459\\
40.8518097098243	-0.244672971955989\\
40.8583645951768	-0.2446734486415\\
40.86491946291	-0.244673925108127\\
40.8714743130322	-0.244674401356005\\
40.8780291455515	-0.244674877385267\\
40.8845839604761	-0.244675353196049\\
40.8911387578141	-0.244675828788484\\
40.8976935375737	-0.244676304162706\\
40.904248299763	-0.24467677931885\\
40.9108030443902	-0.244677254257049\\
40.9173577714634	-0.244677728977437\\
40.9239124809908	-0.244678203480148\\
40.9304671729804	-0.244678677765316\\
40.9370218474403	-0.244679151833073\\
40.9435765043788	-0.244679625683554\\
40.9501311438038	-0.244680099316891\\
40.9566857657236	-0.244680572733218\\
40.9632403701461	-0.244681045932668\\
40.9697949570796	-0.244681518915374\\
40.976349526532	-0.244681991681468\\
40.9829040785116	-0.244682464231084\\
40.9894586130263	-0.244682936564354\\
40.9960131300842	-0.244683408681411\\
41.0025676296934	-0.244683880582387\\
41.0091221118619	-0.244684352267414\\
41.0156765765979	-0.244684823736625\\
41.0222310239094	-0.244685294990152\\
41.0287854538044	-0.244685766028127\\
41.0353398662909	-0.244686236850682\\
41.0418942613771	-0.244686707457948\\
41.0484486390709	-0.244687177850059\\
41.0550029993804	-0.244687648027144\\
41.0615573423135	-0.244688117989336\\
41.0681116678784	-0.244688587736767\\
41.0746659760831	-0.244689057269567\\
41.0812202669354	-0.244689526587868\\
41.0877745404436	-0.2446899956918\\
41.0943287966154	-0.244690464581496\\
41.100883035459	-0.244690933257086\\
41.1074372569824	-0.2446914017187\\
41.1139914611934	-0.24469186996647\\
41.1205456481001	-0.244692338000526\\
41.1270998177106	-0.244692805820999\\
41.1336539700326	-0.244693273428019\\
41.1402081050743	-0.244693740821717\\
41.1467622228435	-0.244694208002222\\
41.1533163233483	-0.244694674969665\\
41.1598704065965	-0.244695141724177\\
41.1664244725961	-0.244695608265886\\
41.1729785213551	-0.244696074594923\\
41.1795325528814	-0.244696540711417\\
41.1860865671829	-0.244697006615498\\
41.1926405642676	-0.244697472307297\\
41.1991945441434	-0.244697937786941\\
41.2057485068181	-0.244698403054561\\
41.2123024522997	-0.244698868110285\\
41.2188563805962	-0.244699332954244\\
41.2254102917153	-0.244699797586565\\
41.231964185665	-0.244700262007378\\
41.2385180624533	-0.244700726216813\\
41.2450719220879	-0.244701190214996\\
41.2516257645768	-0.244701654002058\\
41.2581795899278	-0.244702117578127\\
41.2647333981489	-0.24470258094333\\
41.2712871892478	-0.244703044097798\\
41.2778409632325	-0.244703507041657\\
41.2843947201108	-0.244703969775036\\
41.2909484598905	-0.244704432298063\\
41.2975021825795	-0.244704894610867\\
41.3040558881857	-0.244705356713574\\
41.3106095767169	-0.244705818606313\\
41.3171632481809	-0.244706280289211\\
41.3237169025855	-0.244706741762397\\
41.3302705399386	-0.244707203025997\\
41.336824160248	-0.244707664080138\\
41.3433777635215	-0.244708124924949\\
41.3499313497669	-0.244708585560556\\
41.356484918992	-0.244709045987087\\
41.3630384712046	-0.244709506204668\\
41.3695920064126	-0.244709966213426\\
41.3761455246237	-0.244710426013489\\
41.3826990258456	-0.244710885604982\\
41.3892525100863	-0.244711344988033\\
41.3958059773533	-0.244711804162768\\
41.4023594276546	-0.244712263129313\\
41.4089128609979	-0.244712721887796\\
41.4154662773909	-0.244713180438341\\
41.4220196768415	-0.244713638781075\\
41.4285730593572	-0.244714096916124\\
41.435126424946	-0.244714554843614\\
41.4416797736155	-0.244715012563672\\
41.4482331053736	-0.244715470076422\\
41.4547864202278	-0.24471592738199\\
41.461339718186	-0.244716384480502\\
41.4678929992558	-0.244716841372084\\
41.474446263445	-0.24471729805686\\
41.4809995107614	-0.244717754534956\\
41.4875527412125	-0.244718210806498\\
41.4941059548062	-0.244718666871609\\
41.5006591515501	-0.244719122730416\\
41.5072123314518	-0.244719578383043\\
41.5137654945192	-0.244720033829615\\
41.5203186407599	-0.244720489070257\\
41.5268717701816	-0.244720944105093\\
41.5334248827918	-0.244721398934248\\
41.5399779785985	-0.244721853557845\\
41.546531057609	-0.244722307976011\\
41.5530841198313	-0.244722762188868\\
41.5596371652728	-0.244723216196541\\
41.5661901939413	-0.244723669999154\\
41.5727432058444	-0.24472412359683\\
41.5792962009898	-0.244724576989695\\
41.585849179385	-0.244725030177871\\
41.5924021410378	-0.244725483161481\\
41.5989550859557	-0.244725935940651\\
41.6055080141464	-0.244726388515503\\
41.6120609256175	-0.244726840886161\\
41.6186138203766	-0.244727293052747\\
41.6251666984314	-0.244727745015385\\
41.6317195597893	-0.244728196774199\\
41.6382724044581	-0.24472864832931\\
41.6448252324454	-0.244729099680843\\
41.6513780437586	-0.24472955082892\\
41.6579308384055	-0.244730001773663\\
41.6644836163936	-0.244730452515196\\
41.6710363777304	-0.244730903053641\\
41.6775891224236	-0.244731353389119\\
41.6841418504806	-0.244731803521755\\
41.6906945619092	-0.244732253451669\\
41.6972472567168	-0.244732703178985\\
41.703799934911	-0.244733152703823\\
41.7103525964994	-0.244733602026307\\
41.7169052414894	-0.244734051146558\\
41.7234578698887	-0.244734500064698\\
41.7300104817047	-0.244734948780848\\
41.7365630769451	-0.244735397295131\\
41.7431156556173	-0.244735845607667\\
41.7496682177288	-0.244736293718578\\
41.7562207632872	-0.244736741627987\\
41.7627732923	-0.244737189336012\\
41.7693258047747	-0.244737636842777\\
41.7758783007188	-0.244738084148402\\
41.7824307801398	-0.244738531253008\\
41.7889832430452	-0.244738978156716\\
41.7955356894425	-0.244739424859646\\
41.8020881193392	-0.24473987136192\\
41.8086405327427	-0.244740317663658\\
41.8151929296606	-0.24474076376498\\
41.8217453101002	-0.244741209666008\\
41.8282976740692	-0.24474165536686\\
41.8348500215749	-0.244742100867659\\
41.8414023526247	-0.244742546168523\\
41.8479546672263	-0.244742991269573\\
41.8545069653869	-0.244743436170929\\
41.8610592471141	-0.24474388087271\\
41.8676115124153	-0.244744325375037\\
41.874163761298	-0.244744769678029\\
41.8807159937695	-0.244745213781806\\
41.8872682098372	-0.244745657686488\\
41.8938204095087	-0.244746101392193\\
41.9003725927914	-0.244746544899041\\
41.9069247596925	-0.244746988207152\\
41.9134769102197	-0.244747431316644\\
41.9200290443802	-0.244747874227637\\
41.9265811621814	-0.24474831694025\\
41.9331332636309	-0.244748759454601\\
41.9396853487358	-0.24474920177081\\
41.9462374175037	-0.244749643888995\\
41.952789469942	-0.244750085809274\\
41.9593415060579	-0.244750527531767\\
41.9658935258589	-0.244750969056591\\
41.9724455293523	-0.244751410383866\\
41.9789975165456	-0.244751851513709\\
41.985549487446	-0.244752292446239\\
41.992101442061	-0.244752733181573\\
41.9986533803978	-0.24475317371983\\
42.0052053024639	-0.244753614061127\\
42.0117572082665	-0.244754054205583\\
42.018309097813	-0.244754494153316\\
42.0248609711108	-0.244754933904442\\
42.0314128281671	-0.244755373459079\\
42.0379646689893	-0.244755812817346\\
42.0445164935847	-0.244756251979359\\
42.0510683019607	-0.244756690945235\\
42.0576200941245	-0.244757129715092\\
42.0641718700835	-0.244757568289047\\
42.0707236298449	-0.244758006667218\\
42.077275373416	-0.24475844484972\\
42.0838271008042	-0.244758882836671\\
42.0903788120167	-0.244759320628187\\
42.0969305070609	-0.244759758224386\\
42.1034821859439	-0.244760195625384\\
42.1100338486732	-0.244760632831297\\
42.1165854952558	-0.244761069842242\\
42.1231371256992	-0.244761506658336\\
42.1296887400106	-0.244761943279693\\
42.1362403381972	-0.244762379706432\\
42.1427919202663	-0.244762815938667\\
42.1493434862251	-0.244763251976514\\
42.155895036081	-0.244763687820091\\
42.162446569841	-0.244764123469511\\
42.1689980875125	-0.244764558924892\\
42.1755495891028	-0.244764994186349\\
42.1821010746189	-0.244765429253997\\
42.1886525440682	-0.244765864127952\\
42.1952039974579	-0.244766298808329\\
42.2017554347951	-0.244766733295244\\
42.2083068560872	-0.244767167588812\\
42.2148582613412	-0.244767601689147\\
42.2214096505645	-0.244768035596365\\
42.2279610237642	-0.244768469310581\\
42.2345123809474	-0.24476890283191\\
42.2410637221215	-0.244769336160467\\
42.2476150472935	-0.244769769296366\\
42.2541663564707	-0.244770202239721\\
42.2607176496602	-0.244770634990648\\
42.2672689268692	-0.244771067549261\\
42.2738201881048	-0.244771499915674\\
42.2803714333743	-0.244771932090002\\
42.2869226626848	-0.244772364072358\\
42.2934738760434	-0.244772795862857\\
42.3000250734573	-0.244773227461613\\
42.3065762549337	-0.24477365886874\\
42.3131274204796	-0.244774090084351\\
42.3196785701023	-0.24477452110856\\
42.3262297038088	-0.244774951941482\\
42.3327808216063	-0.244775382583229\\
42.3393319235019	-0.244775813033915\\
42.3458830095027	-0.244776243293654\\
42.3524340796159	-0.244776673362558\\
42.3589851338485	-0.244777103240742\\
42.3655361722077	-0.244777532928318\\
42.3720871947006	-0.2447779624254\\
42.3786382013342	-0.2447783917321\\
42.3851891921156	-0.244778820848531\\
42.3917401670521	-0.244779249774806\\
42.3982911261505	-0.244779678511038\\
42.4048420694181	-0.24478010705734\\
42.4113929968619	-0.244780535413824\\
42.4179439084889	-0.244780963580603\\
42.4244948043063	-0.244781391557789\\
42.431045684321	-0.244781819345494\\
42.4375965485403	-0.244782246943831\\
42.444147396971	-0.244782674352912\\
42.4506982296204	-0.244783101572848\\
42.4572490464953	-0.244783528603753\\
42.4637998476029	-0.244783955445737\\
42.4703506329503	-0.244784382098914\\
42.4769014025443	-0.244784808563393\\
42.4834521563921	-0.244785234839288\\
42.4900028945008	-0.244785660926709\\
42.4965536168772	-0.244786086825768\\
42.5031043235285	-0.244786512536577\\
42.5096550144616	-0.244786938059247\\
42.5162056896836	-0.244787363393889\\
42.5227563492015	-0.244787788540615\\
42.5293069930223	-0.244788213499535\\
42.5358576211529	-0.24478863827076\\
42.5424082336003	-0.244789062854402\\
42.5489588303717	-0.244789487250571\\
42.5555094114738	-0.244789911459378\\
42.5620599769138	-0.244790335480933\\
42.5686105266986	-0.244790759315348\\
42.5751610608351	-0.244791182962733\\
42.5817115793304	-0.244791606423197\\
42.5882620821914	-0.244792029696853\\
42.594812569425	-0.244792452783809\\
42.6013630410382	-0.244792875684176\\
42.6079134970381	-0.244793298398064\\
42.6144639374314	-0.244793720925583\\
42.6210143622252	-0.244794143266844\\
42.6275647714264	-0.244794565421955\\
42.6341151650419	-0.244794987391027\\
42.6406655430787	-0.24479540917417\\
42.6472159055436	-0.244795830771492\\
42.6537662524437	-0.244796252183104\\
42.6603165837859	-0.244796673409115\\
42.6668668995769	-0.244797094449635\\
42.6734171998239	-0.244797515304772\\
42.6799674845335	-0.244797935974636\\
42.6865177537129	-0.244798356459336\\
42.6930680073688	-0.244798776758981\\
42.6996182455082	-0.24479919687368\\
42.7061684681379	-0.244799616803542\\
42.7127186752649	-0.244800036548676\\
42.719268866896	-0.24480045610919\\
42.725819043038	-0.244800875485194\\
42.732369203698	-0.244801294676795\\
42.7389193488827	-0.244801713684102\\
42.7454694785989	-0.244802132507224\\
42.7520195928537	-0.244802551146269\\
42.7585696916537	-0.244802969601345\\
42.765119775006	-0.24480338787256\\
42.7716698429172	-0.244803805960023\\
42.7782198953943	-0.244804223863841\\
42.7847699324441	-0.244804641584123\\
42.7913199540735	-0.244805059120976\\
42.7978699602893	-0.244805476474507\\
42.8044199510982	-0.244805893644826\\
42.8109699265072	-0.244806310632039\\
42.817519886523	-0.244806727436253\\
42.8240698311525	-0.244807144057577\\
42.8306197604025	-0.244807560496118\\
42.8371696742798	-0.244807976751982\\
42.8437195727911	-0.244808392825278\\
42.8502694559434	-0.244808808716112\\
42.8568193237433	-0.244809224424591\\
42.8633691761978	-0.244809639950823\\
42.8699190133135	-0.244810055294913\\
42.8764688350972	-0.24481047045697\\
42.8830186415558	-0.2448108854371\\
42.889568432696	-0.244811300235409\\
42.8961182085246	-0.244811714852005\\
42.9026679690483	-0.244812129286993\\
42.909217714274	-0.244812543540479\\
42.9157674442083	-0.244812957612572\\
42.922317158858	-0.244813371503375\\
42.92886685823	-0.244813785212997\\
42.9354165423308	-0.244814198741543\\
42.9419662111674	-0.244814612089118\\
42.9485158647463	-0.24481502525583\\
42.9550655030744	-0.244815438241783\\
42.9616151261584	-0.244815851047084\\
42.968164734005	-0.244816263671838\\
42.974714326621	-0.244816676116152\\
42.981263904013	-0.244817088380129\\
42.9878134661877	-0.244817500463877\\
42.994363013152	-0.244817912367501\\
43.0009125449124	-0.244818324091105\\
43.0074620614758	-0.244818735634796\\
43.0140115628487	-0.244819146998677\\
43.020561049038	-0.244819558182856\\
43.0271105200502	-0.244819969187435\\
43.0336599758921	-0.244820380012521\\
43.0402094165703	-0.244820790658218\\
43.0467588420916	-0.244821201124631\\
43.0533082524626	-0.244821611411865\\
43.0598576476901	-0.244822021520025\\
43.0664070277805	-0.244822431449214\\
43.0729563927407	-0.244822841199538\\
43.0795057425773	-0.244823250771101\\
43.0860550772969	-0.244823660164008\\
43.0926043969062	-0.244824069378361\\
43.0991537014118	-0.244824478414267\\
43.1057029908205	-0.244824887271828\\
43.1122522651387	-0.244825295951149\\
43.1188015243733	-0.244825704452334\\
43.1253507685307	-0.244826112775487\\
43.1318999976177	-0.244826520920711\\
43.1384492116408	-0.244826928888111\\
43.1449984106067	-0.244827336677789\\
43.151547594522	-0.24482774428985\\
43.1580967633933	-0.244828151724397\\
43.1646459172272	-0.244828558981533\\
43.1711950560303	-0.244828966061362\\
43.1777441798093	-0.244829372963987\\
43.1842932885707	-0.244829779689512\\
43.1908423823211	-0.244830186238039\\
43.1973914610671	-0.244830592609671\\
43.2039405248153	-0.244830998804512\\
43.2104895735723	-0.244831404822664\\
43.2170386073447	-0.24483181066423\\
43.223587626139	-0.244832216329313\\
43.2301366299618	-0.244832621818016\\
43.2366856188197	-0.244833027130441\\
43.2432345927192	-0.24483343226669\\
43.2497835516669	-0.244833837226867\\
43.2563324956693	-0.244834242011073\\
43.2628814247331	-0.244834646619411\\
43.2694303388647	-0.244835051051984\\
43.2759792380707	-0.244835455308892\\
43.2825281223576	-0.244835859390239\\
43.289076991732	-0.244836263296126\\
43.2956258462005	-0.244836667026655\\
43.3021746857694	-0.244837070581928\\
43.3087235104454	-0.244837473962048\\
43.315272320235	-0.244837877167114\\
43.3218211151447	-0.24483828019723\\
43.328369895181	-0.244838683052497\\
43.3349186603504	-0.244839085733016\\
43.3414674106594	-0.244839488238889\\
43.3480161461146	-0.244839890570217\\
43.3545648667224	-0.244840292727101\\
43.3611135724893	-0.244840694709643\\
43.3676622634218	-0.244841096517943\\
43.3742109395264	-0.244841498152102\\
43.3807596008096	-0.244841899612223\\
43.3873082472779	-0.244842300898405\\
43.3938568789377	-0.244842702010749\\
43.4004054957955	-0.244843102949356\\
43.4069540978578	-0.244843503714327\\
43.413502685131	-0.244843904305763\\
43.4200512576216	-0.244844304723763\\
43.4265998153361	-0.244844704968428\\
43.4331483582808	-0.24484510503986\\
43.4396968864624	-0.244845504938157\\
43.4462453998871	-0.244845904663421\\
43.4527938985615	-0.244846304215751\\
43.459342382492	-0.244846703595247\\
43.465890851685	-0.24484710280201\\
43.4724393061469	-0.24484750183614\\
43.4789877458842	-0.244847900697736\\
43.4855361709033	-0.244848299386898\\
43.4920845812106	-0.244848697903726\\
43.4986329768125	-0.24484909624832\\
43.5051813577155	-0.244849494420779\\
43.5117297239259	-0.244849892421203\\
43.5182780754502	-0.24485029024969\\
43.5248264122947	-0.244850687906341\\
43.5313747344658	-0.244851085391255\\
43.53792304197	-0.244851482704531\\
43.5444713348137	-0.244851879846268\\
43.5510196130031	-0.244852276816565\\
43.5575678765448	-0.244852673615522\\
43.564116125445	-0.244853070243237\\
43.5706643597101	-0.244853466699809\\
43.5772125793466	-0.244853862985336\\
43.5837607843608	-0.244854259099918\\
43.590308974759	-0.244854655043654\\
43.5968571505476	-0.244855050816641\\
43.603405311733	-0.244855446418979\\
43.6099534583214	-0.244855841850765\\
43.6165015903194	-0.244856237112098\\
43.6230497077331	-0.244856632203077\\
43.629597810569	-0.2448570271238\\
43.6361458988334	-0.244857421874364\\
43.6426939725326	-0.244857816454868\\
43.6492420316729	-0.24485821086541\\
43.6557900762607	-0.244858605106088\\
43.6623381063023	-0.244858999177\\
43.668886121804	-0.244859393078243\\
43.6754341227721	-0.244859786809915\\
43.681982109213	-0.244860180372114\\
43.6885300811329	-0.244860573764938\\
43.6950780385381	-0.244860966988484\\
43.701625981435	-0.244861360042849\\
43.7081739098298	-0.24486175292813\\
43.7147218237289	-0.244862145644426\\
43.7212697231385	-0.244862538191834\\
43.7278176080649	-0.24486293057045\\
43.7343654785144	-0.244863322780371\\
43.7409133344933	-0.244863714821695\\
43.7474611760078	-0.244864106694519\\
43.7540090030643	-0.244864498398939\\
43.7605568156689	-0.244864889935053\\
43.7671046138279	-0.244865281302956\\
43.7736523975477	-0.244865672502747\\
43.7802001668344	-0.24486606353452\\
43.7867479216944	-0.244866454398374\\
43.7932956621338	-0.244866845094403\\
43.7998433881589	-0.244867235622706\\
43.806391099776	-0.244867625983377\\
43.8129387969912	-0.244868016176514\\
43.8194864798109	-0.244868406202213\\
43.8260341482413	-0.244868796060569\\
43.8325818022885	-0.244869185751678\\
43.8391294419588	-0.244869575275638\\
43.8456770672585	-0.244869964632542\\
43.8522246781937	-0.244870353822489\\
43.8587722747706	-0.244870742845572\\
43.8653198569955	-0.244871131701889\\
43.8718674248746	-0.244871520391534\\
43.878414978414	-0.244871908914603\\
43.8849625176201	-0.244872297271191\\
43.8915100424988	-0.244872685461395\\
43.8980575530566	-0.244873073485309\\
43.9046050492995	-0.244873461343028\\
43.9111525312337	-0.244873849034649\\
43.9176999988654	-0.244874236560266\\
43.9242474522008	-0.244874623919974\\
43.9307948912461	-0.244875011113868\\
43.9373423160073	-0.244875398142043\\
43.9438897264908	-0.244875785004594\\
43.9504371227027	-0.244876171701616\\
43.956984504649	-0.244876558233204\\
43.963531872336	-0.244876944599451\\
43.9700792257699	-0.244877330800454\\
43.9766265649567	-0.244877716836306\\
43.9831738899026	-0.244878102707102\\
43.9897212006138	-0.244878488412936\\
43.9962684970964	-0.244878873953903\\
44.0028157793565	-0.244879259330097\\
44.0093630474003	-0.244879644541612\\
44.0159103012339	-0.244880029588542\\
44.0224575408633	-0.244880414470982\\
44.0290047662948	-0.244880799189025\\
44.0355519775345	-0.244881183742765\\
44.0420991745884	-0.244881568132296\\
44.0486463574627	-0.244881952357712\\
44.0551935261634	-0.244882336419107\\
44.0617406806967	-0.244882720316574\\
44.0682878210688	-0.244883104050207\\
44.0748349472855	-0.2448834876201\\
44.0813820593532	-0.244883871026345\\
44.0879291572778	-0.244884254269037\\
44.0944762410654	-0.244884637348268\\
44.1010233107221	-0.244885020264132\\
44.1075703662541	-0.244885403016723\\
44.1141174076673	-0.244885785606133\\
44.1206644349678	-0.244886168032455\\
44.1272114481618	-0.244886550295782\\
44.1337584472552	-0.244886932396208\\
44.1403054322542	-0.244887314333824\\
44.1468524031647	-0.244887696108725\\
44.1533993599929	-0.244888077721002\\
44.1599463027448	-0.244888459170749\\
44.1664932314264	-0.244888840458057\\
44.1730401460438	-0.24488922158302\\
44.179587046603	-0.24488960254573\\
44.1861339331101	-0.244889983346279\\
44.1926808055711	-0.244890363984759\\
44.1992276639919	-0.244890744461264\\
44.2057745083787	-0.244891124775885\\
44.2123213387375	-0.244891504928714\\
44.2188681550743	-0.244891884919843\\
44.2254149573951	-0.244892264749365\\
44.2319617457058	-0.244892644417371\\
44.2385085200126	-0.244893023923953\\
44.2450552803214	-0.244893403269203\\
44.2516020266383	-0.244893782453213\\
44.2581487589691	-0.244894161476074\\
44.26469547732	-0.244894540337878\\
44.2712421816969	-0.244894919038717\\
44.2777888721057	-0.244895297578682\\
44.2843355485526	-0.244895675957864\\
44.2908822110434	-0.244896054176355\\
44.2974288595841	-0.244896432234246\\
44.3039754941808	-0.244896810131628\\
44.3105221148393	-0.244897187868593\\
44.3170687215657	-0.244897565445231\\
44.3236153143659	-0.244897942861634\\
44.3301618932458	-0.244898320117893\\
44.3367084582115	-0.244898697214098\\
44.3432550092689	-0.24489907415034\\
44.3498015464239	-0.24489945092671\\
44.3563480696824	-0.2448998275433\\
44.3628945790505	-0.244900204000198\\
44.3694410745341	-0.244900580297496\\
44.375987556139	-0.244900956435285\\
44.3825340238713	-0.244901332413655\\
44.3890804777368	-0.244901708232696\\
44.3956269177415	-0.244902083892499\\
44.4021733438913	-0.244902459393154\\
44.4087197561921	-0.244902834734751\\
44.4152661546499	-0.24490320991738\\
44.4218125392705	-0.244903584941131\\
44.4283589100599	-0.244903959806094\\
44.434905267024	-0.24490433451236\\
44.4414516101686	-0.244904709060018\\
44.4479979394997	-0.244905083449157\\
44.4545442550231	-0.244905457679868\\
44.4610905567448	-0.244905831752241\\
44.4676368446707	-0.244906205666364\\
44.4741831188066	-0.244906579422328\\
44.4807293791584	-0.244906953020222\\
44.4872756257319	-0.244907326460135\\
44.4938218585332	-0.244907699742156\\
44.500368077568	-0.244908072866376\\
44.5069142828422	-0.244908445832883\\
44.5134604743616	-0.244908818641767\\
44.5200066521322	-0.244909191293116\\
44.5265528161598	-0.24490956378702\\
44.5330989664502	-0.244909936123567\\
44.5396451030094	-0.244910308302846\\
44.5461912258431	-0.244910680324947\\
44.5527373349571	-0.244911052189958\\
44.5592834303574	-0.244911423897968\\
44.5658295120498	-0.244911795449066\\
44.5723755800401	-0.244912166843339\\
44.5789216343341	-0.244912538080878\\
44.5854676749377	-0.244912909161769\\
44.5920137018567	-0.244913280086102\\
44.5985597150969	-0.244913650853965\\
44.6051057146641	-0.244914021465446\\
44.6116517005642	-0.244914391920633\\
44.6181976728029	-0.244914762219615\\
44.6247436313861	-0.24491513236248\\
44.6312895763196	-0.244915502349315\\
44.6378355076092	-0.244915872180209\\
44.6443814252606	-0.24491624185525\\
44.6509273292797	-0.244916611374526\\
44.6574732196723	-0.244916980738123\\
44.6640190964441	-0.244917349946131\\
44.6705649596009	-0.244917718998636\\
44.6771108091486	-0.244918087895726\\
44.6836566450929	-0.24491845663749\\
44.6902024674395	-0.244918825224013\\
44.6967482761943	-0.244919193655385\\
44.703294071363	-0.244919561931691\\
44.7098398529514	-0.24491993005302\\
44.7163856209652	-0.244920298019458\\
44.7229313754102	-0.244920665831094\\
44.7294771162922	-0.244921033488013\\
44.7360228436169	-0.244921400990303\\
44.7425685573901	-0.244921768338051\\
44.7491142576174	-0.244922135531344\\
44.7556599443047	-0.244922502570268\\
44.7622056174577	-0.244922869454912\\
44.7687512770821	-0.24492323618536\\
44.7752969231837	-0.2449236027617\\
44.7818425557681	-0.244923969184019\\
44.7883881748412	-0.244924335452404\\
44.7949337804086	-0.244924701566939\\
44.8014793724761	-0.244925067527713\\
44.8080249510493	-0.244925433334812\\
44.814570516134	-0.244925798988321\\
44.821116067736	-0.244926164488327\\
44.8276616058608	-0.244926529834917\\
44.8342071305142	-0.244926895028176\\
44.840752641702	-0.24492726006819\\
44.8472981394297	-0.244927624955046\\
44.8538436237032	-0.244927989688829\\
44.860389094528	-0.244928354269625\\
44.8669345519099	-0.244928718697521\\
44.8734799958546	-0.244929082972601\\
44.8800254263677	-0.244929447094953\\
44.886570843455	-0.24492981106466\\
44.893116247122	-0.244930174881809\\
44.8996616373745	-0.244930538546486\\
44.9062070142182	-0.244930902058776\\
44.9127523776586	-0.244931265418764\\
44.9192977277016	-0.244931628626535\\
44.9258430643526	-0.244931991682176\\
44.9323883876174	-0.24493235458577\\
44.9389336975017	-0.244932717337404\\
44.945478994011	-0.244933079937163\\
44.952024277151	-0.24493344238513\\
44.9585695469275	-0.244933804681393\\
44.9651148033459	-0.244934166826034\\
44.971660046412	-0.24493452881914\\
44.9782052761313	-0.244934890660796\\
44.9847504925096	-0.244935252351084\\
44.9912956955524	-0.244935613890092\\
44.9978408852653	-0.244935975277903\\
45.004386061654	-0.244936336514601\\
45.0109312247242	-0.244936697600272\\
45.0174763744813	-0.244937058534999\\
45.024021510931	-0.244937419318868\\
45.030566634079	-0.244937779951962\\
45.0371117439308	-0.244938140434366\\
45.0436568404921	-0.244938500766163\\
45.0502019237684	-0.244938860947439\\
45.0567469937653	-0.244939220978277\\
45.0632920504884	-0.244939580858761\\
45.0698370939433	-0.244939940588976\\
45.0763821241357	-0.244940300169005\\
45.082927141071	-0.244940659598931\\
45.0894721447548	-0.24494101887884\\
45.0960171351928	-0.244941378008814\\
45.1025621123905	-0.244941736988938\\
45.1091070763535	-0.244942095819294\\
45.1156520270873	-0.244942454499967\\
45.1221969645975	-0.24494281303104\\
45.1287418888897	-0.244943171412596\\
45.1352867999694	-0.24494352964472\\
45.1418316978422	-0.244943887727493\\
45.1483765825136	-0.244944245661\\
45.1549214539892	-0.244944603445324\\
45.1614663122745	-0.244944961080547\\
45.1680111573751	-0.244945318566754\\
45.1745559892965	-0.244945675904026\\
45.1811008080442	-0.244946033092448\\
45.1876456136238	-0.244946390132101\\
45.1941904060408	-0.244946747023069\\
45.2007351853007	-0.244947103765435\\
45.207279951409	-0.244947460359282\\
45.2138247043713	-0.244947816804691\\
45.2203694441931	-0.244948173101746\\
45.2269141708799	-0.24494852925053\\
45.2334588844373	-0.244948885251124\\
45.2400035848706	-0.244949241103612\\
45.2465482721854	-0.244949596808075\\
45.2530929463873	-0.244949952364597\\
45.2596376074817	-0.244950307773259\\
45.2661822554741	-0.244950663034143\\
45.27272689037	-0.244951018147333\\
45.2792715121749	-0.244951373112909\\
45.2858161208943	-0.244951727930954\\
45.2923607165337	-0.244952082601551\\
45.2989052990985	-0.24495243712478\\
45.3054498685942	-0.244952791500725\\
45.3119944250263	-0.244953145729466\\
45.3185389684003	-0.244953499811085\\
45.3250834987216	-0.244953853745665\\
45.3316280159957	-0.244954207533286\\
45.3381725202281	-0.244954561174031\\
45.3447170114241	-0.244954914667982\\
45.3512614895894	-0.244955268015218\\
45.3578059547293	-0.244955621215823\\
45.3643504068492	-0.244955974269877\\
45.3708948459547	-0.244956327177462\\
45.3774392720511	-0.244956679938659\\
45.383983685144	-0.244957032553549\\
45.3905280852387	-0.244957385022213\\
45.3970724723406	-0.244957737344733\\
45.4036168464552	-0.24495808952119\\
45.410161207588	-0.244958441551664\\
45.4167055557444	-0.244958793436236\\
45.4232498909297	-0.244959145174988\\
45.4297942131494	-0.244959496767999\\
45.4363385224089	-0.244959848215352\\
45.4428828187136	-0.244960199517127\\
45.449427102069	-0.244960550673403\\
45.4559713724804	-0.244960901684263\\
45.4625156299533	-0.244961252549785\\
45.469059874493	-0.244961603270052\\
45.475604106105	-0.244961953845143\\
45.4821483247945	-0.244962304275139\\
45.4886925305671	-0.24496265456012\\
45.4952367234282	-0.244963004700166\\
45.501780903383	-0.244963354695357\\
45.508325070437	-0.244963704545775\\
45.5148692245956	-0.244964054251497\\
45.5214133658641	-0.244964403812606\\
45.5279574942479	-0.24496475322918\\
45.5345016097524	-0.2449651025013\\
45.5410457123829	-0.244965451629046\\
45.5475898021449	-0.244965800612497\\
45.5541338790437	-0.244966149451733\\
45.5606779430845	-0.244966498146834\\
45.5672219942729	-0.244966846697879\\
45.5737660326141	-0.244967195104949\\
45.5803100581135	-0.244967543368122\\
45.5868540707764	-0.244967891487478\\
45.5933980706082	-0.244968239463097\\
45.5999420576143	-0.244968587295057\\
45.6064860317999	-0.244968934983439\\
45.6130299931704	-0.244969282528321\\
45.6195739417311	-0.244969629929783\\
45.6261178774873	-0.244969977187904\\
45.6326618004445	-0.244970324302763\\
45.6392057106078	-0.244970671274439\\
45.6457496079827	-0.244971018103011\\
45.6522934925744	-0.244971364788558\\
45.6588373643882	-0.244971711331159\\
45.6653812234295	-0.244972057730892\\
45.6719250697036	-0.244972403987837\\
45.6784689032158	-0.244972750102072\\
45.6850127239713	-0.244973096073675\\
45.6915565319755	-0.244973441902726\\
45.6981003272337	-0.244973787589304\\
45.7046441097512	-0.244974133133485\\
45.7111878795331	-0.24497447853535\\
45.717731636585	-0.244974823794976\\
45.7242753809119	-0.244975168912441\\
45.7308191125193	-0.244975513887825\\
45.7373628314124	-0.244975858721204\\
45.7439065375964	-0.244976203412658\\
45.7504502310766	-0.244976547962265\\
45.7569939118584	-0.244976892370102\\
45.7635375799469	-0.244977236636248\\
45.7700812353475	-0.24497758076078\\
45.7766248780653	-0.244977924743776\\
45.7831685081057	-0.244978268585315\\
45.7897121254739	-0.244978612285474\\
45.7962557301752	-0.244978955844331\\
45.8027993222147	-0.244979299261963\\
45.8093429015979	-0.244979642538449\\
45.8158864683298	-0.244979985673865\\
45.8224300224157	-0.24498032866829\\
45.828973563861	-0.2449806715218\\
45.8355170926707	-0.244981014234473\\
45.8420606088502	-0.244981356806388\\
45.8486041124046	-0.24498169923762\\
45.8551476033393	-0.244982041528247\\
45.8616910816593	-0.244982383678346\\
45.86823454737	-0.244982725687995\\
45.8747780004766	-0.244983067557271\\
45.8813214409842	-0.24498340928625\\
45.8878648688981	-0.24498375087501\\
45.8944082842235	-0.244984092323628\\
45.9009516869655	-0.24498443363218\\
45.9074950771295	-0.244984774800744\\
45.9140384547206	-0.244985115829395\\
45.920581819744	-0.244985456718212\\
45.9271251722048	-0.24498579746727\\
45.9336685121083	-0.244986138076647\\
45.9402118394597	-0.244986478546418\\
45.9467551542642	-0.24498681887666\\
45.9532984565269	-0.244987159067451\\
45.959841746253	-0.244987499118865\\
45.9663850234477	-0.24498783903098\\
45.9729282881161	-0.244988178803872\\
45.9794715402635	-0.244988518437617\\
45.986014779895	-0.244988857932292\\
45.9925580070158	-0.244989197287972\\
45.999101221631	-0.244989536504733\\
46.0056444237458	-0.244989875582653\\
46.0121876133653	-0.244990214521806\\
46.0187307904948	-0.244990553322268\\
46.0252739551393	-0.244990891984116\\
46.031817107304	-0.244991230507426\\
46.038360246994	-0.244991568892272\\
46.0449033742146	-0.244991907138732\\
46.0514464889707	-0.24499224524688\\
46.0579895912677	-0.244992583216792\\
46.0645326811105	-0.244992921048545\\
46.0710757585044	-0.244993258742212\\
46.0776188234544	-0.244993596297871\\
46.0841618759657	-0.244993933715595\\
46.0907049160434	-0.244994270995461\\
46.0972479436927	-0.244994608137545\\
46.1037909589186	-0.24499494514192\\
46.1103339617262	-0.244995282008663\\
46.1168769521208	-0.244995618737849\\
46.1234199301073	-0.244995955329552\\
46.1299628956909	-0.244996291783848\\
46.1365058488766	-0.244996628100812\\
46.1430487896697	-0.244996964280519\\
46.1495917180752	-0.244997300323043\\
46.1561346340981	-0.24499763622846\\
46.1626775377436	-0.244997971996845\\
46.1692204290168	-0.244998307628271\\
46.1757633079227	-0.244998643122815\\
46.1823061744664	-0.24499897848055\\
46.1888490286531	-0.244999313701551\\
46.1953918704877	-0.244999648785893\\
46.2019346999754	-0.24499998373365\\
46.2084775171213	-0.245000318544897\\
46.2150203219303	-0.245000653219708\\
46.2215631144076	-0.245000987758157\\
46.2281058945583	-0.245001322160319\\
46.2346486623873	-0.245001656426268\\
46.2411914178998	-0.245001990556078\\
46.2477341611008	-0.245002324549823\\
46.2542768919954	-0.245002658407578\\
46.2608196105886	-0.245002992129416\\
46.2673623168854	-0.245003325715412\\
46.273905010891	-0.245003659165639\\
46.2804476926103	-0.245003992480171\\
46.2869903620484	-0.245004325659083\\
46.2935330192103	-0.245004658702447\\
46.300075664101	-0.245004991610338\\
46.3066182967257	-0.24500532438283\\
46.3131609170892	-0.245005657019996\\
46.3197035251967	-0.245005989521909\\
46.3262461210531	-0.245006321888643\\
46.3327887046636	-0.245006654120273\\
46.339331276033	-0.24500698621687\\
46.3458738351664	-0.245007318178509\\
46.3524163820688	-0.245007650005262\\
46.3589589167453	-0.245007981697204\\
46.3655014392008	-0.245008313254407\\
46.3720439494403	-0.245008644676945\\
46.3785864474689	-0.245008975964891\\
46.3851289332915	-0.245009307118317\\
46.3916714069132	-0.245009638137297\\
46.3982138683388	-0.245009969021905\\
46.4047563175735	-0.245010299772212\\
46.4112987546222	-0.245010630388291\\
46.4178411794898	-0.245010960870217\\
46.4243835921814	-0.24501129121806\\
46.430925992702	-0.245011621431895\\
46.4374683810564	-0.245011951511794\\
46.4440107572498	-0.245012281457829\\
46.450553121287	-0.245012611270073\\
46.457095473173	-0.245012940948599\\
46.4636378129129	-0.245013270493478\\
46.4701801405114	-0.245013599904785\\
46.4767224559737	-0.24501392918259\\
46.4832647593047	-0.245014258326967\\
46.4898070505092	-0.245014587337987\\
46.4963493295924	-0.245014916215724\\
46.5028915965591	-0.245015244960248\\
46.5094338514142	-0.245015573571633\\
46.5159760941627	-0.24501590204995\\
46.5225183248096	-0.245016230395271\\
46.5290605433598	-0.245016558607669\\
46.5356027498182	-0.245016886687215\\
46.5421449441898	-0.245017214633981\\
46.5486871264794	-0.24501754244804\\
46.555229296692	-0.245017870129463\\
46.5617714548326	-0.245018197678321\\
46.5683136009061	-0.245018525094687\\
46.5748557349173	-0.245018852378632\\
46.5813978568712	-0.245019179530228\\
46.5879399667727	-0.245019506549546\\
46.5944820646267	-0.245019833436657\\
46.6010241504382	-0.245020160191635\\
46.607566224212	-0.245020486814549\\
46.614108285953	-0.245020813305471\\
46.6206503356661	-0.245021139664472\\
46.6271923733563	-0.245021465891624\\
46.6337343990284	-0.245021791986999\\
46.6402764126873	-0.245022117950666\\
46.6468184143379	-0.245022443782698\\
46.6533604039851	-0.245022769483165\\
46.6599023816337	-0.245023095052138\\
46.6664443472888	-0.245023420489689\\
46.672986300955	-0.245023745795888\\
46.6795282426374	-0.245024070970807\\
46.6860701723407	-0.245024396014515\\
46.69261209007	-0.245024720927085\\
46.6991539958299	-0.245025045708586\\
46.7056958896254	-0.245025370359089\\
46.7122377714614	-0.245025694878666\\
46.7187796413427	-0.245026019267386\\
46.7253214992741	-0.24502634352532\\
46.7318633452606	-0.245026667652539\\
46.7384051793069	-0.245026991649113\\
46.744947001418	-0.245027315515113\\
46.7514888115986	-0.245027639250608\\
46.7580306098537	-0.24502796285567\\
46.764572396188	-0.245028286330369\\
46.7711141706064	-0.245028609674774\\
46.7776559331137	-0.245028932888956\\
46.7841976837148	-0.245029255972985\\
46.7907394224145	-0.245029578926932\\
46.7972811492176	-0.245029901750865\\
46.803822864129	-0.245030224444856\\
46.8103645671535	-0.245030547008974\\
46.8169062582958	-0.245030869443289\\
46.8234479375609	-0.245031191747871\\
46.8299896049534	-0.24503151392279\\
46.8365312604784	-0.245031835968115\\
46.8430729041404	-0.245032157883916\\
46.8496145359445	-0.245032479670263\\
46.8561561558953	-0.245032801327225\\
46.8626977639976	-0.245033122854873\\
46.8692393602564	-0.245033444253275\\
46.8757809446762	-0.245033765522501\\
46.8823225172621	-0.24503408666262\\
46.8888640780187	-0.245034407673702\\
46.8954056269508	-0.245034728555816\\
46.9019471640632	-0.245035049309032\\
46.9084886893608	-0.245035369933419\\
46.9150302028482	-0.245035690429045\\
46.9215717045304	-0.24503601079598\\
46.9281131944119	-0.245036331034293\\
46.9346546724977	-0.245036651144054\\
46.9411961387925	-0.245036971125331\\
46.947737593301	-0.245037290978192\\
46.954279036028	-0.245037610702708\\
46.9608204669784	-0.245037930298947\\
46.9673618861567	-0.245038249766977\\
46.9739032935679	-0.245038569106868\\
46.9804446892167	-0.245038888318688\\
46.9869860731077	-0.245039207402507\\
46.9935274452459	-0.245039526358391\\
47.0000688056358	-0.245039845186411\\
47.0066101542823	-0.245040163886634\\
47.0131514911901	-0.24504048245913\\
47.019692816364	-0.245040800903966\\
47.0262341298086	-0.245041119221212\\
47.0327754315287	-0.245041437410934\\
47.0393167215291	-0.245041755473203\\
47.0458579998144	-0.245042073408085\\
47.0523992663895	-0.24504239121565\\
47.058940521259	-0.245042708895965\\
47.0654817644276	-0.245043026449098\\
47.0720229959001	-0.245043343875119\\
47.0785642156812	-0.245043661174093\\
47.0851054237756	-0.245043978346091\\
47.091646620188	-0.245044295391179\\
47.0981878049232	-0.245044612309425\\
47.1047289779857	-0.245044929100898\\
47.1112701393804	-0.245045245765665\\
47.117811289112	-0.245045562303794\\
47.124352427185	-0.245045878715352\\
47.1308935536043	-0.245046195000408\\
47.1374346683746	-0.245046511159029\\
47.1439757715004	-0.245046827191282\\
47.1505168629866	-0.245047143097236\\
47.1570579428378	-0.245047458876957\\
47.1635990110586	-0.245047774530513\\
47.1701400676538	-0.245048090057971\\
47.176681112628	-0.2450484054594\\
47.183222145986	-0.245048720734865\\
47.1897631677323	-0.245049035884435\\
47.1963041778717	-0.245049350908176\\
47.2028451764089	-0.245049665806156\\
47.2093861633484	-0.245049980578442\\
47.215927138695	-0.245050295225101\\
47.2224681024533	-0.2450506097462\\
47.229009054628	-0.245050924141806\\
47.2355499952238	-0.245051238411986\\
47.2420909242452	-0.245051552556806\\
47.248631841697	-0.245051866576335\\
47.2551727475837	-0.245052180470638\\
47.2617136419102	-0.245052494239783\\
47.2682545246808	-0.245052807883835\\
47.2747953959005	-0.245053121402863\\
47.2813362555737	-0.245053434796932\\
47.287877103705	-0.245053748066108\\
47.2944179402993	-0.24505406121046\\
47.3009587653609	-0.245054374230052\\
47.3074995788947	-0.245054687124952\\
47.3140403809052	-0.245054999895226\\
47.3205811713971	-0.24505531254094\\
47.3271219503749	-0.245055625062161\\
47.3336627178433	-0.245055937458955\\
47.3402034738069	-0.245056249731388\\
47.3467442182703	-0.245056561879527\\
47.3532849512382	-0.245056873903437\\
47.359825672715	-0.245057185803184\\
47.3663663827056	-0.245057497578836\\
47.3729070812143	-0.245057809230457\\
47.379447768246	-0.245058120758113\\
47.385988443805	-0.245058432161872\\
47.3925291078961	-0.245058743441798\\
47.3990697605239	-0.245059054597957\\
47.4056104016929	-0.245059365630416\\
47.4121510314077	-0.245059676539239\\
47.4186916496729	-0.245059987324493\\
47.4252322564931	-0.245060297986244\\
47.4317728518729	-0.245060608524556\\
47.4383134358169	-0.245060918939496\\
47.4448540083296	-0.245061229231129\\
47.4513945694155	-0.245061539399521\\
47.4579351190794	-0.245061849444736\\
47.4644756573257	-0.245062159366841\\
47.471016184159	-0.245062469165901\\
47.4775566995839	-0.245062778841981\\
47.484097203605	-0.245063088395146\\
47.4906376962267	-0.245063397825462\\
47.4971781774537	-0.245063707132994\\
47.5037186472905	-0.245064016317806\\
47.5102591057417	-0.245064325379965\\
47.5167995528118	-0.245064634319535\\
47.5233399885054	-0.245064943136581\\
47.5298804128269	-0.245065251831168\\
47.5364208257811	-0.245065560403361\\
47.5429612273723	-0.245065868853225\\
47.5495016176052	-0.245066177180825\\
47.5560419964842	-0.245066485386226\\
47.5625823640139	-0.245066793469491\\
47.5691227201989	-0.245067101430687\\
47.5756630650436	-0.245067409269878\\
47.5822033985526	-0.245067716987128\\
47.5887437207304	-0.245068024582502\\
47.5952840315815	-0.245068332056064\\
47.6018243311105	-0.24506863940788\\
47.6083646193218	-0.245068946638012\\
47.6149048962201	-0.245069253746527\\
47.6214451618097	-0.245069560733488\\
47.6279854160952	-0.24506986759896\\
47.6345256590811	-0.245070174343006\\
47.641065890772	-0.245070480965691\\
47.6476061111722	-0.24507078746708\\
47.6541463202863	-0.245071093847236\\
47.6606865181189	-0.245071400106224\\
47.6672267046744	-0.245071706244108\\
47.6737668799572	-0.245072012260951\\
47.680307043972	-0.245072318156818\\
47.6868471967231	-0.245072623931772\\
47.693387338215	-0.245072929585878\\
47.6999274684523	-0.245073235119199\\
47.7064675874395	-0.2450735405318\\
47.7130076951809	-0.245073845823743\\
47.7195477916811	-0.245074150995094\\
47.7260878769445	-0.245074456045914\\
47.7326279509757	-0.245074760976269\\
47.739168013779	-0.245075065786222\\
47.745708065359	-0.245075370475835\\
47.7522481057201	-0.245075675045174\\
47.7587881348668	-0.2450759794943\\
47.7653281528035	-0.245076283823279\\
47.7718681595347	-0.245076588032172\\
47.7784081550649	-0.245076892121044\\
47.7849481393984	-0.245077196089958\\
47.7914881125398	-0.245077499938977\\
47.7980280744935	-0.245077803668164\\
47.8045680252639	-0.245078107277582\\
47.8111079648555	-0.245078410767295\\
47.8176478932727	-0.245078714137366\\
47.8241878105199	-0.245079017387858\\
47.8307277166017	-0.245079320518833\\
47.8372676115224	-0.245079623530355\\
47.8438074952864	-0.245079926422487\\
47.8503473678982	-0.245080229195291\\
47.8568872293622	-0.245080531848831\\
47.8634270796829	-0.24508083438317\\
47.8699669188646	-0.245081136798369\\
47.8765067469118	-0.245081439094492\\
47.8830465638288	-0.245081741271602\\
47.8895863696202	-0.24508204332976\\
47.8961261642903	-0.245082345269031\\
47.9026659478435	-0.245082647089476\\
47.9092057202842	-0.245082948791157\\
47.9157454816168	-0.245083250374138\\
47.9222852318458	-0.245083551838481\\
47.9288249709755	-0.245083853184248\\
47.9353646990104	-0.245084154411501\\
47.9419044159548	-0.245084455520303\\
47.9484441218131	-0.245084756510716\\
47.9549838165897	-0.245085057382803\\
47.961523500289	-0.245085358136625\\
47.9680631729154	-0.245085658772245\\
47.9746028344732	-0.245085959289725\\
47.981142484967	-0.245086259689127\\
47.9876821244009	-0.245086559970512\\
47.9942217527794	-0.245086860133944\\
48.000761370107	-0.245087160179483\\
48.0073009763878	-0.245087460107193\\
48.0138405716264	-0.245087759917133\\
48.0203801558271	-0.245088059609368\\
48.0269197289943	-0.245088359183957\\
48.0334592911322	-0.245088658640964\\
48.0399988422453	-0.245088957980449\\
48.046538382338	-0.245089257202475\\
48.0530779114145	-0.245089556307103\\
48.0596174294793	-0.245089855294394\\
48.0661569365367	-0.24509015416441\\
48.0726964325911	-0.245090452917214\\
48.0792359176467	-0.245090751552865\\
48.0857753917079	-0.245091050071426\\
48.0923148547792	-0.245091348472957\\
48.0988543068647	-0.245091646757521\\
48.1053937479689	-0.245091944925179\\
48.1119331780961	-0.245092242975991\\
48.1184725972506	-0.245092540910019\\
48.1250120054368	-0.245092838727325\\
48.1315514026589	-0.245093136427969\\
48.1380907889213	-0.245093434012013\\
48.1446301642284	-0.245093731479517\\
48.1511695285844	-0.245094028830542\\
48.1577088819937	-0.24509432606515\\
48.1642482244606	-0.245094623183402\\
48.1707875559894	-0.245094920185358\\
48.1773268765844	-0.245095217071079\\
48.1838661862499	-0.245095513840626\\
48.1904054849902	-0.24509581049406\\
48.1969447728097	-0.245096107031442\\
48.2034840497126	-0.245096403452832\\
48.2100233157033	-0.245096699758291\\
48.2165625707859	-0.245096995947879\\
48.223101814965	-0.245097292021658\\
48.2296410482446	-0.245097587979687\\
48.2361802706292	-0.245097883822027\\
48.242719482123	-0.245098179548739\\
48.2492586827303	-0.245098475159884\\
48.2557978724553	-0.24509877065552\\
48.2623370513025	-0.24509906603571\\
48.2688762192759	-0.245099361300512\\
48.27541537638	-0.245099656449988\\
48.281954522619	-0.245099951484198\\
48.2884936579972	-0.245100246403201\\
48.2950327825188	-0.245100541207059\\
48.3015718961881	-0.24510083589583\\
48.3081109990095	-0.245101130469575\\
48.314650090987	-0.245101424928355\\
48.3211891721251	-0.245101719272229\\
48.3277282424279	-0.245102013501256\\
48.3342673018997	-0.245102307615498\\
48.3408063505449	-0.245102601615014\\
48.3473453883676	-0.245102895499863\\
48.353884415372	-0.245103189270106\\
48.3604234315625	-0.245103482925802\\
48.3669624369433	-0.245103776467011\\
48.3735014315186	-0.245104069893793\\
48.3800404152926	-0.245104363206207\\
48.3865793882697	-0.245104656404313\\
48.393118350454	-0.24510494948817\\
48.3996573018498	-0.245105242457839\\
48.4061962424614	-0.245105535313378\\
48.4127351722928	-0.245105828054846\\
48.4192740913485	-0.245106120682304\\
48.4258129996325	-0.245106413195811\\
48.4323518971492	-0.245106705595425\\
48.4388907839027	-0.245106997881207\\
48.4454296598973	-0.245107290053215\\
48.4519685251372	-0.24510758211151\\
48.4585073796266	-0.245107874056149\\
48.4650462233697	-0.245108165887192\\
48.4715850563707	-0.245108457604698\\
48.4781238786339	-0.245108749208726\\
48.4846626901634	-0.245109040699336\\
48.4912014909635	-0.245109332076586\\
48.4977402810383	-0.245109623340534\\
48.504279060392	-0.245109914491241\\
48.5108178290289	-0.245110205528765\\
48.5173565869532	-0.245110496453165\\
48.523895334169	-0.245110787264499\\
48.5304340706805	-0.245111077962826\\
48.5369727964919	-0.245111368548206\\
48.5435115116074	-0.245111659020696\\
48.5500502160313	-0.245111949380355\\
48.5565889097676	-0.245112239627242\\
48.5631275928205	-0.245112529761415\\
48.5696662651943	-0.245112819782934\\
48.5762049268931	-0.245113109691856\\
48.5827435779211	-0.24511339948824\\
48.5892822182824	-0.245113689172144\\
48.5958208479813	-0.245113978743627\\
48.6023594670218	-0.245114268202747\\
48.6088980754082	-0.245114557549562\\
48.6154366731446	-0.24511484678413\\
48.6219752602352	-0.245115135906511\\
48.6285138366842	-0.245115424916761\\
48.6350524024956	-0.245115713814939\\
48.6415909576737	-0.245116002601104\\
48.6481295022225	-0.245116291275312\\
48.6546680361464	-0.245116579837624\\
48.6612065594493	-0.245116868288095\\
48.6677450721354	-0.245117156626784\\
48.674283574209	-0.24511744485375\\
48.6808220656741	-0.24511773296905\\
48.6873605465348	-0.245118020972741\\
48.6938990167953	-0.245118308864882\\
48.7004374764598	-0.245118596645531\\
48.7069759255324	-0.245118884314744\\
48.7135143640171	-0.245119171872581\\
48.7200527919182	-0.245119459319098\\
48.7265912092398	-0.245119746654353\\
48.7331296159859	-0.245120033878404\\
48.7396680121607	-0.245120320991308\\
48.7462063977683	-0.245120607993122\\
48.7527447728129	-0.245120894883905\\
48.7592831372985	-0.245121181663713\\
48.7658214912293	-0.245121468332605\\
48.7723598346094	-0.245121754890637\\
48.7788981674429	-0.245122041337866\\
48.7854364897338	-0.245122327674351\\
48.7919748014864	-0.245122613900147\\
48.7985131027046	-0.245122900015313\\
48.8050513933927	-0.245123186019906\\
48.8115896735547	-0.245123471913982\\
48.8181279431946	-0.245123757697599\\
48.8246662023166	-0.245124043370814\\
48.8312044509249	-0.245124328933683\\
48.8377426890233	-0.245124614386265\\
48.8442809166162	-0.245124899728615\\
48.8508191337075	-0.245125184960791\\
48.8573573403013	-0.24512547008285\\
48.8638955364017	-0.245125755094848\\
48.8704337220128	-0.245126039996842\\
48.8769718971387	-0.245126324788889\\
48.8835100617834	-0.245126609471046\\
48.890048215951	-0.24512689404337\\
48.8965863596456	-0.245127178505916\\
48.9031244928713	-0.245127462858742\\
48.9096626156321	-0.245127747101904\\
48.916200727932	-0.245128031235459\\
48.9227388297752	-0.245128315259463\\
48.9292769211657	-0.245128599173973\\
48.9358150021076	-0.245128882979046\\
48.9423530726048	-0.245129166674737\\
48.9488911326616	-0.245129450261102\\
48.9554291822818	-0.245129733738199\\
48.9619672214697	-0.245130017106084\\
48.9685052502291	-0.245130300364812\\
48.9750432685642	-0.245130583514441\\
48.9815812764791	-0.245130866555025\\
48.9881192739776	-0.245131149486622\\
48.994657261064	-0.245131432309287\\
49.0011952377422	-0.245131715023077\\
49.0077332040162	-0.245131997628048\\
49.0142711598901	-0.245132280124255\\
49.020809105368	-0.245132562511754\\
49.0273470404538	-0.245132844790602\\
49.0338849651515	-0.245133126960854\\
49.0404228794653	-0.245133409022567\\
49.0469607833991	-0.245133690975795\\
49.0534986769569	-0.245133972820595\\
49.0600365601428	-0.245134254557023\\
49.0665744329608	-0.245134536185134\\
49.0731122954149	-0.245134817704984\\
49.079650147509	-0.245135099116628\\
49.0861879892473	-0.245135380420122\\
49.0927258206337	-0.245135661615523\\
49.0992636416722	-0.245135942702884\\
49.1058014523668	-0.245136223682262\\
49.1123392527216	-0.245136504553713\\
49.1188770427405	-0.24513678531729\\
49.1254148224276	-0.245137065973051\\
49.1319525917867	-0.245137346521051\\
49.138490350822	-0.245137626961344\\
49.1450280995373	-0.245137907293986\\
49.1515658379368	-0.245138187519032\\
49.1581035660243	-0.245138467636537\\
49.1646412838039	-0.245138747646557\\
49.1711789912796	-0.245139027549147\\
49.1777166884553	-0.245139307344362\\
49.184254375335	-0.245139587032256\\
49.1907920519227	-0.245139866612886\\
49.1973297182223	-0.245140146086305\\
49.2038673742379	-0.245140425452569\\
49.2104050199733	-0.245140704711733\\
49.2169426554327	-0.245140983863852\\
49.2234802806199	-0.24514126290898\\
49.2300178955388	-0.245141541847173\\
49.2365555001936	-0.245141820678484\\
49.243093094588	-0.24514209940297\\
49.2496306787262	-0.245142378020684\\
49.256168252612	-0.245142656531682\\
49.2627058162493	-0.245142934936017\\
49.2692433696422	-0.245143213233745\\
49.2757809127947	-0.24514349142492\\
49.2823184457105	-0.245143769509597\\
49.2888559683937	-0.24514404748783\\
49.2953934808483	-0.245144325359674\\
49.3019309830781	-0.245144603125183\\
49.3084684750872	-0.245144880784411\\
49.3150059568794	-0.245145158337413\\
49.3215434284587	-0.245145435784243\\
49.328080889829	-0.245145713124956\\
49.3346183409942	-0.245145990359605\\
49.3411557819584	-0.245146267488245\\
49.3476932127253	-0.245146544510931\\
49.354230633299	-0.245146821427715\\
49.3607680436834	-0.245147098238654\\
49.3673054438823	-0.245147374943799\\
49.3738428338997	-0.245147651543206\\
49.3803802137395	-0.245147928036929\\
49.3869175834057	-0.245148204425022\\
49.3934549429021	-0.245148480707537\\
49.3999922922326	-0.245148756884531\\
49.4065296314013	-0.245149032956055\\
49.4130669604119	-0.245149308922165\\
49.4196042792684	-0.245149584782913\\
49.4261415879746	-0.245149860538355\\
49.4326788865346	-0.245150136188543\\
49.4392161749521	-0.245150411733531\\
49.4457534532311	-0.245150687173372\\
49.4522907213755	-0.245150962508122\\
49.4588279793891	-0.245151237737832\\
49.465365227276	-0.245151512862557\\
49.4719024650398	-0.245151787882351\\
49.4784396926846	-0.245152062797266\\
49.4849769102142	-0.245152337607356\\
49.4915141176326	-0.245152612312674\\
49.4980513149435	-0.245152886913275\\
49.5045885021508	-0.245153161409211\\
49.5111256792586	-0.245153435800536\\
49.5176628462705	-0.245153710087303\\
49.5242000031905	-0.245153984269565\\
49.5307371500225	-0.245154258347375\\
49.5372742867703	-0.245154532320787\\
49.5438114134379	-0.245154806189854\\
49.550348530029	-0.245155079954629\\
49.5568856365475	-0.245155353615165\\
49.5634227329973	-0.245155627171516\\
49.5699598193823	-0.245155900623733\\
49.5764968957062	-0.24515617397187\\
49.5830339619731	-0.245156447215981\\
49.5895710181866	-0.245156720356117\\
49.5961080643507	-0.245156993392333\\
49.6026451004692	-0.24515726632468\\
49.609182126546	-0.245157539153211\\
49.615719142585	-0.245157811877981\\
49.6222561485898	-0.24515808449904\\
49.6287931445645	-0.245158357016442\\
49.6353301305128	-0.245158629430239\\
49.6418671064385	-0.245158901740485\\
49.6484040723456	-0.245159173947232\\
49.6549410282378	-0.245159446050532\\
49.661477974119	-0.245159718050438\\
49.668014909993	-0.245159989947002\\
49.6745518358637	-0.245160261740278\\
49.6810887517348	-0.245160533430316\\
49.6876256576101	-0.245160805017171\\
49.6941625534936	-0.245161076500894\\
49.700699439389	-0.245161347881537\\
49.7072363153002	-0.245161619159154\\
49.7137731812309	-0.245161890333795\\
49.720310037185	-0.245162161405514\\
49.7268468831662	-0.245162432374363\\
49.7333837191785	-0.245162703240393\\
49.7399205452256	-0.245162974003658\\
49.7464573613112	-0.245163244664208\\
49.7529941674393	-0.245163515222097\\
49.7595309636136	-0.245163785677376\\
49.766067749838	-0.245164056030097\\
49.7726045261161	-0.245164326280313\\
49.7791412924519	-0.245164596428075\\
49.7856780488491	-0.245164866473435\\
49.7922147953114	-0.245165136416445\\
49.7987515318428	-0.245165406257157\\
49.8052882584469	-0.245165675995622\\
49.8118249751277	-0.245165945631893\\
49.8183616818887	-0.245166215166022\\
49.8248983787339	-0.245166484598059\\
49.8314350656671	-0.245166753928057\\
49.8379717426919	-0.245167023156068\\
49.8445084098122	-0.245167292282142\\
49.8510450670317	-0.245167561306332\\
49.8575817143543	-0.245167830228689\\
49.8641183517836	-0.245168099049265\\
49.8706549793236	-0.245168367768111\\
49.8771915969778	-0.245168636385279\\
49.8837282047502	-0.245168904900819\\
49.8902648026444	-0.245169173314784\\
49.8968013906642	-0.245169441627225\\
49.9033379688134	-0.245169709838193\\
49.9098745370958	-0.245169977947739\\
49.9164110955151	-0.245170245955915\\
49.922947644075	-0.245170513862773\\
49.9294841827793	-0.245170781668362\\
49.9360207116318	-0.245171049372734\\
49.9425572306362	-0.245171316975941\\
49.9490937397962	-0.245171584478034\\
49.9556302391156	-0.245171851879063\\
49.9621667285982	-0.24517211917908\\
49.9687032082477	-0.245172386378136\\
49.9752396780677	-0.245172653476281\\
49.9817761380621	-0.245172920473567\\
49.9883125882347	-0.245173187370044\\
49.994849028589	-0.245173454165764\\
50.0013854591289	-0.245173720860777\\
50.0079218798581	-0.245173987455134\\
50.0144582907802	-0.245174253948886\\
50.0209946918992	-0.245174520342083\\
50.0275310832185	-0.245174786634777\\
50.0340674647421	-0.245175052827018\\
50.0406038364736	-0.245175318918856\\
50.0471401984167	-0.245175584910343\\
50.0536765505751	-0.245175850801529\\
50.0602128929526	-0.245176116592464\\
50.0667492255529	-0.245176382283199\\
50.0732855483796	-0.245176647873785\\
50.0798218614365	-0.245176913364271\\
50.0863581647273	-0.245177178754709\\
50.0928944582557	-0.245177444045148\\
50.0994307420254	-0.24517770923564\\
50.1059670160401	-0.245177974326234\\
50.1125032803035	-0.245178239316981\\
50.1190395348193	-0.245178504207931\\
50.1255757795912	-0.245178768999134\\
50.1321120146229	-0.24517903369064\\
50.138648239918	-0.245179298282501\\
50.1451844554804	-0.245179562774765\\
50.1517206613136	-0.245179827167483\\
50.1582568574213	-0.245180091460705\\
50.1647930438072	-0.245180355654481\\
50.1713292204751	-0.245180619748861\\
50.1778653874285	-0.245180883743895\\
50.1844015446713	-0.245181147639633\\
50.1909376922069	-0.245181411436125\\
50.1974738300392	-0.245181675133421\\
50.2040099581718	-0.245181938731571\\
50.2105460766083	-0.245182202230624\\
50.2170821853525	-0.24518246563063\\
50.223618284408	-0.24518272893164\\
50.2301543737784	-0.245182992133702\\
50.2366904534675	-0.245183255236867\\
50.2432265234789	-0.245183518241184\\
50.2497625838162	-0.245183781146703\\
50.2562986344831	-0.245184043953474\\
50.2628346754833	-0.245184306661545\\
50.2693707068204	-0.245184569270967\\
50.2759067284981	-0.24518483178179\\
50.2824427405201	-0.245185094194061\\
50.2889787428899	-0.245185356507832\\
50.2955147356112	-0.245185618723151\\
50.3020507186877	-0.245185880840069\\
50.308586692123	-0.245186142858633\\
50.3151226559207	-0.245186404778894\\
50.3216586100846	-0.245186666600901\\
50.3281945546181	-0.245186928324703\\
50.3347304895251	-0.245187189950349\\
50.341266414809	-0.245187451477889\\
50.3478023304736	-0.245187712907372\\
50.3543382365224	-0.245187974238846\\
50.3608741329592	-0.245188235472362\\
50.3674100197874	-0.245188496607967\\
50.3739458970108	-0.245188757645712\\
50.380481764633	-0.245189018585645\\
50.3870176226575	-0.245189279427815\\
50.3935534710881	-0.245189540172271\\
50.4000893099283	-0.245189800819063\\
50.4066251391817	-0.245190061368238\\
50.413160958852	-0.245190321819846\\
50.4196967689427	-0.245190582173936\\
50.4262325694576	-0.245190842430556\\
50.4327683604001	-0.245191102589755\\
50.4393041417739	-0.245191362651583\\
50.4458399135826	-0.245191622616087\\
50.4523756758298	-0.245191882483317\\
50.4589114285192	-0.24519214225332\\
50.4654471716542	-0.245192401926146\\
50.4719829052385	-0.245192661501844\\
50.4785186292758	-0.245192920980461\\
50.4850543437695	-0.245193180362047\\
50.4915900487233	-0.245193439646649\\
50.4981257441408	-0.245193698834317\\
50.5046614300255	-0.245193957925099\\
50.5111971063811	-0.245194216919043\\
50.5177327732111	-0.245194475816197\\
50.5242684305192	-0.245194734616611\\
50.5308040783088	-0.245194993320331\\
50.5373397165836	-0.245195251927407\\
50.5438753453472	-0.245195510437887\\
50.5504109646031	-0.245195768851819\\
50.556946574355	-0.245196027169251\\
50.5634821746063	-0.245196285390231\\
50.5700177653606	-0.245196543514808\\
50.5765533466216	-0.24519680154303\\
50.5830889183928	-0.245197059474944\\
50.5896244806777	-0.2451973173106\\
50.5961600334799	-0.245197575050043\\
50.602695576803	-0.245197832693324\\
50.6092311106506	-0.245198090240489\\
50.6157666350261	-0.245198347691587\\
50.6223021499332	-0.245198605046666\\
50.6288376553754	-0.245198862305773\\
50.6353731513563	-0.245199119468956\\
50.6419086378794	-0.245199376536264\\
50.6484441149482	-0.245199633507743\\
50.6549795825663	-0.245199890383442\\
50.6615150407373	-0.245200147163408\\
50.6680504894647	-0.245200403847689\\
50.674585928752	-0.245200660436333\\
50.6811213586028	-0.245200916929387\\
50.6876567790206	-0.245201173326899\\
50.694192190009	-0.245201429628916\\
50.7007275915715	-0.245201685835487\\
50.7072629837115	-0.245201941946658\\
50.7137983664328	-0.245202197962477\\
50.7203337397387	-0.245202453882991\\
50.7268691036328	-0.245202709708249\\
50.7334044581187	-0.245202965438296\\
50.7399398031998	-0.245203221073182\\
50.7464751388797	-0.245203476612952\\
50.7530104651619	-0.245203732057654\\
50.7595457820499	-0.245203987407336\\
50.7660810895473	-0.245204242662045\\
50.7726163876576	-0.245204497821828\\
50.7791516763842	-0.245204752886732\\
50.7856869557306	-0.245205007856805\\
50.7922222257005	-0.245205262732093\\
50.7987574862973	-0.245205517512643\\
50.8052927375245	-0.245205772198503\\
50.8118279793856	-0.24520602678972\\
50.8183632118841	-0.245206281286341\\
50.8248984350235	-0.245206535688412\\
50.8314336488073	-0.245206789995981\\
50.8379688532391	-0.245207044209094\\
50.8445040483223	-0.245207298327799\\
50.8510392340603	-0.245207552352143\\
50.8575744104568	-0.245207806282171\\
50.8641095775152	-0.245208060117931\\
50.8706447352389	-0.245208313859471\\
50.8771798836315	-0.245208567506835\\
50.8837150226964	-0.245208821060072\\
50.8902501524372	-0.245209074519228\\
50.8967852728573	-0.245209327884349\\
50.9033203839602	-0.245209581155482\\
50.9098554857494	-0.245209834332675\\
50.9163905782283	-0.245210087415972\\
50.9229256614005	-0.245210340405422\\
50.9294607352693	-0.24521059330107\\
50.9359957998384	-0.245210846102963\\
50.9425308551111	-0.245211098811147\\
50.9490659010908	-0.245211351425669\\
50.9556009377812	-0.245211603946575\\
50.9621359651856	-0.245211856373912\\
50.9686709833075	-0.245212108707726\\
50.9752059921504	-0.245212360948063\\
50.9817409917177	-0.245212613094969\\
50.9882759820129	-0.245212865148492\\
50.9948109630395	-0.245213117108676\\
51.0013459348008	-0.245213368975569\\
51.0078808973004	-0.245213620749216\\
51.0144158505416	-0.245213872429663\\
51.0209507945281	-0.245214124016957\\
51.0274857292631	-0.245214375511144\\
51.0340206547502	-0.245214626912269\\
51.0405555709927	-0.24521487822038\\
51.0470904779942	-0.245215129435521\\
51.0536253757581	-0.245215380557739\\
51.0601602642877	-0.245215631587079\\
51.0666951435866	-0.245215882523589\\
51.0732300136582	-0.245216133367313\\
51.0797648745059	-0.245216384118297\\
51.0862997261332	-0.245216634776587\\
51.0928345685434	-0.24521688534223\\
51.0993694017401	-0.24521713581527\\
51.1059042257266	-0.245217386195753\\
51.1124390405063	-0.245217636483726\\
51.1189738460827	-0.245217886679234\\
51.1255086424592	-0.245218136782323\\
51.1320434296393	-0.245218386793037\\
51.1385782076263	-0.245218636711424\\
51.1451129764236	-0.245218886537528\\
51.1516477360347	-0.245219136271395\\
51.1581824864631	-0.24521938591307\\
51.164717227712	-0.245219635462599\\
51.1712519597849	-0.245219884920028\\
51.1777866826852	-0.245220134285401\\
51.1843213964164	-0.245220383558765\\
51.1908561009818	-0.245220632740164\\
51.1973907963849	-0.245220881829644\\
51.2039254826289	-0.245221130827251\\
51.2104601597174	-0.245221379733029\\
51.2169948276538	-0.245221628547024\\
51.2235294864413	-0.245221877269281\\
51.2300641360835	-0.245222125899845\\
51.2365987765837	-0.245222374438762\\
51.2431334079453	-0.245222622886077\\
51.2496680301718	-0.245222871241834\\
51.2562026432664	-0.245223119506079\\
51.2627372472325	-0.245223367678857\\
51.2692718420737	-0.245223615760213\\
51.2758064277931	-0.245223863750192\\
51.2823410043943	-0.245224111648839\\
51.2888755718806	-0.245224359456199\\
51.2954101302554	-0.245224607172317\\
51.3019446795221	-0.245224854797237\\
51.3084792196839	-0.245225102331005\\
51.3150137507444	-0.245225349773666\\
51.3215482727068	-0.245225597125264\\
51.3280827855746	-0.245225844385844\\
51.3346172893511	-0.245226091555451\\
51.3411517840397	-0.245226338634129\\
51.3476862696437	-0.245226585621924\\
51.3542207461665	-0.24522683251888\\
51.3607552136115	-0.245227079325041\\
51.3672896719821	-0.245227326040453\\
51.3738241212815	-0.24522757266516\\
51.3803585615131	-0.245227819199206\\
51.3868929926804	-0.245228065642636\\
51.3934274147866	-0.245228311995494\\
51.3999618278351	-0.245228558257826\\
51.4064962318292	-0.245228804429675\\
51.4130306267723	-0.245229050511086\\
51.4195650126678	-0.245229296502103\\
51.426099389519	-0.245229542402771\\
51.4326337573292	-0.245229788213134\\
51.4391681161018	-0.245230033933236\\
51.4457024658401	-0.245230279563122\\
51.4522368065474	-0.245230525102836\\
51.4587711382271	-0.245230770552422\\
51.4653054608826	-0.245231015911925\\
51.4718397745171	-0.245231261181388\\
51.478374079134	-0.245231506360856\\
51.4849083747366	-0.245231751450372\\
51.4914426613282	-0.245231996449982\\
51.4979769389123	-0.245232241359729\\
51.504511207492	-0.245232486179657\\
51.5110454670707	-0.24523273090981\\
51.5175797176518	-0.245232975550232\\
51.5241139592385	-0.245233220100968\\
51.5306481918343	-0.24523346456206\\
51.5371824154423	-0.245233708933554\\
51.5437166300659	-0.245233953215492\\
51.5502508357084	-0.245234197407919\\
51.5567850323732	-0.245234441510879\\
51.5633192200635	-0.245234685524415\\
51.5698533987827	-0.245234929448572\\
51.5763875685341	-0.245235173283392\\
51.5829217293209	-0.24523541702892\\
51.5894558811464	-0.2452356606852\\
51.5959900240141	-0.245235904252274\\
51.6025241579271	-0.245236147730187\\
51.6090582828888	-0.245236391118983\\
51.6155923989025	-0.245236634418704\\
51.6221265059714	-0.245236877629395\\
51.6286606040989	-0.245237120751099\\
51.6351946932882	-0.24523736378386\\
51.6417287735427	-0.24523760672772\\
51.6482628448656	-0.245237849582725\\
51.6547969072602	-0.245238092348916\\
51.6613309607298	-0.245238335026337\\
51.6678650052777	-0.245238577615032\\
51.6743990409072	-0.245238820115045\\
51.6809330676215	-0.245239062526418\\
51.6874670854239	-0.245239304849194\\
51.6940010943178	-0.245239547083418\\
51.7005350943063	-0.245239789229132\\
51.7070690853928	-0.245240031286379\\
51.7136030675805	-0.245240273255203\\
51.7201370408727	-0.245240515135647\\
51.7266710052727	-0.245240756927754\\
51.7332049607838	-0.245240998631567\\
51.7397389074091	-0.245241240247129\\
51.746272845152	-0.245241481774484\\
51.7528067740157	-0.245241723213674\\
51.7593406940035	-0.245241964564742\\
51.7658746051187	-0.245242205827732\\
51.7724085073645	-0.245242447002685\\
51.7789424007442	-0.245242688089647\\
51.785476285261	-0.245242929088658\\
51.7920101609181	-0.245243169999762\\
51.7985440277189	-0.245243410823002\\
51.8050778856666	-0.245243651558421\\
51.8116117347644	-0.245243892206062\\
51.8181455750156	-0.245244132765966\\
51.8246794064234	-0.245244373238178\\
51.831213228991	-0.24524461362274\\
51.8377470427218	-0.245244853919694\\
51.8442808476189	-0.245245094129083\\
51.8508146436856	-0.24524533425095\\
51.8573484309251	-0.245245574285337\\
51.8638822093407	-0.245245814232287\\
51.8704159789356	-0.245246054091843\\
51.876949739713	-0.245246293864047\\
51.8834834916762	-0.245246533548942\\
51.8900172348283	-0.24524677314657\\
51.8965509691727	-0.245247012656973\\
51.9030846947125	-0.245247252080194\\
51.909618411451	-0.245247491416276\\
51.9161521193914	-0.245247730665261\\
51.9226858185369	-0.245247969827191\\
51.9292195088907	-0.245248208902108\\
51.9357531904561	-0.245248447890055\\
51.9422868632362	-0.245248686791075\\
51.9488205272344	-0.245248925605208\\
51.9553541824537	-0.245249164332499\\
51.9618878288974	-0.245249402972988\\
51.9684214665688	-0.245249641526718\\
51.974955095471	-0.245249879993732\\
51.9814887156072	-0.24525011837407\\
51.9880223269807	-0.245250356667776\\
51.9945559295946	-0.245250594874892\\
52.0010895234522	-0.245250832995459\\
52.0076231085566	-0.24525107102952\\
52.0141566849111	-0.245251308977116\\
52.0206902525188	-0.24525154683829\\
52.027223811383	-0.245251784613083\\
52.0337573615068	-0.245252022301538\\
52.0402909028935	-0.245252259903696\\
52.0468244355462	-0.245252497419599\\
52.0533579594681	-0.245252734849289\\
52.0598914746624	-0.245252972192808\\
52.0664249811323	-0.245253209450198\\
52.0729584788809	-0.2452534466215\\
52.0794919679116	-0.245253683706756\\
52.0860254482274	-0.245253920706009\\
52.0925589198315	-0.245254157619299\\
52.0990923827271	-0.245254394446668\\
52.1056258369174	-0.245254631188158\\
52.1121592824056	-0.24525486784381\\
52.1186927191948	-0.245255104413667\\
52.1252261472882	-0.24525534089777\\
52.131759566689	-0.245255577296159\\
52.1382929774003	-0.245255813608878\\
52.1448263794254	-0.245256049835967\\
52.1513597727673	-0.245256285977467\\
52.1578931574293	-0.245256522033421\\
52.1644265334145	-0.245256758003869\\
52.1709599007261	-0.245256993888854\\
52.1774932593673	-0.245257229688415\\
52.1840266093411	-0.245257465402596\\
52.1905599506508	-0.245257701031436\\
52.1970932832995	-0.245257936574978\\
52.2036266072903	-0.245258172033262\\
52.2101599226265	-0.24525840740633\\
52.2166932293112	-0.245258642694224\\
52.2232265273475	-0.245258877896983\\
52.2297598167385	-0.24525911301465\\
52.2362930974875	-0.245259348047265\\
52.2428263695975	-0.245259582994871\\
52.2493596330717	-0.245259817857507\\
52.2558928879133	-0.245260052635214\\
52.2624261341254	-0.245260287328035\\
52.2689593717111	-0.24526052193601\\
52.2754926006736	-0.245260756459179\\
52.282025821016	-0.245260990897584\\
52.2885590327414	-0.245261225251267\\
52.295092235853	-0.245261459520267\\
52.3016254303539	-0.245261693704625\\
52.3081586162472	-0.245261927804383\\
52.3146917935361	-0.245262161819582\\
52.3212249622236	-0.245262395750261\\
52.327758122313	-0.245262629596463\\
52.3342912738074	-0.245262863358227\\
52.3408244167098	-0.245263097035595\\
52.3473575510233	-0.245263330628607\\
52.3538906767512	-0.245263564137304\\
52.3604237938965	-0.245263797561726\\
52.3669569024623	-0.245264030901915\\
52.3734900024518	-0.245264264157911\\
52.3800230938681	-0.245264497329754\\
52.3865561767142	-0.245264730417485\\
52.3930892509933	-0.245264963421145\\
52.3996223167085	-0.245265196340773\\
52.4061553738629	-0.245265429176412\\
52.4126884224597	-0.2452656619281\\
52.4192214625018	-0.245265894595879\\
52.4257544939925	-0.245266127179789\\
52.4322875169348	-0.245266359679871\\
52.4388205313318	-0.245266592096164\\
52.4453535371866	-0.245266824428709\\
52.4518865345023	-0.245267056677547\\
52.458419523282	-0.245267288842717\\
52.4649525035289	-0.245267520924261\\
52.4714854752459	-0.245267752922217\\
52.4780184384363	-0.245267984836628\\
52.484551393103	-0.245268216667532\\
52.4910843392491	-0.24526844841497\\
52.4976172768778	-0.245268680078982\\
52.5041502059922	-0.245268911659608\\
52.5106831265953	-0.245269143156888\\
52.5172160386902	-0.245269374570863\\
52.5237489422799	-0.245269605901573\\
52.5302818373676	-0.245269837149057\\
52.5368147239564	-0.245270068313355\\
52.5433476020493	-0.245270299394508\\
52.5498804716493	-0.245270530392556\\
52.5564133327596	-0.245270761307538\\
52.5629461853833	-0.245270992139494\\
52.5694790295233	-0.245271222888465\\
52.5760118651829	-0.245271453554489\\
52.5825446923649	-0.245271684137608\\
52.5890775110726	-0.245271914637861\\
52.5956103213089	-0.245272145055287\\
52.602143123077	-0.245272375389926\\
52.6086759163799	-0.245272605641819\\
52.6152087012206	-0.245272835811004\\
52.6217414776022	-0.245273065897522\\
52.6282742455278	-0.245273295901412\\
52.6348070050004	-0.245273525822714\\
52.6413397560231	-0.245273755661468\\
52.6478724985989	-0.245273985417713\\
52.6544052327309	-0.245274215091488\\
52.6609379584221	-0.245274444682834\\
52.6674706756756	-0.24527467419179\\
52.6740033844944	-0.245274903618394\\
52.6805360848815	-0.245275132962688\\
52.6870687768401	-0.24527536222471\\
52.6936014603732	-0.2452755914045\\
52.7001341354837	-0.245275820502097\\
52.7066668021748	-0.24527604951754\\
52.7131994604494	-0.24527627845087\\
52.7197321103106	-0.245276507302124\\
52.7262647517615	-0.245276736071343\\
52.732797384805	-0.245276964758566\\
52.7393300094443	-0.245277193363833\\
52.7458626256823	-0.245277421887181\\
52.752395233522	-0.245277650328652\\
52.7589278329665	-0.245277878688283\\
52.7654604240189	-0.245278106966114\\
52.7719930066821	-0.245278335162184\\
52.7785255809591	-0.245278563276533\\
52.7850581468531	-0.245278791309199\\
52.7915907043669	-0.245279019260222\\
52.7981232535037	-0.24527924712964\\
52.8046557942664	-0.245279474917493\\
52.811188326658	-0.245279702623819\\
52.8177208506817	-0.245279930248659\\
52.8242533663403	-0.24528015779205\\
52.8307858736369	-0.245280385254031\\
52.8373183725745	-0.245280612634642\\
52.8438508631561	-0.245280839933921\\
52.8503833453847	-0.245281067151908\\
52.8569158192634	-0.24528129428864\\
52.8634482847951	-0.245281521344158\\
52.8699807419828	-0.245281748318499\\
52.8765131908296	-0.245281975211703\\
52.8830456313384	-0.245282202023808\\
52.8895780635122	-0.245282428754854\\
52.896110487354	-0.245282655404878\\
52.9026429028669	-0.245282881973919\\
52.9091753100537	-0.245283108462017\\
52.9157077089176	-0.24528333486921\\
52.9222400994615	-0.245283561195536\\
52.9287724816884	-0.245283787441033\\
52.9353048556012	-0.245284013605742\\
52.941837221203	-0.245284239689699\\
52.9483695784968	-0.245284465692945\\
52.9549019274855	-0.245284691615516\\
52.9614342681721	-0.245284917457452\\
52.9679666005596	-0.245285143218791\\
52.974498924651	-0.245285368899572\\
52.9810312404492	-0.245285594499833\\
52.9875635479573	-0.245285820019612\\
52.9940958471782	-0.245286045458948\\
53.0006281381148	-0.245286270817879\\
53.0071604207702	-0.245286496096443\\
53.0136926951473	-0.245286721294679\\
53.0202249612491	-0.245286946412625\\
53.0267572190786	-0.245287171450319\\
53.0332894686387	-0.2452873964078\\
53.0398217099324	-0.245287621285106\\
53.0463539429627	-0.245287846082274\\
53.0528861677324	-0.245288070799343\\
53.0594183842447	-0.245288295436352\\
53.0659505925024	-0.245288519993337\\
53.0724827925085	-0.245288744470339\\
53.0790149842659	-0.245288968867393\\
53.0855471677776	-0.245289193184539\\
53.0920793430466	-0.245289417421815\\
53.0986115100759	-0.245289641579258\\
53.1051436688683	-0.245289865656907\\
53.1116758194268	-0.2452900896548\\
53.1182079617543	-0.245290313572973\\
53.1247400958539	-0.245290537411466\\
53.1312722217285	-0.245290761170317\\
53.1378043393809	-0.245290984849562\\
53.1443364488142	-0.24529120844924\\
53.1508685500313	-0.24529143196939\\
53.1574006430351	-0.245291655410047\\
53.1639327278285	-0.245291878771251\\
53.1704648044146	-0.24529210205304\\
53.1769968727962	-0.24529232525545\\
53.1835289329762	-0.245292548378519\\
53.1900609849577	-0.245292771422286\\
53.1965930287435	-0.245292994386788\\
53.2031250643365	-0.245293217272062\\
53.2096570917398	-0.245293440078147\\
53.2161891109561	-0.245293662805079\\
53.2227211219885	-0.245293885452897\\
53.2292531248399	-0.245294108021637\\
53.2357851195132	-0.245294330511338\\
53.2423171060113	-0.245294552922037\\
53.2488490843371	-0.245294775253772\\
53.2553810544936	-0.245294997506579\\
53.2619130164836	-0.245295219680497\\
53.2684449703101	-0.245295441775562\\
53.2749769159761	-0.245295663791813\\
53.2815088534843	-0.245295885729286\\
53.2880407828378	-0.245296107588019\\
53.2945727040394	-0.24529632936805\\
53.301104617092	-0.245296551069414\\
53.3076365219986	-0.245296772692151\\
53.3141684187621	-0.245296994236297\\
53.3207003073853	-0.245297215701889\\
53.3272321878711	-0.245297437088965\\
53.3337640602226	-0.245297658397562\\
53.3402959244425	-0.245297879627716\\
53.3468277805338	-0.245298100779466\\
53.3533596284994	-0.245298321852848\\
53.3598914683421	-0.245298542847899\\
53.3664233000649	-0.245298763764657\\
53.3729551236706	-0.245298984603158\\
53.3794869391621	-0.24529920536344\\
53.3860187465425	-0.245299426045539\\
53.3925505458144	-0.245299646649493\\
53.3990823369808	-0.245299867175338\\
53.4056141200447	-0.245300087623112\\
53.4121458950088	-0.245300307992852\\
53.4186776618761	-0.245300528284593\\
53.4252094206495	-0.245300748498374\\
53.4317411713317	-0.245300968634232\\
53.4382729139258	-0.245301188692202\\
53.4448046484346	-0.245301408672322\\
53.451336374861	-0.245301628574628\\
53.4578680932078	-0.245301848399158\\
53.4643998034779	-0.245302068145948\\
53.4709315056742	-0.245302287815035\\
53.4774631997996	-0.245302507406456\\
53.4839948858569	-0.245302726920247\\
53.4905265638491	-0.245302946356445\\
53.4970582337788	-0.245303165715086\\
53.5035898956492	-0.245303384996208\\
53.5101215494629	-0.245303604199846\\
53.5166531952229	-0.245303823326038\\
53.523184832932	-0.245304042374821\\
53.5297164625932	-0.245304261346229\\
53.5362480842091	-0.245304480240301\\
53.5427796977828	-0.245304699057072\\
53.549311303317	-0.24530491779658\\
53.5558429008147	-0.24530513645886\\
53.5623744902786	-0.245305355043949\\
53.5689060717117	-0.245305573551883\\
53.5754376451167	-0.245305791982699\\
53.5819692104966	-0.245306010336433\\
53.5885007678541	-0.245306228613122\\
53.5950323171922	-0.245306446812801\\
53.6015638585136	-0.245306664935508\\
53.6080953918213	-0.245306882981278\\
53.614626917118	-0.245307100950148\\
53.6211584344066	-0.245307318842154\\
53.6276899436899	-0.245307536657331\\
53.6342214449709	-0.245307754395718\\
53.6407529382522	-0.245307972057348\\
53.6472844235368	-0.24530818964226\\
53.6538159008275	-0.245308407150488\\
53.6603473701271	-0.245308624582069\\
53.6668788314384	-0.245308841937039\\
53.6734102847644	-0.245309059215434\\
53.6799417301077	-0.245309276417291\\
53.6864731674713	-0.245309493542644\\
53.693004596858	-0.245309710591531\\
53.6995360182706	-0.245309927563987\\
53.7060674317119	-0.245310144460048\\
53.7125988371847	-0.24531036127975\\
53.7191302346919	-0.245310578023129\\
53.7256616242363	-0.245310794690221\\
53.7321930058207	-0.245311011281062\\
53.7387243794479	-0.245311227795688\\
53.7452557451208	-0.245311444234134\\
53.7517871028421	-0.245311660596436\\
53.7583184526147	-0.24531187688263\\
53.7648497944413	-0.245312093092753\\
53.7713811283249	-0.245312309226838\\
53.7779124542682	-0.245312525284924\\
53.7844437722739	-0.245312741267044\\
53.790975082345	-0.245312957173235\\
53.7975063844842	-0.245313173003533\\
53.8040376786943	-0.245313388757972\\
53.8105689649782	-0.24531360443659\\
53.8171002433386	-0.24531382003942\\
53.8236315137783	-0.2453140355665\\
53.8301627763002	-0.245314251017864\\
53.836694030907	-0.245314466393548\\
53.8432252776015	-0.245314681693588\\
53.8497565163865	-0.245314896918019\\
53.8562877472649	-0.245315112066876\\
53.8628189702393	-0.245315327140195\\
53.8693501853127	-0.245315542138012\\
53.8758813924877	-0.245315757060361\\
53.8824125917673	-0.245315971907279\\
53.8889437831541	-0.245316186678801\\
53.8954749666509	-0.245316401374961\\
53.9020061422606	-0.245316615995796\\
53.9085373099858	-0.24531683054134\\
53.9150684698295	-0.245317045011629\\
53.9215996217944	-0.245317259406699\\
53.9281307658832	-0.245317473726584\\
53.9346619020988	-0.245317687971319\\
53.9411930304439	-0.245317902140941\\
53.9477241509212	-0.245318116235484\\
53.9542552635337	-0.245318330254983\\
53.9607863682839	-0.245318544199473\\
53.9673174651748	-0.24531875806899\\
53.9738485542091	-0.245318971863568\\
53.9803796353895	-0.245319185583244\\
53.9869107087188	-0.245319399228051\\
53.9934417741998	-0.245319612798025\\
53.9999728318353	-0.2453198262932\\
54.0065038816279	-0.245320039713613\\
54.0130349235805	-0.245320253059298\\
54.0195659576959	-0.245320466330289\\
54.0260969839768	-0.245320679526623\\
54.0326280024259	-0.245320892648332\\
54.039159013046	-0.245321105695454\\
54.0456900158399	-0.245321318668022\\
54.0522210108103	-0.245321531566072\\
54.0587519979599	-0.245321744389637\\
54.0652829772916	-0.245321957138754\\
54.0718139488081	-0.245322169813456\\
54.0783449125121	-0.245322382413779\\
54.0848758684063	-0.245322594939758\\
54.0914068164936	-0.245322807391426\\
54.0979377567767	-0.24532301976882\\
54.1044686892582	-0.245323232071973\\
54.110999613941	-0.24532344430092\\
54.1175305308278	-0.245323656455696\\
54.1240614399214	-0.245323868536335\\
54.1305923412244	-0.245324080542873\\
54.1371232347396	-0.245324292475343\\
54.1436541204698	-0.245324504333781\\
54.1501849984176	-0.24532471611822\\
54.1567158685859	-0.245324927828696\\
54.1632467309773	-0.245325139465243\\
54.1697775855946	-0.245325351027895\\
54.1763084324405	-0.245325562516687\\
54.1828392715177	-0.245325773931653\\
54.189370102829	-0.245325985272829\\
54.1959009263771	-0.245326196540247\\
54.2024317421647	-0.245326407733943\\
54.2089625501945	-0.245326618853951\\
54.2154933504694	-0.245326829900305\\
54.2220241429919	-0.24532704087304\\
54.2285549277648	-0.24532725177219\\
54.2350857047908	-0.245327462597789\\
54.2416164740726	-0.245327673349872\\
54.248147235613	-0.245327884028472\\
54.2546779894147	-0.245328094633625\\
54.2612087354804	-0.245328305165364\\
54.2677394738128	-0.245328515623723\\
54.2742702044145	-0.245328726008737\\
54.2808009272884	-0.245328936320439\\
54.2873316424371	-0.245329146558865\\
54.2938623498634	-0.245329356724048\\
54.3003930495698	-0.245329566816022\\
54.3069237415592	-0.245329776834821\\
54.3134544258343	-0.245329986780479\\
54.3199851023977	-0.24533019665303\\
54.3265157712522	-0.245330406452509\\
54.3330464324004	-0.245330616178949\\
54.339577085845	-0.245330825832385\\
54.3461077315888	-0.24533103541285\\
54.3526383696344	-0.245331244920378\\
54.3591689999846	-0.245331454355003\\
54.365699622642	-0.245331663716759\\
54.3722302376093	-0.24533187300568\\
54.3787608448892	-0.245332082221799\\
54.3852914444844	-0.245332291365152\\
54.3918220363976	-0.24533250043577\\
54.3983526206315	-0.245332709433689\\
54.4048831971887	-0.245332918358942\\
54.4114137660719	-0.245333127211563\\
54.4179443272839	-0.245333335991585\\
54.4244748808273	-0.245333544699042\\
54.4310054267048	-0.245333753333968\\
54.4375359649191	-0.245333961896397\\
54.4440664954727	-0.245334170386362\\
54.4505970183685	-0.245334378803897\\
54.4571275336092	-0.245334587149036\\
54.4636580411972	-0.245334795421811\\
54.4701885411355	-0.245335003622258\\
54.4767190334265	-0.245335211750408\\
54.483249518073	-0.245335419806297\\
54.4897799950777	-0.245335627789956\\
54.4963104644432	-0.245335835701421\\
54.5028409261722	-0.245336043540724\\
54.5093713802673	-0.245336251307899\\
54.5159018267313	-0.245336459002978\\
54.5224322655668	-0.245336666625997\\
54.5289626967764	-0.245336874176988\\
54.5354931203628	-0.245337081655984\\
54.5420235363286	-0.245337289063019\\
54.5485539446766	-0.245337496398126\\
54.5550843454094	-0.245337703661339\\
54.5616147385296	-0.245337910852691\\
54.5681451240399	-0.245338117972214\\
54.5746755019429	-0.245338325019944\\
54.5812058722414	-0.245338531995911\\
54.5877362349379	-0.245338738900151\\
54.594266590035	-0.245338945732696\\
54.6007969375356	-0.245339152493579\\
54.6073272774421	-0.245339359182834\\
54.6138576097572	-0.245339565800493\\
54.6203879344836	-0.24533977234659\\
54.626918251624	-0.245339978821158\\
54.6334485611809	-0.24534018522423\\
54.639978863157	-0.245340391555839\\
54.646509157555	-0.245340597816018\\
54.6530394443775	-0.2453408040048\\
54.659569723627	-0.245341010122219\\
54.6660999953064	-0.245341216168306\\
54.6726302594181	-0.245341422143096\\
54.6791605159648	-0.245341628046621\\
54.6856907649492	-0.245341833878914\\
54.6922210063739	-0.245342039640009\\
54.6987512402415	-0.245342245329937\\
54.7052814665547	-0.245342450948732\\
54.711811685316	-0.245342656496426\\
54.7183418965282	-0.245342861973054\\
54.7248721001937	-0.245343067378646\\
54.7314022963153	-0.245343272713237\\
54.7379324848956	-0.245343477976859\\
54.7444626659372	-0.245343683169545\\
54.7509928394427	-0.245343888291327\\
54.7575230054147	-0.245344093342238\\
54.7640531638558	-0.245344298322311\\
54.7705833147687	-0.24534450323158\\
54.777113458156	-0.245344708070075\\
54.7836435940203	-0.24534491283783\\
54.7901737223642	-0.245345117534879\\
54.7967038431904	-0.245345322161252\\
54.8032339565013	-0.245345526716983\\
54.8097640622997	-0.245345731202105\\
54.8162941605881	-0.24534593561665\\
54.8228242513692	-0.24534613996065\\
54.8293543346456	-0.245346344234139\\
54.8358844104198	-0.245346548437148\\
54.8424144786944	-0.245346752569711\\
54.8489445394722	-0.245346956631859\\
54.8554745927555	-0.245347160623625\\
54.8620046385472	-0.245347364545041\\
54.8685346768497	-0.245347568396141\\
54.8750647076657	-0.245347772176956\\
54.8815947309978	-0.245347975887518\\
54.8881247468485	-0.245348179527861\\
54.8946547552204	-0.245348383098016\\
54.9011847561162	-0.245348586598015\\
54.9077147495384	-0.245348790027892\\
54.9142447354896	-0.245348993387678\\
54.9207747139724	-0.245349196677406\\
54.9273046849894	-0.245349399897107\\
54.9338346485432	-0.245349603046815\\
54.9403646046364	-0.245349806126561\\
54.9468945532715	-0.245350009136377\\
54.9534244944511	-0.245350212076296\\
54.9599544281779	-0.24535041494635\\
54.9664843544543	-0.24535061774657\\
54.9730142732831	-0.24535082047699\\
54.9795441846666	-0.245351023137641\\
54.9860740886076	-0.245351225728555\\
54.9926039851086	-0.245351428249765\\
54.9991338741722	-0.245351630701302\\
55.0056637558009	-0.245351833083198\\
55.0121936299973	-0.245352035395485\\
55.018723496764	-0.245352237638196\\
55.0252533561036	-0.245352439811363\\
55.0317832080186	-0.245352641915017\\
55.0383130525116	-0.245352843949189\\
55.0448428895851	-0.245353045913914\\
55.0513727192418	-0.245353247809221\\
55.0579025414842	-0.245353449635143\\
55.0644323563148	-0.245353651391712\\
55.0709621637363	-0.245353853078959\\
55.0774919637511	-0.245354054696917\\
55.0840217563618	-0.245354256245618\\
55.090551541571	-0.245354457725092\\
55.0970813193813	-0.245354659135372\\
55.1036110897952	-0.24535486047649\\
55.1101408528152	-0.245355061748477\\
55.1166706084439	-0.245355262951366\\
55.1232003566839	-0.245355464085187\\
55.1297300975377	-0.245355665149972\\
55.1362598310078	-0.245355866145753\\
55.1427895570969	-0.245356067072562\\
55.1493192758074	-0.245356267930431\\
55.1558489871419	-0.24535646871939\\
55.162378691103	-0.245356669439472\\
55.1689083876931	-0.245356870090708\\
55.1754380769149	-0.24535707067313\\
55.1819677587708	-0.245357271186769\\
55.1884974332634	-0.245357471631657\\
55.1950271003953	-0.245357672007825\\
55.201556760169	-0.245357872315304\\
55.208086412587	-0.245358072554127\\
55.2146160576519	-0.245358272724324\\
55.2211456953661	-0.245358472825928\\
55.2276753257323	-0.245358672858968\\
55.2342049487529	-0.245358872823478\\
55.2407345644305	-0.245359072719488\\
55.2472641727677	-0.245359272547029\\
55.2537937737669	-0.245359472306133\\
55.2603233674306	-0.245359671996831\\
55.2668529537614	-0.245359871619155\\
55.2733825327619	-0.245360071173136\\
55.2799121044345	-0.245360270658805\\
55.2864416687818	-0.245360470076193\\
55.2929712258063	-0.245360669425331\\
55.2995007755105	-0.245360868706251\\
55.3060303178969	-0.245361067918984\\
55.312559852968	-0.245361267063561\\
55.3190893807264	-0.245361466140014\\
55.3256189011746	-0.245361665148373\\
55.3321484143151	-0.245361864088669\\
55.3386779201504	-0.245362062960934\\
55.345207418683	-0.245362261765199\\
55.3517369099154	-0.245362460501494\\
55.3582663938502	-0.245362659169851\\
55.3647958704898	-0.245362857770301\\
55.3713253398368	-0.245363056302875\\
55.3778548018936	-0.245363254767604\\
55.3843842566628	-0.245363453164518\\
55.3909137041469	-0.24536365149365\\
55.3974431443483	-0.245363849755029\\
55.4039725772696	-0.245364047948686\\
55.4105020029133	-0.245364246074654\\
55.4170314212818	-0.245364444132961\\
55.4235608323777	-0.245364642123641\\
55.4300902362035	-0.245364840046722\\
55.4366196327617	-0.245365037902236\\
55.4431490220547	-0.245365235690214\\
55.449678404085	-0.245365433410687\\
55.4562077788552	-0.245365631063685\\
55.4627371463678	-0.245365828649239\\
55.4692665066251	-0.245366026167381\\
55.4757958596298	-0.24536622361814\\
55.4823252053843	-0.245366421001547\\
55.4888545438911	-0.245366618317634\\
55.4953838751526	-0.24536681556643\\
55.5019131991714	-0.245367012747966\\
55.5084425159499	-0.245367209862274\\
55.5149718254907	-0.245367406909384\\
55.5215011277962	-0.245367603889325\\
55.5280304228688	-0.24536780080213\\
55.5345597107111	-0.245367997647828\\
55.5410889913256	-0.24536819442645\\
55.5476182647147	-0.245368391138027\\
55.5541475308808	-0.245368587782588\\
55.5606767898266	-0.245368784360166\\
55.5672060415543	-0.245368980870789\\
55.5737352860666	-0.245369177314489\\
55.5802645233659	-0.245369373691296\\
55.5867937534546	-0.24536957000124\\
55.5933229763352	-0.245369766244353\\
55.5998521920103	-0.245369962420663\\
55.6063814004822	-0.245370158530202\\
55.6129106017534	-0.245370354573\\
55.6194397958263	-0.245370550549088\\
55.6259689827036	-0.245370746458495\\
55.6324981623875	-0.245370942301252\\
55.6390273348806	-0.245371138077389\\
55.6455565001854	-0.245371333786937\\
55.6520856583042	-0.245371529429926\\
55.6586148092395	-0.245371725006386\\
55.6651439529939	-0.245371920516347\\
55.6716730895697	-0.24537211595984\\
55.6782022189694	-0.245372311336894\\
55.6847313411955	-0.245372506647541\\
55.6912604562504	-0.245372701891809\\
55.6977895641366	-0.24537289706973\\
55.7043186648565	-0.245373092181332\\
55.7108477584125	-0.245373287226648\\
55.7173768448072	-0.245373482205705\\
55.7239059240429	-0.245373677118536\\
55.7304349961221	-0.245373871965169\\
55.7369640610472	-0.245374066745634\\
55.7434931188207	-0.245374261459962\\
55.7500221694451	-0.245374456108183\\
55.7565512129227	-0.245374650690327\\
55.763080249256	-0.245374845206424\\
55.7696092784475	-0.245375039656502\\
55.7761383004995	-0.245375234040594\\
55.7826673154146	-0.245375428358728\\
55.7891963231951	-0.245375622610934\\
55.7957253238435	-0.245375816797243\\
55.8022543173623	-0.245376010917684\\
55.8087833037537	-0.245376204972286\\
55.8153122830204	-0.245376398961081\\
55.8218412551647	-0.245376592884097\\
55.828370220189	-0.245376786741364\\
55.8348991780958	-0.245376980532913\\
55.8414281288874	-0.245377174258773\\
55.8479570725664	-0.245377367918973\\
55.8544860091351	-0.245377561513544\\
55.861014938596	-0.245377755042515\\
55.8675438609515	-0.245377948505916\\
55.874072776204	-0.245378141903777\\
55.8806016843559	-0.245378335236127\\
55.8871305854096	-0.245378528502995\\
55.8936594793676	-0.245378721704413\\
55.9001883662323	-0.245378914840408\\
55.9067172460061	-0.245379107911011\\
55.9132461186913	-0.245379300916251\\
55.9197749842905	-0.245379493856158\\
55.926303842806	-0.245379686730762\\
55.9328326942403	-0.245379879540091\\
55.9393615385957	-0.245380072284176\\
55.9458903758747	-0.245380264963046\\
55.9524192060796	-0.245380457576731\\
55.9589480292129	-0.245380650125259\\
55.965476845277	-0.245380842608661\\
55.9720056542742	-0.245381035026965\\
55.9785344562071	-0.245381227380202\\
55.9850632510779	-0.2453814196684\\
55.9915920388891	-0.24538161189159\\
55.9981208196431	-0.245381804049799\\
56.0046495933423	-0.245381996143059\\
56.011178359989	-0.245382188171397\\
56.0177071195858	-0.245382380134844\\
56.0242358721349	-0.245382572033428\\
56.0307646176387	-0.245382763867179\\
56.0372933560998	-0.245382955636127\\
56.0438220875204	-0.2453831473403\\
56.0503508119029	-0.245383338979727\\
56.0568795292497	-0.245383530554439\\
56.0634082395633	-0.245383722064463\\
56.069936942846	-0.24538391350983\\
56.0764656391002	-0.245384104890568\\
56.0829943283283	-0.245384296206707\\
56.0895230105326	-0.245384487458276\\
56.0960516857156	-0.245384678645303\\
56.1025803538796	-0.245384869767819\\
56.109109015027	-0.245385060825851\\
56.1156376691602	-0.245385251819429\\
56.1221663162816	-0.245385442748583\\
56.1286949563935	-0.245385633613341\\
56.1352235894984	-0.245385824413732\\
56.1417522155985	-0.245386015149785\\
56.1482808346964	-0.245386205821529\\
56.1548094467943	-0.245386396428994\\
56.1613380518946	-0.245386586972207\\
56.1678666499997	-0.245386777451199\\
56.1743952411119	-0.245386967865997\\
56.1809238252337	-0.245387158216632\\
56.1874524023674	-0.245387348503131\\
56.1939809725154	-0.245387538725524\\
56.20050953568	-0.245387728883839\\
56.2070380918636	-0.245387918978106\\
56.2135666410686	-0.245388109008352\\
56.2200951832973	-0.245388298974608\\
56.226623718552	-0.245388488876902\\
56.2331522468353	-0.245388678715262\\
56.2396807681493	-0.245388868489717\\
56.2462092824965	-0.245389058200296\\
56.2527377898792	-0.245389247847028\\
56.2592662902998	-0.245389437429941\\
56.2657947837607	-0.245389626949065\\
56.2723232702641	-0.245389816404427\\
};
\addplot [color=mycolor1,solid,forget plot]
  table[row sep=crcr]{%
56.2723232702641	-0.245389816404427\\
56.2788517498125	-0.245390005796057\\
56.2853802224082	-0.245390195123983\\
56.2919086880535	-0.245390384388234\\
56.2984371467508	-0.245390573588838\\
56.3049655985025	-0.245390762725824\\
56.3114940433108	-0.24539095179922\\
56.3180224811782	-0.245391140809056\\
56.324550912107	-0.245391329755359\\
56.3310793360995	-0.245391518638159\\
56.3376077531581	-0.245391707457483\\
56.3441361632852	-0.24539189621336\\
56.3506645664829	-0.245392084905819\\
56.3571929627538	-0.245392273534888\\
56.3637213521001	-0.245392462100596\\
56.3702497345242	-0.245392650602971\\
56.3767781100284	-0.245392839042041\\
56.3833064786151	-0.245393027417835\\
56.3898348402865	-0.245393215730381\\
56.3963631950451	-0.245393403979708\\
56.4028915428931	-0.245393592165844\\
56.4094198838329	-0.245393780288818\\
56.4159482178668	-0.245393968348656\\
56.4224765449972	-0.245394156345389\\
56.4290048652263	-0.245394344279044\\
56.4355331785566	-0.24539453214965\\
56.4420614849902	-0.245394719957234\\
56.4485897845297	-0.245394907701826\\
56.4551180771772	-0.245395095383452\\
56.4616463629351	-0.245395283002142\\
56.4681746418058	-0.245395470557924\\
56.4747029137915	-0.245395658050826\\
56.4812311788945	-0.245395845480875\\
56.4877594371173	-0.245396032848101\\
56.494287688462	-0.245396220152531\\
56.5008159329311	-0.245396407394193\\
56.5073441705269	-0.245396594573116\\
56.5138724012515	-0.245396781689328\\
56.5204006251075	-0.245396968742856\\
56.526928842097	-0.245397155733729\\
56.5334570522225	-0.245397342661974\\
56.5399852554861	-0.245397529527621\\
56.5465134518903	-0.245397716330696\\
56.5530416414373	-0.245397903071228\\
56.5595698241294	-0.245398089749245\\
56.5660979999689	-0.245398276364774\\
56.5726261689582	-0.245398462917844\\
56.5791543310996	-0.245398649408483\\
56.5856824863953	-0.245398835836718\\
56.5922106348477	-0.245399022202578\\
56.598738776459	-0.24539920850609\\
56.6052669112316	-0.245399394747282\\
56.6117950391678	-0.245399580926182\\
56.6183231602698	-0.245399767042818\\
56.6248512745399	-0.245399953097218\\
56.6313793819806	-0.245400139089409\\
56.6379074825939	-0.245400325019419\\
56.6444355763824	-0.245400510887276\\
56.6509636633481	-0.245400696693009\\
56.6574917434935	-0.245400882436643\\
56.6640198168208	-0.245401068118208\\
56.6705478833323	-0.245401253737731\\
56.6770759430303	-0.24540143929524\\
56.6836039959171	-0.245401624790762\\
56.690132041995	-0.245401810224325\\
56.6966600812663	-0.245401995595956\\
56.7031881137332	-0.245402180905684\\
56.709716139398	-0.245402366153536\\
56.716244158263	-0.245402551339539\\
56.7227721703306	-0.245402736463722\\
56.7293001756029	-0.245402921526111\\
56.7358281740823	-0.245403106526734\\
56.742356165771	-0.245403291465619\\
56.7488841506713	-0.245403476342794\\
56.7554121287856	-0.245403661158285\\
56.761940100116	-0.24540384591212\\
56.7684680646648	-0.245404030604328\\
56.7749960224344	-0.245404215234934\\
56.7815239734269	-0.245404399803968\\
56.7880519176447	-0.245404584311455\\
56.7945798550901	-0.245404768757424\\
56.8011077857652	-0.245404953141902\\
56.8076357096725	-0.245405137464916\\
56.814163626814	-0.245405321726494\\
56.8206915371922	-0.245405505926663\\
56.8272194408092	-0.245405690065451\\
56.8337473376674	-0.245405874142884\\
56.840275227769	-0.24540605815899\\
56.8468031111163	-0.245406242113797\\
56.8533309877115	-0.245406426007331\\
56.8598588575568	-0.24540660983962\\
56.8663867206547	-0.245406793610692\\
56.8729145770072	-0.245406977320573\\
56.8794424266167	-0.24540716096929\\
56.8859702694854	-0.245407344556871\\
56.8924981056156	-0.245407528083343\\
56.8990259350095	-0.245407711548734\\
56.9055537576694	-0.245407894953069\\
56.9120815735976	-0.245408078296378\\
56.9186093827962	-0.245408261578685\\
56.9251371852676	-0.24540844480002\\
56.9316649810139	-0.245408627960408\\
56.9381927700375	-0.245408811059878\\
56.9447205523406	-0.245408994098455\\
56.9512483279254	-0.245409177076167\\
56.9577760967942	-0.245409359993042\\
56.9643038589493	-0.245409542849105\\
56.9708316143928	-0.245409725644385\\
56.977359363127	-0.245409908378907\\
56.9838871051541	-0.2454100910527\\
56.9904148404765	-0.24541027366579\\
56.9969425690962	-0.245410456218204\\
57.0034702910157	-0.245410638709969\\
57.009998006237	-0.245410821141111\\
57.0165257147626	-0.245411003511659\\
57.0230534165944	-0.245411185821638\\
57.0295811117349	-0.245411368071076\\
57.0361088001863	-0.245411550259999\\
57.0426364819507	-0.245411732388434\\
57.0491641570304	-0.245411914456408\\
57.0556918254277	-0.245412096463948\\
57.0622194871448	-0.245412278411081\\
57.0687471421838	-0.245412460297833\\
57.0752747905471	-0.245412642124232\\
57.0818024322369	-0.245412823890303\\
57.0883300672553	-0.245413005596074\\
57.0948576956046	-0.245413187241572\\
57.1013853172871	-0.245413368826823\\
57.1079129323049	-0.245413550351853\\
57.1144405406603	-0.245413731816691\\
57.1209681423555	-0.245413913221361\\
57.1274957373927	-0.245414094565891\\
57.1340233257741	-0.245414275850308\\
57.140550907502	-0.245414457074638\\
57.1470784825786	-0.245414638238907\\
57.153606051006	-0.245414819343143\\
57.1601336127866	-0.245415000387372\\
57.1666611679224	-0.24541518137162\\
57.1731887164158	-0.245415362295915\\
57.1797162582689	-0.245415543160281\\
57.186243793484	-0.245415723964747\\
57.1927713220632	-0.245415904709338\\
57.1992988440088	-0.245416085394082\\
57.205826359323	-0.245416266019004\\
57.2123538680079	-0.245416446584131\\
57.2188813700658	-0.245416627089489\\
57.225408865499	-0.245416807535105\\
57.2319363543095	-0.245416987921006\\
57.2384638364996	-0.245417168247217\\
57.2449913120715	-0.245417348513765\\
57.2515187810275	-0.245417528720677\\
57.2580462433696	-0.245417708867978\\
57.2645736991001	-0.245417888955696\\
57.2711011482213	-0.245418068983856\\
57.2776285907352	-0.245418248952485\\
57.2841560266441	-0.245418428861608\\
57.2906834559503	-0.245418608711254\\
57.2972108786558	-0.245418788501447\\
57.3037382947628	-0.245418968232213\\
57.3102657042737	-0.24541914790358\\
57.3167931071905	-0.245419327515573\\
57.3233205035154	-0.245419507068219\\
57.3298478932507	-0.245419686561544\\
57.3363752763985	-0.245419865995573\\
57.3429026529611	-0.245420045370334\\
57.3494300229405	-0.245420224685852\\
57.355957386339	-0.245420403942153\\
57.3624847431588	-0.245420583139264\\
57.3690120934021	-0.24542076227721\\
57.375539437071	-0.245420941356019\\
57.3820667741677	-0.245421120375715\\
57.3885941046944	-0.245421299336325\\
57.3951214286533	-0.245421478237875\\
57.4016487460465	-0.245421657080391\\
57.4081760568763	-0.245421835863898\\
57.4147033611448	-0.245422014588424\\
57.4212306588541	-0.245422193253994\\
57.4277579500066	-0.245422371860634\\
57.4342852346043	-0.24542255040837\\
57.4408125126493	-0.245422728897228\\
57.447339784144	-0.245422907327233\\
57.4538670490904	-0.245423085698413\\
57.4603943074908	-0.245423264010792\\
57.4669215593472	-0.245423442264397\\
57.4734488046619	-0.245423620459253\\
57.479976043437	-0.245423798595386\\
57.4865032756747	-0.245423976672823\\
57.4930305013772	-0.245424154691589\\
57.4995577205465	-0.245424332651709\\
57.506084933185	-0.245424510553211\\
57.5126121392947	-0.245424688396118\\
57.5191393388779	-0.245424866180458\\
57.5256665319366	-0.245425043906256\\
57.532193718473	-0.245425221573538\\
57.5387208984893	-0.245425399182329\\
57.5452480719876	-0.245425576732655\\
57.5517752389702	-0.245425754224543\\
57.5583023994391	-0.245425931658017\\
57.5648295533965	-0.245426109033103\\
57.5713567008446	-0.245426286349827\\
57.5778838417855	-0.245426463608215\\
57.5844109762214	-0.245426640808292\\
57.5909381041544	-0.245426817950084\\
57.5974652255867	-0.245426995033617\\
57.6039923405204	-0.245427172058915\\
57.6105194489576	-0.245427349026005\\
57.6170465509006	-0.245427525934913\\
57.6235736463514	-0.245427702785663\\
57.6301007353122	-0.245427879578282\\
57.6366278177852	-0.245428056312795\\
57.6431548937725	-0.245428232989227\\
57.6496819632762	-0.245428409607604\\
57.6562090262984	-0.245428586167951\\
57.6627360828414	-0.245428762670294\\
57.6692631329073	-0.245428939114659\\
57.6757901764982	-0.24542911550107\\
57.6823172136161	-0.245429291829553\\
57.6888442442634	-0.245429468100134\\
57.695371268442	-0.245429644312838\\
57.7018982861542	-0.24542982046769\\
57.7084252974021	-0.245429996564717\\
57.7149523021878	-0.245430172603942\\
57.7214793005134	-0.245430348585391\\
57.7280062923811	-0.245430524509091\\
57.7345332777929	-0.245430700375066\\
57.7410602567511	-0.245430876183341\\
57.7475872292578	-0.245431051933941\\
57.7541141953151	-0.245431227626893\\
57.760641154925	-0.245431403262221\\
57.7671681080898	-0.24543157883995\\
57.7736950548115	-0.245431754360106\\
57.7802219950924	-0.245431929822714\\
57.7867489289344	-0.2454321052278\\
57.7932758563398	-0.245432280575387\\
57.7998027773106	-0.245432455865502\\
57.8063296918489	-0.24543263109817\\
57.812856599957	-0.245432806273416\\
57.8193835016369	-0.245432981391264\\
57.8259103968907	-0.245433156451741\\
57.8324372857205	-0.245433331454871\\
57.8389641681285	-0.245433506400679\\
57.8454910441167	-0.245433681289191\\
57.8520179136873	-0.245433856120431\\
57.8585447768424	-0.245434030894425\\
57.8650716335841	-0.245434205611197\\
57.8715984839146	-0.245434380270773\\
57.8781253278358	-0.245434554873177\\
57.88465216535	-0.245434729418435\\
57.8911789964592	-0.245434903906571\\
57.8977058211656	-0.245435078337611\\
57.9042326394712	-0.24543525271158\\
57.9107594513782	-0.245435427028502\\
57.9172862568886	-0.245435601288403\\
57.9238130560046	-0.245435775491306\\
57.9303398487282	-0.245435949637238\\
57.9368666350617	-0.245436123726223\\
57.9433934150069	-0.245436297758286\\
57.9499201885662	-0.245436471733452\\
57.9564469557415	-0.245436645651746\\
57.962973716535	-0.245436819513192\\
57.9695004709487	-0.245436993317815\\
57.9760272189848	-0.245437167065641\\
57.9825539606453	-0.245437340756694\\
57.9890806959324	-0.245437514390998\\
57.9956074248481	-0.245437687968579\\
58.0021341473946	-0.245437861489461\\
58.0086608635738	-0.245438034953669\\
58.015187573388	-0.245438208361228\\
58.0217142768392	-0.245438381712162\\
58.0282409739294	-0.245438555006497\\
58.0347676646608	-0.245438728244256\\
58.0412943490355	-0.245438901425465\\
58.0478210270555	-0.245439074550148\\
58.054347698723	-0.24543924761833\\
58.06087436404	-0.245439420630035\\
58.0674010230085	-0.245439593585289\\
58.0739276756307	-0.245439766484115\\
58.0804543219087	-0.245439939326538\\
58.0869809618446	-0.245440112112583\\
58.0935075954403	-0.245440284842275\\
58.1000342226981	-0.245440457515637\\
58.1065608436199	-0.245440630132696\\
58.1130874582079	-0.245440802693474\\
58.1196140664641	-0.245440975197996\\
58.1261406683905	-0.245441147646288\\
58.1326672639894	-0.245441320038373\\
58.1391938532627	-0.245441492374277\\
58.1457204362125	-0.245441664654023\\
58.152247012841	-0.245441836877635\\
58.15877358315	-0.245442009045139\\
58.1653001471418	-0.245442181156559\\
58.1718267048184	-0.245442353211919\\
58.1783532561819	-0.245442525211244\\
58.1848798012343	-0.245442697154557\\
58.1914063399776	-0.245442869041884\\
58.1979328724141	-0.245443040873249\\
58.2044593985457	-0.245443212648675\\
58.2109859183744	-0.245443384368188\\
58.2175124319024	-0.245443556031812\\
58.2240389391317	-0.24544372763957\\
58.2305654400644	-0.245443899191488\\
58.2370919347025	-0.245444070687589\\
58.2436184230481	-0.245444242127898\\
58.2501449051032	-0.245444413512439\\
58.2566713808699	-0.245444584841236\\
58.2631978503503	-0.245444756114314\\
58.2697243135464	-0.245444927331696\\
58.2762507704603	-0.245445098493407\\
58.2827772210939	-0.245445269599472\\
58.2893036654495	-0.245445440649913\\
58.295830103529	-0.245445611644756\\
58.3023565353344	-0.245445782584025\\
58.3088829608678	-0.245445953467743\\
58.3154093801314	-0.245446124295935\\
58.321935793127	-0.245446295068624\\
58.3284621998568	-0.245446465785836\\
58.3349886003228	-0.245446636447594\\
58.3415149945271	-0.245446807053922\\
58.3480413824716	-0.245446977604843\\
58.3545677641585	-0.245447148100384\\
58.3610941395898	-0.245447318540566\\
58.3676205087675	-0.245447488925414\\
58.3741468716937	-0.245447659254953\\
58.3806732283703	-0.245447829529206\\
58.3871995787995	-0.245447999748197\\
58.3937259229833	-0.245448169911951\\
58.4002522609237	-0.24544834002049\\
58.4067785926228	-0.245448510073839\\
58.4133049180825	-0.245448680072023\\
58.419831237305	-0.245448850015064\\
58.4263575502922	-0.245449019902986\\
58.4328838570462	-0.245449189735815\\
58.439410157569	-0.245449359513573\\
58.4459364518627	-0.245449529236284\\
58.4524627399292	-0.245449698903973\\
58.4589890217706	-0.245449868516662\\
58.465515297389	-0.245450038074376\\
58.4720415667864	-0.245450207577139\\
58.4785678299647	-0.245450377024974\\
58.485094086926	-0.245450546417906\\
58.4916203376724	-0.245450715755957\\
58.4981465822059	-0.245450885039153\\
58.5046728205284	-0.245451054267515\\
58.511199052642	-0.245451223441069\\
58.5177252785488	-0.245451392559838\\
58.5242514982507	-0.245451561623845\\
58.5307777117497	-0.245451730633114\\
58.537303919048	-0.24545189958767\\
58.5438301201474	-0.245452068487535\\
58.5503563150501	-0.245452237332733\\
58.556882503758	-0.245452406123289\\
58.5634086862732	-0.245452574859224\\
58.5699348625976	-0.245452743540564\\
58.5764610327333	-0.245452912167332\\
58.5829871966822	-0.245453080739551\\
58.5895133544465	-0.245453249257245\\
58.5960395060281	-0.245453417720437\\
58.602565651429	-0.245453586129151\\
58.6090917906512	-0.245453754483411\\
58.6156179236968	-0.24545392278324\\
58.6221440505677	-0.245454091028661\\
58.6286701712659	-0.245454259219698\\
58.6351962857935	-0.245454427356375\\
58.6417223941525	-0.245454595438715\\
58.6482484963448	-0.245454763466742\\
58.6547745923725	-0.245454931440478\\
58.6613006822375	-0.245455099359948\\
58.6678267659419	-0.245455267225174\\
58.6743528434877	-0.245455435036181\\
58.6808789148768	-0.245455602792991\\
58.6874049801113	-0.245455770495628\\
58.6939310391932	-0.245455938144115\\
58.7004570921244	-0.245456105738476\\
58.7069831389069	-0.245456273278734\\
58.7135091795428	-0.245456440764913\\
58.7200352140341	-0.245456608197035\\
58.7265612423827	-0.245456775575124\\
58.7330872645906	-0.245456942899203\\
58.7396132806598	-0.245457110169296\\
58.7461392905923	-0.245457277385425\\
58.7526652943901	-0.245457444547615\\
58.7591912920552	-0.245457611655888\\
58.7657172835896	-0.245457778710268\\
58.7722432689952	-0.245457945710777\\
58.7787692482741	-0.245458112657439\\
58.7852952214282	-0.245458279550277\\
58.7918211884595	-0.245458446389315\\
58.7983471493701	-0.245458613174575\\
58.8048731041618	-0.245458779906081\\
58.8113990528366	-0.245458946583856\\
58.8179249953966	-0.245459113207923\\
58.8244509318438	-0.245459279778304\\
58.83097686218	-0.245459446295024\\
58.8375027864073	-0.245459612758105\\
58.8440287045277	-0.245459779167571\\
58.8505546165431	-0.245459945523444\\
58.8570805224555	-0.245460111825747\\
58.8636064222669	-0.245460278074504\\
58.8701323159793	-0.245460444269737\\
58.8766582035946	-0.24546061041147\\
58.8831840851148	-0.245460776499725\\
58.8897099605419	-0.245460942534526\\
58.8962358298778	-0.245461108515896\\
58.9027616931246	-0.245461274443857\\
58.9092875502841	-0.245461440318432\\
58.9158134013584	-0.245461606139645\\
58.9223392463494	-0.245461771907518\\
58.9288650852591	-0.245461937622075\\
58.9353909180894	-0.245462103283337\\
58.9419167448424	-0.245462268891329\\
58.9484425655199	-0.245462434446072\\
58.954968380124	-0.245462599947591\\
58.9614941886566	-0.245462765395907\\
58.9680199911197	-0.245462930791043\\
58.9745457875152	-0.245463096133023\\
58.9810715778451	-0.245463261421869\\
58.9875973621113	-0.245463426657604\\
58.9941231403158	-0.245463591840251\\
59.0006489124606	-0.245463756969833\\
59.0071746785476	-0.245463922046371\\
59.0137004385788	-0.24546408706989\\
59.0202261925561	-0.245464252040412\\
59.0267519404815	-0.245464416957959\\
59.033277682357	-0.245464581822554\\
59.0398034181844	-0.245464746634221\\
59.0463291479657	-0.245464911392981\\
59.052854871703	-0.245465076098858\\
59.0593805893981	-0.245465240751873\\
59.0659063010529	-0.245465405352051\\
59.0724320066695	-0.245465569899413\\
59.0789577062498	-0.245465734393981\\
59.0854833997957	-0.24546589883578\\
59.0920090873092	-0.245466063224831\\
59.0985347687922	-0.245466227561156\\
59.1050604442467	-0.245466391844779\\
59.1115861136745	-0.245466556075722\\
59.1181117770777	-0.245466720254008\\
59.1246374344582	-0.245466884379659\\
59.1311630858179	-0.245467048452697\\
59.1376887311588	-0.245467212473146\\
59.1442143704828	-0.245467376441027\\
59.1507400037918	-0.245467540356364\\
59.1572656310879	-0.245467704219178\\
59.1637912523728	-0.245467868029493\\
59.1703168676485	-0.24546803178733\\
59.1768424769171	-0.245468195492712\\
59.1833680801804	-0.245468359145662\\
59.1898936774403	-0.245468522746202\\
59.1964192686988	-0.245468686294355\\
59.2029448539578	-0.245468849790142\\
59.2094704332193	-0.245469013233586\\
59.2159960064851	-0.24546917662471\\
59.2225215737573	-0.245469339963536\\
59.2290471350376	-0.245469503250087\\
59.2355726903282	-0.245469666484384\\
59.2420982396308	-0.24546982966645\\
59.2486237829474	-0.245469992796308\\
59.25514932028	-0.245470155873979\\
59.2616748516304	-0.245470318899486\\
59.2682003770005	-0.245470481872852\\
59.2747258963924	-0.245470644794098\\
59.2812514098079	-0.245470807663247\\
59.287776917249	-0.245470970480321\\
59.2943024187175	-0.245471133245342\\
59.3008279142154	-0.245471295958333\\
59.3073534037446	-0.245471458619315\\
59.313878887307	-0.245471621228311\\
59.3204043649045	-0.245471783785344\\
59.3269298365391	-0.245471946290435\\
59.3334553022126	-0.245472108743606\\
59.3399807619271	-0.24547227114488\\
59.3465062156843	-0.245472433494279\\
59.3530316634862	-0.245472595791825\\
59.3595571053347	-0.24547275803754\\
59.3660825412317	-0.245472920231446\\
59.3726079711792	-0.245473082373565\\
59.379133395179	-0.24547324446392\\
59.3856588132331	-0.245473406502532\\
59.3921842253433	-0.245473568489424\\
59.3987096315116	-0.245473730424617\\
59.4052350317398	-0.245473892308134\\
59.4117604260299	-0.245474054139996\\
59.4182858143838	-0.245474215920226\\
59.4248111968034	-0.245474377648846\\
59.4313365732905	-0.245474539325877\\
59.4378619438471	-0.245474700951342\\
59.4443873084752	-0.245474862525262\\
59.4509126671765	-0.24547502404766\\
59.4574380199529	-0.245475185518558\\
59.4639633668065	-0.245475346937977\\
59.470488707739	-0.245475508305939\\
59.4770140427524	-0.245475669622467\\
59.4835393718486	-0.245475830887581\\
59.4900646950294	-0.245475992101305\\
59.4965900122968	-0.24547615326366\\
59.5031153236527	-0.245476314374667\\
59.5096406290989	-0.245476475434349\\
59.5161659286373	-0.245476636442728\\
59.5226912222699	-0.245476797399825\\
59.5292165099984	-0.245476958305661\\
59.5357417918249	-0.24547711916026\\
59.5422670677512	-0.245477279963643\\
59.5487923377792	-0.245477440715831\\
59.5553176019107	-0.245477601416846\\
59.5618428601477	-0.245477762066711\\
59.568368112492	-0.245477922665446\\
59.5748933589455	-0.245478083213074\\
59.5814185995102	-0.245478243709616\\
59.5879438341878	-0.245478404155094\\
59.5944690629804	-0.245478564549529\\
59.6009942858896	-0.245478724892944\\
59.6075195029175	-0.24547888518536\\
59.614044714066	-0.245479045426799\\
59.6205699193368	-0.245479205617282\\
59.6270951187319	-0.245479365756831\\
59.6336203122531	-0.245479525845467\\
59.6401454999024	-0.245479685883213\\
59.6466706816815	-0.24547984587009\\
59.6531958575925	-0.245480005806119\\
59.659721027637	-0.245480165691323\\
59.6662461918171	-0.245480325525721\\
59.6727713501346	-0.245480485309337\\
59.6792965025914	-0.245480645042192\\
59.6858216491893	-0.245480804724307\\
59.6923467899302	-0.245480964355704\\
59.698871924816	-0.245481123936404\\
59.7053970538485	-0.245481283466429\\
59.7119221770297	-0.245481442945801\\
59.7184472943613	-0.24548160237454\\
59.7249724058452	-0.245481761752668\\
59.7314975114834	-0.245481921080208\\
59.7380226112777	-0.245482080357179\\
59.7445477052298	-0.245482239583604\\
59.7510727933418	-0.245482398759504\\
59.7575978756155	-0.245482557884901\\
59.7641229520527	-0.245482716959815\\
59.7706480226552	-0.245482875984269\\
59.777173087425	-0.245483034958283\\
59.7836981463639	-0.245483193881879\\
59.7902231994738	-0.245483352755079\\
59.7967482467564	-0.245483511577903\\
59.8032732882138	-0.245483670350374\\
59.8097983238477	-0.245483829072512\\
59.8163233536599	-0.245483987744338\\
59.8228483776525	-0.245484146365875\\
59.8293733958271	-0.245484304937143\\
59.8358984081856	-0.245484463458163\\
59.8424234147299	-0.245484621928958\\
59.8489484154619	-0.245484780349547\\
59.8554734103834	-0.245484938719953\\
59.8619983994963	-0.245485097040196\\
59.8685233828023	-0.245485255310298\\
59.8750483603034	-0.24548541353028\\
59.8815733320013	-0.245485571700163\\
59.888098297898	-0.245485729819969\\
59.8946232579953	-0.245485887889718\\
59.901148212295	-0.245486045909432\\
59.907673160799	-0.245486203879132\\
59.9141981035091	-0.245486361798839\\
59.9207230404271	-0.245486519668574\\
59.927247971555	-0.245486677488359\\
59.9337728968945	-0.245486835258214\\
59.9402978164474	-0.24548699297816\\
59.9468227302157	-0.245487150648219\\
59.9533476382011	-0.245487308268412\\
59.9598725404055	-0.245487465838759\\
59.9663974368308	-0.245487623359282\\
59.9729223274787	-0.245487780830002\\
59.9794472123511	-0.24548793825094\\
59.9859720914499	-0.245488095622117\\
59.9924969647768	-0.245488252943554\\
59.9990218323337	-0.245488410215271\\
60.0055466941225	-0.245488567437291\\
60.0120715501449	-0.245488724609633\\
60.0185964004029	-0.245488881732319\\
60.0251212448981	-0.24548903880537\\
60.0316460836325	-0.245489195828807\\
60.0381709166079	-0.24548935280265\\
60.0446957438261	-0.245489509726921\\
60.051220565289	-0.245489666601641\\
60.0577453809983	-0.24548982342683\\
60.0642701909558	-0.245489980202509\\
60.0707949951635	-0.245490136928699\\
60.0773197936232	-0.245490293605422\\
60.0838445863366	-0.245490450232697\\
60.0903693733056	-0.245490606810546\\
60.0968941545319	-0.24549076333899\\
60.1034189300176	-0.245490919818049\\
60.1099436997642	-0.245491076247744\\
60.1164684637737	-0.245491232628096\\
60.1229932220479	-0.245491388959126\\
60.1295179745886	-0.245491545240855\\
60.1360427213976	-0.245491701473302\\
60.1425674624767	-0.24549185765649\\
60.1490921978278	-0.245492013790439\\
60.1556169274526	-0.24549216987517\\
60.1621416513531	-0.245492325910702\\
60.1686663695309	-0.245492481897058\\
60.1751910819879	-0.245492637834258\\
60.1817157887259	-0.245492793722321\\
60.1882404897467	-0.24549294956127\\
60.1947651850521	-0.245493105351125\\
60.201289874644	-0.245493261091906\\
60.2078145585242	-0.245493416783634\\
60.2143392366944	-0.24549357242633\\
60.2208639091564	-0.245493728020014\\
60.2273885759121	-0.245493883564708\\
60.2339132369633	-0.24549403906043\\
60.2404378923118	-0.245494194507203\\
60.2469625419593	-0.245494349905047\\
60.2534871859078	-0.245494505253982\\
60.2600118241589	-0.245494660554028\\
60.2665364567145	-0.245494815805208\\
60.2730610835764	-0.24549497100754\\
60.2795857047464	-0.245495126161045\\
60.2861103202262	-0.245495281265745\\
60.2926349300178	-0.245495436321659\\
60.2991595341228	-0.245495591328808\\
60.3056841325431	-0.245495746287212\\
60.3122087252805	-0.245495901196893\\
60.3187333123367	-0.24549605605787\\
60.3252578937136	-0.245496210870163\\
60.331782469413	-0.245496365633794\\
60.3383070394366	-0.245496520348783\\
60.3448316037863	-0.24549667501515\\
60.3513561624638	-0.245496829632916\\
60.3578807154709	-0.2454969842021\\
60.3644052628095	-0.245497138722724\\
60.3709298044812	-0.245497293194808\\
60.3774543404879	-0.245497447618371\\
60.3839788708314	-0.245497601993436\\
60.3905033955135	-0.245497756320021\\
60.397027914536	-0.245497910598147\\
60.4035524279005	-0.245498064827834\\
60.410076935609	-0.245498219009104\\
60.4166014376632	-0.245498373141975\\
60.4231259340649	-0.245498527226469\\
60.4296504248159	-0.245498681262606\\
60.4361749099179	-0.245498835250405\\
60.4426993893728	-0.245498989189888\\
60.4492238631823	-0.245499143081074\\
60.4557483313481	-0.245499296923984\\
60.4622727938722	-0.245499450718637\\
60.4687972507562	-0.245499604465055\\
60.4753217020019	-0.245499758163257\\
60.4818461476111	-0.245499911813264\\
60.4883705875857	-0.245500065415095\\
60.4948950219272	-0.245500218968771\\
60.5014194506376	-0.245500372474313\\
60.5079438737186	-0.245500525931739\\
60.514468291172	-0.245500679341071\\
60.5209927029995	-0.245500832702328\\
60.527517109203	-0.245500986015531\\
60.5340415097841	-0.2455011392807\\
60.5405659047447	-0.245501292497855\\
60.5470902940865	-0.245501445667015\\
60.5536146778114	-0.245501598788201\\
60.5601390559209	-0.245501751861434\\
60.5666634284171	-0.245501904886733\\
60.5731877953015	-0.245502057864117\\
60.5797121565759	-0.245502210793608\\
60.5862365122422	-0.245502363675226\\
60.5927608623021	-0.245502516508989\\
60.5992852067574	-0.245502669294919\\
60.6058095456097	-0.245502822033035\\
60.612333878861	-0.245502974723357\\
60.6188582065128	-0.245503127365906\\
60.6253825285671	-0.2455032799607\\
60.6319068450256	-0.245503432507761\\
60.6384311558899	-0.245503585007108\\
60.6449554611619	-0.245503737458761\\
60.6514797608434	-0.24550388986274\\
60.658004054936	-0.245504042219064\\
60.6645283434416	-0.245504194527755\\
60.6710526263619	-0.245504346788831\\
60.6775769036986	-0.245504499002313\\
60.6841011754535	-0.24550465116822\\
60.6906254416284	-0.245504803286572\\
60.697149702225	-0.245504955357389\\
60.7036739572451	-0.245505107380692\\
60.7101982066903	-0.245505259356499\\
60.7167224505626	-0.245505411284831\\
60.7232466888635	-0.245505563165707\\
60.7297709215949	-0.245505714999147\\
60.7362951487585	-0.245505866785172\\
60.7428193703561	-0.2455060185238\\
60.7493435863893	-0.245506170215052\\
60.75586779686	-0.245506321858947\\
60.7623920017699	-0.245506473455505\\
60.7689162011208	-0.245506625004746\\
60.7754403949143	-0.245506776506689\\
60.7819645831522	-0.245506927961355\\
60.7884887658362	-0.245507079368762\\
60.7950129429682	-0.245507230728932\\
60.8015371145498	-0.245507382041882\\
60.8080612805828	-0.245507533307634\\
60.814585441069	-0.245507684526206\\
60.8211095960099	-0.245507835697618\\
60.8276337454075	-0.245507986821891\\
60.8341578892634	-0.245508137899043\\
60.8406820275794	-0.245508288929094\\
60.8472061603571	-0.245508439912064\\
60.8537302875984	-0.245508590847973\\
60.860254409305	-0.245508741736839\\
60.8667785254786	-0.245508892578683\\
60.8733026361209	-0.245509043373525\\
60.8798267412337	-0.245509194121383\\
60.8863508408186	-0.245509344822277\\
60.8928749348775	-0.245509495476227\\
60.8993990234121	-0.245509646083253\\
60.905923106424	-0.245509796643373\\
60.9124471839151	-0.245509947156608\\
60.918971255887	-0.245510097622977\\
60.9254953223414	-0.245510248042499\\
60.9320193832802	-0.245510398415194\\
60.9385434387049	-0.245510548741081\\
60.9450674886175	-0.245510699020181\\
60.9515915330194	-0.245510849252511\\
60.9581155719126	-0.245510999438092\\
60.9646396052987	-0.245511149576943\\
60.9711636331794	-0.245511299669084\\
60.9776876555564	-0.245511449714534\\
60.9842116724315	-0.245511599713312\\
60.9907356838065	-0.245511749665438\\
60.9972596896829	-0.245511899570931\\
61.0037836900625	-0.245512049429811\\
61.0103076849471	-0.245512199242096\\
61.0168316743384	-0.245512349007807\\
61.023355658238	-0.245512498726962\\
61.0298796366478	-0.245512648399581\\
61.0364036095693	-0.245512798025683\\
61.0429275770043	-0.245512947605288\\
61.0494515389546	-0.245513097138414\\
61.0559754954218	-0.245513246625082\\
61.0624994464077	-0.24551339606531\\
61.069023391914	-0.245513545459117\\
61.0755473319423	-0.245513694806524\\
61.0820712664944	-0.245513844107549\\
61.088595195572	-0.245513993362211\\
61.0951191191768	-0.24551414257053\\
61.1016430373105	-0.245514291732524\\
61.1081669499748	-0.245514440848214\\
61.1146908571714	-0.245514589917618\\
61.1212147589021	-0.245514738940755\\
61.1277386551685	-0.245514887917645\\
61.1342625459723	-0.245515036848307\\
61.1407864313153	-0.245515185732759\\
61.1473103111991	-0.245515334571022\\
61.1538341856254	-0.245515483363114\\
61.160358054596	-0.245515632109054\\
61.1668819181125	-0.245515780808862\\
61.1734057761766	-0.245515929462557\\
61.1799296287901	-0.245516078070157\\
61.1864534759547	-0.245516226631682\\
61.1929773176719	-0.245516375147151\\
61.1995011539436	-0.245516523616583\\
61.2060249847714	-0.245516672039996\\
61.212548810157	-0.245516820417411\\
61.2190726301022	-0.245516968748846\\
61.2255964446085	-0.24551711703432\\
61.2321202536777	-0.245517265273852\\
61.2386440573116	-0.245517413467462\\
61.2451678555117	-0.245517561615167\\
61.2516916482798	-0.245517709716988\\
61.2582154356176	-0.245517857772943\\
61.2647392175267	-0.24551800578305\\
61.2712629940088	-0.24551815374733\\
61.2777867650657	-0.245518301665801\\
61.284310530699	-0.245518449538482\\
61.2908342909104	-0.245518597365392\\
61.2973580457016	-0.24551874514655\\
61.3038817950743	-0.245518892881974\\
61.3104055390301	-0.245519040571684\\
61.3169292775708	-0.245519188215698\\
61.323453010698	-0.245519335814036\\
61.3299767384134	-0.245519483366716\\
61.3365004607187	-0.245519630873758\\
61.3430241776155	-0.245519778335179\\
61.3495478891056	-0.245519925750999\\
61.3560715951907	-0.245520073121237\\
61.3625952958723	-0.245520220445911\\
61.3691189911523	-0.245520367725041\\
61.3756426810322	-0.245520514958645\\
61.3821663655138	-0.245520662146741\\
61.3886900445986	-0.24552080928935\\
61.3952137182885	-0.245520956386489\\
61.401737386585	-0.245521103438177\\
61.4082610494899	-0.245521250444433\\
61.4147847070048	-0.245521397405276\\
61.4213083591314	-0.245521544320724\\
61.4278320058713	-0.245521691190797\\
61.4343556472263	-0.245521838015512\\
61.4408792831979	-0.245521984794889\\
61.447402913788	-0.245522131528947\\
61.4539265389981	-0.245522278217703\\
61.4604501588299	-0.245522424861177\\
61.466973773285	-0.245522571459388\\
61.4734973823653	-0.245522718012354\\
61.4800209860722	-0.245522864520093\\
61.4865445844075	-0.245523010982625\\
61.4930681773728	-0.245523157399967\\
61.4995917649699	-0.24552330377214\\
61.5061153472003	-0.24552345009916\\
61.5126389240657	-0.245523596381047\\
61.5191624955679	-0.24552374261782\\
61.5256860617084	-0.245523888809497\\
61.5322096224889	-0.245524034956096\\
61.538733177911	-0.245524181057636\\
61.5452567279766	-0.245524327114136\\
61.5517802726871	-0.245524473125614\\
61.5583038120443	-0.245524619092089\\
61.5648273460497	-0.245524765013579\\
61.5713508747052	-0.245524910890103\\
61.5778743980123	-0.24552505672168\\
61.5843979159726	-0.245525202508327\\
61.5909214285879	-0.245525348250063\\
61.5974449358598	-0.245525493946907\\
61.6039684377899	-0.245525639598878\\
61.6104919343799	-0.245525785205993\\
61.6170154256314	-0.245525930768271\\
61.6235389115462	-0.245526076285731\\
61.6300623921258	-0.24552622175839\\
61.6365858673718	-0.245526367186269\\
61.6431093372861	-0.245526512569384\\
61.6496328018701	-0.245526657907754\\
61.6561562611256	-0.245526803201398\\
61.6626797150541	-0.245526948450334\\
61.6692031636574	-0.24552709365458\\
61.6757266069371	-0.245527238814156\\
61.6822500448948	-0.245527383929078\\
61.6887734775322	-0.245527528999366\\
61.6952969048508	-0.245527674025037\\
61.7018203268525	-0.245527819006111\\
61.7083437435387	-0.245527963942605\\
61.7148671549112	-0.245528108834538\\
61.7213905609716	-0.245528253681928\\
61.7279139617215	-0.245528398484793\\
61.7344373571625	-0.245528543243152\\
61.7409607472964	-0.245528687957023\\
61.7474841321247	-0.245528832626424\\
61.7540075116491	-0.245528977251373\\
61.7605308858711	-0.245529121831889\\
61.7670542547926	-0.24552926636799\\
61.773577618415	-0.245529410859694\\
61.7801009767401	-0.245529555307019\\
61.7866243297694	-0.245529699709984\\
61.7931476775046	-0.245529844068607\\
61.7996710199473	-0.245529988382905\\
61.8061943570992	-0.245530132652898\\
61.8127176889618	-0.245530276878602\\
61.8192410155369	-0.245530421060037\\
61.825764336826	-0.245530565197221\\
61.8322876528308	-0.245530709290172\\
61.8388109635529	-0.245530853338907\\
61.845334268994	-0.245530997343445\\
61.8518575691556	-0.245531141303804\\
61.8583808640394	-0.245531285220002\\
61.864904153647	-0.245531429092058\\
61.87142743798	-0.245531572919989\\
61.8779507170401	-0.245531716703813\\
61.8844739908289	-0.245531860443549\\
61.8909972593481	-0.245532004139214\\
61.8975205225991	-0.245532147790827\\
61.9040437805837	-0.245532291398406\\
61.9105670333035	-0.245532434961968\\
61.9170902807601	-0.245532578481532\\
61.9236135229551	-0.245532721957115\\
61.9301367598902	-0.245532865388736\\
61.9366599915669	-0.245533008776413\\
61.9431832179869	-0.245533152120164\\
61.9497064391518	-0.245533295420006\\
61.9562296550631	-0.245533438675957\\
61.9627528657227	-0.245533581888037\\
61.9692760711319	-0.245533725056261\\
61.9757992712926	-0.245533868180649\\
61.9823224662062	-0.245534011261219\\
61.9888456558744	-0.245534154297987\\
61.9953688402987	-0.245534297290973\\
62.0018920194809	-0.245534440240194\\
62.0084151934225	-0.245534583145668\\
62.0149383621252	-0.245534726007412\\
62.0214615255905	-0.245534868825446\\
62.02798468382	-0.245535011599786\\
62.0345078368155	-0.24553515433045\\
62.0410309845783	-0.245535297017456\\
62.0475541271103	-0.245535439660823\\
62.0540772644129	-0.245535582260568\\
62.0606003964879	-0.245535724816708\\
62.0671235233367	-0.245535867329262\\
62.073646644961	-0.245536009798247\\
62.0801697613625	-0.245536152223682\\
62.0866928725426	-0.245536294605583\\
62.0932159785031	-0.245536436943969\\
62.0997390792455	-0.245536579238857\\
62.1062621747714	-0.245536721490266\\
62.1127852650824	-0.245536863698213\\
62.1193083501801	-0.245537005862715\\
62.1258314300662	-0.245537147983791\\
62.1323545047421	-0.245537290061458\\
62.1388775742096	-0.245537432095733\\
62.1454006384702	-0.245537574086635\\
62.1519236975255	-0.245537716034182\\
62.1584467513771	-0.24553785793839\\
62.1649698000266	-0.245537999799278\\
62.1714928434757	-0.245538141616863\\
62.1780158817258	-0.245538283391163\\
62.1845389147785	-0.245538425122196\\
62.1910619426356	-0.245538566809979\\
62.1975849652985	-0.245538708454529\\
62.2041079827689	-0.245538850055865\\
62.2106309950484	-0.245538991614004\\
62.2171540021385	-0.245539133128964\\
62.2236770040408	-0.245539274600762\\
62.230200000757	-0.245539416029415\\
62.2367229922885	-0.245539557414942\\
62.2432459786371	-0.24553969875736\\
62.2497689598043	-0.245539840056687\\
62.2562919357916	-0.245539981312939\\
62.2628149066007	-0.245540122526135\\
62.2693378722331	-0.245540263696292\\
62.2758608326905	-0.245540404823428\\
62.2823837879744	-0.24554054590756\\
62.2889067380864	-0.245540686948705\\
62.295429683028	-0.245540827946882\\
62.3019526228009	-0.245540968902107\\
62.3084755574067	-0.245541109814398\\
62.3149984868469	-0.245541250683773\\
62.3215214111231	-0.245541391510249\\
62.3280443302369	-0.245541532293844\\
62.3345672441898	-0.245541673034574\\
62.3410901529835	-0.245541813732457\\
62.3476130566195	-0.245541954387512\\
62.3541359550994	-0.245542094999754\\
62.3606588484247	-0.245542235569202\\
62.3671817365972	-0.245542376095873\\
62.3737046196182	-0.245542516579784\\
62.3802274974894	-0.245542657020953\\
62.3867503702124	-0.245542797419397\\
62.3932732377888	-0.245542937775134\\
62.39979610022	-0.24554307808818\\
62.4063189575078	-0.245543218358554\\
62.4128418096536	-0.245543358586271\\
62.419364656659	-0.245543498771351\\
62.4258874985257	-0.24554363891381\\
62.4324103352551	-0.245543779013665\\
62.4389331668488	-0.245543919070934\\
62.4454559933085	-0.245544059085634\\
62.4519788146356	-0.245544199057782\\
62.4585016308318	-0.245544338987396\\
62.4650244418985	-0.245544478874493\\
62.4715472478375	-0.245544618719089\\
62.4780700486501	-0.245544758521203\\
62.4845928443381	-0.245544898280852\\
62.4911156349029	-0.245545037998052\\
62.4976384203462	-0.245545177672821\\
62.5041612006694	-0.245545317305177\\
62.5106839758742	-0.245545456895135\\
62.5172067459621	-0.245545596442715\\
62.5237295109347	-0.245545735947932\\
62.5302522707935	-0.245545875410804\\
62.5367750255401	-0.245546014831349\\
62.543297775176	-0.245546154209582\\
62.5498205197029	-0.245546293545522\\
62.5563432591222	-0.245546432839186\\
62.5628659934355	-0.24554657209059\\
62.5693887226444	-0.245546711299752\\
62.5759114467504	-0.245546850466689\\
62.5824341657551	-0.245546989591418\\
62.5889568796601	-0.245547128673956\\
62.5954795884668	-0.245547267714321\\
62.6020022921769	-0.245547406712529\\
62.6085249907919	-0.245547545668597\\
62.6150476843133	-0.245547684582542\\
62.6215703727427	-0.245547823454382\\
62.6280930560817	-0.245547962284134\\
62.6346157343318	-0.245548101071814\\
62.6411384074945	-0.24554823981744\\
62.6476610755714	-0.245548378521028\\
62.654183738564	-0.245548517182596\\
62.660706396474	-0.245548655802161\\
62.6672290493027	-0.245548794379739\\
62.6737516970519	-0.245548932915348\\
62.680274339723	-0.245549071409004\\
62.6867969773175	-0.245549209860725\\
62.6933196098371	-0.245549348270527\\
62.6998422372832	-0.245549486638428\\
62.7063648596574	-0.245549624964445\\
62.7128874769613	-0.245549763248593\\
62.7194100891964	-0.245549901490891\\
62.7259326963641	-0.245550039691355\\
62.7324552984662	-0.245550177850002\\
62.738977895504	-0.24555031596685\\
62.7455004874792	-0.245550454041914\\
62.7520230743933	-0.245550592075211\\
62.7585456562477	-0.24555073006676\\
62.7650682330442	-0.245550868016575\\
62.7715908047841	-0.245551005924676\\
62.7781133714691	-0.245551143791077\\
62.7846359331006	-0.245551281615796\\
62.7911584896802	-0.245551419398851\\
62.7976810412095	-0.245551557140257\\
62.8042035876899	-0.245551694840031\\
62.810726129123	-0.245551832498191\\
62.8172486655103	-0.245551970114753\\
62.8237711968534	-0.245552107689733\\
62.8302937231538	-0.24555224522315\\
62.836816244413	-0.245552382715018\\
62.8433387606325	-0.245552520165356\\
62.849861271814	-0.24555265757418\\
62.8563837779588	-0.245552794941507\\
62.8629062790686	-0.245552932267353\\
62.8694287751448	-0.245553069551735\\
62.875951266189	-0.24555320679467\\
62.8824737522028	-0.245553343996175\\
62.8889962331876	-0.245553481156266\\
62.8955187091449	-0.24555361827496\\
62.9020411800763	-0.245553755352273\\
62.9085636459834	-0.245553892388223\\
62.9150861068675	-0.245554029382826\\
62.9216085627304	-0.245554166336099\\
62.9281310135734	-0.245554303248058\\
62.9346534593981	-0.24555444011872\\
62.941175900206	-0.245554576948102\\
62.9476983359986	-0.24555471373622\\
62.9542207667775	-0.24555485048309\\
62.9607431925442	-0.24555498718873\\
62.9672656133002	-0.245555123853157\\
62.973788029047	-0.245555260476386\\
62.9803104397861	-0.245555397058434\\
62.986832845519	-0.245555533599318\\
62.9933552462473	-0.245555670099054\\
62.9998776419725	-0.24555580655766\\
63.0064000326961	-0.24555594297515\\
63.0129224184196	-0.245556079351543\\
63.0194447991444	-0.245556215686855\\
63.0259671748723	-0.245556351981102\\
63.0324895456045	-0.2455564882343\\
63.0390119113427	-0.245556624446466\\
63.0455342720883	-0.245556760617617\\
63.0520566278429	-0.24555689674777\\
63.058578978608	-0.24555703283694\\
63.0651013243851	-0.245557168885144\\
63.0716236651757	-0.245557304892398\\
63.0781460009812	-0.24555744085872\\
63.0846683318033	-0.245557576784125\\
63.0911906576434	-0.24555771266863\\
63.097712978503	-0.245557848512251\\
63.1042352943836	-0.245557984315005\\
63.1107576052867	-0.245558120076909\\
63.1172799112139	-0.245558255797978\\
63.1238022121666	-0.245558391478229\\
63.1303245081464	-0.245558527117678\\
63.1368467991547	-0.245558662716343\\
63.143369085193	-0.245558798274238\\
63.1498913662629	-0.245558933791381\\
63.1564136423658	-0.245559069267788\\
63.1629359135033	-0.245559204703476\\
63.1694581796768	-0.245559340098459\\
63.1759804408878	-0.245559475452756\\
63.1825026971379	-0.245559610766382\\
63.1890249484285	-0.245559746039354\\
63.1955471947611	-0.245559881271688\\
63.2020694361372	-0.2455600164634\\
63.2085916725584	-0.245560151614506\\
63.2151139040261	-0.245560286725023\\
63.2216361305418	-0.245560421794968\\
63.2281583521069	-0.245560556824355\\
63.2346805687231	-0.245560691813202\\
63.2412027803918	-0.245560826761526\\
63.2477249871144	-0.245560961669341\\
63.2542471888926	-0.245561096536665\\
63.2607693857276	-0.245561231363513\\
63.2672915776212	-0.245561366149902\\
63.2738137645746	-0.245561500895848\\
63.2803359465895	-0.245561635601367\\
63.2868581236673	-0.245561770266476\\
63.2933802958095	-0.24556190489119\\
63.2999024630176	-0.245562039475526\\
63.3064246252931	-0.245562174019501\\
63.3129467826374	-0.245562308523129\\
63.3194689350521	-0.245562442986427\\
63.3259910825386	-0.245562577409413\\
63.3325132250984	-0.2455627117921\\
63.339035362733	-0.245562846134507\\
63.3455574954439	-0.245562980436648\\
63.3520796232326	-0.24556311469854\\
63.3586017461005	-0.245563248920199\\
63.3651238640491	-0.245563383101642\\
63.3716459770799	-0.245563517242883\\
63.3781680851945	-0.245563651343941\\
63.3846901883941	-0.245563785404829\\
63.3912122866804	-0.245563919425565\\
63.3977343800549	-0.245564053406165\\
63.4042564685189	-0.245564187346644\\
63.410778552074	-0.245564321247019\\
63.4173006307217	-0.245564455107305\\
63.4238227044634	-0.24556458892752\\
63.4303447733005	-0.245564722707678\\
63.4368668372347	-0.245564856447795\\
63.4433888962673	-0.245564990147889\\
63.4499109503998	-0.245565123807974\\
63.4564329996337	-0.245565257428067\\
63.4629550439705	-0.245565391008184\\
63.4694770834116	-0.245565524548341\\
63.4759991179585	-0.245565658048553\\
63.4825211476127	-0.245565791508837\\
63.4890431723757	-0.245565924929208\\
63.4955651922488	-0.245566058309683\\
63.5020872072337	-0.245566191650277\\
63.5086092173317	-0.245566324951007\\
63.5151312225443	-0.245566458211888\\
63.521653222873	-0.245566591432936\\
63.5281752183192	-0.245566724614167\\
63.5346972088845	-0.245566857755597\\
63.5412191945703	-0.245566990857242\\
63.547741175378	-0.245567123919118\\
63.5542631513091	-0.245567256941241\\
63.560785122365	-0.245567389923626\\
63.5673070885474	-0.245567522866289\\
63.5738290498575	-0.245567655769246\\
63.5803510062968	-0.245567788632514\\
63.5868729578669	-0.245567921456107\\
63.5933949045692	-0.245568054240042\\
63.5999168464051	-0.245568186984334\\
63.6064387833761	-0.245568319689\\
63.6129607154837	-0.245568452354055\\
63.6194826427293	-0.245568584979515\\
63.6260045651143	-0.245568717565395\\
63.6325264826403	-0.245568850111712\\
63.6390483953086	-0.245568982618481\\
63.6455703031208	-0.245569115085718\\
63.6520922060783	-0.245569247513438\\
63.6586141041826	-0.245569379901658\\
63.665135997435	-0.245569512250393\\
63.6716578858371	-0.245569644559659\\
63.6781797693903	-0.245569776829472\\
63.6847016480961	-0.245569909059847\\
63.6912235219559	-0.2455700412508\\
63.6977453909711	-0.245570173402347\\
63.7042672551433	-0.245570305514504\\
63.7107891144738	-0.245570437587285\\
63.7173109689641	-0.245570569620707\\
63.7238328186157	-0.245570701614786\\
63.73035466343	-0.245570833569536\\
63.7368765034085	-0.245570965484975\\
63.7433983385525	-0.245571097361117\\
63.7499201688636	-0.245571229197978\\
63.7564419943432	-0.245571360995573\\
63.7629638149928	-0.245571492753919\\
63.7694856308137	-0.24557162447303\\
63.7760074418074	-0.245571756152923\\
63.7825292479754	-0.245571887793614\\
63.7890510493192	-0.245572019395116\\
63.7955728458401	-0.245572150957447\\
63.8020946375395	-0.245572282480622\\
63.8086164244191	-0.245572413964657\\
63.8151382064801	-0.245572545409566\\
63.821659983724	-0.245572676815365\\
63.8281817561522	-0.245572808182071\\
63.8347035237663	-0.245572939509698\\
63.8412252865676	-0.245573070798262\\
63.8477470445576	-0.245573202047779\\
63.8542687977377	-0.245573333258264\\
63.8607905461093	-0.245573464429733\\
63.8673122896739	-0.2455735955622\\
63.8738340284329	-0.245573726655682\\
63.8803557623878	-0.245573857710195\\
63.8868774915399	-0.245573988725752\\
63.8933992158908	-0.245574119702371\\
63.8999209354418	-0.245574250640066\\
63.9064426501944	-0.245574381538853\\
63.9129643601501	-0.245574512398748\\
63.9194860653101	-0.245574643219765\\
63.9260077656761	-0.24557477400192\\
63.9325294612493	-0.245574904745229\\
63.9390511520313	-0.245575035449707\\
63.9455728380235	-0.24557516611537\\
63.9520945192272	-0.245575296742232\\
63.958616195644	-0.245575427330309\\
63.9651378672752	-0.245575557879617\\
63.9716595341223	-0.245575688390171\\
63.9781811961867	-0.245575818861986\\
63.9847028534698	-0.245575949295078\\
63.9912245059731	-0.245576079689462\\
63.997746153698	-0.245576210045153\\
64.0042677966458	-0.245576340362167\\
64.0107894348181	-0.245576470640519\\
64.0173110682163	-0.245576600880224\\
64.0238326968417	-0.245576731081298\\
64.0303543206957	-0.245576861243756\\
64.0368759397799	-0.245576991367613\\
64.0433975540957	-0.245577121452885\\
64.0499191636443	-0.245577251499586\\
64.0564407684274	-0.245577381507732\\
64.0629623684463	-0.245577511477339\\
64.0694839637023	-0.245577641408422\\
64.076005554197	-0.245577771300995\\
64.0825271399317	-0.245577901155074\\
64.0890487209079	-0.245578030970675\\
64.0955702971269	-0.245578160747812\\
64.1020918685903	-0.245578290486501\\
64.1086134352993	-0.245578420186757\\
64.1151349972555	-0.245578549848595\\
64.1216565544602	-0.24557867947203\\
64.1281781069148	-0.245578809057079\\
64.1346996546208	-0.245578938603754\\
64.1412211975796	-0.245579068112073\\
64.1477427357926	-0.24557919758205\\
64.1542642692611	-0.2455793270137\\
64.1607857979867	-0.245579456407039\\
64.1673073219707	-0.245579585762081\\
64.1738288412144	-0.245579715078841\\
64.1803503557195	-0.245579844357336\\
64.1868718654871	-0.245579973597579\\
64.1933933705188	-0.245580102799587\\
64.199914870816	-0.245580231963373\\
64.20643636638	-0.245580361088954\\
64.2129578572123	-0.245580490176345\\
64.2194793433143	-0.245580619225559\\
64.2260008246873	-0.245580748236614\\
64.2325223013328	-0.245580877209522\\
64.2390437732522	-0.245581006144301\\
64.2455652404469	-0.245581135040964\\
64.2520867029183	-0.245581263899527\\
64.2586081606677	-0.245581392720004\\
64.2651296136967	-0.245581521502412\\
64.2716510620065	-0.245581650246764\\
64.2781725055986	-0.245581778953076\\
64.2846939444745	-0.245581907621363\\
64.2912153786354	-0.24558203625164\\
64.2977368080828	-0.245582164843922\\
64.3042582328181	-0.245582293398223\\
64.3107796528427	-0.24558242191456\\
64.317301068158	-0.245582550392946\\
64.3238224787653	-0.245582678833397\\
64.3303438846662	-0.245582807235927\\
64.3368652858619	-0.245582935600553\\
64.3433866823538	-0.245583063927288\\
64.3499080741434	-0.245583192216147\\
64.3564294612321	-0.245583320467146\\
64.3629508436212	-0.2455834486803\\
64.3694722213122	-0.245583576855622\\
64.3759935943064	-0.245583704993129\\
64.3825149626052	-0.245583833092835\\
64.38903632621	-0.245583961154755\\
64.3955576851223	-0.245584089178904\\
64.4020790393433	-0.245584217165297\\
64.4086003888745	-0.245584345113949\\
64.4151217337173	-0.245584473024873\\
64.421643073873	-0.245584600898087\\
64.4281644093431	-0.245584728733603\\
64.4346857401289	-0.245584856531438\\
64.4412070662318	-0.245584984291605\\
64.4477283876533	-0.24558511201412\\
64.4542497043946	-0.245585239698998\\
64.4607710164572	-0.245585367346253\\
64.4672923238425	-0.2455854949559\\
64.4738136265518	-0.245585622527954\\
64.4803349245865	-0.24558575006243\\
64.4868562179481	-0.245585877559342\\
64.4933775066378	-0.245586005018706\\
64.4998987906571	-0.245586132440536\\
64.5064200700074	-0.245586259824846\\
64.51294134469	-0.245586387171652\\
64.5194626147063	-0.245586514480969\\
64.5259838800577	-0.24558664175281\\
64.5325051407456	-0.245586768987191\\
64.5390263967713	-0.245586896184127\\
64.5455476481363	-0.245587023343632\\
64.5520688948419	-0.245587150465721\\
64.5585901368894	-0.245587277550408\\
64.5651113742803	-0.245587404597709\\
64.571632607016	-0.245587531607638\\
64.5781538350977	-0.245587658580209\\
64.584675058527	-0.245587785515438\\
64.591196277305	-0.245587912413339\\
64.5977174914334	-0.245588039273927\\
64.6042387009133	-0.245588166097216\\
64.6107599057462	-0.245588292883221\\
64.6172811059335	-0.245588419631956\\
64.6238023014764	-0.245588546343437\\
64.6303234923765	-0.245588673017678\\
64.636844678635	-0.245588799654693\\
64.6433658602534	-0.245588926254498\\
64.6498870372329	-0.245589052817106\\
64.656408209575	-0.245589179342532\\
64.6629293772811	-0.245589305830791\\
64.6694505403524	-0.245589432281898\\
64.6759716987905	-0.245589558695867\\
64.6824928525965	-0.245589685072712\\
64.689014001772	-0.245589811412449\\
64.6955351463182	-0.245589937715091\\
64.7020562862365	-0.245590063980654\\
64.7085774215284	-0.245590190209151\\
64.7150985521951	-0.245590316400598\\
64.721619678238	-0.245590442555008\\
64.7281407996585	-0.245590568672397\\
64.734661916458	-0.245590694752779\\
64.7411830286378	-0.245590820796168\\
64.7477041361992	-0.245590946802579\\
64.7542252391437	-0.245591072772026\\
64.7607463374725	-0.245591198704524\\
64.7672674311872	-0.245591324600088\\
64.7737885202889	-0.245591450458731\\
64.780309604779	-0.245591576280469\\
64.786830684659	-0.245591702065315\\
64.7933517599302	-0.245591827813284\\
64.7998728305939	-0.245591953524391\\
64.8063938966516	-0.24559207919865\\
64.8129149581044	-0.245592204836075\\
64.8194360149539	-0.245592330436681\\
64.8259570672013	-0.245592456000483\\
64.832478114848	-0.245592581527494\\
64.8389991578955	-0.245592707017729\\
64.8455201963449	-0.245592832471202\\
64.8520412301977	-0.245592957887928\\
64.8585622594552	-0.245593083267922\\
64.8650832841188	-0.245593208611197\\
64.8716043041898	-0.245593333917768\\
64.8781253196696	-0.245593459187649\\
64.8846463305595	-0.245593584420855\\
64.8911673368609	-0.2455937096174\\
64.8976883385752	-0.245593834777298\\
64.9042093357036	-0.245593959900564\\
64.9107303282475	-0.245594084987211\\
64.9172513162083	-0.245594210037255\\
64.9237722995872	-0.245594335050709\\
64.9302932783858	-0.245594460027589\\
64.9368142526053	-0.245594584967907\\
64.943335222247	-0.245594709871679\\
64.9498561873123	-0.245594834738918\\
64.9563771478025	-0.24559495956964\\
64.9628981037191	-0.245595084363857\\
64.9694190550632	-0.245595209121585\\
64.9759400018363	-0.245595333842838\\
64.9824609440398	-0.24559545852763\\
64.9889818816748	-0.245595583175975\\
64.9955028147429	-0.245595707787887\\
65.0020237432453	-0.245595832363381\\
65.0085446671834	-0.245595956902471\\
65.0150655865585	-0.245596081405171\\
65.0215865013719	-0.245596205871495\\
65.028107411625	-0.245596330301457\\
65.0346283173191	-0.245596454695073\\
65.0411492184556	-0.245596579052355\\
65.0476701150358	-0.245596703373318\\
65.0541910070611	-0.245596827657976\\
65.0607118945327	-0.245596951906343\\
65.067232777452	-0.245597076118434\\
65.0737536558204	-0.245597200294263\\
65.0802745296391	-0.245597324433843\\
65.0867953989095	-0.245597448537189\\
65.093316263633	-0.245597572604315\\
65.0998371238108	-0.245597696635236\\
65.1063579794444	-0.245597820629964\\
65.112878830535	-0.245597944588515\\
65.1193996770839	-0.245598068510902\\
65.1259205190926	-0.24559819239714\\
65.1324413565622	-0.245598316247242\\
65.1389621894943	-0.245598440061223\\
65.14548301789	-0.245598563839097\\
65.1520038417507	-0.245598687580877\\
65.1585246610777	-0.245598811286578\\
65.1650454758725	-0.245598934956214\\
65.1715662861362	-0.2455990585898\\
65.1780870918702	-0.245599182187348\\
65.1846078930759	-0.245599305748873\\
65.1911286897546	-0.245599429274389\\
65.1976494819075	-0.24559955276391\\
65.2041702695361	-0.24559967621745\\
65.2106910526417	-0.245599799635023\\
65.2172118312254	-0.245599923016643\\
65.2237326052888	-0.245600046362324\\
65.2302533748331	-0.245600169672081\\
65.2367741398596	-0.245600292945926\\
65.2432949003697	-0.245600416183874\\
65.2498156563646	-0.245600539385939\\
65.2563364078458	-0.245600662552135\\
65.2628571548145	-0.245600785682475\\
65.2693778972719	-0.245600908776974\\
65.2758986352196	-0.245601031835646\\
65.2824193686587	-0.245601154858505\\
65.2889400975906	-0.245601277845564\\
65.2954608220166	-0.245601400796837\\
65.301981541938	-0.245601523712339\\
65.3085022573561	-0.245601646592083\\
65.3150229682723	-0.245601769436083\\
65.3215436746879	-0.245601892244353\\
65.3280643766041	-0.245602015016907\\
65.3345850740224	-0.245602137753758\\
65.3411057669439	-0.245602260454922\\
65.3476264553701	-0.24560238312041\\
65.3541471393022	-0.245602505750238\\
65.3606678187415	-0.245602628344419\\
65.3671884936894	-0.245602750902968\\
65.3737091641471	-0.245602873425896\\
65.3802298301161	-0.24560299591322\\
65.3867504915975	-0.245603118364952\\
65.3932711485927	-0.245603240781106\\
65.399791801103	-0.245603363161697\\
65.4063124491297	-0.245603485506737\\
65.4128330926741	-0.245603607816241\\
65.4193537317376	-0.245603730090222\\
65.4258743663214	-0.245603852328695\\
65.4323949964268	-0.245603974531672\\
65.4389156220552	-0.245604096699168\\
65.4454362432078	-0.245604218831197\\
65.451956859886	-0.245604340927772\\
65.458477472091	-0.245604462988907\\
65.4649980798242	-0.245604585014616\\
65.4715186830869	-0.245604707004912\\
65.4780392818803	-0.24560482895981\\
65.4845598762058	-0.245604950879322\\
65.4910804660647	-0.245605072763463\\
65.4976010514582	-0.245605194612246\\
65.5041216323877	-0.245605316425686\\
65.5106422088545	-0.245605438203795\\
65.5171627808599	-0.245605559946587\\
65.5236833484051	-0.245605681654076\\
65.5302039114915	-0.245605803326277\\
65.5367244701203	-0.245605924963201\\
65.543245024293	-0.245606046564864\\
65.5497655740106	-0.245606168131278\\
65.5562861192747	-0.245606289662458\\
65.5628066600864	-0.245606411158417\\
65.569327196447	-0.245606532619168\\
65.5758477283579	-0.245606654044725\\
65.5823682558203	-0.245606775435103\\
65.5888887788356	-0.245606896790314\\
65.595409297405	-0.245607018110372\\
65.6019298115298	-0.24560713939529\\
65.6084503212113	-0.245607260645083\\
65.6149708264508	-0.245607381859764\\
65.6214913272496	-0.245607503039346\\
65.628011823609	-0.245607624183843\\
65.6345323155303	-0.245607745293269\\
65.6410528030147	-0.245607866367637\\
65.6475732860636	-0.24560798740696\\
65.6540937646782	-0.245608108411253\\
65.6606142388599	-0.245608229380529\\
65.6671347086099	-0.2456083503148\\
65.6736551739295	-0.245608471214082\\
65.68017563482	-0.245608592078387\\
65.6866960912827	-0.245608712907729\\
65.6932165433188	-0.245608833702121\\
65.6997369909297	-0.245608954461577\\
65.7062574341166	-0.24560907518611\\
65.7127778728809	-0.245609195875734\\
65.7192983072238	-0.245609316530462\\
65.7258187371465	-0.245609437150309\\
65.7323391626504	-0.245609557735286\\
65.7388595837368	-0.245609678285408\\
65.7453800004069	-0.245609798800688\\
65.7519004126621	-0.24560991928114\\
65.7584208205035	-0.245610039726777\\
65.7649412239325	-0.245610160137613\\
65.7714616229504	-0.24561028051366\\
65.7779820175584	-0.245610400854933\\
65.7845024077578	-0.245610521161444\\
65.79102279355	-0.245610641433208\\
65.7975431749361	-0.245610761670237\\
65.8040635519174	-0.245610881872545\\
65.8105839244953	-0.245611002040145\\
65.817104292671	-0.245611122173051\\
65.8236246564458	-0.245611242271276\\
65.8301450158209	-0.245611362334834\\
65.8366653707977	-0.245611482363738\\
65.8431857213773	-0.245611602358\\
65.8497060675612	-0.245611722317635\\
65.8562264093505	-0.245611842242656\\
65.8627467467465	-0.245611962133077\\
65.8692670797505	-0.24561208198891\\
65.8757874083638	-0.245612201810168\\
65.8823077325877	-0.245612321596866\\
65.8888280524233	-0.245612441349016\\
65.8953483678721	-0.245612561066633\\
65.9018686789352	-0.245612680749728\\
65.9083889856139	-0.245612800398316\\
65.9149092879095	-0.245612920012409\\
65.9214295858233	-0.245613039592021\\
65.9279498793565	-0.245613159137166\\
65.9344701685104	-0.245613278647856\\
65.9409904532863	-0.245613398124104\\
65.9475107336854	-0.245613517565925\\
65.954031009709	-0.245613636973331\\
65.9605512813584	-0.245613756346336\\
65.9670715486348	-0.245613875684952\\
65.9735918115395	-0.245613994989193\\
65.9801120700738	-0.245614114259073\\
65.9866323242388	-0.245614233494604\\
65.993152574036	-0.245614352695799\\
65.9996728194665	-0.245614471862672\\
66.0061930605316	-0.245614590995236\\
66.0127132972326	-0.245614710093505\\
66.0192335295707	-0.245614829157491\\
66.0257537575472	-0.245614948187207\\
66.0322739811634	-0.245615067182667\\
66.0387942004205	-0.245615186143884\\
66.0453144153197	-0.245615305070871\\
66.0518346258624	-0.245615423963641\\
66.0583548320498	-0.245615542822208\\
66.0648750338831	-0.245615661646584\\
66.0713952313637	-0.245615780436783\\
66.0779154244926	-0.245615899192817\\
66.0844356132714	-0.245616017914701\\
66.090955797701	-0.245616136602446\\
66.097475977783	-0.245616255256066\\
66.1039961535184	-0.245616373875575\\
66.1105163249085	-0.245616492460985\\
66.1170364919546	-0.245616611012309\\
66.123556654658	-0.245616729529561\\
66.1300768130199	-0.245616848012753\\
66.1365969670415	-0.245616966461899\\
66.1431171167242	-0.245617084877011\\
66.1496372620691	-0.245617203258104\\
66.1561574030775	-0.245617321605189\\
66.1626775397506	-0.245617439918279\\
66.1691976720898	-0.245617558197389\\
66.1757178000962	-0.245617676442531\\
66.1822379237712	-0.245617794653717\\
66.1887580431158	-0.245617912830962\\
66.1952781581315	-0.245618030974277\\
66.2017982688195	-0.245618149083676\\
66.2083183751809	-0.245618267159173\\
66.2148384772171	-0.245618385200779\\
66.2213585749293	-0.245618503208508\\
66.2278786683187	-0.245618621182373\\
66.2343987573866	-0.245618739122388\\
66.2409188421342	-0.245618857028563\\
66.2474389225629	-0.245618974900914\\
66.2539589986737	-0.245619092739453\\
66.2604790704679	-0.245619210544192\\
66.2669991379469	-0.245619328315145\\
66.2735192011118	-0.245619446052325\\
66.2800392599639	-0.245619563755744\\
66.2865593145045	-0.245619681425416\\
66.2930793647347	-0.245619799061353\\
66.2995994106558	-0.245619916663568\\
66.3061194522691	-0.245620034232075\\
66.3126394895757	-0.245620151766886\\
66.319159522577	-0.245620269268013\\
66.3256795512742	-0.245620386735471\\
66.3321995756684	-0.245620504169272\\
66.338719595761	-0.245620621569428\\
66.3452396115532	-0.245620738935953\\
66.3517596230462	-0.245620856268859\\
66.3582796302413	-0.245620973568159\\
66.3647996331396	-0.245621090833867\\
66.3713196317425	-0.245621208065995\\
66.3778396260511	-0.245621325264555\\
66.3843596160667	-0.245621442429561\\
66.3908796017905	-0.245621559561025\\
66.3973995832238	-0.245621676658961\\
66.4039195603678	-0.245621793723381\\
66.4104395332238	-0.245621910754298\\
66.4169595017928	-0.245622027751725\\
66.4234794660763	-0.245622144715674\\
66.4299994260754	-0.245622261646158\\
66.4365193817914	-0.24562237854319\\
66.4430393332254	-0.245622495406784\\
66.4495592803787	-0.245622612236951\\
66.4560792232526	-0.245622729033704\\
66.4625991618482	-0.245622845797057\\
66.4691190961669	-0.245622962527021\\
66.4756390262097	-0.245623079223611\\
66.482158951978	-0.245623195886837\\
66.488678873473	-0.245623312516714\\
66.4951987906959	-0.245623429113254\\
66.501718703648	-0.245623545676469\\
66.5082386123303	-0.245623662206373\\
66.5147585167443	-0.245623778702977\\
66.5212784168911	-0.245623895166296\\
66.5277983127719	-0.245624011596341\\
66.5343182043879	-0.245624127993125\\
66.5408380917405	-0.24562424435666\\
66.5473579748307	-0.245624360686961\\
66.5538778536598	-0.245624476984038\\
66.5603977282291	-0.245624593247906\\
66.5669175985398	-0.245624709478575\\
66.573437464593	-0.245624825676061\\
66.5799573263901	-0.245624941840373\\
66.5864771839321	-0.245625057971527\\
66.5929970372204	-0.245625174069533\\
66.5995168862562	-0.245625290134405\\
66.6060367310406	-0.245625406166156\\
66.612556571575	-0.245625522164798\\
66.6190764078605	-0.245625638130343\\
66.6255962398983	-0.245625754062804\\
66.6321160676896	-0.245625869962194\\
66.6386358912357	-0.245625985828526\\
66.6451557105378	-0.245626101661812\\
66.6516755255971	-0.245626217462064\\
66.6581953364148	-0.245626333229295\\
66.6647151429921	-0.245626448963518\\
66.6712349453303	-0.245626564664746\\
66.6777547434305	-0.24562668033299\\
66.684274537294	-0.245626795968264\\
66.6907943269219	-0.24562691157058\\
66.6973141123156	-0.24562702713995\\
66.7038338934761	-0.245627142676387\\
66.7103536704048	-0.245627258179904\\
66.7168734431028	-0.245627373650513\\
66.7233932115713	-0.245627489088227\\
66.7299129758115	-0.245627604493057\\
66.7364327358247	-0.245627719865018\\
66.7429524916121	-0.24562783520412\\
66.7494722431748	-0.245627950510378\\
66.7559919905141	-0.245628065783802\\
66.7625117336312	-0.245628181024406\\
66.7690314725272	-0.245628296232202\\
66.7755512072035	-0.245628411407203\\
66.7820709376611	-0.24562852654942\\
66.7885906639014	-0.245628641658868\\
66.7951103859255	-0.245628756735557\\
66.8016301037346	-0.2456288717795\\
66.8081498173299	-0.245628986790711\\
66.8146695267126	-0.2456291017692\\
66.821189231884	-0.245629216714982\\
66.8277089328452	-0.245629331628067\\
66.8342286295974	-0.245629446508469\\
66.8407483221418	-0.2456295613562\\
66.8472680104797	-0.245629676171272\\
66.8537876946122	-0.245629790953699\\
66.8603073745406	-0.245629905703491\\
66.866827050266	-0.245630020420662\\
66.8733467217896	-0.245630135105223\\
66.8798663891126	-0.245630249757189\\
66.8863860522363	-0.245630364376569\\
66.8929057111618	-0.245630478963378\\
66.8994253658904	-0.245630593517627\\
66.9059450164231	-0.245630708039329\\
66.9124646627613	-0.245630822528496\\
66.9189843049061	-0.245630936985141\\
66.9255039428587	-0.245631051409275\\
66.9320235766203	-0.245631165800912\\
66.9385432061922	-0.245631280160063\\
66.9450628315754	-0.24563139448674\\
66.9515824527712	-0.245631508780957\\
66.9581020697808	-0.245631623042725\\
66.9646216826053	-0.245631737272057\\
66.9711412912461	-0.245631851468965\\
66.9776608957042	-0.245631965633461\\
66.9841804959809	-0.245632079765557\\
66.9907000920772	-0.245632193865267\\
66.9972196839946	-0.245632307932601\\
67.003739271734	-0.245632421967573\\
67.0102588552968	-0.245632535970194\\
67.0167784346841	-0.245632649940478\\
67.0232980098971	-0.245632763878435\\
67.029817580937	-0.245632877784079\\
67.0363371478049	-0.245632991657421\\
67.0428567105021	-0.245633105498475\\
67.0493762690297	-0.245633219307251\\
67.055895823389	-0.245633333083762\\
67.0624153735811	-0.245633446828022\\
67.0689349196072	-0.24563356054004\\
67.0754544614684	-0.245633674219831\\
67.0819739991661	-0.245633787867406\\
67.0884935327013	-0.245633901482777\\
67.0950130620753	-0.245634015065956\\
67.1015325872891	-0.245634128616957\\
67.1080521083441	-0.24563424213579\\
67.1145716252414	-0.245634355622468\\
67.1210911379822	-0.245634469077003\\
67.1276106465676	-0.245634582499407\\
67.1341301509988	-0.245634695889693\\
67.1406496512771	-0.245634809247873\\
67.1471691474036	-0.245634922573958\\
67.1536886393794	-0.245635035867961\\
67.1602081272058	-0.245635149129895\\
67.166727610884	-0.24563526235977\\
67.173247090415	-0.2456353755576\\
67.1797665658002	-0.245635488723395\\
67.1862860370406	-0.24563560185717\\
67.1928055041375	-0.245635714958935\\
67.1993249670919	-0.245635828028702\\
67.2058444259052	-0.245635941066485\\
67.2123638805785	-0.245636054072294\\
67.2188833311129	-0.245636167046142\\
67.2254027775097	-0.245636279988041\\
67.2319222197699	-0.245636392898003\\
67.2384416578949	-0.24563650577604\\
67.2449610918856	-0.245636618622164\\
67.2514805217434	-0.245636731436388\\
67.2579999474694	-0.245636844218722\\
67.2645193690648	-0.24563695696918\\
67.2710387865307	-0.245637069687773\\
67.2775581998684	-0.245637182374514\\
67.2840776090789	-0.245637295029414\\
67.2905970141634	-0.245637407652485\\
67.2971164151232	-0.245637520243739\\
67.3036358119594	-0.245637632803189\\
67.3101552046731	-0.245637745330847\\
67.3166745932656	-0.245637857826723\\
67.323193977738	-0.245637970290831\\
67.3297133580914	-0.245638082723182\\
67.3362327343271	-0.245638195123789\\
67.3427521064462	-0.245638307492663\\
67.3492714744498	-0.245638419829816\\
67.3557908383392	-0.24563853213526\\
67.3623101981155	-0.245638644409007\\
67.3688295537798	-0.245638756651069\\
67.3753489053334	-0.245638868861459\\
67.3818682527773	-0.245638981040187\\
67.3883875961129	-0.245639093187266\\
67.3949069353411	-0.245639205302708\\
67.4014262704632	-0.245639317386524\\
67.4079456014804	-0.245639429438728\\
67.4144649283938	-0.245639541459329\\
67.4209842512046	-0.245639653448342\\
67.4275035699139	-0.245639765405776\\
67.4340228845229	-0.245639877331645\\
67.4405421950327	-0.24563998922596\\
67.4470615014446	-0.245640101088732\\
67.4535808037596	-0.245640212919975\\
67.460100101979	-0.245640324719699\\
67.4666193961039	-0.245640436487917\\
67.4731386861354	-0.24564054822464\\
67.4796579720747	-0.245640659929881\\
67.486177253923	-0.24564077160365\\
67.4926965316814	-0.245640883245961\\
67.499215805351	-0.245640994856825\\
67.5057350749332	-0.245641106436253\\
67.5122543404289	-0.245641217984257\\
67.5187736018393	-0.24564132950085\\
67.5252928591657	-0.245641440986042\\
67.5318121124091	-0.245641552439847\\
67.5383313615707	-0.245641663862275\\
67.5448506066517	-0.245641775253339\\
67.5513698476532	-0.24564188661305\\
67.5578890845763	-0.24564199794142\\
67.5644083174223	-0.245642109238461\\
67.5709275461923	-0.245642220504184\\
67.5774467708873	-0.245642331738602\\
67.5839659915087	-0.245642442941726\\
67.5904852080574	-0.245642554113568\\
67.5970044205348	-0.245642665254139\\
67.6035236289419	-0.245642776363452\\
67.6100428332798	-0.245642887441518\\
67.6165620335497	-0.245642998488349\\
67.6230812297529	-0.245643109503956\\
67.6296004218903	-0.245643220488351\\
67.6361196099632	-0.245643331441547\\
67.6426387939727	-0.245643442363554\\
67.64915797392	-0.245643553254385\\
67.6556771498062	-0.245643664114051\\
67.6621963216324	-0.245643774942563\\
67.6687154893998	-0.245643885739934\\
67.6752346531096	-0.245643996506175\\
67.6817538127628	-0.245644107241299\\
67.6882729683607	-0.245644217945315\\
67.6947921199043	-0.245644328618237\\
67.7013112673949	-0.245644439260076\\
67.7078304108335	-0.245644549870843\\
67.7143495502213	-0.24564466045055\\
67.7208686855594	-0.24564477099921\\
67.727387816849	-0.245644881516832\\
67.7339069440913	-0.24564499200343\\
67.7404260672873	-0.245645102459015\\
67.7469451864382	-0.245645212883597\\
67.7534643015452	-0.24564532327719\\
67.7599834126093	-0.245645433639805\\
67.7665025196317	-0.245645543971452\\
67.7730216226137	-0.245645654272144\\
67.7795407215562	-0.245645764541893\\
67.7860598164604	-0.24564587478071\\
67.7925789073275	-0.245645984988606\\
67.7990979941586	-0.245646095165593\\
67.8056170769549	-0.245646205311684\\
67.8121361557174	-0.245646315426888\\
67.8186552304474	-0.245646425511218\\
67.8251743011459	-0.245646535564686\\
67.8316933678141	-0.245646645587302\\
67.8382124304531	-0.245646755579079\\
67.844731489064	-0.245646865540028\\
67.8512505436481	-0.245646975470161\\
67.8577695942063	-0.245647085369489\\
67.8642886407399	-0.245647195238023\\
67.87080768325	-0.245647305075776\\
67.8773267217377	-0.245647414882758\\
67.8838457562041	-0.245647524658982\\
67.8903647866504	-0.245647634404458\\
67.8968838130777	-0.245647744119198\\
67.9034028354872	-0.245647853803214\\
67.9099218538799	-0.245647963456517\\
67.916440868257	-0.245648073079119\\
67.9229598786196	-0.245648182671031\\
67.9294788849689	-0.245648292232265\\
67.935997887306	-0.245648401762831\\
67.9425168856319	-0.245648511262742\\
67.9490358799479	-0.245648620732009\\
67.955554870255	-0.245648730170644\\
67.9620738565545	-0.245648839578657\\
67.9685928388473	-0.245648948956061\\
67.9751118171347	-0.245649058302867\\
67.9816307914177	-0.245649167619085\\
67.9881497616975	-0.245649276904729\\
67.9946687279752	-0.245649386159808\\
68.001187690252	-0.245649495384335\\
68.0077066485289	-0.24564960457832\\
68.014225602807	-0.245649713741776\\
68.0207445530876	-0.245649822874714\\
68.0272634993717	-0.245649931977144\\
68.0337824416604	-0.245650041049079\\
68.0403013799549	-0.24565015009053\\
68.0468203142563	-0.245650259101508\\
68.0533392445657	-0.245650368082024\\
68.0598581708842	-0.245650477032091\\
68.0663770932129	-0.245650585951718\\
68.072896011553	-0.245650694840919\\
68.0794149259056	-0.245650803699703\\
68.0859338362717	-0.245650912528083\\
68.0924527426526	-0.245651021326069\\
68.0989716450493	-0.245651130093673\\
68.105490543463	-0.245651238830907\\
68.1120094378947	-0.245651347537781\\
68.1185283283456	-0.245651456214307\\
68.1250472148168	-0.245651564860497\\
68.1315660973094	-0.245651673476361\\
68.1380849758246	-0.245651782061911\\
68.1446038503634	-0.245651890617158\\
68.1511227209269	-0.245651999142114\\
68.1576415875163	-0.245652107636789\\
68.1641604501326	-0.245652216101195\\
68.1706793087771	-0.245652324535344\\
68.1771981634508	-0.245652432939246\\
68.1837170141547	-0.245652541312914\\
68.1902358608901	-0.245652649656357\\
68.1967547036581	-0.245652757969588\\
68.2032735424597	-0.245652866252617\\
68.209792377296	-0.245652974505456\\
68.2163112081682	-0.245653082728116\\
68.2228300350774	-0.245653190920609\\
68.2293488580247	-0.245653299082945\\
68.2358676770111	-0.245653407215136\\
68.2423864920379	-0.245653515317193\\
68.2489053031061	-0.245653623389127\\
68.2554241102168	-0.245653731430949\\
68.2619429133711	-0.245653839442671\\
68.2684617125701	-0.245653947424304\\
68.274980507815	-0.245654055375859\\
68.2814992991068	-0.245654163297347\\
68.2880180864467	-0.24565427118878\\
68.2945368698357	-0.245654379050168\\
68.301055649275	-0.245654486881523\\
68.3075744247656	-0.245654594682855\\
68.3140931963087	-0.245654702454177\\
68.3206119639054	-0.245654810195499\\
68.3271307275567	-0.245654917906832\\
68.3336494872638	-0.245655025588188\\
68.3401682430277	-0.245655133239578\\
68.3466869948497	-0.245655240861012\\
68.3532057427307	-0.245655348452502\\
68.3597244866719	-0.245655456014059\\
68.3662432266743	-0.245655563545694\\
68.3727619627391	-0.245655671047419\\
68.3792806948674	-0.245655778519244\\
68.3857994230603	-0.24565588596118\\
68.3923181473189	-0.245655993373239\\
68.3988368676442	-0.245656100755432\\
68.4053555840374	-0.245656208107769\\
68.4118742964995	-0.245656315430262\\
68.4183930050317	-0.245656422722922\\
68.424911709635	-0.24565652998576\\
68.4314304103107	-0.245656637218788\\
68.4379491070596	-0.245656744422015\\
68.444467799883	-0.245656851595453\\
68.450986488782	-0.245656958739114\\
68.4575051737575	-0.245657065853008\\
68.4640238548108	-0.245657172937146\\
68.4705425319429	-0.24565727999154\\
68.477061205155	-0.2456573870162\\
68.483579874448	-0.245657494011137\\
68.4900985398232	-0.245657600976363\\
68.4966172012815	-0.245657707911888\\
68.5031358588241	-0.245657814817724\\
68.5096545124521	-0.245657921693881\\
68.5161731621665	-0.24565802854037\\
68.5226918079685	-0.245658135357204\\
68.5292104498592	-0.245658242144391\\
68.5357290878395	-0.245658348901944\\
68.5422477219107	-0.245658455629873\\
68.5487663520739	-0.24565856232819\\
68.55528497833	-0.245658668996905\\
68.5618036006802	-0.24565877563603\\
68.5683222191255	-0.245658882245574\\
68.5748408336672	-0.24565898882555\\
68.5813594443061	-0.245659095375968\\
68.5878780510436	-0.24565920189684\\
68.5943966538805	-0.245659308388175\\
68.600915252818	-0.245659414849985\\
68.6074338478573	-0.245659521282282\\
68.6139524389993	-0.245659627685075\\
68.6204710262452	-0.245659734058376\\
68.626989609596	-0.245659840402195\\
68.6335081890528	-0.245659946716544\\
68.6400267646168	-0.245660053001434\\
68.6465453362889	-0.245660159256875\\
68.6530639040703	-0.245660265482879\\
68.6595824679621	-0.245660371679455\\
68.6661010279653	-0.245660477846616\\
68.6726195840811	-0.245660583984372\\
68.6791381363104	-0.245660690092733\\
68.6856566846544	-0.245660796171711\\
68.6921752291142	-0.245660902221317\\
68.6986937696908	-0.245661008241561\\
68.7052123063854	-0.245661114232455\\
68.7117308391989	-0.245661220194008\\
68.7182493681325	-0.245661326126233\\
68.7247678931873	-0.245661432029139\\
68.7312864143642	-0.245661537902738\\
68.7378049316646	-0.24566164374704\\
68.7443234450892	-0.245661749562057\\
68.7508419546394	-0.245661855347799\\
68.7573604603161	-0.245661961104277\\
68.7638789621204	-0.245662066831501\\
68.7703974600533	-0.245662172529483\\
68.7769159541161	-0.245662278198234\\
68.7834344443097	-0.245662383837763\\
68.7899529306352	-0.245662489448083\\
68.7964714130936	-0.245662595029203\\
68.8029898916862	-0.245662700581134\\
68.8095083664138	-0.245662806103888\\
68.8160268372777	-0.245662911597475\\
68.8225453042788	-0.245663017061906\\
68.8290637674183	-0.245663122497191\\
68.8355822266972	-0.245663227903341\\
68.8421006821165	-0.245663333280368\\
68.8486191336775	-0.245663438628281\\
68.855137581381	-0.245663543947092\\
68.8616560252283	-0.245663649236811\\
68.8681744652203	-0.245663754497449\\
68.8746929013582	-0.245663859729017\\
68.8812113336429	-0.245663964931525\\
68.8877297620756	-0.245664070104985\\
68.8942481866574	-0.245664175249406\\
68.9007666073892	-0.2456642803648\\
68.9072850242723	-0.245664385451177\\
68.9138034373075	-0.245664490508548\\
68.9203218464961	-0.245664595536924\\
68.9268402518391	-0.245664700536315\\
68.9333586533374	-0.245664805506732\\
68.9398770509923	-0.245664910448186\\
68.9463954448047	-0.245665015360687\\
68.9529138347758	-0.245665120244246\\
68.9594322209065	-0.245665225098874\\
68.965950603198	-0.245665329924582\\
68.9724689816514	-0.245665434721379\\
68.9789873562676	-0.245665539489277\\
68.9855057270477	-0.245665644228286\\
68.9920240939928	-0.245665748938417\\
68.998542457104	-0.24566585361968\\
69.0050608163823	-0.245665958272087\\
69.0115791718288	-0.245666062895648\\
69.0180975234446	-0.245666167490373\\
69.0246158712306	-0.245666272056272\\
69.031134215188	-0.245666376593358\\
69.0376525553179	-0.24566648110164\\
69.0441708916212	-0.245666585581128\\
69.0506892240991	-0.245666690031835\\
69.0572075527525	-0.245666794453769\\
69.0637258775826	-0.245666898846941\\
69.0702441985904	-0.245667003211363\\
69.076762515777	-0.245667107547044\\
69.0832808291434	-0.245667211853996\\
69.0897991386907	-0.245667316132229\\
69.0963174444199	-0.245667420381753\\
69.1028357463321	-0.245667524602579\\
69.1093540444284	-0.245667628794717\\
69.1158723387097	-0.245667732958179\\
69.1223906291772	-0.245667837092974\\
69.1289089158319	-0.245667941199114\\
69.1354271986748	-0.245668045276608\\
69.1419454777071	-0.245668149325467\\
69.1484637529297	-0.245668253345702\\
69.1549820243438	-0.245668357337323\\
69.1615002919503	-0.245668461300341\\
69.1680185557503	-0.245668565234766\\
69.1745368157449	-0.245668669140608\\
69.1810550719351	-0.245668773017879\\
69.1875733243219	-0.245668876866589\\
69.1940915729065	-0.245668980686748\\
69.2006098176899	-0.245669084478366\\
69.2071280586731	-0.245669188241455\\
69.2136462958571	-0.245669291976025\\
69.220164529243	-0.245669395682085\\
69.2266827588319	-0.245669499359647\\
69.2332009846248	-0.245669603008721\\
69.2397192066228	-0.245669706629317\\
69.2462374248268	-0.245669810221447\\
69.252755639238	-0.245669913785119\\
69.2592738498574	-0.245670017320346\\
69.265792056686	-0.245670120827137\\
69.2723102597249	-0.245670224305502\\
69.2788284589751	-0.245670327755452\\
69.2853466544377	-0.245670431176998\\
69.2918648461137	-0.24567053457015\\
69.2983830340042	-0.245670637934918\\
69.3049012181101	-0.245670741271313\\
69.3114193984326	-0.245670844579345\\
};
\end{axis}
\end{tikzpicture}%
\caption{Interaction of four line vortices with $\alpha=0.1$, showing no leapfrogging.}
\label{fig:noleap}
\end{figure}
\item[Instability to small asymmetric disturbances ($0.172<\alpha<0.29$)] 
In this region, leapfrogging occurs, but small asymmetric disturbances in the starting position will cause the system to collapse into instability after a small number of ``leapfroggings''.
The plot from \citet{acheson00} (\cref{fig:achinst}) shows this instability after three ``leapfroggings'', for $\alpha=0.220$ and a pertubation of \num{1E-6} to the first vortex.
The same conditions are used for the MATLAB plot in \cref{fig:instab}.
\begin{figure}[ht]
\centering
\includegraphics[width=\textwidth]{achinst.png}
\caption{Interaction of four line vortices with $\alpha=0.220$, and a pertubation of \num{1E-6}, showing instability. From \citet[p.~271]{acheson00}}
\label{fig:achinst}
\end{figure}
\begin{figure}[ht]
\centering
\setlength\figureheight{7.5cm} 
\setlength\figurewidth{\textwidth}
\input{4vortexinstab2.tikz}
\caption{Interaction of four line vortices with $\alpha=0.220$, and a pertubation of \num{1E-6},  showing instability. This plot from MATLAB is broadly in agreement with the above plot from \citet{acheson00}}
\label{fig:instab}
\end{figure}
\item[Instability with three-vortex motion ($0.29<\alpha<0.382$)]
Within this region, the instability often forms a system of three vortices, as two vortices with the same polarity pass close enough to trigger orbital motion, capturing one of the other vortices as they go.
The plot from \citet{acheson00} (\cref{fig:ach3inst}) shows the system for $\alpha=0.310$ and a pertubation of \num{7E-3} to the first vortex.
The same conditions are used for the MATLAB plot in \cref{fig:3instab}.
\begin{figure}[ht]
\centering
\includegraphics[width=\textwidth]{ach3inst.png}
\caption{Interaction of four line vortices with $\alpha=0.310$, and a pertubation of \num{7E-3}, showing the three vortex interaction. From \citet[p.~272]{acheson00}}
\label{fig:ach3inst}
\end{figure}
\begin{figure}[ht]
\centering
\setlength\figureheight{7.5cm} 
\setlength\figurewidth{\textwidth}
\input{4vortex3instab.tikz}
\caption{Interaction of four line vortices with $\alpha=0.310$, and a pertubation of \num{7E-3},  showing the three vortex interaction. This plot from MATLAB is broadly in agreement with the above plot from \citet{acheson00}}
\label{fig:3instab}
\end{figure}
\item[Stability ($\alpha>0.382$)] 
For higher values of $\alpha$, the system seems stable to small pertubations, at least in similar domains to the other plots.
A plot is shown below (\cref{fig:stable}) for $\alpha=0.5$ and a pertubation of \num{5E-3}.
There does, however appear to be some numerical errors, as the vortex pairs appear to widen.
\begin{figure}[ht]
\centering
\setlength\figureheight{7.5cm} 
\setlength\figurewidth{\textwidth}
\input{stable4vortex.tikz}
\caption{Interaction of four line vortices with $\alpha=0.5$, and a pertubation of \num{5E-3},  showing stability, but possibly some numerical errors.}
\label{fig:stable}
\end{figure}
\end{description}
\clearpage
\section{Errors}
Analysing \cref{fig:stable}, it can be seen that the vortices seems to widen throughout the model run.
This would suggest energy is not staying constant.
To analyse this, the model was run with the same settings, and $\chi$ (\cref{eq:lovechi}) was plotted (\hyperref[chicalc]{\texttt{chicalc.m}}).
This is shown in \cref{fig:chicalc}.
\begin{figure}[ht]
\centering
\setlength\figureheight{7.5cm} 
\setlength\figurewidth{\textwidth}
\input{chicalc.tikz}
\caption{Plot of $\chi$ (\cref{eq:lovechi}) with time, showing loss of energy in the system. The blue plot shows the vortex pair above the $x$ axis, the red the pair below and the green their sum.}
\label{fig:chicalc}
\end{figure}

It can be seen from \cref{fig:chicalc} that there is energy lost from the system as time increases.
The gradual decline, and the effect it has on the paths of the vortices (widening) suggests possibly numerical diffusion, which is a known problem in many computational fluid problems.

\clearpage
\section*{Appendix I: MATLAB Code}\label{sec:ap1}
MATLAB code for the vortex leapfrogging program (\texttt{vortexleap.m}\normalfont) and subroutines (\texttt{vortexf.m}\normalfont,\texttt{vortexg.m}), as well as the analytic program \texttt{chicalc.m} are given below.
\subsection*{\texttt{vortexf.m}}
\label{vortexf}
\begin{verbatim}
function[fi]=vortexf(i,k,x,y,N)
fi=0;
jmat=[1:1:N];
jmat(i)=[];
for j=jmat
    rr=((x(i)-x(j))^2)+((y(i)-y(j))^2);
    fi=fi+(k(j)*(y(j)-y(i))/rr);
end
end
\end{verbatim}
\subsection*{\texttt{vortexg.m}}
\label{vortexg}
\begin{verbatim}
function[gi]=vortexg(i,k,x,y,N)
gi=0;
jmat=[1:1:N];
jmat(i)=[];
for j=jmat
    rr=((x(i)-x(j))^2)+((y(i)-y(j))^2);
    gi=gi+(k(j)*(x(i)-x(j))/rr);
end
end
\end{verbatim}
\subsection*{\texttt{vortexleap.m}}
\label{vortexleap}
\begin{verbatim}
clear all
N=input('number of vortices = ');
%set time step and scales
t=0;
T=input('number of time steps = ');
%set length step and scales
h=input('step-size = ');
%define vortex strengths
for i=1:N
    k(i)=input(['strength of vortex ',num2str(i),' = ']);
end
%define variable for position
x=NaN(1000,N);
y=NaN(1000,N);
for i=1:N
    x(1,i)=input(['x-coordinate of vortex ',num2str(i),' = ']);
    y(1,i)=input(['y-coordinate of vortex ',num2str(i),' = ']);
end
for a=2:T
    clear x1 x2 x3 x4 y1 y2 y3 y4 k1 k2 k3 k4 l1 l2 l3 l4
    x1=x(a-1,:);
    y1=y(a-1,:);
    k1=NaN(1,N);
    k2=NaN(1,N);
    k3=NaN(1,N);
    k4=NaN(1,N);
    l1=NaN(1,N);
    l2=NaN(1,N);
    l3=NaN(1,N);
    l4=NaN(1,N);
    for i=1:N
        k1(i)=h*vortexf(i,k,x1,y1,N);
        l1(i)=h*vortexg(i,k,x1,y1,N);
    end
    x2=x1+(k1/2);
    y2=y1+(y1/2);
    for i=1:N
        k2(i)=h*vortexf(i,k,x2,y2,N);
        l2(i)=h*vortexg(i,k,x2,y2,N);
    end
    x3=x1+(k2/2);
    y3=y1+(y2/2);
    for i=1:N
        k3(i)=h*vortexf(i,k,x3,y3,N);
        l3(i)=h*vortexg(i,k,x3,y3,N);
    end
    x4=x1+(k3);
    y4=y1+(y3);
    for i=1:N
        k4(i)=h*vortexf(i,k,x4,y4,N);
        l4(i)=h*vortexg(i,k,x4,y4,N);
    end
    x(a,:)=x(a-1,:)+((1/6)*(k1+2*k2+2*k3+k4));
    y(a,:)=y(a-1,:)+((1/6)*(l1+2*l2+2*l3+l4));
    percent=100*a/T;
    display([num2str(percent),'% done'])
end
clf
%for viewing evolution of tracks
%multicomet(x,y)
%for viewing static image of tracks
colourmap=['b','r','g','c','m','y','k'];
maxy=1.5*max(max(abs(y)));
hold on
for i=1:N
    plot(x(:,i),y(:,i),colourmap(i))
end
xlabel('x');
ylabel('y');
title([num2str(T),' time steps, step size = ',num2str(h)]);
axis([0 inf -maxy maxy])
hold off
name=input('file name? ');
print('-dpng',[name,'.png']);
\end{verbatim}
The scheme used is a fourth order Runge-Kutta method with non-adaptive step size. The equations plotted are from \citet{acheson00}.
\subsection*{\texttt{chicalc.m}}
\label{chicalc}
\begin{verbatim}
clear chi
for i=1:T
    chi(i)=(k(1)/(2*pi))*log(y(i,1)*y(i,3)*(((x(i,3)-x(i,1))^2 + (y(i,1)+y(i,3))^2)/((x(i,3)-x(i,1))^2 + (y(i,3)-y(i,1))^2)));
end
plot(chi)
hold on
title('plot of \chi for stable vortex interaction simulation')
xlabel('time')
ylabel('\chi')
hold off
\end{verbatim}
\bibliographystyle{dcu}
\bibliography{kwnrefs}
\end{document}
